% Template for ICASSP-2021 paper; to be used with:
%          spconf.sty  - ICASSP/ICIP LaTeX style file, and
%          IEEEbib.bst - IEEE bibliography style file.
% --------------------------------------------------------------------------
\documentclass{article}
\usepackage{spconf,amsmath,graphicx}
\usepackage[dvipsnames]{xcolor}
\usepackage[]{pgf}
\usepackage{tikz}
\usetikzlibrary{shapes,arrows,snakes,backgrounds,matrix,patterns,positioning,fadings}
\usepackage[mode=buildnew]{standalone}
\usepackage{subfig}
\usepackage{upgreek}
\usepackage{nicefrac}
\usepackage{dsfont}
% \usepackage[OT1]{fontenc}
% \renewcommand*\familydefault{\sfdefault}
\usepackage{bm}
\usepackage{cancel}
\usepackage{amsbsy}
\usepackage{amssymb}
\usepackage[ruled]{algorithm2e}
% \usepackage{algpseudocode}
\usepackage{lipsum}
\usepackage{harpoon}
\usepackage[percent]{overpic}
\usepackage{comment}
\newenvironment{note}
	{\par\textcolor{Blue}{\bfseries Note:} \color{Blue}\ignorespaces}
	{\par}
\newenvironment{attention}
	{\par\textcolor{red}{\bfseries Attention:} \color{red}\ignorespaces}
	{\par}
\newenvironment{todo}
	{\par\textcolor{red}{\bfseries TODO:} \color{red}\ignorespaces}
	{\par}
\usepackage{hyperref}
\hypersetup{
	colorlinks,
	linkcolor={blue!80!black},
	citecolor={blue!80!black},
	urlcolor={blue!80!black}
}
% \usepackage{algorithm}
% \usepackage{algpseudocode}
\usepackage[short]{optidef}

\renewcommand{\chapterautorefname}{Ch.}
\renewcommand{\sectionautorefname}{Sec.}
\renewcommand{\subsectionautorefname}{Sec.}
\renewcommand{\figureautorefname}{Fig.}
\newcommand{\subfigureautorefname}{\figureautorefname}
% \newcommand{\algorithmautorefname}{Alg.}

\newcommand{\mtxb}[1]{\bm{\mathrm{#1}}}
\newcommand{\T}{{\mathrm{T}}}
\newcommand{\herm}{{\mathrm{H}}}
\newcommand{\ev}[1]{\mathrm{E} \left\lbrace #1 \right\rbrace}

% Common variables
\newcommand{\h}{\mtxb{h}}
\newcommand{\x}{\mtxb{x}}
\newcommand{\R}{\mtxb{R}}
\newcommand{\W}{\mtxb{W}}
\newcommand{\w}{\mtxb{w}}
\newcommand{\z}{\mtxb{z}}
\newcommand{\y}{\mtxb{y}}
\newcommand{\uu}{\mtxb{u}}
\newcommand{\aRho}{\mtxb{P}}
\newcommand{\hf}{{\bm{h}}}
\newcommand{\xf}{{\bm{x}}}
\newcommand{\Rf}{{\bm{{R}}}}
\newcommand{\Wf}{{\bm{{W}}}}
\newcommand{\Df}{{\bm{{D}}}}
\newcommand{\wf}{{\bm{w}}}
\newcommand{\zf}{{\bm{z}}}
\newcommand{\uuf}{{\bm{u}}}
\newcommand{\yf}{{\bm{y}}}
\newcommand{\aRhof}{{\bm{{P}}}}
\newcommand{\I}{\mtxb{I}}
\newcommand{\Cset}{\mathcal{C}}
\newcommand{\Mset}{\mathcal{M}}
\newcommand{\Tset}{\mathcal{T}}
\newcommand{\Rset}{\mathcal{R}}
\newcommand{\Nset}{\mathcal{N}}



% Title.
% ------
\title{SOMETHING ABOUT THE DISTRIBUTED AVERAGING BASED APPROXIMATION IN SENSOR NETWORKS}
%
% Single address.
% ---------------
\name{Matthias Blochberger\(^1\), Filip Elvander\(^2\), Randall Ali\(^1\), Toon van Waterschoot\(^1\),tbd\thanks{This research work was carried out at the ESAT Laboratory of KU Leuven, in the frame of the SOUNDS European Training Network. This project has received funding from the European Union's Horizon 2020 research and innovation programme under the Marie Skłodowska-Curie grant agreement No.\,956369. This research received funding in part from the Research Foundation - Flanders (FWO) grant 12ZD622N as well as from the European Union's Horizon 2020 research and innovation program / ERC Consolidator Grant: SONORA (No.\,773268). This paper reflects only the authors' views and the Union is not liable for any use that may be made of the contained information. Source code available at \textcolor{red}{ADD URL}}}
\address{\(^1\)KU Leuven, Dept. of Electrical Engineering (ESAT), STADIUS, 3001 Leuven, Belgium\\\(^2\)Aalto University, Dept. of Signal Processing and Acoustics, 02150 Espoo, Finland\\tbd}

\begin{document}
\ninept
%
\maketitle
%
\begin{abstract}
    Distributed signal-processing algorithms in (wireless) sensor networks often aim to decentralize processing tasks as to reduce communication cost and computational complexity or avoid reliance on a single device for processing.
    In this contribution, we extend a distributed adaptive algorithm for blind system identification that relies on the knowledge of a stacked network-wide solution vector at each node, the computation of which requires either broadcasting or relaying of node-specific values to all other nodes.
    The extended algorithm employs a distributed-averaging-based estimation scheme to estimate the network-wide norm value by only using the information provided by neighboring sensor nodes.
    We introduce an adaptive mixing factor between instantaneous and recursion values for adaptivity in a time-varying system.
    Simulation results show that the extension provides estimation results close to the optimal fully-connected-network case while reducing inter-node transmission significantly.
\end{abstract}
%
\begin{keywords}
multi-channel signal processing, distributed signal processing, wireless sensor networks, blind system identification, distributed averaging
\end{keywords}
%
\section{INTRODUCTION}
\label{sec:intro}

Distributed algorithms have been an active area of research for quite some time, with numerous control, optimization, and signal processing applications.
With the ever-growing number of smart multimedia devices in today's surroundings providing ubiquitous processing and communication capabilities, distributed audio and speech signal processing have also found their way into the spotlight.
Algorithms for distributed signal estimation \cite{5483092}, noise control and echo cancellation \cite{9670697}, as well as beamforming \cite{6663655,6329934,MARKOVICHGOLAN20154} to name a few, have been proposed.
The task of distributed single-input-multiple-output (SIMO) blind system identification (BSI) had contributions such as the adaptive cross-relation-based (CR) \cite{yuDistributedBlindSystem2014, liuDistributedBlindIdentification2016}.
Furthermore, we recently introduced an adaptive CR-based algorithm \cite{blochbergerDBSI} using the alternating direction method of multipliers (ADMM) \cite{boydDistributedOptimizationStatistical2011}.
All of the mentioned distributed BSI algorithms rely on information shared between neighboring sensor nodes within the network.
However, the CR-based BSI task necessitates a non-triviality constraint on the full system to be identified (we refer the reader to, e.g., \cite{huangAdaptiveMultichannelLeast2002,huangClassFrequencydomainAdaptive2003}), which manifests itself as one or more global variables.
In this case, the computation of the global variable requries information from all the nodes within the network.
The algorithm in \cite{blochbergerDBSI} relies on node-wise values being relayed throughout the network.
In contrast, to overcome the need for the network to be fully connected, \cite{yuDistributedBlindSystem2014, liuDistributedBlindIdentification2016} use an average consensus \cite{xiaoFastLinearIterations2004} approach where a secondary recursion estimates the global variable for each signal frame.
Both approaches introduce additional transmissions of variables between nodes, the number of which, depending on network and neighborhood size, can be substantial or even unfeasable.

In this paper, we extend \cite{blochbergerDBSI} with a distributed averaging-based \cite{xiaoFastLinearIterations2004} estimation scheme for the global variable.
This approach allows us to compute the global variable without the need of a fully connected network or broadcasting.
Further, we introduce a mixing factor to include instantaneous values into the averaging recursion, which then allows us (i) to reduce the number of secondary iterations significantly (down to 1) and (ii) track time-varying systems.
Simulation results show how the extension leads to BSI performance close to an optimal case where all information is available without the need of broadcasting variables to all nodes or a large number of estimation iterations.

\section{DISTRIBUTED ADAPTIVE BSI IN SENSOR NETWORKS WITH ONLINE-ADMM}
\label{sec:dbsi}
In this section, we briefly outline the adaptive SIMO BSI algorithm proposed in [8], setting the scene for the distributed averaging scheme proposed herein.
% In this section, we will introduce certain parts of an adaptive SIMO BSI algorithm based on Online-ADMM \cite{blochbergerDBSI}, which are necessary for the reader to understand its extension of it.
We refer the reader to the publication cited above for the full derivation of the algorithm.

\subsection[]{Sensor network}
We assume a set \(\Mset \triangleq \{1,\ldots,M\}\) sensor nodes indices and a set of edge tuples \(\mathcal{E}\), which connect the nodes forming a sensor network.
Each edge is an unordered pair of node indices \(\{i,j\} \in \mathcal{E}\), which represents the communication link between them.
We allow \(\{i,i\} \in \mathcal{E}\), in order to simplify notation later, however, this does not represent a link but rather indicates the obvious fact that node \(i\) has information of itself.
Furthermore, let \(\Nset_i = \{j|\{i,j\} \in \mathcal{E}\}\) denote the neighborhood of node \(i\),
% From this follows the neighborhood of a node \(i\), i.e., the other nodes it is directly connected to, defined as \(\Nset_i = \{j|\{i,j\} \in \mathcal{E}\}\).
% It is to note, that we define the set with \(i \in \Nset_i\) for ease of notation later.
% This does however not represent an actual inter-node link.
We can define the symmetric adjacency matrix \(\mtxb{C}\) with elements
\begin{equation}
    C_{ij} = \begin{cases}
        1 \quad \text{if } \{i,j\} \in \mathcal{E}\\
        0 \quad \text{otherwise}.
    \end{cases}
\end{equation}
It may be noted that in \cite{blochbergerDBSI}, the pairs \(\{i,j\} \in \mathcal{E}\) are ordered, which yields a directed graph and leads to two sets of neighborhoods ("transmit" and "receive") and therefore a non-symmetric adjacency matrix.
For the sake of brevity, we limit the explanations in this paper to the undirected case.
The case of directed graphs can be extended directly from the presented results.

\subsection[]{Signal model}
We consider a SIMO system with 
\begin{align}
    \mtxb{s}(n) &= [s(n),\,\ldots,\,s(n-2L+2)]^{\T},\\
    \x_i(n) &= [x_i(n),\,\ldots,\,x_i(n-L+1)]^{\T}, \quad i \in \Mset,
\end{align}
the input signal frame and \(M\) output signal frames, respectively.
Each output \(\x_i(n)\) is the convolution of \(\mtxb{s}(n)\) with the respective channel impulse response \(\h_i\) and an additive noise term \(\mtxb{v}_i(n)\), assumed to be zero-mean and uncorrelated with \(\mtxb{s}(n)\).
The signal model is \(\x_i(n) = \mtxb{H}_i \mtxb{s}(n) + \mtxb{v}_i(n),\)
with \(\mtxb{H}_i\), the \(L \times (2L-1)\) linear convolution matrix of the \(i\)th channel using the elements of \(\h_i\) of length \(L\).
For the purpose of this paper, the length \(L\) of the impulse responses is assumed to be known.

\subsection[]{Distributed BSI with Online-ADMM}
In BSI, the coss-relation problem formulation only uses output signals \(\x_i(n)\), exploiting relative information between them, to identify the system, i.e., the acoustic or communication channels \(\h = [\h_1^\T,\,...,\,\h_M^\T]^\T\).
The solution to this problem is found by the minimization problem (cf. e.g. \cite{langtongBlindIdentificationEqualization1994,huangAdaptiveMultichannelLeast2002,huangClassFrequencydomainAdaptive2003,blochbergerDBSI})
\begin{equation}
    \begin{aligned}
        \hat{\h} = \arg \min_{\h} \quad &\h^\herm \R \h \\
        \text{s.t. } \quad &\|\h\| = 1.
    \end{aligned}\label{eq:frequency_domain:min_prob}
\end{equation}
where the estimate is \(\hat{\h} = [\hat{\h}_1^\T,\,...,\,\hat{\h}_M^\T]^\T\).
In \cite{blochbergerDBSI}, this problem is solved by a distributed adaptive algorithm that is based on separating the problem \eqref{eq:frequency_domain:min_prob} into node-wise subproblems of analogous form, whoever only using a subset channels, provided by neighboring nodes \(\Nset_i \subset \Mset\).
General-form consensus alternating direction method of multipliers (ADMM) \cite{boydDistributedOptimizationStatistical2011} is applied in an adaptive updating scheme (Online-ADMM) \cite{wangOnlineAlternatingDirection2013,hosseiniOnlineDistributedADMM2014}.
This leads to update steps for local, consensus, and dual variables (we refer the reader to \cite{blochbergerDBSI} for details).
The consensus variable in the algorithm is the estimate of the system \(\hat{\h}\).
In the distributed algorithm, each of its subvectors \(\hat{\h}_i\) is computed at the respective node \(i\).
The update step for \(\hat{\h}_i\) is of interest for this extension.
% Separating the cost function and applying general-form consensus ADMM as an adaptive algorithm leads to the update steps
% \begin{align}
%     \w_i^{n+1} &= \underset{\w_i}{\operatorname{argmin}} \, \mathcal{L}_{\rho} (\w,\h^n,\uu^n)\label{eq:general_consensus_admm:local}\\
%     \h^{n+1} &= \underset{\h, \|\h\| = 1}{\operatorname{argmin}}\, \mathcal{L}_{\rho} (\w^{n+1},\h,\uu^n)\label{eq:general_consensus_admm:global}\\
%     \uu_i^{n+1} &= \uu_i^{n} + \rho \left( \w_i^{n+1} - \tilde{\h}_i^{n+1} \right),\label{eq:general_consensus_admm:dual}
% \end{align}
% where \(\w_i\) represents channels estimates of lower dimensional subproblems, solved locally by node \(i\), \(\h = [\hat{\h}_1^\T,\,...,\,\hat{\h}_M^\T]^\T\) the consensus variable (estimate of the system), and \(\uu_i\) are the dual variables.
\begin{todo}
    Is this enough? I fear i have to use a full page introducing all the local, consensus, dual variables, parameter mapping and the lagrangian etc. only to get to \eqref{eq:online_admm:consensus_update} and say that the denominator is what we are looking at...
\end{todo}
It is defined as
\begin{equation}
    \hat{\h_i}^{n+1} = \frac{\bar{\h}_i^{n+1}}{\sqrt{\sum_{j \in \Mset} \|\bar{\h}_j^{n+1}\|^2}}.\label{eq:online_admm:consensus_update}
\end{equation}
% for each node \(i \in \Mset\) with the local unnormalized consensus \(\bar{\h}_i^{n+1}\), a combination of neigborhood averages of \(\w_j\) and \(\uu_j\), \(j \in \Nset_i\).
The computation of the denominator of \eqref{eq:online_admm:consensus_update} requires  \(\|\bar{\h}_j^{n+1}\|\) from all nodes \(j \in \Mset\), which however, is not possible without a fully-connected network or broadcasting.
In \cite{blochbergerDBSI}, we assume that partial squared norms are relayed throughout the network until all nodes have the information necessary.
In \cite{yuDistributedBlindSystem2014,liuDistributedBlindIdentification2016}, a global value is estimated by a distributed averaging approach, iteratively combining values within neighborhoods until convergence to a network-wide average.

Subsequently, we introduce an extension to the algorithm described in \cite{blochbergerDBSI} using a fastest distributed linear averaging (FDLA) approach \cite{xiaoFastLinearIterations2004} to avoid the need of network wide data transmission, with the further introduction of an adaptive mixing factor to include instantaneous values in the recursion.

\section{Adaptive norm estimation}
\label{sec:adaptivenormest}

\subsection[]{Distributed averaging}
The computation of the denominator of \eqref{eq:online_admm:consensus_update} requires knowledge of all \(\|\bar{\h}_i\|^2\) at each node \(i \in \Mset\).
To avoid the large number of transmissions, which would be necessary to facilitate this, we will approximate the squared norm \(\sum_{i \in \Mset} \|\bar{\h}_i\|^2\) at each node \(i\) only using information shared within its neighborhood \(\Nset_i\).

We look at distributed linear combinations with the iteration index \(k=0,...,K\),
\begin{equation}
    \phi_i({k+1}) = \sum_{j \in \Nset_i} W_{ij} \phi_j({k}),\quad i\in \Mset,\label{eq:adaptivenormest:distlincomb}
\end{equation}
which can be written in vector form as \(\bm{\phi}({k+1}) = \W \bm{\phi}({k})\) with \(\bm{\phi}(k) = \begin{bmatrix} \phi_1(k) & \ldots & \phi_M(k) \end{bmatrix}^\T\).
We want to find the matrix \(\mtxb{W}\), which ensure that for any initial value \(\bm{\phi}({0})\), each element of \(\bm{\phi}({k})\) converges to the average of the elements of \(\bm{\phi}({0})\) when \(k \to \infty\).
In this application, the elements of \(\bm{\phi}(0)\) are the initial node-wise squared norm values \(\|\bar{\h}_i\|^2\) which then leads to \(\lim_{k \to \infty} M \phi_i(k) = \sum_{i \in \Mset} \|\bar{\h}_i\|^2\) for all \(i \in \Mset\).
This is achieved by solving the fastest distributed linear averaging (FDLA) problem, introduced in \cite{xiaoFastLinearIterations2004}, in the form of
\begin{equation}
    \begin{aligned}
        \min& \quad \| \W - \bm{1}\bm{1}^\T/M\|_2\\
        \text{s.t.}& \quad \W \in \mathcal{C},\, \bm{1}^\T \W = \bm{1}^\T,\, \W \bm{1}= \bm{1}.
    \end{aligned}\label{eq:adaptivenormest:fdlaminprob}
\end{equation}
Here, the set \(\mathcal{C} = \{\W \in \mathbb{R}^{M \times M} \vert W_{ij} = 0 \text{ if } \{i,j\} \notin \mathcal{E}\}\) describes all matrices with the same sparsity pattern as the adjacency matrix \(\mtxb{C}\).
Now, the \(i\)-th row of \(\W\) contains the neigborhood weights for node \(i\) as non-zero entries.
% \begin{todo}
%     This section reads awkward
% \end{todo}
% In usual applications of the iteration \eqref{eq:adaptivenormest:distlincomb}, it is applied \(K \in \mathbb{N}\) times until convergence of the averages is reached.
In an adaptive algorithm, where the resulting variable is required at each time frame \(n\) such as e.g. \cite{yuDistributedBlindSystem2014, liuDistributedBlindIdentification2016} and \cite{blochbergerDBSI}, it introduces \(K\) secondary iterations per frame.
Each iteration requires information exchange within neighborhoods, adding to the communication cost of the algorithm.
In order to mitigate this, we split iterations over frames \(n\) to keep \(K\) as low as possible.
However, as each frame introduces new data, we need to include this data into the recursion.
To this end, we further introduce an instantaneous weighted neighborhood average of node-wise squared norm values \(\bar{\phi}_i^{n} = \sum_{j \in \Nset_i} W_{ij} \|\bar{\h}_i^n\|^2\) which is then combined with the iterative average into
\begin{equation}
    \hat{\phi}_i^{n+1} = \gamma_i \bar{\phi}_i^{n} + (1-\gamma_i) \phi_i^{n}(K)
\end{equation}
where \(\gamma_i\) is a mixing factor.
With the right choice of mixing factor \(\gamma_i\), this leads to a reduction of distributed averaging iterations \(K\) and improves behaviour of the algorithm when the norms \(\|\h_i\|\) are time-variant (see \autoref{sec:simulations}).
% The factor balances instantaneous values and the distributed averaging iterations, and we shoudl look at the two extreme cases for further insight.
% For \(\gamma_i = 0\), only instantaneous neighborhood averages are used, which effectively leads to suboptimal computation, as \(\hat{\phi}_1^{n} = \hat{\phi}_2^{n} = \ldots = \hat{\phi}_M^{n}\) for large enough \(n\) will never be reached.
% For \(\gamma_i=1\), only the distributed averaging iterations are considered, i.e., we effectively combine the per-frame secondary recursion \eqref{eq:adaptivenormest:distlincomb} from \(K\) to \(mK\) iterations.
% There however, no new data is introduced which leads to performance degradadation in the adaptive algorithm as it only considers inital estimates from \(m=1\), which might be far of the optimum.

\subsection[]{Adaptive mixing factor}
To avoid choosing a fixed value for \(\gamma_i\), we use information provided by \(\bar{\phi}_i^{n}\).
If the absolute difference between subsequent frames \(| \bar{\phi}_i^{n} - \bar{\phi}_i^{n-1} |\) nears 0, then this means that the estimation of \(\bar{\h}_i\) is converging to a steady state.
Is this the case, then the emphasis of the algorithm should lie on norm estimation, i.e. the distributed averaging recursion, which in turn means \(\gamma_i\) should go towards 0 as well.
We set \(\gamma_i^{n}\) proportional to the absolute difference between subsequent frames and set an upper limit at 1,
\begin{equation}
    \gamma_i^{n} = \min \left\lbrace \frac{| \bar{\phi}_i^{n} - \bar{\phi}_i^{n-1} |}{\bar{\phi}_i^{n-1}},\,1\right\rbrace.\label{eq:adaptivenormest:adaptivegamma}
\end{equation}
The initial value for \(\bar{\phi}_i^{n-1}\) for \(n=0\) can be set to an arbitrary real number, in our case 1.
% The combination of instantaneous values and distributed averaging estimate should lead to lower steady-state error when \(\gamma_i\) is chosen right, while with adaptive \(\gamma_i\), convergence speed should increase as well.
Refer to \autoref{alg:davg_norm_est} for the extended algorithm introduced in this section.

\begin{algorithm}[t]
    \caption{ADMM BSI with distributed-averaging-based adaptive estimation of norm values}\label{alg:davg_norm_est}
    \(\W \gets\) \eqref{eq:adaptivenormest:fdlaminprob}\;
    \(\bar{\phi}_i^{n-1} \gets 1, \forall i \in \Mset\)\;
    \For(){\(n=0\dots\)}
    {
        \For(){\(i \in \Mset\)}
        {
            % \(\cdots\)\\
            % \(\w_i^n \gets \operatorname{argmin}_{\w_i}\, \mathcal{L}_{\rho} (\w,\h^n,\uu^n)\)\;
            % Compute \(\bar{\h}_i^{n}\)\;
            % \(\cdots\)\\
            \emph{The steps before are as introduced in }\cite{blochbergerDBSI}\\
            \dotfill\\
            Transmit \(\bar{\h}_i^{n}\) to nodes  \(j \in \Nset_i\)\;
            \(\bar{\phi}_i^{n} \gets \sum_{j \in \Nset_i} W_{ij} \|\bar{\h}_i^n\|^2\)\;
            \(\gamma_i^{n} \gets \min \left\lbrace \frac{| \bar{\phi}_i^{n} - \bar{\phi}_i^{n-1} |}{\bar{\phi}_i^{n-1}},\,1\right\rbrace\)\;
            \eIf(){\(m = 0\)}
            {
                \(\phi_i^{n}(1) \gets \|\bar{\h}_i^n\|^2\)\;
            }
            {
                \(\phi_i^{n}(1) \gets \hat{\phi}_i^{n-1}(K)\)\;
            }
            \For(){\(k=0,\dots,K\)}
            {
                Transmit \(\phi_i^n(k)\) to nodes  \(j \in \Nset_i\)\;
                \(\phi_i^n({k+1}) \gets \sum_{j \in \Nset_i} W_{ij} \phi_j^n({k})\)\;
            }
            \(\hat{\phi}_i^{n} \gets \gamma_i \bar{\phi}_i^{n} + (1-\gamma) \phi_i^{n}(K)\)\;
            \(\hat{\h_i}^{n} \gets \frac{\bar{\h}_i^{n}}{\sqrt{\hat{\phi}_i^{n}M}}\)\;
            \dotfill\\
            \emph{The steps after are as introduced in }\cite{blochbergerDBSI}\\
            % \(\cdots\)\\
            % \(\uu_i^{n} \gets \uu_i^{n-1} + \rho \left( \w_i^{n} - \tilde{\h}_i^{n} \right)\)\;
            % \(\cdots\)\\
            % % \,\\
        }
    }
\end{algorithm}

% \section{Transmission cost}


\section{Evaluation}
\label{sec:simulations}
\subsection[]{Communication cost}
\label{sec:transcost}
% \renewcommand{\arraystretch}{1.2}
% \begin{table}[t]
%     \centering
%     \begin{tabular}{ |l|l| } 
%         \hline
%         & Cost \\
%         \hline\hline
%         (1) Fully connected & \(\mathcal{O}(M)\) \\
%         \hline
%         (2) Broadcast & \(\mathcal{O}(M)\) \\ 
%         \hline
%         (3) Neighborhood & \(\mathcal{O}(\bar{N}_i K)\) \\ 
%         \hline
%     \end{tabular}
%     \caption[]{Comparison of communication cost.}
%     \label{tab:transcost:table}
% \end{table}
% \renewcommand{\arraystretch}{1.0}
\renewcommand{\arraystretch}{1.2}
\begin{table}[t]
    \centering
    \begin{tabular}{ |l|l|l|l| }
        \hline
        & Trans. Ops. & Rec. Ops. & Local Compl. \\
        \hline\hline
        (1) Fully connected & \(M-1\) & \(M-1\) & \(\mathcal{O}(M^2 L^2)\) \\
        \hline
        (2) Broadcast & \(1\) & \(M-1\) & \(\mathcal{O}(N_i^2 L^2)\)  \\ 
        \hline
        (3) Neighborhood & \(N_i K\) & \(N_i K\) & \(\mathcal{O}(N_i^2 L^2)\)\\ 
        \hline
    \end{tabular}
    % \caption[]{Comparison of communication cost (transmission operations \& receive operations) and local problem complexity for a node \(i\) within the network. \(M\) is the number of nodes within the network, \(N_i\) is the number neighboring nodes, \(L\) is the lenght of channel responses, \(K\) is the number of distributed averaging iterations.}
    \caption[]{Comparison of communication cost (transmission operations \& receive operations) and local problem complexity for a node \(i\) within the network.}
    \label{tab:transcost:table}
\end{table}
\renewcommand{\arraystretch}{1.0}
To describe the communication cost within the network, we count the number of variables transmitted per time frame \(n\).
The most useful measure in the case of this work is to analyze the cost per node.
% For this we define an average neighborhood size \(\bar{N}_i = \frac{1}{M} \sum_{i \in \Mset} N_i\) where \(N_i = | \Nset_i |\) is the cardinality/size of the neigborhood set for node \(i\).
We compare the following communication schemes within the context of the algorithm \cite{blochbergerDBSI} this paper proposes the extension to:
\begin{itemize}
    \itemsep-0.2em
    \item[(1)] A fully connected network, where node \(i\) communicates \(\|\bar{\h}_i\|\) to all other nodes \(\{j \in \Nset_i = \Mset \setminus i\}\) directly, i.e., the neighborhood is the full network.
    \item[(2)] The node \(i\) broadcasts \(\|\bar{\h}_i\|\), i.e., it transmit to all other nodes \(\{j \in \Mset \setminus i\}\) without direct links.
    \item[(3)] The node \(i\) communicates \(\|\bar{\h}_i\|\) only to neighboring nodes \(\{j \in \Nset_i \subset \Mset \setminus i\}\) and applies \(K\) distributed averaging iterations.
\end{itemize}
In this comparison, we only consider communication necessary for the computation of \eqref{eq:online_admm:consensus_update} as the rest of the algorithm is the same for our options.
For (1) and (2), the cost per node increases linearly with network size \(M\), whereas for communication scheme (3), the cost is linearly dependent on neighborhood size \(N_i\) and iteration count \(K\) only.
The problem to be solved at each node \(i \in \Mset\) has complexity \(\mathcal{O}(M^2 L^2)\) for (1), with channel response length \(L\) (which means there is no reduction compared to a centralized approach).
For (2) and (3), local complexity is \(\mathcal{O}(N_i^2 L^2)\), where \(N_i\) is the neighborhood size.
Refer to \autoref{tab:transcost:table} for a direct comparison.
% To describe the communication cost within the network, we adopt the Big-\(\mathcal{O}\) notation, similar to the complexity analysis of algorithms.
% We consider the communication of a variable from one node to another as \(\mathcal{O}(1)\).
% The most useful measure in the case of this work is to analyze the cost per node.
% For this we define an average neighborhood size \(\bar{N}_i = \frac{1}{M} \sum_{i \in \Mset} N_i\) where \(N_i = | \Nset_i |\) is the cardinality/size of the neigborhood set for node \(i\).
% We compare the following communication schemes:
% \begin{itemize}
%     \itemsep-0.2em
%     \item[(1)] A fully connected network, i.e., all nodes are linked with all other nodes by direct links, \(\Nset_i = \Mset\).
%     \item[(2)] The nodes employ broadcasting, i.e., the nodes transmit to all other nodes without direct links.
%     \item[(3)] The nodes only communicate with neighboring nodes, \(\Nset_i \subset \Mset\).
% \end{itemize}
% For (1) and (2), the cost per node increases linearly with network size \(M\): \(\mathcal{O}(M)\).
% Using communication scheme (3), as in \cite{yuDistributedBlindSystem2014,liuDistributedBlindIdentification2016} and proposed as an extension to \cite{blochbergerDBSI} in this paper, cost is linearly dependent on neighborhood size \(N_i\) and iteration count \(K\) only: \(\mathcal{O}(\bar{N}_i K)\).
% \begin{todo}
%     Unsure, about this. Filips feedback: "Don't you want the complexity for the whole system here so that you can compare things? That us write the complexity for the fully connected system as well as the distributed systems, parametrized in such away that the reader can compare the two. Here, the reader sees \(M\) and \(\bar{N}_i,K\) and has no idea of how they relate to each other."\\
%     But this shows how the communication cost as defined here goes from \(\mathcal{O}(M)\) to \(\mathcal{O}(\bar{N}_i K)\).\\
%     Other option: not just communication cost: fully connected \(\mathcal{O}(M^2 L^2)\), broadcast \(\)
% \end{todo}
% Refer to \autoref{tab:transcost:table} for the collected orders of communication cost.
% \begin{figure}[t]
%     \centering
%     %% Creator: Matplotlib, PGF backend
%%
%% To include the figure in your LaTeX document, write
%%   \input{<filename>.pgf}
%%
%% Make sure the required packages are loaded in your preamble
%%   \usepackage{pgf}
%%
%% Also ensure that all the required font packages are loaded; for instance,
%% the lmodern package is sometimes necessary when using math font.
%%   \usepackage{lmodern}
%%
%% Figures using additional raster images can only be included by \input if
%% they are in the same directory as the main LaTeX file. For loading figures
%% from other directories you can use the `import` package
%%   \usepackage{import}
%%
%% and then include the figures with
%%   \import{<path to file>}{<filename>.pgf}
%%
%% Matplotlib used the following preamble
%%   \usepackage{fontspec}
%%
\begingroup%
\makeatletter%
\begin{pgfpicture}%
\pgfpathrectangle{\pgfpointorigin}{\pgfqpoint{3.220562in}{1.990417in}}%
\pgfusepath{use as bounding box, clip}%
\begin{pgfscope}%
\pgfsetbuttcap%
\pgfsetmiterjoin%
\definecolor{currentfill}{rgb}{1.000000,1.000000,1.000000}%
\pgfsetfillcolor{currentfill}%
\pgfsetlinewidth{0.000000pt}%
\definecolor{currentstroke}{rgb}{1.000000,1.000000,1.000000}%
\pgfsetstrokecolor{currentstroke}%
\pgfsetstrokeopacity{0.000000}%
\pgfsetdash{}{0pt}%
\pgfpathmoveto{\pgfqpoint{0.000000in}{0.000000in}}%
\pgfpathlineto{\pgfqpoint{3.220562in}{0.000000in}}%
\pgfpathlineto{\pgfqpoint{3.220562in}{1.990417in}}%
\pgfpathlineto{\pgfqpoint{0.000000in}{1.990417in}}%
\pgfpathlineto{\pgfqpoint{0.000000in}{0.000000in}}%
\pgfpathclose%
\pgfusepath{fill}%
\end{pgfscope}%
\begin{pgfscope}%
\pgfsetbuttcap%
\pgfsetmiterjoin%
\definecolor{currentfill}{rgb}{1.000000,1.000000,1.000000}%
\pgfsetfillcolor{currentfill}%
\pgfsetlinewidth{0.000000pt}%
\definecolor{currentstroke}{rgb}{0.000000,0.000000,0.000000}%
\pgfsetstrokecolor{currentstroke}%
\pgfsetstrokeopacity{0.000000}%
\pgfsetdash{}{0pt}%
\pgfpathmoveto{\pgfqpoint{0.562308in}{0.484444in}}%
\pgfpathlineto{\pgfqpoint{3.012228in}{0.484444in}}%
\pgfpathlineto{\pgfqpoint{3.012228in}{1.920972in}}%
\pgfpathlineto{\pgfqpoint{0.562308in}{1.920972in}}%
\pgfpathlineto{\pgfqpoint{0.562308in}{0.484444in}}%
\pgfpathclose%
\pgfusepath{fill}%
\end{pgfscope}%
\begin{pgfscope}%
\pgfpathrectangle{\pgfqpoint{0.562308in}{0.484444in}}{\pgfqpoint{2.449920in}{1.436528in}}%
\pgfusepath{clip}%
\pgfsetrectcap%
\pgfsetroundjoin%
\pgfsetlinewidth{0.803000pt}%
\definecolor{currentstroke}{rgb}{0.690196,0.690196,0.690196}%
\pgfsetstrokecolor{currentstroke}%
\pgfsetdash{}{0pt}%
\pgfpathmoveto{\pgfqpoint{0.562308in}{0.484444in}}%
\pgfpathlineto{\pgfqpoint{0.562308in}{1.920972in}}%
\pgfusepath{stroke}%
\end{pgfscope}%
\begin{pgfscope}%
\pgfsetbuttcap%
\pgfsetroundjoin%
\definecolor{currentfill}{rgb}{0.000000,0.000000,0.000000}%
\pgfsetfillcolor{currentfill}%
\pgfsetlinewidth{0.803000pt}%
\definecolor{currentstroke}{rgb}{0.000000,0.000000,0.000000}%
\pgfsetstrokecolor{currentstroke}%
\pgfsetdash{}{0pt}%
\pgfsys@defobject{currentmarker}{\pgfqpoint{0.000000in}{-0.048611in}}{\pgfqpoint{0.000000in}{0.000000in}}{%
\pgfpathmoveto{\pgfqpoint{0.000000in}{0.000000in}}%
\pgfpathlineto{\pgfqpoint{0.000000in}{-0.048611in}}%
\pgfusepath{stroke,fill}%
}%
\begin{pgfscope}%
\pgfsys@transformshift{0.562308in}{0.484444in}%
\pgfsys@useobject{currentmarker}{}%
\end{pgfscope}%
\end{pgfscope}%
\begin{pgfscope}%
\definecolor{textcolor}{rgb}{0.000000,0.000000,0.000000}%
\pgfsetstrokecolor{textcolor}%
\pgfsetfillcolor{textcolor}%
\pgftext[x=0.562308in,y=0.387222in,,top]{\color{textcolor}\rmfamily\fontsize{10.000000}{12.000000}\selectfont \(\displaystyle {0}\)}%
\end{pgfscope}%
\begin{pgfscope}%
\pgfpathrectangle{\pgfqpoint{0.562308in}{0.484444in}}{\pgfqpoint{2.449920in}{1.436528in}}%
\pgfusepath{clip}%
\pgfsetrectcap%
\pgfsetroundjoin%
\pgfsetlinewidth{0.803000pt}%
\definecolor{currentstroke}{rgb}{0.690196,0.690196,0.690196}%
\pgfsetstrokecolor{currentstroke}%
\pgfsetdash{}{0pt}%
\pgfpathmoveto{\pgfqpoint{1.052292in}{0.484444in}}%
\pgfpathlineto{\pgfqpoint{1.052292in}{1.920972in}}%
\pgfusepath{stroke}%
\end{pgfscope}%
\begin{pgfscope}%
\pgfsetbuttcap%
\pgfsetroundjoin%
\definecolor{currentfill}{rgb}{0.000000,0.000000,0.000000}%
\pgfsetfillcolor{currentfill}%
\pgfsetlinewidth{0.803000pt}%
\definecolor{currentstroke}{rgb}{0.000000,0.000000,0.000000}%
\pgfsetstrokecolor{currentstroke}%
\pgfsetdash{}{0pt}%
\pgfsys@defobject{currentmarker}{\pgfqpoint{0.000000in}{-0.048611in}}{\pgfqpoint{0.000000in}{0.000000in}}{%
\pgfpathmoveto{\pgfqpoint{0.000000in}{0.000000in}}%
\pgfpathlineto{\pgfqpoint{0.000000in}{-0.048611in}}%
\pgfusepath{stroke,fill}%
}%
\begin{pgfscope}%
\pgfsys@transformshift{1.052292in}{0.484444in}%
\pgfsys@useobject{currentmarker}{}%
\end{pgfscope}%
\end{pgfscope}%
\begin{pgfscope}%
\definecolor{textcolor}{rgb}{0.000000,0.000000,0.000000}%
\pgfsetstrokecolor{textcolor}%
\pgfsetfillcolor{textcolor}%
\pgftext[x=1.052292in,y=0.387222in,,top]{\color{textcolor}\rmfamily\fontsize{10.000000}{12.000000}\selectfont \(\displaystyle {1000}\)}%
\end{pgfscope}%
\begin{pgfscope}%
\pgfpathrectangle{\pgfqpoint{0.562308in}{0.484444in}}{\pgfqpoint{2.449920in}{1.436528in}}%
\pgfusepath{clip}%
\pgfsetrectcap%
\pgfsetroundjoin%
\pgfsetlinewidth{0.803000pt}%
\definecolor{currentstroke}{rgb}{0.690196,0.690196,0.690196}%
\pgfsetstrokecolor{currentstroke}%
\pgfsetdash{}{0pt}%
\pgfpathmoveto{\pgfqpoint{1.542276in}{0.484444in}}%
\pgfpathlineto{\pgfqpoint{1.542276in}{1.920972in}}%
\pgfusepath{stroke}%
\end{pgfscope}%
\begin{pgfscope}%
\pgfsetbuttcap%
\pgfsetroundjoin%
\definecolor{currentfill}{rgb}{0.000000,0.000000,0.000000}%
\pgfsetfillcolor{currentfill}%
\pgfsetlinewidth{0.803000pt}%
\definecolor{currentstroke}{rgb}{0.000000,0.000000,0.000000}%
\pgfsetstrokecolor{currentstroke}%
\pgfsetdash{}{0pt}%
\pgfsys@defobject{currentmarker}{\pgfqpoint{0.000000in}{-0.048611in}}{\pgfqpoint{0.000000in}{0.000000in}}{%
\pgfpathmoveto{\pgfqpoint{0.000000in}{0.000000in}}%
\pgfpathlineto{\pgfqpoint{0.000000in}{-0.048611in}}%
\pgfusepath{stroke,fill}%
}%
\begin{pgfscope}%
\pgfsys@transformshift{1.542276in}{0.484444in}%
\pgfsys@useobject{currentmarker}{}%
\end{pgfscope}%
\end{pgfscope}%
\begin{pgfscope}%
\definecolor{textcolor}{rgb}{0.000000,0.000000,0.000000}%
\pgfsetstrokecolor{textcolor}%
\pgfsetfillcolor{textcolor}%
\pgftext[x=1.542276in,y=0.387222in,,top]{\color{textcolor}\rmfamily\fontsize{10.000000}{12.000000}\selectfont \(\displaystyle {2000}\)}%
\end{pgfscope}%
\begin{pgfscope}%
\pgfpathrectangle{\pgfqpoint{0.562308in}{0.484444in}}{\pgfqpoint{2.449920in}{1.436528in}}%
\pgfusepath{clip}%
\pgfsetrectcap%
\pgfsetroundjoin%
\pgfsetlinewidth{0.803000pt}%
\definecolor{currentstroke}{rgb}{0.690196,0.690196,0.690196}%
\pgfsetstrokecolor{currentstroke}%
\pgfsetdash{}{0pt}%
\pgfpathmoveto{\pgfqpoint{2.032260in}{0.484444in}}%
\pgfpathlineto{\pgfqpoint{2.032260in}{1.920972in}}%
\pgfusepath{stroke}%
\end{pgfscope}%
\begin{pgfscope}%
\pgfsetbuttcap%
\pgfsetroundjoin%
\definecolor{currentfill}{rgb}{0.000000,0.000000,0.000000}%
\pgfsetfillcolor{currentfill}%
\pgfsetlinewidth{0.803000pt}%
\definecolor{currentstroke}{rgb}{0.000000,0.000000,0.000000}%
\pgfsetstrokecolor{currentstroke}%
\pgfsetdash{}{0pt}%
\pgfsys@defobject{currentmarker}{\pgfqpoint{0.000000in}{-0.048611in}}{\pgfqpoint{0.000000in}{0.000000in}}{%
\pgfpathmoveto{\pgfqpoint{0.000000in}{0.000000in}}%
\pgfpathlineto{\pgfqpoint{0.000000in}{-0.048611in}}%
\pgfusepath{stroke,fill}%
}%
\begin{pgfscope}%
\pgfsys@transformshift{2.032260in}{0.484444in}%
\pgfsys@useobject{currentmarker}{}%
\end{pgfscope}%
\end{pgfscope}%
\begin{pgfscope}%
\definecolor{textcolor}{rgb}{0.000000,0.000000,0.000000}%
\pgfsetstrokecolor{textcolor}%
\pgfsetfillcolor{textcolor}%
\pgftext[x=2.032260in,y=0.387222in,,top]{\color{textcolor}\rmfamily\fontsize{10.000000}{12.000000}\selectfont \(\displaystyle {3000}\)}%
\end{pgfscope}%
\begin{pgfscope}%
\pgfpathrectangle{\pgfqpoint{0.562308in}{0.484444in}}{\pgfqpoint{2.449920in}{1.436528in}}%
\pgfusepath{clip}%
\pgfsetrectcap%
\pgfsetroundjoin%
\pgfsetlinewidth{0.803000pt}%
\definecolor{currentstroke}{rgb}{0.690196,0.690196,0.690196}%
\pgfsetstrokecolor{currentstroke}%
\pgfsetdash{}{0pt}%
\pgfpathmoveto{\pgfqpoint{2.522244in}{0.484444in}}%
\pgfpathlineto{\pgfqpoint{2.522244in}{1.920972in}}%
\pgfusepath{stroke}%
\end{pgfscope}%
\begin{pgfscope}%
\pgfsetbuttcap%
\pgfsetroundjoin%
\definecolor{currentfill}{rgb}{0.000000,0.000000,0.000000}%
\pgfsetfillcolor{currentfill}%
\pgfsetlinewidth{0.803000pt}%
\definecolor{currentstroke}{rgb}{0.000000,0.000000,0.000000}%
\pgfsetstrokecolor{currentstroke}%
\pgfsetdash{}{0pt}%
\pgfsys@defobject{currentmarker}{\pgfqpoint{0.000000in}{-0.048611in}}{\pgfqpoint{0.000000in}{0.000000in}}{%
\pgfpathmoveto{\pgfqpoint{0.000000in}{0.000000in}}%
\pgfpathlineto{\pgfqpoint{0.000000in}{-0.048611in}}%
\pgfusepath{stroke,fill}%
}%
\begin{pgfscope}%
\pgfsys@transformshift{2.522244in}{0.484444in}%
\pgfsys@useobject{currentmarker}{}%
\end{pgfscope}%
\end{pgfscope}%
\begin{pgfscope}%
\definecolor{textcolor}{rgb}{0.000000,0.000000,0.000000}%
\pgfsetstrokecolor{textcolor}%
\pgfsetfillcolor{textcolor}%
\pgftext[x=2.522244in,y=0.387222in,,top]{\color{textcolor}\rmfamily\fontsize{10.000000}{12.000000}\selectfont \(\displaystyle {4000}\)}%
\end{pgfscope}%
\begin{pgfscope}%
\pgfpathrectangle{\pgfqpoint{0.562308in}{0.484444in}}{\pgfqpoint{2.449920in}{1.436528in}}%
\pgfusepath{clip}%
\pgfsetrectcap%
\pgfsetroundjoin%
\pgfsetlinewidth{0.803000pt}%
\definecolor{currentstroke}{rgb}{0.690196,0.690196,0.690196}%
\pgfsetstrokecolor{currentstroke}%
\pgfsetdash{}{0pt}%
\pgfpathmoveto{\pgfqpoint{3.012228in}{0.484444in}}%
\pgfpathlineto{\pgfqpoint{3.012228in}{1.920972in}}%
\pgfusepath{stroke}%
\end{pgfscope}%
\begin{pgfscope}%
\pgfsetbuttcap%
\pgfsetroundjoin%
\definecolor{currentfill}{rgb}{0.000000,0.000000,0.000000}%
\pgfsetfillcolor{currentfill}%
\pgfsetlinewidth{0.803000pt}%
\definecolor{currentstroke}{rgb}{0.000000,0.000000,0.000000}%
\pgfsetstrokecolor{currentstroke}%
\pgfsetdash{}{0pt}%
\pgfsys@defobject{currentmarker}{\pgfqpoint{0.000000in}{-0.048611in}}{\pgfqpoint{0.000000in}{0.000000in}}{%
\pgfpathmoveto{\pgfqpoint{0.000000in}{0.000000in}}%
\pgfpathlineto{\pgfqpoint{0.000000in}{-0.048611in}}%
\pgfusepath{stroke,fill}%
}%
\begin{pgfscope}%
\pgfsys@transformshift{3.012228in}{0.484444in}%
\pgfsys@useobject{currentmarker}{}%
\end{pgfscope}%
\end{pgfscope}%
\begin{pgfscope}%
\definecolor{textcolor}{rgb}{0.000000,0.000000,0.000000}%
\pgfsetstrokecolor{textcolor}%
\pgfsetfillcolor{textcolor}%
\pgftext[x=3.012228in,y=0.387222in,,top]{\color{textcolor}\rmfamily\fontsize{10.000000}{12.000000}\selectfont \(\displaystyle {5000}\)}%
\end{pgfscope}%
\begin{pgfscope}%
\definecolor{textcolor}{rgb}{0.000000,0.000000,0.000000}%
\pgfsetstrokecolor{textcolor}%
\pgfsetfillcolor{textcolor}%
\pgftext[x=1.787268in,y=0.208333in,,top]{\color{textcolor}\rmfamily\fontsize{10.000000}{12.000000}\selectfont Network size M [1]}%
\end{pgfscope}%
\begin{pgfscope}%
\pgfpathrectangle{\pgfqpoint{0.562308in}{0.484444in}}{\pgfqpoint{2.449920in}{1.436528in}}%
\pgfusepath{clip}%
\pgfsetrectcap%
\pgfsetroundjoin%
\pgfsetlinewidth{0.803000pt}%
\definecolor{currentstroke}{rgb}{0.690196,0.690196,0.690196}%
\pgfsetstrokecolor{currentstroke}%
\pgfsetdash{}{0pt}%
\pgfpathmoveto{\pgfqpoint{0.562308in}{0.549741in}}%
\pgfpathlineto{\pgfqpoint{3.012228in}{0.549741in}}%
\pgfusepath{stroke}%
\end{pgfscope}%
\begin{pgfscope}%
\pgfsetbuttcap%
\pgfsetroundjoin%
\definecolor{currentfill}{rgb}{0.000000,0.000000,0.000000}%
\pgfsetfillcolor{currentfill}%
\pgfsetlinewidth{0.803000pt}%
\definecolor{currentstroke}{rgb}{0.000000,0.000000,0.000000}%
\pgfsetstrokecolor{currentstroke}%
\pgfsetdash{}{0pt}%
\pgfsys@defobject{currentmarker}{\pgfqpoint{-0.048611in}{0.000000in}}{\pgfqpoint{-0.000000in}{0.000000in}}{%
\pgfpathmoveto{\pgfqpoint{-0.000000in}{0.000000in}}%
\pgfpathlineto{\pgfqpoint{-0.048611in}{0.000000in}}%
\pgfusepath{stroke,fill}%
}%
\begin{pgfscope}%
\pgfsys@transformshift{0.562308in}{0.549741in}%
\pgfsys@useobject{currentmarker}{}%
\end{pgfscope}%
\end{pgfscope}%
\begin{pgfscope}%
\definecolor{textcolor}{rgb}{0.000000,0.000000,0.000000}%
\pgfsetstrokecolor{textcolor}%
\pgfsetfillcolor{textcolor}%
\pgftext[x=0.263889in, y=0.501546in, left, base]{\color{textcolor}\rmfamily\fontsize{10.000000}{12.000000}\selectfont \(\displaystyle {10^{1}}\)}%
\end{pgfscope}%
\begin{pgfscope}%
\pgfpathrectangle{\pgfqpoint{0.562308in}{0.484444in}}{\pgfqpoint{2.449920in}{1.436528in}}%
\pgfusepath{clip}%
\pgfsetrectcap%
\pgfsetroundjoin%
\pgfsetlinewidth{0.803000pt}%
\definecolor{currentstroke}{rgb}{0.690196,0.690196,0.690196}%
\pgfsetstrokecolor{currentstroke}%
\pgfsetdash{}{0pt}%
\pgfpathmoveto{\pgfqpoint{0.562308in}{0.985065in}}%
\pgfpathlineto{\pgfqpoint{3.012228in}{0.985065in}}%
\pgfusepath{stroke}%
\end{pgfscope}%
\begin{pgfscope}%
\pgfsetbuttcap%
\pgfsetroundjoin%
\definecolor{currentfill}{rgb}{0.000000,0.000000,0.000000}%
\pgfsetfillcolor{currentfill}%
\pgfsetlinewidth{0.803000pt}%
\definecolor{currentstroke}{rgb}{0.000000,0.000000,0.000000}%
\pgfsetstrokecolor{currentstroke}%
\pgfsetdash{}{0pt}%
\pgfsys@defobject{currentmarker}{\pgfqpoint{-0.048611in}{0.000000in}}{\pgfqpoint{-0.000000in}{0.000000in}}{%
\pgfpathmoveto{\pgfqpoint{-0.000000in}{0.000000in}}%
\pgfpathlineto{\pgfqpoint{-0.048611in}{0.000000in}}%
\pgfusepath{stroke,fill}%
}%
\begin{pgfscope}%
\pgfsys@transformshift{0.562308in}{0.985065in}%
\pgfsys@useobject{currentmarker}{}%
\end{pgfscope}%
\end{pgfscope}%
\begin{pgfscope}%
\definecolor{textcolor}{rgb}{0.000000,0.000000,0.000000}%
\pgfsetstrokecolor{textcolor}%
\pgfsetfillcolor{textcolor}%
\pgftext[x=0.263889in, y=0.936870in, left, base]{\color{textcolor}\rmfamily\fontsize{10.000000}{12.000000}\selectfont \(\displaystyle {10^{2}}\)}%
\end{pgfscope}%
\begin{pgfscope}%
\pgfpathrectangle{\pgfqpoint{0.562308in}{0.484444in}}{\pgfqpoint{2.449920in}{1.436528in}}%
\pgfusepath{clip}%
\pgfsetrectcap%
\pgfsetroundjoin%
\pgfsetlinewidth{0.803000pt}%
\definecolor{currentstroke}{rgb}{0.690196,0.690196,0.690196}%
\pgfsetstrokecolor{currentstroke}%
\pgfsetdash{}{0pt}%
\pgfpathmoveto{\pgfqpoint{0.562308in}{1.420389in}}%
\pgfpathlineto{\pgfqpoint{3.012228in}{1.420389in}}%
\pgfusepath{stroke}%
\end{pgfscope}%
\begin{pgfscope}%
\pgfsetbuttcap%
\pgfsetroundjoin%
\definecolor{currentfill}{rgb}{0.000000,0.000000,0.000000}%
\pgfsetfillcolor{currentfill}%
\pgfsetlinewidth{0.803000pt}%
\definecolor{currentstroke}{rgb}{0.000000,0.000000,0.000000}%
\pgfsetstrokecolor{currentstroke}%
\pgfsetdash{}{0pt}%
\pgfsys@defobject{currentmarker}{\pgfqpoint{-0.048611in}{0.000000in}}{\pgfqpoint{-0.000000in}{0.000000in}}{%
\pgfpathmoveto{\pgfqpoint{-0.000000in}{0.000000in}}%
\pgfpathlineto{\pgfqpoint{-0.048611in}{0.000000in}}%
\pgfusepath{stroke,fill}%
}%
\begin{pgfscope}%
\pgfsys@transformshift{0.562308in}{1.420389in}%
\pgfsys@useobject{currentmarker}{}%
\end{pgfscope}%
\end{pgfscope}%
\begin{pgfscope}%
\definecolor{textcolor}{rgb}{0.000000,0.000000,0.000000}%
\pgfsetstrokecolor{textcolor}%
\pgfsetfillcolor{textcolor}%
\pgftext[x=0.263889in, y=1.372195in, left, base]{\color{textcolor}\rmfamily\fontsize{10.000000}{12.000000}\selectfont \(\displaystyle {10^{3}}\)}%
\end{pgfscope}%
\begin{pgfscope}%
\pgfpathrectangle{\pgfqpoint{0.562308in}{0.484444in}}{\pgfqpoint{2.449920in}{1.436528in}}%
\pgfusepath{clip}%
\pgfsetrectcap%
\pgfsetroundjoin%
\pgfsetlinewidth{0.803000pt}%
\definecolor{currentstroke}{rgb}{0.690196,0.690196,0.690196}%
\pgfsetstrokecolor{currentstroke}%
\pgfsetdash{}{0pt}%
\pgfpathmoveto{\pgfqpoint{0.562308in}{1.855713in}}%
\pgfpathlineto{\pgfqpoint{3.012228in}{1.855713in}}%
\pgfusepath{stroke}%
\end{pgfscope}%
\begin{pgfscope}%
\pgfsetbuttcap%
\pgfsetroundjoin%
\definecolor{currentfill}{rgb}{0.000000,0.000000,0.000000}%
\pgfsetfillcolor{currentfill}%
\pgfsetlinewidth{0.803000pt}%
\definecolor{currentstroke}{rgb}{0.000000,0.000000,0.000000}%
\pgfsetstrokecolor{currentstroke}%
\pgfsetdash{}{0pt}%
\pgfsys@defobject{currentmarker}{\pgfqpoint{-0.048611in}{0.000000in}}{\pgfqpoint{-0.000000in}{0.000000in}}{%
\pgfpathmoveto{\pgfqpoint{-0.000000in}{0.000000in}}%
\pgfpathlineto{\pgfqpoint{-0.048611in}{0.000000in}}%
\pgfusepath{stroke,fill}%
}%
\begin{pgfscope}%
\pgfsys@transformshift{0.562308in}{1.855713in}%
\pgfsys@useobject{currentmarker}{}%
\end{pgfscope}%
\end{pgfscope}%
\begin{pgfscope}%
\definecolor{textcolor}{rgb}{0.000000,0.000000,0.000000}%
\pgfsetstrokecolor{textcolor}%
\pgfsetfillcolor{textcolor}%
\pgftext[x=0.263889in, y=1.807519in, left, base]{\color{textcolor}\rmfamily\fontsize{10.000000}{12.000000}\selectfont \(\displaystyle {10^{4}}\)}%
\end{pgfscope}%
\begin{pgfscope}%
\pgfsetbuttcap%
\pgfsetroundjoin%
\definecolor{currentfill}{rgb}{0.000000,0.000000,0.000000}%
\pgfsetfillcolor{currentfill}%
\pgfsetlinewidth{0.602250pt}%
\definecolor{currentstroke}{rgb}{0.000000,0.000000,0.000000}%
\pgfsetstrokecolor{currentstroke}%
\pgfsetdash{}{0pt}%
\pgfsys@defobject{currentmarker}{\pgfqpoint{-0.027778in}{0.000000in}}{\pgfqpoint{-0.000000in}{0.000000in}}{%
\pgfpathmoveto{\pgfqpoint{-0.000000in}{0.000000in}}%
\pgfpathlineto{\pgfqpoint{-0.027778in}{0.000000in}}%
\pgfusepath{stroke,fill}%
}%
\begin{pgfscope}%
\pgfsys@transformshift{0.562308in}{0.507553in}%
\pgfsys@useobject{currentmarker}{}%
\end{pgfscope}%
\end{pgfscope}%
\begin{pgfscope}%
\pgfsetbuttcap%
\pgfsetroundjoin%
\definecolor{currentfill}{rgb}{0.000000,0.000000,0.000000}%
\pgfsetfillcolor{currentfill}%
\pgfsetlinewidth{0.602250pt}%
\definecolor{currentstroke}{rgb}{0.000000,0.000000,0.000000}%
\pgfsetstrokecolor{currentstroke}%
\pgfsetdash{}{0pt}%
\pgfsys@defobject{currentmarker}{\pgfqpoint{-0.027778in}{0.000000in}}{\pgfqpoint{-0.000000in}{0.000000in}}{%
\pgfpathmoveto{\pgfqpoint{-0.000000in}{0.000000in}}%
\pgfpathlineto{\pgfqpoint{-0.027778in}{0.000000in}}%
\pgfusepath{stroke,fill}%
}%
\begin{pgfscope}%
\pgfsys@transformshift{0.562308in}{0.529821in}%
\pgfsys@useobject{currentmarker}{}%
\end{pgfscope}%
\end{pgfscope}%
\begin{pgfscope}%
\pgfsetbuttcap%
\pgfsetroundjoin%
\definecolor{currentfill}{rgb}{0.000000,0.000000,0.000000}%
\pgfsetfillcolor{currentfill}%
\pgfsetlinewidth{0.602250pt}%
\definecolor{currentstroke}{rgb}{0.000000,0.000000,0.000000}%
\pgfsetstrokecolor{currentstroke}%
\pgfsetdash{}{0pt}%
\pgfsys@defobject{currentmarker}{\pgfqpoint{-0.027778in}{0.000000in}}{\pgfqpoint{-0.000000in}{0.000000in}}{%
\pgfpathmoveto{\pgfqpoint{-0.000000in}{0.000000in}}%
\pgfpathlineto{\pgfqpoint{-0.027778in}{0.000000in}}%
\pgfusepath{stroke,fill}%
}%
\begin{pgfscope}%
\pgfsys@transformshift{0.562308in}{0.680786in}%
\pgfsys@useobject{currentmarker}{}%
\end{pgfscope}%
\end{pgfscope}%
\begin{pgfscope}%
\pgfsetbuttcap%
\pgfsetroundjoin%
\definecolor{currentfill}{rgb}{0.000000,0.000000,0.000000}%
\pgfsetfillcolor{currentfill}%
\pgfsetlinewidth{0.602250pt}%
\definecolor{currentstroke}{rgb}{0.000000,0.000000,0.000000}%
\pgfsetstrokecolor{currentstroke}%
\pgfsetdash{}{0pt}%
\pgfsys@defobject{currentmarker}{\pgfqpoint{-0.027778in}{0.000000in}}{\pgfqpoint{-0.000000in}{0.000000in}}{%
\pgfpathmoveto{\pgfqpoint{-0.000000in}{0.000000in}}%
\pgfpathlineto{\pgfqpoint{-0.027778in}{0.000000in}}%
\pgfusepath{stroke,fill}%
}%
\begin{pgfscope}%
\pgfsys@transformshift{0.562308in}{0.757443in}%
\pgfsys@useobject{currentmarker}{}%
\end{pgfscope}%
\end{pgfscope}%
\begin{pgfscope}%
\pgfsetbuttcap%
\pgfsetroundjoin%
\definecolor{currentfill}{rgb}{0.000000,0.000000,0.000000}%
\pgfsetfillcolor{currentfill}%
\pgfsetlinewidth{0.602250pt}%
\definecolor{currentstroke}{rgb}{0.000000,0.000000,0.000000}%
\pgfsetstrokecolor{currentstroke}%
\pgfsetdash{}{0pt}%
\pgfsys@defobject{currentmarker}{\pgfqpoint{-0.027778in}{0.000000in}}{\pgfqpoint{-0.000000in}{0.000000in}}{%
\pgfpathmoveto{\pgfqpoint{-0.000000in}{0.000000in}}%
\pgfpathlineto{\pgfqpoint{-0.027778in}{0.000000in}}%
\pgfusepath{stroke,fill}%
}%
\begin{pgfscope}%
\pgfsys@transformshift{0.562308in}{0.811832in}%
\pgfsys@useobject{currentmarker}{}%
\end{pgfscope}%
\end{pgfscope}%
\begin{pgfscope}%
\pgfsetbuttcap%
\pgfsetroundjoin%
\definecolor{currentfill}{rgb}{0.000000,0.000000,0.000000}%
\pgfsetfillcolor{currentfill}%
\pgfsetlinewidth{0.602250pt}%
\definecolor{currentstroke}{rgb}{0.000000,0.000000,0.000000}%
\pgfsetstrokecolor{currentstroke}%
\pgfsetdash{}{0pt}%
\pgfsys@defobject{currentmarker}{\pgfqpoint{-0.027778in}{0.000000in}}{\pgfqpoint{-0.000000in}{0.000000in}}{%
\pgfpathmoveto{\pgfqpoint{-0.000000in}{0.000000in}}%
\pgfpathlineto{\pgfqpoint{-0.027778in}{0.000000in}}%
\pgfusepath{stroke,fill}%
}%
\begin{pgfscope}%
\pgfsys@transformshift{0.562308in}{0.854019in}%
\pgfsys@useobject{currentmarker}{}%
\end{pgfscope}%
\end{pgfscope}%
\begin{pgfscope}%
\pgfsetbuttcap%
\pgfsetroundjoin%
\definecolor{currentfill}{rgb}{0.000000,0.000000,0.000000}%
\pgfsetfillcolor{currentfill}%
\pgfsetlinewidth{0.602250pt}%
\definecolor{currentstroke}{rgb}{0.000000,0.000000,0.000000}%
\pgfsetstrokecolor{currentstroke}%
\pgfsetdash{}{0pt}%
\pgfsys@defobject{currentmarker}{\pgfqpoint{-0.027778in}{0.000000in}}{\pgfqpoint{-0.000000in}{0.000000in}}{%
\pgfpathmoveto{\pgfqpoint{-0.000000in}{0.000000in}}%
\pgfpathlineto{\pgfqpoint{-0.027778in}{0.000000in}}%
\pgfusepath{stroke,fill}%
}%
\begin{pgfscope}%
\pgfsys@transformshift{0.562308in}{0.888489in}%
\pgfsys@useobject{currentmarker}{}%
\end{pgfscope}%
\end{pgfscope}%
\begin{pgfscope}%
\pgfsetbuttcap%
\pgfsetroundjoin%
\definecolor{currentfill}{rgb}{0.000000,0.000000,0.000000}%
\pgfsetfillcolor{currentfill}%
\pgfsetlinewidth{0.602250pt}%
\definecolor{currentstroke}{rgb}{0.000000,0.000000,0.000000}%
\pgfsetstrokecolor{currentstroke}%
\pgfsetdash{}{0pt}%
\pgfsys@defobject{currentmarker}{\pgfqpoint{-0.027778in}{0.000000in}}{\pgfqpoint{-0.000000in}{0.000000in}}{%
\pgfpathmoveto{\pgfqpoint{-0.000000in}{0.000000in}}%
\pgfpathlineto{\pgfqpoint{-0.027778in}{0.000000in}}%
\pgfusepath{stroke,fill}%
}%
\begin{pgfscope}%
\pgfsys@transformshift{0.562308in}{0.917632in}%
\pgfsys@useobject{currentmarker}{}%
\end{pgfscope}%
\end{pgfscope}%
\begin{pgfscope}%
\pgfsetbuttcap%
\pgfsetroundjoin%
\definecolor{currentfill}{rgb}{0.000000,0.000000,0.000000}%
\pgfsetfillcolor{currentfill}%
\pgfsetlinewidth{0.602250pt}%
\definecolor{currentstroke}{rgb}{0.000000,0.000000,0.000000}%
\pgfsetstrokecolor{currentstroke}%
\pgfsetdash{}{0pt}%
\pgfsys@defobject{currentmarker}{\pgfqpoint{-0.027778in}{0.000000in}}{\pgfqpoint{-0.000000in}{0.000000in}}{%
\pgfpathmoveto{\pgfqpoint{-0.000000in}{0.000000in}}%
\pgfpathlineto{\pgfqpoint{-0.027778in}{0.000000in}}%
\pgfusepath{stroke,fill}%
}%
\begin{pgfscope}%
\pgfsys@transformshift{0.562308in}{0.942878in}%
\pgfsys@useobject{currentmarker}{}%
\end{pgfscope}%
\end{pgfscope}%
\begin{pgfscope}%
\pgfsetbuttcap%
\pgfsetroundjoin%
\definecolor{currentfill}{rgb}{0.000000,0.000000,0.000000}%
\pgfsetfillcolor{currentfill}%
\pgfsetlinewidth{0.602250pt}%
\definecolor{currentstroke}{rgb}{0.000000,0.000000,0.000000}%
\pgfsetstrokecolor{currentstroke}%
\pgfsetdash{}{0pt}%
\pgfsys@defobject{currentmarker}{\pgfqpoint{-0.027778in}{0.000000in}}{\pgfqpoint{-0.000000in}{0.000000in}}{%
\pgfpathmoveto{\pgfqpoint{-0.000000in}{0.000000in}}%
\pgfpathlineto{\pgfqpoint{-0.027778in}{0.000000in}}%
\pgfusepath{stroke,fill}%
}%
\begin{pgfscope}%
\pgfsys@transformshift{0.562308in}{0.965146in}%
\pgfsys@useobject{currentmarker}{}%
\end{pgfscope}%
\end{pgfscope}%
\begin{pgfscope}%
\pgfsetbuttcap%
\pgfsetroundjoin%
\definecolor{currentfill}{rgb}{0.000000,0.000000,0.000000}%
\pgfsetfillcolor{currentfill}%
\pgfsetlinewidth{0.602250pt}%
\definecolor{currentstroke}{rgb}{0.000000,0.000000,0.000000}%
\pgfsetstrokecolor{currentstroke}%
\pgfsetdash{}{0pt}%
\pgfsys@defobject{currentmarker}{\pgfqpoint{-0.027778in}{0.000000in}}{\pgfqpoint{-0.000000in}{0.000000in}}{%
\pgfpathmoveto{\pgfqpoint{-0.000000in}{0.000000in}}%
\pgfpathlineto{\pgfqpoint{-0.027778in}{0.000000in}}%
\pgfusepath{stroke,fill}%
}%
\begin{pgfscope}%
\pgfsys@transformshift{0.562308in}{1.116111in}%
\pgfsys@useobject{currentmarker}{}%
\end{pgfscope}%
\end{pgfscope}%
\begin{pgfscope}%
\pgfsetbuttcap%
\pgfsetroundjoin%
\definecolor{currentfill}{rgb}{0.000000,0.000000,0.000000}%
\pgfsetfillcolor{currentfill}%
\pgfsetlinewidth{0.602250pt}%
\definecolor{currentstroke}{rgb}{0.000000,0.000000,0.000000}%
\pgfsetstrokecolor{currentstroke}%
\pgfsetdash{}{0pt}%
\pgfsys@defobject{currentmarker}{\pgfqpoint{-0.027778in}{0.000000in}}{\pgfqpoint{-0.000000in}{0.000000in}}{%
\pgfpathmoveto{\pgfqpoint{-0.000000in}{0.000000in}}%
\pgfpathlineto{\pgfqpoint{-0.027778in}{0.000000in}}%
\pgfusepath{stroke,fill}%
}%
\begin{pgfscope}%
\pgfsys@transformshift{0.562308in}{1.192767in}%
\pgfsys@useobject{currentmarker}{}%
\end{pgfscope}%
\end{pgfscope}%
\begin{pgfscope}%
\pgfsetbuttcap%
\pgfsetroundjoin%
\definecolor{currentfill}{rgb}{0.000000,0.000000,0.000000}%
\pgfsetfillcolor{currentfill}%
\pgfsetlinewidth{0.602250pt}%
\definecolor{currentstroke}{rgb}{0.000000,0.000000,0.000000}%
\pgfsetstrokecolor{currentstroke}%
\pgfsetdash{}{0pt}%
\pgfsys@defobject{currentmarker}{\pgfqpoint{-0.027778in}{0.000000in}}{\pgfqpoint{-0.000000in}{0.000000in}}{%
\pgfpathmoveto{\pgfqpoint{-0.000000in}{0.000000in}}%
\pgfpathlineto{\pgfqpoint{-0.027778in}{0.000000in}}%
\pgfusepath{stroke,fill}%
}%
\begin{pgfscope}%
\pgfsys@transformshift{0.562308in}{1.247156in}%
\pgfsys@useobject{currentmarker}{}%
\end{pgfscope}%
\end{pgfscope}%
\begin{pgfscope}%
\pgfsetbuttcap%
\pgfsetroundjoin%
\definecolor{currentfill}{rgb}{0.000000,0.000000,0.000000}%
\pgfsetfillcolor{currentfill}%
\pgfsetlinewidth{0.602250pt}%
\definecolor{currentstroke}{rgb}{0.000000,0.000000,0.000000}%
\pgfsetstrokecolor{currentstroke}%
\pgfsetdash{}{0pt}%
\pgfsys@defobject{currentmarker}{\pgfqpoint{-0.027778in}{0.000000in}}{\pgfqpoint{-0.000000in}{0.000000in}}{%
\pgfpathmoveto{\pgfqpoint{-0.000000in}{0.000000in}}%
\pgfpathlineto{\pgfqpoint{-0.027778in}{0.000000in}}%
\pgfusepath{stroke,fill}%
}%
\begin{pgfscope}%
\pgfsys@transformshift{0.562308in}{1.289343in}%
\pgfsys@useobject{currentmarker}{}%
\end{pgfscope}%
\end{pgfscope}%
\begin{pgfscope}%
\pgfsetbuttcap%
\pgfsetroundjoin%
\definecolor{currentfill}{rgb}{0.000000,0.000000,0.000000}%
\pgfsetfillcolor{currentfill}%
\pgfsetlinewidth{0.602250pt}%
\definecolor{currentstroke}{rgb}{0.000000,0.000000,0.000000}%
\pgfsetstrokecolor{currentstroke}%
\pgfsetdash{}{0pt}%
\pgfsys@defobject{currentmarker}{\pgfqpoint{-0.027778in}{0.000000in}}{\pgfqpoint{-0.000000in}{0.000000in}}{%
\pgfpathmoveto{\pgfqpoint{-0.000000in}{0.000000in}}%
\pgfpathlineto{\pgfqpoint{-0.027778in}{0.000000in}}%
\pgfusepath{stroke,fill}%
}%
\begin{pgfscope}%
\pgfsys@transformshift{0.562308in}{1.323813in}%
\pgfsys@useobject{currentmarker}{}%
\end{pgfscope}%
\end{pgfscope}%
\begin{pgfscope}%
\pgfsetbuttcap%
\pgfsetroundjoin%
\definecolor{currentfill}{rgb}{0.000000,0.000000,0.000000}%
\pgfsetfillcolor{currentfill}%
\pgfsetlinewidth{0.602250pt}%
\definecolor{currentstroke}{rgb}{0.000000,0.000000,0.000000}%
\pgfsetstrokecolor{currentstroke}%
\pgfsetdash{}{0pt}%
\pgfsys@defobject{currentmarker}{\pgfqpoint{-0.027778in}{0.000000in}}{\pgfqpoint{-0.000000in}{0.000000in}}{%
\pgfpathmoveto{\pgfqpoint{-0.000000in}{0.000000in}}%
\pgfpathlineto{\pgfqpoint{-0.027778in}{0.000000in}}%
\pgfusepath{stroke,fill}%
}%
\begin{pgfscope}%
\pgfsys@transformshift{0.562308in}{1.352957in}%
\pgfsys@useobject{currentmarker}{}%
\end{pgfscope}%
\end{pgfscope}%
\begin{pgfscope}%
\pgfsetbuttcap%
\pgfsetroundjoin%
\definecolor{currentfill}{rgb}{0.000000,0.000000,0.000000}%
\pgfsetfillcolor{currentfill}%
\pgfsetlinewidth{0.602250pt}%
\definecolor{currentstroke}{rgb}{0.000000,0.000000,0.000000}%
\pgfsetstrokecolor{currentstroke}%
\pgfsetdash{}{0pt}%
\pgfsys@defobject{currentmarker}{\pgfqpoint{-0.027778in}{0.000000in}}{\pgfqpoint{-0.000000in}{0.000000in}}{%
\pgfpathmoveto{\pgfqpoint{-0.000000in}{0.000000in}}%
\pgfpathlineto{\pgfqpoint{-0.027778in}{0.000000in}}%
\pgfusepath{stroke,fill}%
}%
\begin{pgfscope}%
\pgfsys@transformshift{0.562308in}{1.378202in}%
\pgfsys@useobject{currentmarker}{}%
\end{pgfscope}%
\end{pgfscope}%
\begin{pgfscope}%
\pgfsetbuttcap%
\pgfsetroundjoin%
\definecolor{currentfill}{rgb}{0.000000,0.000000,0.000000}%
\pgfsetfillcolor{currentfill}%
\pgfsetlinewidth{0.602250pt}%
\definecolor{currentstroke}{rgb}{0.000000,0.000000,0.000000}%
\pgfsetstrokecolor{currentstroke}%
\pgfsetdash{}{0pt}%
\pgfsys@defobject{currentmarker}{\pgfqpoint{-0.027778in}{0.000000in}}{\pgfqpoint{-0.000000in}{0.000000in}}{%
\pgfpathmoveto{\pgfqpoint{-0.000000in}{0.000000in}}%
\pgfpathlineto{\pgfqpoint{-0.027778in}{0.000000in}}%
\pgfusepath{stroke,fill}%
}%
\begin{pgfscope}%
\pgfsys@transformshift{0.562308in}{1.400470in}%
\pgfsys@useobject{currentmarker}{}%
\end{pgfscope}%
\end{pgfscope}%
\begin{pgfscope}%
\pgfsetbuttcap%
\pgfsetroundjoin%
\definecolor{currentfill}{rgb}{0.000000,0.000000,0.000000}%
\pgfsetfillcolor{currentfill}%
\pgfsetlinewidth{0.602250pt}%
\definecolor{currentstroke}{rgb}{0.000000,0.000000,0.000000}%
\pgfsetstrokecolor{currentstroke}%
\pgfsetdash{}{0pt}%
\pgfsys@defobject{currentmarker}{\pgfqpoint{-0.027778in}{0.000000in}}{\pgfqpoint{-0.000000in}{0.000000in}}{%
\pgfpathmoveto{\pgfqpoint{-0.000000in}{0.000000in}}%
\pgfpathlineto{\pgfqpoint{-0.027778in}{0.000000in}}%
\pgfusepath{stroke,fill}%
}%
\begin{pgfscope}%
\pgfsys@transformshift{0.562308in}{1.551435in}%
\pgfsys@useobject{currentmarker}{}%
\end{pgfscope}%
\end{pgfscope}%
\begin{pgfscope}%
\pgfsetbuttcap%
\pgfsetroundjoin%
\definecolor{currentfill}{rgb}{0.000000,0.000000,0.000000}%
\pgfsetfillcolor{currentfill}%
\pgfsetlinewidth{0.602250pt}%
\definecolor{currentstroke}{rgb}{0.000000,0.000000,0.000000}%
\pgfsetstrokecolor{currentstroke}%
\pgfsetdash{}{0pt}%
\pgfsys@defobject{currentmarker}{\pgfqpoint{-0.027778in}{0.000000in}}{\pgfqpoint{-0.000000in}{0.000000in}}{%
\pgfpathmoveto{\pgfqpoint{-0.000000in}{0.000000in}}%
\pgfpathlineto{\pgfqpoint{-0.027778in}{0.000000in}}%
\pgfusepath{stroke,fill}%
}%
\begin{pgfscope}%
\pgfsys@transformshift{0.562308in}{1.628092in}%
\pgfsys@useobject{currentmarker}{}%
\end{pgfscope}%
\end{pgfscope}%
\begin{pgfscope}%
\pgfsetbuttcap%
\pgfsetroundjoin%
\definecolor{currentfill}{rgb}{0.000000,0.000000,0.000000}%
\pgfsetfillcolor{currentfill}%
\pgfsetlinewidth{0.602250pt}%
\definecolor{currentstroke}{rgb}{0.000000,0.000000,0.000000}%
\pgfsetstrokecolor{currentstroke}%
\pgfsetdash{}{0pt}%
\pgfsys@defobject{currentmarker}{\pgfqpoint{-0.027778in}{0.000000in}}{\pgfqpoint{-0.000000in}{0.000000in}}{%
\pgfpathmoveto{\pgfqpoint{-0.000000in}{0.000000in}}%
\pgfpathlineto{\pgfqpoint{-0.027778in}{0.000000in}}%
\pgfusepath{stroke,fill}%
}%
\begin{pgfscope}%
\pgfsys@transformshift{0.562308in}{1.682480in}%
\pgfsys@useobject{currentmarker}{}%
\end{pgfscope}%
\end{pgfscope}%
\begin{pgfscope}%
\pgfsetbuttcap%
\pgfsetroundjoin%
\definecolor{currentfill}{rgb}{0.000000,0.000000,0.000000}%
\pgfsetfillcolor{currentfill}%
\pgfsetlinewidth{0.602250pt}%
\definecolor{currentstroke}{rgb}{0.000000,0.000000,0.000000}%
\pgfsetstrokecolor{currentstroke}%
\pgfsetdash{}{0pt}%
\pgfsys@defobject{currentmarker}{\pgfqpoint{-0.027778in}{0.000000in}}{\pgfqpoint{-0.000000in}{0.000000in}}{%
\pgfpathmoveto{\pgfqpoint{-0.000000in}{0.000000in}}%
\pgfpathlineto{\pgfqpoint{-0.027778in}{0.000000in}}%
\pgfusepath{stroke,fill}%
}%
\begin{pgfscope}%
\pgfsys@transformshift{0.562308in}{1.724668in}%
\pgfsys@useobject{currentmarker}{}%
\end{pgfscope}%
\end{pgfscope}%
\begin{pgfscope}%
\pgfsetbuttcap%
\pgfsetroundjoin%
\definecolor{currentfill}{rgb}{0.000000,0.000000,0.000000}%
\pgfsetfillcolor{currentfill}%
\pgfsetlinewidth{0.602250pt}%
\definecolor{currentstroke}{rgb}{0.000000,0.000000,0.000000}%
\pgfsetstrokecolor{currentstroke}%
\pgfsetdash{}{0pt}%
\pgfsys@defobject{currentmarker}{\pgfqpoint{-0.027778in}{0.000000in}}{\pgfqpoint{-0.000000in}{0.000000in}}{%
\pgfpathmoveto{\pgfqpoint{-0.000000in}{0.000000in}}%
\pgfpathlineto{\pgfqpoint{-0.027778in}{0.000000in}}%
\pgfusepath{stroke,fill}%
}%
\begin{pgfscope}%
\pgfsys@transformshift{0.562308in}{1.759137in}%
\pgfsys@useobject{currentmarker}{}%
\end{pgfscope}%
\end{pgfscope}%
\begin{pgfscope}%
\pgfsetbuttcap%
\pgfsetroundjoin%
\definecolor{currentfill}{rgb}{0.000000,0.000000,0.000000}%
\pgfsetfillcolor{currentfill}%
\pgfsetlinewidth{0.602250pt}%
\definecolor{currentstroke}{rgb}{0.000000,0.000000,0.000000}%
\pgfsetstrokecolor{currentstroke}%
\pgfsetdash{}{0pt}%
\pgfsys@defobject{currentmarker}{\pgfqpoint{-0.027778in}{0.000000in}}{\pgfqpoint{-0.000000in}{0.000000in}}{%
\pgfpathmoveto{\pgfqpoint{-0.000000in}{0.000000in}}%
\pgfpathlineto{\pgfqpoint{-0.027778in}{0.000000in}}%
\pgfusepath{stroke,fill}%
}%
\begin{pgfscope}%
\pgfsys@transformshift{0.562308in}{1.788281in}%
\pgfsys@useobject{currentmarker}{}%
\end{pgfscope}%
\end{pgfscope}%
\begin{pgfscope}%
\pgfsetbuttcap%
\pgfsetroundjoin%
\definecolor{currentfill}{rgb}{0.000000,0.000000,0.000000}%
\pgfsetfillcolor{currentfill}%
\pgfsetlinewidth{0.602250pt}%
\definecolor{currentstroke}{rgb}{0.000000,0.000000,0.000000}%
\pgfsetstrokecolor{currentstroke}%
\pgfsetdash{}{0pt}%
\pgfsys@defobject{currentmarker}{\pgfqpoint{-0.027778in}{0.000000in}}{\pgfqpoint{-0.000000in}{0.000000in}}{%
\pgfpathmoveto{\pgfqpoint{-0.000000in}{0.000000in}}%
\pgfpathlineto{\pgfqpoint{-0.027778in}{0.000000in}}%
\pgfusepath{stroke,fill}%
}%
\begin{pgfscope}%
\pgfsys@transformshift{0.562308in}{1.813526in}%
\pgfsys@useobject{currentmarker}{}%
\end{pgfscope}%
\end{pgfscope}%
\begin{pgfscope}%
\pgfsetbuttcap%
\pgfsetroundjoin%
\definecolor{currentfill}{rgb}{0.000000,0.000000,0.000000}%
\pgfsetfillcolor{currentfill}%
\pgfsetlinewidth{0.602250pt}%
\definecolor{currentstroke}{rgb}{0.000000,0.000000,0.000000}%
\pgfsetstrokecolor{currentstroke}%
\pgfsetdash{}{0pt}%
\pgfsys@defobject{currentmarker}{\pgfqpoint{-0.027778in}{0.000000in}}{\pgfqpoint{-0.000000in}{0.000000in}}{%
\pgfpathmoveto{\pgfqpoint{-0.000000in}{0.000000in}}%
\pgfpathlineto{\pgfqpoint{-0.027778in}{0.000000in}}%
\pgfusepath{stroke,fill}%
}%
\begin{pgfscope}%
\pgfsys@transformshift{0.562308in}{1.835794in}%
\pgfsys@useobject{currentmarker}{}%
\end{pgfscope}%
\end{pgfscope}%
\begin{pgfscope}%
\definecolor{textcolor}{rgb}{0.000000,0.000000,0.000000}%
\pgfsetstrokecolor{textcolor}%
\pgfsetfillcolor{textcolor}%
\pgftext[x=0.208333in,y=1.202708in,,bottom,rotate=90.000000]{\color{textcolor}\rmfamily\fontsize{10.000000}{12.000000}\selectfont \(\displaystyle \mathcal{O}(.)\) [1]}%
\end{pgfscope}%
\begin{pgfscope}%
\pgfpathrectangle{\pgfqpoint{0.562308in}{0.484444in}}{\pgfqpoint{2.449920in}{1.436528in}}%
\pgfusepath{clip}%
\pgfsetrectcap%
\pgfsetroundjoin%
\pgfsetlinewidth{1.505625pt}%
\definecolor{currentstroke}{rgb}{0.000000,0.000000,0.000000}%
\pgfsetstrokecolor{currentstroke}%
\pgfsetdash{}{0pt}%
\pgfpathmoveto{\pgfqpoint{0.567208in}{0.680786in}}%
\pgfpathlineto{\pgfqpoint{0.572107in}{0.811832in}}%
\pgfpathlineto{\pgfqpoint{0.578967in}{0.912152in}}%
\pgfpathlineto{\pgfqpoint{0.587297in}{0.988809in}}%
\pgfpathlineto{\pgfqpoint{0.597097in}{1.051360in}}%
\pgfpathlineto{\pgfqpoint{0.608856in}{1.106413in}}%
\pgfpathlineto{\pgfqpoint{0.622086in}{1.153705in}}%
\pgfpathlineto{\pgfqpoint{0.637275in}{1.196511in}}%
\pgfpathlineto{\pgfqpoint{0.654425in}{1.235458in}}%
\pgfpathlineto{\pgfqpoint{0.673534in}{1.271097in}}%
\pgfpathlineto{\pgfqpoint{0.694603in}{1.303894in}}%
\pgfpathlineto{\pgfqpoint{0.718123in}{1.334829in}}%
\pgfpathlineto{\pgfqpoint{0.744582in}{1.364482in}}%
\pgfpathlineto{\pgfqpoint{0.773981in}{1.392752in}}%
\pgfpathlineto{\pgfqpoint{0.806810in}{1.420011in}}%
\pgfpathlineto{\pgfqpoint{0.843559in}{1.446483in}}%
\pgfpathlineto{\pgfqpoint{0.884717in}{1.472304in}}%
\pgfpathlineto{\pgfqpoint{0.931266in}{1.497801in}}%
\pgfpathlineto{\pgfqpoint{0.983694in}{1.522920in}}%
\pgfpathlineto{\pgfqpoint{1.042492in}{1.547615in}}%
\pgfpathlineto{\pgfqpoint{1.109130in}{1.572184in}}%
\pgfpathlineto{\pgfqpoint{1.184098in}{1.596474in}}%
\pgfpathlineto{\pgfqpoint{1.268865in}{1.620636in}}%
\pgfpathlineto{\pgfqpoint{1.364902in}{1.644731in}}%
\pgfpathlineto{\pgfqpoint{1.473188in}{1.668659in}}%
\pgfpathlineto{\pgfqpoint{1.596174in}{1.692603in}}%
\pgfpathlineto{\pgfqpoint{1.735330in}{1.716477in}}%
\pgfpathlineto{\pgfqpoint{1.892614in}{1.740265in}}%
\pgfpathlineto{\pgfqpoint{2.070969in}{1.764051in}}%
\pgfpathlineto{\pgfqpoint{2.272842in}{1.787794in}}%
\pgfpathlineto{\pgfqpoint{2.501665in}{1.811530in}}%
\pgfpathlineto{\pgfqpoint{2.760866in}{1.835247in}}%
\pgfpathlineto{\pgfqpoint{3.011738in}{1.855675in}}%
\pgfpathlineto{\pgfqpoint{3.011738in}{1.855675in}}%
\pgfusepath{stroke}%
\end{pgfscope}%
\begin{pgfscope}%
\pgfpathrectangle{\pgfqpoint{0.562308in}{0.484444in}}{\pgfqpoint{2.449920in}{1.436528in}}%
\pgfusepath{clip}%
\pgfsetbuttcap%
\pgfsetroundjoin%
\pgfsetlinewidth{1.505625pt}%
\definecolor{currentstroke}{rgb}{0.121569,0.466667,0.705882}%
\pgfsetstrokecolor{currentstroke}%
\pgfsetdash{{5.550000pt}{2.400000pt}}{0.000000pt}%
\pgfpathmoveto{\pgfqpoint{0.567208in}{0.567760in}}%
\pgfpathlineto{\pgfqpoint{0.572597in}{0.698806in}}%
\pgfpathlineto{\pgfqpoint{0.579457in}{0.791913in}}%
\pgfpathlineto{\pgfqpoint{0.587787in}{0.865036in}}%
\pgfpathlineto{\pgfqpoint{0.598077in}{0.928138in}}%
\pgfpathlineto{\pgfqpoint{0.609836in}{0.981245in}}%
\pgfpathlineto{\pgfqpoint{0.623556in}{1.028759in}}%
\pgfpathlineto{\pgfqpoint{0.638745in}{1.070345in}}%
\pgfpathlineto{\pgfqpoint{0.655895in}{1.108393in}}%
\pgfpathlineto{\pgfqpoint{0.675004in}{1.143354in}}%
\pgfpathlineto{\pgfqpoint{0.696563in}{1.176317in}}%
\pgfpathlineto{\pgfqpoint{0.720573in}{1.207318in}}%
\pgfpathlineto{\pgfqpoint{0.747032in}{1.236461in}}%
\pgfpathlineto{\pgfqpoint{0.776921in}{1.264745in}}%
\pgfpathlineto{\pgfqpoint{0.810240in}{1.291972in}}%
\pgfpathlineto{\pgfqpoint{0.847478in}{1.318379in}}%
\pgfpathlineto{\pgfqpoint{0.889127in}{1.344110in}}%
\pgfpathlineto{\pgfqpoint{0.936166in}{1.369497in}}%
\pgfpathlineto{\pgfqpoint{0.989084in}{1.394494in}}%
\pgfpathlineto{\pgfqpoint{1.048862in}{1.419251in}}%
\pgfpathlineto{\pgfqpoint{1.115990in}{1.443663in}}%
\pgfpathlineto{\pgfqpoint{1.191937in}{1.467944in}}%
\pgfpathlineto{\pgfqpoint{1.277684in}{1.492065in}}%
\pgfpathlineto{\pgfqpoint{1.374701in}{1.516094in}}%
\pgfpathlineto{\pgfqpoint{1.484458in}{1.540038in}}%
\pgfpathlineto{\pgfqpoint{1.608914in}{1.563961in}}%
\pgfpathlineto{\pgfqpoint{1.749539in}{1.587785in}}%
\pgfpathlineto{\pgfqpoint{1.908784in}{1.611572in}}%
\pgfpathlineto{\pgfqpoint{2.089098in}{1.635325in}}%
\pgfpathlineto{\pgfqpoint{2.293421in}{1.659063in}}%
\pgfpathlineto{\pgfqpoint{2.524694in}{1.682764in}}%
\pgfpathlineto{\pgfqpoint{2.786835in}{1.706463in}}%
\pgfpathlineto{\pgfqpoint{3.011738in}{1.724668in}}%
\pgfpathlineto{\pgfqpoint{3.011738in}{1.724668in}}%
\pgfusepath{stroke}%
\end{pgfscope}%
\begin{pgfscope}%
\pgfpathrectangle{\pgfqpoint{0.562308in}{0.484444in}}{\pgfqpoint{2.449920in}{1.436528in}}%
\pgfusepath{clip}%
\pgfsetrectcap%
\pgfsetroundjoin%
\pgfsetlinewidth{1.505625pt}%
\definecolor{currentstroke}{rgb}{1.000000,0.498039,0.054902}%
\pgfsetstrokecolor{currentstroke}%
\pgfsetdash{}{0pt}%
\pgfpathmoveto{\pgfqpoint{0.567208in}{0.549741in}}%
\pgfpathlineto{\pgfqpoint{3.011738in}{0.549741in}}%
\pgfpathlineto{\pgfqpoint{3.011738in}{0.549741in}}%
\pgfusepath{stroke}%
\end{pgfscope}%
\begin{pgfscope}%
\pgfpathrectangle{\pgfqpoint{0.562308in}{0.484444in}}{\pgfqpoint{2.449920in}{1.436528in}}%
\pgfusepath{clip}%
\pgfsetbuttcap%
\pgfsetroundjoin%
\definecolor{currentfill}{rgb}{1.000000,0.498039,0.054902}%
\pgfsetfillcolor{currentfill}%
\pgfsetlinewidth{1.003750pt}%
\definecolor{currentstroke}{rgb}{1.000000,0.498039,0.054902}%
\pgfsetstrokecolor{currentstroke}%
\pgfsetdash{}{0pt}%
\pgfsys@defobject{currentmarker}{\pgfqpoint{-0.027778in}{-0.027778in}}{\pgfqpoint{0.027778in}{0.027778in}}{%
\pgfpathmoveto{\pgfqpoint{0.000000in}{-0.027778in}}%
\pgfpathcurveto{\pgfqpoint{0.007367in}{-0.027778in}}{\pgfqpoint{0.014433in}{-0.024851in}}{\pgfqpoint{0.019642in}{-0.019642in}}%
\pgfpathcurveto{\pgfqpoint{0.024851in}{-0.014433in}}{\pgfqpoint{0.027778in}{-0.007367in}}{\pgfqpoint{0.027778in}{0.000000in}}%
\pgfpathcurveto{\pgfqpoint{0.027778in}{0.007367in}}{\pgfqpoint{0.024851in}{0.014433in}}{\pgfqpoint{0.019642in}{0.019642in}}%
\pgfpathcurveto{\pgfqpoint{0.014433in}{0.024851in}}{\pgfqpoint{0.007367in}{0.027778in}}{\pgfqpoint{0.000000in}{0.027778in}}%
\pgfpathcurveto{\pgfqpoint{-0.007367in}{0.027778in}}{\pgfqpoint{-0.014433in}{0.024851in}}{\pgfqpoint{-0.019642in}{0.019642in}}%
\pgfpathcurveto{\pgfqpoint{-0.024851in}{0.014433in}}{\pgfqpoint{-0.027778in}{0.007367in}}{\pgfqpoint{-0.027778in}{0.000000in}}%
\pgfpathcurveto{\pgfqpoint{-0.027778in}{-0.007367in}}{\pgfqpoint{-0.024851in}{-0.014433in}}{\pgfqpoint{-0.019642in}{-0.019642in}}%
\pgfpathcurveto{\pgfqpoint{-0.014433in}{-0.024851in}}{\pgfqpoint{-0.007367in}{-0.027778in}}{\pgfqpoint{0.000000in}{-0.027778in}}%
\pgfpathlineto{\pgfqpoint{0.000000in}{-0.027778in}}%
\pgfpathclose%
\pgfusepath{stroke,fill}%
}%
\begin{pgfscope}%
\pgfsys@transformshift{0.567208in}{0.549741in}%
\pgfsys@useobject{currentmarker}{}%
\end{pgfscope}%
\begin{pgfscope}%
\pgfsys@transformshift{0.812200in}{0.549741in}%
\pgfsys@useobject{currentmarker}{}%
\end{pgfscope}%
\begin{pgfscope}%
\pgfsys@transformshift{1.057192in}{0.549741in}%
\pgfsys@useobject{currentmarker}{}%
\end{pgfscope}%
\begin{pgfscope}%
\pgfsys@transformshift{1.302184in}{0.549741in}%
\pgfsys@useobject{currentmarker}{}%
\end{pgfscope}%
\begin{pgfscope}%
\pgfsys@transformshift{1.547176in}{0.549741in}%
\pgfsys@useobject{currentmarker}{}%
\end{pgfscope}%
\begin{pgfscope}%
\pgfsys@transformshift{1.792168in}{0.549741in}%
\pgfsys@useobject{currentmarker}{}%
\end{pgfscope}%
\begin{pgfscope}%
\pgfsys@transformshift{2.037160in}{0.549741in}%
\pgfsys@useobject{currentmarker}{}%
\end{pgfscope}%
\begin{pgfscope}%
\pgfsys@transformshift{2.282152in}{0.549741in}%
\pgfsys@useobject{currentmarker}{}%
\end{pgfscope}%
\begin{pgfscope}%
\pgfsys@transformshift{2.527144in}{0.549741in}%
\pgfsys@useobject{currentmarker}{}%
\end{pgfscope}%
\begin{pgfscope}%
\pgfsys@transformshift{2.772136in}{0.549741in}%
\pgfsys@useobject{currentmarker}{}%
\end{pgfscope}%
\end{pgfscope}%
\begin{pgfscope}%
\pgfpathrectangle{\pgfqpoint{0.562308in}{0.484444in}}{\pgfqpoint{2.449920in}{1.436528in}}%
\pgfusepath{clip}%
\pgfsetrectcap%
\pgfsetroundjoin%
\pgfsetlinewidth{1.505625pt}%
\definecolor{currentstroke}{rgb}{0.172549,0.627451,0.172549}%
\pgfsetstrokecolor{currentstroke}%
\pgfsetdash{}{0pt}%
\pgfpathmoveto{\pgfqpoint{0.567208in}{1.289343in}}%
\pgfpathlineto{\pgfqpoint{3.011738in}{1.289343in}}%
\pgfpathlineto{\pgfqpoint{3.011738in}{1.289343in}}%
\pgfusepath{stroke}%
\end{pgfscope}%
\begin{pgfscope}%
\pgfpathrectangle{\pgfqpoint{0.562308in}{0.484444in}}{\pgfqpoint{2.449920in}{1.436528in}}%
\pgfusepath{clip}%
\pgfsetbuttcap%
\pgfsetmiterjoin%
\definecolor{currentfill}{rgb}{0.172549,0.627451,0.172549}%
\pgfsetfillcolor{currentfill}%
\pgfsetlinewidth{1.003750pt}%
\definecolor{currentstroke}{rgb}{0.172549,0.627451,0.172549}%
\pgfsetstrokecolor{currentstroke}%
\pgfsetdash{}{0pt}%
\pgfsys@defobject{currentmarker}{\pgfqpoint{-0.027778in}{-0.027778in}}{\pgfqpoint{0.027778in}{0.027778in}}{%
\pgfpathmoveto{\pgfqpoint{-0.027778in}{-0.027778in}}%
\pgfpathlineto{\pgfqpoint{0.027778in}{-0.027778in}}%
\pgfpathlineto{\pgfqpoint{0.027778in}{0.027778in}}%
\pgfpathlineto{\pgfqpoint{-0.027778in}{0.027778in}}%
\pgfpathlineto{\pgfqpoint{-0.027778in}{-0.027778in}}%
\pgfpathclose%
\pgfusepath{stroke,fill}%
}%
\begin{pgfscope}%
\pgfsys@transformshift{0.591707in}{1.289343in}%
\pgfsys@useobject{currentmarker}{}%
\end{pgfscope}%
\begin{pgfscope}%
\pgfsys@transformshift{0.836699in}{1.289343in}%
\pgfsys@useobject{currentmarker}{}%
\end{pgfscope}%
\begin{pgfscope}%
\pgfsys@transformshift{1.081691in}{1.289343in}%
\pgfsys@useobject{currentmarker}{}%
\end{pgfscope}%
\begin{pgfscope}%
\pgfsys@transformshift{1.326683in}{1.289343in}%
\pgfsys@useobject{currentmarker}{}%
\end{pgfscope}%
\begin{pgfscope}%
\pgfsys@transformshift{1.571675in}{1.289343in}%
\pgfsys@useobject{currentmarker}{}%
\end{pgfscope}%
\begin{pgfscope}%
\pgfsys@transformshift{1.816667in}{1.289343in}%
\pgfsys@useobject{currentmarker}{}%
\end{pgfscope}%
\begin{pgfscope}%
\pgfsys@transformshift{2.061659in}{1.289343in}%
\pgfsys@useobject{currentmarker}{}%
\end{pgfscope}%
\begin{pgfscope}%
\pgfsys@transformshift{2.306651in}{1.289343in}%
\pgfsys@useobject{currentmarker}{}%
\end{pgfscope}%
\begin{pgfscope}%
\pgfsys@transformshift{2.551643in}{1.289343in}%
\pgfsys@useobject{currentmarker}{}%
\end{pgfscope}%
\begin{pgfscope}%
\pgfsys@transformshift{2.796635in}{1.289343in}%
\pgfsys@useobject{currentmarker}{}%
\end{pgfscope}%
\end{pgfscope}%
\begin{pgfscope}%
\pgfpathrectangle{\pgfqpoint{0.562308in}{0.484444in}}{\pgfqpoint{2.449920in}{1.436528in}}%
\pgfusepath{clip}%
\pgfsetrectcap%
\pgfsetroundjoin%
\pgfsetlinewidth{1.505625pt}%
\definecolor{currentstroke}{rgb}{0.839216,0.152941,0.156863}%
\pgfsetstrokecolor{currentstroke}%
\pgfsetdash{}{0pt}%
\pgfpathmoveto{\pgfqpoint{0.567208in}{0.680786in}}%
\pgfpathlineto{\pgfqpoint{3.011738in}{0.680786in}}%
\pgfpathlineto{\pgfqpoint{3.011738in}{0.680786in}}%
\pgfusepath{stroke}%
\end{pgfscope}%
\begin{pgfscope}%
\pgfpathrectangle{\pgfqpoint{0.562308in}{0.484444in}}{\pgfqpoint{2.449920in}{1.436528in}}%
\pgfusepath{clip}%
\pgfsetbuttcap%
\pgfsetroundjoin%
\definecolor{currentfill}{rgb}{0.839216,0.152941,0.156863}%
\pgfsetfillcolor{currentfill}%
\pgfsetlinewidth{1.003750pt}%
\definecolor{currentstroke}{rgb}{0.839216,0.152941,0.156863}%
\pgfsetstrokecolor{currentstroke}%
\pgfsetdash{}{0pt}%
\pgfsys@defobject{currentmarker}{\pgfqpoint{-0.027778in}{-0.027778in}}{\pgfqpoint{0.027778in}{0.027778in}}{%
\pgfpathmoveto{\pgfqpoint{-0.027778in}{-0.027778in}}%
\pgfpathlineto{\pgfqpoint{0.027778in}{0.027778in}}%
\pgfpathmoveto{\pgfqpoint{-0.027778in}{0.027778in}}%
\pgfpathlineto{\pgfqpoint{0.027778in}{-0.027778in}}%
\pgfusepath{stroke,fill}%
}%
\begin{pgfscope}%
\pgfsys@transformshift{0.616206in}{0.680786in}%
\pgfsys@useobject{currentmarker}{}%
\end{pgfscope}%
\begin{pgfscope}%
\pgfsys@transformshift{0.861198in}{0.680786in}%
\pgfsys@useobject{currentmarker}{}%
\end{pgfscope}%
\begin{pgfscope}%
\pgfsys@transformshift{1.106190in}{0.680786in}%
\pgfsys@useobject{currentmarker}{}%
\end{pgfscope}%
\begin{pgfscope}%
\pgfsys@transformshift{1.351182in}{0.680786in}%
\pgfsys@useobject{currentmarker}{}%
\end{pgfscope}%
\begin{pgfscope}%
\pgfsys@transformshift{1.596174in}{0.680786in}%
\pgfsys@useobject{currentmarker}{}%
\end{pgfscope}%
\begin{pgfscope}%
\pgfsys@transformshift{1.841166in}{0.680786in}%
\pgfsys@useobject{currentmarker}{}%
\end{pgfscope}%
\begin{pgfscope}%
\pgfsys@transformshift{2.086158in}{0.680786in}%
\pgfsys@useobject{currentmarker}{}%
\end{pgfscope}%
\begin{pgfscope}%
\pgfsys@transformshift{2.331150in}{0.680786in}%
\pgfsys@useobject{currentmarker}{}%
\end{pgfscope}%
\begin{pgfscope}%
\pgfsys@transformshift{2.576142in}{0.680786in}%
\pgfsys@useobject{currentmarker}{}%
\end{pgfscope}%
\begin{pgfscope}%
\pgfsys@transformshift{2.821134in}{0.680786in}%
\pgfsys@useobject{currentmarker}{}%
\end{pgfscope}%
\end{pgfscope}%
\begin{pgfscope}%
\pgfpathrectangle{\pgfqpoint{0.562308in}{0.484444in}}{\pgfqpoint{2.449920in}{1.436528in}}%
\pgfusepath{clip}%
\pgfsetrectcap%
\pgfsetroundjoin%
\pgfsetlinewidth{1.505625pt}%
\definecolor{currentstroke}{rgb}{0.580392,0.403922,0.741176}%
\pgfsetstrokecolor{currentstroke}%
\pgfsetdash{}{0pt}%
\pgfpathmoveto{\pgfqpoint{0.567208in}{1.420389in}}%
\pgfpathlineto{\pgfqpoint{3.011738in}{1.420389in}}%
\pgfpathlineto{\pgfqpoint{3.011738in}{1.420389in}}%
\pgfusepath{stroke}%
\end{pgfscope}%
\begin{pgfscope}%
\pgfpathrectangle{\pgfqpoint{0.562308in}{0.484444in}}{\pgfqpoint{2.449920in}{1.436528in}}%
\pgfusepath{clip}%
\pgfsetbuttcap%
\pgfsetroundjoin%
\definecolor{currentfill}{rgb}{0.580392,0.403922,0.741176}%
\pgfsetfillcolor{currentfill}%
\pgfsetlinewidth{1.003750pt}%
\definecolor{currentstroke}{rgb}{0.580392,0.403922,0.741176}%
\pgfsetstrokecolor{currentstroke}%
\pgfsetdash{}{0pt}%
\pgfsys@defobject{currentmarker}{\pgfqpoint{-0.027778in}{-0.027778in}}{\pgfqpoint{0.027778in}{0.027778in}}{%
\pgfpathmoveto{\pgfqpoint{-0.027778in}{0.000000in}}%
\pgfpathlineto{\pgfqpoint{0.027778in}{0.000000in}}%
\pgfpathmoveto{\pgfqpoint{0.000000in}{-0.027778in}}%
\pgfpathlineto{\pgfqpoint{0.000000in}{0.027778in}}%
\pgfusepath{stroke,fill}%
}%
\begin{pgfscope}%
\pgfsys@transformshift{0.640705in}{1.420389in}%
\pgfsys@useobject{currentmarker}{}%
\end{pgfscope}%
\begin{pgfscope}%
\pgfsys@transformshift{0.885697in}{1.420389in}%
\pgfsys@useobject{currentmarker}{}%
\end{pgfscope}%
\begin{pgfscope}%
\pgfsys@transformshift{1.130689in}{1.420389in}%
\pgfsys@useobject{currentmarker}{}%
\end{pgfscope}%
\begin{pgfscope}%
\pgfsys@transformshift{1.375681in}{1.420389in}%
\pgfsys@useobject{currentmarker}{}%
\end{pgfscope}%
\begin{pgfscope}%
\pgfsys@transformshift{1.620673in}{1.420389in}%
\pgfsys@useobject{currentmarker}{}%
\end{pgfscope}%
\begin{pgfscope}%
\pgfsys@transformshift{1.865665in}{1.420389in}%
\pgfsys@useobject{currentmarker}{}%
\end{pgfscope}%
\begin{pgfscope}%
\pgfsys@transformshift{2.110657in}{1.420389in}%
\pgfsys@useobject{currentmarker}{}%
\end{pgfscope}%
\begin{pgfscope}%
\pgfsys@transformshift{2.355649in}{1.420389in}%
\pgfsys@useobject{currentmarker}{}%
\end{pgfscope}%
\begin{pgfscope}%
\pgfsys@transformshift{2.600641in}{1.420389in}%
\pgfsys@useobject{currentmarker}{}%
\end{pgfscope}%
\begin{pgfscope}%
\pgfsys@transformshift{2.845633in}{1.420389in}%
\pgfsys@useobject{currentmarker}{}%
\end{pgfscope}%
\end{pgfscope}%
\begin{pgfscope}%
\pgfsetrectcap%
\pgfsetmiterjoin%
\pgfsetlinewidth{0.803000pt}%
\definecolor{currentstroke}{rgb}{0.000000,0.000000,0.000000}%
\pgfsetstrokecolor{currentstroke}%
\pgfsetdash{}{0pt}%
\pgfpathmoveto{\pgfqpoint{0.562308in}{0.484444in}}%
\pgfpathlineto{\pgfqpoint{0.562308in}{1.920972in}}%
\pgfusepath{stroke}%
\end{pgfscope}%
\begin{pgfscope}%
\pgfsetrectcap%
\pgfsetmiterjoin%
\pgfsetlinewidth{0.803000pt}%
\definecolor{currentstroke}{rgb}{0.000000,0.000000,0.000000}%
\pgfsetstrokecolor{currentstroke}%
\pgfsetdash{}{0pt}%
\pgfpathmoveto{\pgfqpoint{3.012228in}{0.484444in}}%
\pgfpathlineto{\pgfqpoint{3.012228in}{1.920972in}}%
\pgfusepath{stroke}%
\end{pgfscope}%
\begin{pgfscope}%
\pgfsetrectcap%
\pgfsetmiterjoin%
\pgfsetlinewidth{0.803000pt}%
\definecolor{currentstroke}{rgb}{0.000000,0.000000,0.000000}%
\pgfsetstrokecolor{currentstroke}%
\pgfsetdash{}{0pt}%
\pgfpathmoveto{\pgfqpoint{0.562308in}{0.484444in}}%
\pgfpathlineto{\pgfqpoint{3.012228in}{0.484444in}}%
\pgfusepath{stroke}%
\end{pgfscope}%
\begin{pgfscope}%
\pgfsetrectcap%
\pgfsetmiterjoin%
\pgfsetlinewidth{0.803000pt}%
\definecolor{currentstroke}{rgb}{0.000000,0.000000,0.000000}%
\pgfsetstrokecolor{currentstroke}%
\pgfsetdash{}{0pt}%
\pgfpathmoveto{\pgfqpoint{0.562308in}{1.920972in}}%
\pgfpathlineto{\pgfqpoint{3.012228in}{1.920972in}}%
\pgfusepath{stroke}%
\end{pgfscope}%
\begin{pgfscope}%
\pgfsetbuttcap%
\pgfsetmiterjoin%
\definecolor{currentfill}{rgb}{1.000000,1.000000,1.000000}%
\pgfsetfillcolor{currentfill}%
\pgfsetfillopacity{0.800000}%
\pgfsetlinewidth{1.003750pt}%
\definecolor{currentstroke}{rgb}{0.800000,0.800000,0.800000}%
\pgfsetstrokecolor{currentstroke}%
\pgfsetstrokeopacity{0.800000}%
\pgfsetdash{}{0pt}%
\pgfpathmoveto{\pgfqpoint{0.875355in}{0.750836in}}%
\pgfpathlineto{\pgfqpoint{2.880498in}{0.750836in}}%
\pgfpathquadraticcurveto{\pgfqpoint{2.899942in}{0.750836in}}{\pgfqpoint{2.899942in}{0.770280in}}%
\pgfpathlineto{\pgfqpoint{2.899942in}{1.204178in}}%
\pgfpathquadraticcurveto{\pgfqpoint{2.899942in}{1.223622in}}{\pgfqpoint{2.880498in}{1.223622in}}%
\pgfpathlineto{\pgfqpoint{0.875355in}{1.223622in}}%
\pgfpathquadraticcurveto{\pgfqpoint{0.855911in}{1.223622in}}{\pgfqpoint{0.855911in}{1.204178in}}%
\pgfpathlineto{\pgfqpoint{0.855911in}{0.770280in}}%
\pgfpathquadraticcurveto{\pgfqpoint{0.855911in}{0.750836in}}{\pgfqpoint{0.875355in}{0.750836in}}%
\pgfpathlineto{\pgfqpoint{0.875355in}{0.750836in}}%
\pgfpathclose%
\pgfusepath{stroke,fill}%
\end{pgfscope}%
\begin{pgfscope}%
\pgfsetrectcap%
\pgfsetroundjoin%
\pgfsetlinewidth{1.505625pt}%
\definecolor{currentstroke}{rgb}{0.000000,0.000000,0.000000}%
\pgfsetstrokecolor{currentstroke}%
\pgfsetdash{}{0pt}%
\pgfpathmoveto{\pgfqpoint{0.894800in}{1.138360in}}%
\pgfpathlineto{\pgfqpoint{0.992022in}{1.138360in}}%
\pgfpathlineto{\pgfqpoint{1.089244in}{1.138360in}}%
\pgfusepath{stroke}%
\end{pgfscope}%
\begin{pgfscope}%
\definecolor{textcolor}{rgb}{0.000000,0.000000,0.000000}%
\pgfsetstrokecolor{textcolor}%
\pgfsetfillcolor{textcolor}%
\pgftext[x=1.167022in,y=1.104332in,left,base]{\color{textcolor}\rmfamily\fontsize{7.000000}{8.400000}\selectfont full direct}%
\end{pgfscope}%
\begin{pgfscope}%
\pgfsetbuttcap%
\pgfsetroundjoin%
\pgfsetlinewidth{1.505625pt}%
\definecolor{currentstroke}{rgb}{0.121569,0.466667,0.705882}%
\pgfsetstrokecolor{currentstroke}%
\pgfsetdash{{5.550000pt}{2.400000pt}}{0.000000pt}%
\pgfpathmoveto{\pgfqpoint{0.894800in}{1.002832in}}%
\pgfpathlineto{\pgfqpoint{0.992022in}{1.002832in}}%
\pgfpathlineto{\pgfqpoint{1.089244in}{1.002832in}}%
\pgfusepath{stroke}%
\end{pgfscope}%
\begin{pgfscope}%
\definecolor{textcolor}{rgb}{0.000000,0.000000,0.000000}%
\pgfsetstrokecolor{textcolor}%
\pgfsetfillcolor{textcolor}%
\pgftext[x=1.167022in,y=0.968804in,left,base]{\color{textcolor}\rmfamily\fontsize{7.000000}{8.400000}\selectfont broadcast}%
\end{pgfscope}%
\begin{pgfscope}%
\pgfsetrectcap%
\pgfsetroundjoin%
\pgfsetlinewidth{1.505625pt}%
\definecolor{currentstroke}{rgb}{1.000000,0.498039,0.054902}%
\pgfsetstrokecolor{currentstroke}%
\pgfsetdash{}{0pt}%
\pgfpathmoveto{\pgfqpoint{0.894800in}{0.854959in}}%
\pgfpathlineto{\pgfqpoint{0.992022in}{0.854959in}}%
\pgfpathlineto{\pgfqpoint{1.089244in}{0.854959in}}%
\pgfusepath{stroke}%
\end{pgfscope}%
\begin{pgfscope}%
\pgfsetbuttcap%
\pgfsetroundjoin%
\definecolor{currentfill}{rgb}{1.000000,0.498039,0.054902}%
\pgfsetfillcolor{currentfill}%
\pgfsetlinewidth{1.003750pt}%
\definecolor{currentstroke}{rgb}{1.000000,0.498039,0.054902}%
\pgfsetstrokecolor{currentstroke}%
\pgfsetdash{}{0pt}%
\pgfsys@defobject{currentmarker}{\pgfqpoint{-0.027778in}{-0.027778in}}{\pgfqpoint{0.027778in}{0.027778in}}{%
\pgfpathmoveto{\pgfqpoint{0.000000in}{-0.027778in}}%
\pgfpathcurveto{\pgfqpoint{0.007367in}{-0.027778in}}{\pgfqpoint{0.014433in}{-0.024851in}}{\pgfqpoint{0.019642in}{-0.019642in}}%
\pgfpathcurveto{\pgfqpoint{0.024851in}{-0.014433in}}{\pgfqpoint{0.027778in}{-0.007367in}}{\pgfqpoint{0.027778in}{0.000000in}}%
\pgfpathcurveto{\pgfqpoint{0.027778in}{0.007367in}}{\pgfqpoint{0.024851in}{0.014433in}}{\pgfqpoint{0.019642in}{0.019642in}}%
\pgfpathcurveto{\pgfqpoint{0.014433in}{0.024851in}}{\pgfqpoint{0.007367in}{0.027778in}}{\pgfqpoint{0.000000in}{0.027778in}}%
\pgfpathcurveto{\pgfqpoint{-0.007367in}{0.027778in}}{\pgfqpoint{-0.014433in}{0.024851in}}{\pgfqpoint{-0.019642in}{0.019642in}}%
\pgfpathcurveto{\pgfqpoint{-0.024851in}{0.014433in}}{\pgfqpoint{-0.027778in}{0.007367in}}{\pgfqpoint{-0.027778in}{0.000000in}}%
\pgfpathcurveto{\pgfqpoint{-0.027778in}{-0.007367in}}{\pgfqpoint{-0.024851in}{-0.014433in}}{\pgfqpoint{-0.019642in}{-0.019642in}}%
\pgfpathcurveto{\pgfqpoint{-0.014433in}{-0.024851in}}{\pgfqpoint{-0.007367in}{-0.027778in}}{\pgfqpoint{0.000000in}{-0.027778in}}%
\pgfpathlineto{\pgfqpoint{0.000000in}{-0.027778in}}%
\pgfpathclose%
\pgfusepath{stroke,fill}%
}%
\begin{pgfscope}%
\pgfsys@transformshift{0.992022in}{0.854959in}%
\pgfsys@useobject{currentmarker}{}%
\end{pgfscope}%
\end{pgfscope}%
\begin{pgfscope}%
\definecolor{textcolor}{rgb}{0.000000,0.000000,0.000000}%
\pgfsetstrokecolor{textcolor}%
\pgfsetfillcolor{textcolor}%
\pgftext[x=1.167022in,y=0.820931in,left,base]{\color{textcolor}\rmfamily\fontsize{7.000000}{8.400000}\selectfont \(\displaystyle \bar{N}=\)5,\(\displaystyle R=\)1}%
\end{pgfscope}%
\begin{pgfscope}%
\pgfsetrectcap%
\pgfsetroundjoin%
\pgfsetlinewidth{1.505625pt}%
\definecolor{currentstroke}{rgb}{0.172549,0.627451,0.172549}%
\pgfsetstrokecolor{currentstroke}%
\pgfsetdash{}{0pt}%
\pgfpathmoveto{\pgfqpoint{1.919829in}{1.138360in}}%
\pgfpathlineto{\pgfqpoint{2.017051in}{1.138360in}}%
\pgfpathlineto{\pgfqpoint{2.114274in}{1.138360in}}%
\pgfusepath{stroke}%
\end{pgfscope}%
\begin{pgfscope}%
\pgfsetbuttcap%
\pgfsetmiterjoin%
\definecolor{currentfill}{rgb}{0.172549,0.627451,0.172549}%
\pgfsetfillcolor{currentfill}%
\pgfsetlinewidth{1.003750pt}%
\definecolor{currentstroke}{rgb}{0.172549,0.627451,0.172549}%
\pgfsetstrokecolor{currentstroke}%
\pgfsetdash{}{0pt}%
\pgfsys@defobject{currentmarker}{\pgfqpoint{-0.027778in}{-0.027778in}}{\pgfqpoint{0.027778in}{0.027778in}}{%
\pgfpathmoveto{\pgfqpoint{-0.027778in}{-0.027778in}}%
\pgfpathlineto{\pgfqpoint{0.027778in}{-0.027778in}}%
\pgfpathlineto{\pgfqpoint{0.027778in}{0.027778in}}%
\pgfpathlineto{\pgfqpoint{-0.027778in}{0.027778in}}%
\pgfpathlineto{\pgfqpoint{-0.027778in}{-0.027778in}}%
\pgfpathclose%
\pgfusepath{stroke,fill}%
}%
\begin{pgfscope}%
\pgfsys@transformshift{2.017051in}{1.138360in}%
\pgfsys@useobject{currentmarker}{}%
\end{pgfscope}%
\end{pgfscope}%
\begin{pgfscope}%
\definecolor{textcolor}{rgb}{0.000000,0.000000,0.000000}%
\pgfsetstrokecolor{textcolor}%
\pgfsetfillcolor{textcolor}%
\pgftext[x=2.192051in,y=1.104332in,left,base]{\color{textcolor}\rmfamily\fontsize{7.000000}{8.400000}\selectfont \(\displaystyle \bar{N}=\)5,\(\displaystyle R=\)50}%
\end{pgfscope}%
\begin{pgfscope}%
\pgfsetrectcap%
\pgfsetroundjoin%
\pgfsetlinewidth{1.505625pt}%
\definecolor{currentstroke}{rgb}{0.839216,0.152941,0.156863}%
\pgfsetstrokecolor{currentstroke}%
\pgfsetdash{}{0pt}%
\pgfpathmoveto{\pgfqpoint{1.919829in}{0.990487in}}%
\pgfpathlineto{\pgfqpoint{2.017051in}{0.990487in}}%
\pgfpathlineto{\pgfqpoint{2.114274in}{0.990487in}}%
\pgfusepath{stroke}%
\end{pgfscope}%
\begin{pgfscope}%
\pgfsetbuttcap%
\pgfsetroundjoin%
\definecolor{currentfill}{rgb}{0.839216,0.152941,0.156863}%
\pgfsetfillcolor{currentfill}%
\pgfsetlinewidth{1.003750pt}%
\definecolor{currentstroke}{rgb}{0.839216,0.152941,0.156863}%
\pgfsetstrokecolor{currentstroke}%
\pgfsetdash{}{0pt}%
\pgfsys@defobject{currentmarker}{\pgfqpoint{-0.027778in}{-0.027778in}}{\pgfqpoint{0.027778in}{0.027778in}}{%
\pgfpathmoveto{\pgfqpoint{-0.027778in}{-0.027778in}}%
\pgfpathlineto{\pgfqpoint{0.027778in}{0.027778in}}%
\pgfpathmoveto{\pgfqpoint{-0.027778in}{0.027778in}}%
\pgfpathlineto{\pgfqpoint{0.027778in}{-0.027778in}}%
\pgfusepath{stroke,fill}%
}%
\begin{pgfscope}%
\pgfsys@transformshift{2.017051in}{0.990487in}%
\pgfsys@useobject{currentmarker}{}%
\end{pgfscope}%
\end{pgfscope}%
\begin{pgfscope}%
\definecolor{textcolor}{rgb}{0.000000,0.000000,0.000000}%
\pgfsetstrokecolor{textcolor}%
\pgfsetfillcolor{textcolor}%
\pgftext[x=2.192051in,y=0.956459in,left,base]{\color{textcolor}\rmfamily\fontsize{7.000000}{8.400000}\selectfont \(\displaystyle \bar{N}=\)10,\(\displaystyle R=\)1}%
\end{pgfscope}%
\begin{pgfscope}%
\pgfsetrectcap%
\pgfsetroundjoin%
\pgfsetlinewidth{1.505625pt}%
\definecolor{currentstroke}{rgb}{0.580392,0.403922,0.741176}%
\pgfsetstrokecolor{currentstroke}%
\pgfsetdash{}{0pt}%
\pgfpathmoveto{\pgfqpoint{1.919829in}{0.842613in}}%
\pgfpathlineto{\pgfqpoint{2.017051in}{0.842613in}}%
\pgfpathlineto{\pgfqpoint{2.114274in}{0.842613in}}%
\pgfusepath{stroke}%
\end{pgfscope}%
\begin{pgfscope}%
\pgfsetbuttcap%
\pgfsetroundjoin%
\definecolor{currentfill}{rgb}{0.580392,0.403922,0.741176}%
\pgfsetfillcolor{currentfill}%
\pgfsetlinewidth{1.003750pt}%
\definecolor{currentstroke}{rgb}{0.580392,0.403922,0.741176}%
\pgfsetstrokecolor{currentstroke}%
\pgfsetdash{}{0pt}%
\pgfsys@defobject{currentmarker}{\pgfqpoint{-0.027778in}{-0.027778in}}{\pgfqpoint{0.027778in}{0.027778in}}{%
\pgfpathmoveto{\pgfqpoint{-0.027778in}{0.000000in}}%
\pgfpathlineto{\pgfqpoint{0.027778in}{0.000000in}}%
\pgfpathmoveto{\pgfqpoint{0.000000in}{-0.027778in}}%
\pgfpathlineto{\pgfqpoint{0.000000in}{0.027778in}}%
\pgfusepath{stroke,fill}%
}%
\begin{pgfscope}%
\pgfsys@transformshift{2.017051in}{0.842613in}%
\pgfsys@useobject{currentmarker}{}%
\end{pgfscope}%
\end{pgfscope}%
\begin{pgfscope}%
\definecolor{textcolor}{rgb}{0.000000,0.000000,0.000000}%
\pgfsetstrokecolor{textcolor}%
\pgfsetfillcolor{textcolor}%
\pgftext[x=2.192051in,y=0.808586in,left,base]{\color{textcolor}\rmfamily\fontsize{7.000000}{8.400000}\selectfont \(\displaystyle \bar{N}=\)10,\(\displaystyle R=\)50}%
\end{pgfscope}%
\end{pgfpicture}%
\makeatother%
\endgroup%

%     % \vspace*{-0.8cm}
%     \caption[]{A graphical representation of the order of transmission cost of a single node within the network dependent on network size. One can observe that for the neigborhood-only communication scheme, cost stays constant. (This plot might be unnecessary and space consuming)}
%     \label{fig:transcost:bigo}
% \end{figure}

\subsection[]{Simulations}
\begin{figure}[t]
    \centering
    %% Creator: Matplotlib, PGF backend
%%
%% To include the figure in your LaTeX document, write
%%   \input{<filename>.pgf}
%%
%% Make sure the required packages are loaded in your preamble
%%   \usepackage{pgf}
%%
%% Also ensure that all the required font packages are loaded; for instance,
%% the lmodern package is sometimes necessary when using math font.
%%   \usepackage{lmodern}
%%
%% Figures using additional raster images can only be included by \input if
%% they are in the same directory as the main LaTeX file. For loading figures
%% from other directories you can use the `import` package
%%   \usepackage{import}
%%
%% and then include the figures with
%%   \import{<path to file>}{<filename>.pgf}
%%
%% Matplotlib used the following preamble
%%   \usepackage{fontspec}
%%
\begingroup%
\makeatletter%
\begin{pgfpicture}%
\pgfpathrectangle{\pgfpointorigin}{\pgfqpoint{3.390065in}{2.095175in}}%
\pgfusepath{use as bounding box, clip}%
\begin{pgfscope}%
\pgfsetbuttcap%
\pgfsetmiterjoin%
\definecolor{currentfill}{rgb}{1.000000,1.000000,1.000000}%
\pgfsetfillcolor{currentfill}%
\pgfsetlinewidth{0.000000pt}%
\definecolor{currentstroke}{rgb}{1.000000,1.000000,1.000000}%
\pgfsetstrokecolor{currentstroke}%
\pgfsetstrokeopacity{0.000000}%
\pgfsetdash{}{0pt}%
\pgfpathmoveto{\pgfqpoint{0.000000in}{0.000000in}}%
\pgfpathlineto{\pgfqpoint{3.390065in}{0.000000in}}%
\pgfpathlineto{\pgfqpoint{3.390065in}{2.095175in}}%
\pgfpathlineto{\pgfqpoint{0.000000in}{2.095175in}}%
\pgfpathlineto{\pgfqpoint{0.000000in}{0.000000in}}%
\pgfpathclose%
\pgfusepath{fill}%
\end{pgfscope}%
\begin{pgfscope}%
\pgfsetbuttcap%
\pgfsetmiterjoin%
\definecolor{currentfill}{rgb}{1.000000,1.000000,1.000000}%
\pgfsetfillcolor{currentfill}%
\pgfsetlinewidth{0.000000pt}%
\definecolor{currentstroke}{rgb}{0.000000,0.000000,0.000000}%
\pgfsetstrokecolor{currentstroke}%
\pgfsetstrokeopacity{0.000000}%
\pgfsetdash{}{0pt}%
\pgfpathmoveto{\pgfqpoint{0.608025in}{0.484444in}}%
\pgfpathlineto{\pgfqpoint{3.320621in}{0.484444in}}%
\pgfpathlineto{\pgfqpoint{3.320621in}{2.025731in}}%
\pgfpathlineto{\pgfqpoint{0.608025in}{2.025731in}}%
\pgfpathlineto{\pgfqpoint{0.608025in}{0.484444in}}%
\pgfpathclose%
\pgfusepath{fill}%
\end{pgfscope}%
\begin{pgfscope}%
\pgfpathrectangle{\pgfqpoint{0.608025in}{0.484444in}}{\pgfqpoint{2.712595in}{1.541287in}}%
\pgfusepath{clip}%
\pgfsetrectcap%
\pgfsetroundjoin%
\pgfsetlinewidth{0.803000pt}%
\definecolor{currentstroke}{rgb}{0.690196,0.690196,0.690196}%
\pgfsetstrokecolor{currentstroke}%
\pgfsetdash{}{0pt}%
\pgfpathmoveto{\pgfqpoint{0.731325in}{0.484444in}}%
\pgfpathlineto{\pgfqpoint{0.731325in}{2.025731in}}%
\pgfusepath{stroke}%
\end{pgfscope}%
\begin{pgfscope}%
\pgfsetbuttcap%
\pgfsetroundjoin%
\definecolor{currentfill}{rgb}{0.000000,0.000000,0.000000}%
\pgfsetfillcolor{currentfill}%
\pgfsetlinewidth{0.803000pt}%
\definecolor{currentstroke}{rgb}{0.000000,0.000000,0.000000}%
\pgfsetstrokecolor{currentstroke}%
\pgfsetdash{}{0pt}%
\pgfsys@defobject{currentmarker}{\pgfqpoint{0.000000in}{-0.048611in}}{\pgfqpoint{0.000000in}{0.000000in}}{%
\pgfpathmoveto{\pgfqpoint{0.000000in}{0.000000in}}%
\pgfpathlineto{\pgfqpoint{0.000000in}{-0.048611in}}%
\pgfusepath{stroke,fill}%
}%
\begin{pgfscope}%
\pgfsys@transformshift{0.731325in}{0.484444in}%
\pgfsys@useobject{currentmarker}{}%
\end{pgfscope}%
\end{pgfscope}%
\begin{pgfscope}%
\definecolor{textcolor}{rgb}{0.000000,0.000000,0.000000}%
\pgfsetstrokecolor{textcolor}%
\pgfsetfillcolor{textcolor}%
\pgftext[x=0.731325in,y=0.387222in,,top]{\color{textcolor}\rmfamily\fontsize{10.000000}{12.000000}\selectfont \(\displaystyle {0}\)}%
\end{pgfscope}%
\begin{pgfscope}%
\pgfpathrectangle{\pgfqpoint{0.608025in}{0.484444in}}{\pgfqpoint{2.712595in}{1.541287in}}%
\pgfusepath{clip}%
\pgfsetrectcap%
\pgfsetroundjoin%
\pgfsetlinewidth{0.803000pt}%
\definecolor{currentstroke}{rgb}{0.690196,0.690196,0.690196}%
\pgfsetstrokecolor{currentstroke}%
\pgfsetdash{}{0pt}%
\pgfpathmoveto{\pgfqpoint{1.520823in}{0.484444in}}%
\pgfpathlineto{\pgfqpoint{1.520823in}{2.025731in}}%
\pgfusepath{stroke}%
\end{pgfscope}%
\begin{pgfscope}%
\pgfsetbuttcap%
\pgfsetroundjoin%
\definecolor{currentfill}{rgb}{0.000000,0.000000,0.000000}%
\pgfsetfillcolor{currentfill}%
\pgfsetlinewidth{0.803000pt}%
\definecolor{currentstroke}{rgb}{0.000000,0.000000,0.000000}%
\pgfsetstrokecolor{currentstroke}%
\pgfsetdash{}{0pt}%
\pgfsys@defobject{currentmarker}{\pgfqpoint{0.000000in}{-0.048611in}}{\pgfqpoint{0.000000in}{0.000000in}}{%
\pgfpathmoveto{\pgfqpoint{0.000000in}{0.000000in}}%
\pgfpathlineto{\pgfqpoint{0.000000in}{-0.048611in}}%
\pgfusepath{stroke,fill}%
}%
\begin{pgfscope}%
\pgfsys@transformshift{1.520823in}{0.484444in}%
\pgfsys@useobject{currentmarker}{}%
\end{pgfscope}%
\end{pgfscope}%
\begin{pgfscope}%
\definecolor{textcolor}{rgb}{0.000000,0.000000,0.000000}%
\pgfsetstrokecolor{textcolor}%
\pgfsetfillcolor{textcolor}%
\pgftext[x=1.520823in,y=0.387222in,,top]{\color{textcolor}\rmfamily\fontsize{10.000000}{12.000000}\selectfont \(\displaystyle {2000}\)}%
\end{pgfscope}%
\begin{pgfscope}%
\pgfpathrectangle{\pgfqpoint{0.608025in}{0.484444in}}{\pgfqpoint{2.712595in}{1.541287in}}%
\pgfusepath{clip}%
\pgfsetrectcap%
\pgfsetroundjoin%
\pgfsetlinewidth{0.803000pt}%
\definecolor{currentstroke}{rgb}{0.690196,0.690196,0.690196}%
\pgfsetstrokecolor{currentstroke}%
\pgfsetdash{}{0pt}%
\pgfpathmoveto{\pgfqpoint{2.310320in}{0.484444in}}%
\pgfpathlineto{\pgfqpoint{2.310320in}{2.025731in}}%
\pgfusepath{stroke}%
\end{pgfscope}%
\begin{pgfscope}%
\pgfsetbuttcap%
\pgfsetroundjoin%
\definecolor{currentfill}{rgb}{0.000000,0.000000,0.000000}%
\pgfsetfillcolor{currentfill}%
\pgfsetlinewidth{0.803000pt}%
\definecolor{currentstroke}{rgb}{0.000000,0.000000,0.000000}%
\pgfsetstrokecolor{currentstroke}%
\pgfsetdash{}{0pt}%
\pgfsys@defobject{currentmarker}{\pgfqpoint{0.000000in}{-0.048611in}}{\pgfqpoint{0.000000in}{0.000000in}}{%
\pgfpathmoveto{\pgfqpoint{0.000000in}{0.000000in}}%
\pgfpathlineto{\pgfqpoint{0.000000in}{-0.048611in}}%
\pgfusepath{stroke,fill}%
}%
\begin{pgfscope}%
\pgfsys@transformshift{2.310320in}{0.484444in}%
\pgfsys@useobject{currentmarker}{}%
\end{pgfscope}%
\end{pgfscope}%
\begin{pgfscope}%
\definecolor{textcolor}{rgb}{0.000000,0.000000,0.000000}%
\pgfsetstrokecolor{textcolor}%
\pgfsetfillcolor{textcolor}%
\pgftext[x=2.310320in,y=0.387222in,,top]{\color{textcolor}\rmfamily\fontsize{10.000000}{12.000000}\selectfont \(\displaystyle {4000}\)}%
\end{pgfscope}%
\begin{pgfscope}%
\pgfpathrectangle{\pgfqpoint{0.608025in}{0.484444in}}{\pgfqpoint{2.712595in}{1.541287in}}%
\pgfusepath{clip}%
\pgfsetrectcap%
\pgfsetroundjoin%
\pgfsetlinewidth{0.803000pt}%
\definecolor{currentstroke}{rgb}{0.690196,0.690196,0.690196}%
\pgfsetstrokecolor{currentstroke}%
\pgfsetdash{}{0pt}%
\pgfpathmoveto{\pgfqpoint{3.099818in}{0.484444in}}%
\pgfpathlineto{\pgfqpoint{3.099818in}{2.025731in}}%
\pgfusepath{stroke}%
\end{pgfscope}%
\begin{pgfscope}%
\pgfsetbuttcap%
\pgfsetroundjoin%
\definecolor{currentfill}{rgb}{0.000000,0.000000,0.000000}%
\pgfsetfillcolor{currentfill}%
\pgfsetlinewidth{0.803000pt}%
\definecolor{currentstroke}{rgb}{0.000000,0.000000,0.000000}%
\pgfsetstrokecolor{currentstroke}%
\pgfsetdash{}{0pt}%
\pgfsys@defobject{currentmarker}{\pgfqpoint{0.000000in}{-0.048611in}}{\pgfqpoint{0.000000in}{0.000000in}}{%
\pgfpathmoveto{\pgfqpoint{0.000000in}{0.000000in}}%
\pgfpathlineto{\pgfqpoint{0.000000in}{-0.048611in}}%
\pgfusepath{stroke,fill}%
}%
\begin{pgfscope}%
\pgfsys@transformshift{3.099818in}{0.484444in}%
\pgfsys@useobject{currentmarker}{}%
\end{pgfscope}%
\end{pgfscope}%
\begin{pgfscope}%
\definecolor{textcolor}{rgb}{0.000000,0.000000,0.000000}%
\pgfsetstrokecolor{textcolor}%
\pgfsetfillcolor{textcolor}%
\pgftext[x=3.099818in,y=0.387222in,,top]{\color{textcolor}\rmfamily\fontsize{10.000000}{12.000000}\selectfont \(\displaystyle {6000}\)}%
\end{pgfscope}%
\begin{pgfscope}%
\definecolor{textcolor}{rgb}{0.000000,0.000000,0.000000}%
\pgfsetstrokecolor{textcolor}%
\pgfsetfillcolor{textcolor}%
\pgftext[x=1.964323in,y=0.208333in,,top]{\color{textcolor}\rmfamily\fontsize{10.000000}{12.000000}\selectfont Time [frames]}%
\end{pgfscope}%
\begin{pgfscope}%
\pgfpathrectangle{\pgfqpoint{0.608025in}{0.484444in}}{\pgfqpoint{2.712595in}{1.541287in}}%
\pgfusepath{clip}%
\pgfsetrectcap%
\pgfsetroundjoin%
\pgfsetlinewidth{0.803000pt}%
\definecolor{currentstroke}{rgb}{0.690196,0.690196,0.690196}%
\pgfsetstrokecolor{currentstroke}%
\pgfsetdash{}{0pt}%
\pgfpathmoveto{\pgfqpoint{0.608025in}{0.797730in}}%
\pgfpathlineto{\pgfqpoint{3.320621in}{0.797730in}}%
\pgfusepath{stroke}%
\end{pgfscope}%
\begin{pgfscope}%
\pgfsetbuttcap%
\pgfsetroundjoin%
\definecolor{currentfill}{rgb}{0.000000,0.000000,0.000000}%
\pgfsetfillcolor{currentfill}%
\pgfsetlinewidth{0.803000pt}%
\definecolor{currentstroke}{rgb}{0.000000,0.000000,0.000000}%
\pgfsetstrokecolor{currentstroke}%
\pgfsetdash{}{0pt}%
\pgfsys@defobject{currentmarker}{\pgfqpoint{-0.048611in}{0.000000in}}{\pgfqpoint{-0.000000in}{0.000000in}}{%
\pgfpathmoveto{\pgfqpoint{-0.000000in}{0.000000in}}%
\pgfpathlineto{\pgfqpoint{-0.048611in}{0.000000in}}%
\pgfusepath{stroke,fill}%
}%
\begin{pgfscope}%
\pgfsys@transformshift{0.608025in}{0.797730in}%
\pgfsys@useobject{currentmarker}{}%
\end{pgfscope}%
\end{pgfscope}%
\begin{pgfscope}%
\definecolor{textcolor}{rgb}{0.000000,0.000000,0.000000}%
\pgfsetstrokecolor{textcolor}%
\pgfsetfillcolor{textcolor}%
\pgftext[x=0.263889in, y=0.749535in, left, base]{\color{textcolor}\rmfamily\fontsize{10.000000}{12.000000}\selectfont \(\displaystyle {\ensuremath{-}30}\)}%
\end{pgfscope}%
\begin{pgfscope}%
\pgfpathrectangle{\pgfqpoint{0.608025in}{0.484444in}}{\pgfqpoint{2.712595in}{1.541287in}}%
\pgfusepath{clip}%
\pgfsetrectcap%
\pgfsetroundjoin%
\pgfsetlinewidth{0.803000pt}%
\definecolor{currentstroke}{rgb}{0.690196,0.690196,0.690196}%
\pgfsetstrokecolor{currentstroke}%
\pgfsetdash{}{0pt}%
\pgfpathmoveto{\pgfqpoint{0.608025in}{1.184024in}}%
\pgfpathlineto{\pgfqpoint{3.320621in}{1.184024in}}%
\pgfusepath{stroke}%
\end{pgfscope}%
\begin{pgfscope}%
\pgfsetbuttcap%
\pgfsetroundjoin%
\definecolor{currentfill}{rgb}{0.000000,0.000000,0.000000}%
\pgfsetfillcolor{currentfill}%
\pgfsetlinewidth{0.803000pt}%
\definecolor{currentstroke}{rgb}{0.000000,0.000000,0.000000}%
\pgfsetstrokecolor{currentstroke}%
\pgfsetdash{}{0pt}%
\pgfsys@defobject{currentmarker}{\pgfqpoint{-0.048611in}{0.000000in}}{\pgfqpoint{-0.000000in}{0.000000in}}{%
\pgfpathmoveto{\pgfqpoint{-0.000000in}{0.000000in}}%
\pgfpathlineto{\pgfqpoint{-0.048611in}{0.000000in}}%
\pgfusepath{stroke,fill}%
}%
\begin{pgfscope}%
\pgfsys@transformshift{0.608025in}{1.184024in}%
\pgfsys@useobject{currentmarker}{}%
\end{pgfscope}%
\end{pgfscope}%
\begin{pgfscope}%
\definecolor{textcolor}{rgb}{0.000000,0.000000,0.000000}%
\pgfsetstrokecolor{textcolor}%
\pgfsetfillcolor{textcolor}%
\pgftext[x=0.263889in, y=1.135830in, left, base]{\color{textcolor}\rmfamily\fontsize{10.000000}{12.000000}\selectfont \(\displaystyle {\ensuremath{-}20}\)}%
\end{pgfscope}%
\begin{pgfscope}%
\pgfpathrectangle{\pgfqpoint{0.608025in}{0.484444in}}{\pgfqpoint{2.712595in}{1.541287in}}%
\pgfusepath{clip}%
\pgfsetrectcap%
\pgfsetroundjoin%
\pgfsetlinewidth{0.803000pt}%
\definecolor{currentstroke}{rgb}{0.690196,0.690196,0.690196}%
\pgfsetstrokecolor{currentstroke}%
\pgfsetdash{}{0pt}%
\pgfpathmoveto{\pgfqpoint{0.608025in}{1.570319in}}%
\pgfpathlineto{\pgfqpoint{3.320621in}{1.570319in}}%
\pgfusepath{stroke}%
\end{pgfscope}%
\begin{pgfscope}%
\pgfsetbuttcap%
\pgfsetroundjoin%
\definecolor{currentfill}{rgb}{0.000000,0.000000,0.000000}%
\pgfsetfillcolor{currentfill}%
\pgfsetlinewidth{0.803000pt}%
\definecolor{currentstroke}{rgb}{0.000000,0.000000,0.000000}%
\pgfsetstrokecolor{currentstroke}%
\pgfsetdash{}{0pt}%
\pgfsys@defobject{currentmarker}{\pgfqpoint{-0.048611in}{0.000000in}}{\pgfqpoint{-0.000000in}{0.000000in}}{%
\pgfpathmoveto{\pgfqpoint{-0.000000in}{0.000000in}}%
\pgfpathlineto{\pgfqpoint{-0.048611in}{0.000000in}}%
\pgfusepath{stroke,fill}%
}%
\begin{pgfscope}%
\pgfsys@transformshift{0.608025in}{1.570319in}%
\pgfsys@useobject{currentmarker}{}%
\end{pgfscope}%
\end{pgfscope}%
\begin{pgfscope}%
\definecolor{textcolor}{rgb}{0.000000,0.000000,0.000000}%
\pgfsetstrokecolor{textcolor}%
\pgfsetfillcolor{textcolor}%
\pgftext[x=0.263889in, y=1.522124in, left, base]{\color{textcolor}\rmfamily\fontsize{10.000000}{12.000000}\selectfont \(\displaystyle {\ensuremath{-}10}\)}%
\end{pgfscope}%
\begin{pgfscope}%
\pgfpathrectangle{\pgfqpoint{0.608025in}{0.484444in}}{\pgfqpoint{2.712595in}{1.541287in}}%
\pgfusepath{clip}%
\pgfsetrectcap%
\pgfsetroundjoin%
\pgfsetlinewidth{0.803000pt}%
\definecolor{currentstroke}{rgb}{0.690196,0.690196,0.690196}%
\pgfsetstrokecolor{currentstroke}%
\pgfsetdash{}{0pt}%
\pgfpathmoveto{\pgfqpoint{0.608025in}{1.956613in}}%
\pgfpathlineto{\pgfqpoint{3.320621in}{1.956613in}}%
\pgfusepath{stroke}%
\end{pgfscope}%
\begin{pgfscope}%
\pgfsetbuttcap%
\pgfsetroundjoin%
\definecolor{currentfill}{rgb}{0.000000,0.000000,0.000000}%
\pgfsetfillcolor{currentfill}%
\pgfsetlinewidth{0.803000pt}%
\definecolor{currentstroke}{rgb}{0.000000,0.000000,0.000000}%
\pgfsetstrokecolor{currentstroke}%
\pgfsetdash{}{0pt}%
\pgfsys@defobject{currentmarker}{\pgfqpoint{-0.048611in}{0.000000in}}{\pgfqpoint{-0.000000in}{0.000000in}}{%
\pgfpathmoveto{\pgfqpoint{-0.000000in}{0.000000in}}%
\pgfpathlineto{\pgfqpoint{-0.048611in}{0.000000in}}%
\pgfusepath{stroke,fill}%
}%
\begin{pgfscope}%
\pgfsys@transformshift{0.608025in}{1.956613in}%
\pgfsys@useobject{currentmarker}{}%
\end{pgfscope}%
\end{pgfscope}%
\begin{pgfscope}%
\definecolor{textcolor}{rgb}{0.000000,0.000000,0.000000}%
\pgfsetstrokecolor{textcolor}%
\pgfsetfillcolor{textcolor}%
\pgftext[x=0.441358in, y=1.908419in, left, base]{\color{textcolor}\rmfamily\fontsize{10.000000}{12.000000}\selectfont \(\displaystyle {0}\)}%
\end{pgfscope}%
\begin{pgfscope}%
\definecolor{textcolor}{rgb}{0.000000,0.000000,0.000000}%
\pgfsetstrokecolor{textcolor}%
\pgfsetfillcolor{textcolor}%
\pgftext[x=0.208333in,y=1.255087in,,bottom,rotate=90.000000]{\color{textcolor}\rmfamily\fontsize{10.000000}{12.000000}\selectfont NPM [dB]}%
\end{pgfscope}%
\begin{pgfscope}%
\pgfpathrectangle{\pgfqpoint{0.608025in}{0.484444in}}{\pgfqpoint{2.712595in}{1.541287in}}%
\pgfusepath{clip}%
\pgfsetrectcap%
\pgfsetroundjoin%
\pgfsetlinewidth{1.505625pt}%
\definecolor{currentstroke}{rgb}{0.121569,0.466667,0.705882}%
\pgfsetstrokecolor{currentstroke}%
\pgfsetdash{}{0pt}%
\pgfpathmoveto{\pgfqpoint{0.731325in}{1.955572in}}%
\pgfpathlineto{\pgfqpoint{0.735272in}{1.953805in}}%
\pgfpathlineto{\pgfqpoint{0.742773in}{1.948385in}}%
\pgfpathlineto{\pgfqpoint{0.749878in}{1.937440in}}%
\pgfpathlineto{\pgfqpoint{0.759747in}{1.913701in}}%
\pgfpathlineto{\pgfqpoint{0.789353in}{1.824407in}}%
\pgfpathlineto{\pgfqpoint{0.803169in}{1.747076in}}%
\pgfpathlineto{\pgfqpoint{0.809880in}{1.715798in}}%
\pgfpathlineto{\pgfqpoint{0.836328in}{1.613408in}}%
\pgfpathlineto{\pgfqpoint{0.851329in}{1.532205in}}%
\pgfpathlineto{\pgfqpoint{0.862382in}{1.475416in}}%
\pgfpathlineto{\pgfqpoint{0.887645in}{1.380410in}}%
\pgfpathlineto{\pgfqpoint{0.910146in}{1.302399in}}%
\pgfpathlineto{\pgfqpoint{0.928305in}{1.237981in}}%
\pgfpathlineto{\pgfqpoint{0.947647in}{1.187605in}}%
\pgfpathlineto{\pgfqpoint{0.966595in}{1.146449in}}%
\pgfpathlineto{\pgfqpoint{0.989885in}{1.061082in}}%
\pgfpathlineto{\pgfqpoint{1.020676in}{0.995645in}}%
\pgfpathlineto{\pgfqpoint{1.022255in}{0.992012in}}%
\pgfpathlineto{\pgfqpoint{1.027387in}{0.983122in}}%
\pgfpathlineto{\pgfqpoint{1.031334in}{0.977077in}}%
\pgfpathlineto{\pgfqpoint{1.036071in}{0.969265in}}%
\pgfpathlineto{\pgfqpoint{1.054229in}{0.941738in}}%
\pgfpathlineto{\pgfqpoint{1.058572in}{0.933586in}}%
\pgfpathlineto{\pgfqpoint{1.060151in}{0.929972in}}%
\pgfpathlineto{\pgfqpoint{1.067256in}{0.907828in}}%
\pgfpathlineto{\pgfqpoint{1.068440in}{0.908369in}}%
\pgfpathlineto{\pgfqpoint{1.071598in}{0.907985in}}%
\pgfpathlineto{\pgfqpoint{1.094889in}{0.855484in}}%
\pgfpathlineto{\pgfqpoint{1.099231in}{0.850076in}}%
\pgfpathlineto{\pgfqpoint{1.105547in}{0.836578in}}%
\pgfpathlineto{\pgfqpoint{1.107915in}{0.832093in}}%
\pgfpathlineto{\pgfqpoint{1.119363in}{0.803032in}}%
\pgfpathlineto{\pgfqpoint{1.125679in}{0.787975in}}%
\pgfpathlineto{\pgfqpoint{1.131600in}{0.783389in}}%
\pgfpathlineto{\pgfqpoint{1.131995in}{0.783909in}}%
\pgfpathlineto{\pgfqpoint{1.134758in}{0.786988in}}%
\pgfpathlineto{\pgfqpoint{1.135153in}{0.786608in}}%
\pgfpathlineto{\pgfqpoint{1.153311in}{0.735198in}}%
\pgfpathlineto{\pgfqpoint{1.155285in}{0.732051in}}%
\pgfpathlineto{\pgfqpoint{1.155680in}{0.732982in}}%
\pgfpathlineto{\pgfqpoint{1.156075in}{0.732455in}}%
\pgfpathlineto{\pgfqpoint{1.159627in}{0.720116in}}%
\pgfpathlineto{\pgfqpoint{1.160812in}{0.719361in}}%
\pgfpathlineto{\pgfqpoint{1.161206in}{0.720073in}}%
\pgfpathlineto{\pgfqpoint{1.165154in}{0.732769in}}%
\pgfpathlineto{\pgfqpoint{1.165943in}{0.731504in}}%
\pgfpathlineto{\pgfqpoint{1.177391in}{0.695600in}}%
\pgfpathlineto{\pgfqpoint{1.178575in}{0.692422in}}%
\pgfpathlineto{\pgfqpoint{1.178970in}{0.692697in}}%
\pgfpathlineto{\pgfqpoint{1.182918in}{0.701034in}}%
\pgfpathlineto{\pgfqpoint{1.183312in}{0.701275in}}%
\pgfpathlineto{\pgfqpoint{1.190418in}{0.678679in}}%
\pgfpathlineto{\pgfqpoint{1.192392in}{0.676611in}}%
\pgfpathlineto{\pgfqpoint{1.195550in}{0.667624in}}%
\pgfpathlineto{\pgfqpoint{1.196734in}{0.669271in}}%
\pgfpathlineto{\pgfqpoint{1.198708in}{0.667457in}}%
\pgfpathlineto{\pgfqpoint{1.200287in}{0.664693in}}%
\pgfpathlineto{\pgfqpoint{1.200681in}{0.665090in}}%
\pgfpathlineto{\pgfqpoint{1.201866in}{0.663950in}}%
\pgfpathlineto{\pgfqpoint{1.203445in}{0.661761in}}%
\pgfpathlineto{\pgfqpoint{1.203839in}{0.661994in}}%
\pgfpathlineto{\pgfqpoint{1.204629in}{0.660625in}}%
\pgfpathlineto{\pgfqpoint{1.208182in}{0.643604in}}%
\pgfpathlineto{\pgfqpoint{1.208971in}{0.645085in}}%
\pgfpathlineto{\pgfqpoint{1.211734in}{0.648280in}}%
\pgfpathlineto{\pgfqpoint{1.212918in}{0.647101in}}%
\pgfpathlineto{\pgfqpoint{1.213313in}{0.648057in}}%
\pgfpathlineto{\pgfqpoint{1.215287in}{0.651875in}}%
\pgfpathlineto{\pgfqpoint{1.215682in}{0.651349in}}%
\pgfpathlineto{\pgfqpoint{1.219629in}{0.642736in}}%
\pgfpathlineto{\pgfqpoint{1.220419in}{0.643985in}}%
\pgfpathlineto{\pgfqpoint{1.221208in}{0.645025in}}%
\pgfpathlineto{\pgfqpoint{1.221603in}{0.643450in}}%
\pgfpathlineto{\pgfqpoint{1.222787in}{0.642832in}}%
\pgfpathlineto{\pgfqpoint{1.224761in}{0.638338in}}%
\pgfpathlineto{\pgfqpoint{1.225156in}{0.640008in}}%
\pgfpathlineto{\pgfqpoint{1.225945in}{0.637987in}}%
\pgfpathlineto{\pgfqpoint{1.226735in}{0.636719in}}%
\pgfpathlineto{\pgfqpoint{1.227129in}{0.637829in}}%
\pgfpathlineto{\pgfqpoint{1.228708in}{0.645700in}}%
\pgfpathlineto{\pgfqpoint{1.231077in}{0.660910in}}%
\pgfpathlineto{\pgfqpoint{1.231472in}{0.658756in}}%
\pgfpathlineto{\pgfqpoint{1.235419in}{0.650288in}}%
\pgfpathlineto{\pgfqpoint{1.235814in}{0.650526in}}%
\pgfpathlineto{\pgfqpoint{1.236998in}{0.652134in}}%
\pgfpathlineto{\pgfqpoint{1.238182in}{0.659354in}}%
\pgfpathlineto{\pgfqpoint{1.239367in}{0.658431in}}%
\pgfpathlineto{\pgfqpoint{1.240156in}{0.655540in}}%
\pgfpathlineto{\pgfqpoint{1.241340in}{0.657587in}}%
\pgfpathlineto{\pgfqpoint{1.243314in}{0.662185in}}%
\pgfpathlineto{\pgfqpoint{1.245288in}{0.647846in}}%
\pgfpathlineto{\pgfqpoint{1.246077in}{0.648462in}}%
\pgfpathlineto{\pgfqpoint{1.246472in}{0.648322in}}%
\pgfpathlineto{\pgfqpoint{1.247656in}{0.650534in}}%
\pgfpathlineto{\pgfqpoint{1.248051in}{0.649543in}}%
\pgfpathlineto{\pgfqpoint{1.251999in}{0.647604in}}%
\pgfpathlineto{\pgfqpoint{1.252788in}{0.644454in}}%
\pgfpathlineto{\pgfqpoint{1.253578in}{0.646057in}}%
\pgfpathlineto{\pgfqpoint{1.254762in}{0.648554in}}%
\pgfpathlineto{\pgfqpoint{1.257525in}{0.656297in}}%
\pgfpathlineto{\pgfqpoint{1.259499in}{0.656350in}}%
\pgfpathlineto{\pgfqpoint{1.260683in}{0.659363in}}%
\pgfpathlineto{\pgfqpoint{1.261473in}{0.657795in}}%
\pgfpathlineto{\pgfqpoint{1.263446in}{0.653268in}}%
\pgfpathlineto{\pgfqpoint{1.264631in}{0.656098in}}%
\pgfpathlineto{\pgfqpoint{1.265420in}{0.656474in}}%
\pgfpathlineto{\pgfqpoint{1.265815in}{0.655370in}}%
\pgfpathlineto{\pgfqpoint{1.266604in}{0.654143in}}%
\pgfpathlineto{\pgfqpoint{1.266999in}{0.655648in}}%
\pgfpathlineto{\pgfqpoint{1.268973in}{0.660892in}}%
\pgfpathlineto{\pgfqpoint{1.269368in}{0.660434in}}%
\pgfpathlineto{\pgfqpoint{1.270552in}{0.658508in}}%
\pgfpathlineto{\pgfqpoint{1.273315in}{0.648887in}}%
\pgfpathlineto{\pgfqpoint{1.273710in}{0.651585in}}%
\pgfpathlineto{\pgfqpoint{1.274499in}{0.650175in}}%
\pgfpathlineto{\pgfqpoint{1.274894in}{0.648539in}}%
\pgfpathlineto{\pgfqpoint{1.275684in}{0.649379in}}%
\pgfpathlineto{\pgfqpoint{1.278052in}{0.661840in}}%
\pgfpathlineto{\pgfqpoint{1.280421in}{0.665968in}}%
\pgfpathlineto{\pgfqpoint{1.283579in}{0.649829in}}%
\pgfpathlineto{\pgfqpoint{1.284368in}{0.652129in}}%
\pgfpathlineto{\pgfqpoint{1.286737in}{0.654811in}}%
\pgfpathlineto{\pgfqpoint{1.288710in}{0.663566in}}%
\pgfpathlineto{\pgfqpoint{1.289105in}{0.662658in}}%
\pgfpathlineto{\pgfqpoint{1.295026in}{0.645520in}}%
\pgfpathlineto{\pgfqpoint{1.296210in}{0.644456in}}%
\pgfpathlineto{\pgfqpoint{1.297000in}{0.641485in}}%
\pgfpathlineto{\pgfqpoint{1.297789in}{0.641934in}}%
\pgfpathlineto{\pgfqpoint{1.298184in}{0.642889in}}%
\pgfpathlineto{\pgfqpoint{1.298579in}{0.641025in}}%
\pgfpathlineto{\pgfqpoint{1.301342in}{0.640618in}}%
\pgfpathlineto{\pgfqpoint{1.309237in}{0.666014in}}%
\pgfpathlineto{\pgfqpoint{1.309632in}{0.664664in}}%
\pgfpathlineto{\pgfqpoint{1.312395in}{0.657953in}}%
\pgfpathlineto{\pgfqpoint{1.319501in}{0.682290in}}%
\pgfpathlineto{\pgfqpoint{1.319895in}{0.681038in}}%
\pgfpathlineto{\pgfqpoint{1.320685in}{0.683087in}}%
\pgfpathlineto{\pgfqpoint{1.321080in}{0.684780in}}%
\pgfpathlineto{\pgfqpoint{1.321869in}{0.682700in}}%
\pgfpathlineto{\pgfqpoint{1.322659in}{0.681011in}}%
\pgfpathlineto{\pgfqpoint{1.324238in}{0.671607in}}%
\pgfpathlineto{\pgfqpoint{1.325422in}{0.672704in}}%
\pgfpathlineto{\pgfqpoint{1.327790in}{0.677577in}}%
\pgfpathlineto{\pgfqpoint{1.330159in}{0.680136in}}%
\pgfpathlineto{\pgfqpoint{1.338843in}{0.667882in}}%
\pgfpathlineto{\pgfqpoint{1.340422in}{0.665643in}}%
\pgfpathlineto{\pgfqpoint{1.343580in}{0.656755in}}%
\pgfpathlineto{\pgfqpoint{1.343975in}{0.652764in}}%
\pgfpathlineto{\pgfqpoint{1.344765in}{0.655080in}}%
\pgfpathlineto{\pgfqpoint{1.345949in}{0.657121in}}%
\pgfpathlineto{\pgfqpoint{1.346344in}{0.656657in}}%
\pgfpathlineto{\pgfqpoint{1.347133in}{0.653680in}}%
\pgfpathlineto{\pgfqpoint{1.347923in}{0.656361in}}%
\pgfpathlineto{\pgfqpoint{1.349107in}{0.655035in}}%
\pgfpathlineto{\pgfqpoint{1.351870in}{0.646551in}}%
\pgfpathlineto{\pgfqpoint{1.352265in}{0.646574in}}%
\pgfpathlineto{\pgfqpoint{1.358186in}{0.626404in}}%
\pgfpathlineto{\pgfqpoint{1.358581in}{0.626077in}}%
\pgfpathlineto{\pgfqpoint{1.358976in}{0.627673in}}%
\pgfpathlineto{\pgfqpoint{1.360555in}{0.633378in}}%
\pgfpathlineto{\pgfqpoint{1.361739in}{0.631806in}}%
\pgfpathlineto{\pgfqpoint{1.362134in}{0.630763in}}%
\pgfpathlineto{\pgfqpoint{1.363318in}{0.631481in}}%
\pgfpathlineto{\pgfqpoint{1.364502in}{0.632506in}}%
\pgfpathlineto{\pgfqpoint{1.364897in}{0.633218in}}%
\pgfpathlineto{\pgfqpoint{1.365686in}{0.632433in}}%
\pgfpathlineto{\pgfqpoint{1.368055in}{0.628989in}}%
\pgfpathlineto{\pgfqpoint{1.368844in}{0.632096in}}%
\pgfpathlineto{\pgfqpoint{1.369634in}{0.630399in}}%
\pgfpathlineto{\pgfqpoint{1.370029in}{0.630494in}}%
\pgfpathlineto{\pgfqpoint{1.370818in}{0.626078in}}%
\pgfpathlineto{\pgfqpoint{1.371608in}{0.628870in}}%
\pgfpathlineto{\pgfqpoint{1.375950in}{0.647687in}}%
\pgfpathlineto{\pgfqpoint{1.376739in}{0.646219in}}%
\pgfpathlineto{\pgfqpoint{1.378318in}{0.641330in}}%
\pgfpathlineto{\pgfqpoint{1.379108in}{0.642306in}}%
\pgfpathlineto{\pgfqpoint{1.380687in}{0.639748in}}%
\pgfpathlineto{\pgfqpoint{1.385029in}{0.628184in}}%
\pgfpathlineto{\pgfqpoint{1.388187in}{0.636175in}}%
\pgfpathlineto{\pgfqpoint{1.389371in}{0.635557in}}%
\pgfpathlineto{\pgfqpoint{1.392134in}{0.629524in}}%
\pgfpathlineto{\pgfqpoint{1.395687in}{0.609627in}}%
\pgfpathlineto{\pgfqpoint{1.396082in}{0.608849in}}%
\pgfpathlineto{\pgfqpoint{1.398450in}{0.616205in}}%
\pgfpathlineto{\pgfqpoint{1.405556in}{0.619224in}}%
\pgfpathlineto{\pgfqpoint{1.406740in}{0.615715in}}%
\pgfpathlineto{\pgfqpoint{1.408714in}{0.607146in}}%
\pgfpathlineto{\pgfqpoint{1.409109in}{0.607286in}}%
\pgfpathlineto{\pgfqpoint{1.413451in}{0.607236in}}%
\pgfpathlineto{\pgfqpoint{1.414635in}{0.611888in}}%
\pgfpathlineto{\pgfqpoint{1.415425in}{0.609978in}}%
\pgfpathlineto{\pgfqpoint{1.415819in}{0.609427in}}%
\pgfpathlineto{\pgfqpoint{1.416214in}{0.610990in}}%
\pgfpathlineto{\pgfqpoint{1.419372in}{0.621364in}}%
\pgfpathlineto{\pgfqpoint{1.420951in}{0.625869in}}%
\pgfpathlineto{\pgfqpoint{1.421346in}{0.625679in}}%
\pgfpathlineto{\pgfqpoint{1.422135in}{0.627231in}}%
\pgfpathlineto{\pgfqpoint{1.424109in}{0.631700in}}%
\pgfpathlineto{\pgfqpoint{1.426478in}{0.628545in}}%
\pgfpathlineto{\pgfqpoint{1.428451in}{0.612263in}}%
\pgfpathlineto{\pgfqpoint{1.428846in}{0.612355in}}%
\pgfpathlineto{\pgfqpoint{1.430030in}{0.613126in}}%
\pgfpathlineto{\pgfqpoint{1.432399in}{0.601836in}}%
\pgfpathlineto{\pgfqpoint{1.433188in}{0.604656in}}%
\pgfpathlineto{\pgfqpoint{1.436346in}{0.612447in}}%
\pgfpathlineto{\pgfqpoint{1.436741in}{0.610835in}}%
\pgfpathlineto{\pgfqpoint{1.437531in}{0.609334in}}%
\pgfpathlineto{\pgfqpoint{1.437925in}{0.610648in}}%
\pgfpathlineto{\pgfqpoint{1.439504in}{0.612480in}}%
\pgfpathlineto{\pgfqpoint{1.439899in}{0.611733in}}%
\pgfpathlineto{\pgfqpoint{1.440294in}{0.610086in}}%
\pgfpathlineto{\pgfqpoint{1.441083in}{0.613388in}}%
\pgfpathlineto{\pgfqpoint{1.443847in}{0.620246in}}%
\pgfpathlineto{\pgfqpoint{1.444636in}{0.619613in}}%
\pgfpathlineto{\pgfqpoint{1.445031in}{0.619891in}}%
\pgfpathlineto{\pgfqpoint{1.445820in}{0.618601in}}%
\pgfpathlineto{\pgfqpoint{1.447399in}{0.620149in}}%
\pgfpathlineto{\pgfqpoint{1.449768in}{0.625417in}}%
\pgfpathlineto{\pgfqpoint{1.450163in}{0.625196in}}%
\pgfpathlineto{\pgfqpoint{1.451742in}{0.618556in}}%
\pgfpathlineto{\pgfqpoint{1.452926in}{0.612249in}}%
\pgfpathlineto{\pgfqpoint{1.453715in}{0.613530in}}%
\pgfpathlineto{\pgfqpoint{1.454900in}{0.613408in}}%
\pgfpathlineto{\pgfqpoint{1.455294in}{0.612473in}}%
\pgfpathlineto{\pgfqpoint{1.456084in}{0.613766in}}%
\pgfpathlineto{\pgfqpoint{1.458058in}{0.616025in}}%
\pgfpathlineto{\pgfqpoint{1.459242in}{0.621475in}}%
\pgfpathlineto{\pgfqpoint{1.459637in}{0.619683in}}%
\pgfpathlineto{\pgfqpoint{1.462005in}{0.610438in}}%
\pgfpathlineto{\pgfqpoint{1.462794in}{0.614795in}}%
\pgfpathlineto{\pgfqpoint{1.463584in}{0.611367in}}%
\pgfpathlineto{\pgfqpoint{1.463979in}{0.610841in}}%
\pgfpathlineto{\pgfqpoint{1.464768in}{0.606070in}}%
\pgfpathlineto{\pgfqpoint{1.465558in}{0.607298in}}%
\pgfpathlineto{\pgfqpoint{1.466347in}{0.609231in}}%
\pgfpathlineto{\pgfqpoint{1.466742in}{0.607088in}}%
\pgfpathlineto{\pgfqpoint{1.469505in}{0.601915in}}%
\pgfpathlineto{\pgfqpoint{1.472663in}{0.612546in}}%
\pgfpathlineto{\pgfqpoint{1.474242in}{0.609911in}}%
\pgfpathlineto{\pgfqpoint{1.474637in}{0.611176in}}%
\pgfpathlineto{\pgfqpoint{1.477400in}{0.620061in}}%
\pgfpathlineto{\pgfqpoint{1.478190in}{0.623996in}}%
\pgfpathlineto{\pgfqpoint{1.479374in}{0.623544in}}%
\pgfpathlineto{\pgfqpoint{1.480558in}{0.622368in}}%
\pgfpathlineto{\pgfqpoint{1.480953in}{0.623300in}}%
\pgfpathlineto{\pgfqpoint{1.481742in}{0.628028in}}%
\pgfpathlineto{\pgfqpoint{1.482532in}{0.625135in}}%
\pgfpathlineto{\pgfqpoint{1.483321in}{0.623128in}}%
\pgfpathlineto{\pgfqpoint{1.484111in}{0.624702in}}%
\pgfpathlineto{\pgfqpoint{1.486874in}{0.629497in}}%
\pgfpathlineto{\pgfqpoint{1.487269in}{0.627242in}}%
\pgfpathlineto{\pgfqpoint{1.488453in}{0.621354in}}%
\pgfpathlineto{\pgfqpoint{1.489243in}{0.623008in}}%
\pgfpathlineto{\pgfqpoint{1.489637in}{0.623748in}}%
\pgfpathlineto{\pgfqpoint{1.490427in}{0.622091in}}%
\pgfpathlineto{\pgfqpoint{1.491611in}{0.617705in}}%
\pgfpathlineto{\pgfqpoint{1.492795in}{0.620058in}}%
\pgfpathlineto{\pgfqpoint{1.493585in}{0.621155in}}%
\pgfpathlineto{\pgfqpoint{1.494374in}{0.619769in}}%
\pgfpathlineto{\pgfqpoint{1.495559in}{0.619656in}}%
\pgfpathlineto{\pgfqpoint{1.499506in}{0.600560in}}%
\pgfpathlineto{\pgfqpoint{1.502269in}{0.606828in}}%
\pgfpathlineto{\pgfqpoint{1.502664in}{0.605008in}}%
\pgfpathlineto{\pgfqpoint{1.503848in}{0.606306in}}%
\pgfpathlineto{\pgfqpoint{1.505427in}{0.605197in}}%
\pgfpathlineto{\pgfqpoint{1.506612in}{0.603345in}}%
\pgfpathlineto{\pgfqpoint{1.507006in}{0.603669in}}%
\pgfpathlineto{\pgfqpoint{1.509375in}{0.602468in}}%
\pgfpathlineto{\pgfqpoint{1.511349in}{0.598785in}}%
\pgfpathlineto{\pgfqpoint{1.511743in}{0.599178in}}%
\pgfpathlineto{\pgfqpoint{1.514507in}{0.608560in}}%
\pgfpathlineto{\pgfqpoint{1.514901in}{0.607393in}}%
\pgfpathlineto{\pgfqpoint{1.515296in}{0.609133in}}%
\pgfpathlineto{\pgfqpoint{1.518454in}{0.615389in}}%
\pgfpathlineto{\pgfqpoint{1.518849in}{0.615130in}}%
\pgfpathlineto{\pgfqpoint{1.519244in}{0.616276in}}%
\pgfpathlineto{\pgfqpoint{1.520033in}{0.618180in}}%
\pgfpathlineto{\pgfqpoint{1.520428in}{0.616209in}}%
\pgfpathlineto{\pgfqpoint{1.523191in}{0.606092in}}%
\pgfpathlineto{\pgfqpoint{1.528323in}{0.587464in}}%
\pgfpathlineto{\pgfqpoint{1.529112in}{0.590074in}}%
\pgfpathlineto{\pgfqpoint{1.529902in}{0.589858in}}%
\pgfpathlineto{\pgfqpoint{1.534244in}{0.602663in}}%
\pgfpathlineto{\pgfqpoint{1.534639in}{0.602514in}}%
\pgfpathlineto{\pgfqpoint{1.536613in}{0.598498in}}%
\pgfpathlineto{\pgfqpoint{1.537007in}{0.598806in}}%
\pgfpathlineto{\pgfqpoint{1.537797in}{0.599923in}}%
\pgfpathlineto{\pgfqpoint{1.538192in}{0.598662in}}%
\pgfpathlineto{\pgfqpoint{1.538586in}{0.596807in}}%
\pgfpathlineto{\pgfqpoint{1.539376in}{0.597916in}}%
\pgfpathlineto{\pgfqpoint{1.542139in}{0.603925in}}%
\pgfpathlineto{\pgfqpoint{1.545692in}{0.593137in}}%
\pgfpathlineto{\pgfqpoint{1.550429in}{0.607570in}}%
\pgfpathlineto{\pgfqpoint{1.550823in}{0.607357in}}%
\pgfpathlineto{\pgfqpoint{1.551613in}{0.608094in}}%
\pgfpathlineto{\pgfqpoint{1.553981in}{0.601057in}}%
\pgfpathlineto{\pgfqpoint{1.554771in}{0.596679in}}%
\pgfpathlineto{\pgfqpoint{1.555560in}{0.600087in}}%
\pgfpathlineto{\pgfqpoint{1.557139in}{0.603513in}}%
\pgfpathlineto{\pgfqpoint{1.557534in}{0.603316in}}%
\pgfpathlineto{\pgfqpoint{1.558718in}{0.601974in}}%
\pgfpathlineto{\pgfqpoint{1.559508in}{0.599957in}}%
\pgfpathlineto{\pgfqpoint{1.561087in}{0.593264in}}%
\pgfpathlineto{\pgfqpoint{1.561482in}{0.594984in}}%
\pgfpathlineto{\pgfqpoint{1.562666in}{0.597619in}}%
\pgfpathlineto{\pgfqpoint{1.565034in}{0.605282in}}%
\pgfpathlineto{\pgfqpoint{1.565429in}{0.605151in}}%
\pgfpathlineto{\pgfqpoint{1.567008in}{0.605173in}}%
\pgfpathlineto{\pgfqpoint{1.570956in}{0.610890in}}%
\pgfpathlineto{\pgfqpoint{1.571350in}{0.610239in}}%
\pgfpathlineto{\pgfqpoint{1.572535in}{0.607115in}}%
\pgfpathlineto{\pgfqpoint{1.572929in}{0.608416in}}%
\pgfpathlineto{\pgfqpoint{1.574903in}{0.611653in}}%
\pgfpathlineto{\pgfqpoint{1.575693in}{0.610317in}}%
\pgfpathlineto{\pgfqpoint{1.576087in}{0.608783in}}%
\pgfpathlineto{\pgfqpoint{1.577272in}{0.609743in}}%
\pgfpathlineto{\pgfqpoint{1.580035in}{0.614140in}}%
\pgfpathlineto{\pgfqpoint{1.581219in}{0.617301in}}%
\pgfpathlineto{\pgfqpoint{1.585167in}{0.630578in}}%
\pgfpathlineto{\pgfqpoint{1.585956in}{0.629165in}}%
\pgfpathlineto{\pgfqpoint{1.595035in}{0.605721in}}%
\pgfpathlineto{\pgfqpoint{1.595825in}{0.604903in}}%
\pgfpathlineto{\pgfqpoint{1.597404in}{0.599149in}}%
\pgfpathlineto{\pgfqpoint{1.597799in}{0.600268in}}%
\pgfpathlineto{\pgfqpoint{1.599772in}{0.601445in}}%
\pgfpathlineto{\pgfqpoint{1.604509in}{0.605667in}}%
\pgfpathlineto{\pgfqpoint{1.605299in}{0.604373in}}%
\pgfpathlineto{\pgfqpoint{1.608062in}{0.595660in}}%
\pgfpathlineto{\pgfqpoint{1.608852in}{0.599639in}}%
\pgfpathlineto{\pgfqpoint{1.609246in}{0.599409in}}%
\pgfpathlineto{\pgfqpoint{1.609641in}{0.600100in}}%
\pgfpathlineto{\pgfqpoint{1.610036in}{0.600936in}}%
\pgfpathlineto{\pgfqpoint{1.610431in}{0.599956in}}%
\pgfpathlineto{\pgfqpoint{1.614773in}{0.584169in}}%
\pgfpathlineto{\pgfqpoint{1.615168in}{0.588352in}}%
\pgfpathlineto{\pgfqpoint{1.617536in}{0.595952in}}%
\pgfpathlineto{\pgfqpoint{1.618720in}{0.593047in}}%
\pgfpathlineto{\pgfqpoint{1.620694in}{0.589898in}}%
\pgfpathlineto{\pgfqpoint{1.622668in}{0.597702in}}%
\pgfpathlineto{\pgfqpoint{1.623457in}{0.595754in}}%
\pgfpathlineto{\pgfqpoint{1.624247in}{0.589270in}}%
\pgfpathlineto{\pgfqpoint{1.625431in}{0.591143in}}%
\pgfpathlineto{\pgfqpoint{1.628589in}{0.584738in}}%
\pgfpathlineto{\pgfqpoint{1.628984in}{0.585945in}}%
\pgfpathlineto{\pgfqpoint{1.630563in}{0.589045in}}%
\pgfpathlineto{\pgfqpoint{1.631352in}{0.595163in}}%
\pgfpathlineto{\pgfqpoint{1.631747in}{0.599113in}}%
\pgfpathlineto{\pgfqpoint{1.632536in}{0.593836in}}%
\pgfpathlineto{\pgfqpoint{1.635694in}{0.581694in}}%
\pgfpathlineto{\pgfqpoint{1.636089in}{0.581786in}}%
\pgfpathlineto{\pgfqpoint{1.636879in}{0.577719in}}%
\pgfpathlineto{\pgfqpoint{1.637668in}{0.578736in}}%
\pgfpathlineto{\pgfqpoint{1.640431in}{0.586735in}}%
\pgfpathlineto{\pgfqpoint{1.643195in}{0.593353in}}%
\pgfpathlineto{\pgfqpoint{1.644379in}{0.590792in}}%
\pgfpathlineto{\pgfqpoint{1.644774in}{0.591647in}}%
\pgfpathlineto{\pgfqpoint{1.645168in}{0.593316in}}%
\pgfpathlineto{\pgfqpoint{1.645958in}{0.593112in}}%
\pgfpathlineto{\pgfqpoint{1.646747in}{0.590251in}}%
\pgfpathlineto{\pgfqpoint{1.647537in}{0.592415in}}%
\pgfpathlineto{\pgfqpoint{1.650300in}{0.599209in}}%
\pgfpathlineto{\pgfqpoint{1.652669in}{0.593766in}}%
\pgfpathlineto{\pgfqpoint{1.656221in}{0.576994in}}%
\pgfpathlineto{\pgfqpoint{1.657011in}{0.577475in}}%
\pgfpathlineto{\pgfqpoint{1.657800in}{0.577540in}}%
\pgfpathlineto{\pgfqpoint{1.661748in}{0.594099in}}%
\pgfpathlineto{\pgfqpoint{1.662537in}{0.591870in}}%
\pgfpathlineto{\pgfqpoint{1.663327in}{0.590416in}}%
\pgfpathlineto{\pgfqpoint{1.665301in}{0.583839in}}%
\pgfpathlineto{\pgfqpoint{1.666090in}{0.584781in}}%
\pgfpathlineto{\pgfqpoint{1.666880in}{0.586082in}}%
\pgfpathlineto{\pgfqpoint{1.669248in}{0.592948in}}%
\pgfpathlineto{\pgfqpoint{1.669643in}{0.591656in}}%
\pgfpathlineto{\pgfqpoint{1.674380in}{0.593509in}}%
\pgfpathlineto{\pgfqpoint{1.675169in}{0.588320in}}%
\pgfpathlineto{\pgfqpoint{1.675564in}{0.591176in}}%
\pgfpathlineto{\pgfqpoint{1.676354in}{0.594681in}}%
\pgfpathlineto{\pgfqpoint{1.676748in}{0.593478in}}%
\pgfpathlineto{\pgfqpoint{1.677143in}{0.591199in}}%
\pgfpathlineto{\pgfqpoint{1.677933in}{0.592308in}}%
\pgfpathlineto{\pgfqpoint{1.679512in}{0.598378in}}%
\pgfpathlineto{\pgfqpoint{1.680696in}{0.596472in}}%
\pgfpathlineto{\pgfqpoint{1.681485in}{0.597827in}}%
\pgfpathlineto{\pgfqpoint{1.683459in}{0.599428in}}%
\pgfpathlineto{\pgfqpoint{1.687012in}{0.584372in}}%
\pgfpathlineto{\pgfqpoint{1.687801in}{0.587792in}}%
\pgfpathlineto{\pgfqpoint{1.688986in}{0.594092in}}%
\pgfpathlineto{\pgfqpoint{1.689775in}{0.592699in}}%
\pgfpathlineto{\pgfqpoint{1.690170in}{0.589500in}}%
\pgfpathlineto{\pgfqpoint{1.690959in}{0.595692in}}%
\pgfpathlineto{\pgfqpoint{1.692538in}{0.590873in}}%
\pgfpathlineto{\pgfqpoint{1.692933in}{0.593148in}}%
\pgfpathlineto{\pgfqpoint{1.695302in}{0.600199in}}%
\pgfpathlineto{\pgfqpoint{1.696881in}{0.602345in}}%
\pgfpathlineto{\pgfqpoint{1.697670in}{0.600299in}}%
\pgfpathlineto{\pgfqpoint{1.698854in}{0.602236in}}%
\pgfpathlineto{\pgfqpoint{1.702802in}{0.610837in}}%
\pgfpathlineto{\pgfqpoint{1.704776in}{0.605186in}}%
\pgfpathlineto{\pgfqpoint{1.705960in}{0.606635in}}%
\pgfpathlineto{\pgfqpoint{1.706355in}{0.607678in}}%
\pgfpathlineto{\pgfqpoint{1.707144in}{0.605522in}}%
\pgfpathlineto{\pgfqpoint{1.711091in}{0.592676in}}%
\pgfpathlineto{\pgfqpoint{1.711881in}{0.594286in}}%
\pgfpathlineto{\pgfqpoint{1.713460in}{0.600152in}}%
\pgfpathlineto{\pgfqpoint{1.717013in}{0.622400in}}%
\pgfpathlineto{\pgfqpoint{1.719381in}{0.631802in}}%
\pgfpathlineto{\pgfqpoint{1.720960in}{0.627672in}}%
\pgfpathlineto{\pgfqpoint{1.721355in}{0.626805in}}%
\pgfpathlineto{\pgfqpoint{1.722144in}{0.628658in}}%
\pgfpathlineto{\pgfqpoint{1.728066in}{0.640538in}}%
\pgfpathlineto{\pgfqpoint{1.728460in}{0.640191in}}%
\pgfpathlineto{\pgfqpoint{1.735171in}{0.621132in}}%
\pgfpathlineto{\pgfqpoint{1.735961in}{0.622349in}}%
\pgfpathlineto{\pgfqpoint{1.736355in}{0.621011in}}%
\pgfpathlineto{\pgfqpoint{1.746224in}{0.602554in}}%
\pgfpathlineto{\pgfqpoint{1.747408in}{0.597118in}}%
\pgfpathlineto{\pgfqpoint{1.747803in}{0.599339in}}%
\pgfpathlineto{\pgfqpoint{1.749382in}{0.600413in}}%
\pgfpathlineto{\pgfqpoint{1.751751in}{0.588473in}}%
\pgfpathlineto{\pgfqpoint{1.752540in}{0.591179in}}%
\pgfpathlineto{\pgfqpoint{1.753330in}{0.594303in}}%
\pgfpathlineto{\pgfqpoint{1.754119in}{0.591563in}}%
\pgfpathlineto{\pgfqpoint{1.756882in}{0.584063in}}%
\pgfpathlineto{\pgfqpoint{1.757277in}{0.584691in}}%
\pgfpathlineto{\pgfqpoint{1.758067in}{0.583362in}}%
\pgfpathlineto{\pgfqpoint{1.760040in}{0.583358in}}%
\pgfpathlineto{\pgfqpoint{1.762804in}{0.588605in}}%
\pgfpathlineto{\pgfqpoint{1.763198in}{0.587521in}}%
\pgfpathlineto{\pgfqpoint{1.763593in}{0.585461in}}%
\pgfpathlineto{\pgfqpoint{1.764383in}{0.586962in}}%
\pgfpathlineto{\pgfqpoint{1.767146in}{0.593422in}}%
\pgfpathlineto{\pgfqpoint{1.767541in}{0.593160in}}%
\pgfpathlineto{\pgfqpoint{1.770699in}{0.586118in}}%
\pgfpathlineto{\pgfqpoint{1.771093in}{0.586822in}}%
\pgfpathlineto{\pgfqpoint{1.777015in}{0.604905in}}%
\pgfpathlineto{\pgfqpoint{1.777409in}{0.603104in}}%
\pgfpathlineto{\pgfqpoint{1.779383in}{0.594101in}}%
\pgfpathlineto{\pgfqpoint{1.779778in}{0.594354in}}%
\pgfpathlineto{\pgfqpoint{1.782541in}{0.595316in}}%
\pgfpathlineto{\pgfqpoint{1.784120in}{0.591255in}}%
\pgfpathlineto{\pgfqpoint{1.784515in}{0.591450in}}%
\pgfpathlineto{\pgfqpoint{1.785304in}{0.594173in}}%
\pgfpathlineto{\pgfqpoint{1.785699in}{0.592704in}}%
\pgfpathlineto{\pgfqpoint{1.786489in}{0.587803in}}%
\pgfpathlineto{\pgfqpoint{1.787278in}{0.590125in}}%
\pgfpathlineto{\pgfqpoint{1.790041in}{0.597228in}}%
\pgfpathlineto{\pgfqpoint{1.790831in}{0.595174in}}%
\pgfpathlineto{\pgfqpoint{1.793199in}{0.590267in}}%
\pgfpathlineto{\pgfqpoint{1.794383in}{0.587612in}}%
\pgfpathlineto{\pgfqpoint{1.795568in}{0.588746in}}%
\pgfpathlineto{\pgfqpoint{1.796357in}{0.586550in}}%
\pgfpathlineto{\pgfqpoint{1.796752in}{0.585591in}}%
\pgfpathlineto{\pgfqpoint{1.797147in}{0.586356in}}%
\pgfpathlineto{\pgfqpoint{1.799515in}{0.594346in}}%
\pgfpathlineto{\pgfqpoint{1.801094in}{0.592325in}}%
\pgfpathlineto{\pgfqpoint{1.801489in}{0.592744in}}%
\pgfpathlineto{\pgfqpoint{1.801884in}{0.594071in}}%
\pgfpathlineto{\pgfqpoint{1.802673in}{0.591576in}}%
\pgfpathlineto{\pgfqpoint{1.803857in}{0.589222in}}%
\pgfpathlineto{\pgfqpoint{1.804252in}{0.590408in}}%
\pgfpathlineto{\pgfqpoint{1.806226in}{0.591700in}}%
\pgfpathlineto{\pgfqpoint{1.808989in}{0.591393in}}%
\pgfpathlineto{\pgfqpoint{1.810963in}{0.593095in}}%
\pgfpathlineto{\pgfqpoint{1.812147in}{0.584365in}}%
\pgfpathlineto{\pgfqpoint{1.812937in}{0.579864in}}%
\pgfpathlineto{\pgfqpoint{1.813726in}{0.583011in}}%
\pgfpathlineto{\pgfqpoint{1.814121in}{0.583598in}}%
\pgfpathlineto{\pgfqpoint{1.814516in}{0.581742in}}%
\pgfpathlineto{\pgfqpoint{1.815700in}{0.580871in}}%
\pgfpathlineto{\pgfqpoint{1.816095in}{0.582377in}}%
\pgfpathlineto{\pgfqpoint{1.819647in}{0.594640in}}%
\pgfpathlineto{\pgfqpoint{1.821226in}{0.598448in}}%
\pgfpathlineto{\pgfqpoint{1.821621in}{0.597216in}}%
\pgfpathlineto{\pgfqpoint{1.823595in}{0.589027in}}%
\pgfpathlineto{\pgfqpoint{1.824779in}{0.590435in}}%
\pgfpathlineto{\pgfqpoint{1.828727in}{0.597155in}}%
\pgfpathlineto{\pgfqpoint{1.829121in}{0.595727in}}%
\pgfpathlineto{\pgfqpoint{1.831885in}{0.588861in}}%
\pgfpathlineto{\pgfqpoint{1.832674in}{0.592016in}}%
\pgfpathlineto{\pgfqpoint{1.833858in}{0.591356in}}%
\pgfpathlineto{\pgfqpoint{1.834648in}{0.591671in}}%
\pgfpathlineto{\pgfqpoint{1.836622in}{0.593560in}}%
\pgfpathlineto{\pgfqpoint{1.838201in}{0.591259in}}%
\pgfpathlineto{\pgfqpoint{1.838595in}{0.592197in}}%
\pgfpathlineto{\pgfqpoint{1.839780in}{0.593283in}}%
\pgfpathlineto{\pgfqpoint{1.840174in}{0.592713in}}%
\pgfpathlineto{\pgfqpoint{1.842148in}{0.585926in}}%
\pgfpathlineto{\pgfqpoint{1.842543in}{0.586456in}}%
\pgfpathlineto{\pgfqpoint{1.846490in}{0.591146in}}%
\pgfpathlineto{\pgfqpoint{1.847280in}{0.589944in}}%
\pgfpathlineto{\pgfqpoint{1.848464in}{0.584049in}}%
\pgfpathlineto{\pgfqpoint{1.849648in}{0.578748in}}%
\pgfpathlineto{\pgfqpoint{1.850438in}{0.579854in}}%
\pgfpathlineto{\pgfqpoint{1.851622in}{0.579776in}}%
\pgfpathlineto{\pgfqpoint{1.857938in}{0.605511in}}%
\pgfpathlineto{\pgfqpoint{1.858333in}{0.603614in}}%
\pgfpathlineto{\pgfqpoint{1.861886in}{0.592339in}}%
\pgfpathlineto{\pgfqpoint{1.862280in}{0.592956in}}%
\pgfpathlineto{\pgfqpoint{1.864254in}{0.589193in}}%
\pgfpathlineto{\pgfqpoint{1.866623in}{0.597744in}}%
\pgfpathlineto{\pgfqpoint{1.869781in}{0.591794in}}%
\pgfpathlineto{\pgfqpoint{1.870570in}{0.587490in}}%
\pgfpathlineto{\pgfqpoint{1.871754in}{0.588446in}}%
\pgfpathlineto{\pgfqpoint{1.872544in}{0.590704in}}%
\pgfpathlineto{\pgfqpoint{1.872939in}{0.592174in}}%
\pgfpathlineto{\pgfqpoint{1.873728in}{0.589345in}}%
\pgfpathlineto{\pgfqpoint{1.876491in}{0.588703in}}%
\pgfpathlineto{\pgfqpoint{1.877675in}{0.579332in}}%
\pgfpathlineto{\pgfqpoint{1.878465in}{0.584362in}}%
\pgfpathlineto{\pgfqpoint{1.878860in}{0.584590in}}%
\pgfpathlineto{\pgfqpoint{1.881623in}{0.595117in}}%
\pgfpathlineto{\pgfqpoint{1.883202in}{0.592347in}}%
\pgfpathlineto{\pgfqpoint{1.883597in}{0.593441in}}%
\pgfpathlineto{\pgfqpoint{1.884386in}{0.597026in}}%
\pgfpathlineto{\pgfqpoint{1.885176in}{0.594200in}}%
\pgfpathlineto{\pgfqpoint{1.886360in}{0.590547in}}%
\pgfpathlineto{\pgfqpoint{1.887149in}{0.592405in}}%
\pgfpathlineto{\pgfqpoint{1.889913in}{0.595245in}}%
\pgfpathlineto{\pgfqpoint{1.890702in}{0.596538in}}%
\pgfpathlineto{\pgfqpoint{1.891492in}{0.595478in}}%
\pgfpathlineto{\pgfqpoint{1.894255in}{0.587865in}}%
\pgfpathlineto{\pgfqpoint{1.895834in}{0.587673in}}%
\pgfpathlineto{\pgfqpoint{1.899781in}{0.600087in}}%
\pgfpathlineto{\pgfqpoint{1.902939in}{0.596804in}}%
\pgfpathlineto{\pgfqpoint{1.905308in}{0.590305in}}%
\pgfpathlineto{\pgfqpoint{1.905703in}{0.589899in}}%
\pgfpathlineto{\pgfqpoint{1.906097in}{0.591089in}}%
\pgfpathlineto{\pgfqpoint{1.908071in}{0.592331in}}%
\pgfpathlineto{\pgfqpoint{1.909650in}{0.592776in}}%
\pgfpathlineto{\pgfqpoint{1.910045in}{0.591831in}}%
\pgfpathlineto{\pgfqpoint{1.912019in}{0.587028in}}%
\pgfpathlineto{\pgfqpoint{1.912808in}{0.588151in}}%
\pgfpathlineto{\pgfqpoint{1.914782in}{0.585825in}}%
\pgfpathlineto{\pgfqpoint{1.915966in}{0.582869in}}%
\pgfpathlineto{\pgfqpoint{1.916361in}{0.585340in}}%
\pgfpathlineto{\pgfqpoint{1.919124in}{0.587503in}}%
\pgfpathlineto{\pgfqpoint{1.922677in}{0.584368in}}%
\pgfpathlineto{\pgfqpoint{1.923466in}{0.583254in}}%
\pgfpathlineto{\pgfqpoint{1.923861in}{0.582377in}}%
\pgfpathlineto{\pgfqpoint{1.924651in}{0.583514in}}%
\pgfpathlineto{\pgfqpoint{1.925835in}{0.586429in}}%
\pgfpathlineto{\pgfqpoint{1.928203in}{0.580332in}}%
\pgfpathlineto{\pgfqpoint{1.928598in}{0.579069in}}%
\pgfpathlineto{\pgfqpoint{1.928993in}{0.580341in}}%
\pgfpathlineto{\pgfqpoint{1.931756in}{0.586124in}}%
\pgfpathlineto{\pgfqpoint{1.932151in}{0.585441in}}%
\pgfpathlineto{\pgfqpoint{1.932546in}{0.587600in}}%
\pgfpathlineto{\pgfqpoint{1.933730in}{0.591471in}}%
\pgfpathlineto{\pgfqpoint{1.934125in}{0.590452in}}%
\pgfpathlineto{\pgfqpoint{1.936098in}{0.585307in}}%
\pgfpathlineto{\pgfqpoint{1.936493in}{0.585772in}}%
\pgfpathlineto{\pgfqpoint{1.942809in}{0.581832in}}%
\pgfpathlineto{\pgfqpoint{1.943599in}{0.583926in}}%
\pgfpathlineto{\pgfqpoint{1.944388in}{0.582737in}}%
\pgfpathlineto{\pgfqpoint{1.945572in}{0.580003in}}%
\pgfpathlineto{\pgfqpoint{1.947941in}{0.570697in}}%
\pgfpathlineto{\pgfqpoint{1.951099in}{0.583283in}}%
\pgfpathlineto{\pgfqpoint{1.952283in}{0.587992in}}%
\pgfpathlineto{\pgfqpoint{1.952678in}{0.586456in}}%
\pgfpathlineto{\pgfqpoint{1.955046in}{0.580477in}}%
\pgfpathlineto{\pgfqpoint{1.955441in}{0.578169in}}%
\pgfpathlineto{\pgfqpoint{1.956231in}{0.581605in}}%
\pgfpathlineto{\pgfqpoint{1.956625in}{0.583574in}}%
\pgfpathlineto{\pgfqpoint{1.957415in}{0.582703in}}%
\pgfpathlineto{\pgfqpoint{1.962152in}{0.566782in}}%
\pgfpathlineto{\pgfqpoint{1.962546in}{0.567722in}}%
\pgfpathlineto{\pgfqpoint{1.966494in}{0.579459in}}%
\pgfpathlineto{\pgfqpoint{1.967283in}{0.575377in}}%
\pgfpathlineto{\pgfqpoint{1.969257in}{0.566211in}}%
\pgfpathlineto{\pgfqpoint{1.969652in}{0.567432in}}%
\pgfpathlineto{\pgfqpoint{1.973994in}{0.575489in}}%
\pgfpathlineto{\pgfqpoint{1.974389in}{0.574452in}}%
\pgfpathlineto{\pgfqpoint{1.975178in}{0.571423in}}%
\pgfpathlineto{\pgfqpoint{1.975968in}{0.571996in}}%
\pgfpathlineto{\pgfqpoint{1.976363in}{0.573714in}}%
\pgfpathlineto{\pgfqpoint{1.977547in}{0.572415in}}%
\pgfpathlineto{\pgfqpoint{1.981494in}{0.557826in}}%
\pgfpathlineto{\pgfqpoint{1.978731in}{0.573286in}}%
\pgfpathlineto{\pgfqpoint{1.981889in}{0.558839in}}%
\pgfpathlineto{\pgfqpoint{1.986231in}{0.591611in}}%
\pgfpathlineto{\pgfqpoint{1.987021in}{0.591069in}}%
\pgfpathlineto{\pgfqpoint{1.987416in}{0.590331in}}%
\pgfpathlineto{\pgfqpoint{1.987810in}{0.591573in}}%
\pgfpathlineto{\pgfqpoint{1.990968in}{0.597156in}}%
\pgfpathlineto{\pgfqpoint{1.992547in}{0.589497in}}%
\pgfpathlineto{\pgfqpoint{1.993337in}{0.590089in}}%
\pgfpathlineto{\pgfqpoint{1.994916in}{0.590216in}}%
\pgfpathlineto{\pgfqpoint{1.998074in}{0.604628in}}%
\pgfpathlineto{\pgfqpoint{1.998863in}{0.600169in}}%
\pgfpathlineto{\pgfqpoint{2.000048in}{0.595320in}}%
\pgfpathlineto{\pgfqpoint{2.000837in}{0.597131in}}%
\pgfpathlineto{\pgfqpoint{2.002021in}{0.597201in}}%
\pgfpathlineto{\pgfqpoint{2.002416in}{0.595010in}}%
\pgfpathlineto{\pgfqpoint{2.003206in}{0.595985in}}%
\pgfpathlineto{\pgfqpoint{2.005969in}{0.602906in}}%
\pgfpathlineto{\pgfqpoint{2.011101in}{0.608512in}}%
\pgfpathlineto{\pgfqpoint{2.011495in}{0.609747in}}%
\pgfpathlineto{\pgfqpoint{2.011890in}{0.608452in}}%
\pgfpathlineto{\pgfqpoint{2.018601in}{0.588951in}}%
\pgfpathlineto{\pgfqpoint{2.019390in}{0.590745in}}%
\pgfpathlineto{\pgfqpoint{2.020180in}{0.589023in}}%
\pgfpathlineto{\pgfqpoint{2.023338in}{0.594873in}}%
\pgfpathlineto{\pgfqpoint{2.024522in}{0.593465in}}%
\pgfpathlineto{\pgfqpoint{2.027285in}{0.576823in}}%
\pgfpathlineto{\pgfqpoint{2.028864in}{0.577702in}}%
\pgfpathlineto{\pgfqpoint{2.032812in}{0.585934in}}%
\pgfpathlineto{\pgfqpoint{2.035180in}{0.595896in}}%
\pgfpathlineto{\pgfqpoint{2.037944in}{0.597904in}}%
\pgfpathlineto{\pgfqpoint{2.038338in}{0.594807in}}%
\pgfpathlineto{\pgfqpoint{2.039128in}{0.599000in}}%
\pgfpathlineto{\pgfqpoint{2.039917in}{0.603700in}}%
\pgfpathlineto{\pgfqpoint{2.040707in}{0.602807in}}%
\pgfpathlineto{\pgfqpoint{2.041496in}{0.600651in}}%
\pgfpathlineto{\pgfqpoint{2.041891in}{0.603172in}}%
\pgfpathlineto{\pgfqpoint{2.045049in}{0.615657in}}%
\pgfpathlineto{\pgfqpoint{2.045838in}{0.619000in}}%
\pgfpathlineto{\pgfqpoint{2.046628in}{0.616547in}}%
\pgfpathlineto{\pgfqpoint{2.047023in}{0.616207in}}%
\pgfpathlineto{\pgfqpoint{2.047417in}{0.617542in}}%
\pgfpathlineto{\pgfqpoint{2.048207in}{0.617998in}}%
\pgfpathlineto{\pgfqpoint{2.048602in}{0.616926in}}%
\pgfpathlineto{\pgfqpoint{2.050970in}{0.615192in}}%
\pgfpathlineto{\pgfqpoint{2.052154in}{0.608952in}}%
\pgfpathlineto{\pgfqpoint{2.052944in}{0.610754in}}%
\pgfpathlineto{\pgfqpoint{2.053733in}{0.611105in}}%
\pgfpathlineto{\pgfqpoint{2.058470in}{0.599418in}}%
\pgfpathlineto{\pgfqpoint{2.060444in}{0.595166in}}%
\pgfpathlineto{\pgfqpoint{2.060839in}{0.597311in}}%
\pgfpathlineto{\pgfqpoint{2.061234in}{0.597857in}}%
\pgfpathlineto{\pgfqpoint{2.061628in}{0.596156in}}%
\pgfpathlineto{\pgfqpoint{2.062418in}{0.591563in}}%
\pgfpathlineto{\pgfqpoint{2.063207in}{0.593150in}}%
\pgfpathlineto{\pgfqpoint{2.065181in}{0.598167in}}%
\pgfpathlineto{\pgfqpoint{2.070313in}{0.580098in}}%
\pgfpathlineto{\pgfqpoint{2.070708in}{0.581669in}}%
\pgfpathlineto{\pgfqpoint{2.071102in}{0.583035in}}%
\pgfpathlineto{\pgfqpoint{2.072287in}{0.582176in}}%
\pgfpathlineto{\pgfqpoint{2.072681in}{0.581930in}}%
\pgfpathlineto{\pgfqpoint{2.073076in}{0.583424in}}%
\pgfpathlineto{\pgfqpoint{2.074655in}{0.588540in}}%
\pgfpathlineto{\pgfqpoint{2.075050in}{0.586483in}}%
\pgfpathlineto{\pgfqpoint{2.075839in}{0.586919in}}%
\pgfpathlineto{\pgfqpoint{2.077418in}{0.591994in}}%
\pgfpathlineto{\pgfqpoint{2.077813in}{0.591321in}}%
\pgfpathlineto{\pgfqpoint{2.080182in}{0.582439in}}%
\pgfpathlineto{\pgfqpoint{2.080971in}{0.582789in}}%
\pgfpathlineto{\pgfqpoint{2.082550in}{0.584754in}}%
\pgfpathlineto{\pgfqpoint{2.082945in}{0.583585in}}%
\pgfpathlineto{\pgfqpoint{2.083340in}{0.582170in}}%
\pgfpathlineto{\pgfqpoint{2.084129in}{0.584637in}}%
\pgfpathlineto{\pgfqpoint{2.084524in}{0.584313in}}%
\pgfpathlineto{\pgfqpoint{2.084919in}{0.585249in}}%
\pgfpathlineto{\pgfqpoint{2.085708in}{0.591198in}}%
\pgfpathlineto{\pgfqpoint{2.086498in}{0.589118in}}%
\pgfpathlineto{\pgfqpoint{2.088471in}{0.581189in}}%
\pgfpathlineto{\pgfqpoint{2.089656in}{0.584481in}}%
\pgfpathlineto{\pgfqpoint{2.091235in}{0.589086in}}%
\pgfpathlineto{\pgfqpoint{2.091629in}{0.587544in}}%
\pgfpathlineto{\pgfqpoint{2.093208in}{0.581974in}}%
\pgfpathlineto{\pgfqpoint{2.094393in}{0.583742in}}%
\pgfpathlineto{\pgfqpoint{2.095577in}{0.582092in}}%
\pgfpathlineto{\pgfqpoint{2.097156in}{0.579506in}}%
\pgfpathlineto{\pgfqpoint{2.097945in}{0.578251in}}%
\pgfpathlineto{\pgfqpoint{2.099919in}{0.572576in}}%
\pgfpathlineto{\pgfqpoint{2.100314in}{0.572672in}}%
\pgfpathlineto{\pgfqpoint{2.101893in}{0.578472in}}%
\pgfpathlineto{\pgfqpoint{2.102682in}{0.577619in}}%
\pgfpathlineto{\pgfqpoint{2.103077in}{0.575979in}}%
\pgfpathlineto{\pgfqpoint{2.103867in}{0.576874in}}%
\pgfpathlineto{\pgfqpoint{2.105446in}{0.585501in}}%
\pgfpathlineto{\pgfqpoint{2.107419in}{0.583972in}}%
\pgfpathlineto{\pgfqpoint{2.107814in}{0.582430in}}%
\pgfpathlineto{\pgfqpoint{2.108209in}{0.583930in}}%
\pgfpathlineto{\pgfqpoint{2.111367in}{0.595172in}}%
\pgfpathlineto{\pgfqpoint{2.112156in}{0.593025in}}%
\pgfpathlineto{\pgfqpoint{2.112946in}{0.595236in}}%
\pgfpathlineto{\pgfqpoint{2.114130in}{0.598781in}}%
\pgfpathlineto{\pgfqpoint{2.114525in}{0.597516in}}%
\pgfpathlineto{\pgfqpoint{2.116499in}{0.595743in}}%
\pgfpathlineto{\pgfqpoint{2.116893in}{0.597240in}}%
\pgfpathlineto{\pgfqpoint{2.118078in}{0.595870in}}%
\pgfpathlineto{\pgfqpoint{2.119657in}{0.595727in}}%
\pgfpathlineto{\pgfqpoint{2.121236in}{0.591463in}}%
\pgfpathlineto{\pgfqpoint{2.122025in}{0.592234in}}%
\pgfpathlineto{\pgfqpoint{2.122815in}{0.591904in}}%
\pgfpathlineto{\pgfqpoint{2.124788in}{0.595644in}}%
\pgfpathlineto{\pgfqpoint{2.125973in}{0.596191in}}%
\pgfpathlineto{\pgfqpoint{2.127552in}{0.600131in}}%
\pgfpathlineto{\pgfqpoint{2.129130in}{0.598828in}}%
\pgfpathlineto{\pgfqpoint{2.131499in}{0.597011in}}%
\pgfpathlineto{\pgfqpoint{2.132288in}{0.597760in}}%
\pgfpathlineto{\pgfqpoint{2.132683in}{0.597023in}}%
\pgfpathlineto{\pgfqpoint{2.137815in}{0.578283in}}%
\pgfpathlineto{\pgfqpoint{2.140183in}{0.588298in}}%
\pgfpathlineto{\pgfqpoint{2.140973in}{0.583354in}}%
\pgfpathlineto{\pgfqpoint{2.142947in}{0.574141in}}%
\pgfpathlineto{\pgfqpoint{2.143736in}{0.574771in}}%
\pgfpathlineto{\pgfqpoint{2.146499in}{0.565614in}}%
\pgfpathlineto{\pgfqpoint{2.149657in}{0.567208in}}%
\pgfpathlineto{\pgfqpoint{2.153210in}{0.575121in}}%
\pgfpathlineto{\pgfqpoint{2.154394in}{0.574490in}}%
\pgfpathlineto{\pgfqpoint{2.154789in}{0.572542in}}%
\pgfpathlineto{\pgfqpoint{2.155973in}{0.574224in}}%
\pgfpathlineto{\pgfqpoint{2.156763in}{0.575994in}}%
\pgfpathlineto{\pgfqpoint{2.157552in}{0.574082in}}%
\pgfpathlineto{\pgfqpoint{2.158737in}{0.574275in}}%
\pgfpathlineto{\pgfqpoint{2.161500in}{0.570407in}}%
\pgfpathlineto{\pgfqpoint{2.161895in}{0.570725in}}%
\pgfpathlineto{\pgfqpoint{2.170184in}{0.590681in}}%
\pgfpathlineto{\pgfqpoint{2.170974in}{0.589031in}}%
\pgfpathlineto{\pgfqpoint{2.172948in}{0.586609in}}%
\pgfpathlineto{\pgfqpoint{2.179264in}{0.614647in}}%
\pgfpathlineto{\pgfqpoint{2.180053in}{0.612411in}}%
\pgfpathlineto{\pgfqpoint{2.183606in}{0.596477in}}%
\pgfpathlineto{\pgfqpoint{2.190317in}{0.589849in}}%
\pgfpathlineto{\pgfqpoint{2.191501in}{0.591382in}}%
\pgfpathlineto{\pgfqpoint{2.193869in}{0.598562in}}%
\pgfpathlineto{\pgfqpoint{2.197027in}{0.620749in}}%
\pgfpathlineto{\pgfqpoint{2.197422in}{0.619278in}}%
\pgfpathlineto{\pgfqpoint{2.201764in}{0.610239in}}%
\pgfpathlineto{\pgfqpoint{2.202554in}{0.610708in}}%
\pgfpathlineto{\pgfqpoint{2.202949in}{0.611535in}}%
\pgfpathlineto{\pgfqpoint{2.203738in}{0.610818in}}%
\pgfpathlineto{\pgfqpoint{2.206501in}{0.602882in}}%
\pgfpathlineto{\pgfqpoint{2.206896in}{0.603030in}}%
\pgfpathlineto{\pgfqpoint{2.208870in}{0.604624in}}%
\pgfpathlineto{\pgfqpoint{2.209659in}{0.606876in}}%
\pgfpathlineto{\pgfqpoint{2.210449in}{0.605674in}}%
\pgfpathlineto{\pgfqpoint{2.214791in}{0.595405in}}%
\pgfpathlineto{\pgfqpoint{2.215186in}{0.595974in}}%
\pgfpathlineto{\pgfqpoint{2.216370in}{0.595610in}}%
\pgfpathlineto{\pgfqpoint{2.216765in}{0.594281in}}%
\pgfpathlineto{\pgfqpoint{2.217554in}{0.596374in}}%
\pgfpathlineto{\pgfqpoint{2.219923in}{0.602918in}}%
\pgfpathlineto{\pgfqpoint{2.221896in}{0.602539in}}%
\pgfpathlineto{\pgfqpoint{2.227028in}{0.618050in}}%
\pgfpathlineto{\pgfqpoint{2.227818in}{0.617188in}}%
\pgfpathlineto{\pgfqpoint{2.231765in}{0.602702in}}%
\pgfpathlineto{\pgfqpoint{2.233739in}{0.599256in}}%
\pgfpathlineto{\pgfqpoint{2.234134in}{0.600340in}}%
\pgfpathlineto{\pgfqpoint{2.234528in}{0.598971in}}%
\pgfpathlineto{\pgfqpoint{2.236107in}{0.594606in}}%
\pgfpathlineto{\pgfqpoint{2.236502in}{0.596608in}}%
\pgfpathlineto{\pgfqpoint{2.238081in}{0.603290in}}%
\pgfpathlineto{\pgfqpoint{2.239265in}{0.602775in}}%
\pgfpathlineto{\pgfqpoint{2.241239in}{0.597070in}}%
\pgfpathlineto{\pgfqpoint{2.243213in}{0.589573in}}%
\pgfpathlineto{\pgfqpoint{2.243608in}{0.590148in}}%
\pgfpathlineto{\pgfqpoint{2.246371in}{0.594787in}}%
\pgfpathlineto{\pgfqpoint{2.248739in}{0.602038in}}%
\pgfpathlineto{\pgfqpoint{2.249134in}{0.601802in}}%
\pgfpathlineto{\pgfqpoint{2.253082in}{0.591488in}}%
\pgfpathlineto{\pgfqpoint{2.255055in}{0.596071in}}%
\pgfpathlineto{\pgfqpoint{2.255450in}{0.594833in}}%
\pgfpathlineto{\pgfqpoint{2.257424in}{0.589389in}}%
\pgfpathlineto{\pgfqpoint{2.261371in}{0.594025in}}%
\pgfpathlineto{\pgfqpoint{2.262161in}{0.589886in}}%
\pgfpathlineto{\pgfqpoint{2.262950in}{0.592103in}}%
\pgfpathlineto{\pgfqpoint{2.263345in}{0.593105in}}%
\pgfpathlineto{\pgfqpoint{2.264135in}{0.591130in}}%
\pgfpathlineto{\pgfqpoint{2.264924in}{0.588874in}}%
\pgfpathlineto{\pgfqpoint{2.266898in}{0.585679in}}%
\pgfpathlineto{\pgfqpoint{2.267293in}{0.585845in}}%
\pgfpathlineto{\pgfqpoint{2.270056in}{0.595719in}}%
\pgfpathlineto{\pgfqpoint{2.270845in}{0.594082in}}%
\pgfpathlineto{\pgfqpoint{2.272030in}{0.587847in}}%
\pgfpathlineto{\pgfqpoint{2.272819in}{0.592718in}}%
\pgfpathlineto{\pgfqpoint{2.275188in}{0.596258in}}%
\pgfpathlineto{\pgfqpoint{2.277161in}{0.591406in}}%
\pgfpathlineto{\pgfqpoint{2.277556in}{0.592697in}}%
\pgfpathlineto{\pgfqpoint{2.278346in}{0.595016in}}%
\pgfpathlineto{\pgfqpoint{2.278740in}{0.592799in}}%
\pgfpathlineto{\pgfqpoint{2.281109in}{0.582756in}}%
\pgfpathlineto{\pgfqpoint{2.282688in}{0.584065in}}%
\pgfpathlineto{\pgfqpoint{2.283083in}{0.583283in}}%
\pgfpathlineto{\pgfqpoint{2.283872in}{0.580854in}}%
\pgfpathlineto{\pgfqpoint{2.284267in}{0.584153in}}%
\pgfpathlineto{\pgfqpoint{2.286635in}{0.588845in}}%
\pgfpathlineto{\pgfqpoint{2.291767in}{0.576607in}}%
\pgfpathlineto{\pgfqpoint{2.292557in}{0.577155in}}%
\pgfpathlineto{\pgfqpoint{2.293346in}{0.582204in}}%
\pgfpathlineto{\pgfqpoint{2.294136in}{0.580609in}}%
\pgfpathlineto{\pgfqpoint{2.294925in}{0.577966in}}%
\pgfpathlineto{\pgfqpoint{2.295714in}{0.580269in}}%
\pgfpathlineto{\pgfqpoint{2.298872in}{0.584283in}}%
\pgfpathlineto{\pgfqpoint{2.300846in}{0.578415in}}%
\pgfpathlineto{\pgfqpoint{2.302425in}{0.579724in}}%
\pgfpathlineto{\pgfqpoint{2.303609in}{0.584124in}}%
\pgfpathlineto{\pgfqpoint{2.304399in}{0.581934in}}%
\pgfpathlineto{\pgfqpoint{2.304794in}{0.579918in}}%
\pgfpathlineto{\pgfqpoint{2.305978in}{0.581615in}}%
\pgfpathlineto{\pgfqpoint{2.309925in}{0.571173in}}%
\pgfpathlineto{\pgfqpoint{2.310320in}{0.571521in}}%
\pgfpathlineto{\pgfqpoint{2.311504in}{0.575762in}}%
\pgfpathlineto{\pgfqpoint{2.312294in}{0.574056in}}%
\pgfpathlineto{\pgfqpoint{2.313478in}{0.572153in}}%
\pgfpathlineto{\pgfqpoint{2.313873in}{0.573838in}}%
\pgfpathlineto{\pgfqpoint{2.315847in}{0.573709in}}%
\pgfpathlineto{\pgfqpoint{2.316636in}{0.571857in}}%
\pgfpathlineto{\pgfqpoint{2.317426in}{0.573787in}}%
\pgfpathlineto{\pgfqpoint{2.318610in}{0.578289in}}%
\pgfpathlineto{\pgfqpoint{2.319005in}{0.577323in}}%
\pgfpathlineto{\pgfqpoint{2.320189in}{0.574658in}}%
\pgfpathlineto{\pgfqpoint{2.320584in}{0.575728in}}%
\pgfpathlineto{\pgfqpoint{2.322952in}{0.593054in}}%
\pgfpathlineto{\pgfqpoint{2.323347in}{0.592714in}}%
\pgfpathlineto{\pgfqpoint{2.323742in}{0.591123in}}%
\pgfpathlineto{\pgfqpoint{2.324531in}{0.593248in}}%
\pgfpathlineto{\pgfqpoint{2.324926in}{0.592267in}}%
\pgfpathlineto{\pgfqpoint{2.325715in}{0.591078in}}%
\pgfpathlineto{\pgfqpoint{2.328479in}{0.587394in}}%
\pgfpathlineto{\pgfqpoint{2.330058in}{0.586310in}}%
\pgfpathlineto{\pgfqpoint{2.330452in}{0.587509in}}%
\pgfpathlineto{\pgfqpoint{2.330847in}{0.588470in}}%
\pgfpathlineto{\pgfqpoint{2.331242in}{0.585866in}}%
\pgfpathlineto{\pgfqpoint{2.333216in}{0.578387in}}%
\pgfpathlineto{\pgfqpoint{2.333610in}{0.579189in}}%
\pgfpathlineto{\pgfqpoint{2.334400in}{0.576638in}}%
\pgfpathlineto{\pgfqpoint{2.337558in}{0.567732in}}%
\pgfpathlineto{\pgfqpoint{2.339532in}{0.570488in}}%
\pgfpathlineto{\pgfqpoint{2.340716in}{0.575629in}}%
\pgfpathlineto{\pgfqpoint{2.341505in}{0.573853in}}%
\pgfpathlineto{\pgfqpoint{2.341900in}{0.571776in}}%
\pgfpathlineto{\pgfqpoint{2.342690in}{0.573116in}}%
\pgfpathlineto{\pgfqpoint{2.343479in}{0.575822in}}%
\pgfpathlineto{\pgfqpoint{2.343874in}{0.573983in}}%
\pgfpathlineto{\pgfqpoint{2.346637in}{0.566305in}}%
\pgfpathlineto{\pgfqpoint{2.347032in}{0.567602in}}%
\pgfpathlineto{\pgfqpoint{2.349006in}{0.572813in}}%
\pgfpathlineto{\pgfqpoint{2.352164in}{0.589399in}}%
\pgfpathlineto{\pgfqpoint{2.352953in}{0.589696in}}%
\pgfpathlineto{\pgfqpoint{2.354927in}{0.586424in}}%
\pgfpathlineto{\pgfqpoint{2.355716in}{0.586294in}}%
\pgfpathlineto{\pgfqpoint{2.358085in}{0.596640in}}%
\pgfpathlineto{\pgfqpoint{2.360453in}{0.588864in}}%
\pgfpathlineto{\pgfqpoint{2.362427in}{0.577867in}}%
\pgfpathlineto{\pgfqpoint{2.362822in}{0.579679in}}%
\pgfpathlineto{\pgfqpoint{2.365190in}{0.585687in}}%
\pgfpathlineto{\pgfqpoint{2.365585in}{0.585315in}}%
\pgfpathlineto{\pgfqpoint{2.367954in}{0.579090in}}%
\pgfpathlineto{\pgfqpoint{2.370322in}{0.572428in}}%
\pgfpathlineto{\pgfqpoint{2.370717in}{0.574347in}}%
\pgfpathlineto{\pgfqpoint{2.371901in}{0.582098in}}%
\pgfpathlineto{\pgfqpoint{2.372691in}{0.581189in}}%
\pgfpathlineto{\pgfqpoint{2.373480in}{0.577928in}}%
\pgfpathlineto{\pgfqpoint{2.373875in}{0.574857in}}%
\pgfpathlineto{\pgfqpoint{2.374664in}{0.576911in}}%
\pgfpathlineto{\pgfqpoint{2.375059in}{0.577857in}}%
\pgfpathlineto{\pgfqpoint{2.375454in}{0.576981in}}%
\pgfpathlineto{\pgfqpoint{2.377822in}{0.570384in}}%
\pgfpathlineto{\pgfqpoint{2.378217in}{0.571613in}}%
\pgfpathlineto{\pgfqpoint{2.380585in}{0.574732in}}%
\pgfpathlineto{\pgfqpoint{2.381375in}{0.573494in}}%
\pgfpathlineto{\pgfqpoint{2.382164in}{0.569895in}}%
\pgfpathlineto{\pgfqpoint{2.382954in}{0.572714in}}%
\pgfpathlineto{\pgfqpoint{2.383349in}{0.573514in}}%
\pgfpathlineto{\pgfqpoint{2.383743in}{0.572569in}}%
\pgfpathlineto{\pgfqpoint{2.384533in}{0.571159in}}%
\pgfpathlineto{\pgfqpoint{2.384928in}{0.573068in}}%
\pgfpathlineto{\pgfqpoint{2.385717in}{0.574894in}}%
\pgfpathlineto{\pgfqpoint{2.386112in}{0.574608in}}%
\pgfpathlineto{\pgfqpoint{2.386507in}{0.572860in}}%
\pgfpathlineto{\pgfqpoint{2.387296in}{0.575926in}}%
\pgfpathlineto{\pgfqpoint{2.391244in}{0.585541in}}%
\pgfpathlineto{\pgfqpoint{2.391638in}{0.583991in}}%
\pgfpathlineto{\pgfqpoint{2.392428in}{0.581927in}}%
\pgfpathlineto{\pgfqpoint{2.393217in}{0.583580in}}%
\pgfpathlineto{\pgfqpoint{2.395586in}{0.597570in}}%
\pgfpathlineto{\pgfqpoint{2.396375in}{0.594605in}}%
\pgfpathlineto{\pgfqpoint{2.401112in}{0.573835in}}%
\pgfpathlineto{\pgfqpoint{2.401902in}{0.575313in}}%
\pgfpathlineto{\pgfqpoint{2.402297in}{0.574440in}}%
\pgfpathlineto{\pgfqpoint{2.404665in}{0.586176in}}%
\pgfpathlineto{\pgfqpoint{2.405060in}{0.585255in}}%
\pgfpathlineto{\pgfqpoint{2.406639in}{0.581568in}}%
\pgfpathlineto{\pgfqpoint{2.407428in}{0.581925in}}%
\pgfpathlineto{\pgfqpoint{2.407823in}{0.582413in}}%
\pgfpathlineto{\pgfqpoint{2.408218in}{0.581803in}}%
\pgfpathlineto{\pgfqpoint{2.409007in}{0.579537in}}%
\pgfpathlineto{\pgfqpoint{2.409797in}{0.580815in}}%
\pgfpathlineto{\pgfqpoint{2.410981in}{0.583726in}}%
\pgfpathlineto{\pgfqpoint{2.411376in}{0.582157in}}%
\pgfpathlineto{\pgfqpoint{2.412955in}{0.577173in}}%
\pgfpathlineto{\pgfqpoint{2.413744in}{0.577896in}}%
\pgfpathlineto{\pgfqpoint{2.414139in}{0.578266in}}%
\pgfpathlineto{\pgfqpoint{2.414534in}{0.577521in}}%
\pgfpathlineto{\pgfqpoint{2.416113in}{0.575394in}}%
\pgfpathlineto{\pgfqpoint{2.416508in}{0.575587in}}%
\pgfpathlineto{\pgfqpoint{2.417297in}{0.575120in}}%
\pgfpathlineto{\pgfqpoint{2.418087in}{0.573443in}}%
\pgfpathlineto{\pgfqpoint{2.418481in}{0.574961in}}%
\pgfpathlineto{\pgfqpoint{2.418876in}{0.575357in}}%
\pgfpathlineto{\pgfqpoint{2.419271in}{0.573947in}}%
\pgfpathlineto{\pgfqpoint{2.422429in}{0.566132in}}%
\pgfpathlineto{\pgfqpoint{2.423218in}{0.565236in}}%
\pgfpathlineto{\pgfqpoint{2.423613in}{0.566403in}}%
\pgfpathlineto{\pgfqpoint{2.424797in}{0.573412in}}%
\pgfpathlineto{\pgfqpoint{2.425192in}{0.571630in}}%
\pgfpathlineto{\pgfqpoint{2.425587in}{0.566376in}}%
\pgfpathlineto{\pgfqpoint{2.426771in}{0.569701in}}%
\pgfpathlineto{\pgfqpoint{2.427955in}{0.573340in}}%
\pgfpathlineto{\pgfqpoint{2.428350in}{0.573175in}}%
\pgfpathlineto{\pgfqpoint{2.430719in}{0.569022in}}%
\pgfpathlineto{\pgfqpoint{2.431113in}{0.570347in}}%
\pgfpathlineto{\pgfqpoint{2.433877in}{0.575017in}}%
\pgfpathlineto{\pgfqpoint{2.436245in}{0.577570in}}%
\pgfpathlineto{\pgfqpoint{2.437035in}{0.576035in}}%
\pgfpathlineto{\pgfqpoint{2.437429in}{0.577879in}}%
\pgfpathlineto{\pgfqpoint{2.440982in}{0.592930in}}%
\pgfpathlineto{\pgfqpoint{2.441377in}{0.592341in}}%
\pgfpathlineto{\pgfqpoint{2.449272in}{0.569168in}}%
\pgfpathlineto{\pgfqpoint{2.450061in}{0.570634in}}%
\pgfpathlineto{\pgfqpoint{2.450851in}{0.570011in}}%
\pgfpathlineto{\pgfqpoint{2.451246in}{0.568778in}}%
\pgfpathlineto{\pgfqpoint{2.451640in}{0.569688in}}%
\pgfpathlineto{\pgfqpoint{2.452430in}{0.573057in}}%
\pgfpathlineto{\pgfqpoint{2.453219in}{0.570443in}}%
\pgfpathlineto{\pgfqpoint{2.453614in}{0.568692in}}%
\pgfpathlineto{\pgfqpoint{2.454798in}{0.570393in}}%
\pgfpathlineto{\pgfqpoint{2.457167in}{0.567672in}}%
\pgfpathlineto{\pgfqpoint{2.457562in}{0.568882in}}%
\pgfpathlineto{\pgfqpoint{2.459535in}{0.570244in}}%
\pgfpathlineto{\pgfqpoint{2.461114in}{0.568260in}}%
\pgfpathlineto{\pgfqpoint{2.462693in}{0.572239in}}%
\pgfpathlineto{\pgfqpoint{2.463088in}{0.572085in}}%
\pgfpathlineto{\pgfqpoint{2.463877in}{0.569454in}}%
\pgfpathlineto{\pgfqpoint{2.464667in}{0.565224in}}%
\pgfpathlineto{\pgfqpoint{2.465851in}{0.565434in}}%
\pgfpathlineto{\pgfqpoint{2.469799in}{0.575887in}}%
\pgfpathlineto{\pgfqpoint{2.470588in}{0.575352in}}%
\pgfpathlineto{\pgfqpoint{2.470983in}{0.574347in}}%
\pgfpathlineto{\pgfqpoint{2.471772in}{0.575037in}}%
\pgfpathlineto{\pgfqpoint{2.473351in}{0.578020in}}%
\pgfpathlineto{\pgfqpoint{2.473746in}{0.577060in}}%
\pgfpathlineto{\pgfqpoint{2.474930in}{0.575042in}}%
\pgfpathlineto{\pgfqpoint{2.475325in}{0.576444in}}%
\pgfpathlineto{\pgfqpoint{2.475720in}{0.576638in}}%
\pgfpathlineto{\pgfqpoint{2.477299in}{0.580777in}}%
\pgfpathlineto{\pgfqpoint{2.477694in}{0.579485in}}%
\pgfpathlineto{\pgfqpoint{2.478878in}{0.574478in}}%
\pgfpathlineto{\pgfqpoint{2.479667in}{0.577298in}}%
\pgfpathlineto{\pgfqpoint{2.480062in}{0.577782in}}%
\pgfpathlineto{\pgfqpoint{2.480852in}{0.576687in}}%
\pgfpathlineto{\pgfqpoint{2.482431in}{0.572254in}}%
\pgfpathlineto{\pgfqpoint{2.483615in}{0.573222in}}%
\pgfpathlineto{\pgfqpoint{2.484799in}{0.577596in}}%
\pgfpathlineto{\pgfqpoint{2.485194in}{0.575219in}}%
\pgfpathlineto{\pgfqpoint{2.485983in}{0.573404in}}%
\pgfpathlineto{\pgfqpoint{2.486378in}{0.574375in}}%
\pgfpathlineto{\pgfqpoint{2.489536in}{0.578986in}}%
\pgfpathlineto{\pgfqpoint{2.490326in}{0.578781in}}%
\pgfpathlineto{\pgfqpoint{2.490720in}{0.579830in}}%
\pgfpathlineto{\pgfqpoint{2.493089in}{0.580422in}}%
\pgfpathlineto{\pgfqpoint{2.493878in}{0.581728in}}%
\pgfpathlineto{\pgfqpoint{2.497036in}{0.569758in}}%
\pgfpathlineto{\pgfqpoint{2.498615in}{0.572457in}}%
\pgfpathlineto{\pgfqpoint{2.499010in}{0.572056in}}%
\pgfpathlineto{\pgfqpoint{2.501379in}{0.566454in}}%
\pgfpathlineto{\pgfqpoint{2.501773in}{0.569115in}}%
\pgfpathlineto{\pgfqpoint{2.502168in}{0.569243in}}%
\pgfpathlineto{\pgfqpoint{2.504537in}{0.577501in}}%
\pgfpathlineto{\pgfqpoint{2.504931in}{0.576514in}}%
\pgfpathlineto{\pgfqpoint{2.505721in}{0.574727in}}%
\pgfpathlineto{\pgfqpoint{2.506116in}{0.574315in}}%
\pgfpathlineto{\pgfqpoint{2.506905in}{0.575473in}}%
\pgfpathlineto{\pgfqpoint{2.508089in}{0.577398in}}%
\pgfpathlineto{\pgfqpoint{2.510063in}{0.579402in}}%
\pgfpathlineto{\pgfqpoint{2.513221in}{0.574197in}}%
\pgfpathlineto{\pgfqpoint{2.514011in}{0.575228in}}%
\pgfpathlineto{\pgfqpoint{2.515984in}{0.579470in}}%
\pgfpathlineto{\pgfqpoint{2.516379in}{0.579860in}}%
\pgfpathlineto{\pgfqpoint{2.516774in}{0.578229in}}%
\pgfpathlineto{\pgfqpoint{2.517169in}{0.576175in}}%
\pgfpathlineto{\pgfqpoint{2.517958in}{0.579530in}}%
\pgfpathlineto{\pgfqpoint{2.519932in}{0.583790in}}%
\pgfpathlineto{\pgfqpoint{2.520327in}{0.583339in}}%
\pgfpathlineto{\pgfqpoint{2.522300in}{0.584931in}}%
\pgfpathlineto{\pgfqpoint{2.523090in}{0.588896in}}%
\pgfpathlineto{\pgfqpoint{2.524274in}{0.587949in}}%
\pgfpathlineto{\pgfqpoint{2.524669in}{0.588070in}}%
\pgfpathlineto{\pgfqpoint{2.525064in}{0.586769in}}%
\pgfpathlineto{\pgfqpoint{2.526248in}{0.583466in}}%
\pgfpathlineto{\pgfqpoint{2.527037in}{0.584418in}}%
\pgfpathlineto{\pgfqpoint{2.528222in}{0.588089in}}%
\pgfpathlineto{\pgfqpoint{2.529011in}{0.586340in}}%
\pgfpathlineto{\pgfqpoint{2.531774in}{0.580969in}}%
\pgfpathlineto{\pgfqpoint{2.532959in}{0.577537in}}%
\pgfpathlineto{\pgfqpoint{2.533353in}{0.579341in}}%
\pgfpathlineto{\pgfqpoint{2.533748in}{0.579923in}}%
\pgfpathlineto{\pgfqpoint{2.534143in}{0.578840in}}%
\pgfpathlineto{\pgfqpoint{2.536906in}{0.571870in}}%
\pgfpathlineto{\pgfqpoint{2.537301in}{0.572435in}}%
\pgfpathlineto{\pgfqpoint{2.540064in}{0.581038in}}%
\pgfpathlineto{\pgfqpoint{2.545196in}{0.587267in}}%
\pgfpathlineto{\pgfqpoint{2.545590in}{0.586497in}}%
\pgfpathlineto{\pgfqpoint{2.548354in}{0.583091in}}%
\pgfpathlineto{\pgfqpoint{2.549538in}{0.584695in}}%
\pgfpathlineto{\pgfqpoint{2.551906in}{0.589089in}}%
\pgfpathlineto{\pgfqpoint{2.557433in}{0.607128in}}%
\pgfpathlineto{\pgfqpoint{2.558222in}{0.603798in}}%
\pgfpathlineto{\pgfqpoint{2.561380in}{0.587773in}}%
\pgfpathlineto{\pgfqpoint{2.562565in}{0.589741in}}%
\pgfpathlineto{\pgfqpoint{2.564933in}{0.595879in}}%
\pgfpathlineto{\pgfqpoint{2.565328in}{0.595602in}}%
\pgfpathlineto{\pgfqpoint{2.569670in}{0.597879in}}%
\pgfpathlineto{\pgfqpoint{2.576776in}{0.584283in}}%
\pgfpathlineto{\pgfqpoint{2.577170in}{0.587093in}}%
\pgfpathlineto{\pgfqpoint{2.578355in}{0.591274in}}%
\pgfpathlineto{\pgfqpoint{2.579144in}{0.594901in}}%
\pgfpathlineto{\pgfqpoint{2.579934in}{0.593383in}}%
\pgfpathlineto{\pgfqpoint{2.580723in}{0.591974in}}%
\pgfpathlineto{\pgfqpoint{2.581907in}{0.589564in}}%
\pgfpathlineto{\pgfqpoint{2.583881in}{0.594446in}}%
\pgfpathlineto{\pgfqpoint{2.585460in}{0.595267in}}%
\pgfpathlineto{\pgfqpoint{2.589013in}{0.578074in}}%
\pgfpathlineto{\pgfqpoint{2.589802in}{0.580383in}}%
\pgfpathlineto{\pgfqpoint{2.591776in}{0.586815in}}%
\pgfpathlineto{\pgfqpoint{2.592171in}{0.586329in}}%
\pgfpathlineto{\pgfqpoint{2.594539in}{0.580965in}}%
\pgfpathlineto{\pgfqpoint{2.595329in}{0.582011in}}%
\pgfpathlineto{\pgfqpoint{2.605198in}{0.609983in}}%
\pgfpathlineto{\pgfqpoint{2.605592in}{0.606714in}}%
\pgfpathlineto{\pgfqpoint{2.607566in}{0.598675in}}%
\pgfpathlineto{\pgfqpoint{2.607961in}{0.598140in}}%
\pgfpathlineto{\pgfqpoint{2.608356in}{0.598953in}}%
\pgfpathlineto{\pgfqpoint{2.609935in}{0.605593in}}%
\pgfpathlineto{\pgfqpoint{2.610724in}{0.602859in}}%
\pgfpathlineto{\pgfqpoint{2.612303in}{0.593141in}}%
\pgfpathlineto{\pgfqpoint{2.613487in}{0.588835in}}%
\pgfpathlineto{\pgfqpoint{2.613882in}{0.590324in}}%
\pgfpathlineto{\pgfqpoint{2.614672in}{0.588843in}}%
\pgfpathlineto{\pgfqpoint{2.615066in}{0.590488in}}%
\pgfpathlineto{\pgfqpoint{2.617040in}{0.596301in}}%
\pgfpathlineto{\pgfqpoint{2.617435in}{0.593941in}}%
\pgfpathlineto{\pgfqpoint{2.618619in}{0.597537in}}%
\pgfpathlineto{\pgfqpoint{2.619409in}{0.599440in}}%
\pgfpathlineto{\pgfqpoint{2.619803in}{0.597566in}}%
\pgfpathlineto{\pgfqpoint{2.620593in}{0.595327in}}%
\pgfpathlineto{\pgfqpoint{2.620988in}{0.598054in}}%
\pgfpathlineto{\pgfqpoint{2.621382in}{0.597873in}}%
\pgfpathlineto{\pgfqpoint{2.621777in}{0.599020in}}%
\pgfpathlineto{\pgfqpoint{2.622961in}{0.601712in}}%
\pgfpathlineto{\pgfqpoint{2.623356in}{0.601239in}}%
\pgfpathlineto{\pgfqpoint{2.624146in}{0.599138in}}%
\pgfpathlineto{\pgfqpoint{2.624935in}{0.600399in}}%
\pgfpathlineto{\pgfqpoint{2.625725in}{0.602585in}}%
\pgfpathlineto{\pgfqpoint{2.626514in}{0.601111in}}%
\pgfpathlineto{\pgfqpoint{2.629277in}{0.595498in}}%
\pgfpathlineto{\pgfqpoint{2.632040in}{0.585233in}}%
\pgfpathlineto{\pgfqpoint{2.638751in}{0.597323in}}%
\pgfpathlineto{\pgfqpoint{2.641120in}{0.604269in}}%
\pgfpathlineto{\pgfqpoint{2.643488in}{0.597411in}}%
\pgfpathlineto{\pgfqpoint{2.645462in}{0.588448in}}%
\pgfpathlineto{\pgfqpoint{2.645857in}{0.589648in}}%
\pgfpathlineto{\pgfqpoint{2.651383in}{0.609047in}}%
\pgfpathlineto{\pgfqpoint{2.652173in}{0.606765in}}%
\pgfpathlineto{\pgfqpoint{2.658094in}{0.595835in}}%
\pgfpathlineto{\pgfqpoint{2.660068in}{0.599415in}}%
\pgfpathlineto{\pgfqpoint{2.660462in}{0.598643in}}%
\pgfpathlineto{\pgfqpoint{2.665199in}{0.595714in}}%
\pgfpathlineto{\pgfqpoint{2.666778in}{0.593502in}}%
\pgfpathlineto{\pgfqpoint{2.669936in}{0.602845in}}%
\pgfpathlineto{\pgfqpoint{2.670331in}{0.602679in}}%
\pgfpathlineto{\pgfqpoint{2.670726in}{0.604568in}}%
\pgfpathlineto{\pgfqpoint{2.671515in}{0.601699in}}%
\pgfpathlineto{\pgfqpoint{2.672305in}{0.598213in}}%
\pgfpathlineto{\pgfqpoint{2.673094in}{0.601258in}}%
\pgfpathlineto{\pgfqpoint{2.674279in}{0.607715in}}%
\pgfpathlineto{\pgfqpoint{2.675068in}{0.605893in}}%
\pgfpathlineto{\pgfqpoint{2.678621in}{0.594555in}}%
\pgfpathlineto{\pgfqpoint{2.679410in}{0.591680in}}%
\pgfpathlineto{\pgfqpoint{2.680595in}{0.592189in}}%
\pgfpathlineto{\pgfqpoint{2.681384in}{0.594588in}}%
\pgfpathlineto{\pgfqpoint{2.683753in}{0.600725in}}%
\pgfpathlineto{\pgfqpoint{2.684147in}{0.600413in}}%
\pgfpathlineto{\pgfqpoint{2.684937in}{0.601583in}}%
\pgfpathlineto{\pgfqpoint{2.685332in}{0.599339in}}%
\pgfpathlineto{\pgfqpoint{2.686911in}{0.593204in}}%
\pgfpathlineto{\pgfqpoint{2.687305in}{0.593394in}}%
\pgfpathlineto{\pgfqpoint{2.687700in}{0.590737in}}%
\pgfpathlineto{\pgfqpoint{2.688884in}{0.593174in}}%
\pgfpathlineto{\pgfqpoint{2.690069in}{0.596349in}}%
\pgfpathlineto{\pgfqpoint{2.691253in}{0.594660in}}%
\pgfpathlineto{\pgfqpoint{2.694411in}{0.588778in}}%
\pgfpathlineto{\pgfqpoint{2.694806in}{0.589014in}}%
\pgfpathlineto{\pgfqpoint{2.695595in}{0.585867in}}%
\pgfpathlineto{\pgfqpoint{2.696385in}{0.588184in}}%
\pgfpathlineto{\pgfqpoint{2.697964in}{0.592804in}}%
\pgfpathlineto{\pgfqpoint{2.698358in}{0.590744in}}%
\pgfpathlineto{\pgfqpoint{2.700727in}{0.583317in}}%
\pgfpathlineto{\pgfqpoint{2.701516in}{0.583821in}}%
\pgfpathlineto{\pgfqpoint{2.701911in}{0.583360in}}%
\pgfpathlineto{\pgfqpoint{2.702306in}{0.584996in}}%
\pgfpathlineto{\pgfqpoint{2.703885in}{0.589200in}}%
\pgfpathlineto{\pgfqpoint{2.704280in}{0.586925in}}%
\pgfpathlineto{\pgfqpoint{2.705069in}{0.586147in}}%
\pgfpathlineto{\pgfqpoint{2.705464in}{0.587538in}}%
\pgfpathlineto{\pgfqpoint{2.705859in}{0.590068in}}%
\pgfpathlineto{\pgfqpoint{2.706648in}{0.586943in}}%
\pgfpathlineto{\pgfqpoint{2.707438in}{0.583700in}}%
\pgfpathlineto{\pgfqpoint{2.708227in}{0.585604in}}%
\pgfpathlineto{\pgfqpoint{2.708622in}{0.586233in}}%
\pgfpathlineto{\pgfqpoint{2.709017in}{0.585278in}}%
\pgfpathlineto{\pgfqpoint{2.710990in}{0.578443in}}%
\pgfpathlineto{\pgfqpoint{2.711385in}{0.578918in}}%
\pgfpathlineto{\pgfqpoint{2.713359in}{0.585412in}}%
\pgfpathlineto{\pgfqpoint{2.714938in}{0.593758in}}%
\pgfpathlineto{\pgfqpoint{2.716122in}{0.591026in}}%
\pgfpathlineto{\pgfqpoint{2.717701in}{0.585895in}}%
\pgfpathlineto{\pgfqpoint{2.718490in}{0.586600in}}%
\pgfpathlineto{\pgfqpoint{2.720464in}{0.588526in}}%
\pgfpathlineto{\pgfqpoint{2.722833in}{0.585830in}}%
\pgfpathlineto{\pgfqpoint{2.724017in}{0.587967in}}%
\pgfpathlineto{\pgfqpoint{2.724806in}{0.586619in}}%
\pgfpathlineto{\pgfqpoint{2.725596in}{0.583728in}}%
\pgfpathlineto{\pgfqpoint{2.727570in}{0.578225in}}%
\pgfpathlineto{\pgfqpoint{2.730728in}{0.580482in}}%
\pgfpathlineto{\pgfqpoint{2.732701in}{0.578466in}}%
\pgfpathlineto{\pgfqpoint{2.733096in}{0.576538in}}%
\pgfpathlineto{\pgfqpoint{2.734280in}{0.576838in}}%
\pgfpathlineto{\pgfqpoint{2.739017in}{0.585302in}}%
\pgfpathlineto{\pgfqpoint{2.739412in}{0.585241in}}%
\pgfpathlineto{\pgfqpoint{2.741781in}{0.572159in}}%
\pgfpathlineto{\pgfqpoint{2.742570in}{0.574008in}}%
\pgfpathlineto{\pgfqpoint{2.744939in}{0.580521in}}%
\pgfpathlineto{\pgfqpoint{2.745333in}{0.578520in}}%
\pgfpathlineto{\pgfqpoint{2.752439in}{0.564914in}}%
\pgfpathlineto{\pgfqpoint{2.752834in}{0.565952in}}%
\pgfpathlineto{\pgfqpoint{2.756386in}{0.574049in}}%
\pgfpathlineto{\pgfqpoint{2.761123in}{0.590797in}}%
\pgfpathlineto{\pgfqpoint{2.761518in}{0.590207in}}%
\pgfpathlineto{\pgfqpoint{2.761913in}{0.590857in}}%
\pgfpathlineto{\pgfqpoint{2.762702in}{0.590224in}}%
\pgfpathlineto{\pgfqpoint{2.763492in}{0.587790in}}%
\pgfpathlineto{\pgfqpoint{2.764676in}{0.588164in}}%
\pgfpathlineto{\pgfqpoint{2.766255in}{0.587495in}}%
\pgfpathlineto{\pgfqpoint{2.767834in}{0.592973in}}%
\pgfpathlineto{\pgfqpoint{2.768624in}{0.596416in}}%
\pgfpathlineto{\pgfqpoint{2.769413in}{0.595082in}}%
\pgfpathlineto{\pgfqpoint{2.772176in}{0.591668in}}%
\pgfpathlineto{\pgfqpoint{2.773755in}{0.589542in}}%
\pgfpathlineto{\pgfqpoint{2.774150in}{0.590377in}}%
\pgfpathlineto{\pgfqpoint{2.777308in}{0.598233in}}%
\pgfpathlineto{\pgfqpoint{2.778492in}{0.597130in}}%
\pgfpathlineto{\pgfqpoint{2.779282in}{0.596678in}}%
\pgfpathlineto{\pgfqpoint{2.779677in}{0.597819in}}%
\pgfpathlineto{\pgfqpoint{2.781256in}{0.606668in}}%
\pgfpathlineto{\pgfqpoint{2.782045in}{0.602702in}}%
\pgfpathlineto{\pgfqpoint{2.782835in}{0.597055in}}%
\pgfpathlineto{\pgfqpoint{2.783624in}{0.599828in}}%
\pgfpathlineto{\pgfqpoint{2.787177in}{0.606529in}}%
\pgfpathlineto{\pgfqpoint{2.789545in}{0.606588in}}%
\pgfpathlineto{\pgfqpoint{2.789940in}{0.605301in}}%
\pgfpathlineto{\pgfqpoint{2.790730in}{0.606494in}}%
\pgfpathlineto{\pgfqpoint{2.792703in}{0.606436in}}%
\pgfpathlineto{\pgfqpoint{2.794282in}{0.608381in}}%
\pgfpathlineto{\pgfqpoint{2.794677in}{0.610206in}}%
\pgfpathlineto{\pgfqpoint{2.795466in}{0.609070in}}%
\pgfpathlineto{\pgfqpoint{2.799414in}{0.596130in}}%
\pgfpathlineto{\pgfqpoint{2.801388in}{0.588026in}}%
\pgfpathlineto{\pgfqpoint{2.801782in}{0.588224in}}%
\pgfpathlineto{\pgfqpoint{2.802177in}{0.588827in}}%
\pgfpathlineto{\pgfqpoint{2.803361in}{0.586199in}}%
\pgfpathlineto{\pgfqpoint{2.803756in}{0.586778in}}%
\pgfpathlineto{\pgfqpoint{2.807704in}{0.594186in}}%
\pgfpathlineto{\pgfqpoint{2.808098in}{0.593744in}}%
\pgfpathlineto{\pgfqpoint{2.810862in}{0.602690in}}%
\pgfpathlineto{\pgfqpoint{2.812835in}{0.606344in}}%
\pgfpathlineto{\pgfqpoint{2.813230in}{0.605565in}}%
\pgfpathlineto{\pgfqpoint{2.815204in}{0.600196in}}%
\pgfpathlineto{\pgfqpoint{2.815599in}{0.602316in}}%
\pgfpathlineto{\pgfqpoint{2.817572in}{0.607245in}}%
\pgfpathlineto{\pgfqpoint{2.819546in}{0.610737in}}%
\pgfpathlineto{\pgfqpoint{2.820730in}{0.611110in}}%
\pgfpathlineto{\pgfqpoint{2.823099in}{0.602661in}}%
\pgfpathlineto{\pgfqpoint{2.824283in}{0.597639in}}%
\pgfpathlineto{\pgfqpoint{2.825862in}{0.594354in}}%
\pgfpathlineto{\pgfqpoint{2.826652in}{0.595067in}}%
\pgfpathlineto{\pgfqpoint{2.827836in}{0.594250in}}%
\pgfpathlineto{\pgfqpoint{2.828231in}{0.595737in}}%
\pgfpathlineto{\pgfqpoint{2.829415in}{0.596095in}}%
\pgfpathlineto{\pgfqpoint{2.830994in}{0.589526in}}%
\pgfpathlineto{\pgfqpoint{2.831783in}{0.591020in}}%
\pgfpathlineto{\pgfqpoint{2.832178in}{0.593081in}}%
\pgfpathlineto{\pgfqpoint{2.832968in}{0.591705in}}%
\pgfpathlineto{\pgfqpoint{2.834547in}{0.589249in}}%
\pgfpathlineto{\pgfqpoint{2.834941in}{0.589919in}}%
\pgfpathlineto{\pgfqpoint{2.838494in}{0.600053in}}%
\pgfpathlineto{\pgfqpoint{2.838889in}{0.599814in}}%
\pgfpathlineto{\pgfqpoint{2.840073in}{0.599282in}}%
\pgfpathlineto{\pgfqpoint{2.842047in}{0.593903in}}%
\pgfpathlineto{\pgfqpoint{2.843231in}{0.594564in}}%
\pgfpathlineto{\pgfqpoint{2.844415in}{0.600697in}}%
\pgfpathlineto{\pgfqpoint{2.845205in}{0.597687in}}%
\pgfpathlineto{\pgfqpoint{2.845994in}{0.593889in}}%
\pgfpathlineto{\pgfqpoint{2.846389in}{0.595432in}}%
\pgfpathlineto{\pgfqpoint{2.848363in}{0.601085in}}%
\pgfpathlineto{\pgfqpoint{2.848758in}{0.600312in}}%
\pgfpathlineto{\pgfqpoint{2.849152in}{0.600562in}}%
\pgfpathlineto{\pgfqpoint{2.850731in}{0.593631in}}%
\pgfpathlineto{\pgfqpoint{2.851126in}{0.595146in}}%
\pgfpathlineto{\pgfqpoint{2.853100in}{0.600178in}}%
\pgfpathlineto{\pgfqpoint{2.855863in}{0.594098in}}%
\pgfpathlineto{\pgfqpoint{2.858626in}{0.599314in}}%
\pgfpathlineto{\pgfqpoint{2.860205in}{0.593637in}}%
\pgfpathlineto{\pgfqpoint{2.862179in}{0.586816in}}%
\pgfpathlineto{\pgfqpoint{2.869679in}{0.574150in}}%
\pgfpathlineto{\pgfqpoint{2.873232in}{0.595396in}}%
\pgfpathlineto{\pgfqpoint{2.874811in}{0.589856in}}%
\pgfpathlineto{\pgfqpoint{2.875206in}{0.591352in}}%
\pgfpathlineto{\pgfqpoint{2.877180in}{0.597625in}}%
\pgfpathlineto{\pgfqpoint{2.878759in}{0.603471in}}%
\pgfpathlineto{\pgfqpoint{2.879153in}{0.601266in}}%
\pgfpathlineto{\pgfqpoint{2.881127in}{0.594670in}}%
\pgfpathlineto{\pgfqpoint{2.881522in}{0.595214in}}%
\pgfpathlineto{\pgfqpoint{2.882311in}{0.599123in}}%
\pgfpathlineto{\pgfqpoint{2.883101in}{0.597305in}}%
\pgfpathlineto{\pgfqpoint{2.883890in}{0.593671in}}%
\pgfpathlineto{\pgfqpoint{2.888627in}{0.567211in}}%
\pgfpathlineto{\pgfqpoint{2.889811in}{0.569744in}}%
\pgfpathlineto{\pgfqpoint{2.892575in}{0.581707in}}%
\pgfpathlineto{\pgfqpoint{2.892969in}{0.581543in}}%
\pgfpathlineto{\pgfqpoint{2.894154in}{0.580852in}}%
\pgfpathlineto{\pgfqpoint{2.894548in}{0.581476in}}%
\pgfpathlineto{\pgfqpoint{2.895733in}{0.585342in}}%
\pgfpathlineto{\pgfqpoint{2.896917in}{0.582711in}}%
\pgfpathlineto{\pgfqpoint{2.899680in}{0.575387in}}%
\pgfpathlineto{\pgfqpoint{2.900470in}{0.576053in}}%
\pgfpathlineto{\pgfqpoint{2.902443in}{0.577605in}}%
\pgfpathlineto{\pgfqpoint{2.903628in}{0.574020in}}%
\pgfpathlineto{\pgfqpoint{2.904417in}{0.574409in}}%
\pgfpathlineto{\pgfqpoint{2.907180in}{0.586702in}}%
\pgfpathlineto{\pgfqpoint{2.908365in}{0.583012in}}%
\pgfpathlineto{\pgfqpoint{2.908759in}{0.582985in}}%
\pgfpathlineto{\pgfqpoint{2.909549in}{0.578378in}}%
\pgfpathlineto{\pgfqpoint{2.910338in}{0.581399in}}%
\pgfpathlineto{\pgfqpoint{2.913102in}{0.579579in}}%
\pgfpathlineto{\pgfqpoint{2.920602in}{0.594438in}}%
\pgfpathlineto{\pgfqpoint{2.921391in}{0.593541in}}%
\pgfpathlineto{\pgfqpoint{2.922970in}{0.593152in}}%
\pgfpathlineto{\pgfqpoint{2.925339in}{0.601696in}}%
\pgfpathlineto{\pgfqpoint{2.926128in}{0.600964in}}%
\pgfpathlineto{\pgfqpoint{2.927707in}{0.600621in}}%
\pgfpathlineto{\pgfqpoint{2.928497in}{0.597211in}}%
\pgfpathlineto{\pgfqpoint{2.929681in}{0.598526in}}%
\pgfpathlineto{\pgfqpoint{2.933629in}{0.607099in}}%
\pgfpathlineto{\pgfqpoint{2.935997in}{0.600092in}}%
\pgfpathlineto{\pgfqpoint{2.937181in}{0.600839in}}%
\pgfpathlineto{\pgfqpoint{2.937971in}{0.602397in}}%
\pgfpathlineto{\pgfqpoint{2.938366in}{0.600197in}}%
\pgfpathlineto{\pgfqpoint{2.940734in}{0.590428in}}%
\pgfpathlineto{\pgfqpoint{2.941129in}{0.590027in}}%
\pgfpathlineto{\pgfqpoint{2.943103in}{0.594082in}}%
\pgfpathlineto{\pgfqpoint{2.944287in}{0.590888in}}%
\pgfpathlineto{\pgfqpoint{2.944682in}{0.592630in}}%
\pgfpathlineto{\pgfqpoint{2.946261in}{0.598789in}}%
\pgfpathlineto{\pgfqpoint{2.947445in}{0.596963in}}%
\pgfpathlineto{\pgfqpoint{2.948234in}{0.595704in}}%
\pgfpathlineto{\pgfqpoint{2.951787in}{0.586962in}}%
\pgfpathlineto{\pgfqpoint{2.952182in}{0.588835in}}%
\pgfpathlineto{\pgfqpoint{2.953366in}{0.591044in}}%
\pgfpathlineto{\pgfqpoint{2.953761in}{0.589532in}}%
\pgfpathlineto{\pgfqpoint{2.954156in}{0.588326in}}%
\pgfpathlineto{\pgfqpoint{2.954945in}{0.590056in}}%
\pgfpathlineto{\pgfqpoint{2.963235in}{0.607835in}}%
\pgfpathlineto{\pgfqpoint{2.964814in}{0.606911in}}%
\pgfpathlineto{\pgfqpoint{2.965998in}{0.606108in}}%
\pgfpathlineto{\pgfqpoint{2.966393in}{0.606517in}}%
\pgfpathlineto{\pgfqpoint{2.968366in}{0.606111in}}%
\pgfpathlineto{\pgfqpoint{2.970340in}{0.605905in}}%
\pgfpathlineto{\pgfqpoint{2.973103in}{0.604473in}}%
\pgfpathlineto{\pgfqpoint{2.973498in}{0.605106in}}%
\pgfpathlineto{\pgfqpoint{2.974288in}{0.606763in}}%
\pgfpathlineto{\pgfqpoint{2.974682in}{0.606270in}}%
\pgfpathlineto{\pgfqpoint{2.979025in}{0.587074in}}%
\pgfpathlineto{\pgfqpoint{2.980604in}{0.579468in}}%
\pgfpathlineto{\pgfqpoint{2.981393in}{0.580409in}}%
\pgfpathlineto{\pgfqpoint{2.982972in}{0.586167in}}%
\pgfpathlineto{\pgfqpoint{2.983762in}{0.585634in}}%
\pgfpathlineto{\pgfqpoint{2.984156in}{0.585028in}}%
\pgfpathlineto{\pgfqpoint{2.984946in}{0.586600in}}%
\pgfpathlineto{\pgfqpoint{2.985735in}{0.588333in}}%
\pgfpathlineto{\pgfqpoint{2.987709in}{0.591376in}}%
\pgfpathlineto{\pgfqpoint{2.989683in}{0.593284in}}%
\pgfpathlineto{\pgfqpoint{2.990078in}{0.592780in}}%
\pgfpathlineto{\pgfqpoint{2.992051in}{0.586352in}}%
\pgfpathlineto{\pgfqpoint{2.992446in}{0.586837in}}%
\pgfpathlineto{\pgfqpoint{2.995999in}{0.591852in}}%
\pgfpathlineto{\pgfqpoint{2.997973in}{0.597800in}}%
\pgfpathlineto{\pgfqpoint{2.999552in}{0.592793in}}%
\pgfpathlineto{\pgfqpoint{3.000736in}{0.595285in}}%
\pgfpathlineto{\pgfqpoint{3.001131in}{0.595667in}}%
\pgfpathlineto{\pgfqpoint{3.001525in}{0.594658in}}%
\pgfpathlineto{\pgfqpoint{3.002315in}{0.593973in}}%
\pgfpathlineto{\pgfqpoint{3.002710in}{0.594851in}}%
\pgfpathlineto{\pgfqpoint{3.005078in}{0.599725in}}%
\pgfpathlineto{\pgfqpoint{3.005473in}{0.599184in}}%
\pgfpathlineto{\pgfqpoint{3.009815in}{0.587510in}}%
\pgfpathlineto{\pgfqpoint{3.010210in}{0.589232in}}%
\pgfpathlineto{\pgfqpoint{3.012973in}{0.596098in}}%
\pgfpathlineto{\pgfqpoint{3.016131in}{0.598731in}}%
\pgfpathlineto{\pgfqpoint{3.019289in}{0.602175in}}%
\pgfpathlineto{\pgfqpoint{3.020473in}{0.605582in}}%
\pgfpathlineto{\pgfqpoint{3.020868in}{0.603615in}}%
\pgfpathlineto{\pgfqpoint{3.025210in}{0.594538in}}%
\pgfpathlineto{\pgfqpoint{3.026000in}{0.595374in}}%
\pgfpathlineto{\pgfqpoint{3.027579in}{0.589117in}}%
\pgfpathlineto{\pgfqpoint{3.027974in}{0.591661in}}%
\pgfpathlineto{\pgfqpoint{3.029158in}{0.596097in}}%
\pgfpathlineto{\pgfqpoint{3.029553in}{0.594631in}}%
\pgfpathlineto{\pgfqpoint{3.031132in}{0.592517in}}%
\pgfpathlineto{\pgfqpoint{3.031526in}{0.592751in}}%
\pgfpathlineto{\pgfqpoint{3.032316in}{0.595275in}}%
\pgfpathlineto{\pgfqpoint{3.033105in}{0.593098in}}%
\pgfpathlineto{\pgfqpoint{3.034290in}{0.588589in}}%
\pgfpathlineto{\pgfqpoint{3.035474in}{0.591360in}}%
\pgfpathlineto{\pgfqpoint{3.038237in}{0.596666in}}%
\pgfpathlineto{\pgfqpoint{3.039027in}{0.597111in}}%
\pgfpathlineto{\pgfqpoint{3.039421in}{0.596360in}}%
\pgfpathlineto{\pgfqpoint{3.042974in}{0.591193in}}%
\pgfpathlineto{\pgfqpoint{3.044553in}{0.589327in}}%
\pgfpathlineto{\pgfqpoint{3.046527in}{0.592590in}}%
\pgfpathlineto{\pgfqpoint{3.047316in}{0.590321in}}%
\pgfpathlineto{\pgfqpoint{3.048106in}{0.591340in}}%
\pgfpathlineto{\pgfqpoint{3.048500in}{0.591455in}}%
\pgfpathlineto{\pgfqpoint{3.051658in}{0.579380in}}%
\pgfpathlineto{\pgfqpoint{3.054422in}{0.588197in}}%
\pgfpathlineto{\pgfqpoint{3.054816in}{0.586391in}}%
\pgfpathlineto{\pgfqpoint{3.056395in}{0.581714in}}%
\pgfpathlineto{\pgfqpoint{3.057580in}{0.587372in}}%
\pgfpathlineto{\pgfqpoint{3.058369in}{0.585384in}}%
\pgfpathlineto{\pgfqpoint{3.058764in}{0.584758in}}%
\pgfpathlineto{\pgfqpoint{3.059159in}{0.586866in}}%
\pgfpathlineto{\pgfqpoint{3.063106in}{0.599744in}}%
\pgfpathlineto{\pgfqpoint{3.063501in}{0.600752in}}%
\pgfpathlineto{\pgfqpoint{3.063896in}{0.599170in}}%
\pgfpathlineto{\pgfqpoint{3.067843in}{0.589554in}}%
\pgfpathlineto{\pgfqpoint{3.070606in}{0.592576in}}%
\pgfpathlineto{\pgfqpoint{3.074554in}{0.604777in}}%
\pgfpathlineto{\pgfqpoint{3.075343in}{0.603173in}}%
\pgfpathlineto{\pgfqpoint{3.080870in}{0.587601in}}%
\pgfpathlineto{\pgfqpoint{3.082844in}{0.590026in}}%
\pgfpathlineto{\pgfqpoint{3.084423in}{0.594240in}}%
\pgfpathlineto{\pgfqpoint{3.087975in}{0.585319in}}%
\pgfpathlineto{\pgfqpoint{3.088765in}{0.585695in}}%
\pgfpathlineto{\pgfqpoint{3.089160in}{0.584892in}}%
\pgfpathlineto{\pgfqpoint{3.093897in}{0.573561in}}%
\pgfpathlineto{\pgfqpoint{3.095081in}{0.575544in}}%
\pgfpathlineto{\pgfqpoint{3.098239in}{0.574183in}}%
\pgfpathlineto{\pgfqpoint{3.098634in}{0.572958in}}%
\pgfpathlineto{\pgfqpoint{3.099818in}{0.574202in}}%
\pgfpathlineto{\pgfqpoint{3.102976in}{0.571135in}}%
\pgfpathlineto{\pgfqpoint{3.100607in}{0.575148in}}%
\pgfpathlineto{\pgfqpoint{3.103371in}{0.571542in}}%
\pgfpathlineto{\pgfqpoint{3.106529in}{0.575258in}}%
\pgfpathlineto{\pgfqpoint{3.109292in}{0.582232in}}%
\pgfpathlineto{\pgfqpoint{3.109687in}{0.581681in}}%
\pgfpathlineto{\pgfqpoint{3.110476in}{0.579443in}}%
\pgfpathlineto{\pgfqpoint{3.110871in}{0.581354in}}%
\pgfpathlineto{\pgfqpoint{3.113634in}{0.585728in}}%
\pgfpathlineto{\pgfqpoint{3.117187in}{0.571919in}}%
\pgfpathlineto{\pgfqpoint{3.118766in}{0.559947in}}%
\pgfpathlineto{\pgfqpoint{3.119555in}{0.563102in}}%
\pgfpathlineto{\pgfqpoint{3.120740in}{0.561607in}}%
\pgfpathlineto{\pgfqpoint{3.121924in}{0.559788in}}%
\pgfpathlineto{\pgfqpoint{3.122319in}{0.560314in}}%
\pgfpathlineto{\pgfqpoint{3.127450in}{0.570637in}}%
\pgfpathlineto{\pgfqpoint{3.127845in}{0.570098in}}%
\pgfpathlineto{\pgfqpoint{3.129029in}{0.567157in}}%
\pgfpathlineto{\pgfqpoint{3.129819in}{0.568536in}}%
\pgfpathlineto{\pgfqpoint{3.131003in}{0.571567in}}%
\pgfpathlineto{\pgfqpoint{3.131792in}{0.570483in}}%
\pgfpathlineto{\pgfqpoint{3.132582in}{0.571729in}}%
\pgfpathlineto{\pgfqpoint{3.133371in}{0.570407in}}%
\pgfpathlineto{\pgfqpoint{3.134950in}{0.565985in}}%
\pgfpathlineto{\pgfqpoint{3.136135in}{0.567475in}}%
\pgfpathlineto{\pgfqpoint{3.136924in}{0.568777in}}%
\pgfpathlineto{\pgfqpoint{3.137714in}{0.567500in}}%
\pgfpathlineto{\pgfqpoint{3.139687in}{0.561314in}}%
\pgfpathlineto{\pgfqpoint{3.140477in}{0.563756in}}%
\pgfpathlineto{\pgfqpoint{3.142451in}{0.568154in}}%
\pgfpathlineto{\pgfqpoint{3.142845in}{0.567699in}}%
\pgfpathlineto{\pgfqpoint{3.144424in}{0.563944in}}%
\pgfpathlineto{\pgfqpoint{3.145214in}{0.562549in}}%
\pgfpathlineto{\pgfqpoint{3.145609in}{0.563667in}}%
\pgfpathlineto{\pgfqpoint{3.146003in}{0.564169in}}%
\pgfpathlineto{\pgfqpoint{3.148372in}{0.576024in}}%
\pgfpathlineto{\pgfqpoint{3.148767in}{0.574306in}}%
\pgfpathlineto{\pgfqpoint{3.151925in}{0.568286in}}%
\pgfpathlineto{\pgfqpoint{3.153109in}{0.571022in}}%
\pgfpathlineto{\pgfqpoint{3.154293in}{0.572554in}}%
\pgfpathlineto{\pgfqpoint{3.154688in}{0.570501in}}%
\pgfpathlineto{\pgfqpoint{3.157056in}{0.560622in}}%
\pgfpathlineto{\pgfqpoint{3.157451in}{0.561813in}}%
\pgfpathlineto{\pgfqpoint{3.160214in}{0.567873in}}%
\pgfpathlineto{\pgfqpoint{3.161004in}{0.571777in}}%
\pgfpathlineto{\pgfqpoint{3.162978in}{0.578443in}}%
\pgfpathlineto{\pgfqpoint{3.163372in}{0.578267in}}%
\pgfpathlineto{\pgfqpoint{3.165346in}{0.575355in}}%
\pgfpathlineto{\pgfqpoint{3.166925in}{0.582603in}}%
\pgfpathlineto{\pgfqpoint{3.167320in}{0.582274in}}%
\pgfpathlineto{\pgfqpoint{3.172452in}{0.556035in}}%
\pgfpathlineto{\pgfqpoint{3.172846in}{0.556080in}}%
\pgfpathlineto{\pgfqpoint{3.173241in}{0.554503in}}%
\pgfpathlineto{\pgfqpoint{3.173636in}{0.556657in}}%
\pgfpathlineto{\pgfqpoint{3.176399in}{0.564528in}}%
\pgfpathlineto{\pgfqpoint{3.177189in}{0.564178in}}%
\pgfpathlineto{\pgfqpoint{3.177583in}{0.565477in}}%
\pgfpathlineto{\pgfqpoint{3.181926in}{0.580003in}}%
\pgfpathlineto{\pgfqpoint{3.182320in}{0.580631in}}%
\pgfpathlineto{\pgfqpoint{3.182715in}{0.578742in}}%
\pgfpathlineto{\pgfqpoint{3.183899in}{0.575960in}}%
\pgfpathlineto{\pgfqpoint{3.184294in}{0.576334in}}%
\pgfpathlineto{\pgfqpoint{3.188242in}{0.585465in}}%
\pgfpathlineto{\pgfqpoint{3.189821in}{0.583656in}}%
\pgfpathlineto{\pgfqpoint{3.190215in}{0.584172in}}%
\pgfpathlineto{\pgfqpoint{3.194163in}{0.594027in}}%
\pgfpathlineto{\pgfqpoint{3.196926in}{0.597448in}}%
\pgfpathlineto{\pgfqpoint{3.197321in}{0.595333in}}%
\pgfpathlineto{\pgfqpoint{3.197321in}{0.595333in}}%
\pgfusepath{stroke}%
\end{pgfscope}%
\begin{pgfscope}%
\pgfpathrectangle{\pgfqpoint{0.608025in}{0.484444in}}{\pgfqpoint{2.712595in}{1.541287in}}%
\pgfusepath{clip}%
\pgfsetbuttcap%
\pgfsetmiterjoin%
\definecolor{currentfill}{rgb}{0.121569,0.466667,0.705882}%
\pgfsetfillcolor{currentfill}%
\pgfsetlinewidth{1.003750pt}%
\definecolor{currentstroke}{rgb}{0.121569,0.466667,0.705882}%
\pgfsetstrokecolor{currentstroke}%
\pgfsetdash{}{0pt}%
\pgfsys@defobject{currentmarker}{\pgfqpoint{-0.020833in}{-0.020833in}}{\pgfqpoint{0.020833in}{0.020833in}}{%
\pgfpathmoveto{\pgfqpoint{-0.020833in}{-0.020833in}}%
\pgfpathlineto{\pgfqpoint{0.020833in}{-0.020833in}}%
\pgfpathlineto{\pgfqpoint{0.020833in}{0.020833in}}%
\pgfpathlineto{\pgfqpoint{-0.020833in}{0.020833in}}%
\pgfpathlineto{\pgfqpoint{-0.020833in}{-0.020833in}}%
\pgfpathclose%
\pgfusepath{stroke,fill}%
}%
\begin{pgfscope}%
\pgfsys@transformshift{0.731720in}{1.955197in}%
\pgfsys@useobject{currentmarker}{}%
\end{pgfscope}%
\begin{pgfscope}%
\pgfsys@transformshift{0.810669in}{1.712919in}%
\pgfsys@useobject{currentmarker}{}%
\end{pgfscope}%
\begin{pgfscope}%
\pgfsys@transformshift{0.889619in}{1.373857in}%
\pgfsys@useobject{currentmarker}{}%
\end{pgfscope}%
\begin{pgfscope}%
\pgfsys@transformshift{0.968569in}{1.138720in}%
\pgfsys@useobject{currentmarker}{}%
\end{pgfscope}%
\begin{pgfscope}%
\pgfsys@transformshift{1.047519in}{0.952887in}%
\pgfsys@useobject{currentmarker}{}%
\end{pgfscope}%
\begin{pgfscope}%
\pgfsys@transformshift{1.126468in}{0.787480in}%
\pgfsys@useobject{currentmarker}{}%
\end{pgfscope}%
\begin{pgfscope}%
\pgfsys@transformshift{1.205418in}{0.656741in}%
\pgfsys@useobject{currentmarker}{}%
\end{pgfscope}%
\begin{pgfscope}%
\pgfsys@transformshift{1.284368in}{0.652129in}%
\pgfsys@useobject{currentmarker}{}%
\end{pgfscope}%
\begin{pgfscope}%
\pgfsys@transformshift{1.363318in}{0.631481in}%
\pgfsys@useobject{currentmarker}{}%
\end{pgfscope}%
\begin{pgfscope}%
\pgfsys@transformshift{1.442268in}{0.614957in}%
\pgfsys@useobject{currentmarker}{}%
\end{pgfscope}%
\begin{pgfscope}%
\pgfsys@transformshift{1.521217in}{0.613073in}%
\pgfsys@useobject{currentmarker}{}%
\end{pgfscope}%
\begin{pgfscope}%
\pgfsys@transformshift{1.600167in}{0.601740in}%
\pgfsys@useobject{currentmarker}{}%
\end{pgfscope}%
\begin{pgfscope}%
\pgfsys@transformshift{1.679117in}{0.596489in}%
\pgfsys@useobject{currentmarker}{}%
\end{pgfscope}%
\begin{pgfscope}%
\pgfsys@transformshift{1.758067in}{0.583362in}%
\pgfsys@useobject{currentmarker}{}%
\end{pgfscope}%
\begin{pgfscope}%
\pgfsys@transformshift{1.837016in}{0.592134in}%
\pgfsys@useobject{currentmarker}{}%
\end{pgfscope}%
\begin{pgfscope}%
\pgfsys@transformshift{1.915966in}{0.582869in}%
\pgfsys@useobject{currentmarker}{}%
\end{pgfscope}%
\begin{pgfscope}%
\pgfsys@transformshift{1.994916in}{0.590216in}%
\pgfsys@useobject{currentmarker}{}%
\end{pgfscope}%
\begin{pgfscope}%
\pgfsys@transformshift{2.073866in}{0.586482in}%
\pgfsys@useobject{currentmarker}{}%
\end{pgfscope}%
\begin{pgfscope}%
\pgfsys@transformshift{2.152815in}{0.574375in}%
\pgfsys@useobject{currentmarker}{}%
\end{pgfscope}%
\begin{pgfscope}%
\pgfsys@transformshift{2.231765in}{0.602702in}%
\pgfsys@useobject{currentmarker}{}%
\end{pgfscope}%
\begin{pgfscope}%
\pgfsys@transformshift{2.310715in}{0.572786in}%
\pgfsys@useobject{currentmarker}{}%
\end{pgfscope}%
\begin{pgfscope}%
\pgfsys@transformshift{2.389665in}{0.580478in}%
\pgfsys@useobject{currentmarker}{}%
\end{pgfscope}%
\begin{pgfscope}%
\pgfsys@transformshift{2.468614in}{0.572093in}%
\pgfsys@useobject{currentmarker}{}%
\end{pgfscope}%
\begin{pgfscope}%
\pgfsys@transformshift{2.547564in}{0.583451in}%
\pgfsys@useobject{currentmarker}{}%
\end{pgfscope}%
\begin{pgfscope}%
\pgfsys@transformshift{2.626514in}{0.601111in}%
\pgfsys@useobject{currentmarker}{}%
\end{pgfscope}%
\begin{pgfscope}%
\pgfsys@transformshift{2.705464in}{0.587538in}%
\pgfsys@useobject{currentmarker}{}%
\end{pgfscope}%
\begin{pgfscope}%
\pgfsys@transformshift{2.784414in}{0.602306in}%
\pgfsys@useobject{currentmarker}{}%
\end{pgfscope}%
\begin{pgfscope}%
\pgfsys@transformshift{2.863363in}{0.583726in}%
\pgfsys@useobject{currentmarker}{}%
\end{pgfscope}%
\begin{pgfscope}%
\pgfsys@transformshift{2.942313in}{0.591813in}%
\pgfsys@useobject{currentmarker}{}%
\end{pgfscope}%
\begin{pgfscope}%
\pgfsys@transformshift{3.021263in}{0.602519in}%
\pgfsys@useobject{currentmarker}{}%
\end{pgfscope}%
\begin{pgfscope}%
\pgfsys@transformshift{3.100213in}{0.573695in}%
\pgfsys@useobject{currentmarker}{}%
\end{pgfscope}%
\begin{pgfscope}%
\pgfsys@transformshift{3.179162in}{0.572333in}%
\pgfsys@useobject{currentmarker}{}%
\end{pgfscope}%
\end{pgfscope}%
\begin{pgfscope}%
\pgfpathrectangle{\pgfqpoint{0.608025in}{0.484444in}}{\pgfqpoint{2.712595in}{1.541287in}}%
\pgfusepath{clip}%
\pgfsetrectcap%
\pgfsetroundjoin%
\pgfsetlinewidth{1.505625pt}%
\definecolor{currentstroke}{rgb}{1.000000,0.498039,0.054902}%
\pgfsetstrokecolor{currentstroke}%
\pgfsetdash{}{0pt}%
\pgfpathmoveto{\pgfqpoint{0.731325in}{1.955466in}}%
\pgfpathlineto{\pgfqpoint{0.743957in}{1.945714in}}%
\pgfpathlineto{\pgfqpoint{0.745931in}{1.942897in}}%
\pgfpathlineto{\pgfqpoint{0.746325in}{1.943198in}}%
\pgfpathlineto{\pgfqpoint{0.747904in}{1.940387in}}%
\pgfpathlineto{\pgfqpoint{0.756589in}{1.921045in}}%
\pgfpathlineto{\pgfqpoint{0.765668in}{1.903085in}}%
\pgfpathlineto{\pgfqpoint{0.772379in}{1.881289in}}%
\pgfpathlineto{\pgfqpoint{0.787774in}{1.822212in}}%
\pgfpathlineto{\pgfqpoint{0.793300in}{1.799004in}}%
\pgfpathlineto{\pgfqpoint{0.798827in}{1.762161in}}%
\pgfpathlineto{\pgfqpoint{0.826854in}{1.663153in}}%
\pgfpathlineto{\pgfqpoint{0.890803in}{1.344508in}}%
\pgfpathlineto{\pgfqpoint{0.928699in}{1.203963in}}%
\pgfpathlineto{\pgfqpoint{0.952384in}{1.151778in}}%
\pgfpathlineto{\pgfqpoint{0.974095in}{1.110876in}}%
\pgfpathlineto{\pgfqpoint{0.981596in}{1.102494in}}%
\pgfpathlineto{\pgfqpoint{0.989491in}{1.091134in}}%
\pgfpathlineto{\pgfqpoint{0.992649in}{1.080040in}}%
\pgfpathlineto{\pgfqpoint{1.023044in}{0.981074in}}%
\pgfpathlineto{\pgfqpoint{1.024623in}{0.978801in}}%
\pgfpathlineto{\pgfqpoint{1.028571in}{0.971650in}}%
\pgfpathlineto{\pgfqpoint{1.031334in}{0.968099in}}%
\pgfpathlineto{\pgfqpoint{1.035282in}{0.959777in}}%
\pgfpathlineto{\pgfqpoint{1.042387in}{0.943719in}}%
\pgfpathlineto{\pgfqpoint{1.047519in}{0.923919in}}%
\pgfpathlineto{\pgfqpoint{1.053045in}{0.891560in}}%
\pgfpathlineto{\pgfqpoint{1.064888in}{0.830750in}}%
\pgfpathlineto{\pgfqpoint{1.066072in}{0.829059in}}%
\pgfpathlineto{\pgfqpoint{1.068835in}{0.821665in}}%
\pgfpathlineto{\pgfqpoint{1.071598in}{0.821360in}}%
\pgfpathlineto{\pgfqpoint{1.073572in}{0.818321in}}%
\pgfpathlineto{\pgfqpoint{1.074362in}{0.817780in}}%
\pgfpathlineto{\pgfqpoint{1.074756in}{0.819168in}}%
\pgfpathlineto{\pgfqpoint{1.077125in}{0.827759in}}%
\pgfpathlineto{\pgfqpoint{1.082257in}{0.855515in}}%
\pgfpathlineto{\pgfqpoint{1.083441in}{0.854884in}}%
\pgfpathlineto{\pgfqpoint{1.088178in}{0.843965in}}%
\pgfpathlineto{\pgfqpoint{1.092520in}{0.837724in}}%
\pgfpathlineto{\pgfqpoint{1.094494in}{0.835541in}}%
\pgfpathlineto{\pgfqpoint{1.096862in}{0.834821in}}%
\pgfpathlineto{\pgfqpoint{1.100415in}{0.834212in}}%
\pgfpathlineto{\pgfqpoint{1.106336in}{0.826067in}}%
\pgfpathlineto{\pgfqpoint{1.106731in}{0.826521in}}%
\pgfpathlineto{\pgfqpoint{1.110679in}{0.834144in}}%
\pgfpathlineto{\pgfqpoint{1.111073in}{0.833943in}}%
\pgfpathlineto{\pgfqpoint{1.111468in}{0.834109in}}%
\pgfpathlineto{\pgfqpoint{1.111863in}{0.833122in}}%
\pgfpathlineto{\pgfqpoint{1.115810in}{0.829495in}}%
\pgfpathlineto{\pgfqpoint{1.116205in}{0.829947in}}%
\pgfpathlineto{\pgfqpoint{1.116995in}{0.830043in}}%
\pgfpathlineto{\pgfqpoint{1.117389in}{0.828996in}}%
\pgfpathlineto{\pgfqpoint{1.119363in}{0.824811in}}%
\pgfpathlineto{\pgfqpoint{1.123311in}{0.819877in}}%
\pgfpathlineto{\pgfqpoint{1.128047in}{0.823639in}}%
\pgfpathlineto{\pgfqpoint{1.130811in}{0.823844in}}%
\pgfpathlineto{\pgfqpoint{1.132784in}{0.826404in}}%
\pgfpathlineto{\pgfqpoint{1.135153in}{0.827377in}}%
\pgfpathlineto{\pgfqpoint{1.139495in}{0.818767in}}%
\pgfpathlineto{\pgfqpoint{1.145416in}{0.793923in}}%
\pgfpathlineto{\pgfqpoint{1.152917in}{0.739837in}}%
\pgfpathlineto{\pgfqpoint{1.154101in}{0.741849in}}%
\pgfpathlineto{\pgfqpoint{1.155285in}{0.743489in}}%
\pgfpathlineto{\pgfqpoint{1.156075in}{0.746104in}}%
\pgfpathlineto{\pgfqpoint{1.156864in}{0.743669in}}%
\pgfpathlineto{\pgfqpoint{1.159627in}{0.740830in}}%
\pgfpathlineto{\pgfqpoint{1.160022in}{0.741412in}}%
\pgfpathlineto{\pgfqpoint{1.162785in}{0.750396in}}%
\pgfpathlineto{\pgfqpoint{1.164759in}{0.749354in}}%
\pgfpathlineto{\pgfqpoint{1.169101in}{0.740034in}}%
\pgfpathlineto{\pgfqpoint{1.171470in}{0.733691in}}%
\pgfpathlineto{\pgfqpoint{1.179365in}{0.701942in}}%
\pgfpathlineto{\pgfqpoint{1.180944in}{0.700502in}}%
\pgfpathlineto{\pgfqpoint{1.181733in}{0.701516in}}%
\pgfpathlineto{\pgfqpoint{1.183312in}{0.703577in}}%
\pgfpathlineto{\pgfqpoint{1.185681in}{0.707989in}}%
\pgfpathlineto{\pgfqpoint{1.190418in}{0.719874in}}%
\pgfpathlineto{\pgfqpoint{1.191602in}{0.717357in}}%
\pgfpathlineto{\pgfqpoint{1.195155in}{0.705190in}}%
\pgfpathlineto{\pgfqpoint{1.195944in}{0.708254in}}%
\pgfpathlineto{\pgfqpoint{1.197918in}{0.710823in}}%
\pgfpathlineto{\pgfqpoint{1.200287in}{0.715194in}}%
\pgfpathlineto{\pgfqpoint{1.200681in}{0.714223in}}%
\pgfpathlineto{\pgfqpoint{1.205813in}{0.684996in}}%
\pgfpathlineto{\pgfqpoint{1.208576in}{0.669812in}}%
\pgfpathlineto{\pgfqpoint{1.216471in}{0.659216in}}%
\pgfpathlineto{\pgfqpoint{1.219629in}{0.665397in}}%
\pgfpathlineto{\pgfqpoint{1.221998in}{0.662297in}}%
\pgfpathlineto{\pgfqpoint{1.223971in}{0.672001in}}%
\pgfpathlineto{\pgfqpoint{1.224761in}{0.670971in}}%
\pgfpathlineto{\pgfqpoint{1.225550in}{0.669113in}}%
\pgfpathlineto{\pgfqpoint{1.225945in}{0.670988in}}%
\pgfpathlineto{\pgfqpoint{1.228708in}{0.685023in}}%
\pgfpathlineto{\pgfqpoint{1.231077in}{0.680503in}}%
\pgfpathlineto{\pgfqpoint{1.231472in}{0.680731in}}%
\pgfpathlineto{\pgfqpoint{1.235024in}{0.692939in}}%
\pgfpathlineto{\pgfqpoint{1.236209in}{0.689497in}}%
\pgfpathlineto{\pgfqpoint{1.237788in}{0.692803in}}%
\pgfpathlineto{\pgfqpoint{1.241340in}{0.699799in}}%
\pgfpathlineto{\pgfqpoint{1.245683in}{0.712479in}}%
\pgfpathlineto{\pgfqpoint{1.246077in}{0.712439in}}%
\pgfpathlineto{\pgfqpoint{1.248051in}{0.714682in}}%
\pgfpathlineto{\pgfqpoint{1.253972in}{0.706550in}}%
\pgfpathlineto{\pgfqpoint{1.254762in}{0.708813in}}%
\pgfpathlineto{\pgfqpoint{1.257920in}{0.712442in}}%
\pgfpathlineto{\pgfqpoint{1.265025in}{0.705882in}}%
\pgfpathlineto{\pgfqpoint{1.266210in}{0.706048in}}%
\pgfpathlineto{\pgfqpoint{1.266604in}{0.705582in}}%
\pgfpathlineto{\pgfqpoint{1.268578in}{0.705334in}}%
\pgfpathlineto{\pgfqpoint{1.270947in}{0.708699in}}%
\pgfpathlineto{\pgfqpoint{1.271341in}{0.708337in}}%
\pgfpathlineto{\pgfqpoint{1.275684in}{0.699261in}}%
\pgfpathlineto{\pgfqpoint{1.276078in}{0.700271in}}%
\pgfpathlineto{\pgfqpoint{1.276473in}{0.699892in}}%
\pgfpathlineto{\pgfqpoint{1.276868in}{0.700561in}}%
\pgfpathlineto{\pgfqpoint{1.278842in}{0.706224in}}%
\pgfpathlineto{\pgfqpoint{1.279236in}{0.705925in}}%
\pgfpathlineto{\pgfqpoint{1.281210in}{0.705715in}}%
\pgfpathlineto{\pgfqpoint{1.283579in}{0.703214in}}%
\pgfpathlineto{\pgfqpoint{1.286737in}{0.700012in}}%
\pgfpathlineto{\pgfqpoint{1.287921in}{0.702052in}}%
\pgfpathlineto{\pgfqpoint{1.288710in}{0.701057in}}%
\pgfpathlineto{\pgfqpoint{1.291868in}{0.694937in}}%
\pgfpathlineto{\pgfqpoint{1.292658in}{0.693955in}}%
\pgfpathlineto{\pgfqpoint{1.293447in}{0.694885in}}%
\pgfpathlineto{\pgfqpoint{1.294631in}{0.696323in}}%
\pgfpathlineto{\pgfqpoint{1.295026in}{0.695209in}}%
\pgfpathlineto{\pgfqpoint{1.295816in}{0.694809in}}%
\pgfpathlineto{\pgfqpoint{1.297789in}{0.687758in}}%
\pgfpathlineto{\pgfqpoint{1.298184in}{0.687959in}}%
\pgfpathlineto{\pgfqpoint{1.300158in}{0.690438in}}%
\pgfpathlineto{\pgfqpoint{1.300553in}{0.689916in}}%
\pgfpathlineto{\pgfqpoint{1.300947in}{0.688853in}}%
\pgfpathlineto{\pgfqpoint{1.302132in}{0.690052in}}%
\pgfpathlineto{\pgfqpoint{1.303316in}{0.691672in}}%
\pgfpathlineto{\pgfqpoint{1.305684in}{0.697785in}}%
\pgfpathlineto{\pgfqpoint{1.308053in}{0.689424in}}%
\pgfpathlineto{\pgfqpoint{1.309632in}{0.678283in}}%
\pgfpathlineto{\pgfqpoint{1.310421in}{0.679757in}}%
\pgfpathlineto{\pgfqpoint{1.315553in}{0.689808in}}%
\pgfpathlineto{\pgfqpoint{1.318316in}{0.693753in}}%
\pgfpathlineto{\pgfqpoint{1.320685in}{0.701474in}}%
\pgfpathlineto{\pgfqpoint{1.324238in}{0.706396in}}%
\pgfpathlineto{\pgfqpoint{1.325027in}{0.705251in}}%
\pgfpathlineto{\pgfqpoint{1.328580in}{0.701552in}}%
\pgfpathlineto{\pgfqpoint{1.332133in}{0.695111in}}%
\pgfpathlineto{\pgfqpoint{1.336870in}{0.681468in}}%
\pgfpathlineto{\pgfqpoint{1.339633in}{0.664586in}}%
\pgfpathlineto{\pgfqpoint{1.340422in}{0.664546in}}%
\pgfpathlineto{\pgfqpoint{1.344765in}{0.655469in}}%
\pgfpathlineto{\pgfqpoint{1.346738in}{0.657955in}}%
\pgfpathlineto{\pgfqpoint{1.348712in}{0.671576in}}%
\pgfpathlineto{\pgfqpoint{1.349502in}{0.668968in}}%
\pgfpathlineto{\pgfqpoint{1.351081in}{0.664634in}}%
\pgfpathlineto{\pgfqpoint{1.351475in}{0.665287in}}%
\pgfpathlineto{\pgfqpoint{1.354239in}{0.672564in}}%
\pgfpathlineto{\pgfqpoint{1.358186in}{0.673652in}}%
\pgfpathlineto{\pgfqpoint{1.360555in}{0.676653in}}%
\pgfpathlineto{\pgfqpoint{1.362528in}{0.677037in}}%
\pgfpathlineto{\pgfqpoint{1.364502in}{0.676341in}}%
\pgfpathlineto{\pgfqpoint{1.366081in}{0.681874in}}%
\pgfpathlineto{\pgfqpoint{1.366871in}{0.681203in}}%
\pgfpathlineto{\pgfqpoint{1.368055in}{0.681559in}}%
\pgfpathlineto{\pgfqpoint{1.370029in}{0.678775in}}%
\pgfpathlineto{\pgfqpoint{1.371608in}{0.678010in}}%
\pgfpathlineto{\pgfqpoint{1.372792in}{0.676769in}}%
\pgfpathlineto{\pgfqpoint{1.373581in}{0.677279in}}%
\pgfpathlineto{\pgfqpoint{1.375160in}{0.680555in}}%
\pgfpathlineto{\pgfqpoint{1.375950in}{0.678536in}}%
\pgfpathlineto{\pgfqpoint{1.379108in}{0.671366in}}%
\pgfpathlineto{\pgfqpoint{1.379502in}{0.672277in}}%
\pgfpathlineto{\pgfqpoint{1.380292in}{0.670640in}}%
\pgfpathlineto{\pgfqpoint{1.383055in}{0.662353in}}%
\pgfpathlineto{\pgfqpoint{1.383450in}{0.662642in}}%
\pgfpathlineto{\pgfqpoint{1.384634in}{0.664395in}}%
\pgfpathlineto{\pgfqpoint{1.385029in}{0.664080in}}%
\pgfpathlineto{\pgfqpoint{1.387792in}{0.659588in}}%
\pgfpathlineto{\pgfqpoint{1.390950in}{0.651933in}}%
\pgfpathlineto{\pgfqpoint{1.392134in}{0.648346in}}%
\pgfpathlineto{\pgfqpoint{1.392529in}{0.649283in}}%
\pgfpathlineto{\pgfqpoint{1.394503in}{0.650986in}}%
\pgfpathlineto{\pgfqpoint{1.395292in}{0.649898in}}%
\pgfpathlineto{\pgfqpoint{1.396082in}{0.651191in}}%
\pgfpathlineto{\pgfqpoint{1.398056in}{0.647272in}}%
\pgfpathlineto{\pgfqpoint{1.398450in}{0.649511in}}%
\pgfpathlineto{\pgfqpoint{1.398845in}{0.650579in}}%
\pgfpathlineto{\pgfqpoint{1.399240in}{0.648219in}}%
\pgfpathlineto{\pgfqpoint{1.399635in}{0.647441in}}%
\pgfpathlineto{\pgfqpoint{1.400424in}{0.649277in}}%
\pgfpathlineto{\pgfqpoint{1.403187in}{0.653533in}}%
\pgfpathlineto{\pgfqpoint{1.404766in}{0.660898in}}%
\pgfpathlineto{\pgfqpoint{1.405556in}{0.659910in}}%
\pgfpathlineto{\pgfqpoint{1.406740in}{0.657132in}}%
\pgfpathlineto{\pgfqpoint{1.410293in}{0.644828in}}%
\pgfpathlineto{\pgfqpoint{1.411082in}{0.646293in}}%
\pgfpathlineto{\pgfqpoint{1.411477in}{0.646067in}}%
\pgfpathlineto{\pgfqpoint{1.413846in}{0.641674in}}%
\pgfpathlineto{\pgfqpoint{1.417004in}{0.636263in}}%
\pgfpathlineto{\pgfqpoint{1.417398in}{0.637737in}}%
\pgfpathlineto{\pgfqpoint{1.418188in}{0.640187in}}%
\pgfpathlineto{\pgfqpoint{1.418977in}{0.638867in}}%
\pgfpathlineto{\pgfqpoint{1.419372in}{0.637563in}}%
\pgfpathlineto{\pgfqpoint{1.420162in}{0.638423in}}%
\pgfpathlineto{\pgfqpoint{1.421346in}{0.639286in}}%
\pgfpathlineto{\pgfqpoint{1.421741in}{0.639117in}}%
\pgfpathlineto{\pgfqpoint{1.423714in}{0.639243in}}%
\pgfpathlineto{\pgfqpoint{1.426872in}{0.642099in}}%
\pgfpathlineto{\pgfqpoint{1.430030in}{0.624826in}}%
\pgfpathlineto{\pgfqpoint{1.431215in}{0.625793in}}%
\pgfpathlineto{\pgfqpoint{1.431609in}{0.626934in}}%
\pgfpathlineto{\pgfqpoint{1.432399in}{0.624920in}}%
\pgfpathlineto{\pgfqpoint{1.433188in}{0.625178in}}%
\pgfpathlineto{\pgfqpoint{1.433583in}{0.623979in}}%
\pgfpathlineto{\pgfqpoint{1.434767in}{0.621857in}}%
\pgfpathlineto{\pgfqpoint{1.435557in}{0.622127in}}%
\pgfpathlineto{\pgfqpoint{1.436741in}{0.622775in}}%
\pgfpathlineto{\pgfqpoint{1.437136in}{0.621792in}}%
\pgfpathlineto{\pgfqpoint{1.439110in}{0.618289in}}%
\pgfpathlineto{\pgfqpoint{1.439504in}{0.618843in}}%
\pgfpathlineto{\pgfqpoint{1.441478in}{0.614239in}}%
\pgfpathlineto{\pgfqpoint{1.442268in}{0.616520in}}%
\pgfpathlineto{\pgfqpoint{1.448978in}{0.622998in}}%
\pgfpathlineto{\pgfqpoint{1.450952in}{0.621903in}}%
\pgfpathlineto{\pgfqpoint{1.455294in}{0.621625in}}%
\pgfpathlineto{\pgfqpoint{1.456479in}{0.619376in}}%
\pgfpathlineto{\pgfqpoint{1.458847in}{0.624526in}}%
\pgfpathlineto{\pgfqpoint{1.459637in}{0.623146in}}%
\pgfpathlineto{\pgfqpoint{1.462005in}{0.610145in}}%
\pgfpathlineto{\pgfqpoint{1.462400in}{0.610823in}}%
\pgfpathlineto{\pgfqpoint{1.462794in}{0.613509in}}%
\pgfpathlineto{\pgfqpoint{1.463979in}{0.611385in}}%
\pgfpathlineto{\pgfqpoint{1.465952in}{0.607114in}}%
\pgfpathlineto{\pgfqpoint{1.466347in}{0.607648in}}%
\pgfpathlineto{\pgfqpoint{1.471084in}{0.621285in}}%
\pgfpathlineto{\pgfqpoint{1.472268in}{0.622110in}}%
\pgfpathlineto{\pgfqpoint{1.473453in}{0.615267in}}%
\pgfpathlineto{\pgfqpoint{1.473847in}{0.615881in}}%
\pgfpathlineto{\pgfqpoint{1.475821in}{0.622488in}}%
\pgfpathlineto{\pgfqpoint{1.476216in}{0.622252in}}%
\pgfpathlineto{\pgfqpoint{1.477005in}{0.623409in}}%
\pgfpathlineto{\pgfqpoint{1.477795in}{0.626888in}}%
\pgfpathlineto{\pgfqpoint{1.478584in}{0.624999in}}%
\pgfpathlineto{\pgfqpoint{1.479374in}{0.626481in}}%
\pgfpathlineto{\pgfqpoint{1.480163in}{0.624076in}}%
\pgfpathlineto{\pgfqpoint{1.481348in}{0.625488in}}%
\pgfpathlineto{\pgfqpoint{1.481742in}{0.624066in}}%
\pgfpathlineto{\pgfqpoint{1.483716in}{0.614594in}}%
\pgfpathlineto{\pgfqpoint{1.484900in}{0.616969in}}%
\pgfpathlineto{\pgfqpoint{1.487664in}{0.620994in}}%
\pgfpathlineto{\pgfqpoint{1.492401in}{0.620822in}}%
\pgfpathlineto{\pgfqpoint{1.494769in}{0.626031in}}%
\pgfpathlineto{\pgfqpoint{1.496348in}{0.624591in}}%
\pgfpathlineto{\pgfqpoint{1.498717in}{0.621143in}}%
\pgfpathlineto{\pgfqpoint{1.501085in}{0.620183in}}%
\pgfpathlineto{\pgfqpoint{1.501480in}{0.621832in}}%
\pgfpathlineto{\pgfqpoint{1.502269in}{0.620136in}}%
\pgfpathlineto{\pgfqpoint{1.503454in}{0.617454in}}%
\pgfpathlineto{\pgfqpoint{1.504243in}{0.618404in}}%
\pgfpathlineto{\pgfqpoint{1.505033in}{0.619354in}}%
\pgfpathlineto{\pgfqpoint{1.505427in}{0.618563in}}%
\pgfpathlineto{\pgfqpoint{1.506612in}{0.616595in}}%
\pgfpathlineto{\pgfqpoint{1.507401in}{0.619781in}}%
\pgfpathlineto{\pgfqpoint{1.508191in}{0.617244in}}%
\pgfpathlineto{\pgfqpoint{1.509770in}{0.611678in}}%
\pgfpathlineto{\pgfqpoint{1.510559in}{0.612322in}}%
\pgfpathlineto{\pgfqpoint{1.512138in}{0.616214in}}%
\pgfpathlineto{\pgfqpoint{1.515691in}{0.619596in}}%
\pgfpathlineto{\pgfqpoint{1.516086in}{0.619348in}}%
\pgfpathlineto{\pgfqpoint{1.516480in}{0.620788in}}%
\pgfpathlineto{\pgfqpoint{1.518454in}{0.625063in}}%
\pgfpathlineto{\pgfqpoint{1.519638in}{0.621949in}}%
\pgfpathlineto{\pgfqpoint{1.520823in}{0.623279in}}%
\pgfpathlineto{\pgfqpoint{1.521217in}{0.623703in}}%
\pgfpathlineto{\pgfqpoint{1.521612in}{0.622100in}}%
\pgfpathlineto{\pgfqpoint{1.522796in}{0.617735in}}%
\pgfpathlineto{\pgfqpoint{1.523586in}{0.619038in}}%
\pgfpathlineto{\pgfqpoint{1.525165in}{0.616823in}}%
\pgfpathlineto{\pgfqpoint{1.525560in}{0.617434in}}%
\pgfpathlineto{\pgfqpoint{1.526349in}{0.620528in}}%
\pgfpathlineto{\pgfqpoint{1.527139in}{0.617892in}}%
\pgfpathlineto{\pgfqpoint{1.527928in}{0.618097in}}%
\pgfpathlineto{\pgfqpoint{1.529112in}{0.621826in}}%
\pgfpathlineto{\pgfqpoint{1.529507in}{0.619700in}}%
\pgfpathlineto{\pgfqpoint{1.530691in}{0.611243in}}%
\pgfpathlineto{\pgfqpoint{1.531481in}{0.611494in}}%
\pgfpathlineto{\pgfqpoint{1.532270in}{0.611838in}}%
\pgfpathlineto{\pgfqpoint{1.533455in}{0.616273in}}%
\pgfpathlineto{\pgfqpoint{1.534244in}{0.614659in}}%
\pgfpathlineto{\pgfqpoint{1.535034in}{0.613785in}}%
\pgfpathlineto{\pgfqpoint{1.535428in}{0.615236in}}%
\pgfpathlineto{\pgfqpoint{1.538981in}{0.621500in}}%
\pgfpathlineto{\pgfqpoint{1.540955in}{0.615473in}}%
\pgfpathlineto{\pgfqpoint{1.541350in}{0.615601in}}%
\pgfpathlineto{\pgfqpoint{1.544507in}{0.611778in}}%
\pgfpathlineto{\pgfqpoint{1.546481in}{0.612495in}}%
\pgfpathlineto{\pgfqpoint{1.549244in}{0.621194in}}%
\pgfpathlineto{\pgfqpoint{1.551218in}{0.619974in}}%
\pgfpathlineto{\pgfqpoint{1.551613in}{0.621933in}}%
\pgfpathlineto{\pgfqpoint{1.552797in}{0.620952in}}%
\pgfpathlineto{\pgfqpoint{1.555166in}{0.614356in}}%
\pgfpathlineto{\pgfqpoint{1.557139in}{0.620930in}}%
\pgfpathlineto{\pgfqpoint{1.557534in}{0.620457in}}%
\pgfpathlineto{\pgfqpoint{1.559113in}{0.615975in}}%
\pgfpathlineto{\pgfqpoint{1.559508in}{0.616412in}}%
\pgfpathlineto{\pgfqpoint{1.559903in}{0.617378in}}%
\pgfpathlineto{\pgfqpoint{1.560297in}{0.616150in}}%
\pgfpathlineto{\pgfqpoint{1.561482in}{0.614894in}}%
\pgfpathlineto{\pgfqpoint{1.561876in}{0.615206in}}%
\pgfpathlineto{\pgfqpoint{1.565429in}{0.618733in}}%
\pgfpathlineto{\pgfqpoint{1.565824in}{0.618247in}}%
\pgfpathlineto{\pgfqpoint{1.566613in}{0.618550in}}%
\pgfpathlineto{\pgfqpoint{1.570956in}{0.636427in}}%
\pgfpathlineto{\pgfqpoint{1.572929in}{0.633854in}}%
\pgfpathlineto{\pgfqpoint{1.573719in}{0.635237in}}%
\pgfpathlineto{\pgfqpoint{1.578061in}{0.641405in}}%
\pgfpathlineto{\pgfqpoint{1.578456in}{0.640469in}}%
\pgfpathlineto{\pgfqpoint{1.581614in}{0.632885in}}%
\pgfpathlineto{\pgfqpoint{1.582403in}{0.634673in}}%
\pgfpathlineto{\pgfqpoint{1.583193in}{0.635702in}}%
\pgfpathlineto{\pgfqpoint{1.583588in}{0.634129in}}%
\pgfpathlineto{\pgfqpoint{1.584377in}{0.632388in}}%
\pgfpathlineto{\pgfqpoint{1.585167in}{0.634229in}}%
\pgfpathlineto{\pgfqpoint{1.585956in}{0.635797in}}%
\pgfpathlineto{\pgfqpoint{1.586351in}{0.634977in}}%
\pgfpathlineto{\pgfqpoint{1.591088in}{0.618244in}}%
\pgfpathlineto{\pgfqpoint{1.591877in}{0.619627in}}%
\pgfpathlineto{\pgfqpoint{1.593062in}{0.623986in}}%
\pgfpathlineto{\pgfqpoint{1.593851in}{0.622108in}}%
\pgfpathlineto{\pgfqpoint{1.597799in}{0.604904in}}%
\pgfpathlineto{\pgfqpoint{1.599378in}{0.600265in}}%
\pgfpathlineto{\pgfqpoint{1.599772in}{0.602564in}}%
\pgfpathlineto{\pgfqpoint{1.601351in}{0.607423in}}%
\pgfpathlineto{\pgfqpoint{1.601746in}{0.605825in}}%
\pgfpathlineto{\pgfqpoint{1.602536in}{0.604882in}}%
\pgfpathlineto{\pgfqpoint{1.602930in}{0.605968in}}%
\pgfpathlineto{\pgfqpoint{1.603720in}{0.607391in}}%
\pgfpathlineto{\pgfqpoint{1.604115in}{0.605380in}}%
\pgfpathlineto{\pgfqpoint{1.604904in}{0.603824in}}%
\pgfpathlineto{\pgfqpoint{1.605299in}{0.604515in}}%
\pgfpathlineto{\pgfqpoint{1.607667in}{0.611137in}}%
\pgfpathlineto{\pgfqpoint{1.609246in}{0.607031in}}%
\pgfpathlineto{\pgfqpoint{1.610036in}{0.611075in}}%
\pgfpathlineto{\pgfqpoint{1.611220in}{0.610415in}}%
\pgfpathlineto{\pgfqpoint{1.614773in}{0.588846in}}%
\pgfpathlineto{\pgfqpoint{1.615168in}{0.588940in}}%
\pgfpathlineto{\pgfqpoint{1.617931in}{0.595574in}}%
\pgfpathlineto{\pgfqpoint{1.618326in}{0.595668in}}%
\pgfpathlineto{\pgfqpoint{1.619115in}{0.592246in}}%
\pgfpathlineto{\pgfqpoint{1.620299in}{0.593272in}}%
\pgfpathlineto{\pgfqpoint{1.622668in}{0.606973in}}%
\pgfpathlineto{\pgfqpoint{1.623457in}{0.603766in}}%
\pgfpathlineto{\pgfqpoint{1.625036in}{0.599495in}}%
\pgfpathlineto{\pgfqpoint{1.625431in}{0.600092in}}%
\pgfpathlineto{\pgfqpoint{1.626221in}{0.598195in}}%
\pgfpathlineto{\pgfqpoint{1.627010in}{0.600004in}}%
\pgfpathlineto{\pgfqpoint{1.628194in}{0.602548in}}%
\pgfpathlineto{\pgfqpoint{1.628984in}{0.601479in}}%
\pgfpathlineto{\pgfqpoint{1.630563in}{0.610318in}}%
\pgfpathlineto{\pgfqpoint{1.632142in}{0.614473in}}%
\pgfpathlineto{\pgfqpoint{1.632536in}{0.614163in}}%
\pgfpathlineto{\pgfqpoint{1.634510in}{0.604336in}}%
\pgfpathlineto{\pgfqpoint{1.635300in}{0.605978in}}%
\pgfpathlineto{\pgfqpoint{1.636484in}{0.605670in}}%
\pgfpathlineto{\pgfqpoint{1.639642in}{0.598972in}}%
\pgfpathlineto{\pgfqpoint{1.642405in}{0.598276in}}%
\pgfpathlineto{\pgfqpoint{1.643195in}{0.600555in}}%
\pgfpathlineto{\pgfqpoint{1.643984in}{0.598735in}}%
\pgfpathlineto{\pgfqpoint{1.644379in}{0.598496in}}%
\pgfpathlineto{\pgfqpoint{1.644774in}{0.599244in}}%
\pgfpathlineto{\pgfqpoint{1.645958in}{0.600438in}}%
\pgfpathlineto{\pgfqpoint{1.647142in}{0.596879in}}%
\pgfpathlineto{\pgfqpoint{1.647537in}{0.598340in}}%
\pgfpathlineto{\pgfqpoint{1.650300in}{0.607842in}}%
\pgfpathlineto{\pgfqpoint{1.650695in}{0.607609in}}%
\pgfpathlineto{\pgfqpoint{1.651879in}{0.606227in}}%
\pgfpathlineto{\pgfqpoint{1.652274in}{0.607032in}}%
\pgfpathlineto{\pgfqpoint{1.652669in}{0.607512in}}%
\pgfpathlineto{\pgfqpoint{1.657406in}{0.584620in}}%
\pgfpathlineto{\pgfqpoint{1.657800in}{0.584866in}}%
\pgfpathlineto{\pgfqpoint{1.662143in}{0.604721in}}%
\pgfpathlineto{\pgfqpoint{1.663722in}{0.603532in}}%
\pgfpathlineto{\pgfqpoint{1.666485in}{0.600181in}}%
\pgfpathlineto{\pgfqpoint{1.668459in}{0.609543in}}%
\pgfpathlineto{\pgfqpoint{1.669643in}{0.607815in}}%
\pgfpathlineto{\pgfqpoint{1.672011in}{0.609125in}}%
\pgfpathlineto{\pgfqpoint{1.673196in}{0.611359in}}%
\pgfpathlineto{\pgfqpoint{1.673590in}{0.611085in}}%
\pgfpathlineto{\pgfqpoint{1.677143in}{0.598166in}}%
\pgfpathlineto{\pgfqpoint{1.679512in}{0.610220in}}%
\pgfpathlineto{\pgfqpoint{1.681485in}{0.607851in}}%
\pgfpathlineto{\pgfqpoint{1.683064in}{0.607979in}}%
\pgfpathlineto{\pgfqpoint{1.685433in}{0.615002in}}%
\pgfpathlineto{\pgfqpoint{1.686222in}{0.614174in}}%
\pgfpathlineto{\pgfqpoint{1.687012in}{0.613124in}}%
\pgfpathlineto{\pgfqpoint{1.687407in}{0.614312in}}%
\pgfpathlineto{\pgfqpoint{1.688986in}{0.617296in}}%
\pgfpathlineto{\pgfqpoint{1.689380in}{0.615955in}}%
\pgfpathlineto{\pgfqpoint{1.690170in}{0.613222in}}%
\pgfpathlineto{\pgfqpoint{1.690565in}{0.616155in}}%
\pgfpathlineto{\pgfqpoint{1.691354in}{0.620258in}}%
\pgfpathlineto{\pgfqpoint{1.692144in}{0.617841in}}%
\pgfpathlineto{\pgfqpoint{1.694512in}{0.607608in}}%
\pgfpathlineto{\pgfqpoint{1.694907in}{0.608214in}}%
\pgfpathlineto{\pgfqpoint{1.698460in}{0.618438in}}%
\pgfpathlineto{\pgfqpoint{1.698854in}{0.617844in}}%
\pgfpathlineto{\pgfqpoint{1.700433in}{0.623014in}}%
\pgfpathlineto{\pgfqpoint{1.700828in}{0.622566in}}%
\pgfpathlineto{\pgfqpoint{1.701223in}{0.621335in}}%
\pgfpathlineto{\pgfqpoint{1.702012in}{0.623543in}}%
\pgfpathlineto{\pgfqpoint{1.704776in}{0.626441in}}%
\pgfpathlineto{\pgfqpoint{1.705170in}{0.625801in}}%
\pgfpathlineto{\pgfqpoint{1.705565in}{0.630135in}}%
\pgfpathlineto{\pgfqpoint{1.706355in}{0.626257in}}%
\pgfpathlineto{\pgfqpoint{1.710302in}{0.620369in}}%
\pgfpathlineto{\pgfqpoint{1.713065in}{0.624300in}}%
\pgfpathlineto{\pgfqpoint{1.714249in}{0.622084in}}%
\pgfpathlineto{\pgfqpoint{1.717802in}{0.637714in}}%
\pgfpathlineto{\pgfqpoint{1.719776in}{0.639750in}}%
\pgfpathlineto{\pgfqpoint{1.721750in}{0.635803in}}%
\pgfpathlineto{\pgfqpoint{1.722144in}{0.636185in}}%
\pgfpathlineto{\pgfqpoint{1.725302in}{0.641603in}}%
\pgfpathlineto{\pgfqpoint{1.725697in}{0.641361in}}%
\pgfpathlineto{\pgfqpoint{1.726487in}{0.641022in}}%
\pgfpathlineto{\pgfqpoint{1.727671in}{0.646758in}}%
\pgfpathlineto{\pgfqpoint{1.728855in}{0.645875in}}%
\pgfpathlineto{\pgfqpoint{1.731224in}{0.635234in}}%
\pgfpathlineto{\pgfqpoint{1.731618in}{0.636442in}}%
\pgfpathlineto{\pgfqpoint{1.732013in}{0.635744in}}%
\pgfpathlineto{\pgfqpoint{1.732803in}{0.636908in}}%
\pgfpathlineto{\pgfqpoint{1.733197in}{0.637136in}}%
\pgfpathlineto{\pgfqpoint{1.733592in}{0.635854in}}%
\pgfpathlineto{\pgfqpoint{1.734776in}{0.633621in}}%
\pgfpathlineto{\pgfqpoint{1.735171in}{0.634671in}}%
\pgfpathlineto{\pgfqpoint{1.735566in}{0.636010in}}%
\pgfpathlineto{\pgfqpoint{1.735961in}{0.634099in}}%
\pgfpathlineto{\pgfqpoint{1.737540in}{0.631136in}}%
\pgfpathlineto{\pgfqpoint{1.738724in}{0.633026in}}%
\pgfpathlineto{\pgfqpoint{1.739119in}{0.632464in}}%
\pgfpathlineto{\pgfqpoint{1.739908in}{0.629539in}}%
\pgfpathlineto{\pgfqpoint{1.740698in}{0.631142in}}%
\pgfpathlineto{\pgfqpoint{1.741882in}{0.629299in}}%
\pgfpathlineto{\pgfqpoint{1.744250in}{0.622691in}}%
\pgfpathlineto{\pgfqpoint{1.744645in}{0.623476in}}%
\pgfpathlineto{\pgfqpoint{1.746619in}{0.627550in}}%
\pgfpathlineto{\pgfqpoint{1.747408in}{0.625107in}}%
\pgfpathlineto{\pgfqpoint{1.748987in}{0.621253in}}%
\pgfpathlineto{\pgfqpoint{1.753724in}{0.607969in}}%
\pgfpathlineto{\pgfqpoint{1.754909in}{0.609067in}}%
\pgfpathlineto{\pgfqpoint{1.755303in}{0.607441in}}%
\pgfpathlineto{\pgfqpoint{1.756882in}{0.604788in}}%
\pgfpathlineto{\pgfqpoint{1.759646in}{0.599803in}}%
\pgfpathlineto{\pgfqpoint{1.760040in}{0.600281in}}%
\pgfpathlineto{\pgfqpoint{1.765962in}{0.617788in}}%
\pgfpathlineto{\pgfqpoint{1.766356in}{0.618146in}}%
\pgfpathlineto{\pgfqpoint{1.766751in}{0.617016in}}%
\pgfpathlineto{\pgfqpoint{1.773067in}{0.610185in}}%
\pgfpathlineto{\pgfqpoint{1.773462in}{0.611223in}}%
\pgfpathlineto{\pgfqpoint{1.774646in}{0.615855in}}%
\pgfpathlineto{\pgfqpoint{1.775041in}{0.614417in}}%
\pgfpathlineto{\pgfqpoint{1.778199in}{0.609827in}}%
\pgfpathlineto{\pgfqpoint{1.778594in}{0.610552in}}%
\pgfpathlineto{\pgfqpoint{1.780567in}{0.612804in}}%
\pgfpathlineto{\pgfqpoint{1.780962in}{0.611869in}}%
\pgfpathlineto{\pgfqpoint{1.782936in}{0.608733in}}%
\pgfpathlineto{\pgfqpoint{1.784120in}{0.609897in}}%
\pgfpathlineto{\pgfqpoint{1.784910in}{0.611064in}}%
\pgfpathlineto{\pgfqpoint{1.785699in}{0.610319in}}%
\pgfpathlineto{\pgfqpoint{1.787278in}{0.607950in}}%
\pgfpathlineto{\pgfqpoint{1.790041in}{0.619361in}}%
\pgfpathlineto{\pgfqpoint{1.792410in}{0.610292in}}%
\pgfpathlineto{\pgfqpoint{1.793989in}{0.605445in}}%
\pgfpathlineto{\pgfqpoint{1.794778in}{0.605768in}}%
\pgfpathlineto{\pgfqpoint{1.795962in}{0.603402in}}%
\pgfpathlineto{\pgfqpoint{1.798726in}{0.590510in}}%
\pgfpathlineto{\pgfqpoint{1.799120in}{0.591634in}}%
\pgfpathlineto{\pgfqpoint{1.800305in}{0.589842in}}%
\pgfpathlineto{\pgfqpoint{1.800699in}{0.591155in}}%
\pgfpathlineto{\pgfqpoint{1.801884in}{0.594964in}}%
\pgfpathlineto{\pgfqpoint{1.802673in}{0.592889in}}%
\pgfpathlineto{\pgfqpoint{1.803068in}{0.591290in}}%
\pgfpathlineto{\pgfqpoint{1.803857in}{0.593794in}}%
\pgfpathlineto{\pgfqpoint{1.807015in}{0.598380in}}%
\pgfpathlineto{\pgfqpoint{1.807805in}{0.596139in}}%
\pgfpathlineto{\pgfqpoint{1.808594in}{0.597179in}}%
\pgfpathlineto{\pgfqpoint{1.810568in}{0.602760in}}%
\pgfpathlineto{\pgfqpoint{1.814121in}{0.593671in}}%
\pgfpathlineto{\pgfqpoint{1.814516in}{0.593803in}}%
\pgfpathlineto{\pgfqpoint{1.816095in}{0.588357in}}%
\pgfpathlineto{\pgfqpoint{1.816489in}{0.589529in}}%
\pgfpathlineto{\pgfqpoint{1.820437in}{0.606891in}}%
\pgfpathlineto{\pgfqpoint{1.820832in}{0.606583in}}%
\pgfpathlineto{\pgfqpoint{1.823200in}{0.601651in}}%
\pgfpathlineto{\pgfqpoint{1.823595in}{0.603389in}}%
\pgfpathlineto{\pgfqpoint{1.823990in}{0.605278in}}%
\pgfpathlineto{\pgfqpoint{1.824779in}{0.602448in}}%
\pgfpathlineto{\pgfqpoint{1.826753in}{0.593531in}}%
\pgfpathlineto{\pgfqpoint{1.828332in}{0.595802in}}%
\pgfpathlineto{\pgfqpoint{1.831490in}{0.591147in}}%
\pgfpathlineto{\pgfqpoint{1.831885in}{0.592249in}}%
\pgfpathlineto{\pgfqpoint{1.833858in}{0.599327in}}%
\pgfpathlineto{\pgfqpoint{1.834253in}{0.598414in}}%
\pgfpathlineto{\pgfqpoint{1.837016in}{0.604926in}}%
\pgfpathlineto{\pgfqpoint{1.838201in}{0.610564in}}%
\pgfpathlineto{\pgfqpoint{1.838990in}{0.607497in}}%
\pgfpathlineto{\pgfqpoint{1.840174in}{0.608102in}}%
\pgfpathlineto{\pgfqpoint{1.840569in}{0.608523in}}%
\pgfpathlineto{\pgfqpoint{1.840964in}{0.607700in}}%
\pgfpathlineto{\pgfqpoint{1.845701in}{0.595091in}}%
\pgfpathlineto{\pgfqpoint{1.846096in}{0.595758in}}%
\pgfpathlineto{\pgfqpoint{1.846490in}{0.594016in}}%
\pgfpathlineto{\pgfqpoint{1.850438in}{0.580360in}}%
\pgfpathlineto{\pgfqpoint{1.850833in}{0.582682in}}%
\pgfpathlineto{\pgfqpoint{1.852017in}{0.580604in}}%
\pgfpathlineto{\pgfqpoint{1.852412in}{0.580738in}}%
\pgfpathlineto{\pgfqpoint{1.856754in}{0.600366in}}%
\pgfpathlineto{\pgfqpoint{1.862280in}{0.587261in}}%
\pgfpathlineto{\pgfqpoint{1.862675in}{0.588906in}}%
\pgfpathlineto{\pgfqpoint{1.865044in}{0.594407in}}%
\pgfpathlineto{\pgfqpoint{1.865438in}{0.594275in}}%
\pgfpathlineto{\pgfqpoint{1.867807in}{0.599453in}}%
\pgfpathlineto{\pgfqpoint{1.868202in}{0.598780in}}%
\pgfpathlineto{\pgfqpoint{1.870175in}{0.590126in}}%
\pgfpathlineto{\pgfqpoint{1.870570in}{0.590913in}}%
\pgfpathlineto{\pgfqpoint{1.870965in}{0.592062in}}%
\pgfpathlineto{\pgfqpoint{1.871754in}{0.590327in}}%
\pgfpathlineto{\pgfqpoint{1.872149in}{0.590820in}}%
\pgfpathlineto{\pgfqpoint{1.873728in}{0.582903in}}%
\pgfpathlineto{\pgfqpoint{1.874123in}{0.584529in}}%
\pgfpathlineto{\pgfqpoint{1.875702in}{0.595108in}}%
\pgfpathlineto{\pgfqpoint{1.876097in}{0.592257in}}%
\pgfpathlineto{\pgfqpoint{1.877281in}{0.584649in}}%
\pgfpathlineto{\pgfqpoint{1.878070in}{0.585853in}}%
\pgfpathlineto{\pgfqpoint{1.881228in}{0.595856in}}%
\pgfpathlineto{\pgfqpoint{1.882018in}{0.594810in}}%
\pgfpathlineto{\pgfqpoint{1.883597in}{0.594285in}}%
\pgfpathlineto{\pgfqpoint{1.883991in}{0.595067in}}%
\pgfpathlineto{\pgfqpoint{1.885570in}{0.596708in}}%
\pgfpathlineto{\pgfqpoint{1.886360in}{0.594945in}}%
\pgfpathlineto{\pgfqpoint{1.886755in}{0.595762in}}%
\pgfpathlineto{\pgfqpoint{1.890702in}{0.602208in}}%
\pgfpathlineto{\pgfqpoint{1.891492in}{0.601402in}}%
\pgfpathlineto{\pgfqpoint{1.895044in}{0.587364in}}%
\pgfpathlineto{\pgfqpoint{1.896623in}{0.589570in}}%
\pgfpathlineto{\pgfqpoint{1.897018in}{0.588483in}}%
\pgfpathlineto{\pgfqpoint{1.897413in}{0.588467in}}%
\pgfpathlineto{\pgfqpoint{1.901360in}{0.599138in}}%
\pgfpathlineto{\pgfqpoint{1.904913in}{0.589737in}}%
\pgfpathlineto{\pgfqpoint{1.905703in}{0.591069in}}%
\pgfpathlineto{\pgfqpoint{1.906492in}{0.594327in}}%
\pgfpathlineto{\pgfqpoint{1.906887in}{0.593574in}}%
\pgfpathlineto{\pgfqpoint{1.909650in}{0.585374in}}%
\pgfpathlineto{\pgfqpoint{1.910440in}{0.586076in}}%
\pgfpathlineto{\pgfqpoint{1.911624in}{0.591521in}}%
\pgfpathlineto{\pgfqpoint{1.912808in}{0.603335in}}%
\pgfpathlineto{\pgfqpoint{1.913598in}{0.600522in}}%
\pgfpathlineto{\pgfqpoint{1.914387in}{0.596309in}}%
\pgfpathlineto{\pgfqpoint{1.915177in}{0.599421in}}%
\pgfpathlineto{\pgfqpoint{1.915571in}{0.600685in}}%
\pgfpathlineto{\pgfqpoint{1.916361in}{0.598748in}}%
\pgfpathlineto{\pgfqpoint{1.919914in}{0.594318in}}%
\pgfpathlineto{\pgfqpoint{1.920308in}{0.594831in}}%
\pgfpathlineto{\pgfqpoint{1.922282in}{0.596095in}}%
\pgfpathlineto{\pgfqpoint{1.923466in}{0.591980in}}%
\pgfpathlineto{\pgfqpoint{1.924256in}{0.594346in}}%
\pgfpathlineto{\pgfqpoint{1.924651in}{0.594531in}}%
\pgfpathlineto{\pgfqpoint{1.927809in}{0.584520in}}%
\pgfpathlineto{\pgfqpoint{1.928203in}{0.585135in}}%
\pgfpathlineto{\pgfqpoint{1.930572in}{0.594575in}}%
\pgfpathlineto{\pgfqpoint{1.930967in}{0.594032in}}%
\pgfpathlineto{\pgfqpoint{1.931756in}{0.594717in}}%
\pgfpathlineto{\pgfqpoint{1.934519in}{0.601324in}}%
\pgfpathlineto{\pgfqpoint{1.935704in}{0.602020in}}%
\pgfpathlineto{\pgfqpoint{1.941230in}{0.619580in}}%
\pgfpathlineto{\pgfqpoint{1.942020in}{0.616544in}}%
\pgfpathlineto{\pgfqpoint{1.942809in}{0.619414in}}%
\pgfpathlineto{\pgfqpoint{1.943599in}{0.617435in}}%
\pgfpathlineto{\pgfqpoint{1.946362in}{0.604130in}}%
\pgfpathlineto{\pgfqpoint{1.946757in}{0.605181in}}%
\pgfpathlineto{\pgfqpoint{1.947546in}{0.604261in}}%
\pgfpathlineto{\pgfqpoint{1.953467in}{0.585996in}}%
\pgfpathlineto{\pgfqpoint{1.954257in}{0.586557in}}%
\pgfpathlineto{\pgfqpoint{1.956625in}{0.595255in}}%
\pgfpathlineto{\pgfqpoint{1.957415in}{0.593031in}}%
\pgfpathlineto{\pgfqpoint{1.958204in}{0.592992in}}%
\pgfpathlineto{\pgfqpoint{1.958599in}{0.591958in}}%
\pgfpathlineto{\pgfqpoint{1.960178in}{0.583475in}}%
\pgfpathlineto{\pgfqpoint{1.960573in}{0.585218in}}%
\pgfpathlineto{\pgfqpoint{1.962152in}{0.587842in}}%
\pgfpathlineto{\pgfqpoint{1.962546in}{0.586966in}}%
\pgfpathlineto{\pgfqpoint{1.964125in}{0.583074in}}%
\pgfpathlineto{\pgfqpoint{1.964520in}{0.585198in}}%
\pgfpathlineto{\pgfqpoint{1.965310in}{0.587465in}}%
\pgfpathlineto{\pgfqpoint{1.966099in}{0.586685in}}%
\pgfpathlineto{\pgfqpoint{1.966494in}{0.584738in}}%
\pgfpathlineto{\pgfqpoint{1.967283in}{0.587464in}}%
\pgfpathlineto{\pgfqpoint{1.968468in}{0.591798in}}%
\pgfpathlineto{\pgfqpoint{1.968862in}{0.589606in}}%
\pgfpathlineto{\pgfqpoint{1.971626in}{0.582275in}}%
\pgfpathlineto{\pgfqpoint{1.977152in}{0.612774in}}%
\pgfpathlineto{\pgfqpoint{1.977547in}{0.611895in}}%
\pgfpathlineto{\pgfqpoint{1.978336in}{0.610617in}}%
\pgfpathlineto{\pgfqpoint{1.980310in}{0.607775in}}%
\pgfpathlineto{\pgfqpoint{1.982284in}{0.614825in}}%
\pgfpathlineto{\pgfqpoint{1.982679in}{0.614218in}}%
\pgfpathlineto{\pgfqpoint{1.985047in}{0.610156in}}%
\pgfpathlineto{\pgfqpoint{1.985442in}{0.611070in}}%
\pgfpathlineto{\pgfqpoint{1.986231in}{0.611466in}}%
\pgfpathlineto{\pgfqpoint{1.988600in}{0.623227in}}%
\pgfpathlineto{\pgfqpoint{1.989784in}{0.619360in}}%
\pgfpathlineto{\pgfqpoint{1.991363in}{0.612171in}}%
\pgfpathlineto{\pgfqpoint{1.991758in}{0.614470in}}%
\pgfpathlineto{\pgfqpoint{1.993337in}{0.622207in}}%
\pgfpathlineto{\pgfqpoint{1.994126in}{0.621493in}}%
\pgfpathlineto{\pgfqpoint{1.994521in}{0.621065in}}%
\pgfpathlineto{\pgfqpoint{1.994916in}{0.622467in}}%
\pgfpathlineto{\pgfqpoint{1.995311in}{0.623546in}}%
\pgfpathlineto{\pgfqpoint{1.995705in}{0.622018in}}%
\pgfpathlineto{\pgfqpoint{1.997679in}{0.616549in}}%
\pgfpathlineto{\pgfqpoint{1.998863in}{0.618245in}}%
\pgfpathlineto{\pgfqpoint{2.000837in}{0.615248in}}%
\pgfpathlineto{\pgfqpoint{2.002021in}{0.617141in}}%
\pgfpathlineto{\pgfqpoint{2.002416in}{0.616131in}}%
\pgfpathlineto{\pgfqpoint{2.004785in}{0.613075in}}%
\pgfpathlineto{\pgfqpoint{2.005179in}{0.614007in}}%
\pgfpathlineto{\pgfqpoint{2.005574in}{0.614456in}}%
\pgfpathlineto{\pgfqpoint{2.007943in}{0.622512in}}%
\pgfpathlineto{\pgfqpoint{2.009916in}{0.624823in}}%
\pgfpathlineto{\pgfqpoint{2.011495in}{0.625791in}}%
\pgfpathlineto{\pgfqpoint{2.015048in}{0.615078in}}%
\pgfpathlineto{\pgfqpoint{2.017811in}{0.607043in}}%
\pgfpathlineto{\pgfqpoint{2.020180in}{0.605320in}}%
\pgfpathlineto{\pgfqpoint{2.020575in}{0.605379in}}%
\pgfpathlineto{\pgfqpoint{2.022548in}{0.601708in}}%
\pgfpathlineto{\pgfqpoint{2.024522in}{0.609137in}}%
\pgfpathlineto{\pgfqpoint{2.024917in}{0.607220in}}%
\pgfpathlineto{\pgfqpoint{2.029654in}{0.584730in}}%
\pgfpathlineto{\pgfqpoint{2.030443in}{0.587693in}}%
\pgfpathlineto{\pgfqpoint{2.033207in}{0.604310in}}%
\pgfpathlineto{\pgfqpoint{2.035970in}{0.590560in}}%
\pgfpathlineto{\pgfqpoint{2.036759in}{0.593653in}}%
\pgfpathlineto{\pgfqpoint{2.045838in}{0.636886in}}%
\pgfpathlineto{\pgfqpoint{2.046233in}{0.635046in}}%
\pgfpathlineto{\pgfqpoint{2.050181in}{0.621444in}}%
\pgfpathlineto{\pgfqpoint{2.050575in}{0.621894in}}%
\pgfpathlineto{\pgfqpoint{2.051760in}{0.615181in}}%
\pgfpathlineto{\pgfqpoint{2.052549in}{0.617160in}}%
\pgfpathlineto{\pgfqpoint{2.055707in}{0.629208in}}%
\pgfpathlineto{\pgfqpoint{2.056102in}{0.628230in}}%
\pgfpathlineto{\pgfqpoint{2.060839in}{0.612539in}}%
\pgfpathlineto{\pgfqpoint{2.061628in}{0.615396in}}%
\pgfpathlineto{\pgfqpoint{2.062813in}{0.614311in}}%
\pgfpathlineto{\pgfqpoint{2.063997in}{0.615523in}}%
\pgfpathlineto{\pgfqpoint{2.064392in}{0.616166in}}%
\pgfpathlineto{\pgfqpoint{2.064786in}{0.615142in}}%
\pgfpathlineto{\pgfqpoint{2.067944in}{0.605666in}}%
\pgfpathlineto{\pgfqpoint{2.070313in}{0.599946in}}%
\pgfpathlineto{\pgfqpoint{2.070708in}{0.600613in}}%
\pgfpathlineto{\pgfqpoint{2.071892in}{0.601619in}}%
\pgfpathlineto{\pgfqpoint{2.072681in}{0.604781in}}%
\pgfpathlineto{\pgfqpoint{2.073471in}{0.603131in}}%
\pgfpathlineto{\pgfqpoint{2.074655in}{0.604871in}}%
\pgfpathlineto{\pgfqpoint{2.077418in}{0.615860in}}%
\pgfpathlineto{\pgfqpoint{2.078208in}{0.612446in}}%
\pgfpathlineto{\pgfqpoint{2.079392in}{0.614718in}}%
\pgfpathlineto{\pgfqpoint{2.081761in}{0.626091in}}%
\pgfpathlineto{\pgfqpoint{2.082155in}{0.625948in}}%
\pgfpathlineto{\pgfqpoint{2.084919in}{0.624379in}}%
\pgfpathlineto{\pgfqpoint{2.087287in}{0.628385in}}%
\pgfpathlineto{\pgfqpoint{2.087682in}{0.628046in}}%
\pgfpathlineto{\pgfqpoint{2.088866in}{0.626366in}}%
\pgfpathlineto{\pgfqpoint{2.089261in}{0.627420in}}%
\pgfpathlineto{\pgfqpoint{2.090445in}{0.627197in}}%
\pgfpathlineto{\pgfqpoint{2.094787in}{0.612593in}}%
\pgfpathlineto{\pgfqpoint{2.095577in}{0.615236in}}%
\pgfpathlineto{\pgfqpoint{2.097945in}{0.614321in}}%
\pgfpathlineto{\pgfqpoint{2.100709in}{0.603493in}}%
\pgfpathlineto{\pgfqpoint{2.101103in}{0.605812in}}%
\pgfpathlineto{\pgfqpoint{2.104261in}{0.612791in}}%
\pgfpathlineto{\pgfqpoint{2.108998in}{0.609981in}}%
\pgfpathlineto{\pgfqpoint{2.109788in}{0.608379in}}%
\pgfpathlineto{\pgfqpoint{2.110577in}{0.609413in}}%
\pgfpathlineto{\pgfqpoint{2.112156in}{0.612944in}}%
\pgfpathlineto{\pgfqpoint{2.112551in}{0.610306in}}%
\pgfpathlineto{\pgfqpoint{2.114130in}{0.604320in}}%
\pgfpathlineto{\pgfqpoint{2.114525in}{0.604522in}}%
\pgfpathlineto{\pgfqpoint{2.114920in}{0.606324in}}%
\pgfpathlineto{\pgfqpoint{2.115709in}{0.603277in}}%
\pgfpathlineto{\pgfqpoint{2.117288in}{0.599119in}}%
\pgfpathlineto{\pgfqpoint{2.117683in}{0.599395in}}%
\pgfpathlineto{\pgfqpoint{2.118472in}{0.600634in}}%
\pgfpathlineto{\pgfqpoint{2.119657in}{0.603831in}}%
\pgfpathlineto{\pgfqpoint{2.120051in}{0.602164in}}%
\pgfpathlineto{\pgfqpoint{2.122815in}{0.597084in}}%
\pgfpathlineto{\pgfqpoint{2.123999in}{0.601597in}}%
\pgfpathlineto{\pgfqpoint{2.124788in}{0.601200in}}%
\pgfpathlineto{\pgfqpoint{2.126762in}{0.595480in}}%
\pgfpathlineto{\pgfqpoint{2.127157in}{0.597733in}}%
\pgfpathlineto{\pgfqpoint{2.129130in}{0.599827in}}%
\pgfpathlineto{\pgfqpoint{2.129920in}{0.600943in}}%
\pgfpathlineto{\pgfqpoint{2.133473in}{0.629971in}}%
\pgfpathlineto{\pgfqpoint{2.134657in}{0.638605in}}%
\pgfpathlineto{\pgfqpoint{2.135446in}{0.634729in}}%
\pgfpathlineto{\pgfqpoint{2.139789in}{0.610961in}}%
\pgfpathlineto{\pgfqpoint{2.142947in}{0.602683in}}%
\pgfpathlineto{\pgfqpoint{2.143341in}{0.603600in}}%
\pgfpathlineto{\pgfqpoint{2.146105in}{0.610057in}}%
\pgfpathlineto{\pgfqpoint{2.149263in}{0.599047in}}%
\pgfpathlineto{\pgfqpoint{2.151236in}{0.599713in}}%
\pgfpathlineto{\pgfqpoint{2.152026in}{0.600273in}}%
\pgfpathlineto{\pgfqpoint{2.154000in}{0.590084in}}%
\pgfpathlineto{\pgfqpoint{2.154394in}{0.591116in}}%
\pgfpathlineto{\pgfqpoint{2.157158in}{0.596585in}}%
\pgfpathlineto{\pgfqpoint{2.159526in}{0.597015in}}%
\pgfpathlineto{\pgfqpoint{2.161500in}{0.591083in}}%
\pgfpathlineto{\pgfqpoint{2.162289in}{0.591485in}}%
\pgfpathlineto{\pgfqpoint{2.162684in}{0.592242in}}%
\pgfpathlineto{\pgfqpoint{2.163474in}{0.591129in}}%
\pgfpathlineto{\pgfqpoint{2.165053in}{0.590866in}}%
\pgfpathlineto{\pgfqpoint{2.165447in}{0.591932in}}%
\pgfpathlineto{\pgfqpoint{2.168211in}{0.598613in}}%
\pgfpathlineto{\pgfqpoint{2.169000in}{0.599354in}}%
\pgfpathlineto{\pgfqpoint{2.169790in}{0.601806in}}%
\pgfpathlineto{\pgfqpoint{2.170184in}{0.601437in}}%
\pgfpathlineto{\pgfqpoint{2.170579in}{0.598547in}}%
\pgfpathlineto{\pgfqpoint{2.171369in}{0.602020in}}%
\pgfpathlineto{\pgfqpoint{2.176500in}{0.616237in}}%
\pgfpathlineto{\pgfqpoint{2.177685in}{0.617799in}}%
\pgfpathlineto{\pgfqpoint{2.181632in}{0.627103in}}%
\pgfpathlineto{\pgfqpoint{2.182816in}{0.630314in}}%
\pgfpathlineto{\pgfqpoint{2.183211in}{0.628897in}}%
\pgfpathlineto{\pgfqpoint{2.184395in}{0.625155in}}%
\pgfpathlineto{\pgfqpoint{2.184790in}{0.626632in}}%
\pgfpathlineto{\pgfqpoint{2.186369in}{0.633262in}}%
\pgfpathlineto{\pgfqpoint{2.187159in}{0.629064in}}%
\pgfpathlineto{\pgfqpoint{2.188343in}{0.625955in}}%
\pgfpathlineto{\pgfqpoint{2.189132in}{0.627080in}}%
\pgfpathlineto{\pgfqpoint{2.189922in}{0.628467in}}%
\pgfpathlineto{\pgfqpoint{2.195054in}{0.639061in}}%
\pgfpathlineto{\pgfqpoint{2.195843in}{0.636440in}}%
\pgfpathlineto{\pgfqpoint{2.199396in}{0.631452in}}%
\pgfpathlineto{\pgfqpoint{2.202159in}{0.621039in}}%
\pgfpathlineto{\pgfqpoint{2.206501in}{0.629055in}}%
\pgfpathlineto{\pgfqpoint{2.207686in}{0.626125in}}%
\pgfpathlineto{\pgfqpoint{2.208475in}{0.627296in}}%
\pgfpathlineto{\pgfqpoint{2.209659in}{0.626887in}}%
\pgfpathlineto{\pgfqpoint{2.210844in}{0.622498in}}%
\pgfpathlineto{\pgfqpoint{2.211633in}{0.624609in}}%
\pgfpathlineto{\pgfqpoint{2.214396in}{0.633233in}}%
\pgfpathlineto{\pgfqpoint{2.214791in}{0.632981in}}%
\pgfpathlineto{\pgfqpoint{2.215580in}{0.633378in}}%
\pgfpathlineto{\pgfqpoint{2.215975in}{0.631959in}}%
\pgfpathlineto{\pgfqpoint{2.216370in}{0.630911in}}%
\pgfpathlineto{\pgfqpoint{2.217554in}{0.631970in}}%
\pgfpathlineto{\pgfqpoint{2.219923in}{0.638775in}}%
\pgfpathlineto{\pgfqpoint{2.220317in}{0.637432in}}%
\pgfpathlineto{\pgfqpoint{2.221896in}{0.635069in}}%
\pgfpathlineto{\pgfqpoint{2.222291in}{0.635296in}}%
\pgfpathlineto{\pgfqpoint{2.223870in}{0.638250in}}%
\pgfpathlineto{\pgfqpoint{2.224660in}{0.637381in}}%
\pgfpathlineto{\pgfqpoint{2.227028in}{0.630651in}}%
\pgfpathlineto{\pgfqpoint{2.227423in}{0.630749in}}%
\pgfpathlineto{\pgfqpoint{2.228212in}{0.630305in}}%
\pgfpathlineto{\pgfqpoint{2.229002in}{0.632903in}}%
\pgfpathlineto{\pgfqpoint{2.229791in}{0.630661in}}%
\pgfpathlineto{\pgfqpoint{2.233344in}{0.629208in}}%
\pgfpathlineto{\pgfqpoint{2.237292in}{0.621751in}}%
\pgfpathlineto{\pgfqpoint{2.237686in}{0.622308in}}%
\pgfpathlineto{\pgfqpoint{2.238081in}{0.624439in}}%
\pgfpathlineto{\pgfqpoint{2.238871in}{0.622007in}}%
\pgfpathlineto{\pgfqpoint{2.244002in}{0.603093in}}%
\pgfpathlineto{\pgfqpoint{2.247160in}{0.603541in}}%
\pgfpathlineto{\pgfqpoint{2.248345in}{0.608342in}}%
\pgfpathlineto{\pgfqpoint{2.249134in}{0.607193in}}%
\pgfpathlineto{\pgfqpoint{2.251108in}{0.604153in}}%
\pgfpathlineto{\pgfqpoint{2.251503in}{0.604929in}}%
\pgfpathlineto{\pgfqpoint{2.251897in}{0.605909in}}%
\pgfpathlineto{\pgfqpoint{2.254266in}{0.596689in}}%
\pgfpathlineto{\pgfqpoint{2.255055in}{0.601011in}}%
\pgfpathlineto{\pgfqpoint{2.255845in}{0.599292in}}%
\pgfpathlineto{\pgfqpoint{2.257819in}{0.594633in}}%
\pgfpathlineto{\pgfqpoint{2.258213in}{0.596067in}}%
\pgfpathlineto{\pgfqpoint{2.261766in}{0.614486in}}%
\pgfpathlineto{\pgfqpoint{2.262950in}{0.611622in}}%
\pgfpathlineto{\pgfqpoint{2.266898in}{0.595492in}}%
\pgfpathlineto{\pgfqpoint{2.268872in}{0.589441in}}%
\pgfpathlineto{\pgfqpoint{2.269661in}{0.589768in}}%
\pgfpathlineto{\pgfqpoint{2.271635in}{0.597750in}}%
\pgfpathlineto{\pgfqpoint{2.274793in}{0.590790in}}%
\pgfpathlineto{\pgfqpoint{2.275188in}{0.591747in}}%
\pgfpathlineto{\pgfqpoint{2.277556in}{0.595728in}}%
\pgfpathlineto{\pgfqpoint{2.278740in}{0.594671in}}%
\pgfpathlineto{\pgfqpoint{2.281898in}{0.582187in}}%
\pgfpathlineto{\pgfqpoint{2.282688in}{0.583430in}}%
\pgfpathlineto{\pgfqpoint{2.285056in}{0.587620in}}%
\pgfpathlineto{\pgfqpoint{2.287030in}{0.593093in}}%
\pgfpathlineto{\pgfqpoint{2.289004in}{0.586947in}}%
\pgfpathlineto{\pgfqpoint{2.291767in}{0.583312in}}%
\pgfpathlineto{\pgfqpoint{2.292162in}{0.583771in}}%
\pgfpathlineto{\pgfqpoint{2.294136in}{0.590756in}}%
\pgfpathlineto{\pgfqpoint{2.297293in}{0.581265in}}%
\pgfpathlineto{\pgfqpoint{2.297688in}{0.581702in}}%
\pgfpathlineto{\pgfqpoint{2.300451in}{0.588201in}}%
\pgfpathlineto{\pgfqpoint{2.301241in}{0.587413in}}%
\pgfpathlineto{\pgfqpoint{2.302820in}{0.587436in}}%
\pgfpathlineto{\pgfqpoint{2.303215in}{0.587320in}}%
\pgfpathlineto{\pgfqpoint{2.307557in}{0.571568in}}%
\pgfpathlineto{\pgfqpoint{2.309531in}{0.575005in}}%
\pgfpathlineto{\pgfqpoint{2.310320in}{0.573358in}}%
\pgfpathlineto{\pgfqpoint{2.312294in}{0.575857in}}%
\pgfpathlineto{\pgfqpoint{2.312689in}{0.576225in}}%
\pgfpathlineto{\pgfqpoint{2.313083in}{0.575399in}}%
\pgfpathlineto{\pgfqpoint{2.315847in}{0.569735in}}%
\pgfpathlineto{\pgfqpoint{2.319399in}{0.586598in}}%
\pgfpathlineto{\pgfqpoint{2.320189in}{0.579670in}}%
\pgfpathlineto{\pgfqpoint{2.321768in}{0.581575in}}%
\pgfpathlineto{\pgfqpoint{2.324136in}{0.583653in}}%
\pgfpathlineto{\pgfqpoint{2.325321in}{0.585576in}}%
\pgfpathlineto{\pgfqpoint{2.325715in}{0.586917in}}%
\pgfpathlineto{\pgfqpoint{2.326900in}{0.585495in}}%
\pgfpathlineto{\pgfqpoint{2.329663in}{0.577676in}}%
\pgfpathlineto{\pgfqpoint{2.330058in}{0.577821in}}%
\pgfpathlineto{\pgfqpoint{2.332031in}{0.585076in}}%
\pgfpathlineto{\pgfqpoint{2.332821in}{0.583645in}}%
\pgfpathlineto{\pgfqpoint{2.334400in}{0.579279in}}%
\pgfpathlineto{\pgfqpoint{2.335189in}{0.581708in}}%
\pgfpathlineto{\pgfqpoint{2.336374in}{0.582116in}}%
\pgfpathlineto{\pgfqpoint{2.336768in}{0.579703in}}%
\pgfpathlineto{\pgfqpoint{2.337558in}{0.581025in}}%
\pgfpathlineto{\pgfqpoint{2.341505in}{0.593640in}}%
\pgfpathlineto{\pgfqpoint{2.343479in}{0.595026in}}%
\pgfpathlineto{\pgfqpoint{2.346637in}{0.582481in}}%
\pgfpathlineto{\pgfqpoint{2.347821in}{0.583894in}}%
\pgfpathlineto{\pgfqpoint{2.348216in}{0.584533in}}%
\pgfpathlineto{\pgfqpoint{2.348611in}{0.583318in}}%
\pgfpathlineto{\pgfqpoint{2.349795in}{0.580057in}}%
\pgfpathlineto{\pgfqpoint{2.350979in}{0.581880in}}%
\pgfpathlineto{\pgfqpoint{2.351374in}{0.582168in}}%
\pgfpathlineto{\pgfqpoint{2.353348in}{0.588952in}}%
\pgfpathlineto{\pgfqpoint{2.353743in}{0.588377in}}%
\pgfpathlineto{\pgfqpoint{2.355322in}{0.584786in}}%
\pgfpathlineto{\pgfqpoint{2.356111in}{0.582817in}}%
\pgfpathlineto{\pgfqpoint{2.356506in}{0.585070in}}%
\pgfpathlineto{\pgfqpoint{2.358085in}{0.589903in}}%
\pgfpathlineto{\pgfqpoint{2.358874in}{0.588253in}}%
\pgfpathlineto{\pgfqpoint{2.362032in}{0.579963in}}%
\pgfpathlineto{\pgfqpoint{2.363611in}{0.580940in}}%
\pgfpathlineto{\pgfqpoint{2.365980in}{0.590024in}}%
\pgfpathlineto{\pgfqpoint{2.366375in}{0.589951in}}%
\pgfpathlineto{\pgfqpoint{2.367954in}{0.591760in}}%
\pgfpathlineto{\pgfqpoint{2.368348in}{0.591469in}}%
\pgfpathlineto{\pgfqpoint{2.370717in}{0.590396in}}%
\pgfpathlineto{\pgfqpoint{2.371901in}{0.590930in}}%
\pgfpathlineto{\pgfqpoint{2.375454in}{0.579886in}}%
\pgfpathlineto{\pgfqpoint{2.375849in}{0.581171in}}%
\pgfpathlineto{\pgfqpoint{2.377428in}{0.588550in}}%
\pgfpathlineto{\pgfqpoint{2.377822in}{0.585274in}}%
\pgfpathlineto{\pgfqpoint{2.378217in}{0.584226in}}%
\pgfpathlineto{\pgfqpoint{2.379006in}{0.585552in}}%
\pgfpathlineto{\pgfqpoint{2.379796in}{0.589726in}}%
\pgfpathlineto{\pgfqpoint{2.380585in}{0.587331in}}%
\pgfpathlineto{\pgfqpoint{2.382954in}{0.572800in}}%
\pgfpathlineto{\pgfqpoint{2.384533in}{0.574239in}}%
\pgfpathlineto{\pgfqpoint{2.385717in}{0.577072in}}%
\pgfpathlineto{\pgfqpoint{2.386112in}{0.576192in}}%
\pgfpathlineto{\pgfqpoint{2.388086in}{0.570854in}}%
\pgfpathlineto{\pgfqpoint{2.388480in}{0.571000in}}%
\pgfpathlineto{\pgfqpoint{2.389665in}{0.576934in}}%
\pgfpathlineto{\pgfqpoint{2.391638in}{0.586683in}}%
\pgfpathlineto{\pgfqpoint{2.394402in}{0.584258in}}%
\pgfpathlineto{\pgfqpoint{2.395586in}{0.590772in}}%
\pgfpathlineto{\pgfqpoint{2.395981in}{0.589709in}}%
\pgfpathlineto{\pgfqpoint{2.398349in}{0.584082in}}%
\pgfpathlineto{\pgfqpoint{2.402297in}{0.587317in}}%
\pgfpathlineto{\pgfqpoint{2.402691in}{0.584609in}}%
\pgfpathlineto{\pgfqpoint{2.403481in}{0.587717in}}%
\pgfpathlineto{\pgfqpoint{2.405060in}{0.597424in}}%
\pgfpathlineto{\pgfqpoint{2.406639in}{0.606447in}}%
\pgfpathlineto{\pgfqpoint{2.407428in}{0.605530in}}%
\pgfpathlineto{\pgfqpoint{2.408218in}{0.605820in}}%
\pgfpathlineto{\pgfqpoint{2.408613in}{0.604605in}}%
\pgfpathlineto{\pgfqpoint{2.410981in}{0.595557in}}%
\pgfpathlineto{\pgfqpoint{2.411376in}{0.597479in}}%
\pgfpathlineto{\pgfqpoint{2.414534in}{0.605006in}}%
\pgfpathlineto{\pgfqpoint{2.417692in}{0.608108in}}%
\pgfpathlineto{\pgfqpoint{2.418481in}{0.605551in}}%
\pgfpathlineto{\pgfqpoint{2.420455in}{0.597632in}}%
\pgfpathlineto{\pgfqpoint{2.420850in}{0.597849in}}%
\pgfpathlineto{\pgfqpoint{2.421245in}{0.598051in}}%
\pgfpathlineto{\pgfqpoint{2.422429in}{0.602652in}}%
\pgfpathlineto{\pgfqpoint{2.422824in}{0.599572in}}%
\pgfpathlineto{\pgfqpoint{2.424008in}{0.590713in}}%
\pgfpathlineto{\pgfqpoint{2.424797in}{0.595359in}}%
\pgfpathlineto{\pgfqpoint{2.426376in}{0.602599in}}%
\pgfpathlineto{\pgfqpoint{2.426771in}{0.601896in}}%
\pgfpathlineto{\pgfqpoint{2.427955in}{0.599323in}}%
\pgfpathlineto{\pgfqpoint{2.431508in}{0.593355in}}%
\pgfpathlineto{\pgfqpoint{2.431903in}{0.593541in}}%
\pgfpathlineto{\pgfqpoint{2.433482in}{0.595289in}}%
\pgfpathlineto{\pgfqpoint{2.436245in}{0.601532in}}%
\pgfpathlineto{\pgfqpoint{2.440193in}{0.606078in}}%
\pgfpathlineto{\pgfqpoint{2.440982in}{0.601969in}}%
\pgfpathlineto{\pgfqpoint{2.441772in}{0.597353in}}%
\pgfpathlineto{\pgfqpoint{2.442561in}{0.599446in}}%
\pgfpathlineto{\pgfqpoint{2.444535in}{0.607043in}}%
\pgfpathlineto{\pgfqpoint{2.444930in}{0.606103in}}%
\pgfpathlineto{\pgfqpoint{2.449272in}{0.589577in}}%
\pgfpathlineto{\pgfqpoint{2.453614in}{0.580525in}}%
\pgfpathlineto{\pgfqpoint{2.455588in}{0.585911in}}%
\pgfpathlineto{\pgfqpoint{2.455983in}{0.585198in}}%
\pgfpathlineto{\pgfqpoint{2.457167in}{0.584153in}}%
\pgfpathlineto{\pgfqpoint{2.460325in}{0.591702in}}%
\pgfpathlineto{\pgfqpoint{2.462693in}{0.586485in}}%
\pgfpathlineto{\pgfqpoint{2.464667in}{0.589465in}}%
\pgfpathlineto{\pgfqpoint{2.465062in}{0.588702in}}%
\pgfpathlineto{\pgfqpoint{2.466641in}{0.583803in}}%
\pgfpathlineto{\pgfqpoint{2.467430in}{0.584858in}}%
\pgfpathlineto{\pgfqpoint{2.467825in}{0.584941in}}%
\pgfpathlineto{\pgfqpoint{2.470983in}{0.594111in}}%
\pgfpathlineto{\pgfqpoint{2.472562in}{0.597543in}}%
\pgfpathlineto{\pgfqpoint{2.474141in}{0.600521in}}%
\pgfpathlineto{\pgfqpoint{2.474536in}{0.598903in}}%
\pgfpathlineto{\pgfqpoint{2.474930in}{0.598462in}}%
\pgfpathlineto{\pgfqpoint{2.475325in}{0.599231in}}%
\pgfpathlineto{\pgfqpoint{2.476904in}{0.602480in}}%
\pgfpathlineto{\pgfqpoint{2.478088in}{0.597148in}}%
\pgfpathlineto{\pgfqpoint{2.478878in}{0.598436in}}%
\pgfpathlineto{\pgfqpoint{2.479667in}{0.600257in}}%
\pgfpathlineto{\pgfqpoint{2.480062in}{0.599406in}}%
\pgfpathlineto{\pgfqpoint{2.481641in}{0.593826in}}%
\pgfpathlineto{\pgfqpoint{2.482431in}{0.594757in}}%
\pgfpathlineto{\pgfqpoint{2.484010in}{0.596603in}}%
\pgfpathlineto{\pgfqpoint{2.486773in}{0.605878in}}%
\pgfpathlineto{\pgfqpoint{2.487168in}{0.604914in}}%
\pgfpathlineto{\pgfqpoint{2.487562in}{0.605723in}}%
\pgfpathlineto{\pgfqpoint{2.488747in}{0.609842in}}%
\pgfpathlineto{\pgfqpoint{2.489141in}{0.609411in}}%
\pgfpathlineto{\pgfqpoint{2.492694in}{0.598802in}}%
\pgfpathlineto{\pgfqpoint{2.493089in}{0.599550in}}%
\pgfpathlineto{\pgfqpoint{2.493484in}{0.598961in}}%
\pgfpathlineto{\pgfqpoint{2.495852in}{0.590735in}}%
\pgfpathlineto{\pgfqpoint{2.496642in}{0.591121in}}%
\pgfpathlineto{\pgfqpoint{2.498615in}{0.581661in}}%
\pgfpathlineto{\pgfqpoint{2.499010in}{0.582433in}}%
\pgfpathlineto{\pgfqpoint{2.500589in}{0.585264in}}%
\pgfpathlineto{\pgfqpoint{2.500984in}{0.583356in}}%
\pgfpathlineto{\pgfqpoint{2.501773in}{0.583628in}}%
\pgfpathlineto{\pgfqpoint{2.508089in}{0.602265in}}%
\pgfpathlineto{\pgfqpoint{2.508879in}{0.598100in}}%
\pgfpathlineto{\pgfqpoint{2.512432in}{0.587678in}}%
\pgfpathlineto{\pgfqpoint{2.514011in}{0.591774in}}%
\pgfpathlineto{\pgfqpoint{2.514800in}{0.590814in}}%
\pgfpathlineto{\pgfqpoint{2.516774in}{0.591748in}}%
\pgfpathlineto{\pgfqpoint{2.517563in}{0.596297in}}%
\pgfpathlineto{\pgfqpoint{2.517958in}{0.597238in}}%
\pgfpathlineto{\pgfqpoint{2.518748in}{0.595185in}}%
\pgfpathlineto{\pgfqpoint{2.520327in}{0.596663in}}%
\pgfpathlineto{\pgfqpoint{2.523485in}{0.611543in}}%
\pgfpathlineto{\pgfqpoint{2.525458in}{0.612286in}}%
\pgfpathlineto{\pgfqpoint{2.527037in}{0.615806in}}%
\pgfpathlineto{\pgfqpoint{2.527827in}{0.614836in}}%
\pgfpathlineto{\pgfqpoint{2.532564in}{0.603822in}}%
\pgfpathlineto{\pgfqpoint{2.533748in}{0.602726in}}%
\pgfpathlineto{\pgfqpoint{2.534538in}{0.599801in}}%
\pgfpathlineto{\pgfqpoint{2.535327in}{0.602229in}}%
\pgfpathlineto{\pgfqpoint{2.536906in}{0.601444in}}%
\pgfpathlineto{\pgfqpoint{2.539275in}{0.597814in}}%
\pgfpathlineto{\pgfqpoint{2.539669in}{0.597965in}}%
\pgfpathlineto{\pgfqpoint{2.540459in}{0.597741in}}%
\pgfpathlineto{\pgfqpoint{2.541643in}{0.603742in}}%
\pgfpathlineto{\pgfqpoint{2.542433in}{0.600966in}}%
\pgfpathlineto{\pgfqpoint{2.545196in}{0.594032in}}%
\pgfpathlineto{\pgfqpoint{2.545590in}{0.595465in}}%
\pgfpathlineto{\pgfqpoint{2.545985in}{0.593156in}}%
\pgfpathlineto{\pgfqpoint{2.547169in}{0.590532in}}%
\pgfpathlineto{\pgfqpoint{2.547959in}{0.591725in}}%
\pgfpathlineto{\pgfqpoint{2.548748in}{0.595363in}}%
\pgfpathlineto{\pgfqpoint{2.549538in}{0.602230in}}%
\pgfpathlineto{\pgfqpoint{2.550722in}{0.599939in}}%
\pgfpathlineto{\pgfqpoint{2.551512in}{0.597703in}}%
\pgfpathlineto{\pgfqpoint{2.552301in}{0.599621in}}%
\pgfpathlineto{\pgfqpoint{2.554670in}{0.605851in}}%
\pgfpathlineto{\pgfqpoint{2.555459in}{0.607841in}}%
\pgfpathlineto{\pgfqpoint{2.557038in}{0.610363in}}%
\pgfpathlineto{\pgfqpoint{2.557433in}{0.608710in}}%
\pgfpathlineto{\pgfqpoint{2.559407in}{0.604255in}}%
\pgfpathlineto{\pgfqpoint{2.559801in}{0.604793in}}%
\pgfpathlineto{\pgfqpoint{2.560591in}{0.607996in}}%
\pgfpathlineto{\pgfqpoint{2.560986in}{0.606386in}}%
\pgfpathlineto{\pgfqpoint{2.562565in}{0.604272in}}%
\pgfpathlineto{\pgfqpoint{2.565723in}{0.617112in}}%
\pgfpathlineto{\pgfqpoint{2.566512in}{0.614700in}}%
\pgfpathlineto{\pgfqpoint{2.571249in}{0.598642in}}%
\pgfpathlineto{\pgfqpoint{2.571644in}{0.599966in}}%
\pgfpathlineto{\pgfqpoint{2.573618in}{0.596267in}}%
\pgfpathlineto{\pgfqpoint{2.577170in}{0.583293in}}%
\pgfpathlineto{\pgfqpoint{2.577565in}{0.583477in}}%
\pgfpathlineto{\pgfqpoint{2.579144in}{0.600967in}}%
\pgfpathlineto{\pgfqpoint{2.579934in}{0.598852in}}%
\pgfpathlineto{\pgfqpoint{2.581513in}{0.589645in}}%
\pgfpathlineto{\pgfqpoint{2.582697in}{0.590051in}}%
\pgfpathlineto{\pgfqpoint{2.584276in}{0.595586in}}%
\pgfpathlineto{\pgfqpoint{2.585065in}{0.593662in}}%
\pgfpathlineto{\pgfqpoint{2.585460in}{0.591049in}}%
\pgfpathlineto{\pgfqpoint{2.586644in}{0.591648in}}%
\pgfpathlineto{\pgfqpoint{2.587434in}{0.589570in}}%
\pgfpathlineto{\pgfqpoint{2.590987in}{0.576136in}}%
\pgfpathlineto{\pgfqpoint{2.592171in}{0.577326in}}%
\pgfpathlineto{\pgfqpoint{2.593355in}{0.572018in}}%
\pgfpathlineto{\pgfqpoint{2.593750in}{0.575232in}}%
\pgfpathlineto{\pgfqpoint{2.597697in}{0.591245in}}%
\pgfpathlineto{\pgfqpoint{2.599671in}{0.592719in}}%
\pgfpathlineto{\pgfqpoint{2.600066in}{0.591876in}}%
\pgfpathlineto{\pgfqpoint{2.601645in}{0.589903in}}%
\pgfpathlineto{\pgfqpoint{2.603619in}{0.595250in}}%
\pgfpathlineto{\pgfqpoint{2.604013in}{0.594915in}}%
\pgfpathlineto{\pgfqpoint{2.604408in}{0.596261in}}%
\pgfpathlineto{\pgfqpoint{2.604803in}{0.597213in}}%
\pgfpathlineto{\pgfqpoint{2.605592in}{0.595587in}}%
\pgfpathlineto{\pgfqpoint{2.606777in}{0.591005in}}%
\pgfpathlineto{\pgfqpoint{2.607171in}{0.593369in}}%
\pgfpathlineto{\pgfqpoint{2.609540in}{0.598596in}}%
\pgfpathlineto{\pgfqpoint{2.609935in}{0.598698in}}%
\pgfpathlineto{\pgfqpoint{2.612303in}{0.587882in}}%
\pgfpathlineto{\pgfqpoint{2.613093in}{0.589128in}}%
\pgfpathlineto{\pgfqpoint{2.614672in}{0.583344in}}%
\pgfpathlineto{\pgfqpoint{2.615066in}{0.585142in}}%
\pgfpathlineto{\pgfqpoint{2.615856in}{0.589051in}}%
\pgfpathlineto{\pgfqpoint{2.616645in}{0.586213in}}%
\pgfpathlineto{\pgfqpoint{2.618224in}{0.579152in}}%
\pgfpathlineto{\pgfqpoint{2.619014in}{0.580937in}}%
\pgfpathlineto{\pgfqpoint{2.620988in}{0.590479in}}%
\pgfpathlineto{\pgfqpoint{2.623356in}{0.596392in}}%
\pgfpathlineto{\pgfqpoint{2.623751in}{0.595091in}}%
\pgfpathlineto{\pgfqpoint{2.624935in}{0.595578in}}%
\pgfpathlineto{\pgfqpoint{2.626119in}{0.599134in}}%
\pgfpathlineto{\pgfqpoint{2.626909in}{0.598871in}}%
\pgfpathlineto{\pgfqpoint{2.628882in}{0.594891in}}%
\pgfpathlineto{\pgfqpoint{2.630067in}{0.590690in}}%
\pgfpathlineto{\pgfqpoint{2.630461in}{0.592873in}}%
\pgfpathlineto{\pgfqpoint{2.633619in}{0.589945in}}%
\pgfpathlineto{\pgfqpoint{2.634409in}{0.586331in}}%
\pgfpathlineto{\pgfqpoint{2.635198in}{0.588840in}}%
\pgfpathlineto{\pgfqpoint{2.637962in}{0.599819in}}%
\pgfpathlineto{\pgfqpoint{2.638751in}{0.598539in}}%
\pgfpathlineto{\pgfqpoint{2.640330in}{0.593290in}}%
\pgfpathlineto{\pgfqpoint{2.640725in}{0.593926in}}%
\pgfpathlineto{\pgfqpoint{2.642304in}{0.594392in}}%
\pgfpathlineto{\pgfqpoint{2.642699in}{0.593715in}}%
\pgfpathlineto{\pgfqpoint{2.643093in}{0.595298in}}%
\pgfpathlineto{\pgfqpoint{2.644672in}{0.599493in}}%
\pgfpathlineto{\pgfqpoint{2.645067in}{0.597767in}}%
\pgfpathlineto{\pgfqpoint{2.646251in}{0.595009in}}%
\pgfpathlineto{\pgfqpoint{2.647041in}{0.590530in}}%
\pgfpathlineto{\pgfqpoint{2.647830in}{0.592670in}}%
\pgfpathlineto{\pgfqpoint{2.650988in}{0.609010in}}%
\pgfpathlineto{\pgfqpoint{2.652567in}{0.613913in}}%
\pgfpathlineto{\pgfqpoint{2.656515in}{0.595544in}}%
\pgfpathlineto{\pgfqpoint{2.658883in}{0.593162in}}%
\pgfpathlineto{\pgfqpoint{2.661252in}{0.589787in}}%
\pgfpathlineto{\pgfqpoint{2.663620in}{0.596914in}}%
\pgfpathlineto{\pgfqpoint{2.664410in}{0.595813in}}%
\pgfpathlineto{\pgfqpoint{2.665199in}{0.594342in}}%
\pgfpathlineto{\pgfqpoint{2.665989in}{0.595699in}}%
\pgfpathlineto{\pgfqpoint{2.666778in}{0.595193in}}%
\pgfpathlineto{\pgfqpoint{2.667963in}{0.589581in}}%
\pgfpathlineto{\pgfqpoint{2.668752in}{0.590763in}}%
\pgfpathlineto{\pgfqpoint{2.670331in}{0.596899in}}%
\pgfpathlineto{\pgfqpoint{2.672305in}{0.602373in}}%
\pgfpathlineto{\pgfqpoint{2.673884in}{0.609718in}}%
\pgfpathlineto{\pgfqpoint{2.674279in}{0.611561in}}%
\pgfpathlineto{\pgfqpoint{2.675068in}{0.608001in}}%
\pgfpathlineto{\pgfqpoint{2.677437in}{0.602523in}}%
\pgfpathlineto{\pgfqpoint{2.678621in}{0.600275in}}%
\pgfpathlineto{\pgfqpoint{2.679016in}{0.601713in}}%
\pgfpathlineto{\pgfqpoint{2.679805in}{0.605787in}}%
\pgfpathlineto{\pgfqpoint{2.680595in}{0.603523in}}%
\pgfpathlineto{\pgfqpoint{2.680989in}{0.603959in}}%
\pgfpathlineto{\pgfqpoint{2.681384in}{0.603452in}}%
\pgfpathlineto{\pgfqpoint{2.681779in}{0.602255in}}%
\pgfpathlineto{\pgfqpoint{2.682568in}{0.603212in}}%
\pgfpathlineto{\pgfqpoint{2.684542in}{0.607395in}}%
\pgfpathlineto{\pgfqpoint{2.684937in}{0.605694in}}%
\pgfpathlineto{\pgfqpoint{2.685726in}{0.604375in}}%
\pgfpathlineto{\pgfqpoint{2.686911in}{0.605054in}}%
\pgfpathlineto{\pgfqpoint{2.687305in}{0.605201in}}%
\pgfpathlineto{\pgfqpoint{2.688490in}{0.602448in}}%
\pgfpathlineto{\pgfqpoint{2.688884in}{0.604231in}}%
\pgfpathlineto{\pgfqpoint{2.690069in}{0.605351in}}%
\pgfpathlineto{\pgfqpoint{2.690463in}{0.604155in}}%
\pgfpathlineto{\pgfqpoint{2.691253in}{0.604664in}}%
\pgfpathlineto{\pgfqpoint{2.692832in}{0.603078in}}%
\pgfpathlineto{\pgfqpoint{2.693227in}{0.603924in}}%
\pgfpathlineto{\pgfqpoint{2.693621in}{0.602770in}}%
\pgfpathlineto{\pgfqpoint{2.695990in}{0.592420in}}%
\pgfpathlineto{\pgfqpoint{2.696385in}{0.594052in}}%
\pgfpathlineto{\pgfqpoint{2.698753in}{0.600869in}}%
\pgfpathlineto{\pgfqpoint{2.702306in}{0.591779in}}%
\pgfpathlineto{\pgfqpoint{2.702701in}{0.593294in}}%
\pgfpathlineto{\pgfqpoint{2.705464in}{0.598962in}}%
\pgfpathlineto{\pgfqpoint{2.705859in}{0.598661in}}%
\pgfpathlineto{\pgfqpoint{2.707438in}{0.598435in}}%
\pgfpathlineto{\pgfqpoint{2.709017in}{0.601600in}}%
\pgfpathlineto{\pgfqpoint{2.709806in}{0.603034in}}%
\pgfpathlineto{\pgfqpoint{2.710201in}{0.600655in}}%
\pgfpathlineto{\pgfqpoint{2.711780in}{0.590061in}}%
\pgfpathlineto{\pgfqpoint{2.712569in}{0.595301in}}%
\pgfpathlineto{\pgfqpoint{2.714938in}{0.602592in}}%
\pgfpathlineto{\pgfqpoint{2.721254in}{0.586133in}}%
\pgfpathlineto{\pgfqpoint{2.722043in}{0.587429in}}%
\pgfpathlineto{\pgfqpoint{2.722438in}{0.589162in}}%
\pgfpathlineto{\pgfqpoint{2.723227in}{0.588079in}}%
\pgfpathlineto{\pgfqpoint{2.725991in}{0.577264in}}%
\pgfpathlineto{\pgfqpoint{2.727175in}{0.578272in}}%
\pgfpathlineto{\pgfqpoint{2.728359in}{0.576125in}}%
\pgfpathlineto{\pgfqpoint{2.728754in}{0.575737in}}%
\pgfpathlineto{\pgfqpoint{2.729149in}{0.576317in}}%
\pgfpathlineto{\pgfqpoint{2.730333in}{0.584245in}}%
\pgfpathlineto{\pgfqpoint{2.731122in}{0.583259in}}%
\pgfpathlineto{\pgfqpoint{2.731517in}{0.583446in}}%
\pgfpathlineto{\pgfqpoint{2.735070in}{0.575416in}}%
\pgfpathlineto{\pgfqpoint{2.735465in}{0.575204in}}%
\pgfpathlineto{\pgfqpoint{2.735859in}{0.576377in}}%
\pgfpathlineto{\pgfqpoint{2.736649in}{0.577395in}}%
\pgfpathlineto{\pgfqpoint{2.740202in}{0.584806in}}%
\pgfpathlineto{\pgfqpoint{2.740991in}{0.579236in}}%
\pgfpathlineto{\pgfqpoint{2.741781in}{0.583062in}}%
\pgfpathlineto{\pgfqpoint{2.744149in}{0.590265in}}%
\pgfpathlineto{\pgfqpoint{2.745728in}{0.589214in}}%
\pgfpathlineto{\pgfqpoint{2.746518in}{0.588464in}}%
\pgfpathlineto{\pgfqpoint{2.746912in}{0.589259in}}%
\pgfpathlineto{\pgfqpoint{2.748886in}{0.591308in}}%
\pgfpathlineto{\pgfqpoint{2.749676in}{0.590388in}}%
\pgfpathlineto{\pgfqpoint{2.751255in}{0.586532in}}%
\pgfpathlineto{\pgfqpoint{2.751649in}{0.586838in}}%
\pgfpathlineto{\pgfqpoint{2.754807in}{0.591818in}}%
\pgfpathlineto{\pgfqpoint{2.757571in}{0.599569in}}%
\pgfpathlineto{\pgfqpoint{2.759150in}{0.604588in}}%
\pgfpathlineto{\pgfqpoint{2.759939in}{0.604041in}}%
\pgfpathlineto{\pgfqpoint{2.760334in}{0.604473in}}%
\pgfpathlineto{\pgfqpoint{2.760729in}{0.602927in}}%
\pgfpathlineto{\pgfqpoint{2.763492in}{0.595310in}}%
\pgfpathlineto{\pgfqpoint{2.763887in}{0.596518in}}%
\pgfpathlineto{\pgfqpoint{2.765071in}{0.599479in}}%
\pgfpathlineto{\pgfqpoint{2.770992in}{0.620228in}}%
\pgfpathlineto{\pgfqpoint{2.771387in}{0.618736in}}%
\pgfpathlineto{\pgfqpoint{2.772571in}{0.620825in}}%
\pgfpathlineto{\pgfqpoint{2.773361in}{0.623740in}}%
\pgfpathlineto{\pgfqpoint{2.774150in}{0.622350in}}%
\pgfpathlineto{\pgfqpoint{2.775729in}{0.616903in}}%
\pgfpathlineto{\pgfqpoint{2.776124in}{0.617922in}}%
\pgfpathlineto{\pgfqpoint{2.778492in}{0.623473in}}%
\pgfpathlineto{\pgfqpoint{2.779282in}{0.624725in}}%
\pgfpathlineto{\pgfqpoint{2.781256in}{0.628141in}}%
\pgfpathlineto{\pgfqpoint{2.784808in}{0.631113in}}%
\pgfpathlineto{\pgfqpoint{2.785993in}{0.634731in}}%
\pgfpathlineto{\pgfqpoint{2.786782in}{0.634015in}}%
\pgfpathlineto{\pgfqpoint{2.789940in}{0.628219in}}%
\pgfpathlineto{\pgfqpoint{2.791519in}{0.630424in}}%
\pgfpathlineto{\pgfqpoint{2.791914in}{0.629221in}}%
\pgfpathlineto{\pgfqpoint{2.792703in}{0.629189in}}%
\pgfpathlineto{\pgfqpoint{2.793098in}{0.630181in}}%
\pgfpathlineto{\pgfqpoint{2.795072in}{0.631172in}}%
\pgfpathlineto{\pgfqpoint{2.797045in}{0.625270in}}%
\pgfpathlineto{\pgfqpoint{2.797835in}{0.628276in}}%
\pgfpathlineto{\pgfqpoint{2.798230in}{0.626752in}}%
\pgfpathlineto{\pgfqpoint{2.799809in}{0.621511in}}%
\pgfpathlineto{\pgfqpoint{2.800203in}{0.623015in}}%
\pgfpathlineto{\pgfqpoint{2.800993in}{0.621863in}}%
\pgfpathlineto{\pgfqpoint{2.802177in}{0.619345in}}%
\pgfpathlineto{\pgfqpoint{2.806519in}{0.604540in}}%
\pgfpathlineto{\pgfqpoint{2.807704in}{0.602980in}}%
\pgfpathlineto{\pgfqpoint{2.808098in}{0.603404in}}%
\pgfpathlineto{\pgfqpoint{2.812046in}{0.614962in}}%
\pgfpathlineto{\pgfqpoint{2.813230in}{0.616057in}}%
\pgfpathlineto{\pgfqpoint{2.813625in}{0.614761in}}%
\pgfpathlineto{\pgfqpoint{2.814809in}{0.612222in}}%
\pgfpathlineto{\pgfqpoint{2.815204in}{0.612775in}}%
\pgfpathlineto{\pgfqpoint{2.816388in}{0.618027in}}%
\pgfpathlineto{\pgfqpoint{2.817967in}{0.617049in}}%
\pgfpathlineto{\pgfqpoint{2.818362in}{0.615522in}}%
\pgfpathlineto{\pgfqpoint{2.818757in}{0.617411in}}%
\pgfpathlineto{\pgfqpoint{2.820730in}{0.620941in}}%
\pgfpathlineto{\pgfqpoint{2.822704in}{0.613180in}}%
\pgfpathlineto{\pgfqpoint{2.823099in}{0.613902in}}%
\pgfpathlineto{\pgfqpoint{2.823494in}{0.612764in}}%
\pgfpathlineto{\pgfqpoint{2.824283in}{0.610979in}}%
\pgfpathlineto{\pgfqpoint{2.824678in}{0.613258in}}%
\pgfpathlineto{\pgfqpoint{2.826652in}{0.619089in}}%
\pgfpathlineto{\pgfqpoint{2.827046in}{0.618880in}}%
\pgfpathlineto{\pgfqpoint{2.828231in}{0.615318in}}%
\pgfpathlineto{\pgfqpoint{2.829415in}{0.616717in}}%
\pgfpathlineto{\pgfqpoint{2.832573in}{0.619946in}}%
\pgfpathlineto{\pgfqpoint{2.832968in}{0.619120in}}%
\pgfpathlineto{\pgfqpoint{2.833362in}{0.620000in}}%
\pgfpathlineto{\pgfqpoint{2.834547in}{0.624094in}}%
\pgfpathlineto{\pgfqpoint{2.835336in}{0.623707in}}%
\pgfpathlineto{\pgfqpoint{2.836520in}{0.620036in}}%
\pgfpathlineto{\pgfqpoint{2.837310in}{0.620730in}}%
\pgfpathlineto{\pgfqpoint{2.838889in}{0.625109in}}%
\pgfpathlineto{\pgfqpoint{2.839284in}{0.623697in}}%
\pgfpathlineto{\pgfqpoint{2.842047in}{0.610071in}}%
\pgfpathlineto{\pgfqpoint{2.842836in}{0.610450in}}%
\pgfpathlineto{\pgfqpoint{2.843626in}{0.611417in}}%
\pgfpathlineto{\pgfqpoint{2.850337in}{0.637197in}}%
\pgfpathlineto{\pgfqpoint{2.850731in}{0.635318in}}%
\pgfpathlineto{\pgfqpoint{2.852705in}{0.624642in}}%
\pgfpathlineto{\pgfqpoint{2.853495in}{0.626819in}}%
\pgfpathlineto{\pgfqpoint{2.855074in}{0.631755in}}%
\pgfpathlineto{\pgfqpoint{2.855468in}{0.630969in}}%
\pgfpathlineto{\pgfqpoint{2.856653in}{0.628413in}}%
\pgfpathlineto{\pgfqpoint{2.857837in}{0.624962in}}%
\pgfpathlineto{\pgfqpoint{2.859021in}{0.625892in}}%
\pgfpathlineto{\pgfqpoint{2.859811in}{0.625638in}}%
\pgfpathlineto{\pgfqpoint{2.861390in}{0.621902in}}%
\pgfpathlineto{\pgfqpoint{2.861784in}{0.622465in}}%
\pgfpathlineto{\pgfqpoint{2.862179in}{0.623334in}}%
\pgfpathlineto{\pgfqpoint{2.862574in}{0.621200in}}%
\pgfpathlineto{\pgfqpoint{2.862969in}{0.621527in}}%
\pgfpathlineto{\pgfqpoint{2.864153in}{0.617207in}}%
\pgfpathlineto{\pgfqpoint{2.866521in}{0.607634in}}%
\pgfpathlineto{\pgfqpoint{2.867706in}{0.613289in}}%
\pgfpathlineto{\pgfqpoint{2.868495in}{0.611724in}}%
\pgfpathlineto{\pgfqpoint{2.870469in}{0.606813in}}%
\pgfpathlineto{\pgfqpoint{2.870864in}{0.608285in}}%
\pgfpathlineto{\pgfqpoint{2.871258in}{0.608864in}}%
\pgfpathlineto{\pgfqpoint{2.871653in}{0.608081in}}%
\pgfpathlineto{\pgfqpoint{2.872837in}{0.604167in}}%
\pgfpathlineto{\pgfqpoint{2.873627in}{0.605602in}}%
\pgfpathlineto{\pgfqpoint{2.874811in}{0.611121in}}%
\pgfpathlineto{\pgfqpoint{2.876785in}{0.624395in}}%
\pgfpathlineto{\pgfqpoint{2.877180in}{0.623405in}}%
\pgfpathlineto{\pgfqpoint{2.878759in}{0.618975in}}%
\pgfpathlineto{\pgfqpoint{2.879548in}{0.620165in}}%
\pgfpathlineto{\pgfqpoint{2.884680in}{0.620269in}}%
\pgfpathlineto{\pgfqpoint{2.885864in}{0.626664in}}%
\pgfpathlineto{\pgfqpoint{2.886653in}{0.625140in}}%
\pgfpathlineto{\pgfqpoint{2.890206in}{0.613346in}}%
\pgfpathlineto{\pgfqpoint{2.893759in}{0.593828in}}%
\pgfpathlineto{\pgfqpoint{2.894548in}{0.595700in}}%
\pgfpathlineto{\pgfqpoint{2.895338in}{0.596851in}}%
\pgfpathlineto{\pgfqpoint{2.896522in}{0.600926in}}%
\pgfpathlineto{\pgfqpoint{2.896917in}{0.599525in}}%
\pgfpathlineto{\pgfqpoint{2.899285in}{0.595502in}}%
\pgfpathlineto{\pgfqpoint{2.901259in}{0.592938in}}%
\pgfpathlineto{\pgfqpoint{2.902049in}{0.591839in}}%
\pgfpathlineto{\pgfqpoint{2.902443in}{0.593625in}}%
\pgfpathlineto{\pgfqpoint{2.903233in}{0.595004in}}%
\pgfpathlineto{\pgfqpoint{2.905207in}{0.606380in}}%
\pgfpathlineto{\pgfqpoint{2.905996in}{0.605420in}}%
\pgfpathlineto{\pgfqpoint{2.907180in}{0.606432in}}%
\pgfpathlineto{\pgfqpoint{2.907970in}{0.606871in}}%
\pgfpathlineto{\pgfqpoint{2.911523in}{0.595864in}}%
\pgfpathlineto{\pgfqpoint{2.912312in}{0.598493in}}%
\pgfpathlineto{\pgfqpoint{2.913102in}{0.596548in}}%
\pgfpathlineto{\pgfqpoint{2.913891in}{0.594623in}}%
\pgfpathlineto{\pgfqpoint{2.914286in}{0.595504in}}%
\pgfpathlineto{\pgfqpoint{2.916654in}{0.600395in}}%
\pgfpathlineto{\pgfqpoint{2.918628in}{0.613547in}}%
\pgfpathlineto{\pgfqpoint{2.919812in}{0.618288in}}%
\pgfpathlineto{\pgfqpoint{2.920207in}{0.616683in}}%
\pgfpathlineto{\pgfqpoint{2.921391in}{0.616175in}}%
\pgfpathlineto{\pgfqpoint{2.921786in}{0.616892in}}%
\pgfpathlineto{\pgfqpoint{2.924155in}{0.615025in}}%
\pgfpathlineto{\pgfqpoint{2.924549in}{0.616369in}}%
\pgfpathlineto{\pgfqpoint{2.926918in}{0.627647in}}%
\pgfpathlineto{\pgfqpoint{2.927707in}{0.627472in}}%
\pgfpathlineto{\pgfqpoint{2.928892in}{0.623923in}}%
\pgfpathlineto{\pgfqpoint{2.929286in}{0.626855in}}%
\pgfpathlineto{\pgfqpoint{2.932050in}{0.635751in}}%
\pgfpathlineto{\pgfqpoint{2.932444in}{0.634751in}}%
\pgfpathlineto{\pgfqpoint{2.934813in}{0.629375in}}%
\pgfpathlineto{\pgfqpoint{2.935602in}{0.629649in}}%
\pgfpathlineto{\pgfqpoint{2.936392in}{0.629913in}}%
\pgfpathlineto{\pgfqpoint{2.938760in}{0.615782in}}%
\pgfpathlineto{\pgfqpoint{2.941524in}{0.602419in}}%
\pgfpathlineto{\pgfqpoint{2.941918in}{0.602528in}}%
\pgfpathlineto{\pgfqpoint{2.945076in}{0.606809in}}%
\pgfpathlineto{\pgfqpoint{2.945866in}{0.612385in}}%
\pgfpathlineto{\pgfqpoint{2.946655in}{0.609604in}}%
\pgfpathlineto{\pgfqpoint{2.952182in}{0.596614in}}%
\pgfpathlineto{\pgfqpoint{2.952971in}{0.598514in}}%
\pgfpathlineto{\pgfqpoint{2.954550in}{0.602513in}}%
\pgfpathlineto{\pgfqpoint{2.959287in}{0.620775in}}%
\pgfpathlineto{\pgfqpoint{2.960866in}{0.622433in}}%
\pgfpathlineto{\pgfqpoint{2.963235in}{0.627944in}}%
\pgfpathlineto{\pgfqpoint{2.964419in}{0.626318in}}%
\pgfpathlineto{\pgfqpoint{2.964814in}{0.627163in}}%
\pgfpathlineto{\pgfqpoint{2.965208in}{0.628574in}}%
\pgfpathlineto{\pgfqpoint{2.965603in}{0.627732in}}%
\pgfpathlineto{\pgfqpoint{2.967577in}{0.621613in}}%
\pgfpathlineto{\pgfqpoint{2.967972in}{0.621729in}}%
\pgfpathlineto{\pgfqpoint{2.969551in}{0.621741in}}%
\pgfpathlineto{\pgfqpoint{2.971919in}{0.612019in}}%
\pgfpathlineto{\pgfqpoint{2.972314in}{0.612593in}}%
\pgfpathlineto{\pgfqpoint{2.973103in}{0.612887in}}%
\pgfpathlineto{\pgfqpoint{2.973498in}{0.612353in}}%
\pgfpathlineto{\pgfqpoint{2.973893in}{0.611773in}}%
\pgfpathlineto{\pgfqpoint{2.974288in}{0.612510in}}%
\pgfpathlineto{\pgfqpoint{2.974682in}{0.614439in}}%
\pgfpathlineto{\pgfqpoint{2.975472in}{0.613689in}}%
\pgfpathlineto{\pgfqpoint{2.982183in}{0.587198in}}%
\pgfpathlineto{\pgfqpoint{2.982577in}{0.588652in}}%
\pgfpathlineto{\pgfqpoint{2.984156in}{0.592420in}}%
\pgfpathlineto{\pgfqpoint{2.984551in}{0.592338in}}%
\pgfpathlineto{\pgfqpoint{2.986525in}{0.596954in}}%
\pgfpathlineto{\pgfqpoint{2.986920in}{0.596261in}}%
\pgfpathlineto{\pgfqpoint{2.989683in}{0.589571in}}%
\pgfpathlineto{\pgfqpoint{2.987709in}{0.597061in}}%
\pgfpathlineto{\pgfqpoint{2.990078in}{0.589975in}}%
\pgfpathlineto{\pgfqpoint{2.992051in}{0.585130in}}%
\pgfpathlineto{\pgfqpoint{2.994025in}{0.587984in}}%
\pgfpathlineto{\pgfqpoint{2.995604in}{0.589117in}}%
\pgfpathlineto{\pgfqpoint{2.996394in}{0.592578in}}%
\pgfpathlineto{\pgfqpoint{2.998762in}{0.605105in}}%
\pgfpathlineto{\pgfqpoint{3.000341in}{0.601719in}}%
\pgfpathlineto{\pgfqpoint{3.003104in}{0.601631in}}%
\pgfpathlineto{\pgfqpoint{3.004289in}{0.604441in}}%
\pgfpathlineto{\pgfqpoint{3.004683in}{0.602846in}}%
\pgfpathlineto{\pgfqpoint{3.005078in}{0.602304in}}%
\pgfpathlineto{\pgfqpoint{3.005473in}{0.603642in}}%
\pgfpathlineto{\pgfqpoint{3.006657in}{0.605478in}}%
\pgfpathlineto{\pgfqpoint{3.007447in}{0.603918in}}%
\pgfpathlineto{\pgfqpoint{3.010605in}{0.598572in}}%
\pgfpathlineto{\pgfqpoint{3.012973in}{0.603433in}}%
\pgfpathlineto{\pgfqpoint{3.013368in}{0.601972in}}%
\pgfpathlineto{\pgfqpoint{3.014157in}{0.603617in}}%
\pgfpathlineto{\pgfqpoint{3.015736in}{0.605354in}}%
\pgfpathlineto{\pgfqpoint{3.018105in}{0.614198in}}%
\pgfpathlineto{\pgfqpoint{3.018894in}{0.610416in}}%
\pgfpathlineto{\pgfqpoint{3.020473in}{0.606237in}}%
\pgfpathlineto{\pgfqpoint{3.020868in}{0.606624in}}%
\pgfpathlineto{\pgfqpoint{3.021263in}{0.607648in}}%
\pgfpathlineto{\pgfqpoint{3.021658in}{0.605097in}}%
\pgfpathlineto{\pgfqpoint{3.025605in}{0.593956in}}%
\pgfpathlineto{\pgfqpoint{3.028368in}{0.599515in}}%
\pgfpathlineto{\pgfqpoint{3.031132in}{0.597366in}}%
\pgfpathlineto{\pgfqpoint{3.031526in}{0.598053in}}%
\pgfpathlineto{\pgfqpoint{3.033105in}{0.600926in}}%
\pgfpathlineto{\pgfqpoint{3.033500in}{0.600404in}}%
\pgfpathlineto{\pgfqpoint{3.035474in}{0.596011in}}%
\pgfpathlineto{\pgfqpoint{3.039421in}{0.603844in}}%
\pgfpathlineto{\pgfqpoint{3.041000in}{0.608696in}}%
\pgfpathlineto{\pgfqpoint{3.041395in}{0.608426in}}%
\pgfpathlineto{\pgfqpoint{3.042185in}{0.609988in}}%
\pgfpathlineto{\pgfqpoint{3.042974in}{0.612396in}}%
\pgfpathlineto{\pgfqpoint{3.043764in}{0.609881in}}%
\pgfpathlineto{\pgfqpoint{3.044553in}{0.608645in}}%
\pgfpathlineto{\pgfqpoint{3.045737in}{0.613679in}}%
\pgfpathlineto{\pgfqpoint{3.046527in}{0.612196in}}%
\pgfpathlineto{\pgfqpoint{3.048895in}{0.603765in}}%
\pgfpathlineto{\pgfqpoint{3.049290in}{0.604753in}}%
\pgfpathlineto{\pgfqpoint{3.050869in}{0.602891in}}%
\pgfpathlineto{\pgfqpoint{3.052053in}{0.596511in}}%
\pgfpathlineto{\pgfqpoint{3.052843in}{0.597288in}}%
\pgfpathlineto{\pgfqpoint{3.053632in}{0.596304in}}%
\pgfpathlineto{\pgfqpoint{3.055211in}{0.591849in}}%
\pgfpathlineto{\pgfqpoint{3.056395in}{0.588736in}}%
\pgfpathlineto{\pgfqpoint{3.056790in}{0.590945in}}%
\pgfpathlineto{\pgfqpoint{3.057580in}{0.593382in}}%
\pgfpathlineto{\pgfqpoint{3.058764in}{0.592889in}}%
\pgfpathlineto{\pgfqpoint{3.059553in}{0.596542in}}%
\pgfpathlineto{\pgfqpoint{3.062317in}{0.607283in}}%
\pgfpathlineto{\pgfqpoint{3.064685in}{0.615590in}}%
\pgfpathlineto{\pgfqpoint{3.065080in}{0.614135in}}%
\pgfpathlineto{\pgfqpoint{3.067843in}{0.603182in}}%
\pgfpathlineto{\pgfqpoint{3.068238in}{0.604144in}}%
\pgfpathlineto{\pgfqpoint{3.068633in}{0.604046in}}%
\pgfpathlineto{\pgfqpoint{3.069422in}{0.605792in}}%
\pgfpathlineto{\pgfqpoint{3.069817in}{0.605301in}}%
\pgfpathlineto{\pgfqpoint{3.071396in}{0.594936in}}%
\pgfpathlineto{\pgfqpoint{3.071791in}{0.596832in}}%
\pgfpathlineto{\pgfqpoint{3.074949in}{0.609546in}}%
\pgfpathlineto{\pgfqpoint{3.076528in}{0.607001in}}%
\pgfpathlineto{\pgfqpoint{3.079291in}{0.608581in}}%
\pgfpathlineto{\pgfqpoint{3.079686in}{0.607515in}}%
\pgfpathlineto{\pgfqpoint{3.080475in}{0.605659in}}%
\pgfpathlineto{\pgfqpoint{3.080870in}{0.606253in}}%
\pgfpathlineto{\pgfqpoint{3.081265in}{0.607653in}}%
\pgfpathlineto{\pgfqpoint{3.082054in}{0.605957in}}%
\pgfpathlineto{\pgfqpoint{3.082844in}{0.602336in}}%
\pgfpathlineto{\pgfqpoint{3.083633in}{0.603808in}}%
\pgfpathlineto{\pgfqpoint{3.084028in}{0.604831in}}%
\pgfpathlineto{\pgfqpoint{3.084423in}{0.602405in}}%
\pgfpathlineto{\pgfqpoint{3.086791in}{0.595578in}}%
\pgfpathlineto{\pgfqpoint{3.087581in}{0.596325in}}%
\pgfpathlineto{\pgfqpoint{3.088370in}{0.597647in}}%
\pgfpathlineto{\pgfqpoint{3.088765in}{0.595221in}}%
\pgfpathlineto{\pgfqpoint{3.091133in}{0.587428in}}%
\pgfpathlineto{\pgfqpoint{3.091923in}{0.589086in}}%
\pgfpathlineto{\pgfqpoint{3.093107in}{0.593327in}}%
\pgfpathlineto{\pgfqpoint{3.093502in}{0.589862in}}%
\pgfpathlineto{\pgfqpoint{3.095870in}{0.580395in}}%
\pgfpathlineto{\pgfqpoint{3.096660in}{0.581099in}}%
\pgfpathlineto{\pgfqpoint{3.101397in}{0.597110in}}%
\pgfpathlineto{\pgfqpoint{3.101792in}{0.596374in}}%
\pgfpathlineto{\pgfqpoint{3.102581in}{0.594339in}}%
\pgfpathlineto{\pgfqpoint{3.102976in}{0.596320in}}%
\pgfpathlineto{\pgfqpoint{3.103371in}{0.596805in}}%
\pgfpathlineto{\pgfqpoint{3.103765in}{0.596117in}}%
\pgfpathlineto{\pgfqpoint{3.106134in}{0.589483in}}%
\pgfpathlineto{\pgfqpoint{3.106923in}{0.591370in}}%
\pgfpathlineto{\pgfqpoint{3.108108in}{0.596588in}}%
\pgfpathlineto{\pgfqpoint{3.109292in}{0.594589in}}%
\pgfpathlineto{\pgfqpoint{3.110081in}{0.595301in}}%
\pgfpathlineto{\pgfqpoint{3.110476in}{0.592360in}}%
\pgfpathlineto{\pgfqpoint{3.111660in}{0.595112in}}%
\pgfpathlineto{\pgfqpoint{3.112055in}{0.594098in}}%
\pgfpathlineto{\pgfqpoint{3.112450in}{0.595557in}}%
\pgfpathlineto{\pgfqpoint{3.113239in}{0.595269in}}%
\pgfpathlineto{\pgfqpoint{3.114424in}{0.596300in}}%
\pgfpathlineto{\pgfqpoint{3.114818in}{0.595149in}}%
\pgfpathlineto{\pgfqpoint{3.116792in}{0.586921in}}%
\pgfpathlineto{\pgfqpoint{3.117582in}{0.587460in}}%
\pgfpathlineto{\pgfqpoint{3.119161in}{0.587637in}}%
\pgfpathlineto{\pgfqpoint{3.121134in}{0.576670in}}%
\pgfpathlineto{\pgfqpoint{3.121924in}{0.577264in}}%
\pgfpathlineto{\pgfqpoint{3.126266in}{0.585370in}}%
\pgfpathlineto{\pgfqpoint{3.126661in}{0.584220in}}%
\pgfpathlineto{\pgfqpoint{3.127450in}{0.583106in}}%
\pgfpathlineto{\pgfqpoint{3.127845in}{0.583476in}}%
\pgfpathlineto{\pgfqpoint{3.129029in}{0.587911in}}%
\pgfpathlineto{\pgfqpoint{3.130213in}{0.586916in}}%
\pgfpathlineto{\pgfqpoint{3.132187in}{0.581330in}}%
\pgfpathlineto{\pgfqpoint{3.132582in}{0.583324in}}%
\pgfpathlineto{\pgfqpoint{3.134161in}{0.587285in}}%
\pgfpathlineto{\pgfqpoint{3.134556in}{0.585905in}}%
\pgfpathlineto{\pgfqpoint{3.140477in}{0.572147in}}%
\pgfpathlineto{\pgfqpoint{3.144424in}{0.586522in}}%
\pgfpathlineto{\pgfqpoint{3.145609in}{0.589162in}}%
\pgfpathlineto{\pgfqpoint{3.146398in}{0.588024in}}%
\pgfpathlineto{\pgfqpoint{3.148372in}{0.581294in}}%
\pgfpathlineto{\pgfqpoint{3.149161in}{0.582663in}}%
\pgfpathlineto{\pgfqpoint{3.149556in}{0.583497in}}%
\pgfpathlineto{\pgfqpoint{3.149951in}{0.580989in}}%
\pgfpathlineto{\pgfqpoint{3.154293in}{0.565866in}}%
\pgfpathlineto{\pgfqpoint{3.162978in}{0.585834in}}%
\pgfpathlineto{\pgfqpoint{3.164162in}{0.583446in}}%
\pgfpathlineto{\pgfqpoint{3.164557in}{0.584298in}}%
\pgfpathlineto{\pgfqpoint{3.166925in}{0.587369in}}%
\pgfpathlineto{\pgfqpoint{3.167320in}{0.585781in}}%
\pgfpathlineto{\pgfqpoint{3.171662in}{0.574171in}}%
\pgfpathlineto{\pgfqpoint{3.172057in}{0.573562in}}%
\pgfpathlineto{\pgfqpoint{3.172452in}{0.575071in}}%
\pgfpathlineto{\pgfqpoint{3.177583in}{0.589725in}}%
\pgfpathlineto{\pgfqpoint{3.179952in}{0.591075in}}%
\pgfpathlineto{\pgfqpoint{3.180347in}{0.590054in}}%
\pgfpathlineto{\pgfqpoint{3.180741in}{0.591039in}}%
\pgfpathlineto{\pgfqpoint{3.181926in}{0.596486in}}%
\pgfpathlineto{\pgfqpoint{3.182715in}{0.594697in}}%
\pgfpathlineto{\pgfqpoint{3.183505in}{0.592070in}}%
\pgfpathlineto{\pgfqpoint{3.184294in}{0.594376in}}%
\pgfpathlineto{\pgfqpoint{3.185478in}{0.594741in}}%
\pgfpathlineto{\pgfqpoint{3.185873in}{0.593836in}}%
\pgfpathlineto{\pgfqpoint{3.187057in}{0.592292in}}%
\pgfpathlineto{\pgfqpoint{3.187847in}{0.593383in}}%
\pgfpathlineto{\pgfqpoint{3.189821in}{0.595845in}}%
\pgfpathlineto{\pgfqpoint{3.192979in}{0.607155in}}%
\pgfpathlineto{\pgfqpoint{3.193768in}{0.606600in}}%
\pgfpathlineto{\pgfqpoint{3.196137in}{0.601788in}}%
\pgfpathlineto{\pgfqpoint{3.196926in}{0.604785in}}%
\pgfpathlineto{\pgfqpoint{3.197321in}{0.606560in}}%
\pgfpathlineto{\pgfqpoint{3.197321in}{0.606560in}}%
\pgfusepath{stroke}%
\end{pgfscope}%
\begin{pgfscope}%
\pgfpathrectangle{\pgfqpoint{0.608025in}{0.484444in}}{\pgfqpoint{2.712595in}{1.541287in}}%
\pgfusepath{clip}%
\pgfsetbuttcap%
\pgfsetroundjoin%
\definecolor{currentfill}{rgb}{1.000000,0.498039,0.054902}%
\pgfsetfillcolor{currentfill}%
\pgfsetlinewidth{1.003750pt}%
\definecolor{currentstroke}{rgb}{1.000000,0.498039,0.054902}%
\pgfsetstrokecolor{currentstroke}%
\pgfsetdash{}{0pt}%
\pgfsys@defobject{currentmarker}{\pgfqpoint{-0.020833in}{-0.020833in}}{\pgfqpoint{0.020833in}{0.020833in}}{%
\pgfpathmoveto{\pgfqpoint{0.000000in}{-0.020833in}}%
\pgfpathcurveto{\pgfqpoint{0.005525in}{-0.020833in}}{\pgfqpoint{0.010825in}{-0.018638in}}{\pgfqpoint{0.014731in}{-0.014731in}}%
\pgfpathcurveto{\pgfqpoint{0.018638in}{-0.010825in}}{\pgfqpoint{0.020833in}{-0.005525in}}{\pgfqpoint{0.020833in}{0.000000in}}%
\pgfpathcurveto{\pgfqpoint{0.020833in}{0.005525in}}{\pgfqpoint{0.018638in}{0.010825in}}{\pgfqpoint{0.014731in}{0.014731in}}%
\pgfpathcurveto{\pgfqpoint{0.010825in}{0.018638in}}{\pgfqpoint{0.005525in}{0.020833in}}{\pgfqpoint{0.000000in}{0.020833in}}%
\pgfpathcurveto{\pgfqpoint{-0.005525in}{0.020833in}}{\pgfqpoint{-0.010825in}{0.018638in}}{\pgfqpoint{-0.014731in}{0.014731in}}%
\pgfpathcurveto{\pgfqpoint{-0.018638in}{0.010825in}}{\pgfqpoint{-0.020833in}{0.005525in}}{\pgfqpoint{-0.020833in}{0.000000in}}%
\pgfpathcurveto{\pgfqpoint{-0.020833in}{-0.005525in}}{\pgfqpoint{-0.018638in}{-0.010825in}}{\pgfqpoint{-0.014731in}{-0.014731in}}%
\pgfpathcurveto{\pgfqpoint{-0.010825in}{-0.018638in}}{\pgfqpoint{-0.005525in}{-0.020833in}}{\pgfqpoint{0.000000in}{-0.020833in}}%
\pgfpathlineto{\pgfqpoint{0.000000in}{-0.020833in}}%
\pgfpathclose%
\pgfusepath{stroke,fill}%
}%
\begin{pgfscope}%
\pgfsys@transformshift{0.739220in}{1.951233in}%
\pgfsys@useobject{currentmarker}{}%
\end{pgfscope}%
\begin{pgfscope}%
\pgfsys@transformshift{0.818170in}{1.692840in}%
\pgfsys@useobject{currentmarker}{}%
\end{pgfscope}%
\begin{pgfscope}%
\pgfsys@transformshift{0.897119in}{1.321689in}%
\pgfsys@useobject{currentmarker}{}%
\end{pgfscope}%
\begin{pgfscope}%
\pgfsys@transformshift{0.976069in}{1.109257in}%
\pgfsys@useobject{currentmarker}{}%
\end{pgfscope}%
\begin{pgfscope}%
\pgfsys@transformshift{1.055019in}{0.880252in}%
\pgfsys@useobject{currentmarker}{}%
\end{pgfscope}%
\begin{pgfscope}%
\pgfsys@transformshift{1.133969in}{0.827430in}%
\pgfsys@useobject{currentmarker}{}%
\end{pgfscope}%
\begin{pgfscope}%
\pgfsys@transformshift{1.212918in}{0.662056in}%
\pgfsys@useobject{currentmarker}{}%
\end{pgfscope}%
\begin{pgfscope}%
\pgfsys@transformshift{1.291868in}{0.694937in}%
\pgfsys@useobject{currentmarker}{}%
\end{pgfscope}%
\begin{pgfscope}%
\pgfsys@transformshift{1.370818in}{0.678683in}%
\pgfsys@useobject{currentmarker}{}%
\end{pgfscope}%
\begin{pgfscope}%
\pgfsys@transformshift{1.449768in}{0.622196in}%
\pgfsys@useobject{currentmarker}{}%
\end{pgfscope}%
\begin{pgfscope}%
\pgfsys@transformshift{1.528718in}{0.621133in}%
\pgfsys@useobject{currentmarker}{}%
\end{pgfscope}%
\begin{pgfscope}%
\pgfsys@transformshift{1.607667in}{0.611137in}%
\pgfsys@useobject{currentmarker}{}%
\end{pgfscope}%
\begin{pgfscope}%
\pgfsys@transformshift{1.686617in}{0.613531in}%
\pgfsys@useobject{currentmarker}{}%
\end{pgfscope}%
\begin{pgfscope}%
\pgfsys@transformshift{1.765567in}{0.616574in}%
\pgfsys@useobject{currentmarker}{}%
\end{pgfscope}%
\begin{pgfscope}%
\pgfsys@transformshift{1.844517in}{0.595526in}%
\pgfsys@useobject{currentmarker}{}%
\end{pgfscope}%
\begin{pgfscope}%
\pgfsys@transformshift{1.923466in}{0.591980in}%
\pgfsys@useobject{currentmarker}{}%
\end{pgfscope}%
\begin{pgfscope}%
\pgfsys@transformshift{2.002416in}{0.616131in}%
\pgfsys@useobject{currentmarker}{}%
\end{pgfscope}%
\begin{pgfscope}%
\pgfsys@transformshift{2.081366in}{0.625657in}%
\pgfsys@useobject{currentmarker}{}%
\end{pgfscope}%
\begin{pgfscope}%
\pgfsys@transformshift{2.160316in}{0.593774in}%
\pgfsys@useobject{currentmarker}{}%
\end{pgfscope}%
\begin{pgfscope}%
\pgfsys@transformshift{2.239265in}{0.619972in}%
\pgfsys@useobject{currentmarker}{}%
\end{pgfscope}%
\begin{pgfscope}%
\pgfsys@transformshift{2.318215in}{0.581857in}%
\pgfsys@useobject{currentmarker}{}%
\end{pgfscope}%
\begin{pgfscope}%
\pgfsys@transformshift{2.397165in}{0.586175in}%
\pgfsys@useobject{currentmarker}{}%
\end{pgfscope}%
\begin{pgfscope}%
\pgfsys@transformshift{2.476115in}{0.601473in}%
\pgfsys@useobject{currentmarker}{}%
\end{pgfscope}%
\begin{pgfscope}%
\pgfsys@transformshift{2.555064in}{0.606052in}%
\pgfsys@useobject{currentmarker}{}%
\end{pgfscope}%
\begin{pgfscope}%
\pgfsys@transformshift{2.634014in}{0.588050in}%
\pgfsys@useobject{currentmarker}{}%
\end{pgfscope}%
\begin{pgfscope}%
\pgfsys@transformshift{2.712964in}{0.597557in}%
\pgfsys@useobject{currentmarker}{}%
\end{pgfscope}%
\begin{pgfscope}%
\pgfsys@transformshift{2.791914in}{0.629221in}%
\pgfsys@useobject{currentmarker}{}%
\end{pgfscope}%
\begin{pgfscope}%
\pgfsys@transformshift{2.870864in}{0.608285in}%
\pgfsys@useobject{currentmarker}{}%
\end{pgfscope}%
\begin{pgfscope}%
\pgfsys@transformshift{2.949813in}{0.605597in}%
\pgfsys@useobject{currentmarker}{}%
\end{pgfscope}%
\begin{pgfscope}%
\pgfsys@transformshift{3.028763in}{0.599223in}%
\pgfsys@useobject{currentmarker}{}%
\end{pgfscope}%
\begin{pgfscope}%
\pgfsys@transformshift{3.107713in}{0.594299in}%
\pgfsys@useobject{currentmarker}{}%
\end{pgfscope}%
\begin{pgfscope}%
\pgfsys@transformshift{3.186663in}{0.593779in}%
\pgfsys@useobject{currentmarker}{}%
\end{pgfscope}%
\end{pgfscope}%
\begin{pgfscope}%
\pgfpathrectangle{\pgfqpoint{0.608025in}{0.484444in}}{\pgfqpoint{2.712595in}{1.541287in}}%
\pgfusepath{clip}%
\pgfsetrectcap%
\pgfsetroundjoin%
\pgfsetlinewidth{1.505625pt}%
\definecolor{currentstroke}{rgb}{0.172549,0.627451,0.172549}%
\pgfsetstrokecolor{currentstroke}%
\pgfsetdash{}{0pt}%
\pgfpathmoveto{\pgfqpoint{0.731325in}{1.955466in}}%
\pgfpathlineto{\pgfqpoint{0.744746in}{1.945271in}}%
\pgfpathlineto{\pgfqpoint{0.749089in}{1.935578in}}%
\pgfpathlineto{\pgfqpoint{0.764879in}{1.883745in}}%
\pgfpathlineto{\pgfqpoint{0.778695in}{1.842266in}}%
\pgfpathlineto{\pgfqpoint{0.795669in}{1.808149in}}%
\pgfpathlineto{\pgfqpoint{0.809485in}{1.794937in}}%
\pgfpathlineto{\pgfqpoint{0.813827in}{1.786038in}}%
\pgfpathlineto{\pgfqpoint{0.814617in}{1.786939in}}%
\pgfpathlineto{\pgfqpoint{0.815801in}{1.786665in}}%
\pgfpathlineto{\pgfqpoint{0.818170in}{1.772703in}}%
\pgfpathlineto{\pgfqpoint{0.822117in}{1.734140in}}%
\pgfpathlineto{\pgfqpoint{0.826854in}{1.690475in}}%
\pgfpathlineto{\pgfqpoint{0.832775in}{1.637663in}}%
\pgfpathlineto{\pgfqpoint{0.839486in}{1.615184in}}%
\pgfpathlineto{\pgfqpoint{0.845407in}{1.583698in}}%
\pgfpathlineto{\pgfqpoint{0.853302in}{1.543774in}}%
\pgfpathlineto{\pgfqpoint{0.855671in}{1.539904in}}%
\pgfpathlineto{\pgfqpoint{0.856066in}{1.541674in}}%
\pgfpathlineto{\pgfqpoint{0.859224in}{1.548548in}}%
\pgfpathlineto{\pgfqpoint{0.870277in}{1.485003in}}%
\pgfpathlineto{\pgfqpoint{0.870671in}{1.488410in}}%
\pgfpathlineto{\pgfqpoint{0.873829in}{1.531750in}}%
\pgfpathlineto{\pgfqpoint{0.874619in}{1.531190in}}%
\pgfpathlineto{\pgfqpoint{0.875408in}{1.530426in}}%
\pgfpathlineto{\pgfqpoint{0.878961in}{1.543461in}}%
\pgfpathlineto{\pgfqpoint{0.882514in}{1.541822in}}%
\pgfpathlineto{\pgfqpoint{0.884487in}{1.532838in}}%
\pgfpathlineto{\pgfqpoint{0.887251in}{1.501577in}}%
\pgfpathlineto{\pgfqpoint{0.887645in}{1.495863in}}%
\pgfpathlineto{\pgfqpoint{0.888830in}{1.499626in}}%
\pgfpathlineto{\pgfqpoint{0.893961in}{1.514815in}}%
\pgfpathlineto{\pgfqpoint{0.894751in}{1.518885in}}%
\pgfpathlineto{\pgfqpoint{0.895540in}{1.517825in}}%
\pgfpathlineto{\pgfqpoint{0.896330in}{1.515476in}}%
\pgfpathlineto{\pgfqpoint{0.902251in}{1.413359in}}%
\pgfpathlineto{\pgfqpoint{0.902646in}{1.415108in}}%
\pgfpathlineto{\pgfqpoint{0.906988in}{1.439330in}}%
\pgfpathlineto{\pgfqpoint{0.907383in}{1.436196in}}%
\pgfpathlineto{\pgfqpoint{0.912120in}{1.370644in}}%
\pgfpathlineto{\pgfqpoint{0.914883in}{1.361779in}}%
\pgfpathlineto{\pgfqpoint{0.915673in}{1.358447in}}%
\pgfpathlineto{\pgfqpoint{0.916067in}{1.361105in}}%
\pgfpathlineto{\pgfqpoint{0.920015in}{1.376662in}}%
\pgfpathlineto{\pgfqpoint{0.920410in}{1.377409in}}%
\pgfpathlineto{\pgfqpoint{0.923962in}{1.318359in}}%
\pgfpathlineto{\pgfqpoint{0.925147in}{1.327400in}}%
\pgfpathlineto{\pgfqpoint{0.927910in}{1.368438in}}%
\pgfpathlineto{\pgfqpoint{0.928699in}{1.378005in}}%
\pgfpathlineto{\pgfqpoint{0.929489in}{1.371666in}}%
\pgfpathlineto{\pgfqpoint{0.935015in}{1.311456in}}%
\pgfpathlineto{\pgfqpoint{0.935410in}{1.316918in}}%
\pgfpathlineto{\pgfqpoint{0.938173in}{1.377714in}}%
\pgfpathlineto{\pgfqpoint{0.939752in}{1.407401in}}%
\pgfpathlineto{\pgfqpoint{0.940542in}{1.407027in}}%
\pgfpathlineto{\pgfqpoint{0.941726in}{1.404301in}}%
\pgfpathlineto{\pgfqpoint{0.942910in}{1.391538in}}%
\pgfpathlineto{\pgfqpoint{0.943700in}{1.394384in}}%
\pgfpathlineto{\pgfqpoint{0.946463in}{1.400026in}}%
\pgfpathlineto{\pgfqpoint{0.946858in}{1.399888in}}%
\pgfpathlineto{\pgfqpoint{0.948042in}{1.397993in}}%
\pgfpathlineto{\pgfqpoint{0.950411in}{1.385429in}}%
\pgfpathlineto{\pgfqpoint{0.952384in}{1.374350in}}%
\pgfpathlineto{\pgfqpoint{0.952779in}{1.375963in}}%
\pgfpathlineto{\pgfqpoint{0.954753in}{1.385428in}}%
\pgfpathlineto{\pgfqpoint{0.956332in}{1.407072in}}%
\pgfpathlineto{\pgfqpoint{0.957121in}{1.401818in}}%
\pgfpathlineto{\pgfqpoint{0.960279in}{1.369028in}}%
\pgfpathlineto{\pgfqpoint{0.966200in}{1.301631in}}%
\pgfpathlineto{\pgfqpoint{0.967385in}{1.309210in}}%
\pgfpathlineto{\pgfqpoint{0.967779in}{1.306538in}}%
\pgfpathlineto{\pgfqpoint{0.970937in}{1.277443in}}%
\pgfpathlineto{\pgfqpoint{0.972122in}{1.279973in}}%
\pgfpathlineto{\pgfqpoint{0.974095in}{1.288962in}}%
\pgfpathlineto{\pgfqpoint{0.974490in}{1.287168in}}%
\pgfpathlineto{\pgfqpoint{0.976069in}{1.280027in}}%
\pgfpathlineto{\pgfqpoint{0.976859in}{1.283418in}}%
\pgfpathlineto{\pgfqpoint{0.979622in}{1.338353in}}%
\pgfpathlineto{\pgfqpoint{0.980806in}{1.333309in}}%
\pgfpathlineto{\pgfqpoint{0.983175in}{1.265348in}}%
\pgfpathlineto{\pgfqpoint{0.984754in}{1.276358in}}%
\pgfpathlineto{\pgfqpoint{0.987912in}{1.308204in}}%
\pgfpathlineto{\pgfqpoint{0.988701in}{1.307212in}}%
\pgfpathlineto{\pgfqpoint{0.989885in}{1.302575in}}%
\pgfpathlineto{\pgfqpoint{0.991464in}{1.324385in}}%
\pgfpathlineto{\pgfqpoint{0.992649in}{1.319483in}}%
\pgfpathlineto{\pgfqpoint{0.996596in}{1.299788in}}%
\pgfpathlineto{\pgfqpoint{1.000938in}{1.279158in}}%
\pgfpathlineto{\pgfqpoint{1.002912in}{1.239459in}}%
\pgfpathlineto{\pgfqpoint{1.003702in}{1.242224in}}%
\pgfpathlineto{\pgfqpoint{1.006070in}{1.300463in}}%
\pgfpathlineto{\pgfqpoint{1.006860in}{1.288368in}}%
\pgfpathlineto{\pgfqpoint{1.008833in}{1.281119in}}%
\pgfpathlineto{\pgfqpoint{1.009228in}{1.282748in}}%
\pgfpathlineto{\pgfqpoint{1.009623in}{1.278678in}}%
\pgfpathlineto{\pgfqpoint{1.015149in}{1.225402in}}%
\pgfpathlineto{\pgfqpoint{1.017518in}{1.213328in}}%
\pgfpathlineto{\pgfqpoint{1.019492in}{1.210232in}}%
\pgfpathlineto{\pgfqpoint{1.019886in}{1.209241in}}%
\pgfpathlineto{\pgfqpoint{1.020281in}{1.212393in}}%
\pgfpathlineto{\pgfqpoint{1.025808in}{1.270485in}}%
\pgfpathlineto{\pgfqpoint{1.026992in}{1.267995in}}%
\pgfpathlineto{\pgfqpoint{1.031334in}{1.189848in}}%
\pgfpathlineto{\pgfqpoint{1.035676in}{1.135078in}}%
\pgfpathlineto{\pgfqpoint{1.036466in}{1.137173in}}%
\pgfpathlineto{\pgfqpoint{1.040808in}{1.174589in}}%
\pgfpathlineto{\pgfqpoint{1.045940in}{1.258144in}}%
\pgfpathlineto{\pgfqpoint{1.047519in}{1.270097in}}%
\pgfpathlineto{\pgfqpoint{1.049098in}{1.283253in}}%
\pgfpathlineto{\pgfqpoint{1.049492in}{1.280533in}}%
\pgfpathlineto{\pgfqpoint{1.052650in}{1.238777in}}%
\pgfpathlineto{\pgfqpoint{1.053045in}{1.239835in}}%
\pgfpathlineto{\pgfqpoint{1.054229in}{1.260332in}}%
\pgfpathlineto{\pgfqpoint{1.055019in}{1.254064in}}%
\pgfpathlineto{\pgfqpoint{1.058572in}{1.203883in}}%
\pgfpathlineto{\pgfqpoint{1.059361in}{1.213322in}}%
\pgfpathlineto{\pgfqpoint{1.060545in}{1.221636in}}%
\pgfpathlineto{\pgfqpoint{1.060940in}{1.219609in}}%
\pgfpathlineto{\pgfqpoint{1.061730in}{1.207233in}}%
\pgfpathlineto{\pgfqpoint{1.062519in}{1.211341in}}%
\pgfpathlineto{\pgfqpoint{1.066467in}{1.241190in}}%
\pgfpathlineto{\pgfqpoint{1.067256in}{1.234446in}}%
\pgfpathlineto{\pgfqpoint{1.069625in}{1.205031in}}%
\pgfpathlineto{\pgfqpoint{1.070019in}{1.209744in}}%
\pgfpathlineto{\pgfqpoint{1.077125in}{1.275297in}}%
\pgfpathlineto{\pgfqpoint{1.078309in}{1.282450in}}%
\pgfpathlineto{\pgfqpoint{1.078704in}{1.280105in}}%
\pgfpathlineto{\pgfqpoint{1.082257in}{1.226592in}}%
\pgfpathlineto{\pgfqpoint{1.085415in}{1.132346in}}%
\pgfpathlineto{\pgfqpoint{1.086599in}{1.135680in}}%
\pgfpathlineto{\pgfqpoint{1.087388in}{1.137720in}}%
\pgfpathlineto{\pgfqpoint{1.088967in}{1.158593in}}%
\pgfpathlineto{\pgfqpoint{1.089757in}{1.153171in}}%
\pgfpathlineto{\pgfqpoint{1.092125in}{1.134164in}}%
\pgfpathlineto{\pgfqpoint{1.092520in}{1.137744in}}%
\pgfpathlineto{\pgfqpoint{1.092915in}{1.144295in}}%
\pgfpathlineto{\pgfqpoint{1.093310in}{1.142681in}}%
\pgfpathlineto{\pgfqpoint{1.094494in}{1.117072in}}%
\pgfpathlineto{\pgfqpoint{1.095283in}{1.121784in}}%
\pgfpathlineto{\pgfqpoint{1.095678in}{1.126491in}}%
\pgfpathlineto{\pgfqpoint{1.096468in}{1.118051in}}%
\pgfpathlineto{\pgfqpoint{1.097257in}{1.109453in}}%
\pgfpathlineto{\pgfqpoint{1.097652in}{1.119893in}}%
\pgfpathlineto{\pgfqpoint{1.103968in}{1.197823in}}%
\pgfpathlineto{\pgfqpoint{1.104363in}{1.197924in}}%
\pgfpathlineto{\pgfqpoint{1.105152in}{1.209944in}}%
\pgfpathlineto{\pgfqpoint{1.106336in}{1.204442in}}%
\pgfpathlineto{\pgfqpoint{1.106731in}{1.203576in}}%
\pgfpathlineto{\pgfqpoint{1.107126in}{1.206261in}}%
\pgfpathlineto{\pgfqpoint{1.112258in}{1.228341in}}%
\pgfpathlineto{\pgfqpoint{1.113837in}{1.199574in}}%
\pgfpathlineto{\pgfqpoint{1.114626in}{1.204908in}}%
\pgfpathlineto{\pgfqpoint{1.116600in}{1.230463in}}%
\pgfpathlineto{\pgfqpoint{1.116995in}{1.218031in}}%
\pgfpathlineto{\pgfqpoint{1.119758in}{1.173350in}}%
\pgfpathlineto{\pgfqpoint{1.124890in}{1.138553in}}%
\pgfpathlineto{\pgfqpoint{1.125284in}{1.146112in}}%
\pgfpathlineto{\pgfqpoint{1.126863in}{1.177162in}}%
\pgfpathlineto{\pgfqpoint{1.128047in}{1.175517in}}%
\pgfpathlineto{\pgfqpoint{1.129232in}{1.167075in}}%
\pgfpathlineto{\pgfqpoint{1.133179in}{1.095382in}}%
\pgfpathlineto{\pgfqpoint{1.133969in}{1.105008in}}%
\pgfpathlineto{\pgfqpoint{1.135942in}{1.138742in}}%
\pgfpathlineto{\pgfqpoint{1.137127in}{1.132798in}}%
\pgfpathlineto{\pgfqpoint{1.137521in}{1.130125in}}%
\pgfpathlineto{\pgfqpoint{1.137916in}{1.133996in}}%
\pgfpathlineto{\pgfqpoint{1.138706in}{1.132591in}}%
\pgfpathlineto{\pgfqpoint{1.140285in}{1.146314in}}%
\pgfpathlineto{\pgfqpoint{1.140679in}{1.143616in}}%
\pgfpathlineto{\pgfqpoint{1.144627in}{1.110219in}}%
\pgfpathlineto{\pgfqpoint{1.145416in}{1.118138in}}%
\pgfpathlineto{\pgfqpoint{1.150548in}{1.166042in}}%
\pgfpathlineto{\pgfqpoint{1.150943in}{1.165033in}}%
\pgfpathlineto{\pgfqpoint{1.153311in}{1.153366in}}%
\pgfpathlineto{\pgfqpoint{1.157654in}{1.101778in}}%
\pgfpathlineto{\pgfqpoint{1.158838in}{1.110870in}}%
\pgfpathlineto{\pgfqpoint{1.159233in}{1.118317in}}%
\pgfpathlineto{\pgfqpoint{1.160022in}{1.117898in}}%
\pgfpathlineto{\pgfqpoint{1.163180in}{1.077244in}}%
\pgfpathlineto{\pgfqpoint{1.163970in}{1.078381in}}%
\pgfpathlineto{\pgfqpoint{1.165154in}{1.084077in}}%
\pgfpathlineto{\pgfqpoint{1.168707in}{1.128564in}}%
\pgfpathlineto{\pgfqpoint{1.170286in}{1.131990in}}%
\pgfpathlineto{\pgfqpoint{1.170680in}{1.130012in}}%
\pgfpathlineto{\pgfqpoint{1.171075in}{1.127992in}}%
\pgfpathlineto{\pgfqpoint{1.171470in}{1.130457in}}%
\pgfpathlineto{\pgfqpoint{1.174628in}{1.146762in}}%
\pgfpathlineto{\pgfqpoint{1.175023in}{1.145200in}}%
\pgfpathlineto{\pgfqpoint{1.176602in}{1.134723in}}%
\pgfpathlineto{\pgfqpoint{1.176996in}{1.135329in}}%
\pgfpathlineto{\pgfqpoint{1.179760in}{1.157302in}}%
\pgfpathlineto{\pgfqpoint{1.180154in}{1.159632in}}%
\pgfpathlineto{\pgfqpoint{1.180549in}{1.155989in}}%
\pgfpathlineto{\pgfqpoint{1.183312in}{1.134252in}}%
\pgfpathlineto{\pgfqpoint{1.184891in}{1.134862in}}%
\pgfpathlineto{\pgfqpoint{1.187655in}{1.096920in}}%
\pgfpathlineto{\pgfqpoint{1.188049in}{1.095206in}}%
\pgfpathlineto{\pgfqpoint{1.189234in}{1.068213in}}%
\pgfpathlineto{\pgfqpoint{1.190418in}{1.074707in}}%
\pgfpathlineto{\pgfqpoint{1.196734in}{1.006174in}}%
\pgfpathlineto{\pgfqpoint{1.197523in}{1.014287in}}%
\pgfpathlineto{\pgfqpoint{1.201076in}{1.056602in}}%
\pgfpathlineto{\pgfqpoint{1.205024in}{1.081068in}}%
\pgfpathlineto{\pgfqpoint{1.206208in}{1.097999in}}%
\pgfpathlineto{\pgfqpoint{1.207392in}{1.096744in}}%
\pgfpathlineto{\pgfqpoint{1.211734in}{1.138759in}}%
\pgfpathlineto{\pgfqpoint{1.213313in}{1.128950in}}%
\pgfpathlineto{\pgfqpoint{1.214103in}{1.131782in}}%
\pgfpathlineto{\pgfqpoint{1.217261in}{1.155280in}}%
\pgfpathlineto{\pgfqpoint{1.218445in}{1.153694in}}%
\pgfpathlineto{\pgfqpoint{1.220024in}{1.132956in}}%
\pgfpathlineto{\pgfqpoint{1.221603in}{1.105419in}}%
\pgfpathlineto{\pgfqpoint{1.221998in}{1.108222in}}%
\pgfpathlineto{\pgfqpoint{1.222392in}{1.111687in}}%
\pgfpathlineto{\pgfqpoint{1.222787in}{1.106858in}}%
\pgfpathlineto{\pgfqpoint{1.226340in}{1.070231in}}%
\pgfpathlineto{\pgfqpoint{1.226735in}{1.079109in}}%
\pgfpathlineto{\pgfqpoint{1.227129in}{1.082145in}}%
\pgfpathlineto{\pgfqpoint{1.228314in}{1.080505in}}%
\pgfpathlineto{\pgfqpoint{1.228708in}{1.080766in}}%
\pgfpathlineto{\pgfqpoint{1.229498in}{1.086325in}}%
\pgfpathlineto{\pgfqpoint{1.229893in}{1.084990in}}%
\pgfpathlineto{\pgfqpoint{1.230682in}{1.076318in}}%
\pgfpathlineto{\pgfqpoint{1.231472in}{1.079763in}}%
\pgfpathlineto{\pgfqpoint{1.232261in}{1.086505in}}%
\pgfpathlineto{\pgfqpoint{1.232656in}{1.081895in}}%
\pgfpathlineto{\pgfqpoint{1.233840in}{1.062122in}}%
\pgfpathlineto{\pgfqpoint{1.234235in}{1.067841in}}%
\pgfpathlineto{\pgfqpoint{1.236603in}{1.090745in}}%
\pgfpathlineto{\pgfqpoint{1.238577in}{1.070601in}}%
\pgfpathlineto{\pgfqpoint{1.239367in}{1.075283in}}%
\pgfpathlineto{\pgfqpoint{1.242919in}{1.097156in}}%
\pgfpathlineto{\pgfqpoint{1.243314in}{1.096914in}}%
\pgfpathlineto{\pgfqpoint{1.247656in}{1.061629in}}%
\pgfpathlineto{\pgfqpoint{1.248841in}{1.048835in}}%
\pgfpathlineto{\pgfqpoint{1.249630in}{1.051177in}}%
\pgfpathlineto{\pgfqpoint{1.250025in}{1.051718in}}%
\pgfpathlineto{\pgfqpoint{1.253183in}{0.988409in}}%
\pgfpathlineto{\pgfqpoint{1.254367in}{0.993923in}}%
\pgfpathlineto{\pgfqpoint{1.256736in}{0.998921in}}%
\pgfpathlineto{\pgfqpoint{1.257525in}{0.988371in}}%
\pgfpathlineto{\pgfqpoint{1.257920in}{0.993832in}}%
\pgfpathlineto{\pgfqpoint{1.263446in}{1.048303in}}%
\pgfpathlineto{\pgfqpoint{1.266210in}{1.070795in}}%
\pgfpathlineto{\pgfqpoint{1.268578in}{1.088444in}}%
\pgfpathlineto{\pgfqpoint{1.270157in}{1.109748in}}%
\pgfpathlineto{\pgfqpoint{1.274105in}{1.166948in}}%
\pgfpathlineto{\pgfqpoint{1.274499in}{1.170861in}}%
\pgfpathlineto{\pgfqpoint{1.275289in}{1.165251in}}%
\pgfpathlineto{\pgfqpoint{1.278447in}{1.122246in}}%
\pgfpathlineto{\pgfqpoint{1.281605in}{1.087170in}}%
\pgfpathlineto{\pgfqpoint{1.282394in}{1.085037in}}%
\pgfpathlineto{\pgfqpoint{1.283973in}{1.070737in}}%
\pgfpathlineto{\pgfqpoint{1.284368in}{1.071836in}}%
\pgfpathlineto{\pgfqpoint{1.285158in}{1.065889in}}%
\pgfpathlineto{\pgfqpoint{1.285552in}{1.073897in}}%
\pgfpathlineto{\pgfqpoint{1.288710in}{1.115202in}}%
\pgfpathlineto{\pgfqpoint{1.289105in}{1.112331in}}%
\pgfpathlineto{\pgfqpoint{1.293052in}{1.035431in}}%
\pgfpathlineto{\pgfqpoint{1.293447in}{1.039806in}}%
\pgfpathlineto{\pgfqpoint{1.297000in}{1.078639in}}%
\pgfpathlineto{\pgfqpoint{1.297789in}{1.071735in}}%
\pgfpathlineto{\pgfqpoint{1.300553in}{1.038980in}}%
\pgfpathlineto{\pgfqpoint{1.301342in}{1.039080in}}%
\pgfpathlineto{\pgfqpoint{1.301737in}{1.039758in}}%
\pgfpathlineto{\pgfqpoint{1.304500in}{1.068796in}}%
\pgfpathlineto{\pgfqpoint{1.305684in}{1.069557in}}%
\pgfpathlineto{\pgfqpoint{1.308842in}{1.035672in}}%
\pgfpathlineto{\pgfqpoint{1.309237in}{1.035332in}}%
\pgfpathlineto{\pgfqpoint{1.313579in}{1.004530in}}%
\pgfpathlineto{\pgfqpoint{1.315158in}{0.989538in}}%
\pgfpathlineto{\pgfqpoint{1.315948in}{0.965162in}}%
\pgfpathlineto{\pgfqpoint{1.316737in}{0.976843in}}%
\pgfpathlineto{\pgfqpoint{1.317132in}{0.981845in}}%
\pgfpathlineto{\pgfqpoint{1.317922in}{0.977083in}}%
\pgfpathlineto{\pgfqpoint{1.318316in}{0.977168in}}%
\pgfpathlineto{\pgfqpoint{1.318711in}{0.974082in}}%
\pgfpathlineto{\pgfqpoint{1.319106in}{0.975035in}}%
\pgfpathlineto{\pgfqpoint{1.319895in}{0.988156in}}%
\pgfpathlineto{\pgfqpoint{1.320685in}{0.983181in}}%
\pgfpathlineto{\pgfqpoint{1.322659in}{0.962521in}}%
\pgfpathlineto{\pgfqpoint{1.323053in}{0.969083in}}%
\pgfpathlineto{\pgfqpoint{1.325422in}{0.987720in}}%
\pgfpathlineto{\pgfqpoint{1.326211in}{0.985978in}}%
\pgfpathlineto{\pgfqpoint{1.328975in}{1.057858in}}%
\pgfpathlineto{\pgfqpoint{1.329369in}{1.057094in}}%
\pgfpathlineto{\pgfqpoint{1.330948in}{1.028501in}}%
\pgfpathlineto{\pgfqpoint{1.332133in}{1.000835in}}%
\pgfpathlineto{\pgfqpoint{1.332922in}{1.010920in}}%
\pgfpathlineto{\pgfqpoint{1.336080in}{1.060907in}}%
\pgfpathlineto{\pgfqpoint{1.336870in}{1.062092in}}%
\pgfpathlineto{\pgfqpoint{1.339238in}{1.080167in}}%
\pgfpathlineto{\pgfqpoint{1.341212in}{1.082172in}}%
\pgfpathlineto{\pgfqpoint{1.342001in}{1.081669in}}%
\pgfpathlineto{\pgfqpoint{1.345554in}{1.051899in}}%
\pgfpathlineto{\pgfqpoint{1.346738in}{1.065193in}}%
\pgfpathlineto{\pgfqpoint{1.347133in}{1.062252in}}%
\pgfpathlineto{\pgfqpoint{1.351475in}{0.994606in}}%
\pgfpathlineto{\pgfqpoint{1.351870in}{0.994736in}}%
\pgfpathlineto{\pgfqpoint{1.352265in}{1.000128in}}%
\pgfpathlineto{\pgfqpoint{1.352660in}{0.994770in}}%
\pgfpathlineto{\pgfqpoint{1.356212in}{0.958465in}}%
\pgfpathlineto{\pgfqpoint{1.356607in}{0.958442in}}%
\pgfpathlineto{\pgfqpoint{1.357397in}{0.969978in}}%
\pgfpathlineto{\pgfqpoint{1.358186in}{0.961058in}}%
\pgfpathlineto{\pgfqpoint{1.358581in}{0.956022in}}%
\pgfpathlineto{\pgfqpoint{1.359370in}{0.959910in}}%
\pgfpathlineto{\pgfqpoint{1.362923in}{0.986431in}}%
\pgfpathlineto{\pgfqpoint{1.364897in}{0.997345in}}%
\pgfpathlineto{\pgfqpoint{1.365292in}{0.997069in}}%
\pgfpathlineto{\pgfqpoint{1.366081in}{0.994230in}}%
\pgfpathlineto{\pgfqpoint{1.366476in}{0.998170in}}%
\pgfpathlineto{\pgfqpoint{1.366871in}{0.998620in}}%
\pgfpathlineto{\pgfqpoint{1.367265in}{0.996818in}}%
\pgfpathlineto{\pgfqpoint{1.368055in}{0.997751in}}%
\pgfpathlineto{\pgfqpoint{1.369239in}{0.984590in}}%
\pgfpathlineto{\pgfqpoint{1.373581in}{0.943935in}}%
\pgfpathlineto{\pgfqpoint{1.374371in}{0.951421in}}%
\pgfpathlineto{\pgfqpoint{1.376739in}{0.983480in}}%
\pgfpathlineto{\pgfqpoint{1.377923in}{0.974378in}}%
\pgfpathlineto{\pgfqpoint{1.379108in}{0.966806in}}%
\pgfpathlineto{\pgfqpoint{1.380687in}{0.957910in}}%
\pgfpathlineto{\pgfqpoint{1.381081in}{0.961433in}}%
\pgfpathlineto{\pgfqpoint{1.381476in}{0.963980in}}%
\pgfpathlineto{\pgfqpoint{1.381871in}{0.957619in}}%
\pgfpathlineto{\pgfqpoint{1.385818in}{0.914412in}}%
\pgfpathlineto{\pgfqpoint{1.387003in}{0.912997in}}%
\pgfpathlineto{\pgfqpoint{1.388187in}{0.938726in}}%
\pgfpathlineto{\pgfqpoint{1.389371in}{0.927657in}}%
\pgfpathlineto{\pgfqpoint{1.390950in}{0.893246in}}%
\pgfpathlineto{\pgfqpoint{1.392134in}{0.906351in}}%
\pgfpathlineto{\pgfqpoint{1.392529in}{0.905350in}}%
\pgfpathlineto{\pgfqpoint{1.402398in}{1.004619in}}%
\pgfpathlineto{\pgfqpoint{1.402793in}{1.004145in}}%
\pgfpathlineto{\pgfqpoint{1.406740in}{0.973083in}}%
\pgfpathlineto{\pgfqpoint{1.407530in}{0.976311in}}%
\pgfpathlineto{\pgfqpoint{1.407924in}{0.975978in}}%
\pgfpathlineto{\pgfqpoint{1.411082in}{0.941877in}}%
\pgfpathlineto{\pgfqpoint{1.411477in}{0.943113in}}%
\pgfpathlineto{\pgfqpoint{1.413846in}{0.947055in}}%
\pgfpathlineto{\pgfqpoint{1.414240in}{0.946838in}}%
\pgfpathlineto{\pgfqpoint{1.414635in}{0.945993in}}%
\pgfpathlineto{\pgfqpoint{1.417398in}{0.916182in}}%
\pgfpathlineto{\pgfqpoint{1.417793in}{0.918502in}}%
\pgfpathlineto{\pgfqpoint{1.420556in}{0.950020in}}%
\pgfpathlineto{\pgfqpoint{1.420951in}{0.950827in}}%
\pgfpathlineto{\pgfqpoint{1.421346in}{0.956887in}}%
\pgfpathlineto{\pgfqpoint{1.422135in}{0.947079in}}%
\pgfpathlineto{\pgfqpoint{1.423714in}{0.930851in}}%
\pgfpathlineto{\pgfqpoint{1.424899in}{0.932594in}}%
\pgfpathlineto{\pgfqpoint{1.425293in}{0.932940in}}%
\pgfpathlineto{\pgfqpoint{1.426083in}{0.921510in}}%
\pgfpathlineto{\pgfqpoint{1.426478in}{0.930699in}}%
\pgfpathlineto{\pgfqpoint{1.429241in}{0.976129in}}%
\pgfpathlineto{\pgfqpoint{1.433583in}{0.938886in}}%
\pgfpathlineto{\pgfqpoint{1.439899in}{0.978764in}}%
\pgfpathlineto{\pgfqpoint{1.441083in}{0.975997in}}%
\pgfpathlineto{\pgfqpoint{1.445820in}{0.928397in}}%
\pgfpathlineto{\pgfqpoint{1.446215in}{0.930796in}}%
\pgfpathlineto{\pgfqpoint{1.446610in}{0.931695in}}%
\pgfpathlineto{\pgfqpoint{1.448978in}{0.905220in}}%
\pgfpathlineto{\pgfqpoint{1.450557in}{0.879571in}}%
\pgfpathlineto{\pgfqpoint{1.450952in}{0.887253in}}%
\pgfpathlineto{\pgfqpoint{1.452926in}{0.939664in}}%
\pgfpathlineto{\pgfqpoint{1.453715in}{0.938293in}}%
\pgfpathlineto{\pgfqpoint{1.454110in}{0.934062in}}%
\pgfpathlineto{\pgfqpoint{1.454900in}{0.935489in}}%
\pgfpathlineto{\pgfqpoint{1.455689in}{0.942268in}}%
\pgfpathlineto{\pgfqpoint{1.456873in}{0.939829in}}%
\pgfpathlineto{\pgfqpoint{1.458058in}{0.928642in}}%
\pgfpathlineto{\pgfqpoint{1.458847in}{0.938179in}}%
\pgfpathlineto{\pgfqpoint{1.460031in}{0.941165in}}%
\pgfpathlineto{\pgfqpoint{1.464373in}{0.999222in}}%
\pgfpathlineto{\pgfqpoint{1.464768in}{0.999043in}}%
\pgfpathlineto{\pgfqpoint{1.465163in}{0.998748in}}%
\pgfpathlineto{\pgfqpoint{1.468321in}{0.951854in}}%
\pgfpathlineto{\pgfqpoint{1.468716in}{0.952942in}}%
\pgfpathlineto{\pgfqpoint{1.470689in}{0.956346in}}%
\pgfpathlineto{\pgfqpoint{1.475032in}{0.922222in}}%
\pgfpathlineto{\pgfqpoint{1.477005in}{0.901011in}}%
\pgfpathlineto{\pgfqpoint{1.478584in}{0.926125in}}%
\pgfpathlineto{\pgfqpoint{1.479769in}{0.920819in}}%
\pgfpathlineto{\pgfqpoint{1.480163in}{0.918822in}}%
\pgfpathlineto{\pgfqpoint{1.480558in}{0.921233in}}%
\pgfpathlineto{\pgfqpoint{1.484506in}{0.941459in}}%
\pgfpathlineto{\pgfqpoint{1.484900in}{0.934317in}}%
\pgfpathlineto{\pgfqpoint{1.487664in}{0.904782in}}%
\pgfpathlineto{\pgfqpoint{1.488848in}{0.895293in}}%
\pgfpathlineto{\pgfqpoint{1.489243in}{0.896054in}}%
\pgfpathlineto{\pgfqpoint{1.490032in}{0.910659in}}%
\pgfpathlineto{\pgfqpoint{1.490822in}{0.902341in}}%
\pgfpathlineto{\pgfqpoint{1.495164in}{0.879122in}}%
\pgfpathlineto{\pgfqpoint{1.495559in}{0.881010in}}%
\pgfpathlineto{\pgfqpoint{1.498322in}{0.915379in}}%
\pgfpathlineto{\pgfqpoint{1.498717in}{0.913920in}}%
\pgfpathlineto{\pgfqpoint{1.499111in}{0.912036in}}%
\pgfpathlineto{\pgfqpoint{1.499901in}{0.915446in}}%
\pgfpathlineto{\pgfqpoint{1.502664in}{0.930977in}}%
\pgfpathlineto{\pgfqpoint{1.503454in}{0.924028in}}%
\pgfpathlineto{\pgfqpoint{1.504243in}{0.916115in}}%
\pgfpathlineto{\pgfqpoint{1.505033in}{0.919092in}}%
\pgfpathlineto{\pgfqpoint{1.505822in}{0.927763in}}%
\pgfpathlineto{\pgfqpoint{1.509375in}{0.961611in}}%
\pgfpathlineto{\pgfqpoint{1.509770in}{0.963390in}}%
\pgfpathlineto{\pgfqpoint{1.510559in}{0.960081in}}%
\pgfpathlineto{\pgfqpoint{1.514901in}{0.895397in}}%
\pgfpathlineto{\pgfqpoint{1.515296in}{0.900015in}}%
\pgfpathlineto{\pgfqpoint{1.517665in}{0.953474in}}%
\pgfpathlineto{\pgfqpoint{1.518849in}{0.929663in}}%
\pgfpathlineto{\pgfqpoint{1.520033in}{0.921042in}}%
\pgfpathlineto{\pgfqpoint{1.520428in}{0.923614in}}%
\pgfpathlineto{\pgfqpoint{1.522402in}{0.950685in}}%
\pgfpathlineto{\pgfqpoint{1.523191in}{0.944195in}}%
\pgfpathlineto{\pgfqpoint{1.526349in}{0.918123in}}%
\pgfpathlineto{\pgfqpoint{1.526744in}{0.921610in}}%
\pgfpathlineto{\pgfqpoint{1.530691in}{0.971622in}}%
\pgfpathlineto{\pgfqpoint{1.535034in}{0.920438in}}%
\pgfpathlineto{\pgfqpoint{1.535428in}{0.922887in}}%
\pgfpathlineto{\pgfqpoint{1.535823in}{0.924775in}}%
\pgfpathlineto{\pgfqpoint{1.536218in}{0.923588in}}%
\pgfpathlineto{\pgfqpoint{1.538981in}{0.908443in}}%
\pgfpathlineto{\pgfqpoint{1.539376in}{0.911312in}}%
\pgfpathlineto{\pgfqpoint{1.540165in}{0.908038in}}%
\pgfpathlineto{\pgfqpoint{1.541744in}{0.901562in}}%
\pgfpathlineto{\pgfqpoint{1.544113in}{0.866355in}}%
\pgfpathlineto{\pgfqpoint{1.544902in}{0.867167in}}%
\pgfpathlineto{\pgfqpoint{1.545297in}{0.865818in}}%
\pgfpathlineto{\pgfqpoint{1.546086in}{0.868359in}}%
\pgfpathlineto{\pgfqpoint{1.547665in}{0.880643in}}%
\pgfpathlineto{\pgfqpoint{1.548060in}{0.877861in}}%
\pgfpathlineto{\pgfqpoint{1.548455in}{0.873412in}}%
\pgfpathlineto{\pgfqpoint{1.549244in}{0.875277in}}%
\pgfpathlineto{\pgfqpoint{1.550034in}{0.890870in}}%
\pgfpathlineto{\pgfqpoint{1.550823in}{0.884076in}}%
\pgfpathlineto{\pgfqpoint{1.551218in}{0.879294in}}%
\pgfpathlineto{\pgfqpoint{1.551613in}{0.886338in}}%
\pgfpathlineto{\pgfqpoint{1.554376in}{0.920192in}}%
\pgfpathlineto{\pgfqpoint{1.554771in}{0.913840in}}%
\pgfpathlineto{\pgfqpoint{1.557929in}{0.878540in}}%
\pgfpathlineto{\pgfqpoint{1.558718in}{0.874606in}}%
\pgfpathlineto{\pgfqpoint{1.559113in}{0.879108in}}%
\pgfpathlineto{\pgfqpoint{1.559508in}{0.883477in}}%
\pgfpathlineto{\pgfqpoint{1.560297in}{0.877010in}}%
\pgfpathlineto{\pgfqpoint{1.561482in}{0.867614in}}%
\pgfpathlineto{\pgfqpoint{1.562271in}{0.872207in}}%
\pgfpathlineto{\pgfqpoint{1.564640in}{0.879251in}}%
\pgfpathlineto{\pgfqpoint{1.565429in}{0.877224in}}%
\pgfpathlineto{\pgfqpoint{1.567403in}{0.913093in}}%
\pgfpathlineto{\pgfqpoint{1.568192in}{0.912856in}}%
\pgfpathlineto{\pgfqpoint{1.568587in}{0.914216in}}%
\pgfpathlineto{\pgfqpoint{1.568982in}{0.910069in}}%
\pgfpathlineto{\pgfqpoint{1.572140in}{0.890725in}}%
\pgfpathlineto{\pgfqpoint{1.576087in}{0.917446in}}%
\pgfpathlineto{\pgfqpoint{1.577272in}{0.931165in}}%
\pgfpathlineto{\pgfqpoint{1.578061in}{0.925844in}}%
\pgfpathlineto{\pgfqpoint{1.579245in}{0.931407in}}%
\pgfpathlineto{\pgfqpoint{1.582798in}{0.953950in}}%
\pgfpathlineto{\pgfqpoint{1.583588in}{0.954534in}}%
\pgfpathlineto{\pgfqpoint{1.583982in}{0.948737in}}%
\pgfpathlineto{\pgfqpoint{1.587140in}{0.922214in}}%
\pgfpathlineto{\pgfqpoint{1.588719in}{0.940090in}}%
\pgfpathlineto{\pgfqpoint{1.589114in}{0.933346in}}%
\pgfpathlineto{\pgfqpoint{1.590693in}{0.909114in}}%
\pgfpathlineto{\pgfqpoint{1.591877in}{0.913261in}}%
\pgfpathlineto{\pgfqpoint{1.592667in}{0.919327in}}%
\pgfpathlineto{\pgfqpoint{1.593062in}{0.916780in}}%
\pgfpathlineto{\pgfqpoint{1.595035in}{0.898398in}}%
\pgfpathlineto{\pgfqpoint{1.595825in}{0.903942in}}%
\pgfpathlineto{\pgfqpoint{1.596220in}{0.909630in}}%
\pgfpathlineto{\pgfqpoint{1.597009in}{0.904160in}}%
\pgfpathlineto{\pgfqpoint{1.598193in}{0.900272in}}%
\pgfpathlineto{\pgfqpoint{1.598588in}{0.901340in}}%
\pgfpathlineto{\pgfqpoint{1.600167in}{0.905618in}}%
\pgfpathlineto{\pgfqpoint{1.603720in}{0.877735in}}%
\pgfpathlineto{\pgfqpoint{1.608062in}{0.853199in}}%
\pgfpathlineto{\pgfqpoint{1.608457in}{0.857519in}}%
\pgfpathlineto{\pgfqpoint{1.608852in}{0.853357in}}%
\pgfpathlineto{\pgfqpoint{1.610036in}{0.838151in}}%
\pgfpathlineto{\pgfqpoint{1.610431in}{0.840115in}}%
\pgfpathlineto{\pgfqpoint{1.611615in}{0.871253in}}%
\pgfpathlineto{\pgfqpoint{1.612799in}{0.861605in}}%
\pgfpathlineto{\pgfqpoint{1.613194in}{0.861611in}}%
\pgfpathlineto{\pgfqpoint{1.614378in}{0.877684in}}%
\pgfpathlineto{\pgfqpoint{1.615168in}{0.868354in}}%
\pgfpathlineto{\pgfqpoint{1.617141in}{0.839289in}}%
\pgfpathlineto{\pgfqpoint{1.617536in}{0.839679in}}%
\pgfpathlineto{\pgfqpoint{1.619115in}{0.847518in}}%
\pgfpathlineto{\pgfqpoint{1.623063in}{0.903223in}}%
\pgfpathlineto{\pgfqpoint{1.623852in}{0.895812in}}%
\pgfpathlineto{\pgfqpoint{1.627405in}{0.861154in}}%
\pgfpathlineto{\pgfqpoint{1.627799in}{0.861643in}}%
\pgfpathlineto{\pgfqpoint{1.629378in}{0.890297in}}%
\pgfpathlineto{\pgfqpoint{1.629773in}{0.886388in}}%
\pgfpathlineto{\pgfqpoint{1.630168in}{0.882034in}}%
\pgfpathlineto{\pgfqpoint{1.630957in}{0.887293in}}%
\pgfpathlineto{\pgfqpoint{1.633326in}{0.924041in}}%
\pgfpathlineto{\pgfqpoint{1.633721in}{0.917800in}}%
\pgfpathlineto{\pgfqpoint{1.636879in}{0.882660in}}%
\pgfpathlineto{\pgfqpoint{1.638063in}{0.870156in}}%
\pgfpathlineto{\pgfqpoint{1.640037in}{0.846138in}}%
\pgfpathlineto{\pgfqpoint{1.640431in}{0.848954in}}%
\pgfpathlineto{\pgfqpoint{1.643195in}{0.878923in}}%
\pgfpathlineto{\pgfqpoint{1.645958in}{0.906518in}}%
\pgfpathlineto{\pgfqpoint{1.649116in}{0.864507in}}%
\pgfpathlineto{\pgfqpoint{1.649511in}{0.873099in}}%
\pgfpathlineto{\pgfqpoint{1.651484in}{0.904953in}}%
\pgfpathlineto{\pgfqpoint{1.652274in}{0.903986in}}%
\pgfpathlineto{\pgfqpoint{1.653458in}{0.896821in}}%
\pgfpathlineto{\pgfqpoint{1.657011in}{0.862040in}}%
\pgfpathlineto{\pgfqpoint{1.657800in}{0.869716in}}%
\pgfpathlineto{\pgfqpoint{1.660169in}{0.875720in}}%
\pgfpathlineto{\pgfqpoint{1.658590in}{0.869044in}}%
\pgfpathlineto{\pgfqpoint{1.660958in}{0.874787in}}%
\pgfpathlineto{\pgfqpoint{1.662932in}{0.881124in}}%
\pgfpathlineto{\pgfqpoint{1.664511in}{0.867798in}}%
\pgfpathlineto{\pgfqpoint{1.665695in}{0.871301in}}%
\pgfpathlineto{\pgfqpoint{1.666485in}{0.868651in}}%
\pgfpathlineto{\pgfqpoint{1.667669in}{0.884865in}}%
\pgfpathlineto{\pgfqpoint{1.669248in}{0.901380in}}%
\pgfpathlineto{\pgfqpoint{1.669643in}{0.897824in}}%
\pgfpathlineto{\pgfqpoint{1.673590in}{0.845147in}}%
\pgfpathlineto{\pgfqpoint{1.673985in}{0.847393in}}%
\pgfpathlineto{\pgfqpoint{1.674775in}{0.852385in}}%
\pgfpathlineto{\pgfqpoint{1.675169in}{0.845143in}}%
\pgfpathlineto{\pgfqpoint{1.676354in}{0.835381in}}%
\pgfpathlineto{\pgfqpoint{1.676748in}{0.841054in}}%
\pgfpathlineto{\pgfqpoint{1.677143in}{0.846906in}}%
\pgfpathlineto{\pgfqpoint{1.678327in}{0.841577in}}%
\pgfpathlineto{\pgfqpoint{1.678722in}{0.844347in}}%
\pgfpathlineto{\pgfqpoint{1.679512in}{0.840844in}}%
\pgfpathlineto{\pgfqpoint{1.681485in}{0.838600in}}%
\pgfpathlineto{\pgfqpoint{1.683064in}{0.857467in}}%
\pgfpathlineto{\pgfqpoint{1.684249in}{0.854720in}}%
\pgfpathlineto{\pgfqpoint{1.685433in}{0.864610in}}%
\pgfpathlineto{\pgfqpoint{1.686222in}{0.861949in}}%
\pgfpathlineto{\pgfqpoint{1.687012in}{0.856488in}}%
\pgfpathlineto{\pgfqpoint{1.687407in}{0.860216in}}%
\pgfpathlineto{\pgfqpoint{1.689775in}{0.880986in}}%
\pgfpathlineto{\pgfqpoint{1.690565in}{0.880508in}}%
\pgfpathlineto{\pgfqpoint{1.691749in}{0.876219in}}%
\pgfpathlineto{\pgfqpoint{1.696881in}{0.851722in}}%
\pgfpathlineto{\pgfqpoint{1.697275in}{0.853005in}}%
\pgfpathlineto{\pgfqpoint{1.700828in}{0.863096in}}%
\pgfpathlineto{\pgfqpoint{1.701618in}{0.862059in}}%
\pgfpathlineto{\pgfqpoint{1.702407in}{0.865246in}}%
\pgfpathlineto{\pgfqpoint{1.702802in}{0.861968in}}%
\pgfpathlineto{\pgfqpoint{1.703197in}{0.858636in}}%
\pgfpathlineto{\pgfqpoint{1.703986in}{0.863677in}}%
\pgfpathlineto{\pgfqpoint{1.708723in}{0.839211in}}%
\pgfpathlineto{\pgfqpoint{1.709513in}{0.843082in}}%
\pgfpathlineto{\pgfqpoint{1.711486in}{0.858913in}}%
\pgfpathlineto{\pgfqpoint{1.712276in}{0.864094in}}%
\pgfpathlineto{\pgfqpoint{1.713065in}{0.863137in}}%
\pgfpathlineto{\pgfqpoint{1.713855in}{0.863510in}}%
\pgfpathlineto{\pgfqpoint{1.716223in}{0.889973in}}%
\pgfpathlineto{\pgfqpoint{1.716618in}{0.887393in}}%
\pgfpathlineto{\pgfqpoint{1.717802in}{0.874619in}}%
\pgfpathlineto{\pgfqpoint{1.718592in}{0.877348in}}%
\pgfpathlineto{\pgfqpoint{1.720960in}{0.885036in}}%
\pgfpathlineto{\pgfqpoint{1.721355in}{0.883402in}}%
\pgfpathlineto{\pgfqpoint{1.724513in}{0.874026in}}%
\pgfpathlineto{\pgfqpoint{1.724908in}{0.874264in}}%
\pgfpathlineto{\pgfqpoint{1.726487in}{0.879260in}}%
\pgfpathlineto{\pgfqpoint{1.726881in}{0.875627in}}%
\pgfpathlineto{\pgfqpoint{1.727671in}{0.879219in}}%
\pgfpathlineto{\pgfqpoint{1.728855in}{0.882761in}}%
\pgfpathlineto{\pgfqpoint{1.729250in}{0.879493in}}%
\pgfpathlineto{\pgfqpoint{1.736750in}{0.807909in}}%
\pgfpathlineto{\pgfqpoint{1.737145in}{0.814094in}}%
\pgfpathlineto{\pgfqpoint{1.737934in}{0.806585in}}%
\pgfpathlineto{\pgfqpoint{1.739513in}{0.795606in}}%
\pgfpathlineto{\pgfqpoint{1.739908in}{0.797214in}}%
\pgfpathlineto{\pgfqpoint{1.740698in}{0.808630in}}%
\pgfpathlineto{\pgfqpoint{1.741092in}{0.817219in}}%
\pgfpathlineto{\pgfqpoint{1.742277in}{0.810232in}}%
\pgfpathlineto{\pgfqpoint{1.743066in}{0.808976in}}%
\pgfpathlineto{\pgfqpoint{1.743461in}{0.809702in}}%
\pgfpathlineto{\pgfqpoint{1.745829in}{0.820536in}}%
\pgfpathlineto{\pgfqpoint{1.746224in}{0.817041in}}%
\pgfpathlineto{\pgfqpoint{1.747408in}{0.820041in}}%
\pgfpathlineto{\pgfqpoint{1.751751in}{0.844799in}}%
\pgfpathlineto{\pgfqpoint{1.755303in}{0.891880in}}%
\pgfpathlineto{\pgfqpoint{1.756093in}{0.885316in}}%
\pgfpathlineto{\pgfqpoint{1.761619in}{0.846870in}}%
\pgfpathlineto{\pgfqpoint{1.762014in}{0.844974in}}%
\pgfpathlineto{\pgfqpoint{1.762409in}{0.847779in}}%
\pgfpathlineto{\pgfqpoint{1.763593in}{0.860822in}}%
\pgfpathlineto{\pgfqpoint{1.764383in}{0.855318in}}%
\pgfpathlineto{\pgfqpoint{1.767541in}{0.832899in}}%
\pgfpathlineto{\pgfqpoint{1.767935in}{0.833895in}}%
\pgfpathlineto{\pgfqpoint{1.768330in}{0.831428in}}%
\pgfpathlineto{\pgfqpoint{1.768725in}{0.829101in}}%
\pgfpathlineto{\pgfqpoint{1.769120in}{0.829198in}}%
\pgfpathlineto{\pgfqpoint{1.769514in}{0.833803in}}%
\pgfpathlineto{\pgfqpoint{1.770304in}{0.830131in}}%
\pgfpathlineto{\pgfqpoint{1.771093in}{0.826047in}}%
\pgfpathlineto{\pgfqpoint{1.771488in}{0.827820in}}%
\pgfpathlineto{\pgfqpoint{1.774251in}{0.856304in}}%
\pgfpathlineto{\pgfqpoint{1.774646in}{0.850966in}}%
\pgfpathlineto{\pgfqpoint{1.775041in}{0.848677in}}%
\pgfpathlineto{\pgfqpoint{1.775436in}{0.852569in}}%
\pgfpathlineto{\pgfqpoint{1.776225in}{0.858409in}}%
\pgfpathlineto{\pgfqpoint{1.778199in}{0.843304in}}%
\pgfpathlineto{\pgfqpoint{1.779383in}{0.860593in}}%
\pgfpathlineto{\pgfqpoint{1.780173in}{0.855990in}}%
\pgfpathlineto{\pgfqpoint{1.783725in}{0.828110in}}%
\pgfpathlineto{\pgfqpoint{1.784120in}{0.832044in}}%
\pgfpathlineto{\pgfqpoint{1.786883in}{0.859145in}}%
\pgfpathlineto{\pgfqpoint{1.787673in}{0.854143in}}%
\pgfpathlineto{\pgfqpoint{1.788068in}{0.854298in}}%
\pgfpathlineto{\pgfqpoint{1.788857in}{0.859279in}}%
\pgfpathlineto{\pgfqpoint{1.789252in}{0.857617in}}%
\pgfpathlineto{\pgfqpoint{1.789647in}{0.853871in}}%
\pgfpathlineto{\pgfqpoint{1.790831in}{0.857267in}}%
\pgfpathlineto{\pgfqpoint{1.791226in}{0.858279in}}%
\pgfpathlineto{\pgfqpoint{1.791620in}{0.856096in}}%
\pgfpathlineto{\pgfqpoint{1.795568in}{0.808923in}}%
\pgfpathlineto{\pgfqpoint{1.796357in}{0.815720in}}%
\pgfpathlineto{\pgfqpoint{1.797541in}{0.827663in}}%
\pgfpathlineto{\pgfqpoint{1.798331in}{0.825745in}}%
\pgfpathlineto{\pgfqpoint{1.799515in}{0.798582in}}%
\pgfpathlineto{\pgfqpoint{1.800305in}{0.777894in}}%
\pgfpathlineto{\pgfqpoint{1.801094in}{0.792247in}}%
\pgfpathlineto{\pgfqpoint{1.804252in}{0.804652in}}%
\pgfpathlineto{\pgfqpoint{1.805436in}{0.808828in}}%
\pgfpathlineto{\pgfqpoint{1.807015in}{0.831353in}}%
\pgfpathlineto{\pgfqpoint{1.807410in}{0.824482in}}%
\pgfpathlineto{\pgfqpoint{1.807805in}{0.820969in}}%
\pgfpathlineto{\pgfqpoint{1.808594in}{0.823626in}}%
\pgfpathlineto{\pgfqpoint{1.808989in}{0.824285in}}%
\pgfpathlineto{\pgfqpoint{1.809384in}{0.822696in}}%
\pgfpathlineto{\pgfqpoint{1.810568in}{0.811408in}}%
\pgfpathlineto{\pgfqpoint{1.811358in}{0.813410in}}%
\pgfpathlineto{\pgfqpoint{1.812147in}{0.817271in}}%
\pgfpathlineto{\pgfqpoint{1.815700in}{0.860943in}}%
\pgfpathlineto{\pgfqpoint{1.816884in}{0.862206in}}%
\pgfpathlineto{\pgfqpoint{1.818068in}{0.865809in}}%
\pgfpathlineto{\pgfqpoint{1.818463in}{0.864777in}}%
\pgfpathlineto{\pgfqpoint{1.819647in}{0.860537in}}%
\pgfpathlineto{\pgfqpoint{1.820832in}{0.869610in}}%
\pgfpathlineto{\pgfqpoint{1.821621in}{0.868368in}}%
\pgfpathlineto{\pgfqpoint{1.822805in}{0.863072in}}%
\pgfpathlineto{\pgfqpoint{1.823200in}{0.861333in}}%
\pgfpathlineto{\pgfqpoint{1.823990in}{0.861706in}}%
\pgfpathlineto{\pgfqpoint{1.824384in}{0.863451in}}%
\pgfpathlineto{\pgfqpoint{1.824779in}{0.860531in}}%
\pgfpathlineto{\pgfqpoint{1.827542in}{0.835831in}}%
\pgfpathlineto{\pgfqpoint{1.828332in}{0.842700in}}%
\pgfpathlineto{\pgfqpoint{1.828727in}{0.842891in}}%
\pgfpathlineto{\pgfqpoint{1.833069in}{0.823015in}}%
\pgfpathlineto{\pgfqpoint{1.833858in}{0.831740in}}%
\pgfpathlineto{\pgfqpoint{1.835043in}{0.845542in}}%
\pgfpathlineto{\pgfqpoint{1.835832in}{0.840557in}}%
\pgfpathlineto{\pgfqpoint{1.838201in}{0.802617in}}%
\pgfpathlineto{\pgfqpoint{1.842938in}{0.863291in}}%
\pgfpathlineto{\pgfqpoint{1.843727in}{0.859558in}}%
\pgfpathlineto{\pgfqpoint{1.848464in}{0.810447in}}%
\pgfpathlineto{\pgfqpoint{1.850438in}{0.791501in}}%
\pgfpathlineto{\pgfqpoint{1.851227in}{0.797644in}}%
\pgfpathlineto{\pgfqpoint{1.851622in}{0.797999in}}%
\pgfpathlineto{\pgfqpoint{1.853991in}{0.772109in}}%
\pgfpathlineto{\pgfqpoint{1.854385in}{0.773262in}}%
\pgfpathlineto{\pgfqpoint{1.855570in}{0.761002in}}%
\pgfpathlineto{\pgfqpoint{1.856359in}{0.750877in}}%
\pgfpathlineto{\pgfqpoint{1.857149in}{0.759468in}}%
\pgfpathlineto{\pgfqpoint{1.859122in}{0.782170in}}%
\pgfpathlineto{\pgfqpoint{1.859517in}{0.779901in}}%
\pgfpathlineto{\pgfqpoint{1.859912in}{0.782693in}}%
\pgfpathlineto{\pgfqpoint{1.860307in}{0.778731in}}%
\pgfpathlineto{\pgfqpoint{1.862280in}{0.764485in}}%
\pgfpathlineto{\pgfqpoint{1.867807in}{0.836154in}}%
\pgfpathlineto{\pgfqpoint{1.870570in}{0.828362in}}%
\pgfpathlineto{\pgfqpoint{1.872544in}{0.835382in}}%
\pgfpathlineto{\pgfqpoint{1.872939in}{0.836458in}}%
\pgfpathlineto{\pgfqpoint{1.873333in}{0.833683in}}%
\pgfpathlineto{\pgfqpoint{1.874518in}{0.826846in}}%
\pgfpathlineto{\pgfqpoint{1.875307in}{0.829581in}}%
\pgfpathlineto{\pgfqpoint{1.876097in}{0.828318in}}%
\pgfpathlineto{\pgfqpoint{1.876886in}{0.824599in}}%
\pgfpathlineto{\pgfqpoint{1.877675in}{0.827717in}}%
\pgfpathlineto{\pgfqpoint{1.878465in}{0.832022in}}%
\pgfpathlineto{\pgfqpoint{1.878860in}{0.827832in}}%
\pgfpathlineto{\pgfqpoint{1.880439in}{0.809332in}}%
\pgfpathlineto{\pgfqpoint{1.881228in}{0.815878in}}%
\pgfpathlineto{\pgfqpoint{1.883202in}{0.830928in}}%
\pgfpathlineto{\pgfqpoint{1.883597in}{0.827603in}}%
\pgfpathlineto{\pgfqpoint{1.885176in}{0.798571in}}%
\pgfpathlineto{\pgfqpoint{1.886360in}{0.802422in}}%
\pgfpathlineto{\pgfqpoint{1.888728in}{0.807319in}}%
\pgfpathlineto{\pgfqpoint{1.889123in}{0.812296in}}%
\pgfpathlineto{\pgfqpoint{1.889913in}{0.806578in}}%
\pgfpathlineto{\pgfqpoint{1.893071in}{0.761792in}}%
\pgfpathlineto{\pgfqpoint{1.893465in}{0.760591in}}%
\pgfpathlineto{\pgfqpoint{1.899387in}{0.806115in}}%
\pgfpathlineto{\pgfqpoint{1.900571in}{0.810265in}}%
\pgfpathlineto{\pgfqpoint{1.900966in}{0.808139in}}%
\pgfpathlineto{\pgfqpoint{1.902939in}{0.790326in}}%
\pgfpathlineto{\pgfqpoint{1.903729in}{0.799131in}}%
\pgfpathlineto{\pgfqpoint{1.904518in}{0.807856in}}%
\pgfpathlineto{\pgfqpoint{1.905703in}{0.805867in}}%
\pgfpathlineto{\pgfqpoint{1.906887in}{0.809539in}}%
\pgfpathlineto{\pgfqpoint{1.910045in}{0.838771in}}%
\pgfpathlineto{\pgfqpoint{1.910440in}{0.836493in}}%
\pgfpathlineto{\pgfqpoint{1.914782in}{0.794188in}}%
\pgfpathlineto{\pgfqpoint{1.916756in}{0.784808in}}%
\pgfpathlineto{\pgfqpoint{1.917150in}{0.786030in}}%
\pgfpathlineto{\pgfqpoint{1.917545in}{0.789321in}}%
\pgfpathlineto{\pgfqpoint{1.918335in}{0.783046in}}%
\pgfpathlineto{\pgfqpoint{1.919124in}{0.778685in}}%
\pgfpathlineto{\pgfqpoint{1.919519in}{0.783652in}}%
\pgfpathlineto{\pgfqpoint{1.921887in}{0.804898in}}%
\pgfpathlineto{\pgfqpoint{1.922677in}{0.806754in}}%
\pgfpathlineto{\pgfqpoint{1.923072in}{0.806135in}}%
\pgfpathlineto{\pgfqpoint{1.926624in}{0.784080in}}%
\pgfpathlineto{\pgfqpoint{1.927019in}{0.788237in}}%
\pgfpathlineto{\pgfqpoint{1.932546in}{0.851412in}}%
\pgfpathlineto{\pgfqpoint{1.932940in}{0.850406in}}%
\pgfpathlineto{\pgfqpoint{1.933730in}{0.852459in}}%
\pgfpathlineto{\pgfqpoint{1.934125in}{0.853348in}}%
\pgfpathlineto{\pgfqpoint{1.934519in}{0.851551in}}%
\pgfpathlineto{\pgfqpoint{1.935704in}{0.844444in}}%
\pgfpathlineto{\pgfqpoint{1.936493in}{0.848079in}}%
\pgfpathlineto{\pgfqpoint{1.938862in}{0.850578in}}%
\pgfpathlineto{\pgfqpoint{1.939651in}{0.849969in}}%
\pgfpathlineto{\pgfqpoint{1.940046in}{0.852009in}}%
\pgfpathlineto{\pgfqpoint{1.940441in}{0.848749in}}%
\pgfpathlineto{\pgfqpoint{1.945572in}{0.798195in}}%
\pgfpathlineto{\pgfqpoint{1.946757in}{0.797389in}}%
\pgfpathlineto{\pgfqpoint{1.951099in}{0.778390in}}%
\pgfpathlineto{\pgfqpoint{1.951888in}{0.782110in}}%
\pgfpathlineto{\pgfqpoint{1.954257in}{0.795095in}}%
\pgfpathlineto{\pgfqpoint{1.955046in}{0.803657in}}%
\pgfpathlineto{\pgfqpoint{1.955836in}{0.796733in}}%
\pgfpathlineto{\pgfqpoint{1.958599in}{0.785613in}}%
\pgfpathlineto{\pgfqpoint{1.960178in}{0.775309in}}%
\pgfpathlineto{\pgfqpoint{1.960573in}{0.776759in}}%
\pgfpathlineto{\pgfqpoint{1.962546in}{0.795550in}}%
\pgfpathlineto{\pgfqpoint{1.962941in}{0.792279in}}%
\pgfpathlineto{\pgfqpoint{1.963336in}{0.787144in}}%
\pgfpathlineto{\pgfqpoint{1.964125in}{0.793588in}}%
\pgfpathlineto{\pgfqpoint{1.964520in}{0.792766in}}%
\pgfpathlineto{\pgfqpoint{1.964915in}{0.794540in}}%
\pgfpathlineto{\pgfqpoint{1.969257in}{0.819477in}}%
\pgfpathlineto{\pgfqpoint{1.970441in}{0.817296in}}%
\pgfpathlineto{\pgfqpoint{1.971231in}{0.817154in}}%
\pgfpathlineto{\pgfqpoint{1.971626in}{0.817984in}}%
\pgfpathlineto{\pgfqpoint{1.972810in}{0.820214in}}%
\pgfpathlineto{\pgfqpoint{1.973994in}{0.808554in}}%
\pgfpathlineto{\pgfqpoint{1.974784in}{0.810845in}}%
\pgfpathlineto{\pgfqpoint{1.979126in}{0.824126in}}%
\pgfpathlineto{\pgfqpoint{1.980705in}{0.811160in}}%
\pgfpathlineto{\pgfqpoint{1.981494in}{0.816413in}}%
\pgfpathlineto{\pgfqpoint{1.982284in}{0.821841in}}%
\pgfpathlineto{\pgfqpoint{1.982679in}{0.818778in}}%
\pgfpathlineto{\pgfqpoint{1.984652in}{0.807713in}}%
\pgfpathlineto{\pgfqpoint{1.985047in}{0.809629in}}%
\pgfpathlineto{\pgfqpoint{1.987021in}{0.834116in}}%
\pgfpathlineto{\pgfqpoint{1.988205in}{0.828632in}}%
\pgfpathlineto{\pgfqpoint{1.988995in}{0.823612in}}%
\pgfpathlineto{\pgfqpoint{1.992153in}{0.858547in}}%
\pgfpathlineto{\pgfqpoint{1.992942in}{0.858836in}}%
\pgfpathlineto{\pgfqpoint{1.993337in}{0.858050in}}%
\pgfpathlineto{\pgfqpoint{1.994521in}{0.853077in}}%
\pgfpathlineto{\pgfqpoint{2.000837in}{0.808888in}}%
\pgfpathlineto{\pgfqpoint{2.002811in}{0.793439in}}%
\pgfpathlineto{\pgfqpoint{2.003206in}{0.793473in}}%
\pgfpathlineto{\pgfqpoint{2.004390in}{0.792089in}}%
\pgfpathlineto{\pgfqpoint{2.005179in}{0.790210in}}%
\pgfpathlineto{\pgfqpoint{2.005574in}{0.795320in}}%
\pgfpathlineto{\pgfqpoint{2.006364in}{0.791102in}}%
\pgfpathlineto{\pgfqpoint{2.008337in}{0.785646in}}%
\pgfpathlineto{\pgfqpoint{2.008732in}{0.787500in}}%
\pgfpathlineto{\pgfqpoint{2.011101in}{0.794978in}}%
\pgfpathlineto{\pgfqpoint{2.011890in}{0.797406in}}%
\pgfpathlineto{\pgfqpoint{2.012285in}{0.796301in}}%
\pgfpathlineto{\pgfqpoint{2.017811in}{0.748806in}}%
\pgfpathlineto{\pgfqpoint{2.018996in}{0.747998in}}%
\pgfpathlineto{\pgfqpoint{2.020180in}{0.740003in}}%
\pgfpathlineto{\pgfqpoint{2.021364in}{0.743471in}}%
\pgfpathlineto{\pgfqpoint{2.022943in}{0.746161in}}%
\pgfpathlineto{\pgfqpoint{2.024127in}{0.757008in}}%
\pgfpathlineto{\pgfqpoint{2.024522in}{0.751937in}}%
\pgfpathlineto{\pgfqpoint{2.024917in}{0.747301in}}%
\pgfpathlineto{\pgfqpoint{2.025706in}{0.755101in}}%
\pgfpathlineto{\pgfqpoint{2.030443in}{0.825754in}}%
\pgfpathlineto{\pgfqpoint{2.031233in}{0.823367in}}%
\pgfpathlineto{\pgfqpoint{2.031628in}{0.822751in}}%
\pgfpathlineto{\pgfqpoint{2.035970in}{0.781930in}}%
\pgfpathlineto{\pgfqpoint{2.036759in}{0.785838in}}%
\pgfpathlineto{\pgfqpoint{2.039523in}{0.793060in}}%
\pgfpathlineto{\pgfqpoint{2.043075in}{0.834003in}}%
\pgfpathlineto{\pgfqpoint{2.047023in}{0.807593in}}%
\pgfpathlineto{\pgfqpoint{2.047812in}{0.810393in}}%
\pgfpathlineto{\pgfqpoint{2.049391in}{0.818437in}}%
\pgfpathlineto{\pgfqpoint{2.049786in}{0.814176in}}%
\pgfpathlineto{\pgfqpoint{2.052944in}{0.787725in}}%
\pgfpathlineto{\pgfqpoint{2.053339in}{0.786896in}}%
\pgfpathlineto{\pgfqpoint{2.053733in}{0.788316in}}%
\pgfpathlineto{\pgfqpoint{2.054918in}{0.791834in}}%
\pgfpathlineto{\pgfqpoint{2.055312in}{0.790928in}}%
\pgfpathlineto{\pgfqpoint{2.056102in}{0.779909in}}%
\pgfpathlineto{\pgfqpoint{2.057286in}{0.781131in}}%
\pgfpathlineto{\pgfqpoint{2.058076in}{0.779366in}}%
\pgfpathlineto{\pgfqpoint{2.060839in}{0.768719in}}%
\pgfpathlineto{\pgfqpoint{2.062418in}{0.763642in}}%
\pgfpathlineto{\pgfqpoint{2.062813in}{0.761321in}}%
\pgfpathlineto{\pgfqpoint{2.063997in}{0.763391in}}%
\pgfpathlineto{\pgfqpoint{2.065181in}{0.772437in}}%
\pgfpathlineto{\pgfqpoint{2.065576in}{0.771411in}}%
\pgfpathlineto{\pgfqpoint{2.068339in}{0.749757in}}%
\pgfpathlineto{\pgfqpoint{2.073866in}{0.777498in}}%
\pgfpathlineto{\pgfqpoint{2.076234in}{0.785296in}}%
\pgfpathlineto{\pgfqpoint{2.076629in}{0.784484in}}%
\pgfpathlineto{\pgfqpoint{2.077024in}{0.784670in}}%
\pgfpathlineto{\pgfqpoint{2.080576in}{0.766510in}}%
\pgfpathlineto{\pgfqpoint{2.080971in}{0.769044in}}%
\pgfpathlineto{\pgfqpoint{2.083340in}{0.778388in}}%
\pgfpathlineto{\pgfqpoint{2.083734in}{0.777329in}}%
\pgfpathlineto{\pgfqpoint{2.089656in}{0.739375in}}%
\pgfpathlineto{\pgfqpoint{2.090445in}{0.742956in}}%
\pgfpathlineto{\pgfqpoint{2.093603in}{0.762156in}}%
\pgfpathlineto{\pgfqpoint{2.094393in}{0.769089in}}%
\pgfpathlineto{\pgfqpoint{2.096761in}{0.784913in}}%
\pgfpathlineto{\pgfqpoint{2.101103in}{0.800799in}}%
\pgfpathlineto{\pgfqpoint{2.101893in}{0.802156in}}%
\pgfpathlineto{\pgfqpoint{2.102288in}{0.800804in}}%
\pgfpathlineto{\pgfqpoint{2.103867in}{0.789547in}}%
\pgfpathlineto{\pgfqpoint{2.104261in}{0.792992in}}%
\pgfpathlineto{\pgfqpoint{2.105051in}{0.798960in}}%
\pgfpathlineto{\pgfqpoint{2.105840in}{0.796067in}}%
\pgfpathlineto{\pgfqpoint{2.110577in}{0.763933in}}%
\pgfpathlineto{\pgfqpoint{2.112551in}{0.745483in}}%
\pgfpathlineto{\pgfqpoint{2.112946in}{0.745512in}}%
\pgfpathlineto{\pgfqpoint{2.113735in}{0.750151in}}%
\pgfpathlineto{\pgfqpoint{2.114920in}{0.775186in}}%
\pgfpathlineto{\pgfqpoint{2.116104in}{0.771491in}}%
\pgfpathlineto{\pgfqpoint{2.117288in}{0.770912in}}%
\pgfpathlineto{\pgfqpoint{2.118472in}{0.778308in}}%
\pgfpathlineto{\pgfqpoint{2.122420in}{0.824755in}}%
\pgfpathlineto{\pgfqpoint{2.123209in}{0.821581in}}%
\pgfpathlineto{\pgfqpoint{2.128736in}{0.791594in}}%
\pgfpathlineto{\pgfqpoint{2.129130in}{0.792869in}}%
\pgfpathlineto{\pgfqpoint{2.131894in}{0.799749in}}%
\pgfpathlineto{\pgfqpoint{2.135052in}{0.805382in}}%
\pgfpathlineto{\pgfqpoint{2.137420in}{0.814331in}}%
\pgfpathlineto{\pgfqpoint{2.137815in}{0.813527in}}%
\pgfpathlineto{\pgfqpoint{2.140183in}{0.789413in}}%
\pgfpathlineto{\pgfqpoint{2.141762in}{0.762784in}}%
\pgfpathlineto{\pgfqpoint{2.142947in}{0.766110in}}%
\pgfpathlineto{\pgfqpoint{2.143736in}{0.761480in}}%
\pgfpathlineto{\pgfqpoint{2.147289in}{0.741417in}}%
\pgfpathlineto{\pgfqpoint{2.148078in}{0.743040in}}%
\pgfpathlineto{\pgfqpoint{2.149263in}{0.748299in}}%
\pgfpathlineto{\pgfqpoint{2.150447in}{0.745803in}}%
\pgfpathlineto{\pgfqpoint{2.150842in}{0.744178in}}%
\pgfpathlineto{\pgfqpoint{2.151236in}{0.747668in}}%
\pgfpathlineto{\pgfqpoint{2.151631in}{0.747569in}}%
\pgfpathlineto{\pgfqpoint{2.152421in}{0.740852in}}%
\pgfpathlineto{\pgfqpoint{2.153210in}{0.743080in}}%
\pgfpathlineto{\pgfqpoint{2.153605in}{0.743590in}}%
\pgfpathlineto{\pgfqpoint{2.154000in}{0.742615in}}%
\pgfpathlineto{\pgfqpoint{2.158342in}{0.732584in}}%
\pgfpathlineto{\pgfqpoint{2.158737in}{0.733268in}}%
\pgfpathlineto{\pgfqpoint{2.159921in}{0.732194in}}%
\pgfpathlineto{\pgfqpoint{2.164658in}{0.772075in}}%
\pgfpathlineto{\pgfqpoint{2.166632in}{0.767752in}}%
\pgfpathlineto{\pgfqpoint{2.169000in}{0.758712in}}%
\pgfpathlineto{\pgfqpoint{2.169395in}{0.761137in}}%
\pgfpathlineto{\pgfqpoint{2.170184in}{0.757072in}}%
\pgfpathlineto{\pgfqpoint{2.174527in}{0.777408in}}%
\pgfpathlineto{\pgfqpoint{2.177290in}{0.813043in}}%
\pgfpathlineto{\pgfqpoint{2.177685in}{0.811071in}}%
\pgfpathlineto{\pgfqpoint{2.178869in}{0.808087in}}%
\pgfpathlineto{\pgfqpoint{2.180053in}{0.815361in}}%
\pgfpathlineto{\pgfqpoint{2.180843in}{0.812040in}}%
\pgfpathlineto{\pgfqpoint{2.182027in}{0.803899in}}%
\pgfpathlineto{\pgfqpoint{2.185580in}{0.775406in}}%
\pgfpathlineto{\pgfqpoint{2.187948in}{0.768502in}}%
\pgfpathlineto{\pgfqpoint{2.188738in}{0.770152in}}%
\pgfpathlineto{\pgfqpoint{2.190317in}{0.777789in}}%
\pgfpathlineto{\pgfqpoint{2.191896in}{0.775558in}}%
\pgfpathlineto{\pgfqpoint{2.193475in}{0.768983in}}%
\pgfpathlineto{\pgfqpoint{2.194264in}{0.771026in}}%
\pgfpathlineto{\pgfqpoint{2.195843in}{0.777659in}}%
\pgfpathlineto{\pgfqpoint{2.197027in}{0.785668in}}%
\pgfpathlineto{\pgfqpoint{2.197817in}{0.783275in}}%
\pgfpathlineto{\pgfqpoint{2.203738in}{0.755986in}}%
\pgfpathlineto{\pgfqpoint{2.204133in}{0.756338in}}%
\pgfpathlineto{\pgfqpoint{2.204922in}{0.760091in}}%
\pgfpathlineto{\pgfqpoint{2.205712in}{0.758554in}}%
\pgfpathlineto{\pgfqpoint{2.207686in}{0.748677in}}%
\pgfpathlineto{\pgfqpoint{2.208475in}{0.752910in}}%
\pgfpathlineto{\pgfqpoint{2.210844in}{0.764862in}}%
\pgfpathlineto{\pgfqpoint{2.211238in}{0.760172in}}%
\pgfpathlineto{\pgfqpoint{2.211633in}{0.758236in}}%
\pgfpathlineto{\pgfqpoint{2.212028in}{0.760310in}}%
\pgfpathlineto{\pgfqpoint{2.217949in}{0.793937in}}%
\pgfpathlineto{\pgfqpoint{2.219528in}{0.773638in}}%
\pgfpathlineto{\pgfqpoint{2.219923in}{0.775610in}}%
\pgfpathlineto{\pgfqpoint{2.220712in}{0.785370in}}%
\pgfpathlineto{\pgfqpoint{2.221896in}{0.784097in}}%
\pgfpathlineto{\pgfqpoint{2.222686in}{0.782026in}}%
\pgfpathlineto{\pgfqpoint{2.223081in}{0.785403in}}%
\pgfpathlineto{\pgfqpoint{2.224660in}{0.794926in}}%
\pgfpathlineto{\pgfqpoint{2.225054in}{0.790157in}}%
\pgfpathlineto{\pgfqpoint{2.226239in}{0.783765in}}%
\pgfpathlineto{\pgfqpoint{2.226633in}{0.787183in}}%
\pgfpathlineto{\pgfqpoint{2.227423in}{0.787118in}}%
\pgfpathlineto{\pgfqpoint{2.227818in}{0.788410in}}%
\pgfpathlineto{\pgfqpoint{2.228212in}{0.786746in}}%
\pgfpathlineto{\pgfqpoint{2.229397in}{0.782650in}}%
\pgfpathlineto{\pgfqpoint{2.229791in}{0.783924in}}%
\pgfpathlineto{\pgfqpoint{2.230976in}{0.795059in}}%
\pgfpathlineto{\pgfqpoint{2.232160in}{0.790953in}}%
\pgfpathlineto{\pgfqpoint{2.234923in}{0.808260in}}%
\pgfpathlineto{\pgfqpoint{2.236107in}{0.804791in}}%
\pgfpathlineto{\pgfqpoint{2.247160in}{0.743219in}}%
\pgfpathlineto{\pgfqpoint{2.247950in}{0.744761in}}%
\pgfpathlineto{\pgfqpoint{2.248739in}{0.750238in}}%
\pgfpathlineto{\pgfqpoint{2.249529in}{0.749723in}}%
\pgfpathlineto{\pgfqpoint{2.251503in}{0.744714in}}%
\pgfpathlineto{\pgfqpoint{2.251897in}{0.748171in}}%
\pgfpathlineto{\pgfqpoint{2.254266in}{0.763901in}}%
\pgfpathlineto{\pgfqpoint{2.255055in}{0.761801in}}%
\pgfpathlineto{\pgfqpoint{2.258213in}{0.747392in}}%
\pgfpathlineto{\pgfqpoint{2.258608in}{0.749693in}}%
\pgfpathlineto{\pgfqpoint{2.262556in}{0.772632in}}%
\pgfpathlineto{\pgfqpoint{2.263345in}{0.769957in}}%
\pgfpathlineto{\pgfqpoint{2.264135in}{0.767090in}}%
\pgfpathlineto{\pgfqpoint{2.264924in}{0.768970in}}%
\pgfpathlineto{\pgfqpoint{2.265714in}{0.769607in}}%
\pgfpathlineto{\pgfqpoint{2.267293in}{0.773581in}}%
\pgfpathlineto{\pgfqpoint{2.267687in}{0.772442in}}%
\pgfpathlineto{\pgfqpoint{2.271240in}{0.750549in}}%
\pgfpathlineto{\pgfqpoint{2.272030in}{0.746490in}}%
\pgfpathlineto{\pgfqpoint{2.272819in}{0.748308in}}%
\pgfpathlineto{\pgfqpoint{2.274793in}{0.757411in}}%
\pgfpathlineto{\pgfqpoint{2.275582in}{0.752774in}}%
\pgfpathlineto{\pgfqpoint{2.282293in}{0.719326in}}%
\pgfpathlineto{\pgfqpoint{2.283083in}{0.720861in}}%
\pgfpathlineto{\pgfqpoint{2.284267in}{0.726379in}}%
\pgfpathlineto{\pgfqpoint{2.287425in}{0.759277in}}%
\pgfpathlineto{\pgfqpoint{2.288609in}{0.770273in}}%
\pgfpathlineto{\pgfqpoint{2.289399in}{0.769689in}}%
\pgfpathlineto{\pgfqpoint{2.289793in}{0.769751in}}%
\pgfpathlineto{\pgfqpoint{2.290583in}{0.775634in}}%
\pgfpathlineto{\pgfqpoint{2.291372in}{0.773145in}}%
\pgfpathlineto{\pgfqpoint{2.294136in}{0.767557in}}%
\pgfpathlineto{\pgfqpoint{2.295714in}{0.768819in}}%
\pgfpathlineto{\pgfqpoint{2.297293in}{0.765037in}}%
\pgfpathlineto{\pgfqpoint{2.300057in}{0.743591in}}%
\pgfpathlineto{\pgfqpoint{2.303609in}{0.716519in}}%
\pgfpathlineto{\pgfqpoint{2.304004in}{0.720290in}}%
\pgfpathlineto{\pgfqpoint{2.309925in}{0.776828in}}%
\pgfpathlineto{\pgfqpoint{2.310320in}{0.774597in}}%
\pgfpathlineto{\pgfqpoint{2.311504in}{0.765873in}}%
\pgfpathlineto{\pgfqpoint{2.312689in}{0.768872in}}%
\pgfpathlineto{\pgfqpoint{2.314268in}{0.772032in}}%
\pgfpathlineto{\pgfqpoint{2.317031in}{0.759457in}}%
\pgfpathlineto{\pgfqpoint{2.317426in}{0.758949in}}%
\pgfpathlineto{\pgfqpoint{2.319794in}{0.738180in}}%
\pgfpathlineto{\pgfqpoint{2.322952in}{0.730240in}}%
\pgfpathlineto{\pgfqpoint{2.323742in}{0.732327in}}%
\pgfpathlineto{\pgfqpoint{2.324531in}{0.747010in}}%
\pgfpathlineto{\pgfqpoint{2.325321in}{0.742689in}}%
\pgfpathlineto{\pgfqpoint{2.327689in}{0.729578in}}%
\pgfpathlineto{\pgfqpoint{2.328084in}{0.729876in}}%
\pgfpathlineto{\pgfqpoint{2.332031in}{0.742024in}}%
\pgfpathlineto{\pgfqpoint{2.332426in}{0.738914in}}%
\pgfpathlineto{\pgfqpoint{2.333610in}{0.734231in}}%
\pgfpathlineto{\pgfqpoint{2.334005in}{0.734772in}}%
\pgfpathlineto{\pgfqpoint{2.334795in}{0.736529in}}%
\pgfpathlineto{\pgfqpoint{2.335189in}{0.735383in}}%
\pgfpathlineto{\pgfqpoint{2.336374in}{0.731415in}}%
\pgfpathlineto{\pgfqpoint{2.336768in}{0.733826in}}%
\pgfpathlineto{\pgfqpoint{2.339926in}{0.737266in}}%
\pgfpathlineto{\pgfqpoint{2.341505in}{0.737270in}}%
\pgfpathlineto{\pgfqpoint{2.343479in}{0.741329in}}%
\pgfpathlineto{\pgfqpoint{2.343874in}{0.741545in}}%
\pgfpathlineto{\pgfqpoint{2.345848in}{0.760414in}}%
\pgfpathlineto{\pgfqpoint{2.349006in}{0.775357in}}%
\pgfpathlineto{\pgfqpoint{2.351769in}{0.785836in}}%
\pgfpathlineto{\pgfqpoint{2.352164in}{0.784534in}}%
\pgfpathlineto{\pgfqpoint{2.352953in}{0.782407in}}%
\pgfpathlineto{\pgfqpoint{2.353348in}{0.784276in}}%
\pgfpathlineto{\pgfqpoint{2.355716in}{0.794357in}}%
\pgfpathlineto{\pgfqpoint{2.356111in}{0.791289in}}%
\pgfpathlineto{\pgfqpoint{2.357690in}{0.771327in}}%
\pgfpathlineto{\pgfqpoint{2.359664in}{0.775634in}}%
\pgfpathlineto{\pgfqpoint{2.360453in}{0.776535in}}%
\pgfpathlineto{\pgfqpoint{2.361243in}{0.775622in}}%
\pgfpathlineto{\pgfqpoint{2.363611in}{0.769849in}}%
\pgfpathlineto{\pgfqpoint{2.364006in}{0.770023in}}%
\pgfpathlineto{\pgfqpoint{2.364401in}{0.773641in}}%
\pgfpathlineto{\pgfqpoint{2.365190in}{0.769854in}}%
\pgfpathlineto{\pgfqpoint{2.365585in}{0.770856in}}%
\pgfpathlineto{\pgfqpoint{2.365980in}{0.770771in}}%
\pgfpathlineto{\pgfqpoint{2.366375in}{0.766934in}}%
\pgfpathlineto{\pgfqpoint{2.367164in}{0.768970in}}%
\pgfpathlineto{\pgfqpoint{2.368348in}{0.780219in}}%
\pgfpathlineto{\pgfqpoint{2.369927in}{0.779889in}}%
\pgfpathlineto{\pgfqpoint{2.370717in}{0.782366in}}%
\pgfpathlineto{\pgfqpoint{2.371112in}{0.778310in}}%
\pgfpathlineto{\pgfqpoint{2.371901in}{0.781737in}}%
\pgfpathlineto{\pgfqpoint{2.375454in}{0.774150in}}%
\pgfpathlineto{\pgfqpoint{2.375849in}{0.779565in}}%
\pgfpathlineto{\pgfqpoint{2.377033in}{0.775652in}}%
\pgfpathlineto{\pgfqpoint{2.382164in}{0.756964in}}%
\pgfpathlineto{\pgfqpoint{2.384533in}{0.752839in}}%
\pgfpathlineto{\pgfqpoint{2.384928in}{0.754416in}}%
\pgfpathlineto{\pgfqpoint{2.386901in}{0.764946in}}%
\pgfpathlineto{\pgfqpoint{2.390849in}{0.720926in}}%
\pgfpathlineto{\pgfqpoint{2.391638in}{0.724342in}}%
\pgfpathlineto{\pgfqpoint{2.392033in}{0.723200in}}%
\pgfpathlineto{\pgfqpoint{2.392428in}{0.725868in}}%
\pgfpathlineto{\pgfqpoint{2.400323in}{0.775806in}}%
\pgfpathlineto{\pgfqpoint{2.400718in}{0.774964in}}%
\pgfpathlineto{\pgfqpoint{2.401112in}{0.773201in}}%
\pgfpathlineto{\pgfqpoint{2.401507in}{0.774512in}}%
\pgfpathlineto{\pgfqpoint{2.403481in}{0.781337in}}%
\pgfpathlineto{\pgfqpoint{2.405455in}{0.777471in}}%
\pgfpathlineto{\pgfqpoint{2.406639in}{0.775905in}}%
\pgfpathlineto{\pgfqpoint{2.407034in}{0.776273in}}%
\pgfpathlineto{\pgfqpoint{2.407823in}{0.778142in}}%
\pgfpathlineto{\pgfqpoint{2.410981in}{0.786656in}}%
\pgfpathlineto{\pgfqpoint{2.412560in}{0.780628in}}%
\pgfpathlineto{\pgfqpoint{2.412955in}{0.781666in}}%
\pgfpathlineto{\pgfqpoint{2.415323in}{0.788091in}}%
\pgfpathlineto{\pgfqpoint{2.420060in}{0.734660in}}%
\pgfpathlineto{\pgfqpoint{2.420455in}{0.735671in}}%
\pgfpathlineto{\pgfqpoint{2.422429in}{0.748488in}}%
\pgfpathlineto{\pgfqpoint{2.422824in}{0.747975in}}%
\pgfpathlineto{\pgfqpoint{2.424797in}{0.727058in}}%
\pgfpathlineto{\pgfqpoint{2.425192in}{0.730529in}}%
\pgfpathlineto{\pgfqpoint{2.425982in}{0.724966in}}%
\pgfpathlineto{\pgfqpoint{2.426376in}{0.725000in}}%
\pgfpathlineto{\pgfqpoint{2.427955in}{0.737824in}}%
\pgfpathlineto{\pgfqpoint{2.429140in}{0.732367in}}%
\pgfpathlineto{\pgfqpoint{2.429534in}{0.731341in}}%
\pgfpathlineto{\pgfqpoint{2.430324in}{0.732570in}}%
\pgfpathlineto{\pgfqpoint{2.431903in}{0.732960in}}%
\pgfpathlineto{\pgfqpoint{2.432692in}{0.729287in}}%
\pgfpathlineto{\pgfqpoint{2.433087in}{0.731235in}}%
\pgfpathlineto{\pgfqpoint{2.433877in}{0.736858in}}%
\pgfpathlineto{\pgfqpoint{2.434666in}{0.734374in}}%
\pgfpathlineto{\pgfqpoint{2.437429in}{0.724259in}}%
\pgfpathlineto{\pgfqpoint{2.440193in}{0.703484in}}%
\pgfpathlineto{\pgfqpoint{2.444930in}{0.678114in}}%
\pgfpathlineto{\pgfqpoint{2.445719in}{0.678860in}}%
\pgfpathlineto{\pgfqpoint{2.449667in}{0.692747in}}%
\pgfpathlineto{\pgfqpoint{2.450456in}{0.699372in}}%
\pgfpathlineto{\pgfqpoint{2.451640in}{0.697515in}}%
\pgfpathlineto{\pgfqpoint{2.452035in}{0.697170in}}%
\pgfpathlineto{\pgfqpoint{2.453219in}{0.689642in}}%
\pgfpathlineto{\pgfqpoint{2.453614in}{0.692385in}}%
\pgfpathlineto{\pgfqpoint{2.459535in}{0.744628in}}%
\pgfpathlineto{\pgfqpoint{2.460720in}{0.740605in}}%
\pgfpathlineto{\pgfqpoint{2.461509in}{0.742887in}}%
\pgfpathlineto{\pgfqpoint{2.463483in}{0.747553in}}%
\pgfpathlineto{\pgfqpoint{2.464272in}{0.746087in}}%
\pgfpathlineto{\pgfqpoint{2.465456in}{0.745355in}}%
\pgfpathlineto{\pgfqpoint{2.469404in}{0.727495in}}%
\pgfpathlineto{\pgfqpoint{2.469799in}{0.728819in}}%
\pgfpathlineto{\pgfqpoint{2.470983in}{0.730817in}}%
\pgfpathlineto{\pgfqpoint{2.471378in}{0.730623in}}%
\pgfpathlineto{\pgfqpoint{2.473746in}{0.717131in}}%
\pgfpathlineto{\pgfqpoint{2.474536in}{0.722091in}}%
\pgfpathlineto{\pgfqpoint{2.474930in}{0.726313in}}%
\pgfpathlineto{\pgfqpoint{2.475325in}{0.718898in}}%
\pgfpathlineto{\pgfqpoint{2.477694in}{0.692765in}}%
\pgfpathlineto{\pgfqpoint{2.478483in}{0.694506in}}%
\pgfpathlineto{\pgfqpoint{2.479667in}{0.700620in}}%
\pgfpathlineto{\pgfqpoint{2.482036in}{0.678902in}}%
\pgfpathlineto{\pgfqpoint{2.482825in}{0.680890in}}%
\pgfpathlineto{\pgfqpoint{2.483615in}{0.678600in}}%
\pgfpathlineto{\pgfqpoint{2.485589in}{0.669907in}}%
\pgfpathlineto{\pgfqpoint{2.485983in}{0.670041in}}%
\pgfpathlineto{\pgfqpoint{2.490720in}{0.710249in}}%
\pgfpathlineto{\pgfqpoint{2.491510in}{0.702075in}}%
\pgfpathlineto{\pgfqpoint{2.492694in}{0.691457in}}%
\pgfpathlineto{\pgfqpoint{2.493484in}{0.692507in}}%
\pgfpathlineto{\pgfqpoint{2.495457in}{0.699609in}}%
\pgfpathlineto{\pgfqpoint{2.499800in}{0.733140in}}%
\pgfpathlineto{\pgfqpoint{2.501379in}{0.736769in}}%
\pgfpathlineto{\pgfqpoint{2.501773in}{0.735815in}}%
\pgfpathlineto{\pgfqpoint{2.507695in}{0.708112in}}%
\pgfpathlineto{\pgfqpoint{2.509668in}{0.715769in}}%
\pgfpathlineto{\pgfqpoint{2.510063in}{0.714908in}}%
\pgfpathlineto{\pgfqpoint{2.510853in}{0.712590in}}%
\pgfpathlineto{\pgfqpoint{2.511642in}{0.714073in}}%
\pgfpathlineto{\pgfqpoint{2.513616in}{0.719145in}}%
\pgfpathlineto{\pgfqpoint{2.514011in}{0.718686in}}%
\pgfpathlineto{\pgfqpoint{2.514800in}{0.720433in}}%
\pgfpathlineto{\pgfqpoint{2.515590in}{0.722151in}}%
\pgfpathlineto{\pgfqpoint{2.516774in}{0.726668in}}%
\pgfpathlineto{\pgfqpoint{2.517169in}{0.724216in}}%
\pgfpathlineto{\pgfqpoint{2.518353in}{0.719369in}}%
\pgfpathlineto{\pgfqpoint{2.519142in}{0.719900in}}%
\pgfpathlineto{\pgfqpoint{2.519932in}{0.722084in}}%
\pgfpathlineto{\pgfqpoint{2.520327in}{0.721148in}}%
\pgfpathlineto{\pgfqpoint{2.521116in}{0.718210in}}%
\pgfpathlineto{\pgfqpoint{2.521511in}{0.718683in}}%
\pgfpathlineto{\pgfqpoint{2.525458in}{0.737165in}}%
\pgfpathlineto{\pgfqpoint{2.526248in}{0.735991in}}%
\pgfpathlineto{\pgfqpoint{2.529011in}{0.758760in}}%
\pgfpathlineto{\pgfqpoint{2.529801in}{0.756833in}}%
\pgfpathlineto{\pgfqpoint{2.530590in}{0.759320in}}%
\pgfpathlineto{\pgfqpoint{2.531380in}{0.761086in}}%
\pgfpathlineto{\pgfqpoint{2.532959in}{0.750343in}}%
\pgfpathlineto{\pgfqpoint{2.533748in}{0.751530in}}%
\pgfpathlineto{\pgfqpoint{2.535722in}{0.752013in}}%
\pgfpathlineto{\pgfqpoint{2.536511in}{0.752455in}}%
\pgfpathlineto{\pgfqpoint{2.537301in}{0.759932in}}%
\pgfpathlineto{\pgfqpoint{2.538090in}{0.757241in}}%
\pgfpathlineto{\pgfqpoint{2.542038in}{0.737680in}}%
\pgfpathlineto{\pgfqpoint{2.544406in}{0.732842in}}%
\pgfpathlineto{\pgfqpoint{2.545196in}{0.728844in}}%
\pgfpathlineto{\pgfqpoint{2.547564in}{0.717210in}}%
\pgfpathlineto{\pgfqpoint{2.549933in}{0.714789in}}%
\pgfpathlineto{\pgfqpoint{2.551512in}{0.716632in}}%
\pgfpathlineto{\pgfqpoint{2.551906in}{0.715409in}}%
\pgfpathlineto{\pgfqpoint{2.552696in}{0.716982in}}%
\pgfpathlineto{\pgfqpoint{2.553880in}{0.721526in}}%
\pgfpathlineto{\pgfqpoint{2.554670in}{0.720533in}}%
\pgfpathlineto{\pgfqpoint{2.555064in}{0.719499in}}%
\pgfpathlineto{\pgfqpoint{2.555854in}{0.721369in}}%
\pgfpathlineto{\pgfqpoint{2.558222in}{0.731421in}}%
\pgfpathlineto{\pgfqpoint{2.559407in}{0.737382in}}%
\pgfpathlineto{\pgfqpoint{2.560196in}{0.737241in}}%
\pgfpathlineto{\pgfqpoint{2.561380in}{0.739967in}}%
\pgfpathlineto{\pgfqpoint{2.561775in}{0.740784in}}%
\pgfpathlineto{\pgfqpoint{2.562959in}{0.739755in}}%
\pgfpathlineto{\pgfqpoint{2.564144in}{0.739415in}}%
\pgfpathlineto{\pgfqpoint{2.566117in}{0.729505in}}%
\pgfpathlineto{\pgfqpoint{2.568091in}{0.720150in}}%
\pgfpathlineto{\pgfqpoint{2.568486in}{0.720213in}}%
\pgfpathlineto{\pgfqpoint{2.569275in}{0.725486in}}%
\pgfpathlineto{\pgfqpoint{2.570460in}{0.724749in}}%
\pgfpathlineto{\pgfqpoint{2.571644in}{0.725491in}}%
\pgfpathlineto{\pgfqpoint{2.572039in}{0.725085in}}%
\pgfpathlineto{\pgfqpoint{2.573618in}{0.719568in}}%
\pgfpathlineto{\pgfqpoint{2.574407in}{0.720260in}}%
\pgfpathlineto{\pgfqpoint{2.577960in}{0.725067in}}%
\pgfpathlineto{\pgfqpoint{2.578749in}{0.724537in}}%
\pgfpathlineto{\pgfqpoint{2.579539in}{0.716628in}}%
\pgfpathlineto{\pgfqpoint{2.580328in}{0.718603in}}%
\pgfpathlineto{\pgfqpoint{2.582302in}{0.712764in}}%
\pgfpathlineto{\pgfqpoint{2.584671in}{0.727098in}}%
\pgfpathlineto{\pgfqpoint{2.587039in}{0.733946in}}%
\pgfpathlineto{\pgfqpoint{2.589013in}{0.723543in}}%
\pgfpathlineto{\pgfqpoint{2.590197in}{0.708247in}}%
\pgfpathlineto{\pgfqpoint{2.590987in}{0.715289in}}%
\pgfpathlineto{\pgfqpoint{2.595329in}{0.744576in}}%
\pgfpathlineto{\pgfqpoint{2.597697in}{0.730526in}}%
\pgfpathlineto{\pgfqpoint{2.598882in}{0.729810in}}%
\pgfpathlineto{\pgfqpoint{2.602829in}{0.750957in}}%
\pgfpathlineto{\pgfqpoint{2.603224in}{0.750853in}}%
\pgfpathlineto{\pgfqpoint{2.604408in}{0.748599in}}%
\pgfpathlineto{\pgfqpoint{2.604803in}{0.752581in}}%
\pgfpathlineto{\pgfqpoint{2.605987in}{0.748895in}}%
\pgfpathlineto{\pgfqpoint{2.608750in}{0.726876in}}%
\pgfpathlineto{\pgfqpoint{2.609145in}{0.730146in}}%
\pgfpathlineto{\pgfqpoint{2.609540in}{0.734599in}}%
\pgfpathlineto{\pgfqpoint{2.610329in}{0.727232in}}%
\pgfpathlineto{\pgfqpoint{2.613093in}{0.732757in}}%
\pgfpathlineto{\pgfqpoint{2.613487in}{0.732499in}}%
\pgfpathlineto{\pgfqpoint{2.615461in}{0.726023in}}%
\pgfpathlineto{\pgfqpoint{2.615856in}{0.727527in}}%
\pgfpathlineto{\pgfqpoint{2.617040in}{0.732679in}}%
\pgfpathlineto{\pgfqpoint{2.617830in}{0.731898in}}%
\pgfpathlineto{\pgfqpoint{2.618224in}{0.732099in}}%
\pgfpathlineto{\pgfqpoint{2.618619in}{0.731180in}}%
\pgfpathlineto{\pgfqpoint{2.620593in}{0.725207in}}%
\pgfpathlineto{\pgfqpoint{2.624540in}{0.739413in}}%
\pgfpathlineto{\pgfqpoint{2.625330in}{0.739144in}}%
\pgfpathlineto{\pgfqpoint{2.626119in}{0.739858in}}%
\pgfpathlineto{\pgfqpoint{2.626514in}{0.739312in}}%
\pgfpathlineto{\pgfqpoint{2.628093in}{0.733519in}}%
\pgfpathlineto{\pgfqpoint{2.628882in}{0.736835in}}%
\pgfpathlineto{\pgfqpoint{2.629277in}{0.738839in}}%
\pgfpathlineto{\pgfqpoint{2.629672in}{0.736578in}}%
\pgfpathlineto{\pgfqpoint{2.630067in}{0.726972in}}%
\pgfpathlineto{\pgfqpoint{2.631251in}{0.734009in}}%
\pgfpathlineto{\pgfqpoint{2.635593in}{0.749675in}}%
\pgfpathlineto{\pgfqpoint{2.637567in}{0.747116in}}%
\pgfpathlineto{\pgfqpoint{2.639935in}{0.739875in}}%
\pgfpathlineto{\pgfqpoint{2.642304in}{0.722360in}}%
\pgfpathlineto{\pgfqpoint{2.643883in}{0.723143in}}%
\pgfpathlineto{\pgfqpoint{2.644278in}{0.723414in}}%
\pgfpathlineto{\pgfqpoint{2.645067in}{0.722116in}}%
\pgfpathlineto{\pgfqpoint{2.647041in}{0.721763in}}%
\pgfpathlineto{\pgfqpoint{2.651383in}{0.706978in}}%
\pgfpathlineto{\pgfqpoint{2.653357in}{0.702916in}}%
\pgfpathlineto{\pgfqpoint{2.655331in}{0.688556in}}%
\pgfpathlineto{\pgfqpoint{2.656120in}{0.683268in}}%
\pgfpathlineto{\pgfqpoint{2.656910in}{0.687215in}}%
\pgfpathlineto{\pgfqpoint{2.657304in}{0.687506in}}%
\pgfpathlineto{\pgfqpoint{2.658489in}{0.683037in}}%
\pgfpathlineto{\pgfqpoint{2.658883in}{0.685740in}}%
\pgfpathlineto{\pgfqpoint{2.662041in}{0.696137in}}%
\pgfpathlineto{\pgfqpoint{2.662436in}{0.696003in}}%
\pgfpathlineto{\pgfqpoint{2.664015in}{0.688048in}}%
\pgfpathlineto{\pgfqpoint{2.664805in}{0.689811in}}%
\pgfpathlineto{\pgfqpoint{2.666778in}{0.689255in}}%
\pgfpathlineto{\pgfqpoint{2.667963in}{0.688253in}}%
\pgfpathlineto{\pgfqpoint{2.668752in}{0.691923in}}%
\pgfpathlineto{\pgfqpoint{2.669542in}{0.690859in}}%
\pgfpathlineto{\pgfqpoint{2.669936in}{0.691869in}}%
\pgfpathlineto{\pgfqpoint{2.673094in}{0.699298in}}%
\pgfpathlineto{\pgfqpoint{2.673489in}{0.698953in}}%
\pgfpathlineto{\pgfqpoint{2.673884in}{0.700604in}}%
\pgfpathlineto{\pgfqpoint{2.675858in}{0.704700in}}%
\pgfpathlineto{\pgfqpoint{2.678226in}{0.706448in}}%
\pgfpathlineto{\pgfqpoint{2.679016in}{0.706701in}}%
\pgfpathlineto{\pgfqpoint{2.682568in}{0.680654in}}%
\pgfpathlineto{\pgfqpoint{2.683753in}{0.683028in}}%
\pgfpathlineto{\pgfqpoint{2.685332in}{0.684946in}}%
\pgfpathlineto{\pgfqpoint{2.686911in}{0.697250in}}%
\pgfpathlineto{\pgfqpoint{2.687305in}{0.695311in}}%
\pgfpathlineto{\pgfqpoint{2.687700in}{0.693459in}}%
\pgfpathlineto{\pgfqpoint{2.688884in}{0.693943in}}%
\pgfpathlineto{\pgfqpoint{2.689279in}{0.693494in}}%
\pgfpathlineto{\pgfqpoint{2.691648in}{0.701123in}}%
\pgfpathlineto{\pgfqpoint{2.694806in}{0.711416in}}%
\pgfpathlineto{\pgfqpoint{2.695990in}{0.713232in}}%
\pgfpathlineto{\pgfqpoint{2.696385in}{0.712691in}}%
\pgfpathlineto{\pgfqpoint{2.697569in}{0.711522in}}%
\pgfpathlineto{\pgfqpoint{2.697964in}{0.712427in}}%
\pgfpathlineto{\pgfqpoint{2.701516in}{0.734321in}}%
\pgfpathlineto{\pgfqpoint{2.702701in}{0.730676in}}%
\pgfpathlineto{\pgfqpoint{2.704280in}{0.726325in}}%
\pgfpathlineto{\pgfqpoint{2.705069in}{0.728009in}}%
\pgfpathlineto{\pgfqpoint{2.707043in}{0.730374in}}%
\pgfpathlineto{\pgfqpoint{2.707438in}{0.729644in}}%
\pgfpathlineto{\pgfqpoint{2.708227in}{0.725307in}}%
\pgfpathlineto{\pgfqpoint{2.709411in}{0.713188in}}%
\pgfpathlineto{\pgfqpoint{2.710201in}{0.716175in}}%
\pgfpathlineto{\pgfqpoint{2.710596in}{0.715628in}}%
\pgfpathlineto{\pgfqpoint{2.712569in}{0.688669in}}%
\pgfpathlineto{\pgfqpoint{2.712964in}{0.690220in}}%
\pgfpathlineto{\pgfqpoint{2.715332in}{0.694524in}}%
\pgfpathlineto{\pgfqpoint{2.715727in}{0.694048in}}%
\pgfpathlineto{\pgfqpoint{2.717306in}{0.692541in}}%
\pgfpathlineto{\pgfqpoint{2.717701in}{0.693067in}}%
\pgfpathlineto{\pgfqpoint{2.719280in}{0.695676in}}%
\pgfpathlineto{\pgfqpoint{2.721254in}{0.701794in}}%
\pgfpathlineto{\pgfqpoint{2.722438in}{0.693035in}}%
\pgfpathlineto{\pgfqpoint{2.722833in}{0.695752in}}%
\pgfpathlineto{\pgfqpoint{2.724412in}{0.703966in}}%
\pgfpathlineto{\pgfqpoint{2.724806in}{0.703482in}}%
\pgfpathlineto{\pgfqpoint{2.727175in}{0.699738in}}%
\pgfpathlineto{\pgfqpoint{2.728754in}{0.693241in}}%
\pgfpathlineto{\pgfqpoint{2.729543in}{0.697956in}}%
\pgfpathlineto{\pgfqpoint{2.736649in}{0.755552in}}%
\pgfpathlineto{\pgfqpoint{2.737833in}{0.752615in}}%
\pgfpathlineto{\pgfqpoint{2.748491in}{0.689182in}}%
\pgfpathlineto{\pgfqpoint{2.749281in}{0.691163in}}%
\pgfpathlineto{\pgfqpoint{2.750860in}{0.696600in}}%
\pgfpathlineto{\pgfqpoint{2.753623in}{0.709636in}}%
\pgfpathlineto{\pgfqpoint{2.755202in}{0.706541in}}%
\pgfpathlineto{\pgfqpoint{2.755597in}{0.708677in}}%
\pgfpathlineto{\pgfqpoint{2.759544in}{0.727795in}}%
\pgfpathlineto{\pgfqpoint{2.760729in}{0.733350in}}%
\pgfpathlineto{\pgfqpoint{2.762702in}{0.738396in}}%
\pgfpathlineto{\pgfqpoint{2.763097in}{0.737736in}}%
\pgfpathlineto{\pgfqpoint{2.764281in}{0.736063in}}%
\pgfpathlineto{\pgfqpoint{2.764676in}{0.736586in}}%
\pgfpathlineto{\pgfqpoint{2.766650in}{0.744771in}}%
\pgfpathlineto{\pgfqpoint{2.767045in}{0.741598in}}%
\pgfpathlineto{\pgfqpoint{2.768229in}{0.732084in}}%
\pgfpathlineto{\pgfqpoint{2.769413in}{0.735510in}}%
\pgfpathlineto{\pgfqpoint{2.772176in}{0.743640in}}%
\pgfpathlineto{\pgfqpoint{2.777703in}{0.713549in}}%
\pgfpathlineto{\pgfqpoint{2.778492in}{0.716163in}}%
\pgfpathlineto{\pgfqpoint{2.780466in}{0.726265in}}%
\pgfpathlineto{\pgfqpoint{2.780861in}{0.725954in}}%
\pgfpathlineto{\pgfqpoint{2.782835in}{0.716129in}}%
\pgfpathlineto{\pgfqpoint{2.784019in}{0.710780in}}%
\pgfpathlineto{\pgfqpoint{2.784414in}{0.711879in}}%
\pgfpathlineto{\pgfqpoint{2.789151in}{0.736336in}}%
\pgfpathlineto{\pgfqpoint{2.791124in}{0.746336in}}%
\pgfpathlineto{\pgfqpoint{2.791914in}{0.743803in}}%
\pgfpathlineto{\pgfqpoint{2.792309in}{0.742463in}}%
\pgfpathlineto{\pgfqpoint{2.792703in}{0.745715in}}%
\pgfpathlineto{\pgfqpoint{2.793493in}{0.750158in}}%
\pgfpathlineto{\pgfqpoint{2.793888in}{0.746529in}}%
\pgfpathlineto{\pgfqpoint{2.798624in}{0.718300in}}%
\pgfpathlineto{\pgfqpoint{2.799019in}{0.719718in}}%
\pgfpathlineto{\pgfqpoint{2.802572in}{0.727633in}}%
\pgfpathlineto{\pgfqpoint{2.804940in}{0.706417in}}%
\pgfpathlineto{\pgfqpoint{2.805730in}{0.711599in}}%
\pgfpathlineto{\pgfqpoint{2.806914in}{0.718104in}}%
\pgfpathlineto{\pgfqpoint{2.807309in}{0.716063in}}%
\pgfpathlineto{\pgfqpoint{2.808493in}{0.703358in}}%
\pgfpathlineto{\pgfqpoint{2.808888in}{0.706025in}}%
\pgfpathlineto{\pgfqpoint{2.809283in}{0.712095in}}%
\pgfpathlineto{\pgfqpoint{2.810467in}{0.710020in}}%
\pgfpathlineto{\pgfqpoint{2.810862in}{0.708706in}}%
\pgfpathlineto{\pgfqpoint{2.812046in}{0.709791in}}%
\pgfpathlineto{\pgfqpoint{2.814809in}{0.715140in}}%
\pgfpathlineto{\pgfqpoint{2.818362in}{0.738079in}}%
\pgfpathlineto{\pgfqpoint{2.819151in}{0.739992in}}%
\pgfpathlineto{\pgfqpoint{2.820336in}{0.742017in}}%
\pgfpathlineto{\pgfqpoint{2.821915in}{0.734476in}}%
\pgfpathlineto{\pgfqpoint{2.822704in}{0.735697in}}%
\pgfpathlineto{\pgfqpoint{2.825862in}{0.741848in}}%
\pgfpathlineto{\pgfqpoint{2.828625in}{0.743218in}}%
\pgfpathlineto{\pgfqpoint{2.831783in}{0.724940in}}%
\pgfpathlineto{\pgfqpoint{2.835336in}{0.701361in}}%
\pgfpathlineto{\pgfqpoint{2.835731in}{0.703518in}}%
\pgfpathlineto{\pgfqpoint{2.844415in}{0.744628in}}%
\pgfpathlineto{\pgfqpoint{2.844810in}{0.743408in}}%
\pgfpathlineto{\pgfqpoint{2.845205in}{0.742822in}}%
\pgfpathlineto{\pgfqpoint{2.845600in}{0.744077in}}%
\pgfpathlineto{\pgfqpoint{2.848363in}{0.756546in}}%
\pgfpathlineto{\pgfqpoint{2.850337in}{0.772877in}}%
\pgfpathlineto{\pgfqpoint{2.851126in}{0.769993in}}%
\pgfpathlineto{\pgfqpoint{2.859811in}{0.717999in}}%
\pgfpathlineto{\pgfqpoint{2.860205in}{0.718062in}}%
\pgfpathlineto{\pgfqpoint{2.860600in}{0.718657in}}%
\pgfpathlineto{\pgfqpoint{2.862179in}{0.739874in}}%
\pgfpathlineto{\pgfqpoint{2.864153in}{0.737631in}}%
\pgfpathlineto{\pgfqpoint{2.864942in}{0.739264in}}%
\pgfpathlineto{\pgfqpoint{2.866521in}{0.743481in}}%
\pgfpathlineto{\pgfqpoint{2.866916in}{0.741666in}}%
\pgfpathlineto{\pgfqpoint{2.869285in}{0.736135in}}%
\pgfpathlineto{\pgfqpoint{2.871258in}{0.733353in}}%
\pgfpathlineto{\pgfqpoint{2.871653in}{0.734450in}}%
\pgfpathlineto{\pgfqpoint{2.874811in}{0.758275in}}%
\pgfpathlineto{\pgfqpoint{2.875601in}{0.757056in}}%
\pgfpathlineto{\pgfqpoint{2.879153in}{0.751620in}}%
\pgfpathlineto{\pgfqpoint{2.882311in}{0.732209in}}%
\pgfpathlineto{\pgfqpoint{2.883495in}{0.721350in}}%
\pgfpathlineto{\pgfqpoint{2.884285in}{0.711329in}}%
\pgfpathlineto{\pgfqpoint{2.885074in}{0.712188in}}%
\pgfpathlineto{\pgfqpoint{2.886259in}{0.714927in}}%
\pgfpathlineto{\pgfqpoint{2.886653in}{0.714463in}}%
\pgfpathlineto{\pgfqpoint{2.887048in}{0.712702in}}%
\pgfpathlineto{\pgfqpoint{2.888627in}{0.723861in}}%
\pgfpathlineto{\pgfqpoint{2.889022in}{0.723269in}}%
\pgfpathlineto{\pgfqpoint{2.893759in}{0.704701in}}%
\pgfpathlineto{\pgfqpoint{2.894548in}{0.705515in}}%
\pgfpathlineto{\pgfqpoint{2.896522in}{0.709159in}}%
\pgfpathlineto{\pgfqpoint{2.898101in}{0.715334in}}%
\pgfpathlineto{\pgfqpoint{2.899285in}{0.716747in}}%
\pgfpathlineto{\pgfqpoint{2.902838in}{0.705260in}}%
\pgfpathlineto{\pgfqpoint{2.903233in}{0.705368in}}%
\pgfpathlineto{\pgfqpoint{2.906391in}{0.729036in}}%
\pgfpathlineto{\pgfqpoint{2.906786in}{0.732781in}}%
\pgfpathlineto{\pgfqpoint{2.907575in}{0.730204in}}%
\pgfpathlineto{\pgfqpoint{2.911523in}{0.712389in}}%
\pgfpathlineto{\pgfqpoint{2.911917in}{0.713898in}}%
\pgfpathlineto{\pgfqpoint{2.912707in}{0.720744in}}%
\pgfpathlineto{\pgfqpoint{2.913496in}{0.716983in}}%
\pgfpathlineto{\pgfqpoint{2.914681in}{0.716495in}}%
\pgfpathlineto{\pgfqpoint{2.915075in}{0.717416in}}%
\pgfpathlineto{\pgfqpoint{2.916260in}{0.718255in}}%
\pgfpathlineto{\pgfqpoint{2.918628in}{0.733567in}}%
\pgfpathlineto{\pgfqpoint{2.919418in}{0.733337in}}%
\pgfpathlineto{\pgfqpoint{2.922181in}{0.728829in}}%
\pgfpathlineto{\pgfqpoint{2.922576in}{0.727908in}}%
\pgfpathlineto{\pgfqpoint{2.923760in}{0.740287in}}%
\pgfpathlineto{\pgfqpoint{2.924549in}{0.737670in}}%
\pgfpathlineto{\pgfqpoint{2.927707in}{0.728874in}}%
\pgfpathlineto{\pgfqpoint{2.930076in}{0.728451in}}%
\pgfpathlineto{\pgfqpoint{2.930865in}{0.729596in}}%
\pgfpathlineto{\pgfqpoint{2.931260in}{0.728787in}}%
\pgfpathlineto{\pgfqpoint{2.933234in}{0.721200in}}%
\pgfpathlineto{\pgfqpoint{2.935602in}{0.711357in}}%
\pgfpathlineto{\pgfqpoint{2.936787in}{0.708758in}}%
\pgfpathlineto{\pgfqpoint{2.937181in}{0.709793in}}%
\pgfpathlineto{\pgfqpoint{2.937576in}{0.717695in}}%
\pgfpathlineto{\pgfqpoint{2.938760in}{0.713109in}}%
\pgfpathlineto{\pgfqpoint{2.941129in}{0.708249in}}%
\pgfpathlineto{\pgfqpoint{2.944287in}{0.713508in}}%
\pgfpathlineto{\pgfqpoint{2.945076in}{0.716379in}}%
\pgfpathlineto{\pgfqpoint{2.945866in}{0.713651in}}%
\pgfpathlineto{\pgfqpoint{2.948234in}{0.715801in}}%
\pgfpathlineto{\pgfqpoint{2.949024in}{0.714173in}}%
\pgfpathlineto{\pgfqpoint{2.949419in}{0.714672in}}%
\pgfpathlineto{\pgfqpoint{2.952577in}{0.731131in}}%
\pgfpathlineto{\pgfqpoint{2.954156in}{0.726626in}}%
\pgfpathlineto{\pgfqpoint{2.956129in}{0.729862in}}%
\pgfpathlineto{\pgfqpoint{2.956919in}{0.728790in}}%
\pgfpathlineto{\pgfqpoint{2.958893in}{0.723977in}}%
\pgfpathlineto{\pgfqpoint{2.959682in}{0.727423in}}%
\pgfpathlineto{\pgfqpoint{2.960077in}{0.721251in}}%
\pgfpathlineto{\pgfqpoint{2.961261in}{0.725485in}}%
\pgfpathlineto{\pgfqpoint{2.962051in}{0.729511in}}%
\pgfpathlineto{\pgfqpoint{2.962840in}{0.728067in}}%
\pgfpathlineto{\pgfqpoint{2.963235in}{0.728226in}}%
\pgfpathlineto{\pgfqpoint{2.965998in}{0.736979in}}%
\pgfpathlineto{\pgfqpoint{2.969551in}{0.746614in}}%
\pgfpathlineto{\pgfqpoint{2.969945in}{0.745949in}}%
\pgfpathlineto{\pgfqpoint{2.977840in}{0.711558in}}%
\pgfpathlineto{\pgfqpoint{2.980209in}{0.729847in}}%
\pgfpathlineto{\pgfqpoint{2.981788in}{0.735191in}}%
\pgfpathlineto{\pgfqpoint{2.982183in}{0.733579in}}%
\pgfpathlineto{\pgfqpoint{2.982972in}{0.725336in}}%
\pgfpathlineto{\pgfqpoint{2.983762in}{0.728202in}}%
\pgfpathlineto{\pgfqpoint{2.988893in}{0.754276in}}%
\pgfpathlineto{\pgfqpoint{2.989683in}{0.765382in}}%
\pgfpathlineto{\pgfqpoint{2.990867in}{0.763484in}}%
\pgfpathlineto{\pgfqpoint{2.992841in}{0.759138in}}%
\pgfpathlineto{\pgfqpoint{2.998367in}{0.745651in}}%
\pgfpathlineto{\pgfqpoint{2.999157in}{0.745372in}}%
\pgfpathlineto{\pgfqpoint{2.999946in}{0.744474in}}%
\pgfpathlineto{\pgfqpoint{3.000341in}{0.745796in}}%
\pgfpathlineto{\pgfqpoint{3.002710in}{0.748673in}}%
\pgfpathlineto{\pgfqpoint{3.006262in}{0.742156in}}%
\pgfpathlineto{\pgfqpoint{3.007052in}{0.741632in}}%
\pgfpathlineto{\pgfqpoint{3.007447in}{0.744446in}}%
\pgfpathlineto{\pgfqpoint{3.007841in}{0.739940in}}%
\pgfpathlineto{\pgfqpoint{3.009815in}{0.734147in}}%
\pgfpathlineto{\pgfqpoint{3.010605in}{0.735652in}}%
\pgfpathlineto{\pgfqpoint{3.011394in}{0.736670in}}%
\pgfpathlineto{\pgfqpoint{3.011789in}{0.736413in}}%
\pgfpathlineto{\pgfqpoint{3.012578in}{0.732610in}}%
\pgfpathlineto{\pgfqpoint{3.012973in}{0.738036in}}%
\pgfpathlineto{\pgfqpoint{3.013368in}{0.737979in}}%
\pgfpathlineto{\pgfqpoint{3.016526in}{0.749822in}}%
\pgfpathlineto{\pgfqpoint{3.016921in}{0.748869in}}%
\pgfpathlineto{\pgfqpoint{3.022447in}{0.728309in}}%
\pgfpathlineto{\pgfqpoint{3.025210in}{0.708647in}}%
\pgfpathlineto{\pgfqpoint{3.025605in}{0.708722in}}%
\pgfpathlineto{\pgfqpoint{3.028368in}{0.718643in}}%
\pgfpathlineto{\pgfqpoint{3.029553in}{0.717770in}}%
\pgfpathlineto{\pgfqpoint{3.029947in}{0.717313in}}%
\pgfpathlineto{\pgfqpoint{3.030342in}{0.718390in}}%
\pgfpathlineto{\pgfqpoint{3.032316in}{0.720475in}}%
\pgfpathlineto{\pgfqpoint{3.032711in}{0.719882in}}%
\pgfpathlineto{\pgfqpoint{3.035079in}{0.707226in}}%
\pgfpathlineto{\pgfqpoint{3.035869in}{0.704638in}}%
\pgfpathlineto{\pgfqpoint{3.036658in}{0.706263in}}%
\pgfpathlineto{\pgfqpoint{3.037448in}{0.706330in}}%
\pgfpathlineto{\pgfqpoint{3.041000in}{0.734893in}}%
\pgfpathlineto{\pgfqpoint{3.042185in}{0.734346in}}%
\pgfpathlineto{\pgfqpoint{3.045737in}{0.724566in}}%
\pgfpathlineto{\pgfqpoint{3.051658in}{0.704370in}}%
\pgfpathlineto{\pgfqpoint{3.052843in}{0.704773in}}%
\pgfpathlineto{\pgfqpoint{3.055211in}{0.713492in}}%
\pgfpathlineto{\pgfqpoint{3.055606in}{0.711528in}}%
\pgfpathlineto{\pgfqpoint{3.056395in}{0.708977in}}%
\pgfpathlineto{\pgfqpoint{3.057185in}{0.711273in}}%
\pgfpathlineto{\pgfqpoint{3.059553in}{0.722024in}}%
\pgfpathlineto{\pgfqpoint{3.059948in}{0.721257in}}%
\pgfpathlineto{\pgfqpoint{3.061132in}{0.715509in}}%
\pgfpathlineto{\pgfqpoint{3.061922in}{0.719822in}}%
\pgfpathlineto{\pgfqpoint{3.064685in}{0.727287in}}%
\pgfpathlineto{\pgfqpoint{3.067448in}{0.737350in}}%
\pgfpathlineto{\pgfqpoint{3.069817in}{0.746826in}}%
\pgfpathlineto{\pgfqpoint{3.070212in}{0.746239in}}%
\pgfpathlineto{\pgfqpoint{3.082054in}{0.705736in}}%
\pgfpathlineto{\pgfqpoint{3.084028in}{0.691536in}}%
\pgfpathlineto{\pgfqpoint{3.084423in}{0.691718in}}%
\pgfpathlineto{\pgfqpoint{3.084817in}{0.690838in}}%
\pgfpathlineto{\pgfqpoint{3.086396in}{0.688715in}}%
\pgfpathlineto{\pgfqpoint{3.086791in}{0.689361in}}%
\pgfpathlineto{\pgfqpoint{3.087581in}{0.703762in}}%
\pgfpathlineto{\pgfqpoint{3.089160in}{0.715708in}}%
\pgfpathlineto{\pgfqpoint{3.089554in}{0.712803in}}%
\pgfpathlineto{\pgfqpoint{3.091923in}{0.700726in}}%
\pgfpathlineto{\pgfqpoint{3.092712in}{0.701021in}}%
\pgfpathlineto{\pgfqpoint{3.093107in}{0.700666in}}%
\pgfpathlineto{\pgfqpoint{3.093502in}{0.702292in}}%
\pgfpathlineto{\pgfqpoint{3.095476in}{0.710389in}}%
\pgfpathlineto{\pgfqpoint{3.097449in}{0.720106in}}%
\pgfpathlineto{\pgfqpoint{3.098239in}{0.718977in}}%
\pgfpathlineto{\pgfqpoint{3.101792in}{0.716710in}}%
\pgfpathlineto{\pgfqpoint{3.102186in}{0.717674in}}%
\pgfpathlineto{\pgfqpoint{3.108502in}{0.741264in}}%
\pgfpathlineto{\pgfqpoint{3.110871in}{0.710653in}}%
\pgfpathlineto{\pgfqpoint{3.111266in}{0.711337in}}%
\pgfpathlineto{\pgfqpoint{3.113634in}{0.722970in}}%
\pgfpathlineto{\pgfqpoint{3.114424in}{0.716708in}}%
\pgfpathlineto{\pgfqpoint{3.115608in}{0.705564in}}%
\pgfpathlineto{\pgfqpoint{3.116397in}{0.712693in}}%
\pgfpathlineto{\pgfqpoint{3.118766in}{0.723874in}}%
\pgfpathlineto{\pgfqpoint{3.119950in}{0.722420in}}%
\pgfpathlineto{\pgfqpoint{3.121924in}{0.717861in}}%
\pgfpathlineto{\pgfqpoint{3.122319in}{0.718198in}}%
\pgfpathlineto{\pgfqpoint{3.123108in}{0.720359in}}%
\pgfpathlineto{\pgfqpoint{3.123898in}{0.717954in}}%
\pgfpathlineto{\pgfqpoint{3.132187in}{0.689023in}}%
\pgfpathlineto{\pgfqpoint{3.134556in}{0.669817in}}%
\pgfpathlineto{\pgfqpoint{3.135345in}{0.668317in}}%
\pgfpathlineto{\pgfqpoint{3.135740in}{0.669883in}}%
\pgfpathlineto{\pgfqpoint{3.138108in}{0.678358in}}%
\pgfpathlineto{\pgfqpoint{3.138503in}{0.676009in}}%
\pgfpathlineto{\pgfqpoint{3.139293in}{0.671427in}}%
\pgfpathlineto{\pgfqpoint{3.140082in}{0.673092in}}%
\pgfpathlineto{\pgfqpoint{3.140872in}{0.674137in}}%
\pgfpathlineto{\pgfqpoint{3.146398in}{0.696478in}}%
\pgfpathlineto{\pgfqpoint{3.147582in}{0.689748in}}%
\pgfpathlineto{\pgfqpoint{3.149161in}{0.681251in}}%
\pgfpathlineto{\pgfqpoint{3.149951in}{0.682709in}}%
\pgfpathlineto{\pgfqpoint{3.151925in}{0.705255in}}%
\pgfpathlineto{\pgfqpoint{3.154293in}{0.725612in}}%
\pgfpathlineto{\pgfqpoint{3.156267in}{0.734372in}}%
\pgfpathlineto{\pgfqpoint{3.156662in}{0.733127in}}%
\pgfpathlineto{\pgfqpoint{3.159425in}{0.708646in}}%
\pgfpathlineto{\pgfqpoint{3.162188in}{0.683157in}}%
\pgfpathlineto{\pgfqpoint{3.162583in}{0.681542in}}%
\pgfpathlineto{\pgfqpoint{3.162978in}{0.683299in}}%
\pgfpathlineto{\pgfqpoint{3.165741in}{0.713679in}}%
\pgfpathlineto{\pgfqpoint{3.166925in}{0.713483in}}%
\pgfpathlineto{\pgfqpoint{3.171267in}{0.694891in}}%
\pgfpathlineto{\pgfqpoint{3.171662in}{0.696851in}}%
\pgfpathlineto{\pgfqpoint{3.173636in}{0.698441in}}%
\pgfpathlineto{\pgfqpoint{3.174425in}{0.699962in}}%
\pgfpathlineto{\pgfqpoint{3.174820in}{0.700512in}}%
\pgfpathlineto{\pgfqpoint{3.175215in}{0.698598in}}%
\pgfpathlineto{\pgfqpoint{3.176794in}{0.694376in}}%
\pgfpathlineto{\pgfqpoint{3.177189in}{0.694438in}}%
\pgfpathlineto{\pgfqpoint{3.181531in}{0.701468in}}%
\pgfpathlineto{\pgfqpoint{3.181926in}{0.702862in}}%
\pgfpathlineto{\pgfqpoint{3.182320in}{0.702098in}}%
\pgfpathlineto{\pgfqpoint{3.184294in}{0.688703in}}%
\pgfpathlineto{\pgfqpoint{3.185873in}{0.685226in}}%
\pgfpathlineto{\pgfqpoint{3.186268in}{0.686281in}}%
\pgfpathlineto{\pgfqpoint{3.188242in}{0.689338in}}%
\pgfpathlineto{\pgfqpoint{3.191005in}{0.709607in}}%
\pgfpathlineto{\pgfqpoint{3.191400in}{0.710523in}}%
\pgfpathlineto{\pgfqpoint{3.191794in}{0.708449in}}%
\pgfpathlineto{\pgfqpoint{3.192189in}{0.707820in}}%
\pgfpathlineto{\pgfqpoint{3.192979in}{0.709086in}}%
\pgfpathlineto{\pgfqpoint{3.193373in}{0.710389in}}%
\pgfpathlineto{\pgfqpoint{3.194163in}{0.709044in}}%
\pgfpathlineto{\pgfqpoint{3.196926in}{0.685001in}}%
\pgfpathlineto{\pgfqpoint{3.197321in}{0.682398in}}%
\pgfpathlineto{\pgfqpoint{3.197321in}{0.682398in}}%
\pgfusepath{stroke}%
\end{pgfscope}%
\begin{pgfscope}%
\pgfpathrectangle{\pgfqpoint{0.608025in}{0.484444in}}{\pgfqpoint{2.712595in}{1.541287in}}%
\pgfusepath{clip}%
\pgfsetbuttcap%
\pgfsetmiterjoin%
\definecolor{currentfill}{rgb}{0.172549,0.627451,0.172549}%
\pgfsetfillcolor{currentfill}%
\pgfsetlinewidth{1.003750pt}%
\definecolor{currentstroke}{rgb}{0.172549,0.627451,0.172549}%
\pgfsetstrokecolor{currentstroke}%
\pgfsetdash{}{0pt}%
\pgfsys@defobject{currentmarker}{\pgfqpoint{-0.020833in}{-0.020833in}}{\pgfqpoint{0.020833in}{0.020833in}}{%
\pgfpathmoveto{\pgfqpoint{-0.000000in}{-0.020833in}}%
\pgfpathlineto{\pgfqpoint{0.020833in}{0.020833in}}%
\pgfpathlineto{\pgfqpoint{-0.020833in}{0.020833in}}%
\pgfpathlineto{\pgfqpoint{-0.000000in}{-0.020833in}}%
\pgfpathclose%
\pgfusepath{stroke,fill}%
}%
\begin{pgfscope}%
\pgfsys@transformshift{0.747115in}{1.940472in}%
\pgfsys@useobject{currentmarker}{}%
\end{pgfscope}%
\begin{pgfscope}%
\pgfsys@transformshift{0.826065in}{1.698522in}%
\pgfsys@useobject{currentmarker}{}%
\end{pgfscope}%
\begin{pgfscope}%
\pgfsys@transformshift{0.905014in}{1.427778in}%
\pgfsys@useobject{currentmarker}{}%
\end{pgfscope}%
\begin{pgfscope}%
\pgfsys@transformshift{0.983964in}{1.269862in}%
\pgfsys@useobject{currentmarker}{}%
\end{pgfscope}%
\begin{pgfscope}%
\pgfsys@transformshift{1.062914in}{1.213760in}%
\pgfsys@useobject{currentmarker}{}%
\end{pgfscope}%
\begin{pgfscope}%
\pgfsys@transformshift{1.141864in}{1.134252in}%
\pgfsys@useobject{currentmarker}{}%
\end{pgfscope}%
\begin{pgfscope}%
\pgfsys@transformshift{1.220813in}{1.115178in}%
\pgfsys@useobject{currentmarker}{}%
\end{pgfscope}%
\begin{pgfscope}%
\pgfsys@transformshift{1.299763in}{1.045327in}%
\pgfsys@useobject{currentmarker}{}%
\end{pgfscope}%
\begin{pgfscope}%
\pgfsys@transformshift{1.378713in}{0.970929in}%
\pgfsys@useobject{currentmarker}{}%
\end{pgfscope}%
\begin{pgfscope}%
\pgfsys@transformshift{1.457663in}{0.932388in}%
\pgfsys@useobject{currentmarker}{}%
\end{pgfscope}%
\begin{pgfscope}%
\pgfsys@transformshift{1.536613in}{0.919815in}%
\pgfsys@useobject{currentmarker}{}%
\end{pgfscope}%
\begin{pgfscope}%
\pgfsys@transformshift{1.615562in}{0.856427in}%
\pgfsys@useobject{currentmarker}{}%
\end{pgfscope}%
\begin{pgfscope}%
\pgfsys@transformshift{1.694512in}{0.865042in}%
\pgfsys@useobject{currentmarker}{}%
\end{pgfscope}%
\begin{pgfscope}%
\pgfsys@transformshift{1.773462in}{0.849340in}%
\pgfsys@useobject{currentmarker}{}%
\end{pgfscope}%
\begin{pgfscope}%
\pgfsys@transformshift{1.852412in}{0.787180in}%
\pgfsys@useobject{currentmarker}{}%
\end{pgfscope}%
\begin{pgfscope}%
\pgfsys@transformshift{1.931361in}{0.846963in}%
\pgfsys@useobject{currentmarker}{}%
\end{pgfscope}%
\begin{pgfscope}%
\pgfsys@transformshift{2.010311in}{0.791211in}%
\pgfsys@useobject{currentmarker}{}%
\end{pgfscope}%
\begin{pgfscope}%
\pgfsys@transformshift{2.089261in}{0.739418in}%
\pgfsys@useobject{currentmarker}{}%
\end{pgfscope}%
\begin{pgfscope}%
\pgfsys@transformshift{2.168211in}{0.762462in}%
\pgfsys@useobject{currentmarker}{}%
\end{pgfscope}%
\begin{pgfscope}%
\pgfsys@transformshift{2.247160in}{0.743219in}%
\pgfsys@useobject{currentmarker}{}%
\end{pgfscope}%
\begin{pgfscope}%
\pgfsys@transformshift{2.326110in}{0.736516in}%
\pgfsys@useobject{currentmarker}{}%
\end{pgfscope}%
\begin{pgfscope}%
\pgfsys@transformshift{2.405060in}{0.779143in}%
\pgfsys@useobject{currentmarker}{}%
\end{pgfscope}%
\begin{pgfscope}%
\pgfsys@transformshift{2.484010in}{0.676032in}%
\pgfsys@useobject{currentmarker}{}%
\end{pgfscope}%
\begin{pgfscope}%
\pgfsys@transformshift{2.562959in}{0.739755in}%
\pgfsys@useobject{currentmarker}{}%
\end{pgfscope}%
\begin{pgfscope}%
\pgfsys@transformshift{2.641909in}{0.725290in}%
\pgfsys@useobject{currentmarker}{}%
\end{pgfscope}%
\begin{pgfscope}%
\pgfsys@transformshift{2.720859in}{0.701225in}%
\pgfsys@useobject{currentmarker}{}%
\end{pgfscope}%
\begin{pgfscope}%
\pgfsys@transformshift{2.799809in}{0.722532in}%
\pgfsys@useobject{currentmarker}{}%
\end{pgfscope}%
\begin{pgfscope}%
\pgfsys@transformshift{2.878759in}{0.752816in}%
\pgfsys@useobject{currentmarker}{}%
\end{pgfscope}%
\begin{pgfscope}%
\pgfsys@transformshift{2.957708in}{0.724623in}%
\pgfsys@useobject{currentmarker}{}%
\end{pgfscope}%
\begin{pgfscope}%
\pgfsys@transformshift{3.036658in}{0.706263in}%
\pgfsys@useobject{currentmarker}{}%
\end{pgfscope}%
\begin{pgfscope}%
\pgfsys@transformshift{3.115608in}{0.705564in}%
\pgfsys@useobject{currentmarker}{}%
\end{pgfscope}%
\begin{pgfscope}%
\pgfsys@transformshift{3.194558in}{0.706835in}%
\pgfsys@useobject{currentmarker}{}%
\end{pgfscope}%
\end{pgfscope}%
\begin{pgfscope}%
\pgfpathrectangle{\pgfqpoint{0.608025in}{0.484444in}}{\pgfqpoint{2.712595in}{1.541287in}}%
\pgfusepath{clip}%
\pgfsetrectcap%
\pgfsetroundjoin%
\pgfsetlinewidth{1.505625pt}%
\definecolor{currentstroke}{rgb}{0.839216,0.152941,0.156863}%
\pgfsetstrokecolor{currentstroke}%
\pgfsetdash{}{0pt}%
\pgfpathmoveto{\pgfqpoint{0.731325in}{1.955542in}}%
\pgfpathlineto{\pgfqpoint{0.734878in}{1.954172in}}%
\pgfpathlineto{\pgfqpoint{0.736062in}{1.954279in}}%
\pgfpathlineto{\pgfqpoint{0.736457in}{1.953946in}}%
\pgfpathlineto{\pgfqpoint{0.744746in}{1.945032in}}%
\pgfpathlineto{\pgfqpoint{0.749483in}{1.934738in}}%
\pgfpathlineto{\pgfqpoint{0.762510in}{1.888625in}}%
\pgfpathlineto{\pgfqpoint{0.773563in}{1.854587in}}%
\pgfpathlineto{\pgfqpoint{0.787379in}{1.819912in}}%
\pgfpathlineto{\pgfqpoint{0.795669in}{1.800641in}}%
\pgfpathlineto{\pgfqpoint{0.811064in}{1.784601in}}%
\pgfpathlineto{\pgfqpoint{0.815801in}{1.787714in}}%
\pgfpathlineto{\pgfqpoint{0.817775in}{1.779983in}}%
\pgfpathlineto{\pgfqpoint{0.826459in}{1.721173in}}%
\pgfpathlineto{\pgfqpoint{0.831591in}{1.670443in}}%
\pgfpathlineto{\pgfqpoint{0.840276in}{1.620014in}}%
\pgfpathlineto{\pgfqpoint{0.846986in}{1.582129in}}%
\pgfpathlineto{\pgfqpoint{0.854487in}{1.544020in}}%
\pgfpathlineto{\pgfqpoint{0.856855in}{1.540635in}}%
\pgfpathlineto{\pgfqpoint{0.857645in}{1.542571in}}%
\pgfpathlineto{\pgfqpoint{0.858039in}{1.543455in}}%
\pgfpathlineto{\pgfqpoint{0.858434in}{1.541998in}}%
\pgfpathlineto{\pgfqpoint{0.863171in}{1.525653in}}%
\pgfpathlineto{\pgfqpoint{0.864355in}{1.518061in}}%
\pgfpathlineto{\pgfqpoint{0.867513in}{1.498565in}}%
\pgfpathlineto{\pgfqpoint{0.868698in}{1.493758in}}%
\pgfpathlineto{\pgfqpoint{0.869092in}{1.494397in}}%
\pgfpathlineto{\pgfqpoint{0.872250in}{1.538645in}}%
\pgfpathlineto{\pgfqpoint{0.873040in}{1.538065in}}%
\pgfpathlineto{\pgfqpoint{0.873829in}{1.536880in}}%
\pgfpathlineto{\pgfqpoint{0.874224in}{1.538442in}}%
\pgfpathlineto{\pgfqpoint{0.879356in}{1.561277in}}%
\pgfpathlineto{\pgfqpoint{0.880540in}{1.561994in}}%
\pgfpathlineto{\pgfqpoint{0.880935in}{1.561625in}}%
\pgfpathlineto{\pgfqpoint{0.882514in}{1.557177in}}%
\pgfpathlineto{\pgfqpoint{0.884882in}{1.539501in}}%
\pgfpathlineto{\pgfqpoint{0.888435in}{1.488080in}}%
\pgfpathlineto{\pgfqpoint{0.890014in}{1.494482in}}%
\pgfpathlineto{\pgfqpoint{0.893567in}{1.514103in}}%
\pgfpathlineto{\pgfqpoint{0.894751in}{1.510368in}}%
\pgfpathlineto{\pgfqpoint{0.895540in}{1.508512in}}%
\pgfpathlineto{\pgfqpoint{0.903830in}{1.406239in}}%
\pgfpathlineto{\pgfqpoint{0.908962in}{1.437424in}}%
\pgfpathlineto{\pgfqpoint{0.909357in}{1.431960in}}%
\pgfpathlineto{\pgfqpoint{0.913699in}{1.371824in}}%
\pgfpathlineto{\pgfqpoint{0.914488in}{1.370970in}}%
\pgfpathlineto{\pgfqpoint{0.914883in}{1.372272in}}%
\pgfpathlineto{\pgfqpoint{0.920410in}{1.396711in}}%
\pgfpathlineto{\pgfqpoint{0.921594in}{1.399468in}}%
\pgfpathlineto{\pgfqpoint{0.922383in}{1.398444in}}%
\pgfpathlineto{\pgfqpoint{0.923173in}{1.397117in}}%
\pgfpathlineto{\pgfqpoint{0.924752in}{1.405952in}}%
\pgfpathlineto{\pgfqpoint{0.925147in}{1.404280in}}%
\pgfpathlineto{\pgfqpoint{0.926331in}{1.393518in}}%
\pgfpathlineto{\pgfqpoint{0.929489in}{1.353063in}}%
\pgfpathlineto{\pgfqpoint{0.932252in}{1.332903in}}%
\pgfpathlineto{\pgfqpoint{0.933831in}{1.329410in}}%
\pgfpathlineto{\pgfqpoint{0.934226in}{1.327767in}}%
\pgfpathlineto{\pgfqpoint{0.934621in}{1.330942in}}%
\pgfpathlineto{\pgfqpoint{0.939358in}{1.402922in}}%
\pgfpathlineto{\pgfqpoint{0.939752in}{1.402135in}}%
\pgfpathlineto{\pgfqpoint{0.942910in}{1.370520in}}%
\pgfpathlineto{\pgfqpoint{0.943700in}{1.372120in}}%
\pgfpathlineto{\pgfqpoint{0.946858in}{1.378771in}}%
\pgfpathlineto{\pgfqpoint{0.948042in}{1.377036in}}%
\pgfpathlineto{\pgfqpoint{0.949621in}{1.371630in}}%
\pgfpathlineto{\pgfqpoint{0.950411in}{1.373526in}}%
\pgfpathlineto{\pgfqpoint{0.951200in}{1.373966in}}%
\pgfpathlineto{\pgfqpoint{0.951595in}{1.373345in}}%
\pgfpathlineto{\pgfqpoint{0.954358in}{1.365047in}}%
\pgfpathlineto{\pgfqpoint{0.954753in}{1.368286in}}%
\pgfpathlineto{\pgfqpoint{0.956332in}{1.383569in}}%
\pgfpathlineto{\pgfqpoint{0.957121in}{1.379264in}}%
\pgfpathlineto{\pgfqpoint{0.961069in}{1.323701in}}%
\pgfpathlineto{\pgfqpoint{0.963832in}{1.299603in}}%
\pgfpathlineto{\pgfqpoint{0.964227in}{1.297761in}}%
\pgfpathlineto{\pgfqpoint{0.964621in}{1.298027in}}%
\pgfpathlineto{\pgfqpoint{0.965806in}{1.305528in}}%
\pgfpathlineto{\pgfqpoint{0.966595in}{1.304444in}}%
\pgfpathlineto{\pgfqpoint{0.969753in}{1.281019in}}%
\pgfpathlineto{\pgfqpoint{0.973701in}{1.272692in}}%
\pgfpathlineto{\pgfqpoint{0.974885in}{1.275688in}}%
\pgfpathlineto{\pgfqpoint{0.977648in}{1.319448in}}%
\pgfpathlineto{\pgfqpoint{0.978438in}{1.312428in}}%
\pgfpathlineto{\pgfqpoint{0.980806in}{1.281881in}}%
\pgfpathlineto{\pgfqpoint{0.981596in}{1.289014in}}%
\pgfpathlineto{\pgfqpoint{0.983175in}{1.302889in}}%
\pgfpathlineto{\pgfqpoint{0.985148in}{1.252730in}}%
\pgfpathlineto{\pgfqpoint{0.985938in}{1.266261in}}%
\pgfpathlineto{\pgfqpoint{0.989096in}{1.308642in}}%
\pgfpathlineto{\pgfqpoint{0.990280in}{1.324025in}}%
\pgfpathlineto{\pgfqpoint{0.991070in}{1.316686in}}%
\pgfpathlineto{\pgfqpoint{0.993438in}{1.282650in}}%
\pgfpathlineto{\pgfqpoint{0.994622in}{1.286431in}}%
\pgfpathlineto{\pgfqpoint{0.995412in}{1.287577in}}%
\pgfpathlineto{\pgfqpoint{0.995807in}{1.287101in}}%
\pgfpathlineto{\pgfqpoint{0.997386in}{1.278117in}}%
\pgfpathlineto{\pgfqpoint{0.997780in}{1.274332in}}%
\pgfpathlineto{\pgfqpoint{0.998965in}{1.276938in}}%
\pgfpathlineto{\pgfqpoint{1.000149in}{1.273891in}}%
\pgfpathlineto{\pgfqpoint{1.001333in}{1.266723in}}%
\pgfpathlineto{\pgfqpoint{1.001728in}{1.267172in}}%
\pgfpathlineto{\pgfqpoint{1.003702in}{1.299636in}}%
\pgfpathlineto{\pgfqpoint{1.004096in}{1.293245in}}%
\pgfpathlineto{\pgfqpoint{1.005281in}{1.272627in}}%
\pgfpathlineto{\pgfqpoint{1.006070in}{1.282368in}}%
\pgfpathlineto{\pgfqpoint{1.006860in}{1.302197in}}%
\pgfpathlineto{\pgfqpoint{1.008044in}{1.297049in}}%
\pgfpathlineto{\pgfqpoint{1.010807in}{1.268157in}}%
\pgfpathlineto{\pgfqpoint{1.014360in}{1.251793in}}%
\pgfpathlineto{\pgfqpoint{1.018702in}{1.207804in}}%
\pgfpathlineto{\pgfqpoint{1.019492in}{1.215368in}}%
\pgfpathlineto{\pgfqpoint{1.023834in}{1.254538in}}%
\pgfpathlineto{\pgfqpoint{1.024623in}{1.257517in}}%
\pgfpathlineto{\pgfqpoint{1.025018in}{1.254344in}}%
\pgfpathlineto{\pgfqpoint{1.033308in}{1.147657in}}%
\pgfpathlineto{\pgfqpoint{1.034887in}{1.139774in}}%
\pgfpathlineto{\pgfqpoint{1.036861in}{1.163848in}}%
\pgfpathlineto{\pgfqpoint{1.037255in}{1.163390in}}%
\pgfpathlineto{\pgfqpoint{1.038834in}{1.161516in}}%
\pgfpathlineto{\pgfqpoint{1.040413in}{1.175659in}}%
\pgfpathlineto{\pgfqpoint{1.044361in}{1.223699in}}%
\pgfpathlineto{\pgfqpoint{1.047124in}{1.258800in}}%
\pgfpathlineto{\pgfqpoint{1.047913in}{1.257530in}}%
\pgfpathlineto{\pgfqpoint{1.050677in}{1.269865in}}%
\pgfpathlineto{\pgfqpoint{1.051071in}{1.269654in}}%
\pgfpathlineto{\pgfqpoint{1.054229in}{1.224992in}}%
\pgfpathlineto{\pgfqpoint{1.063309in}{1.160837in}}%
\pgfpathlineto{\pgfqpoint{1.064888in}{1.147038in}}%
\pgfpathlineto{\pgfqpoint{1.065282in}{1.152881in}}%
\pgfpathlineto{\pgfqpoint{1.071993in}{1.229304in}}%
\pgfpathlineto{\pgfqpoint{1.075151in}{1.247841in}}%
\pgfpathlineto{\pgfqpoint{1.078704in}{1.266517in}}%
\pgfpathlineto{\pgfqpoint{1.079099in}{1.262999in}}%
\pgfpathlineto{\pgfqpoint{1.082651in}{1.217839in}}%
\pgfpathlineto{\pgfqpoint{1.085809in}{1.163794in}}%
\pgfpathlineto{\pgfqpoint{1.086599in}{1.170287in}}%
\pgfpathlineto{\pgfqpoint{1.088967in}{1.182363in}}%
\pgfpathlineto{\pgfqpoint{1.090941in}{1.140537in}}%
\pgfpathlineto{\pgfqpoint{1.094889in}{1.066721in}}%
\pgfpathlineto{\pgfqpoint{1.095283in}{1.072857in}}%
\pgfpathlineto{\pgfqpoint{1.096073in}{1.076590in}}%
\pgfpathlineto{\pgfqpoint{1.096468in}{1.073842in}}%
\pgfpathlineto{\pgfqpoint{1.096862in}{1.072711in}}%
\pgfpathlineto{\pgfqpoint{1.109889in}{1.234591in}}%
\pgfpathlineto{\pgfqpoint{1.110284in}{1.234947in}}%
\pgfpathlineto{\pgfqpoint{1.111468in}{1.226339in}}%
\pgfpathlineto{\pgfqpoint{1.111863in}{1.230221in}}%
\pgfpathlineto{\pgfqpoint{1.112258in}{1.230961in}}%
\pgfpathlineto{\pgfqpoint{1.112652in}{1.228643in}}%
\pgfpathlineto{\pgfqpoint{1.114626in}{1.215195in}}%
\pgfpathlineto{\pgfqpoint{1.115021in}{1.219599in}}%
\pgfpathlineto{\pgfqpoint{1.116205in}{1.227854in}}%
\pgfpathlineto{\pgfqpoint{1.116600in}{1.224281in}}%
\pgfpathlineto{\pgfqpoint{1.118574in}{1.193281in}}%
\pgfpathlineto{\pgfqpoint{1.121732in}{1.149209in}}%
\pgfpathlineto{\pgfqpoint{1.122916in}{1.151425in}}%
\pgfpathlineto{\pgfqpoint{1.125284in}{1.109685in}}%
\pgfpathlineto{\pgfqpoint{1.125679in}{1.115438in}}%
\pgfpathlineto{\pgfqpoint{1.127258in}{1.153512in}}%
\pgfpathlineto{\pgfqpoint{1.128047in}{1.146421in}}%
\pgfpathlineto{\pgfqpoint{1.131205in}{1.126811in}}%
\pgfpathlineto{\pgfqpoint{1.131600in}{1.127810in}}%
\pgfpathlineto{\pgfqpoint{1.134363in}{1.106401in}}%
\pgfpathlineto{\pgfqpoint{1.136337in}{1.116285in}}%
\pgfpathlineto{\pgfqpoint{1.136732in}{1.114915in}}%
\pgfpathlineto{\pgfqpoint{1.137127in}{1.116926in}}%
\pgfpathlineto{\pgfqpoint{1.138311in}{1.119695in}}%
\pgfpathlineto{\pgfqpoint{1.139100in}{1.118429in}}%
\pgfpathlineto{\pgfqpoint{1.141074in}{1.102810in}}%
\pgfpathlineto{\pgfqpoint{1.142258in}{1.089385in}}%
\pgfpathlineto{\pgfqpoint{1.143443in}{1.089641in}}%
\pgfpathlineto{\pgfqpoint{1.143837in}{1.088632in}}%
\pgfpathlineto{\pgfqpoint{1.149364in}{1.143553in}}%
\pgfpathlineto{\pgfqpoint{1.149759in}{1.145214in}}%
\pgfpathlineto{\pgfqpoint{1.150153in}{1.143075in}}%
\pgfpathlineto{\pgfqpoint{1.154101in}{1.115235in}}%
\pgfpathlineto{\pgfqpoint{1.155285in}{1.106553in}}%
\pgfpathlineto{\pgfqpoint{1.155680in}{1.108633in}}%
\pgfpathlineto{\pgfqpoint{1.156469in}{1.122902in}}%
\pgfpathlineto{\pgfqpoint{1.157259in}{1.117606in}}%
\pgfpathlineto{\pgfqpoint{1.158048in}{1.125093in}}%
\pgfpathlineto{\pgfqpoint{1.158443in}{1.118724in}}%
\pgfpathlineto{\pgfqpoint{1.161996in}{1.063053in}}%
\pgfpathlineto{\pgfqpoint{1.162391in}{1.063930in}}%
\pgfpathlineto{\pgfqpoint{1.163575in}{1.082766in}}%
\pgfpathlineto{\pgfqpoint{1.165943in}{1.104553in}}%
\pgfpathlineto{\pgfqpoint{1.166338in}{1.104638in}}%
\pgfpathlineto{\pgfqpoint{1.167128in}{1.087897in}}%
\pgfpathlineto{\pgfqpoint{1.167917in}{1.094004in}}%
\pgfpathlineto{\pgfqpoint{1.169891in}{1.136073in}}%
\pgfpathlineto{\pgfqpoint{1.170680in}{1.132555in}}%
\pgfpathlineto{\pgfqpoint{1.171470in}{1.125741in}}%
\pgfpathlineto{\pgfqpoint{1.171865in}{1.129555in}}%
\pgfpathlineto{\pgfqpoint{1.173838in}{1.145698in}}%
\pgfpathlineto{\pgfqpoint{1.174233in}{1.145664in}}%
\pgfpathlineto{\pgfqpoint{1.174628in}{1.145392in}}%
\pgfpathlineto{\pgfqpoint{1.175023in}{1.146486in}}%
\pgfpathlineto{\pgfqpoint{1.175812in}{1.151827in}}%
\pgfpathlineto{\pgfqpoint{1.176602in}{1.147299in}}%
\pgfpathlineto{\pgfqpoint{1.177391in}{1.150489in}}%
\pgfpathlineto{\pgfqpoint{1.177786in}{1.147244in}}%
\pgfpathlineto{\pgfqpoint{1.178181in}{1.146459in}}%
\pgfpathlineto{\pgfqpoint{1.178575in}{1.149090in}}%
\pgfpathlineto{\pgfqpoint{1.178970in}{1.151324in}}%
\pgfpathlineto{\pgfqpoint{1.179365in}{1.147113in}}%
\pgfpathlineto{\pgfqpoint{1.180944in}{1.125653in}}%
\pgfpathlineto{\pgfqpoint{1.181733in}{1.129794in}}%
\pgfpathlineto{\pgfqpoint{1.185286in}{1.156237in}}%
\pgfpathlineto{\pgfqpoint{1.185681in}{1.152267in}}%
\pgfpathlineto{\pgfqpoint{1.188839in}{1.105043in}}%
\pgfpathlineto{\pgfqpoint{1.196734in}{1.028315in}}%
\pgfpathlineto{\pgfqpoint{1.197129in}{1.027545in}}%
\pgfpathlineto{\pgfqpoint{1.200287in}{1.058729in}}%
\pgfpathlineto{\pgfqpoint{1.200681in}{1.054587in}}%
\pgfpathlineto{\pgfqpoint{1.201866in}{1.043069in}}%
\pgfpathlineto{\pgfqpoint{1.202260in}{1.048129in}}%
\pgfpathlineto{\pgfqpoint{1.205813in}{1.092157in}}%
\pgfpathlineto{\pgfqpoint{1.206603in}{1.079038in}}%
\pgfpathlineto{\pgfqpoint{1.206997in}{1.072259in}}%
\pgfpathlineto{\pgfqpoint{1.207787in}{1.079807in}}%
\pgfpathlineto{\pgfqpoint{1.211734in}{1.114264in}}%
\pgfpathlineto{\pgfqpoint{1.212524in}{1.122656in}}%
\pgfpathlineto{\pgfqpoint{1.213313in}{1.118007in}}%
\pgfpathlineto{\pgfqpoint{1.215287in}{1.092898in}}%
\pgfpathlineto{\pgfqpoint{1.216471in}{1.095429in}}%
\pgfpathlineto{\pgfqpoint{1.218050in}{1.096442in}}%
\pgfpathlineto{\pgfqpoint{1.222392in}{1.044503in}}%
\pgfpathlineto{\pgfqpoint{1.222787in}{1.047741in}}%
\pgfpathlineto{\pgfqpoint{1.225156in}{1.074304in}}%
\pgfpathlineto{\pgfqpoint{1.225945in}{1.070829in}}%
\pgfpathlineto{\pgfqpoint{1.227129in}{1.049804in}}%
\pgfpathlineto{\pgfqpoint{1.227919in}{1.061892in}}%
\pgfpathlineto{\pgfqpoint{1.229893in}{1.075685in}}%
\pgfpathlineto{\pgfqpoint{1.230287in}{1.076118in}}%
\pgfpathlineto{\pgfqpoint{1.232261in}{1.087664in}}%
\pgfpathlineto{\pgfqpoint{1.233840in}{1.087621in}}%
\pgfpathlineto{\pgfqpoint{1.234235in}{1.086825in}}%
\pgfpathlineto{\pgfqpoint{1.235814in}{1.105862in}}%
\pgfpathlineto{\pgfqpoint{1.237393in}{1.105247in}}%
\pgfpathlineto{\pgfqpoint{1.237788in}{1.106143in}}%
\pgfpathlineto{\pgfqpoint{1.238972in}{1.094027in}}%
\pgfpathlineto{\pgfqpoint{1.239761in}{1.098218in}}%
\pgfpathlineto{\pgfqpoint{1.241340in}{1.113686in}}%
\pgfpathlineto{\pgfqpoint{1.241735in}{1.119002in}}%
\pgfpathlineto{\pgfqpoint{1.242919in}{1.115778in}}%
\pgfpathlineto{\pgfqpoint{1.252393in}{0.979821in}}%
\pgfpathlineto{\pgfqpoint{1.252788in}{0.980900in}}%
\pgfpathlineto{\pgfqpoint{1.253972in}{0.949402in}}%
\pgfpathlineto{\pgfqpoint{1.254367in}{0.941386in}}%
\pgfpathlineto{\pgfqpoint{1.255157in}{0.955663in}}%
\pgfpathlineto{\pgfqpoint{1.257130in}{1.001925in}}%
\pgfpathlineto{\pgfqpoint{1.257920in}{0.994411in}}%
\pgfpathlineto{\pgfqpoint{1.258709in}{0.979862in}}%
\pgfpathlineto{\pgfqpoint{1.259499in}{0.992381in}}%
\pgfpathlineto{\pgfqpoint{1.262262in}{1.056010in}}%
\pgfpathlineto{\pgfqpoint{1.268183in}{1.105434in}}%
\pgfpathlineto{\pgfqpoint{1.270157in}{1.106116in}}%
\pgfpathlineto{\pgfqpoint{1.270552in}{1.105243in}}%
\pgfpathlineto{\pgfqpoint{1.270947in}{1.103719in}}%
\pgfpathlineto{\pgfqpoint{1.272920in}{1.134681in}}%
\pgfpathlineto{\pgfqpoint{1.274105in}{1.133668in}}%
\pgfpathlineto{\pgfqpoint{1.275289in}{1.141509in}}%
\pgfpathlineto{\pgfqpoint{1.275684in}{1.140333in}}%
\pgfpathlineto{\pgfqpoint{1.280026in}{1.082471in}}%
\pgfpathlineto{\pgfqpoint{1.281605in}{1.051714in}}%
\pgfpathlineto{\pgfqpoint{1.282000in}{1.057983in}}%
\pgfpathlineto{\pgfqpoint{1.282789in}{1.069277in}}%
\pgfpathlineto{\pgfqpoint{1.283184in}{1.061431in}}%
\pgfpathlineto{\pgfqpoint{1.286737in}{0.996586in}}%
\pgfpathlineto{\pgfqpoint{1.287131in}{0.995836in}}%
\pgfpathlineto{\pgfqpoint{1.289105in}{1.043552in}}%
\pgfpathlineto{\pgfqpoint{1.289895in}{1.037137in}}%
\pgfpathlineto{\pgfqpoint{1.292658in}{1.016931in}}%
\pgfpathlineto{\pgfqpoint{1.295816in}{1.076336in}}%
\pgfpathlineto{\pgfqpoint{1.297000in}{1.070163in}}%
\pgfpathlineto{\pgfqpoint{1.297789in}{1.062114in}}%
\pgfpathlineto{\pgfqpoint{1.298579in}{1.064938in}}%
\pgfpathlineto{\pgfqpoint{1.300158in}{1.069240in}}%
\pgfpathlineto{\pgfqpoint{1.300553in}{1.068090in}}%
\pgfpathlineto{\pgfqpoint{1.302132in}{1.059698in}}%
\pgfpathlineto{\pgfqpoint{1.302526in}{1.059948in}}%
\pgfpathlineto{\pgfqpoint{1.304105in}{1.078382in}}%
\pgfpathlineto{\pgfqpoint{1.304895in}{1.072856in}}%
\pgfpathlineto{\pgfqpoint{1.311211in}{0.998589in}}%
\pgfpathlineto{\pgfqpoint{1.312000in}{0.990251in}}%
\pgfpathlineto{\pgfqpoint{1.313974in}{0.940163in}}%
\pgfpathlineto{\pgfqpoint{1.315158in}{0.944748in}}%
\pgfpathlineto{\pgfqpoint{1.317132in}{0.938321in}}%
\pgfpathlineto{\pgfqpoint{1.319106in}{0.982540in}}%
\pgfpathlineto{\pgfqpoint{1.319895in}{0.969644in}}%
\pgfpathlineto{\pgfqpoint{1.320290in}{0.965504in}}%
\pgfpathlineto{\pgfqpoint{1.320685in}{0.969112in}}%
\pgfpathlineto{\pgfqpoint{1.324632in}{1.011813in}}%
\pgfpathlineto{\pgfqpoint{1.325422in}{1.022021in}}%
\pgfpathlineto{\pgfqpoint{1.325817in}{1.015053in}}%
\pgfpathlineto{\pgfqpoint{1.326211in}{1.011367in}}%
\pgfpathlineto{\pgfqpoint{1.327001in}{1.012021in}}%
\pgfpathlineto{\pgfqpoint{1.330159in}{1.060233in}}%
\pgfpathlineto{\pgfqpoint{1.330554in}{1.054072in}}%
\pgfpathlineto{\pgfqpoint{1.332133in}{1.023471in}}%
\pgfpathlineto{\pgfqpoint{1.332922in}{1.036112in}}%
\pgfpathlineto{\pgfqpoint{1.336475in}{1.089150in}}%
\pgfpathlineto{\pgfqpoint{1.337264in}{1.094670in}}%
\pgfpathlineto{\pgfqpoint{1.339238in}{1.108146in}}%
\pgfpathlineto{\pgfqpoint{1.339633in}{1.108353in}}%
\pgfpathlineto{\pgfqpoint{1.343975in}{1.050869in}}%
\pgfpathlineto{\pgfqpoint{1.345159in}{1.038177in}}%
\pgfpathlineto{\pgfqpoint{1.345949in}{1.038383in}}%
\pgfpathlineto{\pgfqpoint{1.346344in}{1.038680in}}%
\pgfpathlineto{\pgfqpoint{1.346738in}{1.038002in}}%
\pgfpathlineto{\pgfqpoint{1.347923in}{1.028747in}}%
\pgfpathlineto{\pgfqpoint{1.349107in}{1.006516in}}%
\pgfpathlineto{\pgfqpoint{1.353449in}{0.917108in}}%
\pgfpathlineto{\pgfqpoint{1.353844in}{0.921927in}}%
\pgfpathlineto{\pgfqpoint{1.355028in}{0.947048in}}%
\pgfpathlineto{\pgfqpoint{1.356212in}{0.945680in}}%
\pgfpathlineto{\pgfqpoint{1.359765in}{0.989618in}}%
\pgfpathlineto{\pgfqpoint{1.360555in}{1.000705in}}%
\pgfpathlineto{\pgfqpoint{1.361739in}{0.998676in}}%
\pgfpathlineto{\pgfqpoint{1.362134in}{0.997456in}}%
\pgfpathlineto{\pgfqpoint{1.362923in}{1.006678in}}%
\pgfpathlineto{\pgfqpoint{1.363713in}{0.998813in}}%
\pgfpathlineto{\pgfqpoint{1.364107in}{0.994790in}}%
\pgfpathlineto{\pgfqpoint{1.364502in}{0.999420in}}%
\pgfpathlineto{\pgfqpoint{1.365686in}{1.010419in}}%
\pgfpathlineto{\pgfqpoint{1.366476in}{1.006368in}}%
\pgfpathlineto{\pgfqpoint{1.366871in}{1.004675in}}%
\pgfpathlineto{\pgfqpoint{1.367265in}{1.008455in}}%
\pgfpathlineto{\pgfqpoint{1.368450in}{1.020037in}}%
\pgfpathlineto{\pgfqpoint{1.368844in}{1.015047in}}%
\pgfpathlineto{\pgfqpoint{1.373187in}{0.952779in}}%
\pgfpathlineto{\pgfqpoint{1.373581in}{0.953431in}}%
\pgfpathlineto{\pgfqpoint{1.373976in}{0.955289in}}%
\pgfpathlineto{\pgfqpoint{1.374371in}{0.953339in}}%
\pgfpathlineto{\pgfqpoint{1.374766in}{0.949479in}}%
\pgfpathlineto{\pgfqpoint{1.375555in}{0.955976in}}%
\pgfpathlineto{\pgfqpoint{1.376345in}{0.957622in}}%
\pgfpathlineto{\pgfqpoint{1.378318in}{0.975351in}}%
\pgfpathlineto{\pgfqpoint{1.378713in}{0.970558in}}%
\pgfpathlineto{\pgfqpoint{1.379108in}{0.965191in}}%
\pgfpathlineto{\pgfqpoint{1.379897in}{0.973633in}}%
\pgfpathlineto{\pgfqpoint{1.380292in}{0.978889in}}%
\pgfpathlineto{\pgfqpoint{1.381081in}{0.970120in}}%
\pgfpathlineto{\pgfqpoint{1.381476in}{0.969739in}}%
\pgfpathlineto{\pgfqpoint{1.381871in}{0.970633in}}%
\pgfpathlineto{\pgfqpoint{1.382660in}{0.962582in}}%
\pgfpathlineto{\pgfqpoint{1.383450in}{0.967401in}}%
\pgfpathlineto{\pgfqpoint{1.386608in}{0.987612in}}%
\pgfpathlineto{\pgfqpoint{1.388582in}{0.969691in}}%
\pgfpathlineto{\pgfqpoint{1.388976in}{0.972289in}}%
\pgfpathlineto{\pgfqpoint{1.389371in}{0.976281in}}%
\pgfpathlineto{\pgfqpoint{1.389766in}{0.971351in}}%
\pgfpathlineto{\pgfqpoint{1.393319in}{0.929588in}}%
\pgfpathlineto{\pgfqpoint{1.394503in}{0.933140in}}%
\pgfpathlineto{\pgfqpoint{1.396477in}{0.910078in}}%
\pgfpathlineto{\pgfqpoint{1.397661in}{0.914135in}}%
\pgfpathlineto{\pgfqpoint{1.399635in}{0.943931in}}%
\pgfpathlineto{\pgfqpoint{1.402398in}{0.963925in}}%
\pgfpathlineto{\pgfqpoint{1.402793in}{0.963335in}}%
\pgfpathlineto{\pgfqpoint{1.403582in}{0.964854in}}%
\pgfpathlineto{\pgfqpoint{1.403977in}{0.965237in}}%
\pgfpathlineto{\pgfqpoint{1.405161in}{0.955207in}}%
\pgfpathlineto{\pgfqpoint{1.405556in}{0.955287in}}%
\pgfpathlineto{\pgfqpoint{1.406345in}{0.968380in}}%
\pgfpathlineto{\pgfqpoint{1.407135in}{0.966181in}}%
\pgfpathlineto{\pgfqpoint{1.408319in}{0.961045in}}%
\pgfpathlineto{\pgfqpoint{1.409503in}{0.933704in}}%
\pgfpathlineto{\pgfqpoint{1.410293in}{0.942374in}}%
\pgfpathlineto{\pgfqpoint{1.410688in}{0.945908in}}%
\pgfpathlineto{\pgfqpoint{1.411477in}{0.939496in}}%
\pgfpathlineto{\pgfqpoint{1.412267in}{0.933139in}}%
\pgfpathlineto{\pgfqpoint{1.412661in}{0.936965in}}%
\pgfpathlineto{\pgfqpoint{1.413846in}{0.953159in}}%
\pgfpathlineto{\pgfqpoint{1.414240in}{0.946774in}}%
\pgfpathlineto{\pgfqpoint{1.415425in}{0.927960in}}%
\pgfpathlineto{\pgfqpoint{1.415819in}{0.932733in}}%
\pgfpathlineto{\pgfqpoint{1.417004in}{0.952956in}}%
\pgfpathlineto{\pgfqpoint{1.417793in}{0.947248in}}%
\pgfpathlineto{\pgfqpoint{1.418583in}{0.947270in}}%
\pgfpathlineto{\pgfqpoint{1.420162in}{0.969741in}}%
\pgfpathlineto{\pgfqpoint{1.420556in}{0.961616in}}%
\pgfpathlineto{\pgfqpoint{1.421741in}{0.937005in}}%
\pgfpathlineto{\pgfqpoint{1.422925in}{0.943629in}}%
\pgfpathlineto{\pgfqpoint{1.426083in}{0.956649in}}%
\pgfpathlineto{\pgfqpoint{1.426478in}{0.954069in}}%
\pgfpathlineto{\pgfqpoint{1.427662in}{0.937102in}}%
\pgfpathlineto{\pgfqpoint{1.428451in}{0.939749in}}%
\pgfpathlineto{\pgfqpoint{1.430030in}{0.953380in}}%
\pgfpathlineto{\pgfqpoint{1.430425in}{0.949005in}}%
\pgfpathlineto{\pgfqpoint{1.432004in}{0.943997in}}%
\pgfpathlineto{\pgfqpoint{1.432399in}{0.945079in}}%
\pgfpathlineto{\pgfqpoint{1.432794in}{0.942954in}}%
\pgfpathlineto{\pgfqpoint{1.434373in}{0.929434in}}%
\pgfpathlineto{\pgfqpoint{1.434767in}{0.930985in}}%
\pgfpathlineto{\pgfqpoint{1.435557in}{0.932777in}}%
\pgfpathlineto{\pgfqpoint{1.435952in}{0.932220in}}%
\pgfpathlineto{\pgfqpoint{1.436346in}{0.928871in}}%
\pgfpathlineto{\pgfqpoint{1.436741in}{0.932801in}}%
\pgfpathlineto{\pgfqpoint{1.437925in}{0.963224in}}%
\pgfpathlineto{\pgfqpoint{1.439110in}{0.960081in}}%
\pgfpathlineto{\pgfqpoint{1.440294in}{0.954685in}}%
\pgfpathlineto{\pgfqpoint{1.441083in}{0.958345in}}%
\pgfpathlineto{\pgfqpoint{1.443452in}{0.975061in}}%
\pgfpathlineto{\pgfqpoint{1.443847in}{0.972929in}}%
\pgfpathlineto{\pgfqpoint{1.452926in}{0.897088in}}%
\pgfpathlineto{\pgfqpoint{1.454110in}{0.905758in}}%
\pgfpathlineto{\pgfqpoint{1.456479in}{0.934668in}}%
\pgfpathlineto{\pgfqpoint{1.457268in}{0.924653in}}%
\pgfpathlineto{\pgfqpoint{1.458452in}{0.915254in}}%
\pgfpathlineto{\pgfqpoint{1.458847in}{0.917461in}}%
\pgfpathlineto{\pgfqpoint{1.460031in}{0.927006in}}%
\pgfpathlineto{\pgfqpoint{1.460426in}{0.925199in}}%
\pgfpathlineto{\pgfqpoint{1.461610in}{0.904035in}}%
\pgfpathlineto{\pgfqpoint{1.462400in}{0.912849in}}%
\pgfpathlineto{\pgfqpoint{1.464373in}{0.932084in}}%
\pgfpathlineto{\pgfqpoint{1.465558in}{0.928721in}}%
\pgfpathlineto{\pgfqpoint{1.466347in}{0.924927in}}%
\pgfpathlineto{\pgfqpoint{1.468321in}{0.879292in}}%
\pgfpathlineto{\pgfqpoint{1.469110in}{0.891093in}}%
\pgfpathlineto{\pgfqpoint{1.471084in}{0.908377in}}%
\pgfpathlineto{\pgfqpoint{1.471479in}{0.910717in}}%
\pgfpathlineto{\pgfqpoint{1.471874in}{0.910375in}}%
\pgfpathlineto{\pgfqpoint{1.472663in}{0.903527in}}%
\pgfpathlineto{\pgfqpoint{1.473058in}{0.911688in}}%
\pgfpathlineto{\pgfqpoint{1.474637in}{0.924184in}}%
\pgfpathlineto{\pgfqpoint{1.475032in}{0.923111in}}%
\pgfpathlineto{\pgfqpoint{1.477005in}{0.912198in}}%
\pgfpathlineto{\pgfqpoint{1.478979in}{0.907921in}}%
\pgfpathlineto{\pgfqpoint{1.480163in}{0.913895in}}%
\pgfpathlineto{\pgfqpoint{1.481348in}{0.900701in}}%
\pgfpathlineto{\pgfqpoint{1.481742in}{0.906813in}}%
\pgfpathlineto{\pgfqpoint{1.483716in}{0.912623in}}%
\pgfpathlineto{\pgfqpoint{1.484506in}{0.906744in}}%
\pgfpathlineto{\pgfqpoint{1.485690in}{0.911147in}}%
\pgfpathlineto{\pgfqpoint{1.488848in}{0.868281in}}%
\pgfpathlineto{\pgfqpoint{1.490822in}{0.877405in}}%
\pgfpathlineto{\pgfqpoint{1.492795in}{0.870285in}}%
\pgfpathlineto{\pgfqpoint{1.493190in}{0.871043in}}%
\pgfpathlineto{\pgfqpoint{1.495559in}{0.902805in}}%
\pgfpathlineto{\pgfqpoint{1.497138in}{0.901193in}}%
\pgfpathlineto{\pgfqpoint{1.499506in}{0.888246in}}%
\pgfpathlineto{\pgfqpoint{1.499901in}{0.889083in}}%
\pgfpathlineto{\pgfqpoint{1.502664in}{0.908575in}}%
\pgfpathlineto{\pgfqpoint{1.501085in}{0.886709in}}%
\pgfpathlineto{\pgfqpoint{1.503454in}{0.902824in}}%
\pgfpathlineto{\pgfqpoint{1.505033in}{0.889734in}}%
\pgfpathlineto{\pgfqpoint{1.505427in}{0.891029in}}%
\pgfpathlineto{\pgfqpoint{1.507796in}{0.913797in}}%
\pgfpathlineto{\pgfqpoint{1.510164in}{0.933005in}}%
\pgfpathlineto{\pgfqpoint{1.510559in}{0.934613in}}%
\pgfpathlineto{\pgfqpoint{1.510954in}{0.930952in}}%
\pgfpathlineto{\pgfqpoint{1.512138in}{0.919435in}}%
\pgfpathlineto{\pgfqpoint{1.512928in}{0.924751in}}%
\pgfpathlineto{\pgfqpoint{1.513322in}{0.931716in}}%
\pgfpathlineto{\pgfqpoint{1.514112in}{0.924376in}}%
\pgfpathlineto{\pgfqpoint{1.517665in}{0.895950in}}%
\pgfpathlineto{\pgfqpoint{1.518059in}{0.897459in}}%
\pgfpathlineto{\pgfqpoint{1.518454in}{0.898241in}}%
\pgfpathlineto{\pgfqpoint{1.521217in}{0.872555in}}%
\pgfpathlineto{\pgfqpoint{1.521612in}{0.868344in}}%
\pgfpathlineto{\pgfqpoint{1.522402in}{0.874795in}}%
\pgfpathlineto{\pgfqpoint{1.524375in}{0.903453in}}%
\pgfpathlineto{\pgfqpoint{1.525560in}{0.902524in}}%
\pgfpathlineto{\pgfqpoint{1.529507in}{0.858615in}}%
\pgfpathlineto{\pgfqpoint{1.530297in}{0.853394in}}%
\pgfpathlineto{\pgfqpoint{1.531086in}{0.853798in}}%
\pgfpathlineto{\pgfqpoint{1.532270in}{0.875558in}}%
\pgfpathlineto{\pgfqpoint{1.533849in}{0.908888in}}%
\pgfpathlineto{\pgfqpoint{1.534639in}{0.900804in}}%
\pgfpathlineto{\pgfqpoint{1.536218in}{0.888380in}}%
\pgfpathlineto{\pgfqpoint{1.537007in}{0.896406in}}%
\pgfpathlineto{\pgfqpoint{1.538192in}{0.895920in}}%
\pgfpathlineto{\pgfqpoint{1.538981in}{0.897178in}}%
\pgfpathlineto{\pgfqpoint{1.539771in}{0.896481in}}%
\pgfpathlineto{\pgfqpoint{1.542534in}{0.884287in}}%
\pgfpathlineto{\pgfqpoint{1.545297in}{0.832488in}}%
\pgfpathlineto{\pgfqpoint{1.545692in}{0.833745in}}%
\pgfpathlineto{\pgfqpoint{1.548455in}{0.867035in}}%
\pgfpathlineto{\pgfqpoint{1.549244in}{0.866840in}}%
\pgfpathlineto{\pgfqpoint{1.550034in}{0.861181in}}%
\pgfpathlineto{\pgfqpoint{1.550823in}{0.878248in}}%
\pgfpathlineto{\pgfqpoint{1.551613in}{0.872428in}}%
\pgfpathlineto{\pgfqpoint{1.552008in}{0.867475in}}%
\pgfpathlineto{\pgfqpoint{1.552402in}{0.872709in}}%
\pgfpathlineto{\pgfqpoint{1.553192in}{0.885352in}}%
\pgfpathlineto{\pgfqpoint{1.553981in}{0.879900in}}%
\pgfpathlineto{\pgfqpoint{1.554376in}{0.879926in}}%
\pgfpathlineto{\pgfqpoint{1.555955in}{0.869051in}}%
\pgfpathlineto{\pgfqpoint{1.556350in}{0.863377in}}%
\pgfpathlineto{\pgfqpoint{1.557139in}{0.867337in}}%
\pgfpathlineto{\pgfqpoint{1.559508in}{0.879677in}}%
\pgfpathlineto{\pgfqpoint{1.560297in}{0.880339in}}%
\pgfpathlineto{\pgfqpoint{1.562666in}{0.858901in}}%
\pgfpathlineto{\pgfqpoint{1.563455in}{0.867765in}}%
\pgfpathlineto{\pgfqpoint{1.564245in}{0.877177in}}%
\pgfpathlineto{\pgfqpoint{1.564640in}{0.874031in}}%
\pgfpathlineto{\pgfqpoint{1.565824in}{0.860828in}}%
\pgfpathlineto{\pgfqpoint{1.566613in}{0.862941in}}%
\pgfpathlineto{\pgfqpoint{1.567403in}{0.860045in}}%
\pgfpathlineto{\pgfqpoint{1.567798in}{0.864281in}}%
\pgfpathlineto{\pgfqpoint{1.569377in}{0.884274in}}%
\pgfpathlineto{\pgfqpoint{1.570561in}{0.879256in}}%
\pgfpathlineto{\pgfqpoint{1.570956in}{0.875686in}}%
\pgfpathlineto{\pgfqpoint{1.571745in}{0.878693in}}%
\pgfpathlineto{\pgfqpoint{1.574114in}{0.909005in}}%
\pgfpathlineto{\pgfqpoint{1.574903in}{0.899985in}}%
\pgfpathlineto{\pgfqpoint{1.577666in}{0.872786in}}%
\pgfpathlineto{\pgfqpoint{1.578061in}{0.876941in}}%
\pgfpathlineto{\pgfqpoint{1.578456in}{0.878216in}}%
\pgfpathlineto{\pgfqpoint{1.578851in}{0.875131in}}%
\pgfpathlineto{\pgfqpoint{1.585956in}{0.818939in}}%
\pgfpathlineto{\pgfqpoint{1.586351in}{0.819567in}}%
\pgfpathlineto{\pgfqpoint{1.587535in}{0.823401in}}%
\pgfpathlineto{\pgfqpoint{1.590693in}{0.881728in}}%
\pgfpathlineto{\pgfqpoint{1.591877in}{0.907423in}}%
\pgfpathlineto{\pgfqpoint{1.593062in}{0.904824in}}%
\pgfpathlineto{\pgfqpoint{1.593456in}{0.903060in}}%
\pgfpathlineto{\pgfqpoint{1.593851in}{0.905439in}}%
\pgfpathlineto{\pgfqpoint{1.595035in}{0.916062in}}%
\pgfpathlineto{\pgfqpoint{1.595825in}{0.911954in}}%
\pgfpathlineto{\pgfqpoint{1.597404in}{0.908344in}}%
\pgfpathlineto{\pgfqpoint{1.597799in}{0.908852in}}%
\pgfpathlineto{\pgfqpoint{1.600562in}{0.888701in}}%
\pgfpathlineto{\pgfqpoint{1.603720in}{0.852480in}}%
\pgfpathlineto{\pgfqpoint{1.605694in}{0.835853in}}%
\pgfpathlineto{\pgfqpoint{1.607273in}{0.815965in}}%
\pgfpathlineto{\pgfqpoint{1.607667in}{0.818770in}}%
\pgfpathlineto{\pgfqpoint{1.608062in}{0.818827in}}%
\pgfpathlineto{\pgfqpoint{1.608457in}{0.815154in}}%
\pgfpathlineto{\pgfqpoint{1.609246in}{0.818037in}}%
\pgfpathlineto{\pgfqpoint{1.611220in}{0.831720in}}%
\pgfpathlineto{\pgfqpoint{1.614773in}{0.795240in}}%
\pgfpathlineto{\pgfqpoint{1.616352in}{0.785573in}}%
\pgfpathlineto{\pgfqpoint{1.616747in}{0.785809in}}%
\pgfpathlineto{\pgfqpoint{1.617141in}{0.791085in}}%
\pgfpathlineto{\pgfqpoint{1.618326in}{0.787430in}}%
\pgfpathlineto{\pgfqpoint{1.618720in}{0.786756in}}%
\pgfpathlineto{\pgfqpoint{1.623852in}{0.839838in}}%
\pgfpathlineto{\pgfqpoint{1.624247in}{0.838254in}}%
\pgfpathlineto{\pgfqpoint{1.624642in}{0.840097in}}%
\pgfpathlineto{\pgfqpoint{1.627010in}{0.870724in}}%
\pgfpathlineto{\pgfqpoint{1.628194in}{0.856984in}}%
\pgfpathlineto{\pgfqpoint{1.633721in}{0.831509in}}%
\pgfpathlineto{\pgfqpoint{1.634115in}{0.831912in}}%
\pgfpathlineto{\pgfqpoint{1.634905in}{0.834503in}}%
\pgfpathlineto{\pgfqpoint{1.635300in}{0.834037in}}%
\pgfpathlineto{\pgfqpoint{1.636484in}{0.818424in}}%
\pgfpathlineto{\pgfqpoint{1.637273in}{0.824343in}}%
\pgfpathlineto{\pgfqpoint{1.638063in}{0.832782in}}%
\pgfpathlineto{\pgfqpoint{1.638852in}{0.826728in}}%
\pgfpathlineto{\pgfqpoint{1.641616in}{0.806724in}}%
\pgfpathlineto{\pgfqpoint{1.642010in}{0.812647in}}%
\pgfpathlineto{\pgfqpoint{1.642405in}{0.813071in}}%
\pgfpathlineto{\pgfqpoint{1.642800in}{0.810350in}}%
\pgfpathlineto{\pgfqpoint{1.643195in}{0.811862in}}%
\pgfpathlineto{\pgfqpoint{1.646747in}{0.862098in}}%
\pgfpathlineto{\pgfqpoint{1.647142in}{0.861347in}}%
\pgfpathlineto{\pgfqpoint{1.648721in}{0.845957in}}%
\pgfpathlineto{\pgfqpoint{1.650300in}{0.852340in}}%
\pgfpathlineto{\pgfqpoint{1.652669in}{0.864742in}}%
\pgfpathlineto{\pgfqpoint{1.655037in}{0.846394in}}%
\pgfpathlineto{\pgfqpoint{1.656616in}{0.819229in}}%
\pgfpathlineto{\pgfqpoint{1.657011in}{0.822741in}}%
\pgfpathlineto{\pgfqpoint{1.658985in}{0.855300in}}%
\pgfpathlineto{\pgfqpoint{1.659379in}{0.850479in}}%
\pgfpathlineto{\pgfqpoint{1.663722in}{0.812237in}}%
\pgfpathlineto{\pgfqpoint{1.664116in}{0.819236in}}%
\pgfpathlineto{\pgfqpoint{1.664906in}{0.807103in}}%
\pgfpathlineto{\pgfqpoint{1.666090in}{0.799092in}}%
\pgfpathlineto{\pgfqpoint{1.666880in}{0.801486in}}%
\pgfpathlineto{\pgfqpoint{1.667669in}{0.812740in}}%
\pgfpathlineto{\pgfqpoint{1.668459in}{0.808143in}}%
\pgfpathlineto{\pgfqpoint{1.671617in}{0.777827in}}%
\pgfpathlineto{\pgfqpoint{1.672801in}{0.782520in}}%
\pgfpathlineto{\pgfqpoint{1.674775in}{0.800258in}}%
\pgfpathlineto{\pgfqpoint{1.675169in}{0.799700in}}%
\pgfpathlineto{\pgfqpoint{1.675564in}{0.792382in}}%
\pgfpathlineto{\pgfqpoint{1.676354in}{0.798966in}}%
\pgfpathlineto{\pgfqpoint{1.677933in}{0.817744in}}%
\pgfpathlineto{\pgfqpoint{1.678722in}{0.814764in}}%
\pgfpathlineto{\pgfqpoint{1.679906in}{0.805907in}}%
\pgfpathlineto{\pgfqpoint{1.681485in}{0.772324in}}%
\pgfpathlineto{\pgfqpoint{1.683064in}{0.777048in}}%
\pgfpathlineto{\pgfqpoint{1.688196in}{0.841098in}}%
\pgfpathlineto{\pgfqpoint{1.688986in}{0.829534in}}%
\pgfpathlineto{\pgfqpoint{1.689380in}{0.825786in}}%
\pgfpathlineto{\pgfqpoint{1.690170in}{0.832466in}}%
\pgfpathlineto{\pgfqpoint{1.690959in}{0.830456in}}%
\pgfpathlineto{\pgfqpoint{1.694907in}{0.859678in}}%
\pgfpathlineto{\pgfqpoint{1.695696in}{0.857571in}}%
\pgfpathlineto{\pgfqpoint{1.696091in}{0.860548in}}%
\pgfpathlineto{\pgfqpoint{1.696486in}{0.863839in}}%
\pgfpathlineto{\pgfqpoint{1.696881in}{0.859688in}}%
\pgfpathlineto{\pgfqpoint{1.701223in}{0.808815in}}%
\pgfpathlineto{\pgfqpoint{1.701618in}{0.811384in}}%
\pgfpathlineto{\pgfqpoint{1.702012in}{0.807205in}}%
\pgfpathlineto{\pgfqpoint{1.705170in}{0.780600in}}%
\pgfpathlineto{\pgfqpoint{1.705565in}{0.781851in}}%
\pgfpathlineto{\pgfqpoint{1.705960in}{0.782982in}}%
\pgfpathlineto{\pgfqpoint{1.706749in}{0.773243in}}%
\pgfpathlineto{\pgfqpoint{1.707144in}{0.775192in}}%
\pgfpathlineto{\pgfqpoint{1.711091in}{0.830096in}}%
\pgfpathlineto{\pgfqpoint{1.711881in}{0.823942in}}%
\pgfpathlineto{\pgfqpoint{1.712276in}{0.826467in}}%
\pgfpathlineto{\pgfqpoint{1.713065in}{0.833120in}}%
\pgfpathlineto{\pgfqpoint{1.713460in}{0.827666in}}%
\pgfpathlineto{\pgfqpoint{1.715039in}{0.803526in}}%
\pgfpathlineto{\pgfqpoint{1.716223in}{0.807614in}}%
\pgfpathlineto{\pgfqpoint{1.717013in}{0.802689in}}%
\pgfpathlineto{\pgfqpoint{1.717407in}{0.804872in}}%
\pgfpathlineto{\pgfqpoint{1.717802in}{0.809388in}}%
\pgfpathlineto{\pgfqpoint{1.718986in}{0.809087in}}%
\pgfpathlineto{\pgfqpoint{1.720171in}{0.806877in}}%
\pgfpathlineto{\pgfqpoint{1.720960in}{0.807283in}}%
\pgfpathlineto{\pgfqpoint{1.722144in}{0.815515in}}%
\pgfpathlineto{\pgfqpoint{1.723329in}{0.840576in}}%
\pgfpathlineto{\pgfqpoint{1.724118in}{0.834049in}}%
\pgfpathlineto{\pgfqpoint{1.727671in}{0.810313in}}%
\pgfpathlineto{\pgfqpoint{1.729250in}{0.804730in}}%
\pgfpathlineto{\pgfqpoint{1.733197in}{0.760791in}}%
\pgfpathlineto{\pgfqpoint{1.734382in}{0.766910in}}%
\pgfpathlineto{\pgfqpoint{1.735171in}{0.767140in}}%
\pgfpathlineto{\pgfqpoint{1.735961in}{0.765715in}}%
\pgfpathlineto{\pgfqpoint{1.737540in}{0.782384in}}%
\pgfpathlineto{\pgfqpoint{1.737934in}{0.780088in}}%
\pgfpathlineto{\pgfqpoint{1.740698in}{0.760396in}}%
\pgfpathlineto{\pgfqpoint{1.746224in}{0.814100in}}%
\pgfpathlineto{\pgfqpoint{1.746619in}{0.815681in}}%
\pgfpathlineto{\pgfqpoint{1.748198in}{0.838194in}}%
\pgfpathlineto{\pgfqpoint{1.748593in}{0.834166in}}%
\pgfpathlineto{\pgfqpoint{1.754119in}{0.781663in}}%
\pgfpathlineto{\pgfqpoint{1.755698in}{0.787794in}}%
\pgfpathlineto{\pgfqpoint{1.756488in}{0.787634in}}%
\pgfpathlineto{\pgfqpoint{1.758856in}{0.744751in}}%
\pgfpathlineto{\pgfqpoint{1.760040in}{0.762360in}}%
\pgfpathlineto{\pgfqpoint{1.760830in}{0.763630in}}%
\pgfpathlineto{\pgfqpoint{1.765567in}{0.781512in}}%
\pgfpathlineto{\pgfqpoint{1.767541in}{0.771202in}}%
\pgfpathlineto{\pgfqpoint{1.767935in}{0.770486in}}%
\pgfpathlineto{\pgfqpoint{1.769120in}{0.790485in}}%
\pgfpathlineto{\pgfqpoint{1.771488in}{0.833096in}}%
\pgfpathlineto{\pgfqpoint{1.771883in}{0.828734in}}%
\pgfpathlineto{\pgfqpoint{1.772672in}{0.820462in}}%
\pgfpathlineto{\pgfqpoint{1.773462in}{0.827642in}}%
\pgfpathlineto{\pgfqpoint{1.773857in}{0.826445in}}%
\pgfpathlineto{\pgfqpoint{1.774251in}{0.830367in}}%
\pgfpathlineto{\pgfqpoint{1.774646in}{0.823491in}}%
\pgfpathlineto{\pgfqpoint{1.775436in}{0.813497in}}%
\pgfpathlineto{\pgfqpoint{1.776225in}{0.814139in}}%
\pgfpathlineto{\pgfqpoint{1.777409in}{0.812399in}}%
\pgfpathlineto{\pgfqpoint{1.779383in}{0.780918in}}%
\pgfpathlineto{\pgfqpoint{1.780173in}{0.784374in}}%
\pgfpathlineto{\pgfqpoint{1.780962in}{0.789100in}}%
\pgfpathlineto{\pgfqpoint{1.781357in}{0.786099in}}%
\pgfpathlineto{\pgfqpoint{1.782936in}{0.778647in}}%
\pgfpathlineto{\pgfqpoint{1.783331in}{0.783107in}}%
\pgfpathlineto{\pgfqpoint{1.784910in}{0.799592in}}%
\pgfpathlineto{\pgfqpoint{1.785699in}{0.792318in}}%
\pgfpathlineto{\pgfqpoint{1.786489in}{0.786030in}}%
\pgfpathlineto{\pgfqpoint{1.786883in}{0.789918in}}%
\pgfpathlineto{\pgfqpoint{1.789252in}{0.805520in}}%
\pgfpathlineto{\pgfqpoint{1.789647in}{0.805209in}}%
\pgfpathlineto{\pgfqpoint{1.790831in}{0.800120in}}%
\pgfpathlineto{\pgfqpoint{1.792015in}{0.779773in}}%
\pgfpathlineto{\pgfqpoint{1.793199in}{0.760999in}}%
\pgfpathlineto{\pgfqpoint{1.793989in}{0.767197in}}%
\pgfpathlineto{\pgfqpoint{1.794383in}{0.765002in}}%
\pgfpathlineto{\pgfqpoint{1.794778in}{0.769038in}}%
\pgfpathlineto{\pgfqpoint{1.796752in}{0.778192in}}%
\pgfpathlineto{\pgfqpoint{1.797147in}{0.776735in}}%
\pgfpathlineto{\pgfqpoint{1.800699in}{0.741569in}}%
\pgfpathlineto{\pgfqpoint{1.801094in}{0.737031in}}%
\pgfpathlineto{\pgfqpoint{1.801884in}{0.745036in}}%
\pgfpathlineto{\pgfqpoint{1.803857in}{0.763154in}}%
\pgfpathlineto{\pgfqpoint{1.804252in}{0.762650in}}%
\pgfpathlineto{\pgfqpoint{1.807410in}{0.737800in}}%
\pgfpathlineto{\pgfqpoint{1.807805in}{0.743633in}}%
\pgfpathlineto{\pgfqpoint{1.809779in}{0.754191in}}%
\pgfpathlineto{\pgfqpoint{1.810963in}{0.759272in}}%
\pgfpathlineto{\pgfqpoint{1.811358in}{0.754302in}}%
\pgfpathlineto{\pgfqpoint{1.811752in}{0.750653in}}%
\pgfpathlineto{\pgfqpoint{1.812147in}{0.757545in}}%
\pgfpathlineto{\pgfqpoint{1.814121in}{0.785047in}}%
\pgfpathlineto{\pgfqpoint{1.814516in}{0.779537in}}%
\pgfpathlineto{\pgfqpoint{1.815700in}{0.772013in}}%
\pgfpathlineto{\pgfqpoint{1.816095in}{0.775708in}}%
\pgfpathlineto{\pgfqpoint{1.817674in}{0.800116in}}%
\pgfpathlineto{\pgfqpoint{1.819647in}{0.816021in}}%
\pgfpathlineto{\pgfqpoint{1.820042in}{0.815210in}}%
\pgfpathlineto{\pgfqpoint{1.820832in}{0.809070in}}%
\pgfpathlineto{\pgfqpoint{1.822805in}{0.797124in}}%
\pgfpathlineto{\pgfqpoint{1.823990in}{0.810418in}}%
\pgfpathlineto{\pgfqpoint{1.824384in}{0.803899in}}%
\pgfpathlineto{\pgfqpoint{1.825569in}{0.797462in}}%
\pgfpathlineto{\pgfqpoint{1.825963in}{0.800797in}}%
\pgfpathlineto{\pgfqpoint{1.826753in}{0.798059in}}%
\pgfpathlineto{\pgfqpoint{1.828727in}{0.780424in}}%
\pgfpathlineto{\pgfqpoint{1.831095in}{0.764264in}}%
\pgfpathlineto{\pgfqpoint{1.832279in}{0.779524in}}%
\pgfpathlineto{\pgfqpoint{1.833069in}{0.778669in}}%
\pgfpathlineto{\pgfqpoint{1.837411in}{0.726281in}}%
\pgfpathlineto{\pgfqpoint{1.838201in}{0.732202in}}%
\pgfpathlineto{\pgfqpoint{1.841753in}{0.761315in}}%
\pgfpathlineto{\pgfqpoint{1.843332in}{0.772994in}}%
\pgfpathlineto{\pgfqpoint{1.844122in}{0.768917in}}%
\pgfpathlineto{\pgfqpoint{1.846885in}{0.762612in}}%
\pgfpathlineto{\pgfqpoint{1.848069in}{0.777646in}}%
\pgfpathlineto{\pgfqpoint{1.848464in}{0.783978in}}%
\pgfpathlineto{\pgfqpoint{1.849254in}{0.778275in}}%
\pgfpathlineto{\pgfqpoint{1.852806in}{0.754024in}}%
\pgfpathlineto{\pgfqpoint{1.854385in}{0.745995in}}%
\pgfpathlineto{\pgfqpoint{1.855964in}{0.737502in}}%
\pgfpathlineto{\pgfqpoint{1.857938in}{0.750515in}}%
\pgfpathlineto{\pgfqpoint{1.858333in}{0.750116in}}%
\pgfpathlineto{\pgfqpoint{1.862675in}{0.727899in}}%
\pgfpathlineto{\pgfqpoint{1.863070in}{0.731484in}}%
\pgfpathlineto{\pgfqpoint{1.864254in}{0.749196in}}%
\pgfpathlineto{\pgfqpoint{1.866228in}{0.762099in}}%
\pgfpathlineto{\pgfqpoint{1.866623in}{0.761476in}}%
\pgfpathlineto{\pgfqpoint{1.868596in}{0.750443in}}%
\pgfpathlineto{\pgfqpoint{1.868991in}{0.754656in}}%
\pgfpathlineto{\pgfqpoint{1.869781in}{0.749868in}}%
\pgfpathlineto{\pgfqpoint{1.870570in}{0.746764in}}%
\pgfpathlineto{\pgfqpoint{1.871360in}{0.748025in}}%
\pgfpathlineto{\pgfqpoint{1.871754in}{0.748664in}}%
\pgfpathlineto{\pgfqpoint{1.874912in}{0.721844in}}%
\pgfpathlineto{\pgfqpoint{1.875702in}{0.722806in}}%
\pgfpathlineto{\pgfqpoint{1.876886in}{0.730962in}}%
\pgfpathlineto{\pgfqpoint{1.879254in}{0.758352in}}%
\pgfpathlineto{\pgfqpoint{1.880439in}{0.756801in}}%
\pgfpathlineto{\pgfqpoint{1.881623in}{0.759769in}}%
\pgfpathlineto{\pgfqpoint{1.882412in}{0.763922in}}%
\pgfpathlineto{\pgfqpoint{1.882807in}{0.760075in}}%
\pgfpathlineto{\pgfqpoint{1.884386in}{0.743972in}}%
\pgfpathlineto{\pgfqpoint{1.884781in}{0.748415in}}%
\pgfpathlineto{\pgfqpoint{1.886360in}{0.762321in}}%
\pgfpathlineto{\pgfqpoint{1.886755in}{0.759998in}}%
\pgfpathlineto{\pgfqpoint{1.888334in}{0.724778in}}%
\pgfpathlineto{\pgfqpoint{1.889123in}{0.736950in}}%
\pgfpathlineto{\pgfqpoint{1.891886in}{0.757241in}}%
\pgfpathlineto{\pgfqpoint{1.893071in}{0.742450in}}%
\pgfpathlineto{\pgfqpoint{1.893860in}{0.746852in}}%
\pgfpathlineto{\pgfqpoint{1.895834in}{0.756693in}}%
\pgfpathlineto{\pgfqpoint{1.896229in}{0.759317in}}%
\pgfpathlineto{\pgfqpoint{1.897018in}{0.757229in}}%
\pgfpathlineto{\pgfqpoint{1.899781in}{0.736977in}}%
\pgfpathlineto{\pgfqpoint{1.900966in}{0.743928in}}%
\pgfpathlineto{\pgfqpoint{1.910440in}{0.803260in}}%
\pgfpathlineto{\pgfqpoint{1.911229in}{0.799440in}}%
\pgfpathlineto{\pgfqpoint{1.913598in}{0.784489in}}%
\pgfpathlineto{\pgfqpoint{1.917545in}{0.726923in}}%
\pgfpathlineto{\pgfqpoint{1.918335in}{0.730522in}}%
\pgfpathlineto{\pgfqpoint{1.920308in}{0.739241in}}%
\pgfpathlineto{\pgfqpoint{1.920703in}{0.736850in}}%
\pgfpathlineto{\pgfqpoint{1.924256in}{0.721714in}}%
\pgfpathlineto{\pgfqpoint{1.926230in}{0.754017in}}%
\pgfpathlineto{\pgfqpoint{1.926624in}{0.749764in}}%
\pgfpathlineto{\pgfqpoint{1.928598in}{0.727577in}}%
\pgfpathlineto{\pgfqpoint{1.928993in}{0.731977in}}%
\pgfpathlineto{\pgfqpoint{1.932151in}{0.754924in}}%
\pgfpathlineto{\pgfqpoint{1.933730in}{0.745297in}}%
\pgfpathlineto{\pgfqpoint{1.934125in}{0.745653in}}%
\pgfpathlineto{\pgfqpoint{1.934914in}{0.748087in}}%
\pgfpathlineto{\pgfqpoint{1.936888in}{0.724736in}}%
\pgfpathlineto{\pgfqpoint{1.937677in}{0.729327in}}%
\pgfpathlineto{\pgfqpoint{1.939651in}{0.734936in}}%
\pgfpathlineto{\pgfqpoint{1.940046in}{0.734852in}}%
\pgfpathlineto{\pgfqpoint{1.941625in}{0.729031in}}%
\pgfpathlineto{\pgfqpoint{1.943993in}{0.750387in}}%
\pgfpathlineto{\pgfqpoint{1.944783in}{0.744152in}}%
\pgfpathlineto{\pgfqpoint{1.949125in}{0.709355in}}%
\pgfpathlineto{\pgfqpoint{1.949520in}{0.713446in}}%
\pgfpathlineto{\pgfqpoint{1.951888in}{0.737575in}}%
\pgfpathlineto{\pgfqpoint{1.952678in}{0.738558in}}%
\pgfpathlineto{\pgfqpoint{1.953467in}{0.737453in}}%
\pgfpathlineto{\pgfqpoint{1.954257in}{0.735052in}}%
\pgfpathlineto{\pgfqpoint{1.954652in}{0.738483in}}%
\pgfpathlineto{\pgfqpoint{1.955441in}{0.743791in}}%
\pgfpathlineto{\pgfqpoint{1.955836in}{0.740455in}}%
\pgfpathlineto{\pgfqpoint{1.956625in}{0.736622in}}%
\pgfpathlineto{\pgfqpoint{1.957020in}{0.740493in}}%
\pgfpathlineto{\pgfqpoint{1.958204in}{0.740090in}}%
\pgfpathlineto{\pgfqpoint{1.959783in}{0.736649in}}%
\pgfpathlineto{\pgfqpoint{1.960178in}{0.738214in}}%
\pgfpathlineto{\pgfqpoint{1.962546in}{0.746613in}}%
\pgfpathlineto{\pgfqpoint{1.963336in}{0.751174in}}%
\pgfpathlineto{\pgfqpoint{1.963731in}{0.741669in}}%
\pgfpathlineto{\pgfqpoint{1.964520in}{0.749634in}}%
\pgfpathlineto{\pgfqpoint{1.964915in}{0.751556in}}%
\pgfpathlineto{\pgfqpoint{1.965310in}{0.746882in}}%
\pgfpathlineto{\pgfqpoint{1.965704in}{0.741839in}}%
\pgfpathlineto{\pgfqpoint{1.966494in}{0.750075in}}%
\pgfpathlineto{\pgfqpoint{1.966889in}{0.753918in}}%
\pgfpathlineto{\pgfqpoint{1.967678in}{0.749229in}}%
\pgfpathlineto{\pgfqpoint{1.969257in}{0.740914in}}%
\pgfpathlineto{\pgfqpoint{1.969652in}{0.742636in}}%
\pgfpathlineto{\pgfqpoint{1.970047in}{0.745040in}}%
\pgfpathlineto{\pgfqpoint{1.970836in}{0.741873in}}%
\pgfpathlineto{\pgfqpoint{1.971231in}{0.739314in}}%
\pgfpathlineto{\pgfqpoint{1.972020in}{0.742323in}}%
\pgfpathlineto{\pgfqpoint{1.972810in}{0.747878in}}%
\pgfpathlineto{\pgfqpoint{1.973205in}{0.745035in}}%
\pgfpathlineto{\pgfqpoint{1.974389in}{0.734484in}}%
\pgfpathlineto{\pgfqpoint{1.975178in}{0.738808in}}%
\pgfpathlineto{\pgfqpoint{1.977152in}{0.746670in}}%
\pgfpathlineto{\pgfqpoint{1.977547in}{0.743157in}}%
\pgfpathlineto{\pgfqpoint{1.977942in}{0.739597in}}%
\pgfpathlineto{\pgfqpoint{1.978731in}{0.745138in}}%
\pgfpathlineto{\pgfqpoint{1.979126in}{0.747599in}}%
\pgfpathlineto{\pgfqpoint{1.979915in}{0.742864in}}%
\pgfpathlineto{\pgfqpoint{1.981100in}{0.733641in}}%
\pgfpathlineto{\pgfqpoint{1.981494in}{0.734685in}}%
\pgfpathlineto{\pgfqpoint{1.983073in}{0.757351in}}%
\pgfpathlineto{\pgfqpoint{1.983863in}{0.752963in}}%
\pgfpathlineto{\pgfqpoint{1.987416in}{0.736847in}}%
\pgfpathlineto{\pgfqpoint{1.988995in}{0.722526in}}%
\pgfpathlineto{\pgfqpoint{1.989389in}{0.728277in}}%
\pgfpathlineto{\pgfqpoint{1.989784in}{0.727993in}}%
\pgfpathlineto{\pgfqpoint{1.990179in}{0.721557in}}%
\pgfpathlineto{\pgfqpoint{1.990968in}{0.727634in}}%
\pgfpathlineto{\pgfqpoint{1.994521in}{0.771786in}}%
\pgfpathlineto{\pgfqpoint{1.996495in}{0.776465in}}%
\pgfpathlineto{\pgfqpoint{2.001232in}{0.754504in}}%
\pgfpathlineto{\pgfqpoint{2.002021in}{0.758326in}}%
\pgfpathlineto{\pgfqpoint{2.002416in}{0.755463in}}%
\pgfpathlineto{\pgfqpoint{2.006364in}{0.728461in}}%
\pgfpathlineto{\pgfqpoint{2.007548in}{0.732358in}}%
\pgfpathlineto{\pgfqpoint{2.008337in}{0.744867in}}%
\pgfpathlineto{\pgfqpoint{2.009522in}{0.742008in}}%
\pgfpathlineto{\pgfqpoint{2.010706in}{0.722196in}}%
\pgfpathlineto{\pgfqpoint{2.012680in}{0.706359in}}%
\pgfpathlineto{\pgfqpoint{2.015048in}{0.692171in}}%
\pgfpathlineto{\pgfqpoint{2.015838in}{0.704505in}}%
\pgfpathlineto{\pgfqpoint{2.016627in}{0.698924in}}%
\pgfpathlineto{\pgfqpoint{2.017417in}{0.690767in}}%
\pgfpathlineto{\pgfqpoint{2.018206in}{0.695406in}}%
\pgfpathlineto{\pgfqpoint{2.021759in}{0.715667in}}%
\pgfpathlineto{\pgfqpoint{2.022154in}{0.711530in}}%
\pgfpathlineto{\pgfqpoint{2.022943in}{0.716479in}}%
\pgfpathlineto{\pgfqpoint{2.024917in}{0.750929in}}%
\pgfpathlineto{\pgfqpoint{2.025706in}{0.744762in}}%
\pgfpathlineto{\pgfqpoint{2.026496in}{0.739954in}}%
\pgfpathlineto{\pgfqpoint{2.027285in}{0.741477in}}%
\pgfpathlineto{\pgfqpoint{2.030049in}{0.762451in}}%
\pgfpathlineto{\pgfqpoint{2.030838in}{0.767765in}}%
\pgfpathlineto{\pgfqpoint{2.031628in}{0.763237in}}%
\pgfpathlineto{\pgfqpoint{2.035970in}{0.731134in}}%
\pgfpathlineto{\pgfqpoint{2.038338in}{0.755764in}}%
\pgfpathlineto{\pgfqpoint{2.039128in}{0.752450in}}%
\pgfpathlineto{\pgfqpoint{2.042286in}{0.731693in}}%
\pgfpathlineto{\pgfqpoint{2.043075in}{0.734888in}}%
\pgfpathlineto{\pgfqpoint{2.044654in}{0.743610in}}%
\pgfpathlineto{\pgfqpoint{2.045444in}{0.739469in}}%
\pgfpathlineto{\pgfqpoint{2.050970in}{0.710473in}}%
\pgfpathlineto{\pgfqpoint{2.051365in}{0.714200in}}%
\pgfpathlineto{\pgfqpoint{2.051760in}{0.709043in}}%
\pgfpathlineto{\pgfqpoint{2.052154in}{0.710075in}}%
\pgfpathlineto{\pgfqpoint{2.053733in}{0.698078in}}%
\pgfpathlineto{\pgfqpoint{2.054128in}{0.706614in}}%
\pgfpathlineto{\pgfqpoint{2.054523in}{0.717373in}}%
\pgfpathlineto{\pgfqpoint{2.055707in}{0.709169in}}%
\pgfpathlineto{\pgfqpoint{2.057286in}{0.698716in}}%
\pgfpathlineto{\pgfqpoint{2.058076in}{0.701090in}}%
\pgfpathlineto{\pgfqpoint{2.058470in}{0.703526in}}%
\pgfpathlineto{\pgfqpoint{2.059260in}{0.699492in}}%
\pgfpathlineto{\pgfqpoint{2.059655in}{0.698955in}}%
\pgfpathlineto{\pgfqpoint{2.062418in}{0.708360in}}%
\pgfpathlineto{\pgfqpoint{2.063997in}{0.704506in}}%
\pgfpathlineto{\pgfqpoint{2.065576in}{0.709020in}}%
\pgfpathlineto{\pgfqpoint{2.065971in}{0.708021in}}%
\pgfpathlineto{\pgfqpoint{2.066760in}{0.710762in}}%
\pgfpathlineto{\pgfqpoint{2.067550in}{0.713550in}}%
\pgfpathlineto{\pgfqpoint{2.067944in}{0.708673in}}%
\pgfpathlineto{\pgfqpoint{2.075050in}{0.650104in}}%
\pgfpathlineto{\pgfqpoint{2.076234in}{0.655864in}}%
\pgfpathlineto{\pgfqpoint{2.077813in}{0.667706in}}%
\pgfpathlineto{\pgfqpoint{2.080971in}{0.690054in}}%
\pgfpathlineto{\pgfqpoint{2.081761in}{0.690460in}}%
\pgfpathlineto{\pgfqpoint{2.083340in}{0.679561in}}%
\pgfpathlineto{\pgfqpoint{2.084524in}{0.674179in}}%
\pgfpathlineto{\pgfqpoint{2.085313in}{0.678065in}}%
\pgfpathlineto{\pgfqpoint{2.088471in}{0.713986in}}%
\pgfpathlineto{\pgfqpoint{2.091235in}{0.740935in}}%
\pgfpathlineto{\pgfqpoint{2.092024in}{0.739107in}}%
\pgfpathlineto{\pgfqpoint{2.093998in}{0.709939in}}%
\pgfpathlineto{\pgfqpoint{2.095577in}{0.711106in}}%
\pgfpathlineto{\pgfqpoint{2.095972in}{0.712665in}}%
\pgfpathlineto{\pgfqpoint{2.096761in}{0.710891in}}%
\pgfpathlineto{\pgfqpoint{2.097945in}{0.704175in}}%
\pgfpathlineto{\pgfqpoint{2.099524in}{0.692675in}}%
\pgfpathlineto{\pgfqpoint{2.100314in}{0.697790in}}%
\pgfpathlineto{\pgfqpoint{2.102288in}{0.709012in}}%
\pgfpathlineto{\pgfqpoint{2.103077in}{0.702446in}}%
\pgfpathlineto{\pgfqpoint{2.105446in}{0.678012in}}%
\pgfpathlineto{\pgfqpoint{2.107814in}{0.669451in}}%
\pgfpathlineto{\pgfqpoint{2.108998in}{0.662216in}}%
\pgfpathlineto{\pgfqpoint{2.109393in}{0.666829in}}%
\pgfpathlineto{\pgfqpoint{2.110183in}{0.673792in}}%
\pgfpathlineto{\pgfqpoint{2.110577in}{0.670793in}}%
\pgfpathlineto{\pgfqpoint{2.110972in}{0.666406in}}%
\pgfpathlineto{\pgfqpoint{2.112156in}{0.668655in}}%
\pgfpathlineto{\pgfqpoint{2.114920in}{0.679565in}}%
\pgfpathlineto{\pgfqpoint{2.115709in}{0.671460in}}%
\pgfpathlineto{\pgfqpoint{2.116499in}{0.678043in}}%
\pgfpathlineto{\pgfqpoint{2.116893in}{0.677854in}}%
\pgfpathlineto{\pgfqpoint{2.117288in}{0.673274in}}%
\pgfpathlineto{\pgfqpoint{2.118078in}{0.679521in}}%
\pgfpathlineto{\pgfqpoint{2.120051in}{0.691273in}}%
\pgfpathlineto{\pgfqpoint{2.121630in}{0.687887in}}%
\pgfpathlineto{\pgfqpoint{2.123999in}{0.727143in}}%
\pgfpathlineto{\pgfqpoint{2.125578in}{0.720438in}}%
\pgfpathlineto{\pgfqpoint{2.126762in}{0.713764in}}%
\pgfpathlineto{\pgfqpoint{2.127157in}{0.714988in}}%
\pgfpathlineto{\pgfqpoint{2.129525in}{0.734648in}}%
\pgfpathlineto{\pgfqpoint{2.134657in}{0.689965in}}%
\pgfpathlineto{\pgfqpoint{2.135052in}{0.689603in}}%
\pgfpathlineto{\pgfqpoint{2.135446in}{0.693465in}}%
\pgfpathlineto{\pgfqpoint{2.135841in}{0.691336in}}%
\pgfpathlineto{\pgfqpoint{2.138210in}{0.677745in}}%
\pgfpathlineto{\pgfqpoint{2.138604in}{0.677185in}}%
\pgfpathlineto{\pgfqpoint{2.138999in}{0.678492in}}%
\pgfpathlineto{\pgfqpoint{2.139789in}{0.686327in}}%
\pgfpathlineto{\pgfqpoint{2.140578in}{0.696702in}}%
\pgfpathlineto{\pgfqpoint{2.141762in}{0.694592in}}%
\pgfpathlineto{\pgfqpoint{2.143736in}{0.689125in}}%
\pgfpathlineto{\pgfqpoint{2.144131in}{0.689661in}}%
\pgfpathlineto{\pgfqpoint{2.147684in}{0.710320in}}%
\pgfpathlineto{\pgfqpoint{2.148868in}{0.711102in}}%
\pgfpathlineto{\pgfqpoint{2.149657in}{0.702990in}}%
\pgfpathlineto{\pgfqpoint{2.151236in}{0.685555in}}%
\pgfpathlineto{\pgfqpoint{2.152026in}{0.687631in}}%
\pgfpathlineto{\pgfqpoint{2.152815in}{0.681211in}}%
\pgfpathlineto{\pgfqpoint{2.154000in}{0.682665in}}%
\pgfpathlineto{\pgfqpoint{2.154789in}{0.681529in}}%
\pgfpathlineto{\pgfqpoint{2.155579in}{0.682631in}}%
\pgfpathlineto{\pgfqpoint{2.156368in}{0.683816in}}%
\pgfpathlineto{\pgfqpoint{2.157552in}{0.687406in}}%
\pgfpathlineto{\pgfqpoint{2.157947in}{0.684811in}}%
\pgfpathlineto{\pgfqpoint{2.160710in}{0.673520in}}%
\pgfpathlineto{\pgfqpoint{2.161105in}{0.673953in}}%
\pgfpathlineto{\pgfqpoint{2.163868in}{0.693320in}}%
\pgfpathlineto{\pgfqpoint{2.167026in}{0.704603in}}%
\pgfpathlineto{\pgfqpoint{2.167816in}{0.701258in}}%
\pgfpathlineto{\pgfqpoint{2.168211in}{0.698600in}}%
\pgfpathlineto{\pgfqpoint{2.169000in}{0.700966in}}%
\pgfpathlineto{\pgfqpoint{2.169395in}{0.702321in}}%
\pgfpathlineto{\pgfqpoint{2.169790in}{0.698881in}}%
\pgfpathlineto{\pgfqpoint{2.171369in}{0.691495in}}%
\pgfpathlineto{\pgfqpoint{2.172158in}{0.692583in}}%
\pgfpathlineto{\pgfqpoint{2.175711in}{0.706995in}}%
\pgfpathlineto{\pgfqpoint{2.176500in}{0.706381in}}%
\pgfpathlineto{\pgfqpoint{2.176895in}{0.707678in}}%
\pgfpathlineto{\pgfqpoint{2.179658in}{0.720633in}}%
\pgfpathlineto{\pgfqpoint{2.180843in}{0.724174in}}%
\pgfpathlineto{\pgfqpoint{2.181237in}{0.723753in}}%
\pgfpathlineto{\pgfqpoint{2.187159in}{0.708920in}}%
\pgfpathlineto{\pgfqpoint{2.187948in}{0.710250in}}%
\pgfpathlineto{\pgfqpoint{2.193869in}{0.744795in}}%
\pgfpathlineto{\pgfqpoint{2.198606in}{0.698653in}}%
\pgfpathlineto{\pgfqpoint{2.199396in}{0.691400in}}%
\pgfpathlineto{\pgfqpoint{2.199791in}{0.694123in}}%
\pgfpathlineto{\pgfqpoint{2.201370in}{0.708089in}}%
\pgfpathlineto{\pgfqpoint{2.202159in}{0.704387in}}%
\pgfpathlineto{\pgfqpoint{2.203738in}{0.692529in}}%
\pgfpathlineto{\pgfqpoint{2.204133in}{0.694532in}}%
\pgfpathlineto{\pgfqpoint{2.205712in}{0.703563in}}%
\pgfpathlineto{\pgfqpoint{2.206896in}{0.700654in}}%
\pgfpathlineto{\pgfqpoint{2.208475in}{0.693672in}}%
\pgfpathlineto{\pgfqpoint{2.210844in}{0.707082in}}%
\pgfpathlineto{\pgfqpoint{2.213607in}{0.714637in}}%
\pgfpathlineto{\pgfqpoint{2.216370in}{0.694478in}}%
\pgfpathlineto{\pgfqpoint{2.217949in}{0.698603in}}%
\pgfpathlineto{\pgfqpoint{2.220712in}{0.719469in}}%
\pgfpathlineto{\pgfqpoint{2.221107in}{0.719311in}}%
\pgfpathlineto{\pgfqpoint{2.221502in}{0.717138in}}%
\pgfpathlineto{\pgfqpoint{2.221896in}{0.718276in}}%
\pgfpathlineto{\pgfqpoint{2.223475in}{0.731232in}}%
\pgfpathlineto{\pgfqpoint{2.223870in}{0.730311in}}%
\pgfpathlineto{\pgfqpoint{2.224660in}{0.719950in}}%
\pgfpathlineto{\pgfqpoint{2.225844in}{0.724893in}}%
\pgfpathlineto{\pgfqpoint{2.226633in}{0.721462in}}%
\pgfpathlineto{\pgfqpoint{2.227028in}{0.726283in}}%
\pgfpathlineto{\pgfqpoint{2.227423in}{0.731561in}}%
\pgfpathlineto{\pgfqpoint{2.228212in}{0.725861in}}%
\pgfpathlineto{\pgfqpoint{2.229002in}{0.721289in}}%
\pgfpathlineto{\pgfqpoint{2.229791in}{0.722728in}}%
\pgfpathlineto{\pgfqpoint{2.230186in}{0.723704in}}%
\pgfpathlineto{\pgfqpoint{2.230581in}{0.722532in}}%
\pgfpathlineto{\pgfqpoint{2.231370in}{0.718581in}}%
\pgfpathlineto{\pgfqpoint{2.231765in}{0.722928in}}%
\pgfpathlineto{\pgfqpoint{2.232160in}{0.722016in}}%
\pgfpathlineto{\pgfqpoint{2.232949in}{0.720109in}}%
\pgfpathlineto{\pgfqpoint{2.234134in}{0.733419in}}%
\pgfpathlineto{\pgfqpoint{2.235318in}{0.732423in}}%
\pgfpathlineto{\pgfqpoint{2.235713in}{0.733614in}}%
\pgfpathlineto{\pgfqpoint{2.236502in}{0.726945in}}%
\pgfpathlineto{\pgfqpoint{2.237292in}{0.727710in}}%
\pgfpathlineto{\pgfqpoint{2.238081in}{0.725899in}}%
\pgfpathlineto{\pgfqpoint{2.240450in}{0.713408in}}%
\pgfpathlineto{\pgfqpoint{2.243608in}{0.686507in}}%
\pgfpathlineto{\pgfqpoint{2.245581in}{0.697484in}}%
\pgfpathlineto{\pgfqpoint{2.246371in}{0.695505in}}%
\pgfpathlineto{\pgfqpoint{2.251503in}{0.675400in}}%
\pgfpathlineto{\pgfqpoint{2.252687in}{0.679845in}}%
\pgfpathlineto{\pgfqpoint{2.253871in}{0.686983in}}%
\pgfpathlineto{\pgfqpoint{2.254661in}{0.690823in}}%
\pgfpathlineto{\pgfqpoint{2.255055in}{0.686912in}}%
\pgfpathlineto{\pgfqpoint{2.257424in}{0.683161in}}%
\pgfpathlineto{\pgfqpoint{2.261371in}{0.708680in}}%
\pgfpathlineto{\pgfqpoint{2.262161in}{0.711783in}}%
\pgfpathlineto{\pgfqpoint{2.262950in}{0.709275in}}%
\pgfpathlineto{\pgfqpoint{2.264135in}{0.696179in}}%
\pgfpathlineto{\pgfqpoint{2.264924in}{0.700451in}}%
\pgfpathlineto{\pgfqpoint{2.266503in}{0.705756in}}%
\pgfpathlineto{\pgfqpoint{2.266898in}{0.703336in}}%
\pgfpathlineto{\pgfqpoint{2.269266in}{0.674861in}}%
\pgfpathlineto{\pgfqpoint{2.270845in}{0.676105in}}%
\pgfpathlineto{\pgfqpoint{2.273609in}{0.654347in}}%
\pgfpathlineto{\pgfqpoint{2.274398in}{0.659118in}}%
\pgfpathlineto{\pgfqpoint{2.277556in}{0.680327in}}%
\pgfpathlineto{\pgfqpoint{2.277951in}{0.680098in}}%
\pgfpathlineto{\pgfqpoint{2.278740in}{0.678305in}}%
\pgfpathlineto{\pgfqpoint{2.282293in}{0.658502in}}%
\pgfpathlineto{\pgfqpoint{2.282688in}{0.660995in}}%
\pgfpathlineto{\pgfqpoint{2.284662in}{0.670142in}}%
\pgfpathlineto{\pgfqpoint{2.285451in}{0.666421in}}%
\pgfpathlineto{\pgfqpoint{2.286635in}{0.659666in}}%
\pgfpathlineto{\pgfqpoint{2.287425in}{0.661652in}}%
\pgfpathlineto{\pgfqpoint{2.290188in}{0.670251in}}%
\pgfpathlineto{\pgfqpoint{2.290583in}{0.667370in}}%
\pgfpathlineto{\pgfqpoint{2.290978in}{0.662845in}}%
\pgfpathlineto{\pgfqpoint{2.291767in}{0.670100in}}%
\pgfpathlineto{\pgfqpoint{2.293346in}{0.677885in}}%
\pgfpathlineto{\pgfqpoint{2.293741in}{0.676122in}}%
\pgfpathlineto{\pgfqpoint{2.295714in}{0.662935in}}%
\pgfpathlineto{\pgfqpoint{2.296504in}{0.665322in}}%
\pgfpathlineto{\pgfqpoint{2.298083in}{0.662291in}}%
\pgfpathlineto{\pgfqpoint{2.300451in}{0.650021in}}%
\pgfpathlineto{\pgfqpoint{2.301241in}{0.654190in}}%
\pgfpathlineto{\pgfqpoint{2.303215in}{0.649226in}}%
\pgfpathlineto{\pgfqpoint{2.303609in}{0.649902in}}%
\pgfpathlineto{\pgfqpoint{2.305583in}{0.654959in}}%
\pgfpathlineto{\pgfqpoint{2.307162in}{0.659062in}}%
\pgfpathlineto{\pgfqpoint{2.308346in}{0.664429in}}%
\pgfpathlineto{\pgfqpoint{2.309136in}{0.662258in}}%
\pgfpathlineto{\pgfqpoint{2.309925in}{0.666101in}}%
\pgfpathlineto{\pgfqpoint{2.314662in}{0.688004in}}%
\pgfpathlineto{\pgfqpoint{2.317426in}{0.689167in}}%
\pgfpathlineto{\pgfqpoint{2.321768in}{0.675761in}}%
\pgfpathlineto{\pgfqpoint{2.322557in}{0.678203in}}%
\pgfpathlineto{\pgfqpoint{2.322952in}{0.674624in}}%
\pgfpathlineto{\pgfqpoint{2.325321in}{0.659605in}}%
\pgfpathlineto{\pgfqpoint{2.325715in}{0.661292in}}%
\pgfpathlineto{\pgfqpoint{2.328084in}{0.679773in}}%
\pgfpathlineto{\pgfqpoint{2.329268in}{0.673827in}}%
\pgfpathlineto{\pgfqpoint{2.331637in}{0.661052in}}%
\pgfpathlineto{\pgfqpoint{2.332031in}{0.661387in}}%
\pgfpathlineto{\pgfqpoint{2.332426in}{0.659035in}}%
\pgfpathlineto{\pgfqpoint{2.333216in}{0.661913in}}%
\pgfpathlineto{\pgfqpoint{2.339137in}{0.702728in}}%
\pgfpathlineto{\pgfqpoint{2.339926in}{0.702210in}}%
\pgfpathlineto{\pgfqpoint{2.341111in}{0.701097in}}%
\pgfpathlineto{\pgfqpoint{2.343084in}{0.694819in}}%
\pgfpathlineto{\pgfqpoint{2.345453in}{0.681397in}}%
\pgfpathlineto{\pgfqpoint{2.346637in}{0.686005in}}%
\pgfpathlineto{\pgfqpoint{2.347032in}{0.683174in}}%
\pgfpathlineto{\pgfqpoint{2.348216in}{0.671546in}}%
\pgfpathlineto{\pgfqpoint{2.349006in}{0.676181in}}%
\pgfpathlineto{\pgfqpoint{2.349400in}{0.679402in}}%
\pgfpathlineto{\pgfqpoint{2.350190in}{0.673296in}}%
\pgfpathlineto{\pgfqpoint{2.350979in}{0.675173in}}%
\pgfpathlineto{\pgfqpoint{2.351374in}{0.673859in}}%
\pgfpathlineto{\pgfqpoint{2.352164in}{0.672267in}}%
\pgfpathlineto{\pgfqpoint{2.352558in}{0.675060in}}%
\pgfpathlineto{\pgfqpoint{2.354532in}{0.685763in}}%
\pgfpathlineto{\pgfqpoint{2.354927in}{0.683801in}}%
\pgfpathlineto{\pgfqpoint{2.356901in}{0.678657in}}%
\pgfpathlineto{\pgfqpoint{2.360059in}{0.698577in}}%
\pgfpathlineto{\pgfqpoint{2.360848in}{0.696658in}}%
\pgfpathlineto{\pgfqpoint{2.362032in}{0.689634in}}%
\pgfpathlineto{\pgfqpoint{2.362822in}{0.690921in}}%
\pgfpathlineto{\pgfqpoint{2.363217in}{0.691411in}}%
\pgfpathlineto{\pgfqpoint{2.368348in}{0.655050in}}%
\pgfpathlineto{\pgfqpoint{2.368743in}{0.654279in}}%
\pgfpathlineto{\pgfqpoint{2.372296in}{0.677641in}}%
\pgfpathlineto{\pgfqpoint{2.373085in}{0.676048in}}%
\pgfpathlineto{\pgfqpoint{2.375059in}{0.672920in}}%
\pgfpathlineto{\pgfqpoint{2.375454in}{0.673209in}}%
\pgfpathlineto{\pgfqpoint{2.376638in}{0.660850in}}%
\pgfpathlineto{\pgfqpoint{2.377428in}{0.662300in}}%
\pgfpathlineto{\pgfqpoint{2.379796in}{0.669929in}}%
\pgfpathlineto{\pgfqpoint{2.382559in}{0.680346in}}%
\pgfpathlineto{\pgfqpoint{2.383743in}{0.686181in}}%
\pgfpathlineto{\pgfqpoint{2.384533in}{0.681803in}}%
\pgfpathlineto{\pgfqpoint{2.387691in}{0.665508in}}%
\pgfpathlineto{\pgfqpoint{2.389665in}{0.674729in}}%
\pgfpathlineto{\pgfqpoint{2.390059in}{0.674610in}}%
\pgfpathlineto{\pgfqpoint{2.390454in}{0.675482in}}%
\pgfpathlineto{\pgfqpoint{2.391244in}{0.673996in}}%
\pgfpathlineto{\pgfqpoint{2.392823in}{0.673250in}}%
\pgfpathlineto{\pgfqpoint{2.395191in}{0.658945in}}%
\pgfpathlineto{\pgfqpoint{2.395981in}{0.652475in}}%
\pgfpathlineto{\pgfqpoint{2.396770in}{0.658639in}}%
\pgfpathlineto{\pgfqpoint{2.397560in}{0.665332in}}%
\pgfpathlineto{\pgfqpoint{2.398349in}{0.660515in}}%
\pgfpathlineto{\pgfqpoint{2.399139in}{0.658898in}}%
\pgfpathlineto{\pgfqpoint{2.399533in}{0.660062in}}%
\pgfpathlineto{\pgfqpoint{2.402691in}{0.685398in}}%
\pgfpathlineto{\pgfqpoint{2.403481in}{0.682650in}}%
\pgfpathlineto{\pgfqpoint{2.404665in}{0.673250in}}%
\pgfpathlineto{\pgfqpoint{2.405455in}{0.675160in}}%
\pgfpathlineto{\pgfqpoint{2.406244in}{0.674544in}}%
\pgfpathlineto{\pgfqpoint{2.407034in}{0.665673in}}%
\pgfpathlineto{\pgfqpoint{2.408218in}{0.669331in}}%
\pgfpathlineto{\pgfqpoint{2.411771in}{0.675868in}}%
\pgfpathlineto{\pgfqpoint{2.413744in}{0.674713in}}%
\pgfpathlineto{\pgfqpoint{2.416113in}{0.690795in}}%
\pgfpathlineto{\pgfqpoint{2.416902in}{0.684309in}}%
\pgfpathlineto{\pgfqpoint{2.418876in}{0.663298in}}%
\pgfpathlineto{\pgfqpoint{2.419271in}{0.664594in}}%
\pgfpathlineto{\pgfqpoint{2.419666in}{0.666513in}}%
\pgfpathlineto{\pgfqpoint{2.420455in}{0.663760in}}%
\pgfpathlineto{\pgfqpoint{2.421245in}{0.657698in}}%
\pgfpathlineto{\pgfqpoint{2.421639in}{0.665946in}}%
\pgfpathlineto{\pgfqpoint{2.424008in}{0.687331in}}%
\pgfpathlineto{\pgfqpoint{2.428745in}{0.657215in}}%
\pgfpathlineto{\pgfqpoint{2.429534in}{0.662943in}}%
\pgfpathlineto{\pgfqpoint{2.430324in}{0.660780in}}%
\pgfpathlineto{\pgfqpoint{2.437824in}{0.638143in}}%
\pgfpathlineto{\pgfqpoint{2.438219in}{0.640188in}}%
\pgfpathlineto{\pgfqpoint{2.439403in}{0.647665in}}%
\pgfpathlineto{\pgfqpoint{2.440193in}{0.655631in}}%
\pgfpathlineto{\pgfqpoint{2.440982in}{0.653490in}}%
\pgfpathlineto{\pgfqpoint{2.442561in}{0.646330in}}%
\pgfpathlineto{\pgfqpoint{2.442956in}{0.644773in}}%
\pgfpathlineto{\pgfqpoint{2.443745in}{0.647860in}}%
\pgfpathlineto{\pgfqpoint{2.444140in}{0.648518in}}%
\pgfpathlineto{\pgfqpoint{2.444535in}{0.646174in}}%
\pgfpathlineto{\pgfqpoint{2.445324in}{0.648711in}}%
\pgfpathlineto{\pgfqpoint{2.448877in}{0.656661in}}%
\pgfpathlineto{\pgfqpoint{2.450851in}{0.652221in}}%
\pgfpathlineto{\pgfqpoint{2.451246in}{0.649025in}}%
\pgfpathlineto{\pgfqpoint{2.451640in}{0.653211in}}%
\pgfpathlineto{\pgfqpoint{2.452430in}{0.651548in}}%
\pgfpathlineto{\pgfqpoint{2.453614in}{0.651763in}}%
\pgfpathlineto{\pgfqpoint{2.458351in}{0.625428in}}%
\pgfpathlineto{\pgfqpoint{2.459141in}{0.627708in}}%
\pgfpathlineto{\pgfqpoint{2.459930in}{0.630329in}}%
\pgfpathlineto{\pgfqpoint{2.460325in}{0.632607in}}%
\pgfpathlineto{\pgfqpoint{2.461114in}{0.628926in}}%
\pgfpathlineto{\pgfqpoint{2.461904in}{0.631845in}}%
\pgfpathlineto{\pgfqpoint{2.463088in}{0.630435in}}%
\pgfpathlineto{\pgfqpoint{2.465456in}{0.628854in}}%
\pgfpathlineto{\pgfqpoint{2.467035in}{0.630358in}}%
\pgfpathlineto{\pgfqpoint{2.467430in}{0.629569in}}%
\pgfpathlineto{\pgfqpoint{2.467825in}{0.628241in}}%
\pgfpathlineto{\pgfqpoint{2.468220in}{0.629025in}}%
\pgfpathlineto{\pgfqpoint{2.469009in}{0.635987in}}%
\pgfpathlineto{\pgfqpoint{2.470193in}{0.633960in}}%
\pgfpathlineto{\pgfqpoint{2.471772in}{0.633459in}}%
\pgfpathlineto{\pgfqpoint{2.474930in}{0.654548in}}%
\pgfpathlineto{\pgfqpoint{2.476115in}{0.648797in}}%
\pgfpathlineto{\pgfqpoint{2.480457in}{0.628016in}}%
\pgfpathlineto{\pgfqpoint{2.484010in}{0.615829in}}%
\pgfpathlineto{\pgfqpoint{2.481641in}{0.628536in}}%
\pgfpathlineto{\pgfqpoint{2.484799in}{0.620433in}}%
\pgfpathlineto{\pgfqpoint{2.486378in}{0.629958in}}%
\pgfpathlineto{\pgfqpoint{2.486773in}{0.627740in}}%
\pgfpathlineto{\pgfqpoint{2.487957in}{0.620468in}}%
\pgfpathlineto{\pgfqpoint{2.489141in}{0.621026in}}%
\pgfpathlineto{\pgfqpoint{2.489931in}{0.620072in}}%
\pgfpathlineto{\pgfqpoint{2.491115in}{0.612878in}}%
\pgfpathlineto{\pgfqpoint{2.491510in}{0.615603in}}%
\pgfpathlineto{\pgfqpoint{2.494273in}{0.641406in}}%
\pgfpathlineto{\pgfqpoint{2.494668in}{0.638929in}}%
\pgfpathlineto{\pgfqpoint{2.495063in}{0.637165in}}%
\pgfpathlineto{\pgfqpoint{2.495457in}{0.639936in}}%
\pgfpathlineto{\pgfqpoint{2.495852in}{0.639834in}}%
\pgfpathlineto{\pgfqpoint{2.496642in}{0.638689in}}%
\pgfpathlineto{\pgfqpoint{2.497036in}{0.639326in}}%
\pgfpathlineto{\pgfqpoint{2.497826in}{0.635861in}}%
\pgfpathlineto{\pgfqpoint{2.498221in}{0.637508in}}%
\pgfpathlineto{\pgfqpoint{2.498615in}{0.640935in}}%
\pgfpathlineto{\pgfqpoint{2.499405in}{0.639347in}}%
\pgfpathlineto{\pgfqpoint{2.501379in}{0.629322in}}%
\pgfpathlineto{\pgfqpoint{2.503747in}{0.642350in}}%
\pgfpathlineto{\pgfqpoint{2.504142in}{0.642810in}}%
\pgfpathlineto{\pgfqpoint{2.504537in}{0.640915in}}%
\pgfpathlineto{\pgfqpoint{2.505721in}{0.634013in}}%
\pgfpathlineto{\pgfqpoint{2.507300in}{0.626369in}}%
\pgfpathlineto{\pgfqpoint{2.507695in}{0.626693in}}%
\pgfpathlineto{\pgfqpoint{2.508484in}{0.625640in}}%
\pgfpathlineto{\pgfqpoint{2.510458in}{0.622422in}}%
\pgfpathlineto{\pgfqpoint{2.510853in}{0.624198in}}%
\pgfpathlineto{\pgfqpoint{2.511247in}{0.625215in}}%
\pgfpathlineto{\pgfqpoint{2.511642in}{0.622344in}}%
\pgfpathlineto{\pgfqpoint{2.512826in}{0.614573in}}%
\pgfpathlineto{\pgfqpoint{2.513221in}{0.616584in}}%
\pgfpathlineto{\pgfqpoint{2.517958in}{0.653638in}}%
\pgfpathlineto{\pgfqpoint{2.519142in}{0.655947in}}%
\pgfpathlineto{\pgfqpoint{2.519537in}{0.655048in}}%
\pgfpathlineto{\pgfqpoint{2.521511in}{0.640543in}}%
\pgfpathlineto{\pgfqpoint{2.521906in}{0.643068in}}%
\pgfpathlineto{\pgfqpoint{2.523879in}{0.648492in}}%
\pgfpathlineto{\pgfqpoint{2.525064in}{0.647953in}}%
\pgfpathlineto{\pgfqpoint{2.527037in}{0.654602in}}%
\pgfpathlineto{\pgfqpoint{2.527432in}{0.653934in}}%
\pgfpathlineto{\pgfqpoint{2.531774in}{0.631157in}}%
\pgfpathlineto{\pgfqpoint{2.532169in}{0.633278in}}%
\pgfpathlineto{\pgfqpoint{2.534932in}{0.649928in}}%
\pgfpathlineto{\pgfqpoint{2.536511in}{0.650140in}}%
\pgfpathlineto{\pgfqpoint{2.538880in}{0.663687in}}%
\pgfpathlineto{\pgfqpoint{2.540854in}{0.673611in}}%
\pgfpathlineto{\pgfqpoint{2.541248in}{0.672840in}}%
\pgfpathlineto{\pgfqpoint{2.544012in}{0.663424in}}%
\pgfpathlineto{\pgfqpoint{2.544801in}{0.661360in}}%
\pgfpathlineto{\pgfqpoint{2.545590in}{0.653800in}}%
\pgfpathlineto{\pgfqpoint{2.546380in}{0.659435in}}%
\pgfpathlineto{\pgfqpoint{2.547564in}{0.658990in}}%
\pgfpathlineto{\pgfqpoint{2.552696in}{0.636998in}}%
\pgfpathlineto{\pgfqpoint{2.555459in}{0.631592in}}%
\pgfpathlineto{\pgfqpoint{2.555854in}{0.631666in}}%
\pgfpathlineto{\pgfqpoint{2.556643in}{0.625462in}}%
\pgfpathlineto{\pgfqpoint{2.557433in}{0.629090in}}%
\pgfpathlineto{\pgfqpoint{2.557828in}{0.631130in}}%
\pgfpathlineto{\pgfqpoint{2.558222in}{0.626263in}}%
\pgfpathlineto{\pgfqpoint{2.558617in}{0.622022in}}%
\pgfpathlineto{\pgfqpoint{2.559407in}{0.625999in}}%
\pgfpathlineto{\pgfqpoint{2.560591in}{0.637008in}}%
\pgfpathlineto{\pgfqpoint{2.561775in}{0.636770in}}%
\pgfpathlineto{\pgfqpoint{2.565723in}{0.626831in}}%
\pgfpathlineto{\pgfqpoint{2.566117in}{0.627467in}}%
\pgfpathlineto{\pgfqpoint{2.567302in}{0.640126in}}%
\pgfpathlineto{\pgfqpoint{2.568881in}{0.634823in}}%
\pgfpathlineto{\pgfqpoint{2.572039in}{0.633785in}}%
\pgfpathlineto{\pgfqpoint{2.574407in}{0.621187in}}%
\pgfpathlineto{\pgfqpoint{2.575197in}{0.623070in}}%
\pgfpathlineto{\pgfqpoint{2.575986in}{0.624787in}}%
\pgfpathlineto{\pgfqpoint{2.576381in}{0.622983in}}%
\pgfpathlineto{\pgfqpoint{2.579539in}{0.613283in}}%
\pgfpathlineto{\pgfqpoint{2.579934in}{0.613829in}}%
\pgfpathlineto{\pgfqpoint{2.581513in}{0.620732in}}%
\pgfpathlineto{\pgfqpoint{2.582302in}{0.618005in}}%
\pgfpathlineto{\pgfqpoint{2.582697in}{0.618084in}}%
\pgfpathlineto{\pgfqpoint{2.586644in}{0.645300in}}%
\pgfpathlineto{\pgfqpoint{2.587829in}{0.645058in}}%
\pgfpathlineto{\pgfqpoint{2.592960in}{0.623775in}}%
\pgfpathlineto{\pgfqpoint{2.594145in}{0.625473in}}%
\pgfpathlineto{\pgfqpoint{2.594934in}{0.626908in}}%
\pgfpathlineto{\pgfqpoint{2.596513in}{0.629069in}}%
\pgfpathlineto{\pgfqpoint{2.598882in}{0.625516in}}%
\pgfpathlineto{\pgfqpoint{2.599276in}{0.625559in}}%
\pgfpathlineto{\pgfqpoint{2.602434in}{0.637589in}}%
\pgfpathlineto{\pgfqpoint{2.603224in}{0.637970in}}%
\pgfpathlineto{\pgfqpoint{2.605592in}{0.642098in}}%
\pgfpathlineto{\pgfqpoint{2.606382in}{0.645159in}}%
\pgfpathlineto{\pgfqpoint{2.607171in}{0.642654in}}%
\pgfpathlineto{\pgfqpoint{2.608356in}{0.640837in}}%
\pgfpathlineto{\pgfqpoint{2.608750in}{0.641668in}}%
\pgfpathlineto{\pgfqpoint{2.609145in}{0.646451in}}%
\pgfpathlineto{\pgfqpoint{2.610329in}{0.645596in}}%
\pgfpathlineto{\pgfqpoint{2.611514in}{0.644926in}}%
\pgfpathlineto{\pgfqpoint{2.613093in}{0.638037in}}%
\pgfpathlineto{\pgfqpoint{2.613487in}{0.638932in}}%
\pgfpathlineto{\pgfqpoint{2.615856in}{0.643501in}}%
\pgfpathlineto{\pgfqpoint{2.616645in}{0.643475in}}%
\pgfpathlineto{\pgfqpoint{2.619014in}{0.652271in}}%
\pgfpathlineto{\pgfqpoint{2.619409in}{0.650498in}}%
\pgfpathlineto{\pgfqpoint{2.625330in}{0.621995in}}%
\pgfpathlineto{\pgfqpoint{2.626119in}{0.619785in}}%
\pgfpathlineto{\pgfqpoint{2.626909in}{0.621454in}}%
\pgfpathlineto{\pgfqpoint{2.627304in}{0.628182in}}%
\pgfpathlineto{\pgfqpoint{2.628488in}{0.624684in}}%
\pgfpathlineto{\pgfqpoint{2.630856in}{0.618087in}}%
\pgfpathlineto{\pgfqpoint{2.635593in}{0.628795in}}%
\pgfpathlineto{\pgfqpoint{2.635988in}{0.627524in}}%
\pgfpathlineto{\pgfqpoint{2.638356in}{0.619255in}}%
\pgfpathlineto{\pgfqpoint{2.639146in}{0.621859in}}%
\pgfpathlineto{\pgfqpoint{2.641909in}{0.628636in}}%
\pgfpathlineto{\pgfqpoint{2.644278in}{0.614322in}}%
\pgfpathlineto{\pgfqpoint{2.645067in}{0.618331in}}%
\pgfpathlineto{\pgfqpoint{2.646251in}{0.619926in}}%
\pgfpathlineto{\pgfqpoint{2.647041in}{0.611962in}}%
\pgfpathlineto{\pgfqpoint{2.648225in}{0.614235in}}%
\pgfpathlineto{\pgfqpoint{2.649409in}{0.613538in}}%
\pgfpathlineto{\pgfqpoint{2.652173in}{0.625813in}}%
\pgfpathlineto{\pgfqpoint{2.652962in}{0.622825in}}%
\pgfpathlineto{\pgfqpoint{2.655331in}{0.608385in}}%
\pgfpathlineto{\pgfqpoint{2.656120in}{0.604000in}}%
\pgfpathlineto{\pgfqpoint{2.656910in}{0.606391in}}%
\pgfpathlineto{\pgfqpoint{2.658883in}{0.618642in}}%
\pgfpathlineto{\pgfqpoint{2.659278in}{0.617464in}}%
\pgfpathlineto{\pgfqpoint{2.659673in}{0.616642in}}%
\pgfpathlineto{\pgfqpoint{2.660068in}{0.617906in}}%
\pgfpathlineto{\pgfqpoint{2.660462in}{0.617958in}}%
\pgfpathlineto{\pgfqpoint{2.662041in}{0.618219in}}%
\pgfpathlineto{\pgfqpoint{2.665594in}{0.600846in}}%
\pgfpathlineto{\pgfqpoint{2.666384in}{0.604119in}}%
\pgfpathlineto{\pgfqpoint{2.669147in}{0.618091in}}%
\pgfpathlineto{\pgfqpoint{2.670331in}{0.615557in}}%
\pgfpathlineto{\pgfqpoint{2.671515in}{0.610774in}}%
\pgfpathlineto{\pgfqpoint{2.672305in}{0.612517in}}%
\pgfpathlineto{\pgfqpoint{2.674279in}{0.615149in}}%
\pgfpathlineto{\pgfqpoint{2.675463in}{0.613168in}}%
\pgfpathlineto{\pgfqpoint{2.676252in}{0.602722in}}%
\pgfpathlineto{\pgfqpoint{2.677042in}{0.603885in}}%
\pgfpathlineto{\pgfqpoint{2.677831in}{0.607254in}}%
\pgfpathlineto{\pgfqpoint{2.679805in}{0.619157in}}%
\pgfpathlineto{\pgfqpoint{2.680595in}{0.619739in}}%
\pgfpathlineto{\pgfqpoint{2.681384in}{0.622704in}}%
\pgfpathlineto{\pgfqpoint{2.682174in}{0.619682in}}%
\pgfpathlineto{\pgfqpoint{2.682568in}{0.619553in}}%
\pgfpathlineto{\pgfqpoint{2.683753in}{0.624144in}}%
\pgfpathlineto{\pgfqpoint{2.684147in}{0.622929in}}%
\pgfpathlineto{\pgfqpoint{2.684542in}{0.621318in}}%
\pgfpathlineto{\pgfqpoint{2.684937in}{0.622409in}}%
\pgfpathlineto{\pgfqpoint{2.688095in}{0.633064in}}%
\pgfpathlineto{\pgfqpoint{2.691648in}{0.618958in}}%
\pgfpathlineto{\pgfqpoint{2.695200in}{0.639459in}}%
\pgfpathlineto{\pgfqpoint{2.695990in}{0.636002in}}%
\pgfpathlineto{\pgfqpoint{2.697964in}{0.624672in}}%
\pgfpathlineto{\pgfqpoint{2.698358in}{0.625726in}}%
\pgfpathlineto{\pgfqpoint{2.699543in}{0.629094in}}%
\pgfpathlineto{\pgfqpoint{2.700332in}{0.627450in}}%
\pgfpathlineto{\pgfqpoint{2.701122in}{0.626817in}}%
\pgfpathlineto{\pgfqpoint{2.701911in}{0.624231in}}%
\pgfpathlineto{\pgfqpoint{2.702306in}{0.627178in}}%
\pgfpathlineto{\pgfqpoint{2.703490in}{0.634193in}}%
\pgfpathlineto{\pgfqpoint{2.704280in}{0.632607in}}%
\pgfpathlineto{\pgfqpoint{2.705464in}{0.635412in}}%
\pgfpathlineto{\pgfqpoint{2.705859in}{0.634831in}}%
\pgfpathlineto{\pgfqpoint{2.710990in}{0.614392in}}%
\pgfpathlineto{\pgfqpoint{2.711780in}{0.608918in}}%
\pgfpathlineto{\pgfqpoint{2.712964in}{0.612437in}}%
\pgfpathlineto{\pgfqpoint{2.713359in}{0.611695in}}%
\pgfpathlineto{\pgfqpoint{2.713753in}{0.613231in}}%
\pgfpathlineto{\pgfqpoint{2.715727in}{0.616033in}}%
\pgfpathlineto{\pgfqpoint{2.716911in}{0.620565in}}%
\pgfpathlineto{\pgfqpoint{2.717701in}{0.617786in}}%
\pgfpathlineto{\pgfqpoint{2.720464in}{0.599999in}}%
\pgfpathlineto{\pgfqpoint{2.720859in}{0.603086in}}%
\pgfpathlineto{\pgfqpoint{2.721648in}{0.611399in}}%
\pgfpathlineto{\pgfqpoint{2.722833in}{0.609829in}}%
\pgfpathlineto{\pgfqpoint{2.723622in}{0.607836in}}%
\pgfpathlineto{\pgfqpoint{2.725201in}{0.600561in}}%
\pgfpathlineto{\pgfqpoint{2.725596in}{0.601783in}}%
\pgfpathlineto{\pgfqpoint{2.726385in}{0.608794in}}%
\pgfpathlineto{\pgfqpoint{2.727570in}{0.607045in}}%
\pgfpathlineto{\pgfqpoint{2.727964in}{0.605765in}}%
\pgfpathlineto{\pgfqpoint{2.728359in}{0.606429in}}%
\pgfpathlineto{\pgfqpoint{2.729149in}{0.611802in}}%
\pgfpathlineto{\pgfqpoint{2.729938in}{0.608246in}}%
\pgfpathlineto{\pgfqpoint{2.730333in}{0.608345in}}%
\pgfpathlineto{\pgfqpoint{2.731122in}{0.614820in}}%
\pgfpathlineto{\pgfqpoint{2.732307in}{0.612455in}}%
\pgfpathlineto{\pgfqpoint{2.732701in}{0.613147in}}%
\pgfpathlineto{\pgfqpoint{2.733491in}{0.611639in}}%
\pgfpathlineto{\pgfqpoint{2.734675in}{0.607235in}}%
\pgfpathlineto{\pgfqpoint{2.735070in}{0.605084in}}%
\pgfpathlineto{\pgfqpoint{2.736254in}{0.606187in}}%
\pgfpathlineto{\pgfqpoint{2.737438in}{0.609435in}}%
\pgfpathlineto{\pgfqpoint{2.738228in}{0.608089in}}%
\pgfpathlineto{\pgfqpoint{2.740596in}{0.603815in}}%
\pgfpathlineto{\pgfqpoint{2.741386in}{0.605885in}}%
\pgfpathlineto{\pgfqpoint{2.742175in}{0.604602in}}%
\pgfpathlineto{\pgfqpoint{2.744939in}{0.597312in}}%
\pgfpathlineto{\pgfqpoint{2.746518in}{0.605661in}}%
\pgfpathlineto{\pgfqpoint{2.747702in}{0.614442in}}%
\pgfpathlineto{\pgfqpoint{2.748491in}{0.612525in}}%
\pgfpathlineto{\pgfqpoint{2.749281in}{0.615617in}}%
\pgfpathlineto{\pgfqpoint{2.751649in}{0.599425in}}%
\pgfpathlineto{\pgfqpoint{2.753228in}{0.606450in}}%
\pgfpathlineto{\pgfqpoint{2.755992in}{0.625944in}}%
\pgfpathlineto{\pgfqpoint{2.757571in}{0.632041in}}%
\pgfpathlineto{\pgfqpoint{2.757965in}{0.630018in}}%
\pgfpathlineto{\pgfqpoint{2.760334in}{0.624064in}}%
\pgfpathlineto{\pgfqpoint{2.759150in}{0.631085in}}%
\pgfpathlineto{\pgfqpoint{2.760729in}{0.624451in}}%
\pgfpathlineto{\pgfqpoint{2.761518in}{0.625191in}}%
\pgfpathlineto{\pgfqpoint{2.763492in}{0.626721in}}%
\pgfpathlineto{\pgfqpoint{2.764281in}{0.629282in}}%
\pgfpathlineto{\pgfqpoint{2.765466in}{0.636735in}}%
\pgfpathlineto{\pgfqpoint{2.766255in}{0.633777in}}%
\pgfpathlineto{\pgfqpoint{2.766650in}{0.632477in}}%
\pgfpathlineto{\pgfqpoint{2.767834in}{0.633636in}}%
\pgfpathlineto{\pgfqpoint{2.769413in}{0.632562in}}%
\pgfpathlineto{\pgfqpoint{2.769808in}{0.633505in}}%
\pgfpathlineto{\pgfqpoint{2.770597in}{0.636338in}}%
\pgfpathlineto{\pgfqpoint{2.770992in}{0.635218in}}%
\pgfpathlineto{\pgfqpoint{2.772571in}{0.627632in}}%
\pgfpathlineto{\pgfqpoint{2.773755in}{0.638154in}}%
\pgfpathlineto{\pgfqpoint{2.774545in}{0.635619in}}%
\pgfpathlineto{\pgfqpoint{2.776519in}{0.631450in}}%
\pgfpathlineto{\pgfqpoint{2.777703in}{0.638076in}}%
\pgfpathlineto{\pgfqpoint{2.778492in}{0.636601in}}%
\pgfpathlineto{\pgfqpoint{2.781256in}{0.619975in}}%
\pgfpathlineto{\pgfqpoint{2.781650in}{0.620949in}}%
\pgfpathlineto{\pgfqpoint{2.786387in}{0.644499in}}%
\pgfpathlineto{\pgfqpoint{2.787177in}{0.643222in}}%
\pgfpathlineto{\pgfqpoint{2.788756in}{0.640061in}}%
\pgfpathlineto{\pgfqpoint{2.789151in}{0.641485in}}%
\pgfpathlineto{\pgfqpoint{2.791519in}{0.653900in}}%
\pgfpathlineto{\pgfqpoint{2.795466in}{0.632383in}}%
\pgfpathlineto{\pgfqpoint{2.795861in}{0.633851in}}%
\pgfpathlineto{\pgfqpoint{2.796256in}{0.634056in}}%
\pgfpathlineto{\pgfqpoint{2.797045in}{0.631670in}}%
\pgfpathlineto{\pgfqpoint{2.798230in}{0.632044in}}%
\pgfpathlineto{\pgfqpoint{2.799414in}{0.631871in}}%
\pgfpathlineto{\pgfqpoint{2.800993in}{0.627050in}}%
\pgfpathlineto{\pgfqpoint{2.801388in}{0.627878in}}%
\pgfpathlineto{\pgfqpoint{2.801782in}{0.629167in}}%
\pgfpathlineto{\pgfqpoint{2.803756in}{0.615204in}}%
\pgfpathlineto{\pgfqpoint{2.805335in}{0.614795in}}%
\pgfpathlineto{\pgfqpoint{2.807704in}{0.609515in}}%
\pgfpathlineto{\pgfqpoint{2.811256in}{0.619898in}}%
\pgfpathlineto{\pgfqpoint{2.811651in}{0.618641in}}%
\pgfpathlineto{\pgfqpoint{2.812046in}{0.620281in}}%
\pgfpathlineto{\pgfqpoint{2.813230in}{0.623462in}}%
\pgfpathlineto{\pgfqpoint{2.813625in}{0.621598in}}%
\pgfpathlineto{\pgfqpoint{2.814809in}{0.616562in}}%
\pgfpathlineto{\pgfqpoint{2.815599in}{0.617345in}}%
\pgfpathlineto{\pgfqpoint{2.815993in}{0.617431in}}%
\pgfpathlineto{\pgfqpoint{2.816783in}{0.620060in}}%
\pgfpathlineto{\pgfqpoint{2.817572in}{0.618376in}}%
\pgfpathlineto{\pgfqpoint{2.817967in}{0.619443in}}%
\pgfpathlineto{\pgfqpoint{2.818362in}{0.618616in}}%
\pgfpathlineto{\pgfqpoint{2.819151in}{0.614100in}}%
\pgfpathlineto{\pgfqpoint{2.819941in}{0.616213in}}%
\pgfpathlineto{\pgfqpoint{2.823099in}{0.631159in}}%
\pgfpathlineto{\pgfqpoint{2.824678in}{0.636170in}}%
\pgfpathlineto{\pgfqpoint{2.825073in}{0.634086in}}%
\pgfpathlineto{\pgfqpoint{2.827046in}{0.624914in}}%
\pgfpathlineto{\pgfqpoint{2.827836in}{0.629615in}}%
\pgfpathlineto{\pgfqpoint{2.829020in}{0.629225in}}%
\pgfpathlineto{\pgfqpoint{2.829810in}{0.627908in}}%
\pgfpathlineto{\pgfqpoint{2.830204in}{0.627272in}}%
\pgfpathlineto{\pgfqpoint{2.830599in}{0.627656in}}%
\pgfpathlineto{\pgfqpoint{2.831783in}{0.635129in}}%
\pgfpathlineto{\pgfqpoint{2.832178in}{0.633765in}}%
\pgfpathlineto{\pgfqpoint{2.834547in}{0.615379in}}%
\pgfpathlineto{\pgfqpoint{2.835336in}{0.618704in}}%
\pgfpathlineto{\pgfqpoint{2.835731in}{0.620417in}}%
\pgfpathlineto{\pgfqpoint{2.836520in}{0.617802in}}%
\pgfpathlineto{\pgfqpoint{2.838494in}{0.613011in}}%
\pgfpathlineto{\pgfqpoint{2.840073in}{0.618395in}}%
\pgfpathlineto{\pgfqpoint{2.840468in}{0.617970in}}%
\pgfpathlineto{\pgfqpoint{2.840863in}{0.615260in}}%
\pgfpathlineto{\pgfqpoint{2.841652in}{0.616819in}}%
\pgfpathlineto{\pgfqpoint{2.847573in}{0.632986in}}%
\pgfpathlineto{\pgfqpoint{2.847968in}{0.632901in}}%
\pgfpathlineto{\pgfqpoint{2.849152in}{0.639556in}}%
\pgfpathlineto{\pgfqpoint{2.849942in}{0.636600in}}%
\pgfpathlineto{\pgfqpoint{2.851126in}{0.630295in}}%
\pgfpathlineto{\pgfqpoint{2.851916in}{0.622275in}}%
\pgfpathlineto{\pgfqpoint{2.852705in}{0.626217in}}%
\pgfpathlineto{\pgfqpoint{2.857442in}{0.645456in}}%
\pgfpathlineto{\pgfqpoint{2.860995in}{0.632472in}}%
\pgfpathlineto{\pgfqpoint{2.862179in}{0.629952in}}%
\pgfpathlineto{\pgfqpoint{2.862574in}{0.630303in}}%
\pgfpathlineto{\pgfqpoint{2.865337in}{0.643748in}}%
\pgfpathlineto{\pgfqpoint{2.872837in}{0.657321in}}%
\pgfpathlineto{\pgfqpoint{2.873232in}{0.655156in}}%
\pgfpathlineto{\pgfqpoint{2.875601in}{0.652712in}}%
\pgfpathlineto{\pgfqpoint{2.877574in}{0.651442in}}%
\pgfpathlineto{\pgfqpoint{2.877969in}{0.650103in}}%
\pgfpathlineto{\pgfqpoint{2.878759in}{0.640310in}}%
\pgfpathlineto{\pgfqpoint{2.879548in}{0.647164in}}%
\pgfpathlineto{\pgfqpoint{2.883101in}{0.661701in}}%
\pgfpathlineto{\pgfqpoint{2.883495in}{0.660619in}}%
\pgfpathlineto{\pgfqpoint{2.886653in}{0.652862in}}%
\pgfpathlineto{\pgfqpoint{2.887838in}{0.658467in}}%
\pgfpathlineto{\pgfqpoint{2.888232in}{0.657974in}}%
\pgfpathlineto{\pgfqpoint{2.891390in}{0.637529in}}%
\pgfpathlineto{\pgfqpoint{2.891785in}{0.639082in}}%
\pgfpathlineto{\pgfqpoint{2.894154in}{0.651284in}}%
\pgfpathlineto{\pgfqpoint{2.894548in}{0.650254in}}%
\pgfpathlineto{\pgfqpoint{2.896127in}{0.640857in}}%
\pgfpathlineto{\pgfqpoint{2.896917in}{0.644154in}}%
\pgfpathlineto{\pgfqpoint{2.897312in}{0.645363in}}%
\pgfpathlineto{\pgfqpoint{2.897706in}{0.642771in}}%
\pgfpathlineto{\pgfqpoint{2.898101in}{0.642086in}}%
\pgfpathlineto{\pgfqpoint{2.898891in}{0.643654in}}%
\pgfpathlineto{\pgfqpoint{2.900075in}{0.646151in}}%
\pgfpathlineto{\pgfqpoint{2.900470in}{0.645667in}}%
\pgfpathlineto{\pgfqpoint{2.901259in}{0.641689in}}%
\pgfpathlineto{\pgfqpoint{2.901654in}{0.645023in}}%
\pgfpathlineto{\pgfqpoint{2.902838in}{0.647399in}}%
\pgfpathlineto{\pgfqpoint{2.903233in}{0.647184in}}%
\pgfpathlineto{\pgfqpoint{2.905996in}{0.636711in}}%
\pgfpathlineto{\pgfqpoint{2.906391in}{0.638010in}}%
\pgfpathlineto{\pgfqpoint{2.906786in}{0.635585in}}%
\pgfpathlineto{\pgfqpoint{2.909944in}{0.624442in}}%
\pgfpathlineto{\pgfqpoint{2.910338in}{0.624809in}}%
\pgfpathlineto{\pgfqpoint{2.910733in}{0.623196in}}%
\pgfpathlineto{\pgfqpoint{2.913496in}{0.610474in}}%
\pgfpathlineto{\pgfqpoint{2.914286in}{0.608959in}}%
\pgfpathlineto{\pgfqpoint{2.914681in}{0.610959in}}%
\pgfpathlineto{\pgfqpoint{2.916654in}{0.619950in}}%
\pgfpathlineto{\pgfqpoint{2.918628in}{0.637418in}}%
\pgfpathlineto{\pgfqpoint{2.919023in}{0.636339in}}%
\pgfpathlineto{\pgfqpoint{2.920207in}{0.633636in}}%
\pgfpathlineto{\pgfqpoint{2.920602in}{0.634371in}}%
\pgfpathlineto{\pgfqpoint{2.927313in}{0.654301in}}%
\pgfpathlineto{\pgfqpoint{2.927707in}{0.655446in}}%
\pgfpathlineto{\pgfqpoint{2.928892in}{0.645942in}}%
\pgfpathlineto{\pgfqpoint{2.929681in}{0.649175in}}%
\pgfpathlineto{\pgfqpoint{2.930865in}{0.647814in}}%
\pgfpathlineto{\pgfqpoint{2.934023in}{0.632812in}}%
\pgfpathlineto{\pgfqpoint{2.934813in}{0.635137in}}%
\pgfpathlineto{\pgfqpoint{2.935997in}{0.637115in}}%
\pgfpathlineto{\pgfqpoint{2.936392in}{0.636722in}}%
\pgfpathlineto{\pgfqpoint{2.939550in}{0.613684in}}%
\pgfpathlineto{\pgfqpoint{2.941524in}{0.614883in}}%
\pgfpathlineto{\pgfqpoint{2.944287in}{0.613369in}}%
\pgfpathlineto{\pgfqpoint{2.944682in}{0.615504in}}%
\pgfpathlineto{\pgfqpoint{2.945471in}{0.620027in}}%
\pgfpathlineto{\pgfqpoint{2.946655in}{0.618702in}}%
\pgfpathlineto{\pgfqpoint{2.949813in}{0.625778in}}%
\pgfpathlineto{\pgfqpoint{2.951392in}{0.621477in}}%
\pgfpathlineto{\pgfqpoint{2.951787in}{0.619557in}}%
\pgfpathlineto{\pgfqpoint{2.952577in}{0.621531in}}%
\pgfpathlineto{\pgfqpoint{2.953366in}{0.621009in}}%
\pgfpathlineto{\pgfqpoint{2.954156in}{0.619199in}}%
\pgfpathlineto{\pgfqpoint{2.954550in}{0.620482in}}%
\pgfpathlineto{\pgfqpoint{2.955340in}{0.622982in}}%
\pgfpathlineto{\pgfqpoint{2.955735in}{0.621620in}}%
\pgfpathlineto{\pgfqpoint{2.956129in}{0.618977in}}%
\pgfpathlineto{\pgfqpoint{2.956524in}{0.622833in}}%
\pgfpathlineto{\pgfqpoint{2.958498in}{0.626112in}}%
\pgfpathlineto{\pgfqpoint{2.958893in}{0.625443in}}%
\pgfpathlineto{\pgfqpoint{2.959287in}{0.626708in}}%
\pgfpathlineto{\pgfqpoint{2.962840in}{0.636201in}}%
\pgfpathlineto{\pgfqpoint{2.965998in}{0.623898in}}%
\pgfpathlineto{\pgfqpoint{2.966393in}{0.624283in}}%
\pgfpathlineto{\pgfqpoint{2.967972in}{0.629733in}}%
\pgfpathlineto{\pgfqpoint{2.968761in}{0.629490in}}%
\pgfpathlineto{\pgfqpoint{2.977446in}{0.603602in}}%
\pgfpathlineto{\pgfqpoint{2.979025in}{0.606258in}}%
\pgfpathlineto{\pgfqpoint{2.979419in}{0.605848in}}%
\pgfpathlineto{\pgfqpoint{2.981788in}{0.599765in}}%
\pgfpathlineto{\pgfqpoint{2.984156in}{0.601402in}}%
\pgfpathlineto{\pgfqpoint{2.989683in}{0.629997in}}%
\pgfpathlineto{\pgfqpoint{2.991262in}{0.628274in}}%
\pgfpathlineto{\pgfqpoint{2.995604in}{0.615052in}}%
\pgfpathlineto{\pgfqpoint{2.997578in}{0.609803in}}%
\pgfpathlineto{\pgfqpoint{2.998367in}{0.611927in}}%
\pgfpathlineto{\pgfqpoint{2.999552in}{0.614774in}}%
\pgfpathlineto{\pgfqpoint{2.999946in}{0.612003in}}%
\pgfpathlineto{\pgfqpoint{3.001525in}{0.614192in}}%
\pgfpathlineto{\pgfqpoint{3.002315in}{0.608566in}}%
\pgfpathlineto{\pgfqpoint{3.003104in}{0.609059in}}%
\pgfpathlineto{\pgfqpoint{3.005078in}{0.619661in}}%
\pgfpathlineto{\pgfqpoint{3.006657in}{0.625645in}}%
\pgfpathlineto{\pgfqpoint{3.007052in}{0.624413in}}%
\pgfpathlineto{\pgfqpoint{3.008631in}{0.613889in}}%
\pgfpathlineto{\pgfqpoint{3.009420in}{0.617998in}}%
\pgfpathlineto{\pgfqpoint{3.013368in}{0.636070in}}%
\pgfpathlineto{\pgfqpoint{3.014552in}{0.633482in}}%
\pgfpathlineto{\pgfqpoint{3.016131in}{0.626556in}}%
\pgfpathlineto{\pgfqpoint{3.017315in}{0.628090in}}%
\pgfpathlineto{\pgfqpoint{3.018105in}{0.630959in}}%
\pgfpathlineto{\pgfqpoint{3.019289in}{0.630302in}}%
\pgfpathlineto{\pgfqpoint{3.020473in}{0.626402in}}%
\pgfpathlineto{\pgfqpoint{3.023631in}{0.611559in}}%
\pgfpathlineto{\pgfqpoint{3.024816in}{0.605985in}}%
\pgfpathlineto{\pgfqpoint{3.026000in}{0.607162in}}%
\pgfpathlineto{\pgfqpoint{3.027184in}{0.606221in}}%
\pgfpathlineto{\pgfqpoint{3.029158in}{0.601628in}}%
\pgfpathlineto{\pgfqpoint{3.030737in}{0.606633in}}%
\pgfpathlineto{\pgfqpoint{3.031132in}{0.605504in}}%
\pgfpathlineto{\pgfqpoint{3.032711in}{0.608980in}}%
\pgfpathlineto{\pgfqpoint{3.033895in}{0.615038in}}%
\pgfpathlineto{\pgfqpoint{3.035869in}{0.622727in}}%
\pgfpathlineto{\pgfqpoint{3.036263in}{0.622278in}}%
\pgfpathlineto{\pgfqpoint{3.037842in}{0.621235in}}%
\pgfpathlineto{\pgfqpoint{3.041000in}{0.609396in}}%
\pgfpathlineto{\pgfqpoint{3.044553in}{0.600080in}}%
\pgfpathlineto{\pgfqpoint{3.045343in}{0.602545in}}%
\pgfpathlineto{\pgfqpoint{3.046132in}{0.600728in}}%
\pgfpathlineto{\pgfqpoint{3.048106in}{0.592638in}}%
\pgfpathlineto{\pgfqpoint{3.049290in}{0.594398in}}%
\pgfpathlineto{\pgfqpoint{3.050869in}{0.594238in}}%
\pgfpathlineto{\pgfqpoint{3.052053in}{0.589697in}}%
\pgfpathlineto{\pgfqpoint{3.052448in}{0.591956in}}%
\pgfpathlineto{\pgfqpoint{3.053237in}{0.593287in}}%
\pgfpathlineto{\pgfqpoint{3.053632in}{0.589049in}}%
\pgfpathlineto{\pgfqpoint{3.054422in}{0.595185in}}%
\pgfpathlineto{\pgfqpoint{3.054816in}{0.592268in}}%
\pgfpathlineto{\pgfqpoint{3.055211in}{0.588916in}}%
\pgfpathlineto{\pgfqpoint{3.055606in}{0.596593in}}%
\pgfpathlineto{\pgfqpoint{3.058369in}{0.612346in}}%
\pgfpathlineto{\pgfqpoint{3.058764in}{0.612123in}}%
\pgfpathlineto{\pgfqpoint{3.059159in}{0.613333in}}%
\pgfpathlineto{\pgfqpoint{3.063501in}{0.637407in}}%
\pgfpathlineto{\pgfqpoint{3.065475in}{0.651613in}}%
\pgfpathlineto{\pgfqpoint{3.065869in}{0.651331in}}%
\pgfpathlineto{\pgfqpoint{3.068633in}{0.643203in}}%
\pgfpathlineto{\pgfqpoint{3.071791in}{0.624183in}}%
\pgfpathlineto{\pgfqpoint{3.072185in}{0.625126in}}%
\pgfpathlineto{\pgfqpoint{3.072580in}{0.622831in}}%
\pgfpathlineto{\pgfqpoint{3.074554in}{0.620855in}}%
\pgfpathlineto{\pgfqpoint{3.075343in}{0.620262in}}%
\pgfpathlineto{\pgfqpoint{3.075738in}{0.620808in}}%
\pgfpathlineto{\pgfqpoint{3.076922in}{0.624454in}}%
\pgfpathlineto{\pgfqpoint{3.077712in}{0.622425in}}%
\pgfpathlineto{\pgfqpoint{3.078896in}{0.618454in}}%
\pgfpathlineto{\pgfqpoint{3.079291in}{0.621076in}}%
\pgfpathlineto{\pgfqpoint{3.081265in}{0.623353in}}%
\pgfpathlineto{\pgfqpoint{3.082054in}{0.623261in}}%
\pgfpathlineto{\pgfqpoint{3.084028in}{0.617874in}}%
\pgfpathlineto{\pgfqpoint{3.086396in}{0.612123in}}%
\pgfpathlineto{\pgfqpoint{3.086791in}{0.612356in}}%
\pgfpathlineto{\pgfqpoint{3.089949in}{0.619736in}}%
\pgfpathlineto{\pgfqpoint{3.090344in}{0.619323in}}%
\pgfpathlineto{\pgfqpoint{3.090739in}{0.618655in}}%
\pgfpathlineto{\pgfqpoint{3.091528in}{0.619764in}}%
\pgfpathlineto{\pgfqpoint{3.091923in}{0.620438in}}%
\pgfpathlineto{\pgfqpoint{3.092318in}{0.618902in}}%
\pgfpathlineto{\pgfqpoint{3.097844in}{0.602420in}}%
\pgfpathlineto{\pgfqpoint{3.099028in}{0.599582in}}%
\pgfpathlineto{\pgfqpoint{3.099818in}{0.601899in}}%
\pgfpathlineto{\pgfqpoint{3.100607in}{0.604768in}}%
\pgfpathlineto{\pgfqpoint{3.101002in}{0.602375in}}%
\pgfpathlineto{\pgfqpoint{3.104160in}{0.591312in}}%
\pgfpathlineto{\pgfqpoint{3.104950in}{0.592937in}}%
\pgfpathlineto{\pgfqpoint{3.107318in}{0.600654in}}%
\pgfpathlineto{\pgfqpoint{3.110081in}{0.613832in}}%
\pgfpathlineto{\pgfqpoint{3.111266in}{0.615024in}}%
\pgfpathlineto{\pgfqpoint{3.111660in}{0.614677in}}%
\pgfpathlineto{\pgfqpoint{3.112055in}{0.614394in}}%
\pgfpathlineto{\pgfqpoint{3.113634in}{0.608487in}}%
\pgfpathlineto{\pgfqpoint{3.114029in}{0.609824in}}%
\pgfpathlineto{\pgfqpoint{3.114424in}{0.612701in}}%
\pgfpathlineto{\pgfqpoint{3.115213in}{0.607986in}}%
\pgfpathlineto{\pgfqpoint{3.116397in}{0.601482in}}%
\pgfpathlineto{\pgfqpoint{3.117582in}{0.602622in}}%
\pgfpathlineto{\pgfqpoint{3.121529in}{0.594025in}}%
\pgfpathlineto{\pgfqpoint{3.122713in}{0.588844in}}%
\pgfpathlineto{\pgfqpoint{3.123108in}{0.589047in}}%
\pgfpathlineto{\pgfqpoint{3.125477in}{0.592815in}}%
\pgfpathlineto{\pgfqpoint{3.125871in}{0.593806in}}%
\pgfpathlineto{\pgfqpoint{3.126661in}{0.593196in}}%
\pgfpathlineto{\pgfqpoint{3.127845in}{0.591219in}}%
\pgfpathlineto{\pgfqpoint{3.128240in}{0.591734in}}%
\pgfpathlineto{\pgfqpoint{3.130213in}{0.593437in}}%
\pgfpathlineto{\pgfqpoint{3.133766in}{0.577773in}}%
\pgfpathlineto{\pgfqpoint{3.135345in}{0.576174in}}%
\pgfpathlineto{\pgfqpoint{3.136529in}{0.580373in}}%
\pgfpathlineto{\pgfqpoint{3.136924in}{0.578170in}}%
\pgfpathlineto{\pgfqpoint{3.138898in}{0.569968in}}%
\pgfpathlineto{\pgfqpoint{3.139293in}{0.570366in}}%
\pgfpathlineto{\pgfqpoint{3.142056in}{0.577956in}}%
\pgfpathlineto{\pgfqpoint{3.142845in}{0.577263in}}%
\pgfpathlineto{\pgfqpoint{3.144424in}{0.580974in}}%
\pgfpathlineto{\pgfqpoint{3.145214in}{0.578131in}}%
\pgfpathlineto{\pgfqpoint{3.145609in}{0.579510in}}%
\pgfpathlineto{\pgfqpoint{3.147188in}{0.594441in}}%
\pgfpathlineto{\pgfqpoint{3.147977in}{0.597804in}}%
\pgfpathlineto{\pgfqpoint{3.148767in}{0.596831in}}%
\pgfpathlineto{\pgfqpoint{3.149951in}{0.593730in}}%
\pgfpathlineto{\pgfqpoint{3.153109in}{0.584004in}}%
\pgfpathlineto{\pgfqpoint{3.155477in}{0.584901in}}%
\pgfpathlineto{\pgfqpoint{3.157056in}{0.587134in}}%
\pgfpathlineto{\pgfqpoint{3.157451in}{0.586172in}}%
\pgfpathlineto{\pgfqpoint{3.158635in}{0.582404in}}%
\pgfpathlineto{\pgfqpoint{3.159425in}{0.583506in}}%
\pgfpathlineto{\pgfqpoint{3.159820in}{0.583820in}}%
\pgfpathlineto{\pgfqpoint{3.160609in}{0.580391in}}%
\pgfpathlineto{\pgfqpoint{3.161004in}{0.582936in}}%
\pgfpathlineto{\pgfqpoint{3.162583in}{0.585897in}}%
\pgfpathlineto{\pgfqpoint{3.162978in}{0.585682in}}%
\pgfpathlineto{\pgfqpoint{3.164951in}{0.584831in}}%
\pgfpathlineto{\pgfqpoint{3.167715in}{0.596708in}}%
\pgfpathlineto{\pgfqpoint{3.168109in}{0.595391in}}%
\pgfpathlineto{\pgfqpoint{3.168504in}{0.596828in}}%
\pgfpathlineto{\pgfqpoint{3.168899in}{0.595276in}}%
\pgfpathlineto{\pgfqpoint{3.172846in}{0.582950in}}%
\pgfpathlineto{\pgfqpoint{3.175215in}{0.579420in}}%
\pgfpathlineto{\pgfqpoint{3.181926in}{0.618493in}}%
\pgfpathlineto{\pgfqpoint{3.182715in}{0.616547in}}%
\pgfpathlineto{\pgfqpoint{3.183899in}{0.609381in}}%
\pgfpathlineto{\pgfqpoint{3.185084in}{0.610697in}}%
\pgfpathlineto{\pgfqpoint{3.187057in}{0.610806in}}%
\pgfpathlineto{\pgfqpoint{3.188242in}{0.607144in}}%
\pgfpathlineto{\pgfqpoint{3.188636in}{0.609197in}}%
\pgfpathlineto{\pgfqpoint{3.190215in}{0.615181in}}%
\pgfpathlineto{\pgfqpoint{3.190610in}{0.614218in}}%
\pgfpathlineto{\pgfqpoint{3.193373in}{0.601144in}}%
\pgfpathlineto{\pgfqpoint{3.193768in}{0.602577in}}%
\pgfpathlineto{\pgfqpoint{3.195742in}{0.607079in}}%
\pgfpathlineto{\pgfqpoint{3.197321in}{0.602505in}}%
\pgfpathlineto{\pgfqpoint{3.197321in}{0.602505in}}%
\pgfusepath{stroke}%
\end{pgfscope}%
\begin{pgfscope}%
\pgfpathrectangle{\pgfqpoint{0.608025in}{0.484444in}}{\pgfqpoint{2.712595in}{1.541287in}}%
\pgfusepath{clip}%
\pgfsetbuttcap%
\pgfsetmiterjoin%
\definecolor{currentfill}{rgb}{0.839216,0.152941,0.156863}%
\pgfsetfillcolor{currentfill}%
\pgfsetlinewidth{1.003750pt}%
\definecolor{currentstroke}{rgb}{0.839216,0.152941,0.156863}%
\pgfsetstrokecolor{currentstroke}%
\pgfsetdash{}{0pt}%
\pgfsys@defobject{currentmarker}{\pgfqpoint{-0.020833in}{-0.020833in}}{\pgfqpoint{0.020833in}{0.020833in}}{%
\pgfpathmoveto{\pgfqpoint{0.020833in}{-0.000000in}}%
\pgfpathlineto{\pgfqpoint{-0.020833in}{0.020833in}}%
\pgfpathlineto{\pgfqpoint{-0.020833in}{-0.020833in}}%
\pgfpathlineto{\pgfqpoint{0.020833in}{-0.000000in}}%
\pgfpathclose%
\pgfusepath{stroke,fill}%
}%
\begin{pgfscope}%
\pgfsys@transformshift{0.762905in}{1.887216in}%
\pgfsys@useobject{currentmarker}{}%
\end{pgfscope}%
\begin{pgfscope}%
\pgfsys@transformshift{0.841855in}{1.612378in}%
\pgfsys@useobject{currentmarker}{}%
\end{pgfscope}%
\begin{pgfscope}%
\pgfsys@transformshift{0.920804in}{1.398090in}%
\pgfsys@useobject{currentmarker}{}%
\end{pgfscope}%
\begin{pgfscope}%
\pgfsys@transformshift{0.999754in}{1.275459in}%
\pgfsys@useobject{currentmarker}{}%
\end{pgfscope}%
\begin{pgfscope}%
\pgfsys@transformshift{1.078704in}{1.266517in}%
\pgfsys@useobject{currentmarker}{}%
\end{pgfscope}%
\begin{pgfscope}%
\pgfsys@transformshift{1.157654in}{1.120038in}%
\pgfsys@useobject{currentmarker}{}%
\end{pgfscope}%
\begin{pgfscope}%
\pgfsys@transformshift{1.236603in}{1.101071in}%
\pgfsys@useobject{currentmarker}{}%
\end{pgfscope}%
\begin{pgfscope}%
\pgfsys@transformshift{1.315553in}{0.943310in}%
\pgfsys@useobject{currentmarker}{}%
\end{pgfscope}%
\begin{pgfscope}%
\pgfsys@transformshift{1.394503in}{0.933140in}%
\pgfsys@useobject{currentmarker}{}%
\end{pgfscope}%
\begin{pgfscope}%
\pgfsys@transformshift{1.473453in}{0.923319in}%
\pgfsys@useobject{currentmarker}{}%
\end{pgfscope}%
\begin{pgfscope}%
\pgfsys@transformshift{1.552402in}{0.872709in}%
\pgfsys@useobject{currentmarker}{}%
\end{pgfscope}%
\begin{pgfscope}%
\pgfsys@transformshift{1.631352in}{0.834610in}%
\pgfsys@useobject{currentmarker}{}%
\end{pgfscope}%
\begin{pgfscope}%
\pgfsys@transformshift{1.710302in}{0.822095in}%
\pgfsys@useobject{currentmarker}{}%
\end{pgfscope}%
\begin{pgfscope}%
\pgfsys@transformshift{1.789252in}{0.805520in}%
\pgfsys@useobject{currentmarker}{}%
\end{pgfscope}%
\begin{pgfscope}%
\pgfsys@transformshift{1.868202in}{0.759611in}%
\pgfsys@useobject{currentmarker}{}%
\end{pgfscope}%
\begin{pgfscope}%
\pgfsys@transformshift{1.947151in}{0.730747in}%
\pgfsys@useobject{currentmarker}{}%
\end{pgfscope}%
\begin{pgfscope}%
\pgfsys@transformshift{2.026101in}{0.740323in}%
\pgfsys@useobject{currentmarker}{}%
\end{pgfscope}%
\begin{pgfscope}%
\pgfsys@transformshift{2.105051in}{0.679395in}%
\pgfsys@useobject{currentmarker}{}%
\end{pgfscope}%
\begin{pgfscope}%
\pgfsys@transformshift{2.184001in}{0.720487in}%
\pgfsys@useobject{currentmarker}{}%
\end{pgfscope}%
\begin{pgfscope}%
\pgfsys@transformshift{2.262950in}{0.709275in}%
\pgfsys@useobject{currentmarker}{}%
\end{pgfscope}%
\begin{pgfscope}%
\pgfsys@transformshift{2.341900in}{0.699303in}%
\pgfsys@useobject{currentmarker}{}%
\end{pgfscope}%
\begin{pgfscope}%
\pgfsys@transformshift{2.420850in}{0.661384in}%
\pgfsys@useobject{currentmarker}{}%
\end{pgfscope}%
\begin{pgfscope}%
\pgfsys@transformshift{2.499800in}{0.635051in}%
\pgfsys@useobject{currentmarker}{}%
\end{pgfscope}%
\begin{pgfscope}%
\pgfsys@transformshift{2.578749in}{0.613505in}%
\pgfsys@useobject{currentmarker}{}%
\end{pgfscope}%
\begin{pgfscope}%
\pgfsys@transformshift{2.657699in}{0.611420in}%
\pgfsys@useobject{currentmarker}{}%
\end{pgfscope}%
\begin{pgfscope}%
\pgfsys@transformshift{2.736649in}{0.606990in}%
\pgfsys@useobject{currentmarker}{}%
\end{pgfscope}%
\begin{pgfscope}%
\pgfsys@transformshift{2.815599in}{0.617345in}%
\pgfsys@useobject{currentmarker}{}%
\end{pgfscope}%
\begin{pgfscope}%
\pgfsys@transformshift{2.894548in}{0.650254in}%
\pgfsys@useobject{currentmarker}{}%
\end{pgfscope}%
\begin{pgfscope}%
\pgfsys@transformshift{2.973498in}{0.613333in}%
\pgfsys@useobject{currentmarker}{}%
\end{pgfscope}%
\begin{pgfscope}%
\pgfsys@transformshift{3.052448in}{0.591956in}%
\pgfsys@useobject{currentmarker}{}%
\end{pgfscope}%
\begin{pgfscope}%
\pgfsys@transformshift{3.131398in}{0.587534in}%
\pgfsys@useobject{currentmarker}{}%
\end{pgfscope}%
\end{pgfscope}%
\begin{pgfscope}%
\pgfpathrectangle{\pgfqpoint{0.608025in}{0.484444in}}{\pgfqpoint{2.712595in}{1.541287in}}%
\pgfusepath{clip}%
\pgfsetrectcap%
\pgfsetroundjoin%
\pgfsetlinewidth{1.505625pt}%
\definecolor{currentstroke}{rgb}{0.580392,0.403922,0.741176}%
\pgfsetstrokecolor{currentstroke}%
\pgfsetdash{}{0pt}%
\pgfpathmoveto{\pgfqpoint{0.731325in}{1.955466in}}%
\pgfpathlineto{\pgfqpoint{0.745536in}{1.944631in}}%
\pgfpathlineto{\pgfqpoint{0.758957in}{1.923649in}}%
\pgfpathlineto{\pgfqpoint{0.767247in}{1.899579in}}%
\pgfpathlineto{\pgfqpoint{0.772774in}{1.875393in}}%
\pgfpathlineto{\pgfqpoint{0.780274in}{1.844432in}}%
\pgfpathlineto{\pgfqpoint{0.790143in}{1.805821in}}%
\pgfpathlineto{\pgfqpoint{0.798432in}{1.792448in}}%
\pgfpathlineto{\pgfqpoint{0.801590in}{1.773131in}}%
\pgfpathlineto{\pgfqpoint{0.811459in}{1.703305in}}%
\pgfpathlineto{\pgfqpoint{0.822907in}{1.655998in}}%
\pgfpathlineto{\pgfqpoint{0.830802in}{1.644936in}}%
\pgfpathlineto{\pgfqpoint{0.834749in}{1.623105in}}%
\pgfpathlineto{\pgfqpoint{0.837907in}{1.606372in}}%
\pgfpathlineto{\pgfqpoint{0.843828in}{1.581619in}}%
\pgfpathlineto{\pgfqpoint{0.852908in}{1.536887in}}%
\pgfpathlineto{\pgfqpoint{0.860013in}{1.532442in}}%
\pgfpathlineto{\pgfqpoint{0.860408in}{1.534035in}}%
\pgfpathlineto{\pgfqpoint{0.863171in}{1.552948in}}%
\pgfpathlineto{\pgfqpoint{0.863566in}{1.552913in}}%
\pgfpathlineto{\pgfqpoint{0.864750in}{1.548622in}}%
\pgfpathlineto{\pgfqpoint{0.867119in}{1.520399in}}%
\pgfpathlineto{\pgfqpoint{0.870277in}{1.488441in}}%
\pgfpathlineto{\pgfqpoint{0.874224in}{1.477018in}}%
\pgfpathlineto{\pgfqpoint{0.877777in}{1.475516in}}%
\pgfpathlineto{\pgfqpoint{0.879750in}{1.473696in}}%
\pgfpathlineto{\pgfqpoint{0.882514in}{1.465003in}}%
\pgfpathlineto{\pgfqpoint{0.890014in}{1.440819in}}%
\pgfpathlineto{\pgfqpoint{0.891988in}{1.433673in}}%
\pgfpathlineto{\pgfqpoint{0.895540in}{1.460822in}}%
\pgfpathlineto{\pgfqpoint{0.896330in}{1.459481in}}%
\pgfpathlineto{\pgfqpoint{0.898304in}{1.443513in}}%
\pgfpathlineto{\pgfqpoint{0.904225in}{1.382773in}}%
\pgfpathlineto{\pgfqpoint{0.908172in}{1.363996in}}%
\pgfpathlineto{\pgfqpoint{0.912515in}{1.347669in}}%
\pgfpathlineto{\pgfqpoint{0.919620in}{1.333810in}}%
\pgfpathlineto{\pgfqpoint{0.921199in}{1.328338in}}%
\pgfpathlineto{\pgfqpoint{0.921989in}{1.329754in}}%
\pgfpathlineto{\pgfqpoint{0.924357in}{1.335998in}}%
\pgfpathlineto{\pgfqpoint{0.926726in}{1.370262in}}%
\pgfpathlineto{\pgfqpoint{0.927515in}{1.368260in}}%
\pgfpathlineto{\pgfqpoint{0.930278in}{1.353640in}}%
\pgfpathlineto{\pgfqpoint{0.931857in}{1.334592in}}%
\pgfpathlineto{\pgfqpoint{0.936594in}{1.269206in}}%
\pgfpathlineto{\pgfqpoint{0.940542in}{1.261985in}}%
\pgfpathlineto{\pgfqpoint{0.941726in}{1.260129in}}%
\pgfpathlineto{\pgfqpoint{0.947253in}{1.241901in}}%
\pgfpathlineto{\pgfqpoint{0.951200in}{1.210597in}}%
\pgfpathlineto{\pgfqpoint{0.951990in}{1.213667in}}%
\pgfpathlineto{\pgfqpoint{0.953569in}{1.216330in}}%
\pgfpathlineto{\pgfqpoint{0.953963in}{1.216100in}}%
\pgfpathlineto{\pgfqpoint{0.954753in}{1.214851in}}%
\pgfpathlineto{\pgfqpoint{0.958700in}{1.264128in}}%
\pgfpathlineto{\pgfqpoint{0.959490in}{1.263381in}}%
\pgfpathlineto{\pgfqpoint{0.961463in}{1.254258in}}%
\pgfpathlineto{\pgfqpoint{0.964621in}{1.218838in}}%
\pgfpathlineto{\pgfqpoint{0.970148in}{1.137975in}}%
\pgfpathlineto{\pgfqpoint{0.970937in}{1.139129in}}%
\pgfpathlineto{\pgfqpoint{0.973306in}{1.142377in}}%
\pgfpathlineto{\pgfqpoint{0.973701in}{1.140742in}}%
\pgfpathlineto{\pgfqpoint{0.974095in}{1.140279in}}%
\pgfpathlineto{\pgfqpoint{0.974885in}{1.141538in}}%
\pgfpathlineto{\pgfqpoint{0.976859in}{1.150452in}}%
\pgfpathlineto{\pgfqpoint{0.978043in}{1.147361in}}%
\pgfpathlineto{\pgfqpoint{0.979227in}{1.142602in}}%
\pgfpathlineto{\pgfqpoint{0.980017in}{1.143048in}}%
\pgfpathlineto{\pgfqpoint{0.981990in}{1.161297in}}%
\pgfpathlineto{\pgfqpoint{0.985148in}{1.180324in}}%
\pgfpathlineto{\pgfqpoint{0.986333in}{1.182684in}}%
\pgfpathlineto{\pgfqpoint{0.987122in}{1.190200in}}%
\pgfpathlineto{\pgfqpoint{0.987912in}{1.185479in}}%
\pgfpathlineto{\pgfqpoint{0.988306in}{1.185332in}}%
\pgfpathlineto{\pgfqpoint{0.988701in}{1.186757in}}%
\pgfpathlineto{\pgfqpoint{0.989491in}{1.191102in}}%
\pgfpathlineto{\pgfqpoint{0.990675in}{1.212626in}}%
\pgfpathlineto{\pgfqpoint{0.991464in}{1.209473in}}%
\pgfpathlineto{\pgfqpoint{0.993043in}{1.191503in}}%
\pgfpathlineto{\pgfqpoint{0.996201in}{1.146463in}}%
\pgfpathlineto{\pgfqpoint{0.996991in}{1.138052in}}%
\pgfpathlineto{\pgfqpoint{0.998175in}{1.105154in}}%
\pgfpathlineto{\pgfqpoint{0.999359in}{1.109346in}}%
\pgfpathlineto{\pgfqpoint{1.001728in}{1.122250in}}%
\pgfpathlineto{\pgfqpoint{1.002912in}{1.121314in}}%
\pgfpathlineto{\pgfqpoint{1.005675in}{1.112675in}}%
\pgfpathlineto{\pgfqpoint{1.006860in}{1.104333in}}%
\pgfpathlineto{\pgfqpoint{1.008044in}{1.107001in}}%
\pgfpathlineto{\pgfqpoint{1.009228in}{1.108445in}}%
\pgfpathlineto{\pgfqpoint{1.013176in}{1.144662in}}%
\pgfpathlineto{\pgfqpoint{1.013570in}{1.146029in}}%
\pgfpathlineto{\pgfqpoint{1.013965in}{1.144712in}}%
\pgfpathlineto{\pgfqpoint{1.016334in}{1.130394in}}%
\pgfpathlineto{\pgfqpoint{1.017123in}{1.134459in}}%
\pgfpathlineto{\pgfqpoint{1.019097in}{1.157196in}}%
\pgfpathlineto{\pgfqpoint{1.020281in}{1.153093in}}%
\pgfpathlineto{\pgfqpoint{1.022650in}{1.145006in}}%
\pgfpathlineto{\pgfqpoint{1.023044in}{1.147581in}}%
\pgfpathlineto{\pgfqpoint{1.023439in}{1.151624in}}%
\pgfpathlineto{\pgfqpoint{1.024229in}{1.144015in}}%
\pgfpathlineto{\pgfqpoint{1.030545in}{1.093352in}}%
\pgfpathlineto{\pgfqpoint{1.035676in}{1.066473in}}%
\pgfpathlineto{\pgfqpoint{1.036466in}{1.069190in}}%
\pgfpathlineto{\pgfqpoint{1.037650in}{1.069096in}}%
\pgfpathlineto{\pgfqpoint{1.038440in}{1.067869in}}%
\pgfpathlineto{\pgfqpoint{1.038834in}{1.068391in}}%
\pgfpathlineto{\pgfqpoint{1.039229in}{1.069988in}}%
\pgfpathlineto{\pgfqpoint{1.041992in}{1.101594in}}%
\pgfpathlineto{\pgfqpoint{1.046334in}{1.120663in}}%
\pgfpathlineto{\pgfqpoint{1.046729in}{1.119686in}}%
\pgfpathlineto{\pgfqpoint{1.047124in}{1.117800in}}%
\pgfpathlineto{\pgfqpoint{1.047913in}{1.120154in}}%
\pgfpathlineto{\pgfqpoint{1.050282in}{1.132795in}}%
\pgfpathlineto{\pgfqpoint{1.051466in}{1.140299in}}%
\pgfpathlineto{\pgfqpoint{1.051861in}{1.135196in}}%
\pgfpathlineto{\pgfqpoint{1.053440in}{1.105113in}}%
\pgfpathlineto{\pgfqpoint{1.053835in}{1.108122in}}%
\pgfpathlineto{\pgfqpoint{1.055019in}{1.125977in}}%
\pgfpathlineto{\pgfqpoint{1.055808in}{1.123078in}}%
\pgfpathlineto{\pgfqpoint{1.058572in}{1.087016in}}%
\pgfpathlineto{\pgfqpoint{1.062914in}{1.054336in}}%
\pgfpathlineto{\pgfqpoint{1.063703in}{1.050108in}}%
\pgfpathlineto{\pgfqpoint{1.064888in}{1.041706in}}%
\pgfpathlineto{\pgfqpoint{1.066072in}{1.043340in}}%
\pgfpathlineto{\pgfqpoint{1.066861in}{1.044426in}}%
\pgfpathlineto{\pgfqpoint{1.068440in}{1.053106in}}%
\pgfpathlineto{\pgfqpoint{1.070809in}{1.069099in}}%
\pgfpathlineto{\pgfqpoint{1.071204in}{1.068915in}}%
\pgfpathlineto{\pgfqpoint{1.071598in}{1.068426in}}%
\pgfpathlineto{\pgfqpoint{1.071993in}{1.069400in}}%
\pgfpathlineto{\pgfqpoint{1.073967in}{1.073241in}}%
\pgfpathlineto{\pgfqpoint{1.074362in}{1.072879in}}%
\pgfpathlineto{\pgfqpoint{1.076730in}{1.064909in}}%
\pgfpathlineto{\pgfqpoint{1.078309in}{1.057786in}}%
\pgfpathlineto{\pgfqpoint{1.080283in}{1.086254in}}%
\pgfpathlineto{\pgfqpoint{1.080678in}{1.078035in}}%
\pgfpathlineto{\pgfqpoint{1.081072in}{1.069279in}}%
\pgfpathlineto{\pgfqpoint{1.081862in}{1.076079in}}%
\pgfpathlineto{\pgfqpoint{1.083046in}{1.094301in}}%
\pgfpathlineto{\pgfqpoint{1.084625in}{1.131363in}}%
\pgfpathlineto{\pgfqpoint{1.085020in}{1.131146in}}%
\pgfpathlineto{\pgfqpoint{1.086994in}{1.115033in}}%
\pgfpathlineto{\pgfqpoint{1.092520in}{1.036100in}}%
\pgfpathlineto{\pgfqpoint{1.093704in}{1.039866in}}%
\pgfpathlineto{\pgfqpoint{1.095283in}{1.042305in}}%
\pgfpathlineto{\pgfqpoint{1.096073in}{1.041785in}}%
\pgfpathlineto{\pgfqpoint{1.099626in}{1.037046in}}%
\pgfpathlineto{\pgfqpoint{1.100020in}{1.037336in}}%
\pgfpathlineto{\pgfqpoint{1.100810in}{1.040937in}}%
\pgfpathlineto{\pgfqpoint{1.103178in}{1.049974in}}%
\pgfpathlineto{\pgfqpoint{1.104363in}{1.045653in}}%
\pgfpathlineto{\pgfqpoint{1.107126in}{1.014076in}}%
\pgfpathlineto{\pgfqpoint{1.108310in}{0.996253in}}%
\pgfpathlineto{\pgfqpoint{1.109100in}{1.001874in}}%
\pgfpathlineto{\pgfqpoint{1.110679in}{1.027969in}}%
\pgfpathlineto{\pgfqpoint{1.113442in}{1.060797in}}%
\pgfpathlineto{\pgfqpoint{1.115810in}{1.065087in}}%
\pgfpathlineto{\pgfqpoint{1.118179in}{1.058204in}}%
\pgfpathlineto{\pgfqpoint{1.120547in}{1.044088in}}%
\pgfpathlineto{\pgfqpoint{1.122916in}{1.008682in}}%
\pgfpathlineto{\pgfqpoint{1.123311in}{1.010065in}}%
\pgfpathlineto{\pgfqpoint{1.127258in}{1.027546in}}%
\pgfpathlineto{\pgfqpoint{1.128047in}{1.027006in}}%
\pgfpathlineto{\pgfqpoint{1.129626in}{1.019387in}}%
\pgfpathlineto{\pgfqpoint{1.132784in}{0.986639in}}%
\pgfpathlineto{\pgfqpoint{1.135153in}{0.969052in}}%
\pgfpathlineto{\pgfqpoint{1.135548in}{0.969368in}}%
\pgfpathlineto{\pgfqpoint{1.136337in}{0.968221in}}%
\pgfpathlineto{\pgfqpoint{1.137916in}{0.966243in}}%
\pgfpathlineto{\pgfqpoint{1.138311in}{0.966607in}}%
\pgfpathlineto{\pgfqpoint{1.141074in}{0.995075in}}%
\pgfpathlineto{\pgfqpoint{1.146995in}{1.033063in}}%
\pgfpathlineto{\pgfqpoint{1.152127in}{1.042296in}}%
\pgfpathlineto{\pgfqpoint{1.158443in}{0.969068in}}%
\pgfpathlineto{\pgfqpoint{1.158838in}{0.971565in}}%
\pgfpathlineto{\pgfqpoint{1.162785in}{0.991653in}}%
\pgfpathlineto{\pgfqpoint{1.163180in}{0.991288in}}%
\pgfpathlineto{\pgfqpoint{1.163575in}{0.976254in}}%
\pgfpathlineto{\pgfqpoint{1.164759in}{0.977408in}}%
\pgfpathlineto{\pgfqpoint{1.165154in}{0.977759in}}%
\pgfpathlineto{\pgfqpoint{1.167128in}{0.990550in}}%
\pgfpathlineto{\pgfqpoint{1.167522in}{0.991789in}}%
\pgfpathlineto{\pgfqpoint{1.168312in}{0.990011in}}%
\pgfpathlineto{\pgfqpoint{1.168707in}{0.988160in}}%
\pgfpathlineto{\pgfqpoint{1.169101in}{0.990208in}}%
\pgfpathlineto{\pgfqpoint{1.171865in}{0.997955in}}%
\pgfpathlineto{\pgfqpoint{1.173444in}{1.001540in}}%
\pgfpathlineto{\pgfqpoint{1.174628in}{1.009077in}}%
\pgfpathlineto{\pgfqpoint{1.176602in}{1.017731in}}%
\pgfpathlineto{\pgfqpoint{1.178181in}{1.027670in}}%
\pgfpathlineto{\pgfqpoint{1.179365in}{1.026497in}}%
\pgfpathlineto{\pgfqpoint{1.182128in}{1.015732in}}%
\pgfpathlineto{\pgfqpoint{1.184497in}{0.981733in}}%
\pgfpathlineto{\pgfqpoint{1.186076in}{0.959257in}}%
\pgfpathlineto{\pgfqpoint{1.186470in}{0.963307in}}%
\pgfpathlineto{\pgfqpoint{1.189234in}{0.992044in}}%
\pgfpathlineto{\pgfqpoint{1.190023in}{0.969700in}}%
\pgfpathlineto{\pgfqpoint{1.191207in}{0.977544in}}%
\pgfpathlineto{\pgfqpoint{1.191997in}{0.980859in}}%
\pgfpathlineto{\pgfqpoint{1.194365in}{1.007665in}}%
\pgfpathlineto{\pgfqpoint{1.194760in}{1.008683in}}%
\pgfpathlineto{\pgfqpoint{1.195550in}{1.007024in}}%
\pgfpathlineto{\pgfqpoint{1.197129in}{1.003447in}}%
\pgfpathlineto{\pgfqpoint{1.197523in}{1.005495in}}%
\pgfpathlineto{\pgfqpoint{1.198708in}{1.011207in}}%
\pgfpathlineto{\pgfqpoint{1.199497in}{1.009379in}}%
\pgfpathlineto{\pgfqpoint{1.200681in}{1.003768in}}%
\pgfpathlineto{\pgfqpoint{1.203839in}{0.984328in}}%
\pgfpathlineto{\pgfqpoint{1.205813in}{0.985345in}}%
\pgfpathlineto{\pgfqpoint{1.206603in}{0.987158in}}%
\pgfpathlineto{\pgfqpoint{1.209366in}{1.012888in}}%
\pgfpathlineto{\pgfqpoint{1.209760in}{1.010751in}}%
\pgfpathlineto{\pgfqpoint{1.218840in}{0.954247in}}%
\pgfpathlineto{\pgfqpoint{1.219234in}{0.965064in}}%
\pgfpathlineto{\pgfqpoint{1.220419in}{0.963717in}}%
\pgfpathlineto{\pgfqpoint{1.220813in}{0.963678in}}%
\pgfpathlineto{\pgfqpoint{1.225945in}{0.983222in}}%
\pgfpathlineto{\pgfqpoint{1.228708in}{0.976904in}}%
\pgfpathlineto{\pgfqpoint{1.229893in}{0.978425in}}%
\pgfpathlineto{\pgfqpoint{1.233840in}{0.991043in}}%
\pgfpathlineto{\pgfqpoint{1.235419in}{0.991216in}}%
\pgfpathlineto{\pgfqpoint{1.238577in}{0.982405in}}%
\pgfpathlineto{\pgfqpoint{1.240946in}{0.968218in}}%
\pgfpathlineto{\pgfqpoint{1.241735in}{0.973013in}}%
\pgfpathlineto{\pgfqpoint{1.242525in}{0.976245in}}%
\pgfpathlineto{\pgfqpoint{1.243314in}{0.973886in}}%
\pgfpathlineto{\pgfqpoint{1.248051in}{0.947698in}}%
\pgfpathlineto{\pgfqpoint{1.249235in}{0.944996in}}%
\pgfpathlineto{\pgfqpoint{1.250025in}{0.945292in}}%
\pgfpathlineto{\pgfqpoint{1.252788in}{0.940655in}}%
\pgfpathlineto{\pgfqpoint{1.253183in}{0.941633in}}%
\pgfpathlineto{\pgfqpoint{1.258709in}{0.966320in}}%
\pgfpathlineto{\pgfqpoint{1.259499in}{0.966671in}}%
\pgfpathlineto{\pgfqpoint{1.259894in}{0.965868in}}%
\pgfpathlineto{\pgfqpoint{1.260683in}{0.964203in}}%
\pgfpathlineto{\pgfqpoint{1.261078in}{0.966831in}}%
\pgfpathlineto{\pgfqpoint{1.264631in}{0.979060in}}%
\pgfpathlineto{\pgfqpoint{1.265025in}{0.979544in}}%
\pgfpathlineto{\pgfqpoint{1.265420in}{0.979203in}}%
\pgfpathlineto{\pgfqpoint{1.266210in}{0.976304in}}%
\pgfpathlineto{\pgfqpoint{1.269368in}{0.933159in}}%
\pgfpathlineto{\pgfqpoint{1.272131in}{0.957971in}}%
\pgfpathlineto{\pgfqpoint{1.272920in}{0.952931in}}%
\pgfpathlineto{\pgfqpoint{1.277263in}{0.939280in}}%
\pgfpathlineto{\pgfqpoint{1.279236in}{0.938809in}}%
\pgfpathlineto{\pgfqpoint{1.279631in}{0.939045in}}%
\pgfpathlineto{\pgfqpoint{1.280026in}{0.938072in}}%
\pgfpathlineto{\pgfqpoint{1.280815in}{0.935602in}}%
\pgfpathlineto{\pgfqpoint{1.283184in}{0.909582in}}%
\pgfpathlineto{\pgfqpoint{1.284368in}{0.912096in}}%
\pgfpathlineto{\pgfqpoint{1.287921in}{0.934434in}}%
\pgfpathlineto{\pgfqpoint{1.292263in}{0.966954in}}%
\pgfpathlineto{\pgfqpoint{1.292658in}{0.967876in}}%
\pgfpathlineto{\pgfqpoint{1.293052in}{0.967560in}}%
\pgfpathlineto{\pgfqpoint{1.295421in}{0.939994in}}%
\pgfpathlineto{\pgfqpoint{1.296210in}{0.945591in}}%
\pgfpathlineto{\pgfqpoint{1.297000in}{0.940287in}}%
\pgfpathlineto{\pgfqpoint{1.298974in}{0.924741in}}%
\pgfpathlineto{\pgfqpoint{1.299763in}{0.926984in}}%
\pgfpathlineto{\pgfqpoint{1.306474in}{0.948717in}}%
\pgfpathlineto{\pgfqpoint{1.307658in}{0.947230in}}%
\pgfpathlineto{\pgfqpoint{1.310421in}{0.935763in}}%
\pgfpathlineto{\pgfqpoint{1.315948in}{0.894691in}}%
\pgfpathlineto{\pgfqpoint{1.317922in}{0.884100in}}%
\pgfpathlineto{\pgfqpoint{1.321869in}{0.922085in}}%
\pgfpathlineto{\pgfqpoint{1.323053in}{0.918664in}}%
\pgfpathlineto{\pgfqpoint{1.327790in}{0.898561in}}%
\pgfpathlineto{\pgfqpoint{1.328185in}{0.899330in}}%
\pgfpathlineto{\pgfqpoint{1.328975in}{0.902425in}}%
\pgfpathlineto{\pgfqpoint{1.332922in}{0.929505in}}%
\pgfpathlineto{\pgfqpoint{1.337264in}{0.935810in}}%
\pgfpathlineto{\pgfqpoint{1.339633in}{0.938862in}}%
\pgfpathlineto{\pgfqpoint{1.340422in}{0.934331in}}%
\pgfpathlineto{\pgfqpoint{1.347528in}{0.854411in}}%
\pgfpathlineto{\pgfqpoint{1.349107in}{0.861776in}}%
\pgfpathlineto{\pgfqpoint{1.352265in}{0.875584in}}%
\pgfpathlineto{\pgfqpoint{1.353449in}{0.872295in}}%
\pgfpathlineto{\pgfqpoint{1.357397in}{0.857299in}}%
\pgfpathlineto{\pgfqpoint{1.357791in}{0.857782in}}%
\pgfpathlineto{\pgfqpoint{1.359765in}{0.865032in}}%
\pgfpathlineto{\pgfqpoint{1.360160in}{0.864806in}}%
\pgfpathlineto{\pgfqpoint{1.360949in}{0.863629in}}%
\pgfpathlineto{\pgfqpoint{1.361344in}{0.864979in}}%
\pgfpathlineto{\pgfqpoint{1.367265in}{0.919784in}}%
\pgfpathlineto{\pgfqpoint{1.370029in}{0.931111in}}%
\pgfpathlineto{\pgfqpoint{1.370423in}{0.930815in}}%
\pgfpathlineto{\pgfqpoint{1.371608in}{0.926815in}}%
\pgfpathlineto{\pgfqpoint{1.377134in}{0.872179in}}%
\pgfpathlineto{\pgfqpoint{1.379108in}{0.860581in}}%
\pgfpathlineto{\pgfqpoint{1.379897in}{0.862692in}}%
\pgfpathlineto{\pgfqpoint{1.381871in}{0.856843in}}%
\pgfpathlineto{\pgfqpoint{1.383055in}{0.851082in}}%
\pgfpathlineto{\pgfqpoint{1.383845in}{0.854117in}}%
\pgfpathlineto{\pgfqpoint{1.391345in}{0.882916in}}%
\pgfpathlineto{\pgfqpoint{1.391740in}{0.882129in}}%
\pgfpathlineto{\pgfqpoint{1.393319in}{0.878322in}}%
\pgfpathlineto{\pgfqpoint{1.397266in}{0.905693in}}%
\pgfpathlineto{\pgfqpoint{1.397661in}{0.904997in}}%
\pgfpathlineto{\pgfqpoint{1.398056in}{0.900657in}}%
\pgfpathlineto{\pgfqpoint{1.398845in}{0.905704in}}%
\pgfpathlineto{\pgfqpoint{1.401214in}{0.926986in}}%
\pgfpathlineto{\pgfqpoint{1.402003in}{0.928172in}}%
\pgfpathlineto{\pgfqpoint{1.402793in}{0.927550in}}%
\pgfpathlineto{\pgfqpoint{1.404372in}{0.922948in}}%
\pgfpathlineto{\pgfqpoint{1.413056in}{0.877047in}}%
\pgfpathlineto{\pgfqpoint{1.416214in}{0.871799in}}%
\pgfpathlineto{\pgfqpoint{1.416609in}{0.872296in}}%
\pgfpathlineto{\pgfqpoint{1.418583in}{0.876051in}}%
\pgfpathlineto{\pgfqpoint{1.422925in}{0.905194in}}%
\pgfpathlineto{\pgfqpoint{1.423320in}{0.903517in}}%
\pgfpathlineto{\pgfqpoint{1.428451in}{0.889619in}}%
\pgfpathlineto{\pgfqpoint{1.433978in}{0.901640in}}%
\pgfpathlineto{\pgfqpoint{1.437136in}{0.906249in}}%
\pgfpathlineto{\pgfqpoint{1.438715in}{0.905775in}}%
\pgfpathlineto{\pgfqpoint{1.443452in}{0.895680in}}%
\pgfpathlineto{\pgfqpoint{1.446610in}{0.884743in}}%
\pgfpathlineto{\pgfqpoint{1.451347in}{0.893033in}}%
\pgfpathlineto{\pgfqpoint{1.452926in}{0.892470in}}%
\pgfpathlineto{\pgfqpoint{1.457268in}{0.885134in}}%
\pgfpathlineto{\pgfqpoint{1.460821in}{0.894235in}}%
\pgfpathlineto{\pgfqpoint{1.463979in}{0.886542in}}%
\pgfpathlineto{\pgfqpoint{1.465163in}{0.888335in}}%
\pgfpathlineto{\pgfqpoint{1.467926in}{0.890759in}}%
\pgfpathlineto{\pgfqpoint{1.469110in}{0.889475in}}%
\pgfpathlineto{\pgfqpoint{1.469900in}{0.890276in}}%
\pgfpathlineto{\pgfqpoint{1.475426in}{0.899512in}}%
\pgfpathlineto{\pgfqpoint{1.475821in}{0.898345in}}%
\pgfpathlineto{\pgfqpoint{1.477400in}{0.898030in}}%
\pgfpathlineto{\pgfqpoint{1.477795in}{0.898551in}}%
\pgfpathlineto{\pgfqpoint{1.480558in}{0.903167in}}%
\pgfpathlineto{\pgfqpoint{1.484111in}{0.913640in}}%
\pgfpathlineto{\pgfqpoint{1.490427in}{0.938300in}}%
\pgfpathlineto{\pgfqpoint{1.495559in}{0.947375in}}%
\pgfpathlineto{\pgfqpoint{1.497927in}{0.945785in}}%
\pgfpathlineto{\pgfqpoint{1.498717in}{0.942223in}}%
\pgfpathlineto{\pgfqpoint{1.501085in}{0.920266in}}%
\pgfpathlineto{\pgfqpoint{1.501480in}{0.921384in}}%
\pgfpathlineto{\pgfqpoint{1.502664in}{0.929959in}}%
\pgfpathlineto{\pgfqpoint{1.503454in}{0.925595in}}%
\pgfpathlineto{\pgfqpoint{1.507401in}{0.916929in}}%
\pgfpathlineto{\pgfqpoint{1.508980in}{0.916031in}}%
\pgfpathlineto{\pgfqpoint{1.509375in}{0.916414in}}%
\pgfpathlineto{\pgfqpoint{1.511349in}{0.916394in}}%
\pgfpathlineto{\pgfqpoint{1.517270in}{0.932030in}}%
\pgfpathlineto{\pgfqpoint{1.523191in}{0.957737in}}%
\pgfpathlineto{\pgfqpoint{1.526349in}{0.925963in}}%
\pgfpathlineto{\pgfqpoint{1.527139in}{0.931360in}}%
\pgfpathlineto{\pgfqpoint{1.528323in}{0.949416in}}%
\pgfpathlineto{\pgfqpoint{1.529507in}{0.944351in}}%
\pgfpathlineto{\pgfqpoint{1.536613in}{0.910457in}}%
\pgfpathlineto{\pgfqpoint{1.538981in}{0.908412in}}%
\pgfpathlineto{\pgfqpoint{1.540165in}{0.908881in}}%
\pgfpathlineto{\pgfqpoint{1.540560in}{0.908021in}}%
\pgfpathlineto{\pgfqpoint{1.540955in}{0.907224in}}%
\pgfpathlineto{\pgfqpoint{1.541744in}{0.908956in}}%
\pgfpathlineto{\pgfqpoint{1.547665in}{0.925518in}}%
\pgfpathlineto{\pgfqpoint{1.552008in}{0.925838in}}%
\pgfpathlineto{\pgfqpoint{1.554771in}{0.929688in}}%
\pgfpathlineto{\pgfqpoint{1.555560in}{0.928983in}}%
\pgfpathlineto{\pgfqpoint{1.561876in}{0.914798in}}%
\pgfpathlineto{\pgfqpoint{1.563455in}{0.915816in}}%
\pgfpathlineto{\pgfqpoint{1.565824in}{0.915079in}}%
\pgfpathlineto{\pgfqpoint{1.571350in}{0.909733in}}%
\pgfpathlineto{\pgfqpoint{1.571745in}{0.910912in}}%
\pgfpathlineto{\pgfqpoint{1.574114in}{0.918744in}}%
\pgfpathlineto{\pgfqpoint{1.574508in}{0.916314in}}%
\pgfpathlineto{\pgfqpoint{1.576482in}{0.904294in}}%
\pgfpathlineto{\pgfqpoint{1.576877in}{0.904509in}}%
\pgfpathlineto{\pgfqpoint{1.578456in}{0.920319in}}%
\pgfpathlineto{\pgfqpoint{1.580824in}{0.916984in}}%
\pgfpathlineto{\pgfqpoint{1.585167in}{0.907873in}}%
\pgfpathlineto{\pgfqpoint{1.585561in}{0.908070in}}%
\pgfpathlineto{\pgfqpoint{1.585956in}{0.906663in}}%
\pgfpathlineto{\pgfqpoint{1.588325in}{0.901393in}}%
\pgfpathlineto{\pgfqpoint{1.589114in}{0.900235in}}%
\pgfpathlineto{\pgfqpoint{1.589904in}{0.901688in}}%
\pgfpathlineto{\pgfqpoint{1.592272in}{0.904053in}}%
\pgfpathlineto{\pgfqpoint{1.595430in}{0.908948in}}%
\pgfpathlineto{\pgfqpoint{1.599378in}{0.913416in}}%
\pgfpathlineto{\pgfqpoint{1.601746in}{0.897729in}}%
\pgfpathlineto{\pgfqpoint{1.602536in}{0.901300in}}%
\pgfpathlineto{\pgfqpoint{1.603720in}{0.915972in}}%
\pgfpathlineto{\pgfqpoint{1.605299in}{0.915768in}}%
\pgfpathlineto{\pgfqpoint{1.607667in}{0.917823in}}%
\pgfpathlineto{\pgfqpoint{1.610431in}{0.916974in}}%
\pgfpathlineto{\pgfqpoint{1.613589in}{0.917931in}}%
\pgfpathlineto{\pgfqpoint{1.619115in}{0.937714in}}%
\pgfpathlineto{\pgfqpoint{1.620299in}{0.935679in}}%
\pgfpathlineto{\pgfqpoint{1.622273in}{0.926480in}}%
\pgfpathlineto{\pgfqpoint{1.626615in}{0.897604in}}%
\pgfpathlineto{\pgfqpoint{1.627010in}{0.898500in}}%
\pgfpathlineto{\pgfqpoint{1.628194in}{0.905336in}}%
\pgfpathlineto{\pgfqpoint{1.629378in}{0.919231in}}%
\pgfpathlineto{\pgfqpoint{1.630563in}{0.916104in}}%
\pgfpathlineto{\pgfqpoint{1.632142in}{0.912385in}}%
\pgfpathlineto{\pgfqpoint{1.634115in}{0.908210in}}%
\pgfpathlineto{\pgfqpoint{1.634510in}{0.908525in}}%
\pgfpathlineto{\pgfqpoint{1.635694in}{0.906857in}}%
\pgfpathlineto{\pgfqpoint{1.638458in}{0.901984in}}%
\pgfpathlineto{\pgfqpoint{1.640826in}{0.903019in}}%
\pgfpathlineto{\pgfqpoint{1.641616in}{0.902772in}}%
\pgfpathlineto{\pgfqpoint{1.649511in}{0.937309in}}%
\pgfpathlineto{\pgfqpoint{1.650695in}{0.939076in}}%
\pgfpathlineto{\pgfqpoint{1.651090in}{0.937400in}}%
\pgfpathlineto{\pgfqpoint{1.652274in}{0.935280in}}%
\pgfpathlineto{\pgfqpoint{1.652669in}{0.935741in}}%
\pgfpathlineto{\pgfqpoint{1.655037in}{0.942997in}}%
\pgfpathlineto{\pgfqpoint{1.655432in}{0.942743in}}%
\pgfpathlineto{\pgfqpoint{1.656616in}{0.941011in}}%
\pgfpathlineto{\pgfqpoint{1.664116in}{0.917207in}}%
\pgfpathlineto{\pgfqpoint{1.666880in}{0.914314in}}%
\pgfpathlineto{\pgfqpoint{1.668064in}{0.915848in}}%
\pgfpathlineto{\pgfqpoint{1.670432in}{0.920968in}}%
\pgfpathlineto{\pgfqpoint{1.673196in}{0.920377in}}%
\pgfpathlineto{\pgfqpoint{1.674775in}{0.920890in}}%
\pgfpathlineto{\pgfqpoint{1.676748in}{0.924224in}}%
\pgfpathlineto{\pgfqpoint{1.681091in}{0.936781in}}%
\pgfpathlineto{\pgfqpoint{1.684249in}{0.939238in}}%
\pgfpathlineto{\pgfqpoint{1.684643in}{0.938500in}}%
\pgfpathlineto{\pgfqpoint{1.685433in}{0.939722in}}%
\pgfpathlineto{\pgfqpoint{1.688591in}{0.939333in}}%
\pgfpathlineto{\pgfqpoint{1.693328in}{0.931849in}}%
\pgfpathlineto{\pgfqpoint{1.697275in}{0.925263in}}%
\pgfpathlineto{\pgfqpoint{1.698460in}{0.926241in}}%
\pgfpathlineto{\pgfqpoint{1.700433in}{0.931702in}}%
\pgfpathlineto{\pgfqpoint{1.700828in}{0.929850in}}%
\pgfpathlineto{\pgfqpoint{1.702407in}{0.925352in}}%
\pgfpathlineto{\pgfqpoint{1.702802in}{0.927081in}}%
\pgfpathlineto{\pgfqpoint{1.705170in}{0.933634in}}%
\pgfpathlineto{\pgfqpoint{1.707144in}{0.933100in}}%
\pgfpathlineto{\pgfqpoint{1.710302in}{0.933914in}}%
\pgfpathlineto{\pgfqpoint{1.717802in}{0.945072in}}%
\pgfpathlineto{\pgfqpoint{1.720171in}{0.940658in}}%
\pgfpathlineto{\pgfqpoint{1.724908in}{0.918952in}}%
\pgfpathlineto{\pgfqpoint{1.726881in}{0.908360in}}%
\pgfpathlineto{\pgfqpoint{1.727276in}{0.908770in}}%
\pgfpathlineto{\pgfqpoint{1.731618in}{0.925470in}}%
\pgfpathlineto{\pgfqpoint{1.733197in}{0.927437in}}%
\pgfpathlineto{\pgfqpoint{1.733987in}{0.926905in}}%
\pgfpathlineto{\pgfqpoint{1.736750in}{0.929050in}}%
\pgfpathlineto{\pgfqpoint{1.739119in}{0.932967in}}%
\pgfpathlineto{\pgfqpoint{1.743856in}{0.942529in}}%
\pgfpathlineto{\pgfqpoint{1.745829in}{0.944152in}}%
\pgfpathlineto{\pgfqpoint{1.747803in}{0.935024in}}%
\pgfpathlineto{\pgfqpoint{1.750961in}{0.916187in}}%
\pgfpathlineto{\pgfqpoint{1.751356in}{0.916653in}}%
\pgfpathlineto{\pgfqpoint{1.752540in}{0.923473in}}%
\pgfpathlineto{\pgfqpoint{1.753330in}{0.927659in}}%
\pgfpathlineto{\pgfqpoint{1.754119in}{0.925332in}}%
\pgfpathlineto{\pgfqpoint{1.761619in}{0.905861in}}%
\pgfpathlineto{\pgfqpoint{1.764383in}{0.906875in}}%
\pgfpathlineto{\pgfqpoint{1.766356in}{0.911113in}}%
\pgfpathlineto{\pgfqpoint{1.769120in}{0.917002in}}%
\pgfpathlineto{\pgfqpoint{1.774646in}{0.929807in}}%
\pgfpathlineto{\pgfqpoint{1.775830in}{0.929605in}}%
\pgfpathlineto{\pgfqpoint{1.777804in}{0.936278in}}%
\pgfpathlineto{\pgfqpoint{1.780962in}{0.941269in}}%
\pgfpathlineto{\pgfqpoint{1.781357in}{0.941061in}}%
\pgfpathlineto{\pgfqpoint{1.783331in}{0.939626in}}%
\pgfpathlineto{\pgfqpoint{1.785304in}{0.938368in}}%
\pgfpathlineto{\pgfqpoint{1.788462in}{0.932886in}}%
\pgfpathlineto{\pgfqpoint{1.790436in}{0.930704in}}%
\pgfpathlineto{\pgfqpoint{1.790831in}{0.931204in}}%
\pgfpathlineto{\pgfqpoint{1.796357in}{0.931763in}}%
\pgfpathlineto{\pgfqpoint{1.797541in}{0.931389in}}%
\pgfpathlineto{\pgfqpoint{1.797936in}{0.932048in}}%
\pgfpathlineto{\pgfqpoint{1.798331in}{0.932741in}}%
\pgfpathlineto{\pgfqpoint{1.798726in}{0.930543in}}%
\pgfpathlineto{\pgfqpoint{1.801094in}{0.919419in}}%
\pgfpathlineto{\pgfqpoint{1.801489in}{0.919612in}}%
\pgfpathlineto{\pgfqpoint{1.804647in}{0.937240in}}%
\pgfpathlineto{\pgfqpoint{1.805831in}{0.933419in}}%
\pgfpathlineto{\pgfqpoint{1.809384in}{0.927329in}}%
\pgfpathlineto{\pgfqpoint{1.812542in}{0.923051in}}%
\pgfpathlineto{\pgfqpoint{1.812937in}{0.923790in}}%
\pgfpathlineto{\pgfqpoint{1.817674in}{0.931076in}}%
\pgfpathlineto{\pgfqpoint{1.823990in}{0.928827in}}%
\pgfpathlineto{\pgfqpoint{1.825174in}{0.927610in}}%
\pgfpathlineto{\pgfqpoint{1.825569in}{0.927865in}}%
\pgfpathlineto{\pgfqpoint{1.828727in}{0.934008in}}%
\pgfpathlineto{\pgfqpoint{1.829121in}{0.933459in}}%
\pgfpathlineto{\pgfqpoint{1.833464in}{0.927058in}}%
\pgfpathlineto{\pgfqpoint{1.838201in}{0.931508in}}%
\pgfpathlineto{\pgfqpoint{1.841753in}{0.926132in}}%
\pgfpathlineto{\pgfqpoint{1.843332in}{0.923313in}}%
\pgfpathlineto{\pgfqpoint{1.843727in}{0.924370in}}%
\pgfpathlineto{\pgfqpoint{1.846096in}{0.930616in}}%
\pgfpathlineto{\pgfqpoint{1.846885in}{0.929173in}}%
\pgfpathlineto{\pgfqpoint{1.850438in}{0.918287in}}%
\pgfpathlineto{\pgfqpoint{1.850833in}{0.918654in}}%
\pgfpathlineto{\pgfqpoint{1.852412in}{0.924246in}}%
\pgfpathlineto{\pgfqpoint{1.853201in}{0.922557in}}%
\pgfpathlineto{\pgfqpoint{1.855570in}{0.922835in}}%
\pgfpathlineto{\pgfqpoint{1.859517in}{0.926080in}}%
\pgfpathlineto{\pgfqpoint{1.863465in}{0.930128in}}%
\pgfpathlineto{\pgfqpoint{1.867807in}{0.933998in}}%
\pgfpathlineto{\pgfqpoint{1.868991in}{0.931843in}}%
\pgfpathlineto{\pgfqpoint{1.872149in}{0.921858in}}%
\pgfpathlineto{\pgfqpoint{1.873728in}{0.916168in}}%
\pgfpathlineto{\pgfqpoint{1.874518in}{0.917245in}}%
\pgfpathlineto{\pgfqpoint{1.875307in}{0.918285in}}%
\pgfpathlineto{\pgfqpoint{1.876097in}{0.917418in}}%
\pgfpathlineto{\pgfqpoint{1.882807in}{0.910536in}}%
\pgfpathlineto{\pgfqpoint{1.883991in}{0.909018in}}%
\pgfpathlineto{\pgfqpoint{1.884386in}{0.909527in}}%
\pgfpathlineto{\pgfqpoint{1.885176in}{0.910272in}}%
\pgfpathlineto{\pgfqpoint{1.885965in}{0.909346in}}%
\pgfpathlineto{\pgfqpoint{1.887939in}{0.909869in}}%
\pgfpathlineto{\pgfqpoint{1.897808in}{0.923214in}}%
\pgfpathlineto{\pgfqpoint{1.900176in}{0.919387in}}%
\pgfpathlineto{\pgfqpoint{1.900571in}{0.920466in}}%
\pgfpathlineto{\pgfqpoint{1.904124in}{0.927447in}}%
\pgfpathlineto{\pgfqpoint{1.905703in}{0.928250in}}%
\pgfpathlineto{\pgfqpoint{1.906097in}{0.927791in}}%
\pgfpathlineto{\pgfqpoint{1.908071in}{0.927117in}}%
\pgfpathlineto{\pgfqpoint{1.910834in}{0.925312in}}%
\pgfpathlineto{\pgfqpoint{1.919914in}{0.914314in}}%
\pgfpathlineto{\pgfqpoint{1.920308in}{0.914965in}}%
\pgfpathlineto{\pgfqpoint{1.925045in}{0.923190in}}%
\pgfpathlineto{\pgfqpoint{1.928993in}{0.937660in}}%
\pgfpathlineto{\pgfqpoint{1.935309in}{0.956327in}}%
\pgfpathlineto{\pgfqpoint{1.938072in}{0.958984in}}%
\pgfpathlineto{\pgfqpoint{1.943204in}{0.956009in}}%
\pgfpathlineto{\pgfqpoint{1.949915in}{0.936066in}}%
\pgfpathlineto{\pgfqpoint{1.951494in}{0.941263in}}%
\pgfpathlineto{\pgfqpoint{1.955046in}{0.952175in}}%
\pgfpathlineto{\pgfqpoint{1.960967in}{0.944332in}}%
\pgfpathlineto{\pgfqpoint{1.961757in}{0.946364in}}%
\pgfpathlineto{\pgfqpoint{1.963336in}{0.948908in}}%
\pgfpathlineto{\pgfqpoint{1.963731in}{0.947655in}}%
\pgfpathlineto{\pgfqpoint{1.965704in}{0.945338in}}%
\pgfpathlineto{\pgfqpoint{1.966099in}{0.945537in}}%
\pgfpathlineto{\pgfqpoint{1.968073in}{0.945291in}}%
\pgfpathlineto{\pgfqpoint{1.969652in}{0.942981in}}%
\pgfpathlineto{\pgfqpoint{1.974784in}{0.932484in}}%
\pgfpathlineto{\pgfqpoint{1.975178in}{0.932741in}}%
\pgfpathlineto{\pgfqpoint{1.976757in}{0.938408in}}%
\pgfpathlineto{\pgfqpoint{1.978336in}{0.940376in}}%
\pgfpathlineto{\pgfqpoint{1.981100in}{0.939190in}}%
\pgfpathlineto{\pgfqpoint{1.981494in}{0.940079in}}%
\pgfpathlineto{\pgfqpoint{1.985442in}{0.947856in}}%
\pgfpathlineto{\pgfqpoint{1.988600in}{0.956297in}}%
\pgfpathlineto{\pgfqpoint{1.990179in}{0.954961in}}%
\pgfpathlineto{\pgfqpoint{1.994126in}{0.954488in}}%
\pgfpathlineto{\pgfqpoint{1.996890in}{0.953405in}}%
\pgfpathlineto{\pgfqpoint{1.999653in}{0.951239in}}%
\pgfpathlineto{\pgfqpoint{2.000837in}{0.954333in}}%
\pgfpathlineto{\pgfqpoint{2.004785in}{0.968817in}}%
\pgfpathlineto{\pgfqpoint{2.005969in}{0.968433in}}%
\pgfpathlineto{\pgfqpoint{2.017022in}{0.955182in}}%
\pgfpathlineto{\pgfqpoint{2.021759in}{0.945274in}}%
\pgfpathlineto{\pgfqpoint{2.022943in}{0.943716in}}%
\pgfpathlineto{\pgfqpoint{2.023733in}{0.944589in}}%
\pgfpathlineto{\pgfqpoint{2.025706in}{0.944497in}}%
\pgfpathlineto{\pgfqpoint{2.026496in}{0.945785in}}%
\pgfpathlineto{\pgfqpoint{2.030049in}{0.956247in}}%
\pgfpathlineto{\pgfqpoint{2.032022in}{0.955000in}}%
\pgfpathlineto{\pgfqpoint{2.037549in}{0.948851in}}%
\pgfpathlineto{\pgfqpoint{2.039523in}{0.947238in}}%
\pgfpathlineto{\pgfqpoint{2.044654in}{0.935016in}}%
\pgfpathlineto{\pgfqpoint{2.047812in}{0.934674in}}%
\pgfpathlineto{\pgfqpoint{2.050181in}{0.941505in}}%
\pgfpathlineto{\pgfqpoint{2.050970in}{0.945305in}}%
\pgfpathlineto{\pgfqpoint{2.051760in}{0.942699in}}%
\pgfpathlineto{\pgfqpoint{2.059655in}{0.911082in}}%
\pgfpathlineto{\pgfqpoint{2.060049in}{0.911426in}}%
\pgfpathlineto{\pgfqpoint{2.060839in}{0.910610in}}%
\pgfpathlineto{\pgfqpoint{2.061628in}{0.911589in}}%
\pgfpathlineto{\pgfqpoint{2.062813in}{0.912010in}}%
\pgfpathlineto{\pgfqpoint{2.063207in}{0.911774in}}%
\pgfpathlineto{\pgfqpoint{2.063997in}{0.910527in}}%
\pgfpathlineto{\pgfqpoint{2.069523in}{0.903279in}}%
\pgfpathlineto{\pgfqpoint{2.070313in}{0.904137in}}%
\pgfpathlineto{\pgfqpoint{2.083734in}{0.932325in}}%
\pgfpathlineto{\pgfqpoint{2.084129in}{0.931788in}}%
\pgfpathlineto{\pgfqpoint{2.088471in}{0.926789in}}%
\pgfpathlineto{\pgfqpoint{2.093208in}{0.916402in}}%
\pgfpathlineto{\pgfqpoint{2.095972in}{0.914878in}}%
\pgfpathlineto{\pgfqpoint{2.097945in}{0.915331in}}%
\pgfpathlineto{\pgfqpoint{2.100709in}{0.910934in}}%
\pgfpathlineto{\pgfqpoint{2.104261in}{0.901793in}}%
\pgfpathlineto{\pgfqpoint{2.106235in}{0.902454in}}%
\pgfpathlineto{\pgfqpoint{2.110577in}{0.905429in}}%
\pgfpathlineto{\pgfqpoint{2.113735in}{0.900970in}}%
\pgfpathlineto{\pgfqpoint{2.117683in}{0.892572in}}%
\pgfpathlineto{\pgfqpoint{2.118078in}{0.893332in}}%
\pgfpathlineto{\pgfqpoint{2.125973in}{0.905781in}}%
\pgfpathlineto{\pgfqpoint{2.126367in}{0.905215in}}%
\pgfpathlineto{\pgfqpoint{2.127946in}{0.905340in}}%
\pgfpathlineto{\pgfqpoint{2.138999in}{0.918810in}}%
\pgfpathlineto{\pgfqpoint{2.140578in}{0.916523in}}%
\pgfpathlineto{\pgfqpoint{2.143736in}{0.910341in}}%
\pgfpathlineto{\pgfqpoint{2.153605in}{0.890780in}}%
\pgfpathlineto{\pgfqpoint{2.160316in}{0.901007in}}%
\pgfpathlineto{\pgfqpoint{2.160710in}{0.899927in}}%
\pgfpathlineto{\pgfqpoint{2.164263in}{0.899135in}}%
\pgfpathlineto{\pgfqpoint{2.167026in}{0.901698in}}%
\pgfpathlineto{\pgfqpoint{2.170184in}{0.898393in}}%
\pgfpathlineto{\pgfqpoint{2.178869in}{0.913583in}}%
\pgfpathlineto{\pgfqpoint{2.179658in}{0.913057in}}%
\pgfpathlineto{\pgfqpoint{2.180053in}{0.913249in}}%
\pgfpathlineto{\pgfqpoint{2.182027in}{0.911154in}}%
\pgfpathlineto{\pgfqpoint{2.186369in}{0.910518in}}%
\pgfpathlineto{\pgfqpoint{2.189527in}{0.907493in}}%
\pgfpathlineto{\pgfqpoint{2.192290in}{0.906788in}}%
\pgfpathlineto{\pgfqpoint{2.195843in}{0.901942in}}%
\pgfpathlineto{\pgfqpoint{2.198212in}{0.898614in}}%
\pgfpathlineto{\pgfqpoint{2.202554in}{0.890543in}}%
\pgfpathlineto{\pgfqpoint{2.206107in}{0.887090in}}%
\pgfpathlineto{\pgfqpoint{2.213212in}{0.877032in}}%
\pgfpathlineto{\pgfqpoint{2.213607in}{0.877373in}}%
\pgfpathlineto{\pgfqpoint{2.217554in}{0.880316in}}%
\pgfpathlineto{\pgfqpoint{2.221107in}{0.877950in}}%
\pgfpathlineto{\pgfqpoint{2.225844in}{0.883510in}}%
\pgfpathlineto{\pgfqpoint{2.226239in}{0.883091in}}%
\pgfpathlineto{\pgfqpoint{2.226633in}{0.884564in}}%
\pgfpathlineto{\pgfqpoint{2.231765in}{0.897263in}}%
\pgfpathlineto{\pgfqpoint{2.232160in}{0.897067in}}%
\pgfpathlineto{\pgfqpoint{2.234528in}{0.895753in}}%
\pgfpathlineto{\pgfqpoint{2.236897in}{0.893248in}}%
\pgfpathlineto{\pgfqpoint{2.240450in}{0.892267in}}%
\pgfpathlineto{\pgfqpoint{2.242423in}{0.892949in}}%
\pgfpathlineto{\pgfqpoint{2.244792in}{0.900846in}}%
\pgfpathlineto{\pgfqpoint{2.248345in}{0.909278in}}%
\pgfpathlineto{\pgfqpoint{2.251503in}{0.911175in}}%
\pgfpathlineto{\pgfqpoint{2.259003in}{0.924549in}}%
\pgfpathlineto{\pgfqpoint{2.259398in}{0.923492in}}%
\pgfpathlineto{\pgfqpoint{2.263345in}{0.915357in}}%
\pgfpathlineto{\pgfqpoint{2.265714in}{0.910575in}}%
\pgfpathlineto{\pgfqpoint{2.266898in}{0.907400in}}%
\pgfpathlineto{\pgfqpoint{2.271635in}{0.896944in}}%
\pgfpathlineto{\pgfqpoint{2.272819in}{0.897899in}}%
\pgfpathlineto{\pgfqpoint{2.277161in}{0.902099in}}%
\pgfpathlineto{\pgfqpoint{2.279135in}{0.902120in}}%
\pgfpathlineto{\pgfqpoint{2.285451in}{0.900887in}}%
\pgfpathlineto{\pgfqpoint{2.289004in}{0.900737in}}%
\pgfpathlineto{\pgfqpoint{2.294530in}{0.891811in}}%
\pgfpathlineto{\pgfqpoint{2.297293in}{0.887321in}}%
\pgfpathlineto{\pgfqpoint{2.300057in}{0.885721in}}%
\pgfpathlineto{\pgfqpoint{2.301636in}{0.885544in}}%
\pgfpathlineto{\pgfqpoint{2.303215in}{0.887012in}}%
\pgfpathlineto{\pgfqpoint{2.303609in}{0.886724in}}%
\pgfpathlineto{\pgfqpoint{2.306767in}{0.884442in}}%
\pgfpathlineto{\pgfqpoint{2.307557in}{0.885122in}}%
\pgfpathlineto{\pgfqpoint{2.318215in}{0.913663in}}%
\pgfpathlineto{\pgfqpoint{2.321373in}{0.917386in}}%
\pgfpathlineto{\pgfqpoint{2.321768in}{0.916787in}}%
\pgfpathlineto{\pgfqpoint{2.324531in}{0.910729in}}%
\pgfpathlineto{\pgfqpoint{2.325715in}{0.909200in}}%
\pgfpathlineto{\pgfqpoint{2.331242in}{0.904962in}}%
\pgfpathlineto{\pgfqpoint{2.335189in}{0.909977in}}%
\pgfpathlineto{\pgfqpoint{2.341111in}{0.929843in}}%
\pgfpathlineto{\pgfqpoint{2.342690in}{0.931457in}}%
\pgfpathlineto{\pgfqpoint{2.343084in}{0.931060in}}%
\pgfpathlineto{\pgfqpoint{2.343874in}{0.930587in}}%
\pgfpathlineto{\pgfqpoint{2.344269in}{0.931788in}}%
\pgfpathlineto{\pgfqpoint{2.347032in}{0.935402in}}%
\pgfpathlineto{\pgfqpoint{2.347821in}{0.934429in}}%
\pgfpathlineto{\pgfqpoint{2.348611in}{0.935809in}}%
\pgfpathlineto{\pgfqpoint{2.350979in}{0.936577in}}%
\pgfpathlineto{\pgfqpoint{2.353743in}{0.931677in}}%
\pgfpathlineto{\pgfqpoint{2.361243in}{0.901159in}}%
\pgfpathlineto{\pgfqpoint{2.363611in}{0.899688in}}%
\pgfpathlineto{\pgfqpoint{2.365190in}{0.900991in}}%
\pgfpathlineto{\pgfqpoint{2.367559in}{0.905457in}}%
\pgfpathlineto{\pgfqpoint{2.369138in}{0.904142in}}%
\pgfpathlineto{\pgfqpoint{2.371506in}{0.905362in}}%
\pgfpathlineto{\pgfqpoint{2.374270in}{0.901863in}}%
\pgfpathlineto{\pgfqpoint{2.374664in}{0.902165in}}%
\pgfpathlineto{\pgfqpoint{2.375849in}{0.903007in}}%
\pgfpathlineto{\pgfqpoint{2.376243in}{0.902242in}}%
\pgfpathlineto{\pgfqpoint{2.376638in}{0.902401in}}%
\pgfpathlineto{\pgfqpoint{2.377033in}{0.901238in}}%
\pgfpathlineto{\pgfqpoint{2.380980in}{0.887915in}}%
\pgfpathlineto{\pgfqpoint{2.385717in}{0.874903in}}%
\pgfpathlineto{\pgfqpoint{2.389270in}{0.871270in}}%
\pgfpathlineto{\pgfqpoint{2.390849in}{0.869638in}}%
\pgfpathlineto{\pgfqpoint{2.391244in}{0.869892in}}%
\pgfpathlineto{\pgfqpoint{2.397165in}{0.880663in}}%
\pgfpathlineto{\pgfqpoint{2.397560in}{0.880400in}}%
\pgfpathlineto{\pgfqpoint{2.398744in}{0.881394in}}%
\pgfpathlineto{\pgfqpoint{2.405455in}{0.892264in}}%
\pgfpathlineto{\pgfqpoint{2.415718in}{0.890821in}}%
\pgfpathlineto{\pgfqpoint{2.416902in}{0.889465in}}%
\pgfpathlineto{\pgfqpoint{2.417297in}{0.889812in}}%
\pgfpathlineto{\pgfqpoint{2.417692in}{0.890929in}}%
\pgfpathlineto{\pgfqpoint{2.418876in}{0.890211in}}%
\pgfpathlineto{\pgfqpoint{2.422429in}{0.891663in}}%
\pgfpathlineto{\pgfqpoint{2.422824in}{0.892043in}}%
\pgfpathlineto{\pgfqpoint{2.423218in}{0.891441in}}%
\pgfpathlineto{\pgfqpoint{2.424403in}{0.890011in}}%
\pgfpathlineto{\pgfqpoint{2.424797in}{0.890199in}}%
\pgfpathlineto{\pgfqpoint{2.438219in}{0.912210in}}%
\pgfpathlineto{\pgfqpoint{2.440587in}{0.910506in}}%
\pgfpathlineto{\pgfqpoint{2.453219in}{0.885280in}}%
\pgfpathlineto{\pgfqpoint{2.453614in}{0.885935in}}%
\pgfpathlineto{\pgfqpoint{2.455588in}{0.891293in}}%
\pgfpathlineto{\pgfqpoint{2.455983in}{0.891194in}}%
\pgfpathlineto{\pgfqpoint{2.457167in}{0.892020in}}%
\pgfpathlineto{\pgfqpoint{2.460325in}{0.898238in}}%
\pgfpathlineto{\pgfqpoint{2.463088in}{0.900679in}}%
\pgfpathlineto{\pgfqpoint{2.464667in}{0.901925in}}%
\pgfpathlineto{\pgfqpoint{2.465062in}{0.901379in}}%
\pgfpathlineto{\pgfqpoint{2.465851in}{0.902580in}}%
\pgfpathlineto{\pgfqpoint{2.468614in}{0.905720in}}%
\pgfpathlineto{\pgfqpoint{2.470193in}{0.905900in}}%
\pgfpathlineto{\pgfqpoint{2.477694in}{0.892820in}}%
\pgfpathlineto{\pgfqpoint{2.480457in}{0.888878in}}%
\pgfpathlineto{\pgfqpoint{2.484010in}{0.883300in}}%
\pgfpathlineto{\pgfqpoint{2.485983in}{0.880013in}}%
\pgfpathlineto{\pgfqpoint{2.486773in}{0.878744in}}%
\pgfpathlineto{\pgfqpoint{2.487168in}{0.879999in}}%
\pgfpathlineto{\pgfqpoint{2.490326in}{0.886179in}}%
\pgfpathlineto{\pgfqpoint{2.492694in}{0.887245in}}%
\pgfpathlineto{\pgfqpoint{2.493089in}{0.886667in}}%
\pgfpathlineto{\pgfqpoint{2.496247in}{0.883964in}}%
\pgfpathlineto{\pgfqpoint{2.497036in}{0.884069in}}%
\pgfpathlineto{\pgfqpoint{2.497431in}{0.883407in}}%
\pgfpathlineto{\pgfqpoint{2.500984in}{0.878240in}}%
\pgfpathlineto{\pgfqpoint{2.502958in}{0.876972in}}%
\pgfpathlineto{\pgfqpoint{2.509668in}{0.879507in}}%
\pgfpathlineto{\pgfqpoint{2.514800in}{0.874808in}}%
\pgfpathlineto{\pgfqpoint{2.515195in}{0.875382in}}%
\pgfpathlineto{\pgfqpoint{2.516774in}{0.875362in}}%
\pgfpathlineto{\pgfqpoint{2.519932in}{0.880523in}}%
\pgfpathlineto{\pgfqpoint{2.524669in}{0.882954in}}%
\pgfpathlineto{\pgfqpoint{2.530590in}{0.891388in}}%
\pgfpathlineto{\pgfqpoint{2.532564in}{0.891931in}}%
\pgfpathlineto{\pgfqpoint{2.540459in}{0.893259in}}%
\pgfpathlineto{\pgfqpoint{2.541248in}{0.892442in}}%
\pgfpathlineto{\pgfqpoint{2.541643in}{0.892822in}}%
\pgfpathlineto{\pgfqpoint{2.544012in}{0.896361in}}%
\pgfpathlineto{\pgfqpoint{2.545196in}{0.895634in}}%
\pgfpathlineto{\pgfqpoint{2.546775in}{0.895583in}}%
\pgfpathlineto{\pgfqpoint{2.549143in}{0.897751in}}%
\pgfpathlineto{\pgfqpoint{2.549933in}{0.897127in}}%
\pgfpathlineto{\pgfqpoint{2.550327in}{0.898053in}}%
\pgfpathlineto{\pgfqpoint{2.552696in}{0.902020in}}%
\pgfpathlineto{\pgfqpoint{2.553091in}{0.901536in}}%
\pgfpathlineto{\pgfqpoint{2.553485in}{0.901230in}}%
\pgfpathlineto{\pgfqpoint{2.554275in}{0.902570in}}%
\pgfpathlineto{\pgfqpoint{2.564144in}{0.929164in}}%
\pgfpathlineto{\pgfqpoint{2.565328in}{0.925950in}}%
\pgfpathlineto{\pgfqpoint{2.570460in}{0.916746in}}%
\pgfpathlineto{\pgfqpoint{2.570854in}{0.917094in}}%
\pgfpathlineto{\pgfqpoint{2.574012in}{0.920099in}}%
\pgfpathlineto{\pgfqpoint{2.576776in}{0.921031in}}%
\pgfpathlineto{\pgfqpoint{2.580328in}{0.920945in}}%
\pgfpathlineto{\pgfqpoint{2.587829in}{0.909770in}}%
\pgfpathlineto{\pgfqpoint{2.593750in}{0.895330in}}%
\pgfpathlineto{\pgfqpoint{2.594145in}{0.895582in}}%
\pgfpathlineto{\pgfqpoint{2.600461in}{0.908252in}}%
\pgfpathlineto{\pgfqpoint{2.605198in}{0.916379in}}%
\pgfpathlineto{\pgfqpoint{2.605987in}{0.916681in}}%
\pgfpathlineto{\pgfqpoint{2.609145in}{0.920854in}}%
\pgfpathlineto{\pgfqpoint{2.612303in}{0.921271in}}%
\pgfpathlineto{\pgfqpoint{2.613487in}{0.920452in}}%
\pgfpathlineto{\pgfqpoint{2.617435in}{0.913054in}}%
\pgfpathlineto{\pgfqpoint{2.617830in}{0.913226in}}%
\pgfpathlineto{\pgfqpoint{2.619803in}{0.913761in}}%
\pgfpathlineto{\pgfqpoint{2.621382in}{0.916080in}}%
\pgfpathlineto{\pgfqpoint{2.629277in}{0.934851in}}%
\pgfpathlineto{\pgfqpoint{2.635198in}{0.936986in}}%
\pgfpathlineto{\pgfqpoint{2.640725in}{0.922889in}}%
\pgfpathlineto{\pgfqpoint{2.643093in}{0.919641in}}%
\pgfpathlineto{\pgfqpoint{2.646251in}{0.922212in}}%
\pgfpathlineto{\pgfqpoint{2.650988in}{0.930330in}}%
\pgfpathlineto{\pgfqpoint{2.654146in}{0.928662in}}%
\pgfpathlineto{\pgfqpoint{2.655331in}{0.927193in}}%
\pgfpathlineto{\pgfqpoint{2.655725in}{0.927679in}}%
\pgfpathlineto{\pgfqpoint{2.658489in}{0.929786in}}%
\pgfpathlineto{\pgfqpoint{2.659278in}{0.929026in}}%
\pgfpathlineto{\pgfqpoint{2.659673in}{0.929789in}}%
\pgfpathlineto{\pgfqpoint{2.661647in}{0.931485in}}%
\pgfpathlineto{\pgfqpoint{2.664015in}{0.927371in}}%
\pgfpathlineto{\pgfqpoint{2.666778in}{0.923008in}}%
\pgfpathlineto{\pgfqpoint{2.668752in}{0.920047in}}%
\pgfpathlineto{\pgfqpoint{2.674279in}{0.912762in}}%
\pgfpathlineto{\pgfqpoint{2.677437in}{0.912871in}}%
\pgfpathlineto{\pgfqpoint{2.680989in}{0.906559in}}%
\pgfpathlineto{\pgfqpoint{2.686121in}{0.901965in}}%
\pgfpathlineto{\pgfqpoint{2.687305in}{0.903413in}}%
\pgfpathlineto{\pgfqpoint{2.694806in}{0.911336in}}%
\pgfpathlineto{\pgfqpoint{2.702306in}{0.910807in}}%
\pgfpathlineto{\pgfqpoint{2.705069in}{0.914953in}}%
\pgfpathlineto{\pgfqpoint{2.705859in}{0.913876in}}%
\pgfpathlineto{\pgfqpoint{2.708227in}{0.914029in}}%
\pgfpathlineto{\pgfqpoint{2.710990in}{0.914335in}}%
\pgfpathlineto{\pgfqpoint{2.714148in}{0.912780in}}%
\pgfpathlineto{\pgfqpoint{2.716911in}{0.910183in}}%
\pgfpathlineto{\pgfqpoint{2.719280in}{0.911866in}}%
\pgfpathlineto{\pgfqpoint{2.722438in}{0.914945in}}%
\pgfpathlineto{\pgfqpoint{2.724017in}{0.914960in}}%
\pgfpathlineto{\pgfqpoint{2.726780in}{0.914140in}}%
\pgfpathlineto{\pgfqpoint{2.729938in}{0.916763in}}%
\pgfpathlineto{\pgfqpoint{2.732307in}{0.918243in}}%
\pgfpathlineto{\pgfqpoint{2.732701in}{0.917579in}}%
\pgfpathlineto{\pgfqpoint{2.734280in}{0.918064in}}%
\pgfpathlineto{\pgfqpoint{2.736649in}{0.918693in}}%
\pgfpathlineto{\pgfqpoint{2.742965in}{0.912719in}}%
\pgfpathlineto{\pgfqpoint{2.749676in}{0.916310in}}%
\pgfpathlineto{\pgfqpoint{2.752834in}{0.916496in}}%
\pgfpathlineto{\pgfqpoint{2.760729in}{0.929505in}}%
\pgfpathlineto{\pgfqpoint{2.763887in}{0.928968in}}%
\pgfpathlineto{\pgfqpoint{2.766650in}{0.927167in}}%
\pgfpathlineto{\pgfqpoint{2.770597in}{0.927466in}}%
\pgfpathlineto{\pgfqpoint{2.772176in}{0.928429in}}%
\pgfpathlineto{\pgfqpoint{2.775729in}{0.931462in}}%
\pgfpathlineto{\pgfqpoint{2.779677in}{0.931123in}}%
\pgfpathlineto{\pgfqpoint{2.793098in}{0.935728in}}%
\pgfpathlineto{\pgfqpoint{2.795466in}{0.937909in}}%
\pgfpathlineto{\pgfqpoint{2.795861in}{0.937643in}}%
\pgfpathlineto{\pgfqpoint{2.799809in}{0.936203in}}%
\pgfpathlineto{\pgfqpoint{2.806914in}{0.930201in}}%
\pgfpathlineto{\pgfqpoint{2.807704in}{0.932200in}}%
\pgfpathlineto{\pgfqpoint{2.810467in}{0.935130in}}%
\pgfpathlineto{\pgfqpoint{2.814414in}{0.943329in}}%
\pgfpathlineto{\pgfqpoint{2.815993in}{0.942680in}}%
\pgfpathlineto{\pgfqpoint{2.816783in}{0.944521in}}%
\pgfpathlineto{\pgfqpoint{2.817572in}{0.943283in}}%
\pgfpathlineto{\pgfqpoint{2.822309in}{0.936702in}}%
\pgfpathlineto{\pgfqpoint{2.828625in}{0.940862in}}%
\pgfpathlineto{\pgfqpoint{2.830599in}{0.941473in}}%
\pgfpathlineto{\pgfqpoint{2.834152in}{0.944991in}}%
\pgfpathlineto{\pgfqpoint{2.835336in}{0.944162in}}%
\pgfpathlineto{\pgfqpoint{2.835731in}{0.944473in}}%
\pgfpathlineto{\pgfqpoint{2.839678in}{0.950164in}}%
\pgfpathlineto{\pgfqpoint{2.840863in}{0.948261in}}%
\pgfpathlineto{\pgfqpoint{2.843626in}{0.946518in}}%
\pgfpathlineto{\pgfqpoint{2.845994in}{0.944086in}}%
\pgfpathlineto{\pgfqpoint{2.849942in}{0.937434in}}%
\pgfpathlineto{\pgfqpoint{2.851126in}{0.938599in}}%
\pgfpathlineto{\pgfqpoint{2.852705in}{0.941759in}}%
\pgfpathlineto{\pgfqpoint{2.853889in}{0.945678in}}%
\pgfpathlineto{\pgfqpoint{2.855074in}{0.945098in}}%
\pgfpathlineto{\pgfqpoint{2.858626in}{0.943220in}}%
\pgfpathlineto{\pgfqpoint{2.863758in}{0.939890in}}%
\pgfpathlineto{\pgfqpoint{2.866127in}{0.937766in}}%
\pgfpathlineto{\pgfqpoint{2.869285in}{0.936140in}}%
\pgfpathlineto{\pgfqpoint{2.870469in}{0.937524in}}%
\pgfpathlineto{\pgfqpoint{2.873232in}{0.940548in}}%
\pgfpathlineto{\pgfqpoint{2.875995in}{0.941747in}}%
\pgfpathlineto{\pgfqpoint{2.876390in}{0.940743in}}%
\pgfpathlineto{\pgfqpoint{2.878364in}{0.938321in}}%
\pgfpathlineto{\pgfqpoint{2.880732in}{0.935577in}}%
\pgfpathlineto{\pgfqpoint{2.883890in}{0.930824in}}%
\pgfpathlineto{\pgfqpoint{2.887443in}{0.928671in}}%
\pgfpathlineto{\pgfqpoint{2.890996in}{0.935089in}}%
\pgfpathlineto{\pgfqpoint{2.892969in}{0.934973in}}%
\pgfpathlineto{\pgfqpoint{2.894548in}{0.935784in}}%
\pgfpathlineto{\pgfqpoint{2.899285in}{0.940208in}}%
\pgfpathlineto{\pgfqpoint{2.900470in}{0.940773in}}%
\pgfpathlineto{\pgfqpoint{2.903233in}{0.936842in}}%
\pgfpathlineto{\pgfqpoint{2.910338in}{0.936095in}}%
\pgfpathlineto{\pgfqpoint{2.911917in}{0.937121in}}%
\pgfpathlineto{\pgfqpoint{2.913496in}{0.937747in}}%
\pgfpathlineto{\pgfqpoint{2.913891in}{0.937417in}}%
\pgfpathlineto{\pgfqpoint{2.921391in}{0.927854in}}%
\pgfpathlineto{\pgfqpoint{2.930865in}{0.899415in}}%
\pgfpathlineto{\pgfqpoint{2.931655in}{0.901457in}}%
\pgfpathlineto{\pgfqpoint{2.932444in}{0.899949in}}%
\pgfpathlineto{\pgfqpoint{2.934023in}{0.899980in}}%
\pgfpathlineto{\pgfqpoint{2.936787in}{0.901511in}}%
\pgfpathlineto{\pgfqpoint{2.937181in}{0.900739in}}%
\pgfpathlineto{\pgfqpoint{2.938760in}{0.899688in}}%
\pgfpathlineto{\pgfqpoint{2.939155in}{0.899952in}}%
\pgfpathlineto{\pgfqpoint{2.955735in}{0.921856in}}%
\pgfpathlineto{\pgfqpoint{2.960866in}{0.927922in}}%
\pgfpathlineto{\pgfqpoint{2.964024in}{0.925464in}}%
\pgfpathlineto{\pgfqpoint{2.966787in}{0.925133in}}%
\pgfpathlineto{\pgfqpoint{2.969551in}{0.928204in}}%
\pgfpathlineto{\pgfqpoint{2.974288in}{0.934825in}}%
\pgfpathlineto{\pgfqpoint{2.982183in}{0.935539in}}%
\pgfpathlineto{\pgfqpoint{2.986920in}{0.931799in}}%
\pgfpathlineto{\pgfqpoint{2.989683in}{0.927255in}}%
\pgfpathlineto{\pgfqpoint{2.991262in}{0.925664in}}%
\pgfpathlineto{\pgfqpoint{2.991657in}{0.926988in}}%
\pgfpathlineto{\pgfqpoint{2.997183in}{0.944502in}}%
\pgfpathlineto{\pgfqpoint{3.002710in}{0.957216in}}%
\pgfpathlineto{\pgfqpoint{3.003104in}{0.957072in}}%
\pgfpathlineto{\pgfqpoint{3.018894in}{0.911722in}}%
\pgfpathlineto{\pgfqpoint{3.020079in}{0.913030in}}%
\pgfpathlineto{\pgfqpoint{3.020473in}{0.912728in}}%
\pgfpathlineto{\pgfqpoint{3.023631in}{0.907030in}}%
\pgfpathlineto{\pgfqpoint{3.024816in}{0.906229in}}%
\pgfpathlineto{\pgfqpoint{3.026000in}{0.905086in}}%
\pgfpathlineto{\pgfqpoint{3.026395in}{0.905339in}}%
\pgfpathlineto{\pgfqpoint{3.028368in}{0.907607in}}%
\pgfpathlineto{\pgfqpoint{3.028763in}{0.907394in}}%
\pgfpathlineto{\pgfqpoint{3.031132in}{0.909723in}}%
\pgfpathlineto{\pgfqpoint{3.033500in}{0.911774in}}%
\pgfpathlineto{\pgfqpoint{3.037842in}{0.906385in}}%
\pgfpathlineto{\pgfqpoint{3.038237in}{0.906798in}}%
\pgfpathlineto{\pgfqpoint{3.044553in}{0.909957in}}%
\pgfpathlineto{\pgfqpoint{3.046921in}{0.909294in}}%
\pgfpathlineto{\pgfqpoint{3.051264in}{0.904964in}}%
\pgfpathlineto{\pgfqpoint{3.051658in}{0.905795in}}%
\pgfpathlineto{\pgfqpoint{3.054816in}{0.912770in}}%
\pgfpathlineto{\pgfqpoint{3.057185in}{0.916512in}}%
\pgfpathlineto{\pgfqpoint{3.058764in}{0.917722in}}%
\pgfpathlineto{\pgfqpoint{3.062317in}{0.922090in}}%
\pgfpathlineto{\pgfqpoint{3.066659in}{0.930128in}}%
\pgfpathlineto{\pgfqpoint{3.070212in}{0.938510in}}%
\pgfpathlineto{\pgfqpoint{3.080080in}{0.951749in}}%
\pgfpathlineto{\pgfqpoint{3.084423in}{0.952001in}}%
\pgfpathlineto{\pgfqpoint{3.088765in}{0.947476in}}%
\pgfpathlineto{\pgfqpoint{3.091133in}{0.945519in}}%
\pgfpathlineto{\pgfqpoint{3.093897in}{0.944229in}}%
\pgfpathlineto{\pgfqpoint{3.100607in}{0.936661in}}%
\pgfpathlineto{\pgfqpoint{3.106529in}{0.930745in}}%
\pgfpathlineto{\pgfqpoint{3.109687in}{0.927693in}}%
\pgfpathlineto{\pgfqpoint{3.117582in}{0.918022in}}%
\pgfpathlineto{\pgfqpoint{3.120740in}{0.918050in}}%
\pgfpathlineto{\pgfqpoint{3.129424in}{0.919402in}}%
\pgfpathlineto{\pgfqpoint{3.132187in}{0.920592in}}%
\pgfpathlineto{\pgfqpoint{3.134556in}{0.921170in}}%
\pgfpathlineto{\pgfqpoint{3.137319in}{0.922249in}}%
\pgfpathlineto{\pgfqpoint{3.142451in}{0.913610in}}%
\pgfpathlineto{\pgfqpoint{3.145214in}{0.913179in}}%
\pgfpathlineto{\pgfqpoint{3.148372in}{0.917127in}}%
\pgfpathlineto{\pgfqpoint{3.157451in}{0.934025in}}%
\pgfpathlineto{\pgfqpoint{3.160609in}{0.930545in}}%
\pgfpathlineto{\pgfqpoint{3.164557in}{0.928174in}}%
\pgfpathlineto{\pgfqpoint{3.165741in}{0.926652in}}%
\pgfpathlineto{\pgfqpoint{3.167320in}{0.924388in}}%
\pgfpathlineto{\pgfqpoint{3.174820in}{0.906226in}}%
\pgfpathlineto{\pgfqpoint{3.176399in}{0.906039in}}%
\pgfpathlineto{\pgfqpoint{3.179162in}{0.903797in}}%
\pgfpathlineto{\pgfqpoint{3.180347in}{0.902620in}}%
\pgfpathlineto{\pgfqpoint{3.184294in}{0.900289in}}%
\pgfpathlineto{\pgfqpoint{3.187057in}{0.903080in}}%
\pgfpathlineto{\pgfqpoint{3.189426in}{0.903743in}}%
\pgfpathlineto{\pgfqpoint{3.194952in}{0.900838in}}%
\pgfpathlineto{\pgfqpoint{3.197321in}{0.895721in}}%
\pgfpathlineto{\pgfqpoint{3.197321in}{0.895721in}}%
\pgfusepath{stroke}%
\end{pgfscope}%
\begin{pgfscope}%
\pgfpathrectangle{\pgfqpoint{0.608025in}{0.484444in}}{\pgfqpoint{2.712595in}{1.541287in}}%
\pgfusepath{clip}%
\pgfsetbuttcap%
\pgfsetmiterjoin%
\definecolor{currentfill}{rgb}{0.580392,0.403922,0.741176}%
\pgfsetfillcolor{currentfill}%
\pgfsetlinewidth{1.003750pt}%
\definecolor{currentstroke}{rgb}{0.580392,0.403922,0.741176}%
\pgfsetstrokecolor{currentstroke}%
\pgfsetdash{}{0pt}%
\pgfsys@defobject{currentmarker}{\pgfqpoint{-0.020833in}{-0.020833in}}{\pgfqpoint{0.020833in}{0.020833in}}{%
\pgfpathmoveto{\pgfqpoint{-0.020833in}{0.000000in}}%
\pgfpathlineto{\pgfqpoint{0.020833in}{-0.020833in}}%
\pgfpathlineto{\pgfqpoint{0.020833in}{0.020833in}}%
\pgfpathlineto{\pgfqpoint{-0.020833in}{0.000000in}}%
\pgfpathclose%
\pgfusepath{stroke,fill}%
}%
\begin{pgfscope}%
\pgfsys@transformshift{0.747115in}{1.942487in}%
\pgfsys@useobject{currentmarker}{}%
\end{pgfscope}%
\begin{pgfscope}%
\pgfsys@transformshift{0.826065in}{1.653225in}%
\pgfsys@useobject{currentmarker}{}%
\end{pgfscope}%
\begin{pgfscope}%
\pgfsys@transformshift{0.905014in}{1.379906in}%
\pgfsys@useobject{currentmarker}{}%
\end{pgfscope}%
\begin{pgfscope}%
\pgfsys@transformshift{0.983964in}{1.176832in}%
\pgfsys@useobject{currentmarker}{}%
\end{pgfscope}%
\begin{pgfscope}%
\pgfsys@transformshift{1.062914in}{1.054336in}%
\pgfsys@useobject{currentmarker}{}%
\end{pgfscope}%
\begin{pgfscope}%
\pgfsys@transformshift{1.141864in}{1.001695in}%
\pgfsys@useobject{currentmarker}{}%
\end{pgfscope}%
\begin{pgfscope}%
\pgfsys@transformshift{1.220813in}{0.963678in}%
\pgfsys@useobject{currentmarker}{}%
\end{pgfscope}%
\begin{pgfscope}%
\pgfsys@transformshift{1.299763in}{0.926984in}%
\pgfsys@useobject{currentmarker}{}%
\end{pgfscope}%
\begin{pgfscope}%
\pgfsys@transformshift{1.378713in}{0.861347in}%
\pgfsys@useobject{currentmarker}{}%
\end{pgfscope}%
\begin{pgfscope}%
\pgfsys@transformshift{1.457663in}{0.886802in}%
\pgfsys@useobject{currentmarker}{}%
\end{pgfscope}%
\begin{pgfscope}%
\pgfsys@transformshift{1.536613in}{0.910457in}%
\pgfsys@useobject{currentmarker}{}%
\end{pgfscope}%
\begin{pgfscope}%
\pgfsys@transformshift{1.615562in}{0.926396in}%
\pgfsys@useobject{currentmarker}{}%
\end{pgfscope}%
\begin{pgfscope}%
\pgfsys@transformshift{1.694512in}{0.929012in}%
\pgfsys@useobject{currentmarker}{}%
\end{pgfscope}%
\begin{pgfscope}%
\pgfsys@transformshift{1.773462in}{0.927511in}%
\pgfsys@useobject{currentmarker}{}%
\end{pgfscope}%
\begin{pgfscope}%
\pgfsys@transformshift{1.852412in}{0.924246in}%
\pgfsys@useobject{currentmarker}{}%
\end{pgfscope}%
\begin{pgfscope}%
\pgfsys@transformshift{1.931361in}{0.945577in}%
\pgfsys@useobject{currentmarker}{}%
\end{pgfscope}%
\begin{pgfscope}%
\pgfsys@transformshift{2.010311in}{0.964053in}%
\pgfsys@useobject{currentmarker}{}%
\end{pgfscope}%
\begin{pgfscope}%
\pgfsys@transformshift{2.089261in}{0.924392in}%
\pgfsys@useobject{currentmarker}{}%
\end{pgfscope}%
\begin{pgfscope}%
\pgfsys@transformshift{2.168211in}{0.900213in}%
\pgfsys@useobject{currentmarker}{}%
\end{pgfscope}%
\begin{pgfscope}%
\pgfsys@transformshift{2.247160in}{0.907358in}%
\pgfsys@useobject{currentmarker}{}%
\end{pgfscope}%
\begin{pgfscope}%
\pgfsys@transformshift{2.326110in}{0.908778in}%
\pgfsys@useobject{currentmarker}{}%
\end{pgfscope}%
\begin{pgfscope}%
\pgfsys@transformshift{2.405060in}{0.892266in}%
\pgfsys@useobject{currentmarker}{}%
\end{pgfscope}%
\begin{pgfscope}%
\pgfsys@transformshift{2.484010in}{0.883300in}%
\pgfsys@useobject{currentmarker}{}%
\end{pgfscope}%
\begin{pgfscope}%
\pgfsys@transformshift{2.562959in}{0.926789in}%
\pgfsys@useobject{currentmarker}{}%
\end{pgfscope}%
\begin{pgfscope}%
\pgfsys@transformshift{2.641909in}{0.920374in}%
\pgfsys@useobject{currentmarker}{}%
\end{pgfscope}%
\begin{pgfscope}%
\pgfsys@transformshift{2.720859in}{0.914172in}%
\pgfsys@useobject{currentmarker}{}%
\end{pgfscope}%
\begin{pgfscope}%
\pgfsys@transformshift{2.799809in}{0.936203in}%
\pgfsys@useobject{currentmarker}{}%
\end{pgfscope}%
\begin{pgfscope}%
\pgfsys@transformshift{2.878759in}{0.938087in}%
\pgfsys@useobject{currentmarker}{}%
\end{pgfscope}%
\begin{pgfscope}%
\pgfsys@transformshift{2.957708in}{0.925200in}%
\pgfsys@useobject{currentmarker}{}%
\end{pgfscope}%
\begin{pgfscope}%
\pgfsys@transformshift{3.036658in}{0.907418in}%
\pgfsys@useobject{currentmarker}{}%
\end{pgfscope}%
\begin{pgfscope}%
\pgfsys@transformshift{3.115608in}{0.919333in}%
\pgfsys@useobject{currentmarker}{}%
\end{pgfscope}%
\begin{pgfscope}%
\pgfsys@transformshift{3.194558in}{0.901086in}%
\pgfsys@useobject{currentmarker}{}%
\end{pgfscope}%
\end{pgfscope}%
\begin{pgfscope}%
\pgfpathrectangle{\pgfqpoint{0.608025in}{0.484444in}}{\pgfqpoint{2.712595in}{1.541287in}}%
\pgfusepath{clip}%
\pgfsetrectcap%
\pgfsetroundjoin%
\pgfsetlinewidth{1.505625pt}%
\definecolor{currentstroke}{rgb}{0.549020,0.337255,0.294118}%
\pgfsetstrokecolor{currentstroke}%
\pgfsetdash{}{0pt}%
\pgfpathmoveto{\pgfqpoint{0.731325in}{1.955542in}}%
\pgfpathlineto{\pgfqpoint{0.734878in}{1.954184in}}%
\pgfpathlineto{\pgfqpoint{0.736062in}{1.954200in}}%
\pgfpathlineto{\pgfqpoint{0.736457in}{1.953872in}}%
\pgfpathlineto{\pgfqpoint{0.738430in}{1.952332in}}%
\pgfpathlineto{\pgfqpoint{0.741588in}{1.949137in}}%
\pgfpathlineto{\pgfqpoint{0.758563in}{1.921031in}}%
\pgfpathlineto{\pgfqpoint{0.770405in}{1.888024in}}%
\pgfpathlineto{\pgfqpoint{0.785800in}{1.828901in}}%
\pgfpathlineto{\pgfqpoint{0.790537in}{1.821369in}}%
\pgfpathlineto{\pgfqpoint{0.797643in}{1.814439in}}%
\pgfpathlineto{\pgfqpoint{0.801195in}{1.802556in}}%
\pgfpathlineto{\pgfqpoint{0.807511in}{1.767330in}}%
\pgfpathlineto{\pgfqpoint{0.816196in}{1.710817in}}%
\pgfpathlineto{\pgfqpoint{0.821722in}{1.690130in}}%
\pgfpathlineto{\pgfqpoint{0.825275in}{1.686801in}}%
\pgfpathlineto{\pgfqpoint{0.829223in}{1.682632in}}%
\pgfpathlineto{\pgfqpoint{0.831591in}{1.673015in}}%
\pgfpathlineto{\pgfqpoint{0.835539in}{1.640567in}}%
\pgfpathlineto{\pgfqpoint{0.841065in}{1.601729in}}%
\pgfpathlineto{\pgfqpoint{0.853697in}{1.549782in}}%
\pgfpathlineto{\pgfqpoint{0.858434in}{1.546398in}}%
\pgfpathlineto{\pgfqpoint{0.860408in}{1.540205in}}%
\pgfpathlineto{\pgfqpoint{0.860803in}{1.543359in}}%
\pgfpathlineto{\pgfqpoint{0.862382in}{1.549953in}}%
\pgfpathlineto{\pgfqpoint{0.863171in}{1.548932in}}%
\pgfpathlineto{\pgfqpoint{0.864750in}{1.538146in}}%
\pgfpathlineto{\pgfqpoint{0.871461in}{1.475576in}}%
\pgfpathlineto{\pgfqpoint{0.876198in}{1.468899in}}%
\pgfpathlineto{\pgfqpoint{0.879750in}{1.457807in}}%
\pgfpathlineto{\pgfqpoint{0.885277in}{1.439873in}}%
\pgfpathlineto{\pgfqpoint{0.890803in}{1.440684in}}%
\pgfpathlineto{\pgfqpoint{0.891198in}{1.439627in}}%
\pgfpathlineto{\pgfqpoint{0.891593in}{1.441578in}}%
\pgfpathlineto{\pgfqpoint{0.894356in}{1.461932in}}%
\pgfpathlineto{\pgfqpoint{0.895146in}{1.461843in}}%
\pgfpathlineto{\pgfqpoint{0.896725in}{1.452866in}}%
\pgfpathlineto{\pgfqpoint{0.899488in}{1.412078in}}%
\pgfpathlineto{\pgfqpoint{0.901462in}{1.386458in}}%
\pgfpathlineto{\pgfqpoint{0.902251in}{1.386568in}}%
\pgfpathlineto{\pgfqpoint{0.904225in}{1.385035in}}%
\pgfpathlineto{\pgfqpoint{0.906593in}{1.374319in}}%
\pgfpathlineto{\pgfqpoint{0.907383in}{1.371482in}}%
\pgfpathlineto{\pgfqpoint{0.908172in}{1.372440in}}%
\pgfpathlineto{\pgfqpoint{0.909357in}{1.371030in}}%
\pgfpathlineto{\pgfqpoint{0.911725in}{1.360157in}}%
\pgfpathlineto{\pgfqpoint{0.915278in}{1.321908in}}%
\pgfpathlineto{\pgfqpoint{0.916067in}{1.316739in}}%
\pgfpathlineto{\pgfqpoint{0.916857in}{1.317529in}}%
\pgfpathlineto{\pgfqpoint{0.920015in}{1.324861in}}%
\pgfpathlineto{\pgfqpoint{0.921989in}{1.333908in}}%
\pgfpathlineto{\pgfqpoint{0.923962in}{1.361597in}}%
\pgfpathlineto{\pgfqpoint{0.924357in}{1.358837in}}%
\pgfpathlineto{\pgfqpoint{0.926331in}{1.354684in}}%
\pgfpathlineto{\pgfqpoint{0.927910in}{1.349914in}}%
\pgfpathlineto{\pgfqpoint{0.930673in}{1.328529in}}%
\pgfpathlineto{\pgfqpoint{0.936989in}{1.278376in}}%
\pgfpathlineto{\pgfqpoint{0.938568in}{1.279320in}}%
\pgfpathlineto{\pgfqpoint{0.938963in}{1.278656in}}%
\pgfpathlineto{\pgfqpoint{0.940147in}{1.275155in}}%
\pgfpathlineto{\pgfqpoint{0.942516in}{1.245262in}}%
\pgfpathlineto{\pgfqpoint{0.946858in}{1.215333in}}%
\pgfpathlineto{\pgfqpoint{0.948437in}{1.216642in}}%
\pgfpathlineto{\pgfqpoint{0.948832in}{1.214776in}}%
\pgfpathlineto{\pgfqpoint{0.950411in}{1.210682in}}%
\pgfpathlineto{\pgfqpoint{0.950805in}{1.210878in}}%
\pgfpathlineto{\pgfqpoint{0.951990in}{1.214815in}}%
\pgfpathlineto{\pgfqpoint{0.953963in}{1.267082in}}%
\pgfpathlineto{\pgfqpoint{0.954753in}{1.259399in}}%
\pgfpathlineto{\pgfqpoint{0.957121in}{1.273552in}}%
\pgfpathlineto{\pgfqpoint{0.957911in}{1.272688in}}%
\pgfpathlineto{\pgfqpoint{0.961069in}{1.259705in}}%
\pgfpathlineto{\pgfqpoint{0.965806in}{1.199180in}}%
\pgfpathlineto{\pgfqpoint{0.968964in}{1.130750in}}%
\pgfpathlineto{\pgfqpoint{0.972516in}{1.107538in}}%
\pgfpathlineto{\pgfqpoint{0.972911in}{1.105997in}}%
\pgfpathlineto{\pgfqpoint{0.973306in}{1.107001in}}%
\pgfpathlineto{\pgfqpoint{0.973701in}{1.110769in}}%
\pgfpathlineto{\pgfqpoint{0.974490in}{1.107709in}}%
\pgfpathlineto{\pgfqpoint{0.975280in}{1.104490in}}%
\pgfpathlineto{\pgfqpoint{0.976069in}{1.106962in}}%
\pgfpathlineto{\pgfqpoint{0.977253in}{1.108059in}}%
\pgfpathlineto{\pgfqpoint{0.977648in}{1.107249in}}%
\pgfpathlineto{\pgfqpoint{0.978438in}{1.104504in}}%
\pgfpathlineto{\pgfqpoint{0.978832in}{1.107757in}}%
\pgfpathlineto{\pgfqpoint{0.979622in}{1.114520in}}%
\pgfpathlineto{\pgfqpoint{0.980411in}{1.108950in}}%
\pgfpathlineto{\pgfqpoint{0.981201in}{1.105247in}}%
\pgfpathlineto{\pgfqpoint{0.981596in}{1.109240in}}%
\pgfpathlineto{\pgfqpoint{0.985543in}{1.160440in}}%
\pgfpathlineto{\pgfqpoint{0.987517in}{1.187059in}}%
\pgfpathlineto{\pgfqpoint{0.988701in}{1.189277in}}%
\pgfpathlineto{\pgfqpoint{0.989096in}{1.187497in}}%
\pgfpathlineto{\pgfqpoint{0.992649in}{1.134293in}}%
\pgfpathlineto{\pgfqpoint{0.994622in}{1.119881in}}%
\pgfpathlineto{\pgfqpoint{0.996596in}{1.148104in}}%
\pgfpathlineto{\pgfqpoint{0.996991in}{1.147123in}}%
\pgfpathlineto{\pgfqpoint{0.998570in}{1.112892in}}%
\pgfpathlineto{\pgfqpoint{0.999359in}{1.118141in}}%
\pgfpathlineto{\pgfqpoint{1.000149in}{1.121670in}}%
\pgfpathlineto{\pgfqpoint{1.000544in}{1.117054in}}%
\pgfpathlineto{\pgfqpoint{1.007254in}{1.043460in}}%
\pgfpathlineto{\pgfqpoint{1.007649in}{1.042140in}}%
\pgfpathlineto{\pgfqpoint{1.008439in}{1.043041in}}%
\pgfpathlineto{\pgfqpoint{1.009623in}{1.046110in}}%
\pgfpathlineto{\pgfqpoint{1.010018in}{1.045557in}}%
\pgfpathlineto{\pgfqpoint{1.012386in}{1.024682in}}%
\pgfpathlineto{\pgfqpoint{1.013176in}{1.031741in}}%
\pgfpathlineto{\pgfqpoint{1.017518in}{1.093187in}}%
\pgfpathlineto{\pgfqpoint{1.022650in}{1.073275in}}%
\pgfpathlineto{\pgfqpoint{1.023044in}{1.077431in}}%
\pgfpathlineto{\pgfqpoint{1.023834in}{1.091480in}}%
\pgfpathlineto{\pgfqpoint{1.024623in}{1.089561in}}%
\pgfpathlineto{\pgfqpoint{1.028176in}{1.055833in}}%
\pgfpathlineto{\pgfqpoint{1.038834in}{0.954814in}}%
\pgfpathlineto{\pgfqpoint{1.039624in}{0.952384in}}%
\pgfpathlineto{\pgfqpoint{1.040019in}{0.955945in}}%
\pgfpathlineto{\pgfqpoint{1.043966in}{1.012181in}}%
\pgfpathlineto{\pgfqpoint{1.047124in}{1.038851in}}%
\pgfpathlineto{\pgfqpoint{1.047519in}{1.038793in}}%
\pgfpathlineto{\pgfqpoint{1.049492in}{1.030518in}}%
\pgfpathlineto{\pgfqpoint{1.052256in}{1.004442in}}%
\pgfpathlineto{\pgfqpoint{1.052650in}{1.011112in}}%
\pgfpathlineto{\pgfqpoint{1.053045in}{1.016593in}}%
\pgfpathlineto{\pgfqpoint{1.054229in}{1.014701in}}%
\pgfpathlineto{\pgfqpoint{1.059361in}{0.998023in}}%
\pgfpathlineto{\pgfqpoint{1.059756in}{0.998305in}}%
\pgfpathlineto{\pgfqpoint{1.064493in}{0.933660in}}%
\pgfpathlineto{\pgfqpoint{1.064888in}{0.930967in}}%
\pgfpathlineto{\pgfqpoint{1.065282in}{0.935859in}}%
\pgfpathlineto{\pgfqpoint{1.066072in}{0.941226in}}%
\pgfpathlineto{\pgfqpoint{1.066861in}{0.938418in}}%
\pgfpathlineto{\pgfqpoint{1.067256in}{0.937073in}}%
\pgfpathlineto{\pgfqpoint{1.067651in}{0.938324in}}%
\pgfpathlineto{\pgfqpoint{1.068835in}{0.941009in}}%
\pgfpathlineto{\pgfqpoint{1.069230in}{0.940246in}}%
\pgfpathlineto{\pgfqpoint{1.071598in}{0.928764in}}%
\pgfpathlineto{\pgfqpoint{1.071993in}{0.930610in}}%
\pgfpathlineto{\pgfqpoint{1.072388in}{0.934183in}}%
\pgfpathlineto{\pgfqpoint{1.073572in}{0.932942in}}%
\pgfpathlineto{\pgfqpoint{1.073967in}{0.931532in}}%
\pgfpathlineto{\pgfqpoint{1.074362in}{0.933752in}}%
\pgfpathlineto{\pgfqpoint{1.075151in}{0.936467in}}%
\pgfpathlineto{\pgfqpoint{1.075546in}{0.931844in}}%
\pgfpathlineto{\pgfqpoint{1.076730in}{0.921902in}}%
\pgfpathlineto{\pgfqpoint{1.077125in}{0.923775in}}%
\pgfpathlineto{\pgfqpoint{1.078309in}{0.938326in}}%
\pgfpathlineto{\pgfqpoint{1.078704in}{0.935202in}}%
\pgfpathlineto{\pgfqpoint{1.080283in}{0.927227in}}%
\pgfpathlineto{\pgfqpoint{1.080678in}{0.929298in}}%
\pgfpathlineto{\pgfqpoint{1.084625in}{0.956968in}}%
\pgfpathlineto{\pgfqpoint{1.085809in}{0.958619in}}%
\pgfpathlineto{\pgfqpoint{1.086994in}{0.962788in}}%
\pgfpathlineto{\pgfqpoint{1.087388in}{0.961738in}}%
\pgfpathlineto{\pgfqpoint{1.088967in}{0.948151in}}%
\pgfpathlineto{\pgfqpoint{1.090152in}{0.950821in}}%
\pgfpathlineto{\pgfqpoint{1.090546in}{0.953405in}}%
\pgfpathlineto{\pgfqpoint{1.091336in}{0.949298in}}%
\pgfpathlineto{\pgfqpoint{1.095678in}{0.868630in}}%
\pgfpathlineto{\pgfqpoint{1.097257in}{0.861559in}}%
\pgfpathlineto{\pgfqpoint{1.097652in}{0.862941in}}%
\pgfpathlineto{\pgfqpoint{1.098441in}{0.864318in}}%
\pgfpathlineto{\pgfqpoint{1.100810in}{0.878110in}}%
\pgfpathlineto{\pgfqpoint{1.102784in}{0.894656in}}%
\pgfpathlineto{\pgfqpoint{1.103178in}{0.892965in}}%
\pgfpathlineto{\pgfqpoint{1.103968in}{0.883609in}}%
\pgfpathlineto{\pgfqpoint{1.104757in}{0.868961in}}%
\pgfpathlineto{\pgfqpoint{1.105547in}{0.875548in}}%
\pgfpathlineto{\pgfqpoint{1.105942in}{0.876264in}}%
\pgfpathlineto{\pgfqpoint{1.106336in}{0.871751in}}%
\pgfpathlineto{\pgfqpoint{1.106731in}{0.877525in}}%
\pgfpathlineto{\pgfqpoint{1.107126in}{0.884503in}}%
\pgfpathlineto{\pgfqpoint{1.108310in}{0.878295in}}%
\pgfpathlineto{\pgfqpoint{1.109100in}{0.889449in}}%
\pgfpathlineto{\pgfqpoint{1.109494in}{0.894444in}}%
\pgfpathlineto{\pgfqpoint{1.110679in}{0.890628in}}%
\pgfpathlineto{\pgfqpoint{1.111468in}{0.894209in}}%
\pgfpathlineto{\pgfqpoint{1.117389in}{0.951533in}}%
\pgfpathlineto{\pgfqpoint{1.117784in}{0.950904in}}%
\pgfpathlineto{\pgfqpoint{1.119363in}{0.935527in}}%
\pgfpathlineto{\pgfqpoint{1.124495in}{0.843525in}}%
\pgfpathlineto{\pgfqpoint{1.126074in}{0.847303in}}%
\pgfpathlineto{\pgfqpoint{1.127258in}{0.851224in}}%
\pgfpathlineto{\pgfqpoint{1.127653in}{0.847977in}}%
\pgfpathlineto{\pgfqpoint{1.129232in}{0.829885in}}%
\pgfpathlineto{\pgfqpoint{1.130021in}{0.835710in}}%
\pgfpathlineto{\pgfqpoint{1.130811in}{0.843744in}}%
\pgfpathlineto{\pgfqpoint{1.131600in}{0.839711in}}%
\pgfpathlineto{\pgfqpoint{1.133179in}{0.835324in}}%
\pgfpathlineto{\pgfqpoint{1.134363in}{0.841209in}}%
\pgfpathlineto{\pgfqpoint{1.134758in}{0.838414in}}%
\pgfpathlineto{\pgfqpoint{1.135548in}{0.834758in}}%
\pgfpathlineto{\pgfqpoint{1.137916in}{0.864326in}}%
\pgfpathlineto{\pgfqpoint{1.138311in}{0.860965in}}%
\pgfpathlineto{\pgfqpoint{1.141469in}{0.839334in}}%
\pgfpathlineto{\pgfqpoint{1.142258in}{0.850608in}}%
\pgfpathlineto{\pgfqpoint{1.142653in}{0.859579in}}%
\pgfpathlineto{\pgfqpoint{1.143443in}{0.852872in}}%
\pgfpathlineto{\pgfqpoint{1.144627in}{0.842929in}}%
\pgfpathlineto{\pgfqpoint{1.145022in}{0.851707in}}%
\pgfpathlineto{\pgfqpoint{1.146601in}{0.874836in}}%
\pgfpathlineto{\pgfqpoint{1.147390in}{0.869001in}}%
\pgfpathlineto{\pgfqpoint{1.150153in}{0.857842in}}%
\pgfpathlineto{\pgfqpoint{1.150943in}{0.858837in}}%
\pgfpathlineto{\pgfqpoint{1.153311in}{0.828371in}}%
\pgfpathlineto{\pgfqpoint{1.153706in}{0.830766in}}%
\pgfpathlineto{\pgfqpoint{1.154496in}{0.835528in}}%
\pgfpathlineto{\pgfqpoint{1.154890in}{0.829703in}}%
\pgfpathlineto{\pgfqpoint{1.155285in}{0.829505in}}%
\pgfpathlineto{\pgfqpoint{1.162391in}{0.761554in}}%
\pgfpathlineto{\pgfqpoint{1.164364in}{0.779375in}}%
\pgfpathlineto{\pgfqpoint{1.166338in}{0.791295in}}%
\pgfpathlineto{\pgfqpoint{1.166733in}{0.791203in}}%
\pgfpathlineto{\pgfqpoint{1.167522in}{0.783018in}}%
\pgfpathlineto{\pgfqpoint{1.168312in}{0.790149in}}%
\pgfpathlineto{\pgfqpoint{1.168707in}{0.787620in}}%
\pgfpathlineto{\pgfqpoint{1.169101in}{0.792121in}}%
\pgfpathlineto{\pgfqpoint{1.169891in}{0.789023in}}%
\pgfpathlineto{\pgfqpoint{1.171075in}{0.783895in}}%
\pgfpathlineto{\pgfqpoint{1.172259in}{0.776528in}}%
\pgfpathlineto{\pgfqpoint{1.173049in}{0.778247in}}%
\pgfpathlineto{\pgfqpoint{1.174233in}{0.785987in}}%
\pgfpathlineto{\pgfqpoint{1.174628in}{0.790102in}}%
\pgfpathlineto{\pgfqpoint{1.175417in}{0.783197in}}%
\pgfpathlineto{\pgfqpoint{1.177391in}{0.796520in}}%
\pgfpathlineto{\pgfqpoint{1.180944in}{0.845272in}}%
\pgfpathlineto{\pgfqpoint{1.182523in}{0.823489in}}%
\pgfpathlineto{\pgfqpoint{1.182918in}{0.814300in}}%
\pgfpathlineto{\pgfqpoint{1.184102in}{0.817577in}}%
\pgfpathlineto{\pgfqpoint{1.184497in}{0.817731in}}%
\pgfpathlineto{\pgfqpoint{1.184891in}{0.816372in}}%
\pgfpathlineto{\pgfqpoint{1.187260in}{0.798474in}}%
\pgfpathlineto{\pgfqpoint{1.191602in}{0.744144in}}%
\pgfpathlineto{\pgfqpoint{1.191997in}{0.742856in}}%
\pgfpathlineto{\pgfqpoint{1.193971in}{0.754131in}}%
\pgfpathlineto{\pgfqpoint{1.195550in}{0.743904in}}%
\pgfpathlineto{\pgfqpoint{1.195944in}{0.747080in}}%
\pgfpathlineto{\pgfqpoint{1.201076in}{0.797730in}}%
\pgfpathlineto{\pgfqpoint{1.201866in}{0.793480in}}%
\pgfpathlineto{\pgfqpoint{1.204629in}{0.737492in}}%
\pgfpathlineto{\pgfqpoint{1.205418in}{0.747334in}}%
\pgfpathlineto{\pgfqpoint{1.205813in}{0.748992in}}%
\pgfpathlineto{\pgfqpoint{1.206208in}{0.744167in}}%
\pgfpathlineto{\pgfqpoint{1.207392in}{0.731201in}}%
\pgfpathlineto{\pgfqpoint{1.207787in}{0.739182in}}%
\pgfpathlineto{\pgfqpoint{1.211734in}{0.805640in}}%
\pgfpathlineto{\pgfqpoint{1.212524in}{0.802852in}}%
\pgfpathlineto{\pgfqpoint{1.214103in}{0.781147in}}%
\pgfpathlineto{\pgfqpoint{1.214497in}{0.776371in}}%
\pgfpathlineto{\pgfqpoint{1.215287in}{0.782414in}}%
\pgfpathlineto{\pgfqpoint{1.216076in}{0.785724in}}%
\pgfpathlineto{\pgfqpoint{1.216471in}{0.780700in}}%
\pgfpathlineto{\pgfqpoint{1.220419in}{0.700726in}}%
\pgfpathlineto{\pgfqpoint{1.221603in}{0.704832in}}%
\pgfpathlineto{\pgfqpoint{1.221998in}{0.704585in}}%
\pgfpathlineto{\pgfqpoint{1.225945in}{0.761969in}}%
\pgfpathlineto{\pgfqpoint{1.227919in}{0.751792in}}%
\pgfpathlineto{\pgfqpoint{1.230682in}{0.757883in}}%
\pgfpathlineto{\pgfqpoint{1.232656in}{0.775346in}}%
\pgfpathlineto{\pgfqpoint{1.233445in}{0.771040in}}%
\pgfpathlineto{\pgfqpoint{1.234630in}{0.760614in}}%
\pgfpathlineto{\pgfqpoint{1.236998in}{0.716533in}}%
\pgfpathlineto{\pgfqpoint{1.237393in}{0.716737in}}%
\pgfpathlineto{\pgfqpoint{1.239761in}{0.747557in}}%
\pgfpathlineto{\pgfqpoint{1.240156in}{0.745075in}}%
\pgfpathlineto{\pgfqpoint{1.242525in}{0.737253in}}%
\pgfpathlineto{\pgfqpoint{1.243709in}{0.724225in}}%
\pgfpathlineto{\pgfqpoint{1.247262in}{0.685408in}}%
\pgfpathlineto{\pgfqpoint{1.249630in}{0.736389in}}%
\pgfpathlineto{\pgfqpoint{1.251209in}{0.725825in}}%
\pgfpathlineto{\pgfqpoint{1.253183in}{0.711599in}}%
\pgfpathlineto{\pgfqpoint{1.253578in}{0.713943in}}%
\pgfpathlineto{\pgfqpoint{1.253972in}{0.720593in}}%
\pgfpathlineto{\pgfqpoint{1.255157in}{0.717305in}}%
\pgfpathlineto{\pgfqpoint{1.257525in}{0.706772in}}%
\pgfpathlineto{\pgfqpoint{1.263052in}{0.761636in}}%
\pgfpathlineto{\pgfqpoint{1.263841in}{0.764084in}}%
\pgfpathlineto{\pgfqpoint{1.265025in}{0.762866in}}%
\pgfpathlineto{\pgfqpoint{1.265420in}{0.761016in}}%
\pgfpathlineto{\pgfqpoint{1.265815in}{0.763011in}}%
\pgfpathlineto{\pgfqpoint{1.266999in}{0.768520in}}%
\pgfpathlineto{\pgfqpoint{1.267394in}{0.763166in}}%
\pgfpathlineto{\pgfqpoint{1.269762in}{0.746784in}}%
\pgfpathlineto{\pgfqpoint{1.270947in}{0.760103in}}%
\pgfpathlineto{\pgfqpoint{1.271736in}{0.755071in}}%
\pgfpathlineto{\pgfqpoint{1.280421in}{0.688783in}}%
\pgfpathlineto{\pgfqpoint{1.281210in}{0.693871in}}%
\pgfpathlineto{\pgfqpoint{1.282000in}{0.703624in}}%
\pgfpathlineto{\pgfqpoint{1.282789in}{0.699469in}}%
\pgfpathlineto{\pgfqpoint{1.283579in}{0.695823in}}%
\pgfpathlineto{\pgfqpoint{1.284763in}{0.697830in}}%
\pgfpathlineto{\pgfqpoint{1.285158in}{0.697389in}}%
\pgfpathlineto{\pgfqpoint{1.286342in}{0.691709in}}%
\pgfpathlineto{\pgfqpoint{1.286737in}{0.693801in}}%
\pgfpathlineto{\pgfqpoint{1.287131in}{0.694657in}}%
\pgfpathlineto{\pgfqpoint{1.287526in}{0.692579in}}%
\pgfpathlineto{\pgfqpoint{1.287921in}{0.690988in}}%
\pgfpathlineto{\pgfqpoint{1.288316in}{0.692935in}}%
\pgfpathlineto{\pgfqpoint{1.292658in}{0.727485in}}%
\pgfpathlineto{\pgfqpoint{1.293052in}{0.725512in}}%
\pgfpathlineto{\pgfqpoint{1.294631in}{0.717107in}}%
\pgfpathlineto{\pgfqpoint{1.296210in}{0.730732in}}%
\pgfpathlineto{\pgfqpoint{1.296605in}{0.729695in}}%
\pgfpathlineto{\pgfqpoint{1.297395in}{0.722487in}}%
\pgfpathlineto{\pgfqpoint{1.300158in}{0.700135in}}%
\pgfpathlineto{\pgfqpoint{1.303711in}{0.691187in}}%
\pgfpathlineto{\pgfqpoint{1.307263in}{0.703782in}}%
\pgfpathlineto{\pgfqpoint{1.308053in}{0.700035in}}%
\pgfpathlineto{\pgfqpoint{1.312000in}{0.677560in}}%
\pgfpathlineto{\pgfqpoint{1.312790in}{0.674177in}}%
\pgfpathlineto{\pgfqpoint{1.313185in}{0.675121in}}%
\pgfpathlineto{\pgfqpoint{1.315948in}{0.692531in}}%
\pgfpathlineto{\pgfqpoint{1.316737in}{0.688697in}}%
\pgfpathlineto{\pgfqpoint{1.317527in}{0.692347in}}%
\pgfpathlineto{\pgfqpoint{1.317922in}{0.691619in}}%
\pgfpathlineto{\pgfqpoint{1.318316in}{0.687562in}}%
\pgfpathlineto{\pgfqpoint{1.319106in}{0.692987in}}%
\pgfpathlineto{\pgfqpoint{1.326606in}{0.731238in}}%
\pgfpathlineto{\pgfqpoint{1.327396in}{0.726548in}}%
\pgfpathlineto{\pgfqpoint{1.334896in}{0.688422in}}%
\pgfpathlineto{\pgfqpoint{1.336475in}{0.677633in}}%
\pgfpathlineto{\pgfqpoint{1.336870in}{0.679870in}}%
\pgfpathlineto{\pgfqpoint{1.337264in}{0.681260in}}%
\pgfpathlineto{\pgfqpoint{1.337659in}{0.679092in}}%
\pgfpathlineto{\pgfqpoint{1.338054in}{0.679013in}}%
\pgfpathlineto{\pgfqpoint{1.339633in}{0.665975in}}%
\pgfpathlineto{\pgfqpoint{1.340028in}{0.666809in}}%
\pgfpathlineto{\pgfqpoint{1.342396in}{0.679875in}}%
\pgfpathlineto{\pgfqpoint{1.343186in}{0.671997in}}%
\pgfpathlineto{\pgfqpoint{1.343975in}{0.664591in}}%
\pgfpathlineto{\pgfqpoint{1.345159in}{0.665740in}}%
\pgfpathlineto{\pgfqpoint{1.345554in}{0.665569in}}%
\pgfpathlineto{\pgfqpoint{1.345949in}{0.666636in}}%
\pgfpathlineto{\pgfqpoint{1.348317in}{0.681008in}}%
\pgfpathlineto{\pgfqpoint{1.348712in}{0.677912in}}%
\pgfpathlineto{\pgfqpoint{1.349107in}{0.673098in}}%
\pgfpathlineto{\pgfqpoint{1.349896in}{0.679178in}}%
\pgfpathlineto{\pgfqpoint{1.352660in}{0.694210in}}%
\pgfpathlineto{\pgfqpoint{1.353449in}{0.694614in}}%
\pgfpathlineto{\pgfqpoint{1.353844in}{0.694246in}}%
\pgfpathlineto{\pgfqpoint{1.354633in}{0.695049in}}%
\pgfpathlineto{\pgfqpoint{1.355818in}{0.683755in}}%
\pgfpathlineto{\pgfqpoint{1.359765in}{0.663253in}}%
\pgfpathlineto{\pgfqpoint{1.360160in}{0.665409in}}%
\pgfpathlineto{\pgfqpoint{1.360949in}{0.667721in}}%
\pgfpathlineto{\pgfqpoint{1.361344in}{0.666067in}}%
\pgfpathlineto{\pgfqpoint{1.362528in}{0.653158in}}%
\pgfpathlineto{\pgfqpoint{1.363318in}{0.656887in}}%
\pgfpathlineto{\pgfqpoint{1.365686in}{0.664411in}}%
\pgfpathlineto{\pgfqpoint{1.366081in}{0.662584in}}%
\pgfpathlineto{\pgfqpoint{1.366871in}{0.664347in}}%
\pgfpathlineto{\pgfqpoint{1.371213in}{0.685678in}}%
\pgfpathlineto{\pgfqpoint{1.372002in}{0.682994in}}%
\pgfpathlineto{\pgfqpoint{1.373976in}{0.672066in}}%
\pgfpathlineto{\pgfqpoint{1.374766in}{0.675257in}}%
\pgfpathlineto{\pgfqpoint{1.375950in}{0.683911in}}%
\pgfpathlineto{\pgfqpoint{1.376739in}{0.679820in}}%
\pgfpathlineto{\pgfqpoint{1.379108in}{0.671567in}}%
\pgfpathlineto{\pgfqpoint{1.380292in}{0.677361in}}%
\pgfpathlineto{\pgfqpoint{1.380687in}{0.675787in}}%
\pgfpathlineto{\pgfqpoint{1.386608in}{0.649434in}}%
\pgfpathlineto{\pgfqpoint{1.387003in}{0.649623in}}%
\pgfpathlineto{\pgfqpoint{1.390555in}{0.627031in}}%
\pgfpathlineto{\pgfqpoint{1.391345in}{0.631298in}}%
\pgfpathlineto{\pgfqpoint{1.394503in}{0.638034in}}%
\pgfpathlineto{\pgfqpoint{1.395292in}{0.630178in}}%
\pgfpathlineto{\pgfqpoint{1.396082in}{0.634574in}}%
\pgfpathlineto{\pgfqpoint{1.396477in}{0.635891in}}%
\pgfpathlineto{\pgfqpoint{1.397266in}{0.633840in}}%
\pgfpathlineto{\pgfqpoint{1.397661in}{0.634254in}}%
\pgfpathlineto{\pgfqpoint{1.404766in}{0.685266in}}%
\pgfpathlineto{\pgfqpoint{1.406345in}{0.677800in}}%
\pgfpathlineto{\pgfqpoint{1.406740in}{0.681923in}}%
\pgfpathlineto{\pgfqpoint{1.407530in}{0.689909in}}%
\pgfpathlineto{\pgfqpoint{1.408319in}{0.683653in}}%
\pgfpathlineto{\pgfqpoint{1.413846in}{0.653222in}}%
\pgfpathlineto{\pgfqpoint{1.414635in}{0.656796in}}%
\pgfpathlineto{\pgfqpoint{1.415030in}{0.652608in}}%
\pgfpathlineto{\pgfqpoint{1.416609in}{0.637285in}}%
\pgfpathlineto{\pgfqpoint{1.417398in}{0.644084in}}%
\pgfpathlineto{\pgfqpoint{1.420556in}{0.669407in}}%
\pgfpathlineto{\pgfqpoint{1.420951in}{0.668417in}}%
\pgfpathlineto{\pgfqpoint{1.424504in}{0.648736in}}%
\pgfpathlineto{\pgfqpoint{1.426478in}{0.640289in}}%
\pgfpathlineto{\pgfqpoint{1.426872in}{0.640862in}}%
\pgfpathlineto{\pgfqpoint{1.428846in}{0.639728in}}%
\pgfpathlineto{\pgfqpoint{1.429636in}{0.641509in}}%
\pgfpathlineto{\pgfqpoint{1.431215in}{0.655681in}}%
\pgfpathlineto{\pgfqpoint{1.432399in}{0.654533in}}%
\pgfpathlineto{\pgfqpoint{1.432794in}{0.652772in}}%
\pgfpathlineto{\pgfqpoint{1.433188in}{0.655934in}}%
\pgfpathlineto{\pgfqpoint{1.434373in}{0.665075in}}%
\pgfpathlineto{\pgfqpoint{1.435557in}{0.662532in}}%
\pgfpathlineto{\pgfqpoint{1.439110in}{0.653979in}}%
\pgfpathlineto{\pgfqpoint{1.439899in}{0.655792in}}%
\pgfpathlineto{\pgfqpoint{1.441083in}{0.650138in}}%
\pgfpathlineto{\pgfqpoint{1.441873in}{0.651925in}}%
\pgfpathlineto{\pgfqpoint{1.442268in}{0.654555in}}%
\pgfpathlineto{\pgfqpoint{1.443452in}{0.653048in}}%
\pgfpathlineto{\pgfqpoint{1.444241in}{0.652382in}}%
\pgfpathlineto{\pgfqpoint{1.446610in}{0.645550in}}%
\pgfpathlineto{\pgfqpoint{1.447399in}{0.646723in}}%
\pgfpathlineto{\pgfqpoint{1.448584in}{0.650936in}}%
\pgfpathlineto{\pgfqpoint{1.448978in}{0.647674in}}%
\pgfpathlineto{\pgfqpoint{1.449373in}{0.644565in}}%
\pgfpathlineto{\pgfqpoint{1.450163in}{0.647012in}}%
\pgfpathlineto{\pgfqpoint{1.450952in}{0.650996in}}%
\pgfpathlineto{\pgfqpoint{1.451742in}{0.648232in}}%
\pgfpathlineto{\pgfqpoint{1.452531in}{0.645475in}}%
\pgfpathlineto{\pgfqpoint{1.452926in}{0.648147in}}%
\pgfpathlineto{\pgfqpoint{1.453715in}{0.650680in}}%
\pgfpathlineto{\pgfqpoint{1.456873in}{0.625334in}}%
\pgfpathlineto{\pgfqpoint{1.457268in}{0.625652in}}%
\pgfpathlineto{\pgfqpoint{1.458452in}{0.635065in}}%
\pgfpathlineto{\pgfqpoint{1.459242in}{0.639439in}}%
\pgfpathlineto{\pgfqpoint{1.460031in}{0.638508in}}%
\pgfpathlineto{\pgfqpoint{1.460426in}{0.637284in}}%
\pgfpathlineto{\pgfqpoint{1.463189in}{0.650964in}}%
\pgfpathlineto{\pgfqpoint{1.463584in}{0.650706in}}%
\pgfpathlineto{\pgfqpoint{1.464373in}{0.652659in}}%
\pgfpathlineto{\pgfqpoint{1.464768in}{0.649255in}}%
\pgfpathlineto{\pgfqpoint{1.466347in}{0.644114in}}%
\pgfpathlineto{\pgfqpoint{1.467926in}{0.646566in}}%
\pgfpathlineto{\pgfqpoint{1.468716in}{0.646038in}}%
\pgfpathlineto{\pgfqpoint{1.470295in}{0.638707in}}%
\pgfpathlineto{\pgfqpoint{1.471084in}{0.641614in}}%
\pgfpathlineto{\pgfqpoint{1.472268in}{0.644685in}}%
\pgfpathlineto{\pgfqpoint{1.472663in}{0.642491in}}%
\pgfpathlineto{\pgfqpoint{1.473058in}{0.641592in}}%
\pgfpathlineto{\pgfqpoint{1.473453in}{0.641944in}}%
\pgfpathlineto{\pgfqpoint{1.473847in}{0.643918in}}%
\pgfpathlineto{\pgfqpoint{1.474637in}{0.641848in}}%
\pgfpathlineto{\pgfqpoint{1.475426in}{0.632585in}}%
\pgfpathlineto{\pgfqpoint{1.476216in}{0.635803in}}%
\pgfpathlineto{\pgfqpoint{1.478979in}{0.648330in}}%
\pgfpathlineto{\pgfqpoint{1.479769in}{0.652604in}}%
\pgfpathlineto{\pgfqpoint{1.480163in}{0.649276in}}%
\pgfpathlineto{\pgfqpoint{1.482137in}{0.640504in}}%
\pgfpathlineto{\pgfqpoint{1.486085in}{0.665396in}}%
\pgfpathlineto{\pgfqpoint{1.487269in}{0.662503in}}%
\pgfpathlineto{\pgfqpoint{1.487664in}{0.663479in}}%
\pgfpathlineto{\pgfqpoint{1.488058in}{0.663019in}}%
\pgfpathlineto{\pgfqpoint{1.489637in}{0.656657in}}%
\pgfpathlineto{\pgfqpoint{1.490822in}{0.661821in}}%
\pgfpathlineto{\pgfqpoint{1.491216in}{0.659733in}}%
\pgfpathlineto{\pgfqpoint{1.492401in}{0.665345in}}%
\pgfpathlineto{\pgfqpoint{1.493190in}{0.663836in}}%
\pgfpathlineto{\pgfqpoint{1.497138in}{0.637245in}}%
\pgfpathlineto{\pgfqpoint{1.497927in}{0.640441in}}%
\pgfpathlineto{\pgfqpoint{1.500296in}{0.647632in}}%
\pgfpathlineto{\pgfqpoint{1.500690in}{0.648601in}}%
\pgfpathlineto{\pgfqpoint{1.501085in}{0.647159in}}%
\pgfpathlineto{\pgfqpoint{1.501480in}{0.645923in}}%
\pgfpathlineto{\pgfqpoint{1.502269in}{0.647863in}}%
\pgfpathlineto{\pgfqpoint{1.502664in}{0.648491in}}%
\pgfpathlineto{\pgfqpoint{1.503059in}{0.646845in}}%
\pgfpathlineto{\pgfqpoint{1.505427in}{0.647448in}}%
\pgfpathlineto{\pgfqpoint{1.507401in}{0.655001in}}%
\pgfpathlineto{\pgfqpoint{1.510164in}{0.643356in}}%
\pgfpathlineto{\pgfqpoint{1.510559in}{0.644412in}}%
\pgfpathlineto{\pgfqpoint{1.511349in}{0.646069in}}%
\pgfpathlineto{\pgfqpoint{1.511743in}{0.644116in}}%
\pgfpathlineto{\pgfqpoint{1.513717in}{0.641805in}}%
\pgfpathlineto{\pgfqpoint{1.514112in}{0.640760in}}%
\pgfpathlineto{\pgfqpoint{1.514901in}{0.642744in}}%
\pgfpathlineto{\pgfqpoint{1.515296in}{0.642709in}}%
\pgfpathlineto{\pgfqpoint{1.516086in}{0.640730in}}%
\pgfpathlineto{\pgfqpoint{1.516480in}{0.642718in}}%
\pgfpathlineto{\pgfqpoint{1.519244in}{0.655526in}}%
\pgfpathlineto{\pgfqpoint{1.520033in}{0.662371in}}%
\pgfpathlineto{\pgfqpoint{1.520823in}{0.661420in}}%
\pgfpathlineto{\pgfqpoint{1.525165in}{0.650453in}}%
\pgfpathlineto{\pgfqpoint{1.529507in}{0.624632in}}%
\pgfpathlineto{\pgfqpoint{1.529902in}{0.625719in}}%
\pgfpathlineto{\pgfqpoint{1.533060in}{0.652269in}}%
\pgfpathlineto{\pgfqpoint{1.535034in}{0.656978in}}%
\pgfpathlineto{\pgfqpoint{1.535428in}{0.656621in}}%
\pgfpathlineto{\pgfqpoint{1.540560in}{0.627659in}}%
\pgfpathlineto{\pgfqpoint{1.541744in}{0.628955in}}%
\pgfpathlineto{\pgfqpoint{1.544902in}{0.618099in}}%
\pgfpathlineto{\pgfqpoint{1.545692in}{0.621943in}}%
\pgfpathlineto{\pgfqpoint{1.551218in}{0.643609in}}%
\pgfpathlineto{\pgfqpoint{1.553192in}{0.623836in}}%
\pgfpathlineto{\pgfqpoint{1.553981in}{0.625879in}}%
\pgfpathlineto{\pgfqpoint{1.554771in}{0.628062in}}%
\pgfpathlineto{\pgfqpoint{1.557139in}{0.635510in}}%
\pgfpathlineto{\pgfqpoint{1.558718in}{0.642730in}}%
\pgfpathlineto{\pgfqpoint{1.563061in}{0.662250in}}%
\pgfpathlineto{\pgfqpoint{1.565429in}{0.665950in}}%
\pgfpathlineto{\pgfqpoint{1.566219in}{0.664369in}}%
\pgfpathlineto{\pgfqpoint{1.568587in}{0.660163in}}%
\pgfpathlineto{\pgfqpoint{1.570561in}{0.649028in}}%
\pgfpathlineto{\pgfqpoint{1.571745in}{0.647746in}}%
\pgfpathlineto{\pgfqpoint{1.573719in}{0.633538in}}%
\pgfpathlineto{\pgfqpoint{1.575298in}{0.648122in}}%
\pgfpathlineto{\pgfqpoint{1.576877in}{0.646447in}}%
\pgfpathlineto{\pgfqpoint{1.578851in}{0.651938in}}%
\pgfpathlineto{\pgfqpoint{1.582009in}{0.662712in}}%
\pgfpathlineto{\pgfqpoint{1.582798in}{0.661111in}}%
\pgfpathlineto{\pgfqpoint{1.585167in}{0.645951in}}%
\pgfpathlineto{\pgfqpoint{1.585561in}{0.646818in}}%
\pgfpathlineto{\pgfqpoint{1.586746in}{0.648303in}}%
\pgfpathlineto{\pgfqpoint{1.589904in}{0.659937in}}%
\pgfpathlineto{\pgfqpoint{1.591088in}{0.661316in}}%
\pgfpathlineto{\pgfqpoint{1.591483in}{0.660901in}}%
\pgfpathlineto{\pgfqpoint{1.595035in}{0.646789in}}%
\pgfpathlineto{\pgfqpoint{1.595430in}{0.647018in}}%
\pgfpathlineto{\pgfqpoint{1.596220in}{0.643250in}}%
\pgfpathlineto{\pgfqpoint{1.598983in}{0.622786in}}%
\pgfpathlineto{\pgfqpoint{1.599772in}{0.625663in}}%
\pgfpathlineto{\pgfqpoint{1.605299in}{0.649898in}}%
\pgfpathlineto{\pgfqpoint{1.614378in}{0.615499in}}%
\pgfpathlineto{\pgfqpoint{1.615562in}{0.618702in}}%
\pgfpathlineto{\pgfqpoint{1.616352in}{0.620583in}}%
\pgfpathlineto{\pgfqpoint{1.617141in}{0.616105in}}%
\pgfpathlineto{\pgfqpoint{1.617931in}{0.618439in}}%
\pgfpathlineto{\pgfqpoint{1.622668in}{0.623470in}}%
\pgfpathlineto{\pgfqpoint{1.623457in}{0.611487in}}%
\pgfpathlineto{\pgfqpoint{1.624247in}{0.617169in}}%
\pgfpathlineto{\pgfqpoint{1.626221in}{0.633520in}}%
\pgfpathlineto{\pgfqpoint{1.627405in}{0.632748in}}%
\pgfpathlineto{\pgfqpoint{1.628589in}{0.631104in}}%
\pgfpathlineto{\pgfqpoint{1.630168in}{0.636302in}}%
\pgfpathlineto{\pgfqpoint{1.630563in}{0.635328in}}%
\pgfpathlineto{\pgfqpoint{1.634905in}{0.623350in}}%
\pgfpathlineto{\pgfqpoint{1.637273in}{0.615186in}}%
\pgfpathlineto{\pgfqpoint{1.639247in}{0.624859in}}%
\pgfpathlineto{\pgfqpoint{1.639642in}{0.624564in}}%
\pgfpathlineto{\pgfqpoint{1.641616in}{0.628161in}}%
\pgfpathlineto{\pgfqpoint{1.642405in}{0.627633in}}%
\pgfpathlineto{\pgfqpoint{1.642800in}{0.627124in}}%
\pgfpathlineto{\pgfqpoint{1.644774in}{0.617625in}}%
\pgfpathlineto{\pgfqpoint{1.645168in}{0.618424in}}%
\pgfpathlineto{\pgfqpoint{1.647932in}{0.624629in}}%
\pgfpathlineto{\pgfqpoint{1.648721in}{0.614656in}}%
\pgfpathlineto{\pgfqpoint{1.649905in}{0.615749in}}%
\pgfpathlineto{\pgfqpoint{1.650300in}{0.625396in}}%
\pgfpathlineto{\pgfqpoint{1.651484in}{0.620211in}}%
\pgfpathlineto{\pgfqpoint{1.651879in}{0.618317in}}%
\pgfpathlineto{\pgfqpoint{1.653063in}{0.619510in}}%
\pgfpathlineto{\pgfqpoint{1.654642in}{0.614027in}}%
\pgfpathlineto{\pgfqpoint{1.657406in}{0.599264in}}%
\pgfpathlineto{\pgfqpoint{1.662143in}{0.630138in}}%
\pgfpathlineto{\pgfqpoint{1.662537in}{0.628785in}}%
\pgfpathlineto{\pgfqpoint{1.662932in}{0.629714in}}%
\pgfpathlineto{\pgfqpoint{1.665301in}{0.641922in}}%
\pgfpathlineto{\pgfqpoint{1.666880in}{0.646954in}}%
\pgfpathlineto{\pgfqpoint{1.667274in}{0.645158in}}%
\pgfpathlineto{\pgfqpoint{1.671617in}{0.620342in}}%
\pgfpathlineto{\pgfqpoint{1.672011in}{0.622035in}}%
\pgfpathlineto{\pgfqpoint{1.672801in}{0.614731in}}%
\pgfpathlineto{\pgfqpoint{1.673590in}{0.616446in}}%
\pgfpathlineto{\pgfqpoint{1.674775in}{0.619226in}}%
\pgfpathlineto{\pgfqpoint{1.676748in}{0.637056in}}%
\pgfpathlineto{\pgfqpoint{1.679117in}{0.633108in}}%
\pgfpathlineto{\pgfqpoint{1.679512in}{0.632675in}}%
\pgfpathlineto{\pgfqpoint{1.679906in}{0.634267in}}%
\pgfpathlineto{\pgfqpoint{1.681880in}{0.639083in}}%
\pgfpathlineto{\pgfqpoint{1.683854in}{0.632494in}}%
\pgfpathlineto{\pgfqpoint{1.685433in}{0.627787in}}%
\pgfpathlineto{\pgfqpoint{1.687012in}{0.636570in}}%
\pgfpathlineto{\pgfqpoint{1.687407in}{0.631889in}}%
\pgfpathlineto{\pgfqpoint{1.687801in}{0.629549in}}%
\pgfpathlineto{\pgfqpoint{1.688591in}{0.633453in}}%
\pgfpathlineto{\pgfqpoint{1.689775in}{0.645513in}}%
\pgfpathlineto{\pgfqpoint{1.690565in}{0.642233in}}%
\pgfpathlineto{\pgfqpoint{1.692144in}{0.638428in}}%
\pgfpathlineto{\pgfqpoint{1.692538in}{0.638738in}}%
\pgfpathlineto{\pgfqpoint{1.693723in}{0.648526in}}%
\pgfpathlineto{\pgfqpoint{1.694512in}{0.643961in}}%
\pgfpathlineto{\pgfqpoint{1.696486in}{0.636419in}}%
\pgfpathlineto{\pgfqpoint{1.695696in}{0.644818in}}%
\pgfpathlineto{\pgfqpoint{1.696881in}{0.639141in}}%
\pgfpathlineto{\pgfqpoint{1.698065in}{0.645098in}}%
\pgfpathlineto{\pgfqpoint{1.698460in}{0.646877in}}%
\pgfpathlineto{\pgfqpoint{1.698854in}{0.642994in}}%
\pgfpathlineto{\pgfqpoint{1.699249in}{0.644532in}}%
\pgfpathlineto{\pgfqpoint{1.699644in}{0.643942in}}%
\pgfpathlineto{\pgfqpoint{1.700433in}{0.653976in}}%
\pgfpathlineto{\pgfqpoint{1.701223in}{0.651968in}}%
\pgfpathlineto{\pgfqpoint{1.703197in}{0.651188in}}%
\pgfpathlineto{\pgfqpoint{1.703591in}{0.651756in}}%
\pgfpathlineto{\pgfqpoint{1.706355in}{0.659358in}}%
\pgfpathlineto{\pgfqpoint{1.707144in}{0.656641in}}%
\pgfpathlineto{\pgfqpoint{1.707934in}{0.657946in}}%
\pgfpathlineto{\pgfqpoint{1.709118in}{0.654733in}}%
\pgfpathlineto{\pgfqpoint{1.710697in}{0.642308in}}%
\pgfpathlineto{\pgfqpoint{1.711486in}{0.632241in}}%
\pgfpathlineto{\pgfqpoint{1.712276in}{0.632982in}}%
\pgfpathlineto{\pgfqpoint{1.715039in}{0.651723in}}%
\pgfpathlineto{\pgfqpoint{1.715828in}{0.653727in}}%
\pgfpathlineto{\pgfqpoint{1.717802in}{0.664779in}}%
\pgfpathlineto{\pgfqpoint{1.718197in}{0.663374in}}%
\pgfpathlineto{\pgfqpoint{1.722539in}{0.646220in}}%
\pgfpathlineto{\pgfqpoint{1.722934in}{0.646949in}}%
\pgfpathlineto{\pgfqpoint{1.723723in}{0.648297in}}%
\pgfpathlineto{\pgfqpoint{1.730039in}{0.663939in}}%
\pgfpathlineto{\pgfqpoint{1.731618in}{0.660643in}}%
\pgfpathlineto{\pgfqpoint{1.732013in}{0.661487in}}%
\pgfpathlineto{\pgfqpoint{1.734776in}{0.665501in}}%
\pgfpathlineto{\pgfqpoint{1.738724in}{0.655686in}}%
\pgfpathlineto{\pgfqpoint{1.739119in}{0.655645in}}%
\pgfpathlineto{\pgfqpoint{1.739513in}{0.657375in}}%
\pgfpathlineto{\pgfqpoint{1.740303in}{0.654122in}}%
\pgfpathlineto{\pgfqpoint{1.743066in}{0.644415in}}%
\pgfpathlineto{\pgfqpoint{1.745829in}{0.637485in}}%
\pgfpathlineto{\pgfqpoint{1.746224in}{0.637999in}}%
\pgfpathlineto{\pgfqpoint{1.747408in}{0.640631in}}%
\pgfpathlineto{\pgfqpoint{1.747803in}{0.640030in}}%
\pgfpathlineto{\pgfqpoint{1.748987in}{0.634981in}}%
\pgfpathlineto{\pgfqpoint{1.749777in}{0.643085in}}%
\pgfpathlineto{\pgfqpoint{1.750566in}{0.641378in}}%
\pgfpathlineto{\pgfqpoint{1.751751in}{0.643218in}}%
\pgfpathlineto{\pgfqpoint{1.752540in}{0.638032in}}%
\pgfpathlineto{\pgfqpoint{1.753330in}{0.639877in}}%
\pgfpathlineto{\pgfqpoint{1.756882in}{0.651069in}}%
\pgfpathlineto{\pgfqpoint{1.757672in}{0.649704in}}%
\pgfpathlineto{\pgfqpoint{1.758461in}{0.650141in}}%
\pgfpathlineto{\pgfqpoint{1.760830in}{0.657588in}}%
\pgfpathlineto{\pgfqpoint{1.761225in}{0.651125in}}%
\pgfpathlineto{\pgfqpoint{1.762409in}{0.652414in}}%
\pgfpathlineto{\pgfqpoint{1.764383in}{0.651167in}}%
\pgfpathlineto{\pgfqpoint{1.765172in}{0.655400in}}%
\pgfpathlineto{\pgfqpoint{1.765962in}{0.651462in}}%
\pgfpathlineto{\pgfqpoint{1.767541in}{0.651057in}}%
\pgfpathlineto{\pgfqpoint{1.768330in}{0.654689in}}%
\pgfpathlineto{\pgfqpoint{1.769120in}{0.652549in}}%
\pgfpathlineto{\pgfqpoint{1.771093in}{0.640789in}}%
\pgfpathlineto{\pgfqpoint{1.773857in}{0.625624in}}%
\pgfpathlineto{\pgfqpoint{1.776225in}{0.644338in}}%
\pgfpathlineto{\pgfqpoint{1.778199in}{0.642076in}}%
\pgfpathlineto{\pgfqpoint{1.778594in}{0.643359in}}%
\pgfpathlineto{\pgfqpoint{1.780567in}{0.648937in}}%
\pgfpathlineto{\pgfqpoint{1.783725in}{0.637387in}}%
\pgfpathlineto{\pgfqpoint{1.785304in}{0.632966in}}%
\pgfpathlineto{\pgfqpoint{1.786094in}{0.620675in}}%
\pgfpathlineto{\pgfqpoint{1.786883in}{0.624049in}}%
\pgfpathlineto{\pgfqpoint{1.790041in}{0.638521in}}%
\pgfpathlineto{\pgfqpoint{1.793594in}{0.653607in}}%
\pgfpathlineto{\pgfqpoint{1.795173in}{0.646368in}}%
\pgfpathlineto{\pgfqpoint{1.800305in}{0.620835in}}%
\pgfpathlineto{\pgfqpoint{1.801094in}{0.621709in}}%
\pgfpathlineto{\pgfqpoint{1.805436in}{0.647574in}}%
\pgfpathlineto{\pgfqpoint{1.806226in}{0.646355in}}%
\pgfpathlineto{\pgfqpoint{1.807015in}{0.647429in}}%
\pgfpathlineto{\pgfqpoint{1.807805in}{0.648250in}}%
\pgfpathlineto{\pgfqpoint{1.808200in}{0.646683in}}%
\pgfpathlineto{\pgfqpoint{1.810963in}{0.633682in}}%
\pgfpathlineto{\pgfqpoint{1.812542in}{0.625223in}}%
\pgfpathlineto{\pgfqpoint{1.812937in}{0.627523in}}%
\pgfpathlineto{\pgfqpoint{1.813331in}{0.629846in}}%
\pgfpathlineto{\pgfqpoint{1.813726in}{0.625895in}}%
\pgfpathlineto{\pgfqpoint{1.814516in}{0.622301in}}%
\pgfpathlineto{\pgfqpoint{1.815305in}{0.623905in}}%
\pgfpathlineto{\pgfqpoint{1.816489in}{0.625641in}}%
\pgfpathlineto{\pgfqpoint{1.818068in}{0.616220in}}%
\pgfpathlineto{\pgfqpoint{1.818858in}{0.616788in}}%
\pgfpathlineto{\pgfqpoint{1.820042in}{0.623764in}}%
\pgfpathlineto{\pgfqpoint{1.821226in}{0.622698in}}%
\pgfpathlineto{\pgfqpoint{1.822411in}{0.623289in}}%
\pgfpathlineto{\pgfqpoint{1.822805in}{0.621793in}}%
\pgfpathlineto{\pgfqpoint{1.823200in}{0.623177in}}%
\pgfpathlineto{\pgfqpoint{1.825569in}{0.637264in}}%
\pgfpathlineto{\pgfqpoint{1.825963in}{0.636742in}}%
\pgfpathlineto{\pgfqpoint{1.826358in}{0.636186in}}%
\pgfpathlineto{\pgfqpoint{1.827148in}{0.637375in}}%
\pgfpathlineto{\pgfqpoint{1.829911in}{0.642845in}}%
\pgfpathlineto{\pgfqpoint{1.830306in}{0.642442in}}%
\pgfpathlineto{\pgfqpoint{1.831885in}{0.641541in}}%
\pgfpathlineto{\pgfqpoint{1.834648in}{0.621022in}}%
\pgfpathlineto{\pgfqpoint{1.836622in}{0.625405in}}%
\pgfpathlineto{\pgfqpoint{1.838595in}{0.629129in}}%
\pgfpathlineto{\pgfqpoint{1.839385in}{0.628452in}}%
\pgfpathlineto{\pgfqpoint{1.840174in}{0.627660in}}%
\pgfpathlineto{\pgfqpoint{1.842543in}{0.620621in}}%
\pgfpathlineto{\pgfqpoint{1.843332in}{0.621801in}}%
\pgfpathlineto{\pgfqpoint{1.844122in}{0.620694in}}%
\pgfpathlineto{\pgfqpoint{1.844517in}{0.620336in}}%
\pgfpathlineto{\pgfqpoint{1.846490in}{0.604785in}}%
\pgfpathlineto{\pgfqpoint{1.847280in}{0.605402in}}%
\pgfpathlineto{\pgfqpoint{1.848859in}{0.602625in}}%
\pgfpathlineto{\pgfqpoint{1.849254in}{0.610310in}}%
\pgfpathlineto{\pgfqpoint{1.850438in}{0.607319in}}%
\pgfpathlineto{\pgfqpoint{1.850833in}{0.608403in}}%
\pgfpathlineto{\pgfqpoint{1.851227in}{0.606094in}}%
\pgfpathlineto{\pgfqpoint{1.852412in}{0.602466in}}%
\pgfpathlineto{\pgfqpoint{1.852806in}{0.604138in}}%
\pgfpathlineto{\pgfqpoint{1.859122in}{0.619228in}}%
\pgfpathlineto{\pgfqpoint{1.859517in}{0.617714in}}%
\pgfpathlineto{\pgfqpoint{1.860307in}{0.606185in}}%
\pgfpathlineto{\pgfqpoint{1.861096in}{0.608062in}}%
\pgfpathlineto{\pgfqpoint{1.862280in}{0.613067in}}%
\pgfpathlineto{\pgfqpoint{1.864254in}{0.619749in}}%
\pgfpathlineto{\pgfqpoint{1.867017in}{0.610601in}}%
\pgfpathlineto{\pgfqpoint{1.867412in}{0.611596in}}%
\pgfpathlineto{\pgfqpoint{1.868202in}{0.616894in}}%
\pgfpathlineto{\pgfqpoint{1.868991in}{0.613204in}}%
\pgfpathlineto{\pgfqpoint{1.869781in}{0.608145in}}%
\pgfpathlineto{\pgfqpoint{1.870570in}{0.609349in}}%
\pgfpathlineto{\pgfqpoint{1.872544in}{0.609570in}}%
\pgfpathlineto{\pgfqpoint{1.872939in}{0.617328in}}%
\pgfpathlineto{\pgfqpoint{1.874123in}{0.616345in}}%
\pgfpathlineto{\pgfqpoint{1.875307in}{0.617247in}}%
\pgfpathlineto{\pgfqpoint{1.876097in}{0.619944in}}%
\pgfpathlineto{\pgfqpoint{1.876491in}{0.617253in}}%
\pgfpathlineto{\pgfqpoint{1.876886in}{0.612893in}}%
\pgfpathlineto{\pgfqpoint{1.878070in}{0.614029in}}%
\pgfpathlineto{\pgfqpoint{1.878860in}{0.621160in}}%
\pgfpathlineto{\pgfqpoint{1.880044in}{0.618405in}}%
\pgfpathlineto{\pgfqpoint{1.885176in}{0.628367in}}%
\pgfpathlineto{\pgfqpoint{1.885570in}{0.627038in}}%
\pgfpathlineto{\pgfqpoint{1.886360in}{0.614067in}}%
\pgfpathlineto{\pgfqpoint{1.887544in}{0.617184in}}%
\pgfpathlineto{\pgfqpoint{1.893071in}{0.633121in}}%
\pgfpathlineto{\pgfqpoint{1.894255in}{0.632163in}}%
\pgfpathlineto{\pgfqpoint{1.896229in}{0.623407in}}%
\pgfpathlineto{\pgfqpoint{1.896623in}{0.624160in}}%
\pgfpathlineto{\pgfqpoint{1.898202in}{0.627577in}}%
\pgfpathlineto{\pgfqpoint{1.898597in}{0.627129in}}%
\pgfpathlineto{\pgfqpoint{1.898992in}{0.626034in}}%
\pgfpathlineto{\pgfqpoint{1.899387in}{0.635472in}}%
\pgfpathlineto{\pgfqpoint{1.900571in}{0.631485in}}%
\pgfpathlineto{\pgfqpoint{1.902939in}{0.629555in}}%
\pgfpathlineto{\pgfqpoint{1.904913in}{0.624303in}}%
\pgfpathlineto{\pgfqpoint{1.905703in}{0.627892in}}%
\pgfpathlineto{\pgfqpoint{1.906492in}{0.626226in}}%
\pgfpathlineto{\pgfqpoint{1.907282in}{0.624721in}}%
\pgfpathlineto{\pgfqpoint{1.908071in}{0.626220in}}%
\pgfpathlineto{\pgfqpoint{1.908466in}{0.629443in}}%
\pgfpathlineto{\pgfqpoint{1.909255in}{0.623309in}}%
\pgfpathlineto{\pgfqpoint{1.911229in}{0.629891in}}%
\pgfpathlineto{\pgfqpoint{1.912413in}{0.628387in}}%
\pgfpathlineto{\pgfqpoint{1.913992in}{0.620567in}}%
\pgfpathlineto{\pgfqpoint{1.914387in}{0.620675in}}%
\pgfpathlineto{\pgfqpoint{1.915966in}{0.628371in}}%
\pgfpathlineto{\pgfqpoint{1.916756in}{0.626179in}}%
\pgfpathlineto{\pgfqpoint{1.921098in}{0.615965in}}%
\pgfpathlineto{\pgfqpoint{1.921493in}{0.617052in}}%
\pgfpathlineto{\pgfqpoint{1.925045in}{0.635810in}}%
\pgfpathlineto{\pgfqpoint{1.925440in}{0.633237in}}%
\pgfpathlineto{\pgfqpoint{1.926230in}{0.631181in}}%
\pgfpathlineto{\pgfqpoint{1.926624in}{0.632131in}}%
\pgfpathlineto{\pgfqpoint{1.928993in}{0.635992in}}%
\pgfpathlineto{\pgfqpoint{1.930177in}{0.635210in}}%
\pgfpathlineto{\pgfqpoint{1.931361in}{0.638289in}}%
\pgfpathlineto{\pgfqpoint{1.931756in}{0.637769in}}%
\pgfpathlineto{\pgfqpoint{1.934125in}{0.633325in}}%
\pgfpathlineto{\pgfqpoint{1.935704in}{0.630287in}}%
\pgfpathlineto{\pgfqpoint{1.936493in}{0.631888in}}%
\pgfpathlineto{\pgfqpoint{1.938862in}{0.633445in}}%
\pgfpathlineto{\pgfqpoint{1.939256in}{0.633086in}}%
\pgfpathlineto{\pgfqpoint{1.939651in}{0.634318in}}%
\pgfpathlineto{\pgfqpoint{1.940046in}{0.634973in}}%
\pgfpathlineto{\pgfqpoint{1.940835in}{0.633810in}}%
\pgfpathlineto{\pgfqpoint{1.941230in}{0.633684in}}%
\pgfpathlineto{\pgfqpoint{1.943993in}{0.625588in}}%
\pgfpathlineto{\pgfqpoint{1.944388in}{0.627304in}}%
\pgfpathlineto{\pgfqpoint{1.945178in}{0.624598in}}%
\pgfpathlineto{\pgfqpoint{1.947151in}{0.621899in}}%
\pgfpathlineto{\pgfqpoint{1.951494in}{0.634543in}}%
\pgfpathlineto{\pgfqpoint{1.951888in}{0.633450in}}%
\pgfpathlineto{\pgfqpoint{1.953073in}{0.634549in}}%
\pgfpathlineto{\pgfqpoint{1.955836in}{0.640710in}}%
\pgfpathlineto{\pgfqpoint{1.958204in}{0.619026in}}%
\pgfpathlineto{\pgfqpoint{1.959783in}{0.621457in}}%
\pgfpathlineto{\pgfqpoint{1.960967in}{0.630756in}}%
\pgfpathlineto{\pgfqpoint{1.965704in}{0.660124in}}%
\pgfpathlineto{\pgfqpoint{1.966889in}{0.661414in}}%
\pgfpathlineto{\pgfqpoint{1.968073in}{0.661534in}}%
\pgfpathlineto{\pgfqpoint{1.969652in}{0.646332in}}%
\pgfpathlineto{\pgfqpoint{1.972020in}{0.647541in}}%
\pgfpathlineto{\pgfqpoint{1.973205in}{0.645348in}}%
\pgfpathlineto{\pgfqpoint{1.973599in}{0.646309in}}%
\pgfpathlineto{\pgfqpoint{1.973994in}{0.648479in}}%
\pgfpathlineto{\pgfqpoint{1.974389in}{0.646184in}}%
\pgfpathlineto{\pgfqpoint{1.976363in}{0.641820in}}%
\pgfpathlineto{\pgfqpoint{1.976757in}{0.641467in}}%
\pgfpathlineto{\pgfqpoint{1.977152in}{0.642580in}}%
\pgfpathlineto{\pgfqpoint{1.979915in}{0.660408in}}%
\pgfpathlineto{\pgfqpoint{1.980705in}{0.665532in}}%
\pgfpathlineto{\pgfqpoint{1.981889in}{0.663509in}}%
\pgfpathlineto{\pgfqpoint{1.983468in}{0.648174in}}%
\pgfpathlineto{\pgfqpoint{1.983863in}{0.648457in}}%
\pgfpathlineto{\pgfqpoint{1.984652in}{0.649688in}}%
\pgfpathlineto{\pgfqpoint{1.985047in}{0.648848in}}%
\pgfpathlineto{\pgfqpoint{1.988205in}{0.636083in}}%
\pgfpathlineto{\pgfqpoint{1.988995in}{0.630763in}}%
\pgfpathlineto{\pgfqpoint{1.989784in}{0.632932in}}%
\pgfpathlineto{\pgfqpoint{1.993337in}{0.645124in}}%
\pgfpathlineto{\pgfqpoint{1.994916in}{0.641055in}}%
\pgfpathlineto{\pgfqpoint{1.999258in}{0.625659in}}%
\pgfpathlineto{\pgfqpoint{2.000048in}{0.633809in}}%
\pgfpathlineto{\pgfqpoint{2.001232in}{0.629824in}}%
\pgfpathlineto{\pgfqpoint{2.002416in}{0.627445in}}%
\pgfpathlineto{\pgfqpoint{2.002811in}{0.629048in}}%
\pgfpathlineto{\pgfqpoint{2.006364in}{0.633808in}}%
\pgfpathlineto{\pgfqpoint{2.007153in}{0.638842in}}%
\pgfpathlineto{\pgfqpoint{2.007548in}{0.637884in}}%
\pgfpathlineto{\pgfqpoint{2.008732in}{0.631136in}}%
\pgfpathlineto{\pgfqpoint{2.009127in}{0.631365in}}%
\pgfpathlineto{\pgfqpoint{2.011101in}{0.637452in}}%
\pgfpathlineto{\pgfqpoint{2.011890in}{0.636620in}}%
\pgfpathlineto{\pgfqpoint{2.014259in}{0.630011in}}%
\pgfpathlineto{\pgfqpoint{2.015048in}{0.632203in}}%
\pgfpathlineto{\pgfqpoint{2.018206in}{0.642891in}}%
\pgfpathlineto{\pgfqpoint{2.020969in}{0.631581in}}%
\pgfpathlineto{\pgfqpoint{2.022154in}{0.632825in}}%
\pgfpathlineto{\pgfqpoint{2.023733in}{0.635184in}}%
\pgfpathlineto{\pgfqpoint{2.024127in}{0.644426in}}%
\pgfpathlineto{\pgfqpoint{2.025312in}{0.642417in}}%
\pgfpathlineto{\pgfqpoint{2.027285in}{0.636690in}}%
\pgfpathlineto{\pgfqpoint{2.027680in}{0.638880in}}%
\pgfpathlineto{\pgfqpoint{2.028075in}{0.638654in}}%
\pgfpathlineto{\pgfqpoint{2.029654in}{0.648715in}}%
\pgfpathlineto{\pgfqpoint{2.030443in}{0.645294in}}%
\pgfpathlineto{\pgfqpoint{2.033996in}{0.636150in}}%
\pgfpathlineto{\pgfqpoint{2.035180in}{0.636972in}}%
\pgfpathlineto{\pgfqpoint{2.035970in}{0.642017in}}%
\pgfpathlineto{\pgfqpoint{2.036759in}{0.639465in}}%
\pgfpathlineto{\pgfqpoint{2.039523in}{0.643596in}}%
\pgfpathlineto{\pgfqpoint{2.041496in}{0.649014in}}%
\pgfpathlineto{\pgfqpoint{2.042286in}{0.648250in}}%
\pgfpathlineto{\pgfqpoint{2.044654in}{0.646395in}}%
\pgfpathlineto{\pgfqpoint{2.047023in}{0.640245in}}%
\pgfpathlineto{\pgfqpoint{2.047417in}{0.639827in}}%
\pgfpathlineto{\pgfqpoint{2.048207in}{0.651245in}}%
\pgfpathlineto{\pgfqpoint{2.049786in}{0.646041in}}%
\pgfpathlineto{\pgfqpoint{2.053733in}{0.639964in}}%
\pgfpathlineto{\pgfqpoint{2.054128in}{0.640683in}}%
\pgfpathlineto{\pgfqpoint{2.054523in}{0.639752in}}%
\pgfpathlineto{\pgfqpoint{2.056891in}{0.631973in}}%
\pgfpathlineto{\pgfqpoint{2.057286in}{0.632844in}}%
\pgfpathlineto{\pgfqpoint{2.058865in}{0.636484in}}%
\pgfpathlineto{\pgfqpoint{2.059260in}{0.633441in}}%
\pgfpathlineto{\pgfqpoint{2.062418in}{0.616700in}}%
\pgfpathlineto{\pgfqpoint{2.065181in}{0.622463in}}%
\pgfpathlineto{\pgfqpoint{2.065576in}{0.620548in}}%
\pgfpathlineto{\pgfqpoint{2.070313in}{0.600164in}}%
\pgfpathlineto{\pgfqpoint{2.070708in}{0.600826in}}%
\pgfpathlineto{\pgfqpoint{2.073076in}{0.604456in}}%
\pgfpathlineto{\pgfqpoint{2.075445in}{0.620670in}}%
\pgfpathlineto{\pgfqpoint{2.077024in}{0.623539in}}%
\pgfpathlineto{\pgfqpoint{2.077418in}{0.622119in}}%
\pgfpathlineto{\pgfqpoint{2.078208in}{0.620533in}}%
\pgfpathlineto{\pgfqpoint{2.079787in}{0.629025in}}%
\pgfpathlineto{\pgfqpoint{2.080182in}{0.626995in}}%
\pgfpathlineto{\pgfqpoint{2.083734in}{0.610288in}}%
\pgfpathlineto{\pgfqpoint{2.084524in}{0.611589in}}%
\pgfpathlineto{\pgfqpoint{2.087682in}{0.617688in}}%
\pgfpathlineto{\pgfqpoint{2.088077in}{0.617106in}}%
\pgfpathlineto{\pgfqpoint{2.088471in}{0.618853in}}%
\pgfpathlineto{\pgfqpoint{2.090445in}{0.628596in}}%
\pgfpathlineto{\pgfqpoint{2.092419in}{0.628649in}}%
\pgfpathlineto{\pgfqpoint{2.097156in}{0.617476in}}%
\pgfpathlineto{\pgfqpoint{2.097551in}{0.619403in}}%
\pgfpathlineto{\pgfqpoint{2.108209in}{0.634267in}}%
\pgfpathlineto{\pgfqpoint{2.108998in}{0.633150in}}%
\pgfpathlineto{\pgfqpoint{2.112156in}{0.620127in}}%
\pgfpathlineto{\pgfqpoint{2.112946in}{0.621819in}}%
\pgfpathlineto{\pgfqpoint{2.113735in}{0.619949in}}%
\pgfpathlineto{\pgfqpoint{2.115709in}{0.616630in}}%
\pgfpathlineto{\pgfqpoint{2.116104in}{0.617004in}}%
\pgfpathlineto{\pgfqpoint{2.116499in}{0.617614in}}%
\pgfpathlineto{\pgfqpoint{2.116893in}{0.616146in}}%
\pgfpathlineto{\pgfqpoint{2.117288in}{0.615736in}}%
\pgfpathlineto{\pgfqpoint{2.117683in}{0.616429in}}%
\pgfpathlineto{\pgfqpoint{2.118867in}{0.619373in}}%
\pgfpathlineto{\pgfqpoint{2.119262in}{0.618604in}}%
\pgfpathlineto{\pgfqpoint{2.120446in}{0.615333in}}%
\pgfpathlineto{\pgfqpoint{2.120841in}{0.616605in}}%
\pgfpathlineto{\pgfqpoint{2.122420in}{0.626373in}}%
\pgfpathlineto{\pgfqpoint{2.123209in}{0.626163in}}%
\pgfpathlineto{\pgfqpoint{2.124788in}{0.624640in}}%
\pgfpathlineto{\pgfqpoint{2.125578in}{0.620082in}}%
\pgfpathlineto{\pgfqpoint{2.125973in}{0.622520in}}%
\pgfpathlineto{\pgfqpoint{2.128736in}{0.631934in}}%
\pgfpathlineto{\pgfqpoint{2.130315in}{0.627229in}}%
\pgfpathlineto{\pgfqpoint{2.131104in}{0.629919in}}%
\pgfpathlineto{\pgfqpoint{2.132288in}{0.632643in}}%
\pgfpathlineto{\pgfqpoint{2.133867in}{0.642269in}}%
\pgfpathlineto{\pgfqpoint{2.134262in}{0.640861in}}%
\pgfpathlineto{\pgfqpoint{2.138999in}{0.606982in}}%
\pgfpathlineto{\pgfqpoint{2.140578in}{0.611197in}}%
\pgfpathlineto{\pgfqpoint{2.142157in}{0.610597in}}%
\pgfpathlineto{\pgfqpoint{2.148473in}{0.614242in}}%
\pgfpathlineto{\pgfqpoint{2.149263in}{0.612715in}}%
\pgfpathlineto{\pgfqpoint{2.150052in}{0.613439in}}%
\pgfpathlineto{\pgfqpoint{2.150842in}{0.615063in}}%
\pgfpathlineto{\pgfqpoint{2.151631in}{0.613780in}}%
\pgfpathlineto{\pgfqpoint{2.155579in}{0.603236in}}%
\pgfpathlineto{\pgfqpoint{2.155973in}{0.602528in}}%
\pgfpathlineto{\pgfqpoint{2.156368in}{0.603049in}}%
\pgfpathlineto{\pgfqpoint{2.159921in}{0.616994in}}%
\pgfpathlineto{\pgfqpoint{2.161105in}{0.614193in}}%
\pgfpathlineto{\pgfqpoint{2.162289in}{0.615071in}}%
\pgfpathlineto{\pgfqpoint{2.163474in}{0.618778in}}%
\pgfpathlineto{\pgfqpoint{2.163868in}{0.617202in}}%
\pgfpathlineto{\pgfqpoint{2.164263in}{0.614708in}}%
\pgfpathlineto{\pgfqpoint{2.165447in}{0.615366in}}%
\pgfpathlineto{\pgfqpoint{2.168211in}{0.624344in}}%
\pgfpathlineto{\pgfqpoint{2.171369in}{0.639483in}}%
\pgfpathlineto{\pgfqpoint{2.173737in}{0.634109in}}%
\pgfpathlineto{\pgfqpoint{2.174132in}{0.635231in}}%
\pgfpathlineto{\pgfqpoint{2.181237in}{0.660351in}}%
\pgfpathlineto{\pgfqpoint{2.183211in}{0.662095in}}%
\pgfpathlineto{\pgfqpoint{2.186764in}{0.660969in}}%
\pgfpathlineto{\pgfqpoint{2.191106in}{0.646976in}}%
\pgfpathlineto{\pgfqpoint{2.191501in}{0.649735in}}%
\pgfpathlineto{\pgfqpoint{2.193475in}{0.659063in}}%
\pgfpathlineto{\pgfqpoint{2.194264in}{0.658161in}}%
\pgfpathlineto{\pgfqpoint{2.197817in}{0.644956in}}%
\pgfpathlineto{\pgfqpoint{2.198212in}{0.646007in}}%
\pgfpathlineto{\pgfqpoint{2.199396in}{0.644647in}}%
\pgfpathlineto{\pgfqpoint{2.201370in}{0.632841in}}%
\pgfpathlineto{\pgfqpoint{2.202554in}{0.633236in}}%
\pgfpathlineto{\pgfqpoint{2.204922in}{0.639884in}}%
\pgfpathlineto{\pgfqpoint{2.206501in}{0.646605in}}%
\pgfpathlineto{\pgfqpoint{2.207291in}{0.645392in}}%
\pgfpathlineto{\pgfqpoint{2.207686in}{0.645670in}}%
\pgfpathlineto{\pgfqpoint{2.208080in}{0.644483in}}%
\pgfpathlineto{\pgfqpoint{2.210449in}{0.637649in}}%
\pgfpathlineto{\pgfqpoint{2.211238in}{0.640305in}}%
\pgfpathlineto{\pgfqpoint{2.212422in}{0.645718in}}%
\pgfpathlineto{\pgfqpoint{2.212817in}{0.642971in}}%
\pgfpathlineto{\pgfqpoint{2.214001in}{0.637815in}}%
\pgfpathlineto{\pgfqpoint{2.214791in}{0.639084in}}%
\pgfpathlineto{\pgfqpoint{2.218738in}{0.646549in}}%
\pgfpathlineto{\pgfqpoint{2.219528in}{0.645026in}}%
\pgfpathlineto{\pgfqpoint{2.220317in}{0.645593in}}%
\pgfpathlineto{\pgfqpoint{2.222291in}{0.638711in}}%
\pgfpathlineto{\pgfqpoint{2.223081in}{0.640375in}}%
\pgfpathlineto{\pgfqpoint{2.223870in}{0.641675in}}%
\pgfpathlineto{\pgfqpoint{2.224660in}{0.640568in}}%
\pgfpathlineto{\pgfqpoint{2.225054in}{0.639320in}}%
\pgfpathlineto{\pgfqpoint{2.225844in}{0.640234in}}%
\pgfpathlineto{\pgfqpoint{2.227818in}{0.644617in}}%
\pgfpathlineto{\pgfqpoint{2.228212in}{0.644181in}}%
\pgfpathlineto{\pgfqpoint{2.229397in}{0.638772in}}%
\pgfpathlineto{\pgfqpoint{2.230976in}{0.627958in}}%
\pgfpathlineto{\pgfqpoint{2.231765in}{0.629361in}}%
\pgfpathlineto{\pgfqpoint{2.233739in}{0.633973in}}%
\pgfpathlineto{\pgfqpoint{2.234528in}{0.631661in}}%
\pgfpathlineto{\pgfqpoint{2.234923in}{0.634084in}}%
\pgfpathlineto{\pgfqpoint{2.235713in}{0.636366in}}%
\pgfpathlineto{\pgfqpoint{2.236897in}{0.641750in}}%
\pgfpathlineto{\pgfqpoint{2.238081in}{0.641341in}}%
\pgfpathlineto{\pgfqpoint{2.239265in}{0.643067in}}%
\pgfpathlineto{\pgfqpoint{2.239660in}{0.642967in}}%
\pgfpathlineto{\pgfqpoint{2.241634in}{0.640484in}}%
\pgfpathlineto{\pgfqpoint{2.245187in}{0.650021in}}%
\pgfpathlineto{\pgfqpoint{2.245976in}{0.647725in}}%
\pgfpathlineto{\pgfqpoint{2.246766in}{0.649960in}}%
\pgfpathlineto{\pgfqpoint{2.251108in}{0.657601in}}%
\pgfpathlineto{\pgfqpoint{2.251897in}{0.656074in}}%
\pgfpathlineto{\pgfqpoint{2.254266in}{0.651130in}}%
\pgfpathlineto{\pgfqpoint{2.255055in}{0.653111in}}%
\pgfpathlineto{\pgfqpoint{2.255450in}{0.652151in}}%
\pgfpathlineto{\pgfqpoint{2.256634in}{0.649263in}}%
\pgfpathlineto{\pgfqpoint{2.257424in}{0.649544in}}%
\pgfpathlineto{\pgfqpoint{2.259003in}{0.652549in}}%
\pgfpathlineto{\pgfqpoint{2.260187in}{0.651668in}}%
\pgfpathlineto{\pgfqpoint{2.262950in}{0.646696in}}%
\pgfpathlineto{\pgfqpoint{2.267293in}{0.636336in}}%
\pgfpathlineto{\pgfqpoint{2.269266in}{0.636991in}}%
\pgfpathlineto{\pgfqpoint{2.270845in}{0.633938in}}%
\pgfpathlineto{\pgfqpoint{2.271240in}{0.634422in}}%
\pgfpathlineto{\pgfqpoint{2.272030in}{0.636135in}}%
\pgfpathlineto{\pgfqpoint{2.272819in}{0.637740in}}%
\pgfpathlineto{\pgfqpoint{2.273609in}{0.636392in}}%
\pgfpathlineto{\pgfqpoint{2.275582in}{0.633603in}}%
\pgfpathlineto{\pgfqpoint{2.276372in}{0.632164in}}%
\pgfpathlineto{\pgfqpoint{2.277161in}{0.633100in}}%
\pgfpathlineto{\pgfqpoint{2.278740in}{0.632072in}}%
\pgfpathlineto{\pgfqpoint{2.279135in}{0.632863in}}%
\pgfpathlineto{\pgfqpoint{2.282293in}{0.638810in}}%
\pgfpathlineto{\pgfqpoint{2.283083in}{0.635974in}}%
\pgfpathlineto{\pgfqpoint{2.283477in}{0.632977in}}%
\pgfpathlineto{\pgfqpoint{2.284662in}{0.633924in}}%
\pgfpathlineto{\pgfqpoint{2.285451in}{0.634836in}}%
\pgfpathlineto{\pgfqpoint{2.285846in}{0.634306in}}%
\pgfpathlineto{\pgfqpoint{2.287030in}{0.628317in}}%
\pgfpathlineto{\pgfqpoint{2.288214in}{0.629532in}}%
\pgfpathlineto{\pgfqpoint{2.290583in}{0.626461in}}%
\pgfpathlineto{\pgfqpoint{2.291767in}{0.624097in}}%
\pgfpathlineto{\pgfqpoint{2.292162in}{0.625512in}}%
\pgfpathlineto{\pgfqpoint{2.293741in}{0.628797in}}%
\pgfpathlineto{\pgfqpoint{2.294136in}{0.629352in}}%
\pgfpathlineto{\pgfqpoint{2.294925in}{0.628193in}}%
\pgfpathlineto{\pgfqpoint{2.296504in}{0.624950in}}%
\pgfpathlineto{\pgfqpoint{2.296899in}{0.627023in}}%
\pgfpathlineto{\pgfqpoint{2.298083in}{0.629599in}}%
\pgfpathlineto{\pgfqpoint{2.299662in}{0.634736in}}%
\pgfpathlineto{\pgfqpoint{2.300057in}{0.633638in}}%
\pgfpathlineto{\pgfqpoint{2.300846in}{0.635318in}}%
\pgfpathlineto{\pgfqpoint{2.301636in}{0.633440in}}%
\pgfpathlineto{\pgfqpoint{2.304399in}{0.623962in}}%
\pgfpathlineto{\pgfqpoint{2.305583in}{0.624609in}}%
\pgfpathlineto{\pgfqpoint{2.306767in}{0.625708in}}%
\pgfpathlineto{\pgfqpoint{2.307162in}{0.624935in}}%
\pgfpathlineto{\pgfqpoint{2.307952in}{0.623835in}}%
\pgfpathlineto{\pgfqpoint{2.308346in}{0.624607in}}%
\pgfpathlineto{\pgfqpoint{2.309531in}{0.626806in}}%
\pgfpathlineto{\pgfqpoint{2.309925in}{0.625818in}}%
\pgfpathlineto{\pgfqpoint{2.313083in}{0.617884in}}%
\pgfpathlineto{\pgfqpoint{2.313873in}{0.619403in}}%
\pgfpathlineto{\pgfqpoint{2.314662in}{0.620821in}}%
\pgfpathlineto{\pgfqpoint{2.315452in}{0.619513in}}%
\pgfpathlineto{\pgfqpoint{2.315847in}{0.619531in}}%
\pgfpathlineto{\pgfqpoint{2.317031in}{0.614979in}}%
\pgfpathlineto{\pgfqpoint{2.317426in}{0.615589in}}%
\pgfpathlineto{\pgfqpoint{2.318215in}{0.618878in}}%
\pgfpathlineto{\pgfqpoint{2.319005in}{0.616820in}}%
\pgfpathlineto{\pgfqpoint{2.321373in}{0.607002in}}%
\pgfpathlineto{\pgfqpoint{2.326505in}{0.617230in}}%
\pgfpathlineto{\pgfqpoint{2.328084in}{0.616137in}}%
\pgfpathlineto{\pgfqpoint{2.332031in}{0.602858in}}%
\pgfpathlineto{\pgfqpoint{2.332821in}{0.603871in}}%
\pgfpathlineto{\pgfqpoint{2.336374in}{0.609430in}}%
\pgfpathlineto{\pgfqpoint{2.336768in}{0.609070in}}%
\pgfpathlineto{\pgfqpoint{2.339137in}{0.601574in}}%
\pgfpathlineto{\pgfqpoint{2.339926in}{0.603445in}}%
\pgfpathlineto{\pgfqpoint{2.343084in}{0.613807in}}%
\pgfpathlineto{\pgfqpoint{2.343479in}{0.612346in}}%
\pgfpathlineto{\pgfqpoint{2.346637in}{0.613958in}}%
\pgfpathlineto{\pgfqpoint{2.349006in}{0.623321in}}%
\pgfpathlineto{\pgfqpoint{2.351769in}{0.624294in}}%
\pgfpathlineto{\pgfqpoint{2.352558in}{0.622441in}}%
\pgfpathlineto{\pgfqpoint{2.353348in}{0.617316in}}%
\pgfpathlineto{\pgfqpoint{2.354137in}{0.619914in}}%
\pgfpathlineto{\pgfqpoint{2.354927in}{0.621117in}}%
\pgfpathlineto{\pgfqpoint{2.355322in}{0.619726in}}%
\pgfpathlineto{\pgfqpoint{2.357690in}{0.613742in}}%
\pgfpathlineto{\pgfqpoint{2.358480in}{0.614050in}}%
\pgfpathlineto{\pgfqpoint{2.358874in}{0.612768in}}%
\pgfpathlineto{\pgfqpoint{2.360453in}{0.610524in}}%
\pgfpathlineto{\pgfqpoint{2.367164in}{0.623202in}}%
\pgfpathlineto{\pgfqpoint{2.369533in}{0.614217in}}%
\pgfpathlineto{\pgfqpoint{2.370322in}{0.615467in}}%
\pgfpathlineto{\pgfqpoint{2.373480in}{0.625013in}}%
\pgfpathlineto{\pgfqpoint{2.374270in}{0.622781in}}%
\pgfpathlineto{\pgfqpoint{2.374664in}{0.622301in}}%
\pgfpathlineto{\pgfqpoint{2.375059in}{0.623346in}}%
\pgfpathlineto{\pgfqpoint{2.375849in}{0.625599in}}%
\pgfpathlineto{\pgfqpoint{2.376638in}{0.630845in}}%
\pgfpathlineto{\pgfqpoint{2.377428in}{0.627971in}}%
\pgfpathlineto{\pgfqpoint{2.379401in}{0.620922in}}%
\pgfpathlineto{\pgfqpoint{2.381375in}{0.609050in}}%
\pgfpathlineto{\pgfqpoint{2.381770in}{0.610337in}}%
\pgfpathlineto{\pgfqpoint{2.386112in}{0.622455in}}%
\pgfpathlineto{\pgfqpoint{2.386901in}{0.621667in}}%
\pgfpathlineto{\pgfqpoint{2.388480in}{0.621770in}}%
\pgfpathlineto{\pgfqpoint{2.392823in}{0.627531in}}%
\pgfpathlineto{\pgfqpoint{2.393217in}{0.627003in}}%
\pgfpathlineto{\pgfqpoint{2.394007in}{0.623558in}}%
\pgfpathlineto{\pgfqpoint{2.394402in}{0.625345in}}%
\pgfpathlineto{\pgfqpoint{2.396375in}{0.632384in}}%
\pgfpathlineto{\pgfqpoint{2.397165in}{0.631269in}}%
\pgfpathlineto{\pgfqpoint{2.397560in}{0.631048in}}%
\pgfpathlineto{\pgfqpoint{2.397954in}{0.632490in}}%
\pgfpathlineto{\pgfqpoint{2.400323in}{0.639925in}}%
\pgfpathlineto{\pgfqpoint{2.400718in}{0.639443in}}%
\pgfpathlineto{\pgfqpoint{2.403481in}{0.635225in}}%
\pgfpathlineto{\pgfqpoint{2.405849in}{0.634025in}}%
\pgfpathlineto{\pgfqpoint{2.407823in}{0.630651in}}%
\pgfpathlineto{\pgfqpoint{2.408613in}{0.631641in}}%
\pgfpathlineto{\pgfqpoint{2.410192in}{0.633102in}}%
\pgfpathlineto{\pgfqpoint{2.410981in}{0.630286in}}%
\pgfpathlineto{\pgfqpoint{2.411771in}{0.631658in}}%
\pgfpathlineto{\pgfqpoint{2.412560in}{0.631251in}}%
\pgfpathlineto{\pgfqpoint{2.415718in}{0.623296in}}%
\pgfpathlineto{\pgfqpoint{2.418481in}{0.627064in}}%
\pgfpathlineto{\pgfqpoint{2.416508in}{0.622569in}}%
\pgfpathlineto{\pgfqpoint{2.418876in}{0.625680in}}%
\pgfpathlineto{\pgfqpoint{2.421245in}{0.619367in}}%
\pgfpathlineto{\pgfqpoint{2.422824in}{0.615818in}}%
\pgfpathlineto{\pgfqpoint{2.425192in}{0.607393in}}%
\pgfpathlineto{\pgfqpoint{2.425982in}{0.609210in}}%
\pgfpathlineto{\pgfqpoint{2.427166in}{0.610621in}}%
\pgfpathlineto{\pgfqpoint{2.428745in}{0.612587in}}%
\pgfpathlineto{\pgfqpoint{2.432298in}{0.608754in}}%
\pgfpathlineto{\pgfqpoint{2.435456in}{0.620778in}}%
\pgfpathlineto{\pgfqpoint{2.436640in}{0.616412in}}%
\pgfpathlineto{\pgfqpoint{2.439403in}{0.612483in}}%
\pgfpathlineto{\pgfqpoint{2.442166in}{0.606764in}}%
\pgfpathlineto{\pgfqpoint{2.442561in}{0.608194in}}%
\pgfpathlineto{\pgfqpoint{2.443745in}{0.611771in}}%
\pgfpathlineto{\pgfqpoint{2.444535in}{0.610937in}}%
\pgfpathlineto{\pgfqpoint{2.448877in}{0.618958in}}%
\pgfpathlineto{\pgfqpoint{2.449272in}{0.617005in}}%
\pgfpathlineto{\pgfqpoint{2.452035in}{0.606386in}}%
\pgfpathlineto{\pgfqpoint{2.454404in}{0.590620in}}%
\pgfpathlineto{\pgfqpoint{2.454798in}{0.593237in}}%
\pgfpathlineto{\pgfqpoint{2.458746in}{0.602545in}}%
\pgfpathlineto{\pgfqpoint{2.459930in}{0.607896in}}%
\pgfpathlineto{\pgfqpoint{2.460720in}{0.605694in}}%
\pgfpathlineto{\pgfqpoint{2.462693in}{0.602930in}}%
\pgfpathlineto{\pgfqpoint{2.463483in}{0.599559in}}%
\pgfpathlineto{\pgfqpoint{2.463877in}{0.601911in}}%
\pgfpathlineto{\pgfqpoint{2.467430in}{0.612713in}}%
\pgfpathlineto{\pgfqpoint{2.467825in}{0.611675in}}%
\pgfpathlineto{\pgfqpoint{2.470983in}{0.605802in}}%
\pgfpathlineto{\pgfqpoint{2.471378in}{0.606981in}}%
\pgfpathlineto{\pgfqpoint{2.473746in}{0.615080in}}%
\pgfpathlineto{\pgfqpoint{2.474536in}{0.613835in}}%
\pgfpathlineto{\pgfqpoint{2.474930in}{0.614455in}}%
\pgfpathlineto{\pgfqpoint{2.475720in}{0.612922in}}%
\pgfpathlineto{\pgfqpoint{2.476509in}{0.612324in}}%
\pgfpathlineto{\pgfqpoint{2.477299in}{0.606818in}}%
\pgfpathlineto{\pgfqpoint{2.478483in}{0.608322in}}%
\pgfpathlineto{\pgfqpoint{2.479273in}{0.606477in}}%
\pgfpathlineto{\pgfqpoint{2.479667in}{0.607921in}}%
\pgfpathlineto{\pgfqpoint{2.489141in}{0.621989in}}%
\pgfpathlineto{\pgfqpoint{2.493484in}{0.613403in}}%
\pgfpathlineto{\pgfqpoint{2.495457in}{0.614794in}}%
\pgfpathlineto{\pgfqpoint{2.497036in}{0.618199in}}%
\pgfpathlineto{\pgfqpoint{2.497431in}{0.617216in}}%
\pgfpathlineto{\pgfqpoint{2.498221in}{0.616541in}}%
\pgfpathlineto{\pgfqpoint{2.498615in}{0.617895in}}%
\pgfpathlineto{\pgfqpoint{2.502168in}{0.623249in}}%
\pgfpathlineto{\pgfqpoint{2.502563in}{0.624817in}}%
\pgfpathlineto{\pgfqpoint{2.503352in}{0.622482in}}%
\pgfpathlineto{\pgfqpoint{2.505721in}{0.619566in}}%
\pgfpathlineto{\pgfqpoint{2.507695in}{0.611436in}}%
\pgfpathlineto{\pgfqpoint{2.508089in}{0.612136in}}%
\pgfpathlineto{\pgfqpoint{2.512432in}{0.626512in}}%
\pgfpathlineto{\pgfqpoint{2.513616in}{0.629819in}}%
\pgfpathlineto{\pgfqpoint{2.514011in}{0.629628in}}%
\pgfpathlineto{\pgfqpoint{2.515590in}{0.627315in}}%
\pgfpathlineto{\pgfqpoint{2.518748in}{0.634238in}}%
\pgfpathlineto{\pgfqpoint{2.519142in}{0.634420in}}%
\pgfpathlineto{\pgfqpoint{2.520721in}{0.628879in}}%
\pgfpathlineto{\pgfqpoint{2.521511in}{0.630200in}}%
\pgfpathlineto{\pgfqpoint{2.524274in}{0.642202in}}%
\pgfpathlineto{\pgfqpoint{2.524669in}{0.640336in}}%
\pgfpathlineto{\pgfqpoint{2.525458in}{0.637697in}}%
\pgfpathlineto{\pgfqpoint{2.525853in}{0.638855in}}%
\pgfpathlineto{\pgfqpoint{2.527827in}{0.645751in}}%
\pgfpathlineto{\pgfqpoint{2.528616in}{0.644602in}}%
\pgfpathlineto{\pgfqpoint{2.529801in}{0.642938in}}%
\pgfpathlineto{\pgfqpoint{2.530195in}{0.643844in}}%
\pgfpathlineto{\pgfqpoint{2.531774in}{0.645982in}}%
\pgfpathlineto{\pgfqpoint{2.532169in}{0.645516in}}%
\pgfpathlineto{\pgfqpoint{2.535327in}{0.639221in}}%
\pgfpathlineto{\pgfqpoint{2.536906in}{0.641677in}}%
\pgfpathlineto{\pgfqpoint{2.537301in}{0.641055in}}%
\pgfpathlineto{\pgfqpoint{2.539275in}{0.638249in}}%
\pgfpathlineto{\pgfqpoint{2.540854in}{0.640598in}}%
\pgfpathlineto{\pgfqpoint{2.541248in}{0.638228in}}%
\pgfpathlineto{\pgfqpoint{2.542827in}{0.635976in}}%
\pgfpathlineto{\pgfqpoint{2.543222in}{0.636182in}}%
\pgfpathlineto{\pgfqpoint{2.547169in}{0.645590in}}%
\pgfpathlineto{\pgfqpoint{2.549933in}{0.659865in}}%
\pgfpathlineto{\pgfqpoint{2.550327in}{0.659345in}}%
\pgfpathlineto{\pgfqpoint{2.554275in}{0.648885in}}%
\pgfpathlineto{\pgfqpoint{2.554670in}{0.649828in}}%
\pgfpathlineto{\pgfqpoint{2.557433in}{0.657781in}}%
\pgfpathlineto{\pgfqpoint{2.557828in}{0.656829in}}%
\pgfpathlineto{\pgfqpoint{2.559012in}{0.656674in}}%
\pgfpathlineto{\pgfqpoint{2.562170in}{0.651488in}}%
\pgfpathlineto{\pgfqpoint{2.562565in}{0.652257in}}%
\pgfpathlineto{\pgfqpoint{2.563354in}{0.652981in}}%
\pgfpathlineto{\pgfqpoint{2.566512in}{0.642138in}}%
\pgfpathlineto{\pgfqpoint{2.567696in}{0.645442in}}%
\pgfpathlineto{\pgfqpoint{2.570460in}{0.652422in}}%
\pgfpathlineto{\pgfqpoint{2.572433in}{0.653877in}}%
\pgfpathlineto{\pgfqpoint{2.572828in}{0.653497in}}%
\pgfpathlineto{\pgfqpoint{2.573618in}{0.654266in}}%
\pgfpathlineto{\pgfqpoint{2.574012in}{0.653282in}}%
\pgfpathlineto{\pgfqpoint{2.579539in}{0.636022in}}%
\pgfpathlineto{\pgfqpoint{2.580723in}{0.637910in}}%
\pgfpathlineto{\pgfqpoint{2.581513in}{0.636750in}}%
\pgfpathlineto{\pgfqpoint{2.583881in}{0.629596in}}%
\pgfpathlineto{\pgfqpoint{2.584276in}{0.630073in}}%
\pgfpathlineto{\pgfqpoint{2.585065in}{0.631330in}}%
\pgfpathlineto{\pgfqpoint{2.585855in}{0.630198in}}%
\pgfpathlineto{\pgfqpoint{2.589013in}{0.634499in}}%
\pgfpathlineto{\pgfqpoint{2.589408in}{0.633695in}}%
\pgfpathlineto{\pgfqpoint{2.591381in}{0.628795in}}%
\pgfpathlineto{\pgfqpoint{2.592171in}{0.629273in}}%
\pgfpathlineto{\pgfqpoint{2.592566in}{0.628644in}}%
\pgfpathlineto{\pgfqpoint{2.594539in}{0.627891in}}%
\pgfpathlineto{\pgfqpoint{2.596118in}{0.631252in}}%
\pgfpathlineto{\pgfqpoint{2.596513in}{0.630878in}}%
\pgfpathlineto{\pgfqpoint{2.598092in}{0.631154in}}%
\pgfpathlineto{\pgfqpoint{2.598487in}{0.630637in}}%
\pgfpathlineto{\pgfqpoint{2.598882in}{0.631657in}}%
\pgfpathlineto{\pgfqpoint{2.600461in}{0.633941in}}%
\pgfpathlineto{\pgfqpoint{2.600855in}{0.633106in}}%
\pgfpathlineto{\pgfqpoint{2.603619in}{0.628764in}}%
\pgfpathlineto{\pgfqpoint{2.605592in}{0.625176in}}%
\pgfpathlineto{\pgfqpoint{2.606382in}{0.627838in}}%
\pgfpathlineto{\pgfqpoint{2.607566in}{0.631190in}}%
\pgfpathlineto{\pgfqpoint{2.608356in}{0.629011in}}%
\pgfpathlineto{\pgfqpoint{2.610329in}{0.622344in}}%
\pgfpathlineto{\pgfqpoint{2.610724in}{0.622858in}}%
\pgfpathlineto{\pgfqpoint{2.611514in}{0.622137in}}%
\pgfpathlineto{\pgfqpoint{2.611908in}{0.623822in}}%
\pgfpathlineto{\pgfqpoint{2.613093in}{0.627449in}}%
\pgfpathlineto{\pgfqpoint{2.613882in}{0.625732in}}%
\pgfpathlineto{\pgfqpoint{2.617040in}{0.630074in}}%
\pgfpathlineto{\pgfqpoint{2.619014in}{0.644779in}}%
\pgfpathlineto{\pgfqpoint{2.619803in}{0.641883in}}%
\pgfpathlineto{\pgfqpoint{2.621382in}{0.629966in}}%
\pgfpathlineto{\pgfqpoint{2.622567in}{0.631053in}}%
\pgfpathlineto{\pgfqpoint{2.623356in}{0.631150in}}%
\pgfpathlineto{\pgfqpoint{2.623751in}{0.630279in}}%
\pgfpathlineto{\pgfqpoint{2.624540in}{0.627213in}}%
\pgfpathlineto{\pgfqpoint{2.626119in}{0.621087in}}%
\pgfpathlineto{\pgfqpoint{2.626514in}{0.622308in}}%
\pgfpathlineto{\pgfqpoint{2.626909in}{0.624296in}}%
\pgfpathlineto{\pgfqpoint{2.627698in}{0.622636in}}%
\pgfpathlineto{\pgfqpoint{2.629277in}{0.618778in}}%
\pgfpathlineto{\pgfqpoint{2.630067in}{0.616407in}}%
\pgfpathlineto{\pgfqpoint{2.630856in}{0.618499in}}%
\pgfpathlineto{\pgfqpoint{2.633225in}{0.614741in}}%
\pgfpathlineto{\pgfqpoint{2.633619in}{0.615946in}}%
\pgfpathlineto{\pgfqpoint{2.634014in}{0.616121in}}%
\pgfpathlineto{\pgfqpoint{2.635593in}{0.612100in}}%
\pgfpathlineto{\pgfqpoint{2.635988in}{0.613398in}}%
\pgfpathlineto{\pgfqpoint{2.637172in}{0.612544in}}%
\pgfpathlineto{\pgfqpoint{2.638356in}{0.608864in}}%
\pgfpathlineto{\pgfqpoint{2.638751in}{0.611611in}}%
\pgfpathlineto{\pgfqpoint{2.639541in}{0.613288in}}%
\pgfpathlineto{\pgfqpoint{2.639935in}{0.612822in}}%
\pgfpathlineto{\pgfqpoint{2.642699in}{0.606096in}}%
\pgfpathlineto{\pgfqpoint{2.644278in}{0.609507in}}%
\pgfpathlineto{\pgfqpoint{2.646251in}{0.613868in}}%
\pgfpathlineto{\pgfqpoint{2.649015in}{0.621601in}}%
\pgfpathlineto{\pgfqpoint{2.650988in}{0.630052in}}%
\pgfpathlineto{\pgfqpoint{2.651383in}{0.629657in}}%
\pgfpathlineto{\pgfqpoint{2.652567in}{0.628266in}}%
\pgfpathlineto{\pgfqpoint{2.652962in}{0.629318in}}%
\pgfpathlineto{\pgfqpoint{2.654541in}{0.632446in}}%
\pgfpathlineto{\pgfqpoint{2.656910in}{0.640764in}}%
\pgfpathlineto{\pgfqpoint{2.657304in}{0.639434in}}%
\pgfpathlineto{\pgfqpoint{2.658094in}{0.635977in}}%
\pgfpathlineto{\pgfqpoint{2.658883in}{0.637774in}}%
\pgfpathlineto{\pgfqpoint{2.659278in}{0.640564in}}%
\pgfpathlineto{\pgfqpoint{2.660068in}{0.637258in}}%
\pgfpathlineto{\pgfqpoint{2.662436in}{0.631753in}}%
\pgfpathlineto{\pgfqpoint{2.663226in}{0.633492in}}%
\pgfpathlineto{\pgfqpoint{2.664015in}{0.632044in}}%
\pgfpathlineto{\pgfqpoint{2.667173in}{0.628108in}}%
\pgfpathlineto{\pgfqpoint{2.671121in}{0.609954in}}%
\pgfpathlineto{\pgfqpoint{2.671515in}{0.609948in}}%
\pgfpathlineto{\pgfqpoint{2.675068in}{0.621479in}}%
\pgfpathlineto{\pgfqpoint{2.677437in}{0.617804in}}%
\pgfpathlineto{\pgfqpoint{2.677831in}{0.619198in}}%
\pgfpathlineto{\pgfqpoint{2.678226in}{0.615176in}}%
\pgfpathlineto{\pgfqpoint{2.682174in}{0.620809in}}%
\pgfpathlineto{\pgfqpoint{2.682963in}{0.622087in}}%
\pgfpathlineto{\pgfqpoint{2.683358in}{0.620839in}}%
\pgfpathlineto{\pgfqpoint{2.683753in}{0.620360in}}%
\pgfpathlineto{\pgfqpoint{2.685332in}{0.610855in}}%
\pgfpathlineto{\pgfqpoint{2.686121in}{0.611226in}}%
\pgfpathlineto{\pgfqpoint{2.687700in}{0.611681in}}%
\pgfpathlineto{\pgfqpoint{2.688490in}{0.613601in}}%
\pgfpathlineto{\pgfqpoint{2.689279in}{0.612772in}}%
\pgfpathlineto{\pgfqpoint{2.690858in}{0.611066in}}%
\pgfpathlineto{\pgfqpoint{2.691648in}{0.608735in}}%
\pgfpathlineto{\pgfqpoint{2.692437in}{0.610900in}}%
\pgfpathlineto{\pgfqpoint{2.695200in}{0.628377in}}%
\pgfpathlineto{\pgfqpoint{2.695990in}{0.626636in}}%
\pgfpathlineto{\pgfqpoint{2.700332in}{0.602869in}}%
\pgfpathlineto{\pgfqpoint{2.703095in}{0.601744in}}%
\pgfpathlineto{\pgfqpoint{2.704674in}{0.603517in}}%
\pgfpathlineto{\pgfqpoint{2.705069in}{0.603366in}}%
\pgfpathlineto{\pgfqpoint{2.707043in}{0.601397in}}%
\pgfpathlineto{\pgfqpoint{2.707438in}{0.602632in}}%
\pgfpathlineto{\pgfqpoint{2.708622in}{0.609499in}}%
\pgfpathlineto{\pgfqpoint{2.709411in}{0.606311in}}%
\pgfpathlineto{\pgfqpoint{2.711780in}{0.607707in}}%
\pgfpathlineto{\pgfqpoint{2.712174in}{0.607322in}}%
\pgfpathlineto{\pgfqpoint{2.712964in}{0.607187in}}%
\pgfpathlineto{\pgfqpoint{2.714148in}{0.612276in}}%
\pgfpathlineto{\pgfqpoint{2.714543in}{0.610531in}}%
\pgfpathlineto{\pgfqpoint{2.714938in}{0.609388in}}%
\pgfpathlineto{\pgfqpoint{2.715332in}{0.610946in}}%
\pgfpathlineto{\pgfqpoint{2.716517in}{0.614108in}}%
\pgfpathlineto{\pgfqpoint{2.716911in}{0.611418in}}%
\pgfpathlineto{\pgfqpoint{2.718885in}{0.605502in}}%
\pgfpathlineto{\pgfqpoint{2.719280in}{0.606352in}}%
\pgfpathlineto{\pgfqpoint{2.720069in}{0.606963in}}%
\pgfpathlineto{\pgfqpoint{2.720464in}{0.606107in}}%
\pgfpathlineto{\pgfqpoint{2.722043in}{0.602395in}}%
\pgfpathlineto{\pgfqpoint{2.722833in}{0.603456in}}%
\pgfpathlineto{\pgfqpoint{2.723622in}{0.603879in}}%
\pgfpathlineto{\pgfqpoint{2.724017in}{0.603335in}}%
\pgfpathlineto{\pgfqpoint{2.725991in}{0.601133in}}%
\pgfpathlineto{\pgfqpoint{2.727570in}{0.601103in}}%
\pgfpathlineto{\pgfqpoint{2.729149in}{0.587866in}}%
\pgfpathlineto{\pgfqpoint{2.730333in}{0.589280in}}%
\pgfpathlineto{\pgfqpoint{2.733491in}{0.601028in}}%
\pgfpathlineto{\pgfqpoint{2.735465in}{0.599552in}}%
\pgfpathlineto{\pgfqpoint{2.737044in}{0.597621in}}%
\pgfpathlineto{\pgfqpoint{2.737438in}{0.599280in}}%
\pgfpathlineto{\pgfqpoint{2.738228in}{0.602384in}}%
\pgfpathlineto{\pgfqpoint{2.739017in}{0.601204in}}%
\pgfpathlineto{\pgfqpoint{2.740596in}{0.601202in}}%
\pgfpathlineto{\pgfqpoint{2.743754in}{0.619327in}}%
\pgfpathlineto{\pgfqpoint{2.744149in}{0.619075in}}%
\pgfpathlineto{\pgfqpoint{2.744939in}{0.613560in}}%
\pgfpathlineto{\pgfqpoint{2.745728in}{0.617437in}}%
\pgfpathlineto{\pgfqpoint{2.746123in}{0.618251in}}%
\pgfpathlineto{\pgfqpoint{2.746518in}{0.615798in}}%
\pgfpathlineto{\pgfqpoint{2.749676in}{0.598203in}}%
\pgfpathlineto{\pgfqpoint{2.750465in}{0.598760in}}%
\pgfpathlineto{\pgfqpoint{2.752439in}{0.603758in}}%
\pgfpathlineto{\pgfqpoint{2.756386in}{0.620653in}}%
\pgfpathlineto{\pgfqpoint{2.757571in}{0.624595in}}%
\pgfpathlineto{\pgfqpoint{2.759544in}{0.639390in}}%
\pgfpathlineto{\pgfqpoint{2.760729in}{0.636017in}}%
\pgfpathlineto{\pgfqpoint{2.761518in}{0.637007in}}%
\pgfpathlineto{\pgfqpoint{2.761913in}{0.635192in}}%
\pgfpathlineto{\pgfqpoint{2.763887in}{0.630432in}}%
\pgfpathlineto{\pgfqpoint{2.764281in}{0.631324in}}%
\pgfpathlineto{\pgfqpoint{2.765860in}{0.635978in}}%
\pgfpathlineto{\pgfqpoint{2.766255in}{0.633747in}}%
\pgfpathlineto{\pgfqpoint{2.767045in}{0.632240in}}%
\pgfpathlineto{\pgfqpoint{2.767439in}{0.633596in}}%
\pgfpathlineto{\pgfqpoint{2.770203in}{0.636413in}}%
\pgfpathlineto{\pgfqpoint{2.770992in}{0.638347in}}%
\pgfpathlineto{\pgfqpoint{2.772176in}{0.642199in}}%
\pgfpathlineto{\pgfqpoint{2.772571in}{0.640330in}}%
\pgfpathlineto{\pgfqpoint{2.774545in}{0.631422in}}%
\pgfpathlineto{\pgfqpoint{2.774940in}{0.631665in}}%
\pgfpathlineto{\pgfqpoint{2.776519in}{0.635072in}}%
\pgfpathlineto{\pgfqpoint{2.776913in}{0.634051in}}%
\pgfpathlineto{\pgfqpoint{2.778098in}{0.631457in}}%
\pgfpathlineto{\pgfqpoint{2.778492in}{0.633489in}}%
\pgfpathlineto{\pgfqpoint{2.780466in}{0.637311in}}%
\pgfpathlineto{\pgfqpoint{2.782440in}{0.636171in}}%
\pgfpathlineto{\pgfqpoint{2.782835in}{0.634655in}}%
\pgfpathlineto{\pgfqpoint{2.783624in}{0.636095in}}%
\pgfpathlineto{\pgfqpoint{2.785598in}{0.639693in}}%
\pgfpathlineto{\pgfqpoint{2.786782in}{0.637209in}}%
\pgfpathlineto{\pgfqpoint{2.788756in}{0.631770in}}%
\pgfpathlineto{\pgfqpoint{2.789151in}{0.632995in}}%
\pgfpathlineto{\pgfqpoint{2.790730in}{0.631481in}}%
\pgfpathlineto{\pgfqpoint{2.792703in}{0.627740in}}%
\pgfpathlineto{\pgfqpoint{2.793493in}{0.629218in}}%
\pgfpathlineto{\pgfqpoint{2.794282in}{0.627977in}}%
\pgfpathlineto{\pgfqpoint{2.797045in}{0.619441in}}%
\pgfpathlineto{\pgfqpoint{2.797440in}{0.621699in}}%
\pgfpathlineto{\pgfqpoint{2.798230in}{0.624713in}}%
\pgfpathlineto{\pgfqpoint{2.799414in}{0.624480in}}%
\pgfpathlineto{\pgfqpoint{2.800203in}{0.621886in}}%
\pgfpathlineto{\pgfqpoint{2.802967in}{0.610823in}}%
\pgfpathlineto{\pgfqpoint{2.803756in}{0.612892in}}%
\pgfpathlineto{\pgfqpoint{2.804546in}{0.611215in}}%
\pgfpathlineto{\pgfqpoint{2.806125in}{0.615715in}}%
\pgfpathlineto{\pgfqpoint{2.807309in}{0.614672in}}%
\pgfpathlineto{\pgfqpoint{2.807704in}{0.613564in}}%
\pgfpathlineto{\pgfqpoint{2.808493in}{0.615558in}}%
\pgfpathlineto{\pgfqpoint{2.814414in}{0.626743in}}%
\pgfpathlineto{\pgfqpoint{2.814809in}{0.626153in}}%
\pgfpathlineto{\pgfqpoint{2.817572in}{0.618166in}}%
\pgfpathlineto{\pgfqpoint{2.818362in}{0.619925in}}%
\pgfpathlineto{\pgfqpoint{2.819151in}{0.619471in}}%
\pgfpathlineto{\pgfqpoint{2.821915in}{0.612632in}}%
\pgfpathlineto{\pgfqpoint{2.822309in}{0.613316in}}%
\pgfpathlineto{\pgfqpoint{2.829020in}{0.630573in}}%
\pgfpathlineto{\pgfqpoint{2.832573in}{0.625404in}}%
\pgfpathlineto{\pgfqpoint{2.835336in}{0.619004in}}%
\pgfpathlineto{\pgfqpoint{2.837705in}{0.614910in}}%
\pgfpathlineto{\pgfqpoint{2.838494in}{0.616501in}}%
\pgfpathlineto{\pgfqpoint{2.841652in}{0.629926in}}%
\pgfpathlineto{\pgfqpoint{2.842047in}{0.629646in}}%
\pgfpathlineto{\pgfqpoint{2.842442in}{0.630983in}}%
\pgfpathlineto{\pgfqpoint{2.844810in}{0.635600in}}%
\pgfpathlineto{\pgfqpoint{2.845205in}{0.635107in}}%
\pgfpathlineto{\pgfqpoint{2.845994in}{0.636514in}}%
\pgfpathlineto{\pgfqpoint{2.849152in}{0.642413in}}%
\pgfpathlineto{\pgfqpoint{2.849547in}{0.642174in}}%
\pgfpathlineto{\pgfqpoint{2.855468in}{0.631569in}}%
\pgfpathlineto{\pgfqpoint{2.855863in}{0.632542in}}%
\pgfpathlineto{\pgfqpoint{2.856258in}{0.631426in}}%
\pgfpathlineto{\pgfqpoint{2.858232in}{0.629197in}}%
\pgfpathlineto{\pgfqpoint{2.859811in}{0.627105in}}%
\pgfpathlineto{\pgfqpoint{2.861390in}{0.616357in}}%
\pgfpathlineto{\pgfqpoint{2.862574in}{0.616871in}}%
\pgfpathlineto{\pgfqpoint{2.866521in}{0.626570in}}%
\pgfpathlineto{\pgfqpoint{2.867311in}{0.626280in}}%
\pgfpathlineto{\pgfqpoint{2.868890in}{0.633564in}}%
\pgfpathlineto{\pgfqpoint{2.869679in}{0.632317in}}%
\pgfpathlineto{\pgfqpoint{2.872048in}{0.624615in}}%
\pgfpathlineto{\pgfqpoint{2.874022in}{0.622264in}}%
\pgfpathlineto{\pgfqpoint{2.877969in}{0.642871in}}%
\pgfpathlineto{\pgfqpoint{2.878759in}{0.639144in}}%
\pgfpathlineto{\pgfqpoint{2.881522in}{0.632776in}}%
\pgfpathlineto{\pgfqpoint{2.882311in}{0.628935in}}%
\pgfpathlineto{\pgfqpoint{2.883101in}{0.630847in}}%
\pgfpathlineto{\pgfqpoint{2.883495in}{0.629603in}}%
\pgfpathlineto{\pgfqpoint{2.884285in}{0.631899in}}%
\pgfpathlineto{\pgfqpoint{2.887443in}{0.643588in}}%
\pgfpathlineto{\pgfqpoint{2.887838in}{0.643219in}}%
\pgfpathlineto{\pgfqpoint{2.890996in}{0.636815in}}%
\pgfpathlineto{\pgfqpoint{2.892180in}{0.638459in}}%
\pgfpathlineto{\pgfqpoint{2.892575in}{0.638001in}}%
\pgfpathlineto{\pgfqpoint{2.896127in}{0.626253in}}%
\pgfpathlineto{\pgfqpoint{2.896917in}{0.627672in}}%
\pgfpathlineto{\pgfqpoint{2.898891in}{0.635993in}}%
\pgfpathlineto{\pgfqpoint{2.900470in}{0.639827in}}%
\pgfpathlineto{\pgfqpoint{2.901259in}{0.639599in}}%
\pgfpathlineto{\pgfqpoint{2.904812in}{0.642951in}}%
\pgfpathlineto{\pgfqpoint{2.905207in}{0.644272in}}%
\pgfpathlineto{\pgfqpoint{2.905996in}{0.642301in}}%
\pgfpathlineto{\pgfqpoint{2.909944in}{0.623744in}}%
\pgfpathlineto{\pgfqpoint{2.911523in}{0.618189in}}%
\pgfpathlineto{\pgfqpoint{2.911917in}{0.620352in}}%
\pgfpathlineto{\pgfqpoint{2.915075in}{0.632305in}}%
\pgfpathlineto{\pgfqpoint{2.919023in}{0.629383in}}%
\pgfpathlineto{\pgfqpoint{2.919812in}{0.630375in}}%
\pgfpathlineto{\pgfqpoint{2.920207in}{0.628700in}}%
\pgfpathlineto{\pgfqpoint{2.920602in}{0.629457in}}%
\pgfpathlineto{\pgfqpoint{2.922181in}{0.629562in}}%
\pgfpathlineto{\pgfqpoint{2.923365in}{0.627988in}}%
\pgfpathlineto{\pgfqpoint{2.923760in}{0.628086in}}%
\pgfpathlineto{\pgfqpoint{2.925339in}{0.631719in}}%
\pgfpathlineto{\pgfqpoint{2.925734in}{0.630337in}}%
\pgfpathlineto{\pgfqpoint{2.926128in}{0.629014in}}%
\pgfpathlineto{\pgfqpoint{2.926918in}{0.629729in}}%
\pgfpathlineto{\pgfqpoint{2.930076in}{0.641176in}}%
\pgfpathlineto{\pgfqpoint{2.931655in}{0.639741in}}%
\pgfpathlineto{\pgfqpoint{2.932444in}{0.635432in}}%
\pgfpathlineto{\pgfqpoint{2.933234in}{0.638402in}}%
\pgfpathlineto{\pgfqpoint{2.934023in}{0.640700in}}%
\pgfpathlineto{\pgfqpoint{2.934418in}{0.638399in}}%
\pgfpathlineto{\pgfqpoint{2.935997in}{0.634121in}}%
\pgfpathlineto{\pgfqpoint{2.936392in}{0.634213in}}%
\pgfpathlineto{\pgfqpoint{2.937971in}{0.632141in}}%
\pgfpathlineto{\pgfqpoint{2.938366in}{0.633261in}}%
\pgfpathlineto{\pgfqpoint{2.939550in}{0.634778in}}%
\pgfpathlineto{\pgfqpoint{2.939945in}{0.634083in}}%
\pgfpathlineto{\pgfqpoint{2.940734in}{0.634495in}}%
\pgfpathlineto{\pgfqpoint{2.941524in}{0.636580in}}%
\pgfpathlineto{\pgfqpoint{2.942313in}{0.635082in}}%
\pgfpathlineto{\pgfqpoint{2.943892in}{0.636396in}}%
\pgfpathlineto{\pgfqpoint{2.944287in}{0.634586in}}%
\pgfpathlineto{\pgfqpoint{2.944682in}{0.637902in}}%
\pgfpathlineto{\pgfqpoint{2.947445in}{0.649033in}}%
\pgfpathlineto{\pgfqpoint{2.948629in}{0.651691in}}%
\pgfpathlineto{\pgfqpoint{2.949419in}{0.649604in}}%
\pgfpathlineto{\pgfqpoint{2.951392in}{0.646035in}}%
\pgfpathlineto{\pgfqpoint{2.951787in}{0.647035in}}%
\pgfpathlineto{\pgfqpoint{2.953761in}{0.648013in}}%
\pgfpathlineto{\pgfqpoint{2.955340in}{0.646919in}}%
\pgfpathlineto{\pgfqpoint{2.957708in}{0.641636in}}%
\pgfpathlineto{\pgfqpoint{2.958103in}{0.642602in}}%
\pgfpathlineto{\pgfqpoint{2.958498in}{0.641639in}}%
\pgfpathlineto{\pgfqpoint{2.958893in}{0.639797in}}%
\pgfpathlineto{\pgfqpoint{2.959287in}{0.641498in}}%
\pgfpathlineto{\pgfqpoint{2.960472in}{0.646489in}}%
\pgfpathlineto{\pgfqpoint{2.960866in}{0.645798in}}%
\pgfpathlineto{\pgfqpoint{2.963629in}{0.640274in}}%
\pgfpathlineto{\pgfqpoint{2.966787in}{0.631608in}}%
\pgfpathlineto{\pgfqpoint{2.967577in}{0.632637in}}%
\pgfpathlineto{\pgfqpoint{2.967972in}{0.632689in}}%
\pgfpathlineto{\pgfqpoint{2.969945in}{0.630222in}}%
\pgfpathlineto{\pgfqpoint{2.970340in}{0.629553in}}%
\pgfpathlineto{\pgfqpoint{2.970735in}{0.630770in}}%
\pgfpathlineto{\pgfqpoint{2.973103in}{0.637322in}}%
\pgfpathlineto{\pgfqpoint{2.974288in}{0.635648in}}%
\pgfpathlineto{\pgfqpoint{2.976656in}{0.628509in}}%
\pgfpathlineto{\pgfqpoint{2.983367in}{0.634601in}}%
\pgfpathlineto{\pgfqpoint{2.986130in}{0.636584in}}%
\pgfpathlineto{\pgfqpoint{2.988104in}{0.638539in}}%
\pgfpathlineto{\pgfqpoint{2.991262in}{0.636754in}}%
\pgfpathlineto{\pgfqpoint{2.992051in}{0.637831in}}%
\pgfpathlineto{\pgfqpoint{2.992841in}{0.636736in}}%
\pgfpathlineto{\pgfqpoint{2.997578in}{0.633357in}}%
\pgfpathlineto{\pgfqpoint{2.998367in}{0.639404in}}%
\pgfpathlineto{\pgfqpoint{2.999157in}{0.637317in}}%
\pgfpathlineto{\pgfqpoint{3.001525in}{0.632094in}}%
\pgfpathlineto{\pgfqpoint{3.001920in}{0.634204in}}%
\pgfpathlineto{\pgfqpoint{3.003894in}{0.639612in}}%
\pgfpathlineto{\pgfqpoint{3.005868in}{0.633234in}}%
\pgfpathlineto{\pgfqpoint{3.008631in}{0.619201in}}%
\pgfpathlineto{\pgfqpoint{3.009815in}{0.621967in}}%
\pgfpathlineto{\pgfqpoint{3.012578in}{0.628211in}}%
\pgfpathlineto{\pgfqpoint{3.014552in}{0.628013in}}%
\pgfpathlineto{\pgfqpoint{3.015342in}{0.630256in}}%
\pgfpathlineto{\pgfqpoint{3.015736in}{0.627875in}}%
\pgfpathlineto{\pgfqpoint{3.017710in}{0.626237in}}%
\pgfpathlineto{\pgfqpoint{3.019289in}{0.626115in}}%
\pgfpathlineto{\pgfqpoint{3.022052in}{0.622040in}}%
\pgfpathlineto{\pgfqpoint{3.023237in}{0.623142in}}%
\pgfpathlineto{\pgfqpoint{3.026395in}{0.625484in}}%
\pgfpathlineto{\pgfqpoint{3.026789in}{0.624915in}}%
\pgfpathlineto{\pgfqpoint{3.028368in}{0.623691in}}%
\pgfpathlineto{\pgfqpoint{3.028763in}{0.624368in}}%
\pgfpathlineto{\pgfqpoint{3.029947in}{0.623492in}}%
\pgfpathlineto{\pgfqpoint{3.031921in}{0.622590in}}%
\pgfpathlineto{\pgfqpoint{3.034290in}{0.626600in}}%
\pgfpathlineto{\pgfqpoint{3.037053in}{0.631152in}}%
\pgfpathlineto{\pgfqpoint{3.038237in}{0.628899in}}%
\pgfpathlineto{\pgfqpoint{3.039421in}{0.624931in}}%
\pgfpathlineto{\pgfqpoint{3.039816in}{0.627279in}}%
\pgfpathlineto{\pgfqpoint{3.040211in}{0.628278in}}%
\pgfpathlineto{\pgfqpoint{3.041000in}{0.626684in}}%
\pgfpathlineto{\pgfqpoint{3.041790in}{0.624945in}}%
\pgfpathlineto{\pgfqpoint{3.042579in}{0.626112in}}%
\pgfpathlineto{\pgfqpoint{3.042974in}{0.626933in}}%
\pgfpathlineto{\pgfqpoint{3.043369in}{0.624985in}}%
\pgfpathlineto{\pgfqpoint{3.044553in}{0.621028in}}%
\pgfpathlineto{\pgfqpoint{3.044948in}{0.622935in}}%
\pgfpathlineto{\pgfqpoint{3.049685in}{0.634679in}}%
\pgfpathlineto{\pgfqpoint{3.050079in}{0.634200in}}%
\pgfpathlineto{\pgfqpoint{3.051658in}{0.631909in}}%
\pgfpathlineto{\pgfqpoint{3.052053in}{0.632772in}}%
\pgfpathlineto{\pgfqpoint{3.052843in}{0.631107in}}%
\pgfpathlineto{\pgfqpoint{3.057580in}{0.628112in}}%
\pgfpathlineto{\pgfqpoint{3.060343in}{0.625025in}}%
\pgfpathlineto{\pgfqpoint{3.061922in}{0.625959in}}%
\pgfpathlineto{\pgfqpoint{3.063501in}{0.632676in}}%
\pgfpathlineto{\pgfqpoint{3.064290in}{0.629330in}}%
\pgfpathlineto{\pgfqpoint{3.065475in}{0.624715in}}%
\pgfpathlineto{\pgfqpoint{3.066264in}{0.627265in}}%
\pgfpathlineto{\pgfqpoint{3.067448in}{0.629641in}}%
\pgfpathlineto{\pgfqpoint{3.068238in}{0.629097in}}%
\pgfpathlineto{\pgfqpoint{3.071396in}{0.629021in}}%
\pgfpathlineto{\pgfqpoint{3.072580in}{0.625281in}}%
\pgfpathlineto{\pgfqpoint{3.073370in}{0.627066in}}%
\pgfpathlineto{\pgfqpoint{3.074949in}{0.627062in}}%
\pgfpathlineto{\pgfqpoint{3.075738in}{0.625103in}}%
\pgfpathlineto{\pgfqpoint{3.076133in}{0.627279in}}%
\pgfpathlineto{\pgfqpoint{3.076528in}{0.627548in}}%
\pgfpathlineto{\pgfqpoint{3.076922in}{0.626204in}}%
\pgfpathlineto{\pgfqpoint{3.078896in}{0.622272in}}%
\pgfpathlineto{\pgfqpoint{3.081265in}{0.629211in}}%
\pgfpathlineto{\pgfqpoint{3.082054in}{0.629905in}}%
\pgfpathlineto{\pgfqpoint{3.082449in}{0.628695in}}%
\pgfpathlineto{\pgfqpoint{3.085212in}{0.619674in}}%
\pgfpathlineto{\pgfqpoint{3.085607in}{0.620256in}}%
\pgfpathlineto{\pgfqpoint{3.090344in}{0.627922in}}%
\pgfpathlineto{\pgfqpoint{3.092712in}{0.635718in}}%
\pgfpathlineto{\pgfqpoint{3.093107in}{0.635414in}}%
\pgfpathlineto{\pgfqpoint{3.095870in}{0.624262in}}%
\pgfpathlineto{\pgfqpoint{3.097055in}{0.623296in}}%
\pgfpathlineto{\pgfqpoint{3.098239in}{0.617427in}}%
\pgfpathlineto{\pgfqpoint{3.099028in}{0.618905in}}%
\pgfpathlineto{\pgfqpoint{3.102186in}{0.610046in}}%
\pgfpathlineto{\pgfqpoint{3.107318in}{0.624529in}}%
\pgfpathlineto{\pgfqpoint{3.107713in}{0.624076in}}%
\pgfpathlineto{\pgfqpoint{3.108897in}{0.623872in}}%
\pgfpathlineto{\pgfqpoint{3.109292in}{0.624719in}}%
\pgfpathlineto{\pgfqpoint{3.110476in}{0.628636in}}%
\pgfpathlineto{\pgfqpoint{3.110871in}{0.627336in}}%
\pgfpathlineto{\pgfqpoint{3.112845in}{0.623011in}}%
\pgfpathlineto{\pgfqpoint{3.114029in}{0.623282in}}%
\pgfpathlineto{\pgfqpoint{3.116792in}{0.616558in}}%
\pgfpathlineto{\pgfqpoint{3.117187in}{0.618537in}}%
\pgfpathlineto{\pgfqpoint{3.121134in}{0.626270in}}%
\pgfpathlineto{\pgfqpoint{3.123503in}{0.634700in}}%
\pgfpathlineto{\pgfqpoint{3.123898in}{0.633362in}}%
\pgfpathlineto{\pgfqpoint{3.125477in}{0.629899in}}%
\pgfpathlineto{\pgfqpoint{3.127845in}{0.625824in}}%
\pgfpathlineto{\pgfqpoint{3.128240in}{0.625349in}}%
\pgfpathlineto{\pgfqpoint{3.128635in}{0.626082in}}%
\pgfpathlineto{\pgfqpoint{3.129819in}{0.628225in}}%
\pgfpathlineto{\pgfqpoint{3.130213in}{0.627473in}}%
\pgfpathlineto{\pgfqpoint{3.131398in}{0.628193in}}%
\pgfpathlineto{\pgfqpoint{3.131792in}{0.626369in}}%
\pgfpathlineto{\pgfqpoint{3.133371in}{0.630872in}}%
\pgfpathlineto{\pgfqpoint{3.133766in}{0.629355in}}%
\pgfpathlineto{\pgfqpoint{3.134950in}{0.630289in}}%
\pgfpathlineto{\pgfqpoint{3.136924in}{0.636200in}}%
\pgfpathlineto{\pgfqpoint{3.137319in}{0.636091in}}%
\pgfpathlineto{\pgfqpoint{3.140872in}{0.625957in}}%
\pgfpathlineto{\pgfqpoint{3.141661in}{0.629106in}}%
\pgfpathlineto{\pgfqpoint{3.144424in}{0.623715in}}%
\pgfpathlineto{\pgfqpoint{3.145214in}{0.624600in}}%
\pgfpathlineto{\pgfqpoint{3.148767in}{0.630383in}}%
\pgfpathlineto{\pgfqpoint{3.149161in}{0.629613in}}%
\pgfpathlineto{\pgfqpoint{3.149556in}{0.630483in}}%
\pgfpathlineto{\pgfqpoint{3.150740in}{0.631833in}}%
\pgfpathlineto{\pgfqpoint{3.151135in}{0.631598in}}%
\pgfpathlineto{\pgfqpoint{3.152714in}{0.628184in}}%
\pgfpathlineto{\pgfqpoint{3.154688in}{0.621452in}}%
\pgfpathlineto{\pgfqpoint{3.155083in}{0.622476in}}%
\pgfpathlineto{\pgfqpoint{3.155872in}{0.624710in}}%
\pgfpathlineto{\pgfqpoint{3.156267in}{0.621973in}}%
\pgfpathlineto{\pgfqpoint{3.157451in}{0.617321in}}%
\pgfpathlineto{\pgfqpoint{3.158241in}{0.619512in}}%
\pgfpathlineto{\pgfqpoint{3.159425in}{0.621391in}}%
\pgfpathlineto{\pgfqpoint{3.159820in}{0.620392in}}%
\pgfpathlineto{\pgfqpoint{3.161004in}{0.622012in}}%
\pgfpathlineto{\pgfqpoint{3.161793in}{0.624534in}}%
\pgfpathlineto{\pgfqpoint{3.162583in}{0.623399in}}%
\pgfpathlineto{\pgfqpoint{3.164557in}{0.625793in}}%
\pgfpathlineto{\pgfqpoint{3.164951in}{0.628150in}}%
\pgfpathlineto{\pgfqpoint{3.165741in}{0.625099in}}%
\pgfpathlineto{\pgfqpoint{3.166925in}{0.627106in}}%
\pgfpathlineto{\pgfqpoint{3.167320in}{0.626617in}}%
\pgfpathlineto{\pgfqpoint{3.167715in}{0.625125in}}%
\pgfpathlineto{\pgfqpoint{3.168504in}{0.627022in}}%
\pgfpathlineto{\pgfqpoint{3.170873in}{0.629797in}}%
\pgfpathlineto{\pgfqpoint{3.174820in}{0.632441in}}%
\pgfpathlineto{\pgfqpoint{3.176004in}{0.629593in}}%
\pgfpathlineto{\pgfqpoint{3.176399in}{0.630551in}}%
\pgfpathlineto{\pgfqpoint{3.176794in}{0.631498in}}%
\pgfpathlineto{\pgfqpoint{3.177583in}{0.630538in}}%
\pgfpathlineto{\pgfqpoint{3.181531in}{0.625161in}}%
\pgfpathlineto{\pgfqpoint{3.183899in}{0.630613in}}%
\pgfpathlineto{\pgfqpoint{3.185084in}{0.631215in}}%
\pgfpathlineto{\pgfqpoint{3.185478in}{0.629926in}}%
\pgfpathlineto{\pgfqpoint{3.188242in}{0.625238in}}%
\pgfpathlineto{\pgfqpoint{3.188636in}{0.626705in}}%
\pgfpathlineto{\pgfqpoint{3.191400in}{0.633767in}}%
\pgfpathlineto{\pgfqpoint{3.192189in}{0.634974in}}%
\pgfpathlineto{\pgfqpoint{3.193768in}{0.639989in}}%
\pgfpathlineto{\pgfqpoint{3.194558in}{0.638432in}}%
\pgfpathlineto{\pgfqpoint{3.196531in}{0.632053in}}%
\pgfpathlineto{\pgfqpoint{3.196926in}{0.632263in}}%
\pgfpathlineto{\pgfqpoint{3.197321in}{0.634154in}}%
\pgfpathlineto{\pgfqpoint{3.197321in}{0.634154in}}%
\pgfusepath{stroke}%
\end{pgfscope}%
\begin{pgfscope}%
\pgfpathrectangle{\pgfqpoint{0.608025in}{0.484444in}}{\pgfqpoint{2.712595in}{1.541287in}}%
\pgfusepath{clip}%
\pgfsetbuttcap%
\pgfsetroundjoin%
\definecolor{currentfill}{rgb}{0.549020,0.337255,0.294118}%
\pgfsetfillcolor{currentfill}%
\pgfsetlinewidth{1.003750pt}%
\definecolor{currentstroke}{rgb}{0.549020,0.337255,0.294118}%
\pgfsetstrokecolor{currentstroke}%
\pgfsetdash{}{0pt}%
\pgfsys@defobject{currentmarker}{\pgfqpoint{-0.020833in}{-0.020833in}}{\pgfqpoint{0.020833in}{0.020833in}}{%
\pgfpathmoveto{\pgfqpoint{-0.020833in}{-0.020833in}}%
\pgfpathlineto{\pgfqpoint{0.020833in}{0.020833in}}%
\pgfpathmoveto{\pgfqpoint{-0.020833in}{0.020833in}}%
\pgfpathlineto{\pgfqpoint{0.020833in}{-0.020833in}}%
\pgfusepath{stroke,fill}%
}%
\begin{pgfscope}%
\pgfsys@transformshift{0.762905in}{1.910689in}%
\pgfsys@useobject{currentmarker}{}%
\end{pgfscope}%
\begin{pgfscope}%
\pgfsys@transformshift{0.841855in}{1.597952in}%
\pgfsys@useobject{currentmarker}{}%
\end{pgfscope}%
\begin{pgfscope}%
\pgfsys@transformshift{0.920804in}{1.327057in}%
\pgfsys@useobject{currentmarker}{}%
\end{pgfscope}%
\begin{pgfscope}%
\pgfsys@transformshift{0.999754in}{1.120366in}%
\pgfsys@useobject{currentmarker}{}%
\end{pgfscope}%
\begin{pgfscope}%
\pgfsys@transformshift{1.078704in}{0.935202in}%
\pgfsys@useobject{currentmarker}{}%
\end{pgfscope}%
\begin{pgfscope}%
\pgfsys@transformshift{1.157654in}{0.789483in}%
\pgfsys@useobject{currentmarker}{}%
\end{pgfscope}%
\begin{pgfscope}%
\pgfsys@transformshift{1.236603in}{0.720261in}%
\pgfsys@useobject{currentmarker}{}%
\end{pgfscope}%
\begin{pgfscope}%
\pgfsys@transformshift{1.315553in}{0.687976in}%
\pgfsys@useobject{currentmarker}{}%
\end{pgfscope}%
\begin{pgfscope}%
\pgfsys@transformshift{1.394503in}{0.638034in}%
\pgfsys@useobject{currentmarker}{}%
\end{pgfscope}%
\begin{pgfscope}%
\pgfsys@transformshift{1.473453in}{0.641944in}%
\pgfsys@useobject{currentmarker}{}%
\end{pgfscope}%
\begin{pgfscope}%
\pgfsys@transformshift{1.552402in}{0.626082in}%
\pgfsys@useobject{currentmarker}{}%
\end{pgfscope}%
\begin{pgfscope}%
\pgfsys@transformshift{1.631352in}{0.634115in}%
\pgfsys@useobject{currentmarker}{}%
\end{pgfscope}%
\begin{pgfscope}%
\pgfsys@transformshift{1.710302in}{0.648693in}%
\pgfsys@useobject{currentmarker}{}%
\end{pgfscope}%
\begin{pgfscope}%
\pgfsys@transformshift{1.789252in}{0.636149in}%
\pgfsys@useobject{currentmarker}{}%
\end{pgfscope}%
\begin{pgfscope}%
\pgfsys@transformshift{1.868202in}{0.616894in}%
\pgfsys@useobject{currentmarker}{}%
\end{pgfscope}%
\begin{pgfscope}%
\pgfsys@transformshift{1.947151in}{0.621899in}%
\pgfsys@useobject{currentmarker}{}%
\end{pgfscope}%
\begin{pgfscope}%
\pgfsys@transformshift{2.026101in}{0.639022in}%
\pgfsys@useobject{currentmarker}{}%
\end{pgfscope}%
\begin{pgfscope}%
\pgfsys@transformshift{2.105051in}{0.631255in}%
\pgfsys@useobject{currentmarker}{}%
\end{pgfscope}%
\begin{pgfscope}%
\pgfsys@transformshift{2.184001in}{0.661606in}%
\pgfsys@useobject{currentmarker}{}%
\end{pgfscope}%
\begin{pgfscope}%
\pgfsys@transformshift{2.262950in}{0.646696in}%
\pgfsys@useobject{currentmarker}{}%
\end{pgfscope}%
\begin{pgfscope}%
\pgfsys@transformshift{2.341900in}{0.610201in}%
\pgfsys@useobject{currentmarker}{}%
\end{pgfscope}%
\begin{pgfscope}%
\pgfsys@transformshift{2.420850in}{0.619751in}%
\pgfsys@useobject{currentmarker}{}%
\end{pgfscope}%
\begin{pgfscope}%
\pgfsys@transformshift{2.499800in}{0.617734in}%
\pgfsys@useobject{currentmarker}{}%
\end{pgfscope}%
\begin{pgfscope}%
\pgfsys@transformshift{2.578749in}{0.637794in}%
\pgfsys@useobject{currentmarker}{}%
\end{pgfscope}%
\begin{pgfscope}%
\pgfsys@transformshift{2.657699in}{0.636988in}%
\pgfsys@useobject{currentmarker}{}%
\end{pgfscope}%
\begin{pgfscope}%
\pgfsys@transformshift{2.736649in}{0.597916in}%
\pgfsys@useobject{currentmarker}{}%
\end{pgfscope}%
\begin{pgfscope}%
\pgfsys@transformshift{2.815599in}{0.623155in}%
\pgfsys@useobject{currentmarker}{}%
\end{pgfscope}%
\begin{pgfscope}%
\pgfsys@transformshift{2.894548in}{0.632260in}%
\pgfsys@useobject{currentmarker}{}%
\end{pgfscope}%
\begin{pgfscope}%
\pgfsys@transformshift{2.973498in}{0.636942in}%
\pgfsys@useobject{currentmarker}{}%
\end{pgfscope}%
\begin{pgfscope}%
\pgfsys@transformshift{3.052448in}{0.632099in}%
\pgfsys@useobject{currentmarker}{}%
\end{pgfscope}%
\begin{pgfscope}%
\pgfsys@transformshift{3.131398in}{0.628193in}%
\pgfsys@useobject{currentmarker}{}%
\end{pgfscope}%
\end{pgfscope}%
\begin{pgfscope}%
\pgfsetrectcap%
\pgfsetmiterjoin%
\pgfsetlinewidth{0.803000pt}%
\definecolor{currentstroke}{rgb}{0.000000,0.000000,0.000000}%
\pgfsetstrokecolor{currentstroke}%
\pgfsetdash{}{0pt}%
\pgfpathmoveto{\pgfqpoint{0.608025in}{0.484444in}}%
\pgfpathlineto{\pgfqpoint{0.608025in}{2.025731in}}%
\pgfusepath{stroke}%
\end{pgfscope}%
\begin{pgfscope}%
\pgfsetrectcap%
\pgfsetmiterjoin%
\pgfsetlinewidth{0.803000pt}%
\definecolor{currentstroke}{rgb}{0.000000,0.000000,0.000000}%
\pgfsetstrokecolor{currentstroke}%
\pgfsetdash{}{0pt}%
\pgfpathmoveto{\pgfqpoint{3.320621in}{0.484444in}}%
\pgfpathlineto{\pgfqpoint{3.320621in}{2.025731in}}%
\pgfusepath{stroke}%
\end{pgfscope}%
\begin{pgfscope}%
\pgfsetrectcap%
\pgfsetmiterjoin%
\pgfsetlinewidth{0.803000pt}%
\definecolor{currentstroke}{rgb}{0.000000,0.000000,0.000000}%
\pgfsetstrokecolor{currentstroke}%
\pgfsetdash{}{0pt}%
\pgfpathmoveto{\pgfqpoint{0.608025in}{0.484444in}}%
\pgfpathlineto{\pgfqpoint{3.320621in}{0.484444in}}%
\pgfusepath{stroke}%
\end{pgfscope}%
\begin{pgfscope}%
\pgfsetrectcap%
\pgfsetmiterjoin%
\pgfsetlinewidth{0.803000pt}%
\definecolor{currentstroke}{rgb}{0.000000,0.000000,0.000000}%
\pgfsetstrokecolor{currentstroke}%
\pgfsetdash{}{0pt}%
\pgfpathmoveto{\pgfqpoint{0.608025in}{2.025731in}}%
\pgfpathlineto{\pgfqpoint{3.320621in}{2.025731in}}%
\pgfusepath{stroke}%
\end{pgfscope}%
\begin{pgfscope}%
\pgfsetbuttcap%
\pgfsetmiterjoin%
\definecolor{currentfill}{rgb}{1.000000,1.000000,1.000000}%
\pgfsetfillcolor{currentfill}%
\pgfsetfillopacity{0.800000}%
\pgfsetlinewidth{1.003750pt}%
\definecolor{currentstroke}{rgb}{0.800000,0.800000,0.800000}%
\pgfsetstrokecolor{currentstroke}%
\pgfsetstrokeopacity{0.800000}%
\pgfsetdash{}{0pt}%
\pgfpathmoveto{\pgfqpoint{0.891116in}{1.540074in}}%
\pgfpathlineto{\pgfqpoint{3.252565in}{1.540074in}}%
\pgfpathquadraticcurveto{\pgfqpoint{3.272009in}{1.540074in}}{\pgfqpoint{3.272009in}{1.559518in}}%
\pgfpathlineto{\pgfqpoint{3.272009in}{1.957675in}}%
\pgfpathquadraticcurveto{\pgfqpoint{3.272009in}{1.977120in}}{\pgfqpoint{3.252565in}{1.977120in}}%
\pgfpathlineto{\pgfqpoint{0.891116in}{1.977120in}}%
\pgfpathquadraticcurveto{\pgfqpoint{0.871671in}{1.977120in}}{\pgfqpoint{0.871671in}{1.957675in}}%
\pgfpathlineto{\pgfqpoint{0.871671in}{1.559518in}}%
\pgfpathquadraticcurveto{\pgfqpoint{0.871671in}{1.540074in}}{\pgfqpoint{0.891116in}{1.540074in}}%
\pgfpathlineto{\pgfqpoint{0.891116in}{1.540074in}}%
\pgfpathclose%
\pgfusepath{stroke,fill}%
\end{pgfscope}%
\begin{pgfscope}%
\pgfsetrectcap%
\pgfsetroundjoin%
\pgfsetlinewidth{1.505625pt}%
\definecolor{currentstroke}{rgb}{0.121569,0.466667,0.705882}%
\pgfsetstrokecolor{currentstroke}%
\pgfsetdash{}{0pt}%
\pgfpathmoveto{\pgfqpoint{0.910560in}{1.903814in}}%
\pgfpathlineto{\pgfqpoint{1.007782in}{1.903814in}}%
\pgfpathlineto{\pgfqpoint{1.105005in}{1.903814in}}%
\pgfusepath{stroke}%
\end{pgfscope}%
\begin{pgfscope}%
\pgfsetbuttcap%
\pgfsetmiterjoin%
\definecolor{currentfill}{rgb}{0.121569,0.466667,0.705882}%
\pgfsetfillcolor{currentfill}%
\pgfsetlinewidth{1.003750pt}%
\definecolor{currentstroke}{rgb}{0.121569,0.466667,0.705882}%
\pgfsetstrokecolor{currentstroke}%
\pgfsetdash{}{0pt}%
\pgfsys@defobject{currentmarker}{\pgfqpoint{-0.020833in}{-0.020833in}}{\pgfqpoint{0.020833in}{0.020833in}}{%
\pgfpathmoveto{\pgfqpoint{-0.020833in}{-0.020833in}}%
\pgfpathlineto{\pgfqpoint{0.020833in}{-0.020833in}}%
\pgfpathlineto{\pgfqpoint{0.020833in}{0.020833in}}%
\pgfpathlineto{\pgfqpoint{-0.020833in}{0.020833in}}%
\pgfpathlineto{\pgfqpoint{-0.020833in}{-0.020833in}}%
\pgfpathclose%
\pgfusepath{stroke,fill}%
}%
\begin{pgfscope}%
\pgfsys@transformshift{1.007782in}{1.903814in}%
\pgfsys@useobject{currentmarker}{}%
\end{pgfscope}%
\end{pgfscope}%
\begin{pgfscope}%
\definecolor{textcolor}{rgb}{0.000000,0.000000,0.000000}%
\pgfsetstrokecolor{textcolor}%
\pgfsetfillcolor{textcolor}%
\pgftext[x=1.182782in,y=1.869787in,left,base]{\color{textcolor}\rmfamily\fontsize{7.000000}{8.400000}\selectfont optimal}%
\end{pgfscope}%
\begin{pgfscope}%
\pgfsetrectcap%
\pgfsetroundjoin%
\pgfsetlinewidth{1.505625pt}%
\definecolor{currentstroke}{rgb}{1.000000,0.498039,0.054902}%
\pgfsetstrokecolor{currentstroke}%
\pgfsetdash{}{0pt}%
\pgfpathmoveto{\pgfqpoint{0.910560in}{1.768287in}}%
\pgfpathlineto{\pgfqpoint{1.007782in}{1.768287in}}%
\pgfpathlineto{\pgfqpoint{1.105005in}{1.768287in}}%
\pgfusepath{stroke}%
\end{pgfscope}%
\begin{pgfscope}%
\pgfsetbuttcap%
\pgfsetroundjoin%
\definecolor{currentfill}{rgb}{1.000000,0.498039,0.054902}%
\pgfsetfillcolor{currentfill}%
\pgfsetlinewidth{1.003750pt}%
\definecolor{currentstroke}{rgb}{1.000000,0.498039,0.054902}%
\pgfsetstrokecolor{currentstroke}%
\pgfsetdash{}{0pt}%
\pgfsys@defobject{currentmarker}{\pgfqpoint{-0.020833in}{-0.020833in}}{\pgfqpoint{0.020833in}{0.020833in}}{%
\pgfpathmoveto{\pgfqpoint{0.000000in}{-0.020833in}}%
\pgfpathcurveto{\pgfqpoint{0.005525in}{-0.020833in}}{\pgfqpoint{0.010825in}{-0.018638in}}{\pgfqpoint{0.014731in}{-0.014731in}}%
\pgfpathcurveto{\pgfqpoint{0.018638in}{-0.010825in}}{\pgfqpoint{0.020833in}{-0.005525in}}{\pgfqpoint{0.020833in}{0.000000in}}%
\pgfpathcurveto{\pgfqpoint{0.020833in}{0.005525in}}{\pgfqpoint{0.018638in}{0.010825in}}{\pgfqpoint{0.014731in}{0.014731in}}%
\pgfpathcurveto{\pgfqpoint{0.010825in}{0.018638in}}{\pgfqpoint{0.005525in}{0.020833in}}{\pgfqpoint{0.000000in}{0.020833in}}%
\pgfpathcurveto{\pgfqpoint{-0.005525in}{0.020833in}}{\pgfqpoint{-0.010825in}{0.018638in}}{\pgfqpoint{-0.014731in}{0.014731in}}%
\pgfpathcurveto{\pgfqpoint{-0.018638in}{0.010825in}}{\pgfqpoint{-0.020833in}{0.005525in}}{\pgfqpoint{-0.020833in}{0.000000in}}%
\pgfpathcurveto{\pgfqpoint{-0.020833in}{-0.005525in}}{\pgfqpoint{-0.018638in}{-0.010825in}}{\pgfqpoint{-0.014731in}{-0.014731in}}%
\pgfpathcurveto{\pgfqpoint{-0.010825in}{-0.018638in}}{\pgfqpoint{-0.005525in}{-0.020833in}}{\pgfqpoint{0.000000in}{-0.020833in}}%
\pgfpathlineto{\pgfqpoint{0.000000in}{-0.020833in}}%
\pgfpathclose%
\pgfusepath{stroke,fill}%
}%
\begin{pgfscope}%
\pgfsys@transformshift{1.007782in}{1.768287in}%
\pgfsys@useobject{currentmarker}{}%
\end{pgfscope}%
\end{pgfscope}%
\begin{pgfscope}%
\definecolor{textcolor}{rgb}{0.000000,0.000000,0.000000}%
\pgfsetstrokecolor{textcolor}%
\pgfsetfillcolor{textcolor}%
\pgftext[x=1.182782in,y=1.734259in,left,base]{\color{textcolor}\rmfamily\fontsize{7.000000}{8.400000}\selectfont adaptive \(\displaystyle \gamma_i,\,K=1\)}%
\end{pgfscope}%
\begin{pgfscope}%
\pgfsetrectcap%
\pgfsetroundjoin%
\pgfsetlinewidth{1.505625pt}%
\definecolor{currentstroke}{rgb}{0.172549,0.627451,0.172549}%
\pgfsetstrokecolor{currentstroke}%
\pgfsetdash{}{0pt}%
\pgfpathmoveto{\pgfqpoint{0.910560in}{1.632327in}}%
\pgfpathlineto{\pgfqpoint{1.007782in}{1.632327in}}%
\pgfpathlineto{\pgfqpoint{1.105005in}{1.632327in}}%
\pgfusepath{stroke}%
\end{pgfscope}%
\begin{pgfscope}%
\pgfsetbuttcap%
\pgfsetmiterjoin%
\definecolor{currentfill}{rgb}{0.172549,0.627451,0.172549}%
\pgfsetfillcolor{currentfill}%
\pgfsetlinewidth{1.003750pt}%
\definecolor{currentstroke}{rgb}{0.172549,0.627451,0.172549}%
\pgfsetstrokecolor{currentstroke}%
\pgfsetdash{}{0pt}%
\pgfsys@defobject{currentmarker}{\pgfqpoint{-0.020833in}{-0.020833in}}{\pgfqpoint{0.020833in}{0.020833in}}{%
\pgfpathmoveto{\pgfqpoint{-0.000000in}{-0.020833in}}%
\pgfpathlineto{\pgfqpoint{0.020833in}{0.020833in}}%
\pgfpathlineto{\pgfqpoint{-0.020833in}{0.020833in}}%
\pgfpathlineto{\pgfqpoint{-0.000000in}{-0.020833in}}%
\pgfpathclose%
\pgfusepath{stroke,fill}%
}%
\begin{pgfscope}%
\pgfsys@transformshift{1.007782in}{1.632327in}%
\pgfsys@useobject{currentmarker}{}%
\end{pgfscope}%
\end{pgfscope}%
\begin{pgfscope}%
\definecolor{textcolor}{rgb}{0.000000,0.000000,0.000000}%
\pgfsetstrokecolor{textcolor}%
\pgfsetfillcolor{textcolor}%
\pgftext[x=1.182782in,y=1.598299in,left,base]{\color{textcolor}\rmfamily\fontsize{7.000000}{8.400000}\selectfont fixed \(\displaystyle \gamma_i,\,K=1\)}%
\end{pgfscope}%
\begin{pgfscope}%
\pgfsetrectcap%
\pgfsetroundjoin%
\pgfsetlinewidth{1.505625pt}%
\definecolor{currentstroke}{rgb}{0.839216,0.152941,0.156863}%
\pgfsetstrokecolor{currentstroke}%
\pgfsetdash{}{0pt}%
\pgfpathmoveto{\pgfqpoint{2.158881in}{1.903814in}}%
\pgfpathlineto{\pgfqpoint{2.256103in}{1.903814in}}%
\pgfpathlineto{\pgfqpoint{2.353326in}{1.903814in}}%
\pgfusepath{stroke}%
\end{pgfscope}%
\begin{pgfscope}%
\pgfsetbuttcap%
\pgfsetmiterjoin%
\definecolor{currentfill}{rgb}{0.839216,0.152941,0.156863}%
\pgfsetfillcolor{currentfill}%
\pgfsetlinewidth{1.003750pt}%
\definecolor{currentstroke}{rgb}{0.839216,0.152941,0.156863}%
\pgfsetstrokecolor{currentstroke}%
\pgfsetdash{}{0pt}%
\pgfsys@defobject{currentmarker}{\pgfqpoint{-0.020833in}{-0.020833in}}{\pgfqpoint{0.020833in}{0.020833in}}{%
\pgfpathmoveto{\pgfqpoint{0.020833in}{-0.000000in}}%
\pgfpathlineto{\pgfqpoint{-0.020833in}{0.020833in}}%
\pgfpathlineto{\pgfqpoint{-0.020833in}{-0.020833in}}%
\pgfpathlineto{\pgfqpoint{0.020833in}{-0.000000in}}%
\pgfpathclose%
\pgfusepath{stroke,fill}%
}%
\begin{pgfscope}%
\pgfsys@transformshift{2.256103in}{1.903814in}%
\pgfsys@useobject{currentmarker}{}%
\end{pgfscope}%
\end{pgfscope}%
\begin{pgfscope}%
\definecolor{textcolor}{rgb}{0.000000,0.000000,0.000000}%
\pgfsetstrokecolor{textcolor}%
\pgfsetfillcolor{textcolor}%
\pgftext[x=2.431103in,y=1.869787in,left,base]{\color{textcolor}\rmfamily\fontsize{7.000000}{8.400000}\selectfont fixed \(\displaystyle \gamma_i,\,K=10\)}%
\end{pgfscope}%
\begin{pgfscope}%
\pgfsetrectcap%
\pgfsetroundjoin%
\pgfsetlinewidth{1.505625pt}%
\definecolor{currentstroke}{rgb}{0.580392,0.403922,0.741176}%
\pgfsetstrokecolor{currentstroke}%
\pgfsetdash{}{0pt}%
\pgfpathmoveto{\pgfqpoint{2.158881in}{1.767855in}}%
\pgfpathlineto{\pgfqpoint{2.256103in}{1.767855in}}%
\pgfpathlineto{\pgfqpoint{2.353326in}{1.767855in}}%
\pgfusepath{stroke}%
\end{pgfscope}%
\begin{pgfscope}%
\pgfsetbuttcap%
\pgfsetmiterjoin%
\definecolor{currentfill}{rgb}{0.580392,0.403922,0.741176}%
\pgfsetfillcolor{currentfill}%
\pgfsetlinewidth{1.003750pt}%
\definecolor{currentstroke}{rgb}{0.580392,0.403922,0.741176}%
\pgfsetstrokecolor{currentstroke}%
\pgfsetdash{}{0pt}%
\pgfsys@defobject{currentmarker}{\pgfqpoint{-0.020833in}{-0.020833in}}{\pgfqpoint{0.020833in}{0.020833in}}{%
\pgfpathmoveto{\pgfqpoint{-0.020833in}{0.000000in}}%
\pgfpathlineto{\pgfqpoint{0.020833in}{-0.020833in}}%
\pgfpathlineto{\pgfqpoint{0.020833in}{0.020833in}}%
\pgfpathlineto{\pgfqpoint{-0.020833in}{0.000000in}}%
\pgfpathclose%
\pgfusepath{stroke,fill}%
}%
\begin{pgfscope}%
\pgfsys@transformshift{2.256103in}{1.767855in}%
\pgfsys@useobject{currentmarker}{}%
\end{pgfscope}%
\end{pgfscope}%
\begin{pgfscope}%
\definecolor{textcolor}{rgb}{0.000000,0.000000,0.000000}%
\pgfsetstrokecolor{textcolor}%
\pgfsetfillcolor{textcolor}%
\pgftext[x=2.431103in,y=1.733827in,left,base]{\color{textcolor}\rmfamily\fontsize{7.000000}{8.400000}\selectfont fixed \(\displaystyle \gamma_i,\,K=1\)}%
\end{pgfscope}%
\begin{pgfscope}%
\pgfsetrectcap%
\pgfsetroundjoin%
\pgfsetlinewidth{1.505625pt}%
\definecolor{currentstroke}{rgb}{0.549020,0.337255,0.294118}%
\pgfsetstrokecolor{currentstroke}%
\pgfsetdash{}{0pt}%
\pgfpathmoveto{\pgfqpoint{2.158881in}{1.631895in}}%
\pgfpathlineto{\pgfqpoint{2.256103in}{1.631895in}}%
\pgfpathlineto{\pgfqpoint{2.353326in}{1.631895in}}%
\pgfusepath{stroke}%
\end{pgfscope}%
\begin{pgfscope}%
\pgfsetbuttcap%
\pgfsetroundjoin%
\definecolor{currentfill}{rgb}{0.549020,0.337255,0.294118}%
\pgfsetfillcolor{currentfill}%
\pgfsetlinewidth{1.003750pt}%
\definecolor{currentstroke}{rgb}{0.549020,0.337255,0.294118}%
\pgfsetstrokecolor{currentstroke}%
\pgfsetdash{}{0pt}%
\pgfsys@defobject{currentmarker}{\pgfqpoint{-0.020833in}{-0.020833in}}{\pgfqpoint{0.020833in}{0.020833in}}{%
\pgfpathmoveto{\pgfqpoint{-0.020833in}{-0.020833in}}%
\pgfpathlineto{\pgfqpoint{0.020833in}{0.020833in}}%
\pgfpathmoveto{\pgfqpoint{-0.020833in}{0.020833in}}%
\pgfpathlineto{\pgfqpoint{0.020833in}{-0.020833in}}%
\pgfusepath{stroke,fill}%
}%
\begin{pgfscope}%
\pgfsys@transformshift{2.256103in}{1.631895in}%
\pgfsys@useobject{currentmarker}{}%
\end{pgfscope}%
\end{pgfscope}%
\begin{pgfscope}%
\definecolor{textcolor}{rgb}{0.000000,0.000000,0.000000}%
\pgfsetstrokecolor{textcolor}%
\pgfsetfillcolor{textcolor}%
\pgftext[x=2.431103in,y=1.597867in,left,base]{\color{textcolor}\rmfamily\fontsize{7.000000}{8.400000}\selectfont fixed \(\displaystyle \gamma_i,\,K=10\)}%
\end{pgfscope}%
\end{pgfpicture}%
\makeatother%
\endgroup%

    \vspace*{-0.6cm}
    \caption[]{Median NPM of 30 Monte-Carlo runs over time for optimal case where all node information is available (fully connected network/broadcasting), fixed values of \(\gamma_i\) with \(K \in \{1,10\}\), adaptive \(\gamma_i\) with \(K=1\). A moving average filter was applied to result curves for better readability (compare transparent and opaque lines). Hatched area indicates frames used in computations for \autoref{fig:simulations:avgNPMgamma}.}
    \label{fig:simulations:NPMtime}
\end{figure}
\begin{figure}[t]
    \centering
    %% Creator: Matplotlib, PGF backend
%%
%% To include the figure in your LaTeX document, write
%%   \input{<filename>.pgf}
%%
%% Make sure the required packages are loaded in your preamble
%%   \usepackage{pgf}
%%
%% Also ensure that all the required font packages are loaded; for instance,
%% the lmodern package is sometimes necessary when using math font.
%%   \usepackage{lmodern}
%%
%% Figures using additional raster images can only be included by \input if
%% they are in the same directory as the main LaTeX file. For loading figures
%% from other directories you can use the `import` package
%%   \usepackage{import}
%%
%% and then include the figures with
%%   \import{<path to file>}{<filename>.pgf}
%%
%% Matplotlib used the following preamble
%%   \usepackage{fontspec}
%%
\begingroup%
\makeatletter%
\begin{pgfpicture}%
\pgfpathrectangle{\pgfpointorigin}{\pgfqpoint{3.390065in}{2.095175in}}%
\pgfusepath{use as bounding box, clip}%
\begin{pgfscope}%
\pgfsetbuttcap%
\pgfsetmiterjoin%
\definecolor{currentfill}{rgb}{1.000000,1.000000,1.000000}%
\pgfsetfillcolor{currentfill}%
\pgfsetlinewidth{0.000000pt}%
\definecolor{currentstroke}{rgb}{1.000000,1.000000,1.000000}%
\pgfsetstrokecolor{currentstroke}%
\pgfsetstrokeopacity{0.000000}%
\pgfsetdash{}{0pt}%
\pgfpathmoveto{\pgfqpoint{0.000000in}{0.000000in}}%
\pgfpathlineto{\pgfqpoint{3.390065in}{0.000000in}}%
\pgfpathlineto{\pgfqpoint{3.390065in}{2.095175in}}%
\pgfpathlineto{\pgfqpoint{0.000000in}{2.095175in}}%
\pgfpathlineto{\pgfqpoint{0.000000in}{0.000000in}}%
\pgfpathclose%
\pgfusepath{fill}%
\end{pgfscope}%
\begin{pgfscope}%
\pgfsetbuttcap%
\pgfsetmiterjoin%
\definecolor{currentfill}{rgb}{1.000000,1.000000,1.000000}%
\pgfsetfillcolor{currentfill}%
\pgfsetlinewidth{0.000000pt}%
\definecolor{currentstroke}{rgb}{0.000000,0.000000,0.000000}%
\pgfsetstrokecolor{currentstroke}%
\pgfsetstrokeopacity{0.000000}%
\pgfsetdash{}{0pt}%
\pgfpathmoveto{\pgfqpoint{0.568671in}{0.451277in}}%
\pgfpathlineto{\pgfqpoint{3.213368in}{0.451277in}}%
\pgfpathlineto{\pgfqpoint{3.213368in}{1.989300in}}%
\pgfpathlineto{\pgfqpoint{0.568671in}{1.989300in}}%
\pgfpathlineto{\pgfqpoint{0.568671in}{0.451277in}}%
\pgfpathclose%
\pgfusepath{fill}%
\end{pgfscope}%
\begin{pgfscope}%
\pgfpathrectangle{\pgfqpoint{0.568671in}{0.451277in}}{\pgfqpoint{2.644697in}{1.538023in}}%
\pgfusepath{clip}%
\pgfsetrectcap%
\pgfsetroundjoin%
\pgfsetlinewidth{0.803000pt}%
\definecolor{currentstroke}{rgb}{0.690196,0.690196,0.690196}%
\pgfsetstrokecolor{currentstroke}%
\pgfsetdash{}{0pt}%
\pgfpathmoveto{\pgfqpoint{0.568671in}{0.451277in}}%
\pgfpathlineto{\pgfqpoint{0.568671in}{1.989300in}}%
\pgfusepath{stroke}%
\end{pgfscope}%
\begin{pgfscope}%
\pgfsetbuttcap%
\pgfsetroundjoin%
\definecolor{currentfill}{rgb}{0.000000,0.000000,0.000000}%
\pgfsetfillcolor{currentfill}%
\pgfsetlinewidth{0.803000pt}%
\definecolor{currentstroke}{rgb}{0.000000,0.000000,0.000000}%
\pgfsetstrokecolor{currentstroke}%
\pgfsetdash{}{0pt}%
\pgfsys@defobject{currentmarker}{\pgfqpoint{0.000000in}{-0.048611in}}{\pgfqpoint{0.000000in}{0.000000in}}{%
\pgfpathmoveto{\pgfqpoint{0.000000in}{0.000000in}}%
\pgfpathlineto{\pgfqpoint{0.000000in}{-0.048611in}}%
\pgfusepath{stroke,fill}%
}%
\begin{pgfscope}%
\pgfsys@transformshift{0.568671in}{0.451277in}%
\pgfsys@useobject{currentmarker}{}%
\end{pgfscope}%
\end{pgfscope}%
\begin{pgfscope}%
\definecolor{textcolor}{rgb}{0.000000,0.000000,0.000000}%
\pgfsetstrokecolor{textcolor}%
\pgfsetfillcolor{textcolor}%
\pgftext[x=0.568671in,y=0.354055in,,top]{\color{textcolor}\rmfamily\fontsize{9.000000}{10.800000}\selectfont \(\displaystyle {0.00}\)}%
\end{pgfscope}%
\begin{pgfscope}%
\pgfpathrectangle{\pgfqpoint{0.568671in}{0.451277in}}{\pgfqpoint{2.644697in}{1.538023in}}%
\pgfusepath{clip}%
\pgfsetrectcap%
\pgfsetroundjoin%
\pgfsetlinewidth{0.803000pt}%
\definecolor{currentstroke}{rgb}{0.690196,0.690196,0.690196}%
\pgfsetstrokecolor{currentstroke}%
\pgfsetdash{}{0pt}%
\pgfpathmoveto{\pgfqpoint{1.229845in}{0.451277in}}%
\pgfpathlineto{\pgfqpoint{1.229845in}{1.989300in}}%
\pgfusepath{stroke}%
\end{pgfscope}%
\begin{pgfscope}%
\pgfsetbuttcap%
\pgfsetroundjoin%
\definecolor{currentfill}{rgb}{0.000000,0.000000,0.000000}%
\pgfsetfillcolor{currentfill}%
\pgfsetlinewidth{0.803000pt}%
\definecolor{currentstroke}{rgb}{0.000000,0.000000,0.000000}%
\pgfsetstrokecolor{currentstroke}%
\pgfsetdash{}{0pt}%
\pgfsys@defobject{currentmarker}{\pgfqpoint{0.000000in}{-0.048611in}}{\pgfqpoint{0.000000in}{0.000000in}}{%
\pgfpathmoveto{\pgfqpoint{0.000000in}{0.000000in}}%
\pgfpathlineto{\pgfqpoint{0.000000in}{-0.048611in}}%
\pgfusepath{stroke,fill}%
}%
\begin{pgfscope}%
\pgfsys@transformshift{1.229845in}{0.451277in}%
\pgfsys@useobject{currentmarker}{}%
\end{pgfscope}%
\end{pgfscope}%
\begin{pgfscope}%
\definecolor{textcolor}{rgb}{0.000000,0.000000,0.000000}%
\pgfsetstrokecolor{textcolor}%
\pgfsetfillcolor{textcolor}%
\pgftext[x=1.229845in,y=0.354055in,,top]{\color{textcolor}\rmfamily\fontsize{9.000000}{10.800000}\selectfont \(\displaystyle {0.01}\)}%
\end{pgfscope}%
\begin{pgfscope}%
\pgfpathrectangle{\pgfqpoint{0.568671in}{0.451277in}}{\pgfqpoint{2.644697in}{1.538023in}}%
\pgfusepath{clip}%
\pgfsetrectcap%
\pgfsetroundjoin%
\pgfsetlinewidth{0.803000pt}%
\definecolor{currentstroke}{rgb}{0.690196,0.690196,0.690196}%
\pgfsetstrokecolor{currentstroke}%
\pgfsetdash{}{0pt}%
\pgfpathmoveto{\pgfqpoint{1.891020in}{0.451277in}}%
\pgfpathlineto{\pgfqpoint{1.891020in}{1.989300in}}%
\pgfusepath{stroke}%
\end{pgfscope}%
\begin{pgfscope}%
\pgfsetbuttcap%
\pgfsetroundjoin%
\definecolor{currentfill}{rgb}{0.000000,0.000000,0.000000}%
\pgfsetfillcolor{currentfill}%
\pgfsetlinewidth{0.803000pt}%
\definecolor{currentstroke}{rgb}{0.000000,0.000000,0.000000}%
\pgfsetstrokecolor{currentstroke}%
\pgfsetdash{}{0pt}%
\pgfsys@defobject{currentmarker}{\pgfqpoint{0.000000in}{-0.048611in}}{\pgfqpoint{0.000000in}{0.000000in}}{%
\pgfpathmoveto{\pgfqpoint{0.000000in}{0.000000in}}%
\pgfpathlineto{\pgfqpoint{0.000000in}{-0.048611in}}%
\pgfusepath{stroke,fill}%
}%
\begin{pgfscope}%
\pgfsys@transformshift{1.891020in}{0.451277in}%
\pgfsys@useobject{currentmarker}{}%
\end{pgfscope}%
\end{pgfscope}%
\begin{pgfscope}%
\definecolor{textcolor}{rgb}{0.000000,0.000000,0.000000}%
\pgfsetstrokecolor{textcolor}%
\pgfsetfillcolor{textcolor}%
\pgftext[x=1.891020in,y=0.354055in,,top]{\color{textcolor}\rmfamily\fontsize{9.000000}{10.800000}\selectfont \(\displaystyle {0.02}\)}%
\end{pgfscope}%
\begin{pgfscope}%
\pgfpathrectangle{\pgfqpoint{0.568671in}{0.451277in}}{\pgfqpoint{2.644697in}{1.538023in}}%
\pgfusepath{clip}%
\pgfsetrectcap%
\pgfsetroundjoin%
\pgfsetlinewidth{0.803000pt}%
\definecolor{currentstroke}{rgb}{0.690196,0.690196,0.690196}%
\pgfsetstrokecolor{currentstroke}%
\pgfsetdash{}{0pt}%
\pgfpathmoveto{\pgfqpoint{2.552194in}{0.451277in}}%
\pgfpathlineto{\pgfqpoint{2.552194in}{1.989300in}}%
\pgfusepath{stroke}%
\end{pgfscope}%
\begin{pgfscope}%
\pgfsetbuttcap%
\pgfsetroundjoin%
\definecolor{currentfill}{rgb}{0.000000,0.000000,0.000000}%
\pgfsetfillcolor{currentfill}%
\pgfsetlinewidth{0.803000pt}%
\definecolor{currentstroke}{rgb}{0.000000,0.000000,0.000000}%
\pgfsetstrokecolor{currentstroke}%
\pgfsetdash{}{0pt}%
\pgfsys@defobject{currentmarker}{\pgfqpoint{0.000000in}{-0.048611in}}{\pgfqpoint{0.000000in}{0.000000in}}{%
\pgfpathmoveto{\pgfqpoint{0.000000in}{0.000000in}}%
\pgfpathlineto{\pgfqpoint{0.000000in}{-0.048611in}}%
\pgfusepath{stroke,fill}%
}%
\begin{pgfscope}%
\pgfsys@transformshift{2.552194in}{0.451277in}%
\pgfsys@useobject{currentmarker}{}%
\end{pgfscope}%
\end{pgfscope}%
\begin{pgfscope}%
\definecolor{textcolor}{rgb}{0.000000,0.000000,0.000000}%
\pgfsetstrokecolor{textcolor}%
\pgfsetfillcolor{textcolor}%
\pgftext[x=2.552194in,y=0.354055in,,top]{\color{textcolor}\rmfamily\fontsize{9.000000}{10.800000}\selectfont \(\displaystyle {0.03}\)}%
\end{pgfscope}%
\begin{pgfscope}%
\pgfpathrectangle{\pgfqpoint{0.568671in}{0.451277in}}{\pgfqpoint{2.644697in}{1.538023in}}%
\pgfusepath{clip}%
\pgfsetrectcap%
\pgfsetroundjoin%
\pgfsetlinewidth{0.803000pt}%
\definecolor{currentstroke}{rgb}{0.690196,0.690196,0.690196}%
\pgfsetstrokecolor{currentstroke}%
\pgfsetdash{}{0pt}%
\pgfpathmoveto{\pgfqpoint{3.213368in}{0.451277in}}%
\pgfpathlineto{\pgfqpoint{3.213368in}{1.989300in}}%
\pgfusepath{stroke}%
\end{pgfscope}%
\begin{pgfscope}%
\pgfsetbuttcap%
\pgfsetroundjoin%
\definecolor{currentfill}{rgb}{0.000000,0.000000,0.000000}%
\pgfsetfillcolor{currentfill}%
\pgfsetlinewidth{0.803000pt}%
\definecolor{currentstroke}{rgb}{0.000000,0.000000,0.000000}%
\pgfsetstrokecolor{currentstroke}%
\pgfsetdash{}{0pt}%
\pgfsys@defobject{currentmarker}{\pgfqpoint{0.000000in}{-0.048611in}}{\pgfqpoint{0.000000in}{0.000000in}}{%
\pgfpathmoveto{\pgfqpoint{0.000000in}{0.000000in}}%
\pgfpathlineto{\pgfqpoint{0.000000in}{-0.048611in}}%
\pgfusepath{stroke,fill}%
}%
\begin{pgfscope}%
\pgfsys@transformshift{3.213368in}{0.451277in}%
\pgfsys@useobject{currentmarker}{}%
\end{pgfscope}%
\end{pgfscope}%
\begin{pgfscope}%
\definecolor{textcolor}{rgb}{0.000000,0.000000,0.000000}%
\pgfsetstrokecolor{textcolor}%
\pgfsetfillcolor{textcolor}%
\pgftext[x=3.213368in,y=0.354055in,,top]{\color{textcolor}\rmfamily\fontsize{9.000000}{10.800000}\selectfont \(\displaystyle {0.04}\)}%
\end{pgfscope}%
\begin{pgfscope}%
\definecolor{textcolor}{rgb}{0.000000,0.000000,0.000000}%
\pgfsetstrokecolor{textcolor}%
\pgfsetfillcolor{textcolor}%
\pgftext[x=1.891020in,y=0.187500in,,top]{\color{textcolor}\rmfamily\fontsize{9.000000}{10.800000}\selectfont Mixing factor \(\displaystyle \gamma\) [1]}%
\end{pgfscope}%
\begin{pgfscope}%
\pgfpathrectangle{\pgfqpoint{0.568671in}{0.451277in}}{\pgfqpoint{2.644697in}{1.538023in}}%
\pgfusepath{clip}%
\pgfsetrectcap%
\pgfsetroundjoin%
\pgfsetlinewidth{0.803000pt}%
\definecolor{currentstroke}{rgb}{0.690196,0.690196,0.690196}%
\pgfsetstrokecolor{currentstroke}%
\pgfsetdash{}{0pt}%
\pgfpathmoveto{\pgfqpoint{0.568671in}{0.451277in}}%
\pgfpathlineto{\pgfqpoint{3.213368in}{0.451277in}}%
\pgfusepath{stroke}%
\end{pgfscope}%
\begin{pgfscope}%
\pgfsetbuttcap%
\pgfsetroundjoin%
\definecolor{currentfill}{rgb}{0.000000,0.000000,0.000000}%
\pgfsetfillcolor{currentfill}%
\pgfsetlinewidth{0.803000pt}%
\definecolor{currentstroke}{rgb}{0.000000,0.000000,0.000000}%
\pgfsetstrokecolor{currentstroke}%
\pgfsetdash{}{0pt}%
\pgfsys@defobject{currentmarker}{\pgfqpoint{-0.048611in}{0.000000in}}{\pgfqpoint{-0.000000in}{0.000000in}}{%
\pgfpathmoveto{\pgfqpoint{-0.000000in}{0.000000in}}%
\pgfpathlineto{\pgfqpoint{-0.048611in}{0.000000in}}%
\pgfusepath{stroke,fill}%
}%
\begin{pgfscope}%
\pgfsys@transformshift{0.568671in}{0.451277in}%
\pgfsys@useobject{currentmarker}{}%
\end{pgfscope}%
\end{pgfscope}%
\begin{pgfscope}%
\definecolor{textcolor}{rgb}{0.000000,0.000000,0.000000}%
\pgfsetstrokecolor{textcolor}%
\pgfsetfillcolor{textcolor}%
\pgftext[x=0.243055in, y=0.407902in, left, base]{\color{textcolor}\rmfamily\fontsize{9.000000}{10.800000}\selectfont \(\displaystyle {\ensuremath{-}40}\)}%
\end{pgfscope}%
\begin{pgfscope}%
\pgfpathrectangle{\pgfqpoint{0.568671in}{0.451277in}}{\pgfqpoint{2.644697in}{1.538023in}}%
\pgfusepath{clip}%
\pgfsetrectcap%
\pgfsetroundjoin%
\pgfsetlinewidth{0.803000pt}%
\definecolor{currentstroke}{rgb}{0.690196,0.690196,0.690196}%
\pgfsetstrokecolor{currentstroke}%
\pgfsetdash{}{0pt}%
\pgfpathmoveto{\pgfqpoint{0.568671in}{0.835783in}}%
\pgfpathlineto{\pgfqpoint{3.213368in}{0.835783in}}%
\pgfusepath{stroke}%
\end{pgfscope}%
\begin{pgfscope}%
\pgfsetbuttcap%
\pgfsetroundjoin%
\definecolor{currentfill}{rgb}{0.000000,0.000000,0.000000}%
\pgfsetfillcolor{currentfill}%
\pgfsetlinewidth{0.803000pt}%
\definecolor{currentstroke}{rgb}{0.000000,0.000000,0.000000}%
\pgfsetstrokecolor{currentstroke}%
\pgfsetdash{}{0pt}%
\pgfsys@defobject{currentmarker}{\pgfqpoint{-0.048611in}{0.000000in}}{\pgfqpoint{-0.000000in}{0.000000in}}{%
\pgfpathmoveto{\pgfqpoint{-0.000000in}{0.000000in}}%
\pgfpathlineto{\pgfqpoint{-0.048611in}{0.000000in}}%
\pgfusepath{stroke,fill}%
}%
\begin{pgfscope}%
\pgfsys@transformshift{0.568671in}{0.835783in}%
\pgfsys@useobject{currentmarker}{}%
\end{pgfscope}%
\end{pgfscope}%
\begin{pgfscope}%
\definecolor{textcolor}{rgb}{0.000000,0.000000,0.000000}%
\pgfsetstrokecolor{textcolor}%
\pgfsetfillcolor{textcolor}%
\pgftext[x=0.243055in, y=0.792408in, left, base]{\color{textcolor}\rmfamily\fontsize{9.000000}{10.800000}\selectfont \(\displaystyle {\ensuremath{-}35}\)}%
\end{pgfscope}%
\begin{pgfscope}%
\pgfpathrectangle{\pgfqpoint{0.568671in}{0.451277in}}{\pgfqpoint{2.644697in}{1.538023in}}%
\pgfusepath{clip}%
\pgfsetrectcap%
\pgfsetroundjoin%
\pgfsetlinewidth{0.803000pt}%
\definecolor{currentstroke}{rgb}{0.690196,0.690196,0.690196}%
\pgfsetstrokecolor{currentstroke}%
\pgfsetdash{}{0pt}%
\pgfpathmoveto{\pgfqpoint{0.568671in}{1.220289in}}%
\pgfpathlineto{\pgfqpoint{3.213368in}{1.220289in}}%
\pgfusepath{stroke}%
\end{pgfscope}%
\begin{pgfscope}%
\pgfsetbuttcap%
\pgfsetroundjoin%
\definecolor{currentfill}{rgb}{0.000000,0.000000,0.000000}%
\pgfsetfillcolor{currentfill}%
\pgfsetlinewidth{0.803000pt}%
\definecolor{currentstroke}{rgb}{0.000000,0.000000,0.000000}%
\pgfsetstrokecolor{currentstroke}%
\pgfsetdash{}{0pt}%
\pgfsys@defobject{currentmarker}{\pgfqpoint{-0.048611in}{0.000000in}}{\pgfqpoint{-0.000000in}{0.000000in}}{%
\pgfpathmoveto{\pgfqpoint{-0.000000in}{0.000000in}}%
\pgfpathlineto{\pgfqpoint{-0.048611in}{0.000000in}}%
\pgfusepath{stroke,fill}%
}%
\begin{pgfscope}%
\pgfsys@transformshift{0.568671in}{1.220289in}%
\pgfsys@useobject{currentmarker}{}%
\end{pgfscope}%
\end{pgfscope}%
\begin{pgfscope}%
\definecolor{textcolor}{rgb}{0.000000,0.000000,0.000000}%
\pgfsetstrokecolor{textcolor}%
\pgfsetfillcolor{textcolor}%
\pgftext[x=0.243055in, y=1.176914in, left, base]{\color{textcolor}\rmfamily\fontsize{9.000000}{10.800000}\selectfont \(\displaystyle {\ensuremath{-}30}\)}%
\end{pgfscope}%
\begin{pgfscope}%
\pgfpathrectangle{\pgfqpoint{0.568671in}{0.451277in}}{\pgfqpoint{2.644697in}{1.538023in}}%
\pgfusepath{clip}%
\pgfsetrectcap%
\pgfsetroundjoin%
\pgfsetlinewidth{0.803000pt}%
\definecolor{currentstroke}{rgb}{0.690196,0.690196,0.690196}%
\pgfsetstrokecolor{currentstroke}%
\pgfsetdash{}{0pt}%
\pgfpathmoveto{\pgfqpoint{0.568671in}{1.604795in}}%
\pgfpathlineto{\pgfqpoint{3.213368in}{1.604795in}}%
\pgfusepath{stroke}%
\end{pgfscope}%
\begin{pgfscope}%
\pgfsetbuttcap%
\pgfsetroundjoin%
\definecolor{currentfill}{rgb}{0.000000,0.000000,0.000000}%
\pgfsetfillcolor{currentfill}%
\pgfsetlinewidth{0.803000pt}%
\definecolor{currentstroke}{rgb}{0.000000,0.000000,0.000000}%
\pgfsetstrokecolor{currentstroke}%
\pgfsetdash{}{0pt}%
\pgfsys@defobject{currentmarker}{\pgfqpoint{-0.048611in}{0.000000in}}{\pgfqpoint{-0.000000in}{0.000000in}}{%
\pgfpathmoveto{\pgfqpoint{-0.000000in}{0.000000in}}%
\pgfpathlineto{\pgfqpoint{-0.048611in}{0.000000in}}%
\pgfusepath{stroke,fill}%
}%
\begin{pgfscope}%
\pgfsys@transformshift{0.568671in}{1.604795in}%
\pgfsys@useobject{currentmarker}{}%
\end{pgfscope}%
\end{pgfscope}%
\begin{pgfscope}%
\definecolor{textcolor}{rgb}{0.000000,0.000000,0.000000}%
\pgfsetstrokecolor{textcolor}%
\pgfsetfillcolor{textcolor}%
\pgftext[x=0.243055in, y=1.561420in, left, base]{\color{textcolor}\rmfamily\fontsize{9.000000}{10.800000}\selectfont \(\displaystyle {\ensuremath{-}25}\)}%
\end{pgfscope}%
\begin{pgfscope}%
\pgfpathrectangle{\pgfqpoint{0.568671in}{0.451277in}}{\pgfqpoint{2.644697in}{1.538023in}}%
\pgfusepath{clip}%
\pgfsetrectcap%
\pgfsetroundjoin%
\pgfsetlinewidth{0.803000pt}%
\definecolor{currentstroke}{rgb}{0.690196,0.690196,0.690196}%
\pgfsetstrokecolor{currentstroke}%
\pgfsetdash{}{0pt}%
\pgfpathmoveto{\pgfqpoint{0.568671in}{1.989300in}}%
\pgfpathlineto{\pgfqpoint{3.213368in}{1.989300in}}%
\pgfusepath{stroke}%
\end{pgfscope}%
\begin{pgfscope}%
\pgfsetbuttcap%
\pgfsetroundjoin%
\definecolor{currentfill}{rgb}{0.000000,0.000000,0.000000}%
\pgfsetfillcolor{currentfill}%
\pgfsetlinewidth{0.803000pt}%
\definecolor{currentstroke}{rgb}{0.000000,0.000000,0.000000}%
\pgfsetstrokecolor{currentstroke}%
\pgfsetdash{}{0pt}%
\pgfsys@defobject{currentmarker}{\pgfqpoint{-0.048611in}{0.000000in}}{\pgfqpoint{-0.000000in}{0.000000in}}{%
\pgfpathmoveto{\pgfqpoint{-0.000000in}{0.000000in}}%
\pgfpathlineto{\pgfqpoint{-0.048611in}{0.000000in}}%
\pgfusepath{stroke,fill}%
}%
\begin{pgfscope}%
\pgfsys@transformshift{0.568671in}{1.989300in}%
\pgfsys@useobject{currentmarker}{}%
\end{pgfscope}%
\end{pgfscope}%
\begin{pgfscope}%
\definecolor{textcolor}{rgb}{0.000000,0.000000,0.000000}%
\pgfsetstrokecolor{textcolor}%
\pgfsetfillcolor{textcolor}%
\pgftext[x=0.243055in, y=1.945925in, left, base]{\color{textcolor}\rmfamily\fontsize{9.000000}{10.800000}\selectfont \(\displaystyle {\ensuremath{-}20}\)}%
\end{pgfscope}%
\begin{pgfscope}%
\definecolor{textcolor}{rgb}{0.000000,0.000000,0.000000}%
\pgfsetstrokecolor{textcolor}%
\pgfsetfillcolor{textcolor}%
\pgftext[x=0.187500in,y=1.220289in,,bottom,rotate=90.000000]{\color{textcolor}\rmfamily\fontsize{9.000000}{10.800000}\selectfont Avg. NPM [dB]}%
\end{pgfscope}%
\begin{pgfscope}%
\pgfpathrectangle{\pgfqpoint{0.568671in}{0.451277in}}{\pgfqpoint{2.644697in}{1.538023in}}%
\pgfusepath{clip}%
\pgfsetrectcap%
\pgfsetroundjoin%
\pgfsetlinewidth{1.505625pt}%
\definecolor{currentstroke}{rgb}{0.000000,0.000000,0.000000}%
\pgfsetstrokecolor{currentstroke}%
\pgfsetdash{}{0pt}%
\pgfpathmoveto{\pgfqpoint{0.568671in}{0.770304in}}%
\pgfpathlineto{\pgfqpoint{0.700906in}{0.770304in}}%
\pgfpathlineto{\pgfqpoint{0.833141in}{0.770304in}}%
\pgfpathlineto{\pgfqpoint{0.965376in}{0.770304in}}%
\pgfpathlineto{\pgfqpoint{1.097611in}{0.770304in}}%
\pgfpathlineto{\pgfqpoint{1.229845in}{0.770304in}}%
\pgfpathlineto{\pgfqpoint{1.362080in}{0.770304in}}%
\pgfpathlineto{\pgfqpoint{1.494315in}{0.770304in}}%
\pgfpathlineto{\pgfqpoint{1.626550in}{0.770304in}}%
\pgfpathlineto{\pgfqpoint{1.758785in}{0.770304in}}%
\pgfpathlineto{\pgfqpoint{1.891020in}{0.770304in}}%
\pgfpathlineto{\pgfqpoint{2.023255in}{0.770304in}}%
\pgfpathlineto{\pgfqpoint{2.155489in}{0.770304in}}%
\pgfpathlineto{\pgfqpoint{2.287724in}{0.770304in}}%
\pgfpathlineto{\pgfqpoint{2.419959in}{0.770304in}}%
\pgfpathlineto{\pgfqpoint{2.552194in}{0.770304in}}%
\pgfpathlineto{\pgfqpoint{2.684429in}{0.770304in}}%
\pgfpathlineto{\pgfqpoint{2.816664in}{0.770304in}}%
\pgfpathlineto{\pgfqpoint{2.948899in}{0.770304in}}%
\pgfpathlineto{\pgfqpoint{3.081133in}{0.770304in}}%
\pgfpathlineto{\pgfqpoint{3.213368in}{0.770304in}}%
\pgfusepath{stroke}%
\end{pgfscope}%
\begin{pgfscope}%
\pgfpathrectangle{\pgfqpoint{0.568671in}{0.451277in}}{\pgfqpoint{2.644697in}{1.538023in}}%
\pgfusepath{clip}%
\pgfsetrectcap%
\pgfsetroundjoin%
\pgfsetlinewidth{1.505625pt}%
\definecolor{currentstroke}{rgb}{0.121569,0.466667,0.705882}%
\pgfsetstrokecolor{currentstroke}%
\pgfsetdash{}{0pt}%
\pgfpathmoveto{\pgfqpoint{0.568671in}{0.796535in}}%
\pgfpathlineto{\pgfqpoint{0.700906in}{0.796535in}}%
\pgfpathlineto{\pgfqpoint{0.833141in}{0.796535in}}%
\pgfpathlineto{\pgfqpoint{0.965376in}{0.796535in}}%
\pgfpathlineto{\pgfqpoint{1.097611in}{0.796535in}}%
\pgfpathlineto{\pgfqpoint{1.229845in}{0.796535in}}%
\pgfpathlineto{\pgfqpoint{1.362080in}{0.796535in}}%
\pgfpathlineto{\pgfqpoint{1.494315in}{0.796535in}}%
\pgfpathlineto{\pgfqpoint{1.626550in}{0.796535in}}%
\pgfpathlineto{\pgfqpoint{1.758785in}{0.796535in}}%
\pgfpathlineto{\pgfqpoint{1.891020in}{0.796535in}}%
\pgfpathlineto{\pgfqpoint{2.023255in}{0.796535in}}%
\pgfpathlineto{\pgfqpoint{2.155489in}{0.796535in}}%
\pgfpathlineto{\pgfqpoint{2.287724in}{0.796535in}}%
\pgfpathlineto{\pgfqpoint{2.419959in}{0.796535in}}%
\pgfpathlineto{\pgfqpoint{2.552194in}{0.796535in}}%
\pgfpathlineto{\pgfqpoint{2.684429in}{0.796535in}}%
\pgfpathlineto{\pgfqpoint{2.816664in}{0.796535in}}%
\pgfpathlineto{\pgfqpoint{2.948899in}{0.796535in}}%
\pgfpathlineto{\pgfqpoint{3.081133in}{0.796535in}}%
\pgfpathlineto{\pgfqpoint{3.213368in}{0.796535in}}%
\pgfusepath{stroke}%
\end{pgfscope}%
\begin{pgfscope}%
\pgfpathrectangle{\pgfqpoint{0.568671in}{0.451277in}}{\pgfqpoint{2.644697in}{1.538023in}}%
\pgfusepath{clip}%
\pgfsetbuttcap%
\pgfsetroundjoin%
\definecolor{currentfill}{rgb}{0.000000,0.000000,0.000000}%
\pgfsetfillcolor{currentfill}%
\pgfsetfillopacity{0.000000}%
\pgfsetlinewidth{1.003750pt}%
\definecolor{currentstroke}{rgb}{0.121569,0.466667,0.705882}%
\pgfsetstrokecolor{currentstroke}%
\pgfsetdash{}{0pt}%
\pgfsys@defobject{currentmarker}{\pgfqpoint{-0.027778in}{-0.027778in}}{\pgfqpoint{0.027778in}{0.027778in}}{%
\pgfpathmoveto{\pgfqpoint{0.000000in}{-0.027778in}}%
\pgfpathcurveto{\pgfqpoint{0.007367in}{-0.027778in}}{\pgfqpoint{0.014433in}{-0.024851in}}{\pgfqpoint{0.019642in}{-0.019642in}}%
\pgfpathcurveto{\pgfqpoint{0.024851in}{-0.014433in}}{\pgfqpoint{0.027778in}{-0.007367in}}{\pgfqpoint{0.027778in}{0.000000in}}%
\pgfpathcurveto{\pgfqpoint{0.027778in}{0.007367in}}{\pgfqpoint{0.024851in}{0.014433in}}{\pgfqpoint{0.019642in}{0.019642in}}%
\pgfpathcurveto{\pgfqpoint{0.014433in}{0.024851in}}{\pgfqpoint{0.007367in}{0.027778in}}{\pgfqpoint{0.000000in}{0.027778in}}%
\pgfpathcurveto{\pgfqpoint{-0.007367in}{0.027778in}}{\pgfqpoint{-0.014433in}{0.024851in}}{\pgfqpoint{-0.019642in}{0.019642in}}%
\pgfpathcurveto{\pgfqpoint{-0.024851in}{0.014433in}}{\pgfqpoint{-0.027778in}{0.007367in}}{\pgfqpoint{-0.027778in}{0.000000in}}%
\pgfpathcurveto{\pgfqpoint{-0.027778in}{-0.007367in}}{\pgfqpoint{-0.024851in}{-0.014433in}}{\pgfqpoint{-0.019642in}{-0.019642in}}%
\pgfpathcurveto{\pgfqpoint{-0.014433in}{-0.024851in}}{\pgfqpoint{-0.007367in}{-0.027778in}}{\pgfqpoint{0.000000in}{-0.027778in}}%
\pgfpathlineto{\pgfqpoint{0.000000in}{-0.027778in}}%
\pgfpathclose%
\pgfusepath{stroke,fill}%
}%
\begin{pgfscope}%
\pgfsys@transformshift{0.833141in}{0.796535in}%
\pgfsys@useobject{currentmarker}{}%
\end{pgfscope}%
\begin{pgfscope}%
\pgfsys@transformshift{1.097611in}{0.796535in}%
\pgfsys@useobject{currentmarker}{}%
\end{pgfscope}%
\begin{pgfscope}%
\pgfsys@transformshift{1.362080in}{0.796535in}%
\pgfsys@useobject{currentmarker}{}%
\end{pgfscope}%
\begin{pgfscope}%
\pgfsys@transformshift{1.626550in}{0.796535in}%
\pgfsys@useobject{currentmarker}{}%
\end{pgfscope}%
\begin{pgfscope}%
\pgfsys@transformshift{1.891020in}{0.796535in}%
\pgfsys@useobject{currentmarker}{}%
\end{pgfscope}%
\begin{pgfscope}%
\pgfsys@transformshift{2.155489in}{0.796535in}%
\pgfsys@useobject{currentmarker}{}%
\end{pgfscope}%
\begin{pgfscope}%
\pgfsys@transformshift{2.419959in}{0.796535in}%
\pgfsys@useobject{currentmarker}{}%
\end{pgfscope}%
\begin{pgfscope}%
\pgfsys@transformshift{2.684429in}{0.796535in}%
\pgfsys@useobject{currentmarker}{}%
\end{pgfscope}%
\begin{pgfscope}%
\pgfsys@transformshift{2.948899in}{0.796535in}%
\pgfsys@useobject{currentmarker}{}%
\end{pgfscope}%
\begin{pgfscope}%
\pgfsys@transformshift{3.213368in}{0.796535in}%
\pgfsys@useobject{currentmarker}{}%
\end{pgfscope}%
\end{pgfscope}%
\begin{pgfscope}%
\pgfpathrectangle{\pgfqpoint{0.568671in}{0.451277in}}{\pgfqpoint{2.644697in}{1.538023in}}%
\pgfusepath{clip}%
\pgfsetrectcap%
\pgfsetroundjoin%
\pgfsetlinewidth{1.505625pt}%
\definecolor{currentstroke}{rgb}{1.000000,0.498039,0.054902}%
\pgfsetstrokecolor{currentstroke}%
\pgfsetdash{}{0pt}%
\pgfpathmoveto{\pgfqpoint{0.663221in}{2.003189in}}%
\pgfpathlineto{\pgfqpoint{0.700906in}{1.527012in}}%
\pgfpathlineto{\pgfqpoint{0.833141in}{1.158562in}}%
\pgfpathlineto{\pgfqpoint{0.965376in}{1.062626in}}%
\pgfpathlineto{\pgfqpoint{1.097611in}{1.020148in}}%
\pgfpathlineto{\pgfqpoint{1.229845in}{1.023919in}}%
\pgfpathlineto{\pgfqpoint{1.362080in}{1.051901in}}%
\pgfpathlineto{\pgfqpoint{1.494315in}{1.071252in}}%
\pgfpathlineto{\pgfqpoint{1.626550in}{1.094269in}}%
\pgfpathlineto{\pgfqpoint{1.758785in}{1.131922in}}%
\pgfpathlineto{\pgfqpoint{1.891020in}{1.162576in}}%
\pgfpathlineto{\pgfqpoint{2.023255in}{1.200499in}}%
\pgfpathlineto{\pgfqpoint{2.155489in}{1.231310in}}%
\pgfpathlineto{\pgfqpoint{2.287724in}{1.268983in}}%
\pgfpathlineto{\pgfqpoint{2.419959in}{1.300429in}}%
\pgfpathlineto{\pgfqpoint{2.552194in}{1.321574in}}%
\pgfpathlineto{\pgfqpoint{2.684429in}{1.353015in}}%
\pgfpathlineto{\pgfqpoint{2.816664in}{1.376453in}}%
\pgfpathlineto{\pgfqpoint{2.948899in}{1.402560in}}%
\pgfpathlineto{\pgfqpoint{3.081133in}{1.431235in}}%
\pgfpathlineto{\pgfqpoint{3.213368in}{1.455607in}}%
\pgfusepath{stroke}%
\end{pgfscope}%
\begin{pgfscope}%
\pgfpathrectangle{\pgfqpoint{0.568671in}{0.451277in}}{\pgfqpoint{2.644697in}{1.538023in}}%
\pgfusepath{clip}%
\pgfsetbuttcap%
\pgfsetmiterjoin%
\definecolor{currentfill}{rgb}{0.000000,0.000000,0.000000}%
\pgfsetfillcolor{currentfill}%
\pgfsetfillopacity{0.000000}%
\pgfsetlinewidth{1.003750pt}%
\definecolor{currentstroke}{rgb}{1.000000,0.498039,0.054902}%
\pgfsetstrokecolor{currentstroke}%
\pgfsetdash{}{0pt}%
\pgfsys@defobject{currentmarker}{\pgfqpoint{-0.027778in}{-0.027778in}}{\pgfqpoint{0.027778in}{0.027778in}}{%
\pgfpathmoveto{\pgfqpoint{-0.027778in}{-0.027778in}}%
\pgfpathlineto{\pgfqpoint{0.027778in}{-0.027778in}}%
\pgfpathlineto{\pgfqpoint{0.027778in}{0.027778in}}%
\pgfpathlineto{\pgfqpoint{-0.027778in}{0.027778in}}%
\pgfpathlineto{\pgfqpoint{-0.027778in}{-0.027778in}}%
\pgfpathclose%
\pgfusepath{stroke,fill}%
}%
\begin{pgfscope}%
\pgfsys@transformshift{0.700906in}{1.527012in}%
\pgfsys@useobject{currentmarker}{}%
\end{pgfscope}%
\begin{pgfscope}%
\pgfsys@transformshift{0.965376in}{1.062626in}%
\pgfsys@useobject{currentmarker}{}%
\end{pgfscope}%
\begin{pgfscope}%
\pgfsys@transformshift{1.229845in}{1.023919in}%
\pgfsys@useobject{currentmarker}{}%
\end{pgfscope}%
\begin{pgfscope}%
\pgfsys@transformshift{1.494315in}{1.071252in}%
\pgfsys@useobject{currentmarker}{}%
\end{pgfscope}%
\begin{pgfscope}%
\pgfsys@transformshift{1.758785in}{1.131922in}%
\pgfsys@useobject{currentmarker}{}%
\end{pgfscope}%
\begin{pgfscope}%
\pgfsys@transformshift{2.023255in}{1.200499in}%
\pgfsys@useobject{currentmarker}{}%
\end{pgfscope}%
\begin{pgfscope}%
\pgfsys@transformshift{2.287724in}{1.268983in}%
\pgfsys@useobject{currentmarker}{}%
\end{pgfscope}%
\begin{pgfscope}%
\pgfsys@transformshift{2.552194in}{1.321574in}%
\pgfsys@useobject{currentmarker}{}%
\end{pgfscope}%
\begin{pgfscope}%
\pgfsys@transformshift{2.816664in}{1.376453in}%
\pgfsys@useobject{currentmarker}{}%
\end{pgfscope}%
\begin{pgfscope}%
\pgfsys@transformshift{3.081133in}{1.431235in}%
\pgfsys@useobject{currentmarker}{}%
\end{pgfscope}%
\end{pgfscope}%
\begin{pgfscope}%
\pgfpathrectangle{\pgfqpoint{0.568671in}{0.451277in}}{\pgfqpoint{2.644697in}{1.538023in}}%
\pgfusepath{clip}%
\pgfsetrectcap%
\pgfsetroundjoin%
\pgfsetlinewidth{1.505625pt}%
\definecolor{currentstroke}{rgb}{0.172549,0.627451,0.172549}%
\pgfsetstrokecolor{currentstroke}%
\pgfsetdash{}{0pt}%
\pgfpathmoveto{\pgfqpoint{0.667508in}{2.003189in}}%
\pgfpathlineto{\pgfqpoint{0.700906in}{1.599492in}}%
\pgfpathlineto{\pgfqpoint{0.833141in}{1.161125in}}%
\pgfpathlineto{\pgfqpoint{0.965376in}{0.973572in}}%
\pgfpathlineto{\pgfqpoint{1.097611in}{0.962358in}}%
\pgfpathlineto{\pgfqpoint{1.229845in}{0.935926in}}%
\pgfpathlineto{\pgfqpoint{1.362080in}{0.940678in}}%
\pgfpathlineto{\pgfqpoint{1.494315in}{0.962047in}}%
\pgfpathlineto{\pgfqpoint{1.626550in}{0.962398in}}%
\pgfpathlineto{\pgfqpoint{1.758785in}{0.994588in}}%
\pgfpathlineto{\pgfqpoint{1.891020in}{1.007431in}}%
\pgfpathlineto{\pgfqpoint{2.023255in}{1.026581in}}%
\pgfpathlineto{\pgfqpoint{2.155489in}{1.027377in}}%
\pgfpathlineto{\pgfqpoint{2.287724in}{1.036696in}}%
\pgfpathlineto{\pgfqpoint{2.419959in}{1.053653in}}%
\pgfpathlineto{\pgfqpoint{2.552194in}{1.059263in}}%
\pgfpathlineto{\pgfqpoint{2.684429in}{1.078174in}}%
\pgfpathlineto{\pgfqpoint{2.816664in}{1.088275in}}%
\pgfpathlineto{\pgfqpoint{2.948899in}{1.099597in}}%
\pgfpathlineto{\pgfqpoint{3.081133in}{1.116675in}}%
\pgfpathlineto{\pgfqpoint{3.213368in}{1.136070in}}%
\pgfusepath{stroke}%
\end{pgfscope}%
\begin{pgfscope}%
\pgfpathrectangle{\pgfqpoint{0.568671in}{0.451277in}}{\pgfqpoint{2.644697in}{1.538023in}}%
\pgfusepath{clip}%
\pgfsetbuttcap%
\pgfsetroundjoin%
\definecolor{currentfill}{rgb}{0.000000,0.000000,0.000000}%
\pgfsetfillcolor{currentfill}%
\pgfsetfillopacity{0.000000}%
\pgfsetlinewidth{1.003750pt}%
\definecolor{currentstroke}{rgb}{0.172549,0.627451,0.172549}%
\pgfsetstrokecolor{currentstroke}%
\pgfsetdash{}{0pt}%
\pgfsys@defobject{currentmarker}{\pgfqpoint{-0.027778in}{-0.027778in}}{\pgfqpoint{0.027778in}{0.027778in}}{%
\pgfpathmoveto{\pgfqpoint{-0.027778in}{-0.027778in}}%
\pgfpathlineto{\pgfqpoint{0.027778in}{0.027778in}}%
\pgfpathmoveto{\pgfqpoint{-0.027778in}{0.027778in}}%
\pgfpathlineto{\pgfqpoint{0.027778in}{-0.027778in}}%
\pgfusepath{stroke,fill}%
}%
\begin{pgfscope}%
\pgfsys@transformshift{0.833141in}{1.161125in}%
\pgfsys@useobject{currentmarker}{}%
\end{pgfscope}%
\begin{pgfscope}%
\pgfsys@transformshift{1.097611in}{0.962358in}%
\pgfsys@useobject{currentmarker}{}%
\end{pgfscope}%
\begin{pgfscope}%
\pgfsys@transformshift{1.362080in}{0.940678in}%
\pgfsys@useobject{currentmarker}{}%
\end{pgfscope}%
\begin{pgfscope}%
\pgfsys@transformshift{1.626550in}{0.962398in}%
\pgfsys@useobject{currentmarker}{}%
\end{pgfscope}%
\begin{pgfscope}%
\pgfsys@transformshift{1.891020in}{1.007431in}%
\pgfsys@useobject{currentmarker}{}%
\end{pgfscope}%
\begin{pgfscope}%
\pgfsys@transformshift{2.155489in}{1.027377in}%
\pgfsys@useobject{currentmarker}{}%
\end{pgfscope}%
\begin{pgfscope}%
\pgfsys@transformshift{2.419959in}{1.053653in}%
\pgfsys@useobject{currentmarker}{}%
\end{pgfscope}%
\begin{pgfscope}%
\pgfsys@transformshift{2.684429in}{1.078174in}%
\pgfsys@useobject{currentmarker}{}%
\end{pgfscope}%
\begin{pgfscope}%
\pgfsys@transformshift{2.948899in}{1.099597in}%
\pgfsys@useobject{currentmarker}{}%
\end{pgfscope}%
\begin{pgfscope}%
\pgfsys@transformshift{3.213368in}{1.136070in}%
\pgfsys@useobject{currentmarker}{}%
\end{pgfscope}%
\end{pgfscope}%
\begin{pgfscope}%
\pgfpathrectangle{\pgfqpoint{0.568671in}{0.451277in}}{\pgfqpoint{2.644697in}{1.538023in}}%
\pgfusepath{clip}%
\pgfsetrectcap%
\pgfsetroundjoin%
\pgfsetlinewidth{1.505625pt}%
\definecolor{currentstroke}{rgb}{0.839216,0.152941,0.156863}%
\pgfsetstrokecolor{currentstroke}%
\pgfsetdash{}{0pt}%
\pgfpathmoveto{\pgfqpoint{0.662116in}{2.003189in}}%
\pgfpathlineto{\pgfqpoint{0.700906in}{1.507248in}}%
\pgfpathlineto{\pgfqpoint{0.833141in}{1.074231in}}%
\pgfpathlineto{\pgfqpoint{0.965376in}{0.934749in}}%
\pgfpathlineto{\pgfqpoint{1.097611in}{0.844174in}}%
\pgfpathlineto{\pgfqpoint{1.229845in}{0.810498in}}%
\pgfpathlineto{\pgfqpoint{1.362080in}{0.813917in}}%
\pgfpathlineto{\pgfqpoint{1.494315in}{0.808017in}}%
\pgfpathlineto{\pgfqpoint{1.626550in}{0.822081in}}%
\pgfpathlineto{\pgfqpoint{1.758785in}{0.831996in}}%
\pgfpathlineto{\pgfqpoint{1.891020in}{0.824858in}}%
\pgfpathlineto{\pgfqpoint{2.023255in}{0.834683in}}%
\pgfpathlineto{\pgfqpoint{2.155489in}{0.845484in}}%
\pgfpathlineto{\pgfqpoint{2.287724in}{0.853264in}}%
\pgfpathlineto{\pgfqpoint{2.419959in}{0.859527in}}%
\pgfpathlineto{\pgfqpoint{2.552194in}{0.863563in}}%
\pgfpathlineto{\pgfqpoint{2.684429in}{0.867381in}}%
\pgfpathlineto{\pgfqpoint{2.816664in}{0.868712in}}%
\pgfpathlineto{\pgfqpoint{2.948899in}{0.875991in}}%
\pgfpathlineto{\pgfqpoint{3.081133in}{0.878849in}}%
\pgfpathlineto{\pgfqpoint{3.213368in}{0.883353in}}%
\pgfusepath{stroke}%
\end{pgfscope}%
\begin{pgfscope}%
\pgfpathrectangle{\pgfqpoint{0.568671in}{0.451277in}}{\pgfqpoint{2.644697in}{1.538023in}}%
\pgfusepath{clip}%
\pgfsetbuttcap%
\pgfsetroundjoin%
\definecolor{currentfill}{rgb}{0.000000,0.000000,0.000000}%
\pgfsetfillcolor{currentfill}%
\pgfsetfillopacity{0.000000}%
\pgfsetlinewidth{1.003750pt}%
\definecolor{currentstroke}{rgb}{0.839216,0.152941,0.156863}%
\pgfsetstrokecolor{currentstroke}%
\pgfsetdash{}{0pt}%
\pgfsys@defobject{currentmarker}{\pgfqpoint{-0.027778in}{-0.027778in}}{\pgfqpoint{0.027778in}{0.027778in}}{%
\pgfpathmoveto{\pgfqpoint{-0.027778in}{0.000000in}}%
\pgfpathlineto{\pgfqpoint{0.027778in}{0.000000in}}%
\pgfpathmoveto{\pgfqpoint{0.000000in}{-0.027778in}}%
\pgfpathlineto{\pgfqpoint{0.000000in}{0.027778in}}%
\pgfusepath{stroke,fill}%
}%
\begin{pgfscope}%
\pgfsys@transformshift{0.700906in}{1.507248in}%
\pgfsys@useobject{currentmarker}{}%
\end{pgfscope}%
\begin{pgfscope}%
\pgfsys@transformshift{0.965376in}{0.934749in}%
\pgfsys@useobject{currentmarker}{}%
\end{pgfscope}%
\begin{pgfscope}%
\pgfsys@transformshift{1.229845in}{0.810498in}%
\pgfsys@useobject{currentmarker}{}%
\end{pgfscope}%
\begin{pgfscope}%
\pgfsys@transformshift{1.494315in}{0.808017in}%
\pgfsys@useobject{currentmarker}{}%
\end{pgfscope}%
\begin{pgfscope}%
\pgfsys@transformshift{1.758785in}{0.831996in}%
\pgfsys@useobject{currentmarker}{}%
\end{pgfscope}%
\begin{pgfscope}%
\pgfsys@transformshift{2.023255in}{0.834683in}%
\pgfsys@useobject{currentmarker}{}%
\end{pgfscope}%
\begin{pgfscope}%
\pgfsys@transformshift{2.287724in}{0.853264in}%
\pgfsys@useobject{currentmarker}{}%
\end{pgfscope}%
\begin{pgfscope}%
\pgfsys@transformshift{2.552194in}{0.863563in}%
\pgfsys@useobject{currentmarker}{}%
\end{pgfscope}%
\begin{pgfscope}%
\pgfsys@transformshift{2.816664in}{0.868712in}%
\pgfsys@useobject{currentmarker}{}%
\end{pgfscope}%
\begin{pgfscope}%
\pgfsys@transformshift{3.081133in}{0.878849in}%
\pgfsys@useobject{currentmarker}{}%
\end{pgfscope}%
\end{pgfscope}%
\begin{pgfscope}%
\pgfsetrectcap%
\pgfsetmiterjoin%
\pgfsetlinewidth{0.803000pt}%
\definecolor{currentstroke}{rgb}{0.000000,0.000000,0.000000}%
\pgfsetstrokecolor{currentstroke}%
\pgfsetdash{}{0pt}%
\pgfpathmoveto{\pgfqpoint{0.568671in}{0.451277in}}%
\pgfpathlineto{\pgfqpoint{0.568671in}{1.989300in}}%
\pgfusepath{stroke}%
\end{pgfscope}%
\begin{pgfscope}%
\pgfsetrectcap%
\pgfsetmiterjoin%
\pgfsetlinewidth{0.803000pt}%
\definecolor{currentstroke}{rgb}{0.000000,0.000000,0.000000}%
\pgfsetstrokecolor{currentstroke}%
\pgfsetdash{}{0pt}%
\pgfpathmoveto{\pgfqpoint{3.213368in}{0.451277in}}%
\pgfpathlineto{\pgfqpoint{3.213368in}{1.989300in}}%
\pgfusepath{stroke}%
\end{pgfscope}%
\begin{pgfscope}%
\pgfsetrectcap%
\pgfsetmiterjoin%
\pgfsetlinewidth{0.803000pt}%
\definecolor{currentstroke}{rgb}{0.000000,0.000000,0.000000}%
\pgfsetstrokecolor{currentstroke}%
\pgfsetdash{}{0pt}%
\pgfpathmoveto{\pgfqpoint{0.568671in}{0.451277in}}%
\pgfpathlineto{\pgfqpoint{3.213368in}{0.451277in}}%
\pgfusepath{stroke}%
\end{pgfscope}%
\begin{pgfscope}%
\pgfsetrectcap%
\pgfsetmiterjoin%
\pgfsetlinewidth{0.803000pt}%
\definecolor{currentstroke}{rgb}{0.000000,0.000000,0.000000}%
\pgfsetstrokecolor{currentstroke}%
\pgfsetdash{}{0pt}%
\pgfpathmoveto{\pgfqpoint{0.568671in}{1.989300in}}%
\pgfpathlineto{\pgfqpoint{3.213368in}{1.989300in}}%
\pgfusepath{stroke}%
\end{pgfscope}%
\begin{pgfscope}%
\pgfsetbuttcap%
\pgfsetmiterjoin%
\definecolor{currentfill}{rgb}{1.000000,1.000000,1.000000}%
\pgfsetfillcolor{currentfill}%
\pgfsetfillopacity{0.800000}%
\pgfsetlinewidth{1.003750pt}%
\definecolor{currentstroke}{rgb}{0.800000,0.800000,0.800000}%
\pgfsetstrokecolor{currentstroke}%
\pgfsetstrokeopacity{0.800000}%
\pgfsetdash{}{0pt}%
\pgfpathmoveto{\pgfqpoint{0.783863in}{1.504076in}}%
\pgfpathlineto{\pgfqpoint{3.145313in}{1.504076in}}%
\pgfpathquadraticcurveto{\pgfqpoint{3.164757in}{1.504076in}}{\pgfqpoint{3.164757in}{1.523520in}}%
\pgfpathlineto{\pgfqpoint{3.164757in}{1.921245in}}%
\pgfpathquadraticcurveto{\pgfqpoint{3.164757in}{1.940689in}}{\pgfqpoint{3.145313in}{1.940689in}}%
\pgfpathlineto{\pgfqpoint{0.783863in}{1.940689in}}%
\pgfpathquadraticcurveto{\pgfqpoint{0.764419in}{1.940689in}}{\pgfqpoint{0.764419in}{1.921245in}}%
\pgfpathlineto{\pgfqpoint{0.764419in}{1.523520in}}%
\pgfpathquadraticcurveto{\pgfqpoint{0.764419in}{1.504076in}}{\pgfqpoint{0.783863in}{1.504076in}}%
\pgfpathlineto{\pgfqpoint{0.783863in}{1.504076in}}%
\pgfpathclose%
\pgfusepath{stroke,fill}%
\end{pgfscope}%
\begin{pgfscope}%
\pgfsetrectcap%
\pgfsetroundjoin%
\pgfsetlinewidth{1.505625pt}%
\definecolor{currentstroke}{rgb}{0.000000,0.000000,0.000000}%
\pgfsetstrokecolor{currentstroke}%
\pgfsetdash{}{0pt}%
\pgfpathmoveto{\pgfqpoint{0.803308in}{1.867384in}}%
\pgfpathlineto{\pgfqpoint{0.900530in}{1.867384in}}%
\pgfpathlineto{\pgfqpoint{0.997752in}{1.867384in}}%
\pgfusepath{stroke}%
\end{pgfscope}%
\begin{pgfscope}%
\definecolor{textcolor}{rgb}{0.000000,0.000000,0.000000}%
\pgfsetstrokecolor{textcolor}%
\pgfsetfillcolor{textcolor}%
\pgftext[x=1.075530in,y=1.833356in,left,base]{\color{textcolor}\rmfamily\fontsize{7.000000}{8.400000}\selectfont optimal}%
\end{pgfscope}%
\begin{pgfscope}%
\pgfsetrectcap%
\pgfsetroundjoin%
\pgfsetlinewidth{1.505625pt}%
\definecolor{currentstroke}{rgb}{0.121569,0.466667,0.705882}%
\pgfsetstrokecolor{currentstroke}%
\pgfsetdash{}{0pt}%
\pgfpathmoveto{\pgfqpoint{0.803308in}{1.731856in}}%
\pgfpathlineto{\pgfqpoint{0.900530in}{1.731856in}}%
\pgfpathlineto{\pgfqpoint{0.997752in}{1.731856in}}%
\pgfusepath{stroke}%
\end{pgfscope}%
\begin{pgfscope}%
\pgfsetbuttcap%
\pgfsetroundjoin%
\definecolor{currentfill}{rgb}{0.000000,0.000000,0.000000}%
\pgfsetfillcolor{currentfill}%
\pgfsetfillopacity{0.000000}%
\pgfsetlinewidth{1.003750pt}%
\definecolor{currentstroke}{rgb}{0.121569,0.466667,0.705882}%
\pgfsetstrokecolor{currentstroke}%
\pgfsetdash{}{0pt}%
\pgfsys@defobject{currentmarker}{\pgfqpoint{-0.027778in}{-0.027778in}}{\pgfqpoint{0.027778in}{0.027778in}}{%
\pgfpathmoveto{\pgfqpoint{0.000000in}{-0.027778in}}%
\pgfpathcurveto{\pgfqpoint{0.007367in}{-0.027778in}}{\pgfqpoint{0.014433in}{-0.024851in}}{\pgfqpoint{0.019642in}{-0.019642in}}%
\pgfpathcurveto{\pgfqpoint{0.024851in}{-0.014433in}}{\pgfqpoint{0.027778in}{-0.007367in}}{\pgfqpoint{0.027778in}{0.000000in}}%
\pgfpathcurveto{\pgfqpoint{0.027778in}{0.007367in}}{\pgfqpoint{0.024851in}{0.014433in}}{\pgfqpoint{0.019642in}{0.019642in}}%
\pgfpathcurveto{\pgfqpoint{0.014433in}{0.024851in}}{\pgfqpoint{0.007367in}{0.027778in}}{\pgfqpoint{0.000000in}{0.027778in}}%
\pgfpathcurveto{\pgfqpoint{-0.007367in}{0.027778in}}{\pgfqpoint{-0.014433in}{0.024851in}}{\pgfqpoint{-0.019642in}{0.019642in}}%
\pgfpathcurveto{\pgfqpoint{-0.024851in}{0.014433in}}{\pgfqpoint{-0.027778in}{0.007367in}}{\pgfqpoint{-0.027778in}{0.000000in}}%
\pgfpathcurveto{\pgfqpoint{-0.027778in}{-0.007367in}}{\pgfqpoint{-0.024851in}{-0.014433in}}{\pgfqpoint{-0.019642in}{-0.019642in}}%
\pgfpathcurveto{\pgfqpoint{-0.014433in}{-0.024851in}}{\pgfqpoint{-0.007367in}{-0.027778in}}{\pgfqpoint{0.000000in}{-0.027778in}}%
\pgfpathlineto{\pgfqpoint{0.000000in}{-0.027778in}}%
\pgfpathclose%
\pgfusepath{stroke,fill}%
}%
\begin{pgfscope}%
\pgfsys@transformshift{0.900530in}{1.731856in}%
\pgfsys@useobject{currentmarker}{}%
\end{pgfscope}%
\end{pgfscope}%
\begin{pgfscope}%
\definecolor{textcolor}{rgb}{0.000000,0.000000,0.000000}%
\pgfsetstrokecolor{textcolor}%
\pgfsetfillcolor{textcolor}%
\pgftext[x=1.075530in,y=1.697828in,left,base]{\color{textcolor}\rmfamily\fontsize{7.000000}{8.400000}\selectfont adaptive \(\displaystyle \gamma_i,\,K=1\)}%
\end{pgfscope}%
\begin{pgfscope}%
\pgfsetrectcap%
\pgfsetroundjoin%
\pgfsetlinewidth{1.505625pt}%
\definecolor{currentstroke}{rgb}{1.000000,0.498039,0.054902}%
\pgfsetstrokecolor{currentstroke}%
\pgfsetdash{}{0pt}%
\pgfpathmoveto{\pgfqpoint{0.803308in}{1.595896in}}%
\pgfpathlineto{\pgfqpoint{0.900530in}{1.595896in}}%
\pgfpathlineto{\pgfqpoint{0.997752in}{1.595896in}}%
\pgfusepath{stroke}%
\end{pgfscope}%
\begin{pgfscope}%
\pgfsetbuttcap%
\pgfsetmiterjoin%
\definecolor{currentfill}{rgb}{0.000000,0.000000,0.000000}%
\pgfsetfillcolor{currentfill}%
\pgfsetfillopacity{0.000000}%
\pgfsetlinewidth{1.003750pt}%
\definecolor{currentstroke}{rgb}{1.000000,0.498039,0.054902}%
\pgfsetstrokecolor{currentstroke}%
\pgfsetdash{}{0pt}%
\pgfsys@defobject{currentmarker}{\pgfqpoint{-0.027778in}{-0.027778in}}{\pgfqpoint{0.027778in}{0.027778in}}{%
\pgfpathmoveto{\pgfqpoint{-0.027778in}{-0.027778in}}%
\pgfpathlineto{\pgfqpoint{0.027778in}{-0.027778in}}%
\pgfpathlineto{\pgfqpoint{0.027778in}{0.027778in}}%
\pgfpathlineto{\pgfqpoint{-0.027778in}{0.027778in}}%
\pgfpathlineto{\pgfqpoint{-0.027778in}{-0.027778in}}%
\pgfpathclose%
\pgfusepath{stroke,fill}%
}%
\begin{pgfscope}%
\pgfsys@transformshift{0.900530in}{1.595896in}%
\pgfsys@useobject{currentmarker}{}%
\end{pgfscope}%
\end{pgfscope}%
\begin{pgfscope}%
\definecolor{textcolor}{rgb}{0.000000,0.000000,0.000000}%
\pgfsetstrokecolor{textcolor}%
\pgfsetfillcolor{textcolor}%
\pgftext[x=1.075530in,y=1.561869in,left,base]{\color{textcolor}\rmfamily\fontsize{7.000000}{8.400000}\selectfont fixed \(\displaystyle \gamma_i,\,K=1\)}%
\end{pgfscope}%
\begin{pgfscope}%
\pgfsetrectcap%
\pgfsetroundjoin%
\pgfsetlinewidth{1.505625pt}%
\definecolor{currentstroke}{rgb}{0.172549,0.627451,0.172549}%
\pgfsetstrokecolor{currentstroke}%
\pgfsetdash{}{0pt}%
\pgfpathmoveto{\pgfqpoint{2.051629in}{1.867384in}}%
\pgfpathlineto{\pgfqpoint{2.148851in}{1.867384in}}%
\pgfpathlineto{\pgfqpoint{2.246073in}{1.867384in}}%
\pgfusepath{stroke}%
\end{pgfscope}%
\begin{pgfscope}%
\pgfsetbuttcap%
\pgfsetroundjoin%
\definecolor{currentfill}{rgb}{0.000000,0.000000,0.000000}%
\pgfsetfillcolor{currentfill}%
\pgfsetfillopacity{0.000000}%
\pgfsetlinewidth{1.003750pt}%
\definecolor{currentstroke}{rgb}{0.172549,0.627451,0.172549}%
\pgfsetstrokecolor{currentstroke}%
\pgfsetdash{}{0pt}%
\pgfsys@defobject{currentmarker}{\pgfqpoint{-0.027778in}{-0.027778in}}{\pgfqpoint{0.027778in}{0.027778in}}{%
\pgfpathmoveto{\pgfqpoint{-0.027778in}{-0.027778in}}%
\pgfpathlineto{\pgfqpoint{0.027778in}{0.027778in}}%
\pgfpathmoveto{\pgfqpoint{-0.027778in}{0.027778in}}%
\pgfpathlineto{\pgfqpoint{0.027778in}{-0.027778in}}%
\pgfusepath{stroke,fill}%
}%
\begin{pgfscope}%
\pgfsys@transformshift{2.148851in}{1.867384in}%
\pgfsys@useobject{currentmarker}{}%
\end{pgfscope}%
\end{pgfscope}%
\begin{pgfscope}%
\definecolor{textcolor}{rgb}{0.000000,0.000000,0.000000}%
\pgfsetstrokecolor{textcolor}%
\pgfsetfillcolor{textcolor}%
\pgftext[x=2.323851in,y=1.833356in,left,base]{\color{textcolor}\rmfamily\fontsize{7.000000}{8.400000}\selectfont fixed \(\displaystyle \gamma_i,\,K=2\)}%
\end{pgfscope}%
\begin{pgfscope}%
\pgfsetrectcap%
\pgfsetroundjoin%
\pgfsetlinewidth{1.505625pt}%
\definecolor{currentstroke}{rgb}{0.839216,0.152941,0.156863}%
\pgfsetstrokecolor{currentstroke}%
\pgfsetdash{}{0pt}%
\pgfpathmoveto{\pgfqpoint{2.051629in}{1.731424in}}%
\pgfpathlineto{\pgfqpoint{2.148851in}{1.731424in}}%
\pgfpathlineto{\pgfqpoint{2.246073in}{1.731424in}}%
\pgfusepath{stroke}%
\end{pgfscope}%
\begin{pgfscope}%
\pgfsetbuttcap%
\pgfsetroundjoin%
\definecolor{currentfill}{rgb}{0.000000,0.000000,0.000000}%
\pgfsetfillcolor{currentfill}%
\pgfsetfillopacity{0.000000}%
\pgfsetlinewidth{1.003750pt}%
\definecolor{currentstroke}{rgb}{0.839216,0.152941,0.156863}%
\pgfsetstrokecolor{currentstroke}%
\pgfsetdash{}{0pt}%
\pgfsys@defobject{currentmarker}{\pgfqpoint{-0.027778in}{-0.027778in}}{\pgfqpoint{0.027778in}{0.027778in}}{%
\pgfpathmoveto{\pgfqpoint{-0.027778in}{0.000000in}}%
\pgfpathlineto{\pgfqpoint{0.027778in}{0.000000in}}%
\pgfpathmoveto{\pgfqpoint{0.000000in}{-0.027778in}}%
\pgfpathlineto{\pgfqpoint{0.000000in}{0.027778in}}%
\pgfusepath{stroke,fill}%
}%
\begin{pgfscope}%
\pgfsys@transformshift{2.148851in}{1.731424in}%
\pgfsys@useobject{currentmarker}{}%
\end{pgfscope}%
\end{pgfscope}%
\begin{pgfscope}%
\definecolor{textcolor}{rgb}{0.000000,0.000000,0.000000}%
\pgfsetstrokecolor{textcolor}%
\pgfsetfillcolor{textcolor}%
\pgftext[x=2.323851in,y=1.697396in,left,base]{\color{textcolor}\rmfamily\fontsize{7.000000}{8.400000}\selectfont fixed \(\displaystyle \gamma_i,\,K=10\)}%
\end{pgfscope}%
\end{pgfpicture}%
\makeatother%
\endgroup%

    \vspace*{-0.6cm}
    \caption[]{Mean of 500 post-convergence frames of median NPM of 30 Monte-Carlo runs for fixed \(\gamma_i\) on the interval \([0.0, 0.04]\) and \(K \in \{1,2,10\}\). Results for optimal method, and adaptive \(\gamma_i\) included for comparison. Hatched area in \autoref{fig:simulations:NPMtime} indicates frames used for computation of mean NPM.}
    \label{fig:simulations:avgNPMgamma}
\end{figure}
To evaluate the effectiveness of the proposed extension, simulations were run.
For this, we define the error measure as the normalized projection misalignment
\begin{equation}
    \begin{aligned}
        \text{NPM}(n) &= 20\,\log_{10} \frac{\left\| \mtxb{e} \right\|}{\left\|\hat{\h}(n)\right\|},\\
        \text{with}\quad \mtxb{e} &= \hat{\h}(n) - \frac{\hat{\h}^\T (n) \h}{\h^\T \h}\hat{\h}(n),
    \end{aligned}
\end{equation}
commonly used in BSI to compare estimated \(\hat{\h}(n)\) and true \(\h\).

The first simulation setup is a network with \(M=5\) nodes arranged in a ring topology, each node having \(|N_i|=2\) neighbors.
The input signal is zero-mean white Gaussian noise (WGN), the impulse responses of length \(L=16\) are drawn from a normal distribution, and at each channel, independent WGN is added with \(\text{SNR}=10\,\text{dB}\).
The norms of the impulse responses are scaled to random values drawn from the uniform distribution \(\mathcal{U}_{[0.5,2.0]}\).
A comparison is made between the following cases:
\begin{itemize}
    \itemsep-0.2em
    \item[(a)] \(\|\bar{\h}_i\|^2\) of all \(i \in \Mset\) is available for global norm computation ("optimal" in \autoref{fig:simulations:NPMtime} \& \autoref{fig:simulations:avgNPMgamma}),
    \item[(b)] Inter-neighborhood communication with fixed \(\gamma_i \in [0.0, 0.4]\) and \(K \in \{1,2,10\}\),
    \item[(c)] Inter-neighborhood communication with adaptive \(\gamma_i\) \eqref{eq:adaptivenormest:adaptivegamma} and \(K=1\).
\end{itemize}
\autoref{fig:simulations:NPMtime} shows the median of 30 Monte-Carlo runs.
We can observe that when fixing \(\gamma_i\), a tradeoff between convergence speed and steady-state error arises at low iteration counts (cf. also \autoref{fig:simulations:avgNPMgamma}).
With \(K=10\), the performance comes close to the optimal case at the cost of additional communication between nodes.
Compare that to the case with adaptive \(\gamma_i\) where even at \(K=1\), i.e., no additional communication for the recursive estimation scheme, convergence speed and steady-state error are close to the optimal case.

The second simulation setup is a \(3\)-node ring topology with a neighborhood size of \(|N_i|=2\).
The input signal, impulse responses, additive noise, and SNR have the same parameters as the first simulation.
To test the algorithms response to time-variant scenarios, the 3 impulse responses are then scaled to norms of \([2.2, 0.5, 1.2],\, [2.2, 1.0, 1.2],\, [2.2, 0.5, 2.0]\) at frames \(0,\, 5000,\, 10000\) respectively.
\autoref{fig:simulations:NPMtimedyn} (bottom) shows the median NPM of 30 Monte-Carlo runs, where we can observe very similar behavior of the optimal and distributed-averaging-based algorithms, where both algorithms reach a converged state after the rescaling events.
It further shows the damped oscillations that appear in the estimates of \(\|\h^n\|\) around the target value 1 \autoref{fig:simulations:NPMtimedyn} (top) and \(\|\hat{\h}_i^n\|\) around the rescaled true norms \autoref{fig:simulations:NPMtimedyn} (middle), respectively.

In summary, the simulations show that the right choice of fixed \(\gamma_i\) leads to results close to an optimal case with \(K=1\) distributed averaging iterations.
Further, introducing an adaptive \(\gamma_i^n\) allows the algorithm to converge faster and deal with time-variant systems.
This is an improvement over \cite{blochbergerDBSI} as communication is only required within neighborhoods and over \cite{liuDistributedBlindIdentification2016,liuDistributedRecursiveBlind2017} needing only \(K=1\) instead of \(K=50\) iterations.


\begin{figure}[t]
    \centering
    %% Creator: Matplotlib, PGF backend
%%
%% To include the figure in your LaTeX document, write
%%   \input{<filename>.pgf}
%%
%% Make sure the required packages are loaded in your preamble
%%   \usepackage{pgf}
%%
%% Also ensure that all the required font packages are loaded; for instance,
%% the lmodern package is sometimes necessary when using math font.
%%   \usepackage{lmodern}
%%
%% Figures using additional raster images can only be included by \input if
%% they are in the same directory as the main LaTeX file. For loading figures
%% from other directories you can use the `import` package
%%   \usepackage{import}
%%
%% and then include the figures with
%%   \import{<path to file>}{<filename>.pgf}
%%
%% Matplotlib used the following preamble
%%   \usepackage{fontspec}
%%
\begingroup%
\makeatletter%
\begin{pgfpicture}%
\pgfpathrectangle{\pgfpointorigin}{\pgfqpoint{3.390065in}{1.356026in}}%
\pgfusepath{use as bounding box, clip}%
\begin{pgfscope}%
\pgfsetbuttcap%
\pgfsetmiterjoin%
\definecolor{currentfill}{rgb}{1.000000,1.000000,1.000000}%
\pgfsetfillcolor{currentfill}%
\pgfsetlinewidth{0.000000pt}%
\definecolor{currentstroke}{rgb}{1.000000,1.000000,1.000000}%
\pgfsetstrokecolor{currentstroke}%
\pgfsetstrokeopacity{0.000000}%
\pgfsetdash{}{0pt}%
\pgfpathmoveto{\pgfqpoint{0.000000in}{0.000000in}}%
\pgfpathlineto{\pgfqpoint{3.390065in}{0.000000in}}%
\pgfpathlineto{\pgfqpoint{3.390065in}{1.356026in}}%
\pgfpathlineto{\pgfqpoint{0.000000in}{1.356026in}}%
\pgfpathlineto{\pgfqpoint{0.000000in}{0.000000in}}%
\pgfpathclose%
\pgfusepath{fill}%
\end{pgfscope}%
\begin{pgfscope}%
\pgfsetbuttcap%
\pgfsetmiterjoin%
\definecolor{currentfill}{rgb}{1.000000,1.000000,1.000000}%
\pgfsetfillcolor{currentfill}%
\pgfsetlinewidth{0.000000pt}%
\definecolor{currentstroke}{rgb}{0.000000,0.000000,0.000000}%
\pgfsetstrokecolor{currentstroke}%
\pgfsetstrokeopacity{0.000000}%
\pgfsetdash{}{0pt}%
\pgfpathmoveto{\pgfqpoint{0.568671in}{0.451277in}}%
\pgfpathlineto{\pgfqpoint{3.166976in}{0.451277in}}%
\pgfpathlineto{\pgfqpoint{3.166976in}{1.250151in}}%
\pgfpathlineto{\pgfqpoint{0.568671in}{1.250151in}}%
\pgfpathlineto{\pgfqpoint{0.568671in}{0.451277in}}%
\pgfpathclose%
\pgfusepath{fill}%
\end{pgfscope}%
\begin{pgfscope}%
\pgfpathrectangle{\pgfqpoint{0.568671in}{0.451277in}}{\pgfqpoint{2.598305in}{0.798874in}}%
\pgfusepath{clip}%
\pgfsetrectcap%
\pgfsetroundjoin%
\pgfsetlinewidth{0.803000pt}%
\definecolor{currentstroke}{rgb}{0.690196,0.690196,0.690196}%
\pgfsetstrokecolor{currentstroke}%
\pgfsetdash{}{0pt}%
\pgfpathmoveto{\pgfqpoint{0.568671in}{0.451277in}}%
\pgfpathlineto{\pgfqpoint{0.568671in}{1.250151in}}%
\pgfusepath{stroke}%
\end{pgfscope}%
\begin{pgfscope}%
\pgfsetbuttcap%
\pgfsetroundjoin%
\definecolor{currentfill}{rgb}{0.000000,0.000000,0.000000}%
\pgfsetfillcolor{currentfill}%
\pgfsetlinewidth{0.803000pt}%
\definecolor{currentstroke}{rgb}{0.000000,0.000000,0.000000}%
\pgfsetstrokecolor{currentstroke}%
\pgfsetdash{}{0pt}%
\pgfsys@defobject{currentmarker}{\pgfqpoint{0.000000in}{-0.048611in}}{\pgfqpoint{0.000000in}{0.000000in}}{%
\pgfpathmoveto{\pgfqpoint{0.000000in}{0.000000in}}%
\pgfpathlineto{\pgfqpoint{0.000000in}{-0.048611in}}%
\pgfusepath{stroke,fill}%
}%
\begin{pgfscope}%
\pgfsys@transformshift{0.568671in}{0.451277in}%
\pgfsys@useobject{currentmarker}{}%
\end{pgfscope}%
\end{pgfscope}%
\begin{pgfscope}%
\pgfpathrectangle{\pgfqpoint{0.568671in}{0.451277in}}{\pgfqpoint{2.598305in}{0.798874in}}%
\pgfusepath{clip}%
\pgfsetrectcap%
\pgfsetroundjoin%
\pgfsetlinewidth{0.803000pt}%
\definecolor{currentstroke}{rgb}{0.690196,0.690196,0.690196}%
\pgfsetstrokecolor{currentstroke}%
\pgfsetdash{}{0pt}%
\pgfpathmoveto{\pgfqpoint{1.001722in}{0.451277in}}%
\pgfpathlineto{\pgfqpoint{1.001722in}{1.250151in}}%
\pgfusepath{stroke}%
\end{pgfscope}%
\begin{pgfscope}%
\pgfsetbuttcap%
\pgfsetroundjoin%
\definecolor{currentfill}{rgb}{0.000000,0.000000,0.000000}%
\pgfsetfillcolor{currentfill}%
\pgfsetlinewidth{0.803000pt}%
\definecolor{currentstroke}{rgb}{0.000000,0.000000,0.000000}%
\pgfsetstrokecolor{currentstroke}%
\pgfsetdash{}{0pt}%
\pgfsys@defobject{currentmarker}{\pgfqpoint{0.000000in}{-0.048611in}}{\pgfqpoint{0.000000in}{0.000000in}}{%
\pgfpathmoveto{\pgfqpoint{0.000000in}{0.000000in}}%
\pgfpathlineto{\pgfqpoint{0.000000in}{-0.048611in}}%
\pgfusepath{stroke,fill}%
}%
\begin{pgfscope}%
\pgfsys@transformshift{1.001722in}{0.451277in}%
\pgfsys@useobject{currentmarker}{}%
\end{pgfscope}%
\end{pgfscope}%
\begin{pgfscope}%
\pgfpathrectangle{\pgfqpoint{0.568671in}{0.451277in}}{\pgfqpoint{2.598305in}{0.798874in}}%
\pgfusepath{clip}%
\pgfsetrectcap%
\pgfsetroundjoin%
\pgfsetlinewidth{0.803000pt}%
\definecolor{currentstroke}{rgb}{0.690196,0.690196,0.690196}%
\pgfsetstrokecolor{currentstroke}%
\pgfsetdash{}{0pt}%
\pgfpathmoveto{\pgfqpoint{1.434773in}{0.451277in}}%
\pgfpathlineto{\pgfqpoint{1.434773in}{1.250151in}}%
\pgfusepath{stroke}%
\end{pgfscope}%
\begin{pgfscope}%
\pgfsetbuttcap%
\pgfsetroundjoin%
\definecolor{currentfill}{rgb}{0.000000,0.000000,0.000000}%
\pgfsetfillcolor{currentfill}%
\pgfsetlinewidth{0.803000pt}%
\definecolor{currentstroke}{rgb}{0.000000,0.000000,0.000000}%
\pgfsetstrokecolor{currentstroke}%
\pgfsetdash{}{0pt}%
\pgfsys@defobject{currentmarker}{\pgfqpoint{0.000000in}{-0.048611in}}{\pgfqpoint{0.000000in}{0.000000in}}{%
\pgfpathmoveto{\pgfqpoint{0.000000in}{0.000000in}}%
\pgfpathlineto{\pgfqpoint{0.000000in}{-0.048611in}}%
\pgfusepath{stroke,fill}%
}%
\begin{pgfscope}%
\pgfsys@transformshift{1.434773in}{0.451277in}%
\pgfsys@useobject{currentmarker}{}%
\end{pgfscope}%
\end{pgfscope}%
\begin{pgfscope}%
\pgfpathrectangle{\pgfqpoint{0.568671in}{0.451277in}}{\pgfqpoint{2.598305in}{0.798874in}}%
\pgfusepath{clip}%
\pgfsetrectcap%
\pgfsetroundjoin%
\pgfsetlinewidth{0.803000pt}%
\definecolor{currentstroke}{rgb}{0.690196,0.690196,0.690196}%
\pgfsetstrokecolor{currentstroke}%
\pgfsetdash{}{0pt}%
\pgfpathmoveto{\pgfqpoint{1.867823in}{0.451277in}}%
\pgfpathlineto{\pgfqpoint{1.867823in}{1.250151in}}%
\pgfusepath{stroke}%
\end{pgfscope}%
\begin{pgfscope}%
\pgfsetbuttcap%
\pgfsetroundjoin%
\definecolor{currentfill}{rgb}{0.000000,0.000000,0.000000}%
\pgfsetfillcolor{currentfill}%
\pgfsetlinewidth{0.803000pt}%
\definecolor{currentstroke}{rgb}{0.000000,0.000000,0.000000}%
\pgfsetstrokecolor{currentstroke}%
\pgfsetdash{}{0pt}%
\pgfsys@defobject{currentmarker}{\pgfqpoint{0.000000in}{-0.048611in}}{\pgfqpoint{0.000000in}{0.000000in}}{%
\pgfpathmoveto{\pgfqpoint{0.000000in}{0.000000in}}%
\pgfpathlineto{\pgfqpoint{0.000000in}{-0.048611in}}%
\pgfusepath{stroke,fill}%
}%
\begin{pgfscope}%
\pgfsys@transformshift{1.867823in}{0.451277in}%
\pgfsys@useobject{currentmarker}{}%
\end{pgfscope}%
\end{pgfscope}%
\begin{pgfscope}%
\pgfpathrectangle{\pgfqpoint{0.568671in}{0.451277in}}{\pgfqpoint{2.598305in}{0.798874in}}%
\pgfusepath{clip}%
\pgfsetrectcap%
\pgfsetroundjoin%
\pgfsetlinewidth{0.803000pt}%
\definecolor{currentstroke}{rgb}{0.690196,0.690196,0.690196}%
\pgfsetstrokecolor{currentstroke}%
\pgfsetdash{}{0pt}%
\pgfpathmoveto{\pgfqpoint{2.300874in}{0.451277in}}%
\pgfpathlineto{\pgfqpoint{2.300874in}{1.250151in}}%
\pgfusepath{stroke}%
\end{pgfscope}%
\begin{pgfscope}%
\pgfsetbuttcap%
\pgfsetroundjoin%
\definecolor{currentfill}{rgb}{0.000000,0.000000,0.000000}%
\pgfsetfillcolor{currentfill}%
\pgfsetlinewidth{0.803000pt}%
\definecolor{currentstroke}{rgb}{0.000000,0.000000,0.000000}%
\pgfsetstrokecolor{currentstroke}%
\pgfsetdash{}{0pt}%
\pgfsys@defobject{currentmarker}{\pgfqpoint{0.000000in}{-0.048611in}}{\pgfqpoint{0.000000in}{0.000000in}}{%
\pgfpathmoveto{\pgfqpoint{0.000000in}{0.000000in}}%
\pgfpathlineto{\pgfqpoint{0.000000in}{-0.048611in}}%
\pgfusepath{stroke,fill}%
}%
\begin{pgfscope}%
\pgfsys@transformshift{2.300874in}{0.451277in}%
\pgfsys@useobject{currentmarker}{}%
\end{pgfscope}%
\end{pgfscope}%
\begin{pgfscope}%
\pgfpathrectangle{\pgfqpoint{0.568671in}{0.451277in}}{\pgfqpoint{2.598305in}{0.798874in}}%
\pgfusepath{clip}%
\pgfsetrectcap%
\pgfsetroundjoin%
\pgfsetlinewidth{0.803000pt}%
\definecolor{currentstroke}{rgb}{0.690196,0.690196,0.690196}%
\pgfsetstrokecolor{currentstroke}%
\pgfsetdash{}{0pt}%
\pgfpathmoveto{\pgfqpoint{2.733925in}{0.451277in}}%
\pgfpathlineto{\pgfqpoint{2.733925in}{1.250151in}}%
\pgfusepath{stroke}%
\end{pgfscope}%
\begin{pgfscope}%
\pgfsetbuttcap%
\pgfsetroundjoin%
\definecolor{currentfill}{rgb}{0.000000,0.000000,0.000000}%
\pgfsetfillcolor{currentfill}%
\pgfsetlinewidth{0.803000pt}%
\definecolor{currentstroke}{rgb}{0.000000,0.000000,0.000000}%
\pgfsetstrokecolor{currentstroke}%
\pgfsetdash{}{0pt}%
\pgfsys@defobject{currentmarker}{\pgfqpoint{0.000000in}{-0.048611in}}{\pgfqpoint{0.000000in}{0.000000in}}{%
\pgfpathmoveto{\pgfqpoint{0.000000in}{0.000000in}}%
\pgfpathlineto{\pgfqpoint{0.000000in}{-0.048611in}}%
\pgfusepath{stroke,fill}%
}%
\begin{pgfscope}%
\pgfsys@transformshift{2.733925in}{0.451277in}%
\pgfsys@useobject{currentmarker}{}%
\end{pgfscope}%
\end{pgfscope}%
\begin{pgfscope}%
\pgfpathrectangle{\pgfqpoint{0.568671in}{0.451277in}}{\pgfqpoint{2.598305in}{0.798874in}}%
\pgfusepath{clip}%
\pgfsetrectcap%
\pgfsetroundjoin%
\pgfsetlinewidth{0.803000pt}%
\definecolor{currentstroke}{rgb}{0.690196,0.690196,0.690196}%
\pgfsetstrokecolor{currentstroke}%
\pgfsetdash{}{0pt}%
\pgfpathmoveto{\pgfqpoint{3.166976in}{0.451277in}}%
\pgfpathlineto{\pgfqpoint{3.166976in}{1.250151in}}%
\pgfusepath{stroke}%
\end{pgfscope}%
\begin{pgfscope}%
\pgfsetbuttcap%
\pgfsetroundjoin%
\definecolor{currentfill}{rgb}{0.000000,0.000000,0.000000}%
\pgfsetfillcolor{currentfill}%
\pgfsetlinewidth{0.803000pt}%
\definecolor{currentstroke}{rgb}{0.000000,0.000000,0.000000}%
\pgfsetstrokecolor{currentstroke}%
\pgfsetdash{}{0pt}%
\pgfsys@defobject{currentmarker}{\pgfqpoint{0.000000in}{-0.048611in}}{\pgfqpoint{0.000000in}{0.000000in}}{%
\pgfpathmoveto{\pgfqpoint{0.000000in}{0.000000in}}%
\pgfpathlineto{\pgfqpoint{0.000000in}{-0.048611in}}%
\pgfusepath{stroke,fill}%
}%
\begin{pgfscope}%
\pgfsys@transformshift{3.166976in}{0.451277in}%
\pgfsys@useobject{currentmarker}{}%
\end{pgfscope}%
\end{pgfscope}%
\begin{pgfscope}%
\pgfpathrectangle{\pgfqpoint{0.568671in}{0.451277in}}{\pgfqpoint{2.598305in}{0.798874in}}%
\pgfusepath{clip}%
\pgfsetrectcap%
\pgfsetroundjoin%
\pgfsetlinewidth{0.803000pt}%
\definecolor{currentstroke}{rgb}{0.690196,0.690196,0.690196}%
\pgfsetstrokecolor{currentstroke}%
\pgfsetdash{}{0pt}%
\pgfpathmoveto{\pgfqpoint{0.568671in}{0.531165in}}%
\pgfpathlineto{\pgfqpoint{3.166976in}{0.531165in}}%
\pgfusepath{stroke}%
\end{pgfscope}%
\begin{pgfscope}%
\pgfsetbuttcap%
\pgfsetroundjoin%
\definecolor{currentfill}{rgb}{0.000000,0.000000,0.000000}%
\pgfsetfillcolor{currentfill}%
\pgfsetlinewidth{0.803000pt}%
\definecolor{currentstroke}{rgb}{0.000000,0.000000,0.000000}%
\pgfsetstrokecolor{currentstroke}%
\pgfsetdash{}{0pt}%
\pgfsys@defobject{currentmarker}{\pgfqpoint{-0.048611in}{0.000000in}}{\pgfqpoint{-0.000000in}{0.000000in}}{%
\pgfpathmoveto{\pgfqpoint{-0.000000in}{0.000000in}}%
\pgfpathlineto{\pgfqpoint{-0.048611in}{0.000000in}}%
\pgfusepath{stroke,fill}%
}%
\begin{pgfscope}%
\pgfsys@transformshift{0.568671in}{0.531165in}%
\pgfsys@useobject{currentmarker}{}%
\end{pgfscope}%
\end{pgfscope}%
\begin{pgfscope}%
\definecolor{textcolor}{rgb}{0.000000,0.000000,0.000000}%
\pgfsetstrokecolor{textcolor}%
\pgfsetfillcolor{textcolor}%
\pgftext[x=0.307291in, y=0.487790in, left, base]{\color{textcolor}\rmfamily\fontsize{9.000000}{10.800000}\selectfont \(\displaystyle {0.8}\)}%
\end{pgfscope}%
\begin{pgfscope}%
\pgfpathrectangle{\pgfqpoint{0.568671in}{0.451277in}}{\pgfqpoint{2.598305in}{0.798874in}}%
\pgfusepath{clip}%
\pgfsetrectcap%
\pgfsetroundjoin%
\pgfsetlinewidth{0.803000pt}%
\definecolor{currentstroke}{rgb}{0.690196,0.690196,0.690196}%
\pgfsetstrokecolor{currentstroke}%
\pgfsetdash{}{0pt}%
\pgfpathmoveto{\pgfqpoint{0.568671in}{0.850714in}}%
\pgfpathlineto{\pgfqpoint{3.166976in}{0.850714in}}%
\pgfusepath{stroke}%
\end{pgfscope}%
\begin{pgfscope}%
\pgfsetbuttcap%
\pgfsetroundjoin%
\definecolor{currentfill}{rgb}{0.000000,0.000000,0.000000}%
\pgfsetfillcolor{currentfill}%
\pgfsetlinewidth{0.803000pt}%
\definecolor{currentstroke}{rgb}{0.000000,0.000000,0.000000}%
\pgfsetstrokecolor{currentstroke}%
\pgfsetdash{}{0pt}%
\pgfsys@defobject{currentmarker}{\pgfqpoint{-0.048611in}{0.000000in}}{\pgfqpoint{-0.000000in}{0.000000in}}{%
\pgfpathmoveto{\pgfqpoint{-0.000000in}{0.000000in}}%
\pgfpathlineto{\pgfqpoint{-0.048611in}{0.000000in}}%
\pgfusepath{stroke,fill}%
}%
\begin{pgfscope}%
\pgfsys@transformshift{0.568671in}{0.850714in}%
\pgfsys@useobject{currentmarker}{}%
\end{pgfscope}%
\end{pgfscope}%
\begin{pgfscope}%
\definecolor{textcolor}{rgb}{0.000000,0.000000,0.000000}%
\pgfsetstrokecolor{textcolor}%
\pgfsetfillcolor{textcolor}%
\pgftext[x=0.307291in, y=0.807339in, left, base]{\color{textcolor}\rmfamily\fontsize{9.000000}{10.800000}\selectfont \(\displaystyle {1.0}\)}%
\end{pgfscope}%
\begin{pgfscope}%
\pgfpathrectangle{\pgfqpoint{0.568671in}{0.451277in}}{\pgfqpoint{2.598305in}{0.798874in}}%
\pgfusepath{clip}%
\pgfsetrectcap%
\pgfsetroundjoin%
\pgfsetlinewidth{0.803000pt}%
\definecolor{currentstroke}{rgb}{0.690196,0.690196,0.690196}%
\pgfsetstrokecolor{currentstroke}%
\pgfsetdash{}{0pt}%
\pgfpathmoveto{\pgfqpoint{0.568671in}{1.170264in}}%
\pgfpathlineto{\pgfqpoint{3.166976in}{1.170264in}}%
\pgfusepath{stroke}%
\end{pgfscope}%
\begin{pgfscope}%
\pgfsetbuttcap%
\pgfsetroundjoin%
\definecolor{currentfill}{rgb}{0.000000,0.000000,0.000000}%
\pgfsetfillcolor{currentfill}%
\pgfsetlinewidth{0.803000pt}%
\definecolor{currentstroke}{rgb}{0.000000,0.000000,0.000000}%
\pgfsetstrokecolor{currentstroke}%
\pgfsetdash{}{0pt}%
\pgfsys@defobject{currentmarker}{\pgfqpoint{-0.048611in}{0.000000in}}{\pgfqpoint{-0.000000in}{0.000000in}}{%
\pgfpathmoveto{\pgfqpoint{-0.000000in}{0.000000in}}%
\pgfpathlineto{\pgfqpoint{-0.048611in}{0.000000in}}%
\pgfusepath{stroke,fill}%
}%
\begin{pgfscope}%
\pgfsys@transformshift{0.568671in}{1.170264in}%
\pgfsys@useobject{currentmarker}{}%
\end{pgfscope}%
\end{pgfscope}%
\begin{pgfscope}%
\definecolor{textcolor}{rgb}{0.000000,0.000000,0.000000}%
\pgfsetstrokecolor{textcolor}%
\pgfsetfillcolor{textcolor}%
\pgftext[x=0.307291in, y=1.126889in, left, base]{\color{textcolor}\rmfamily\fontsize{9.000000}{10.800000}\selectfont \(\displaystyle {1.2}\)}%
\end{pgfscope}%
\begin{pgfscope}%
\definecolor{textcolor}{rgb}{0.000000,0.000000,0.000000}%
\pgfsetstrokecolor{textcolor}%
\pgfsetfillcolor{textcolor}%
\pgftext[x=0.251735in,y=0.850714in,,bottom,rotate=90.000000]{\color{textcolor}\rmfamily\fontsize{9.000000}{10.800000}\selectfont \(\displaystyle \|\mathbf{h}\|\) [1]}%
\end{pgfscope}%
\begin{pgfscope}%
\pgfpathrectangle{\pgfqpoint{0.568671in}{0.451277in}}{\pgfqpoint{2.598305in}{0.798874in}}%
\pgfusepath{clip}%
\pgfsetbuttcap%
\pgfsetroundjoin%
\pgfsetlinewidth{0.752812pt}%
\definecolor{currentstroke}{rgb}{0.000000,0.000000,0.000000}%
\pgfsetstrokecolor{currentstroke}%
\pgfsetdash{{2.775000pt}{1.200000pt}}{0.000000pt}%
\pgfpathmoveto{\pgfqpoint{1.434773in}{0.451277in}}%
\pgfpathlineto{\pgfqpoint{1.434773in}{1.250151in}}%
\pgfusepath{stroke}%
\end{pgfscope}%
\begin{pgfscope}%
\pgfpathrectangle{\pgfqpoint{0.568671in}{0.451277in}}{\pgfqpoint{2.598305in}{0.798874in}}%
\pgfusepath{clip}%
\pgfsetbuttcap%
\pgfsetroundjoin%
\pgfsetlinewidth{0.752812pt}%
\definecolor{currentstroke}{rgb}{0.000000,0.000000,0.000000}%
\pgfsetstrokecolor{currentstroke}%
\pgfsetdash{{2.775000pt}{1.200000pt}}{0.000000pt}%
\pgfpathmoveto{\pgfqpoint{2.300874in}{0.451277in}}%
\pgfpathlineto{\pgfqpoint{2.300874in}{1.250151in}}%
\pgfusepath{stroke}%
\end{pgfscope}%
\begin{pgfscope}%
\pgfpathrectangle{\pgfqpoint{0.568671in}{0.451277in}}{\pgfqpoint{2.598305in}{0.798874in}}%
\pgfusepath{clip}%
\pgfsetrectcap%
\pgfsetroundjoin%
\pgfsetlinewidth{1.505625pt}%
\definecolor{currentstroke}{rgb}{0.121569,0.466667,0.705882}%
\pgfsetstrokecolor{currentstroke}%
\pgfsetdash{}{0pt}%
\pgfpathmoveto{\pgfqpoint{0.568671in}{0.854716in}}%
\pgfpathlineto{\pgfqpoint{0.568844in}{0.511885in}}%
\pgfpathlineto{\pgfqpoint{0.569884in}{1.169838in}}%
\pgfpathlineto{\pgfqpoint{0.570577in}{1.248293in}}%
\pgfpathlineto{\pgfqpoint{0.571789in}{1.198932in}}%
\pgfpathlineto{\pgfqpoint{0.572309in}{1.183968in}}%
\pgfpathlineto{\pgfqpoint{0.575253in}{1.081666in}}%
\pgfpathlineto{\pgfqpoint{0.575427in}{1.083758in}}%
\pgfpathlineto{\pgfqpoint{0.575600in}{1.078251in}}%
\pgfpathlineto{\pgfqpoint{0.576466in}{1.078324in}}%
\pgfpathlineto{\pgfqpoint{0.578891in}{1.065606in}}%
\pgfpathlineto{\pgfqpoint{0.579238in}{1.065708in}}%
\pgfpathlineto{\pgfqpoint{0.582702in}{1.071245in}}%
\pgfpathlineto{\pgfqpoint{0.583048in}{1.070980in}}%
\pgfpathlineto{\pgfqpoint{0.585127in}{1.070358in}}%
\pgfpathlineto{\pgfqpoint{0.588418in}{1.067768in}}%
\pgfpathlineto{\pgfqpoint{0.592749in}{1.046627in}}%
\pgfpathlineto{\pgfqpoint{0.594134in}{1.050471in}}%
\pgfpathlineto{\pgfqpoint{0.595693in}{1.053849in}}%
\pgfpathlineto{\pgfqpoint{0.596213in}{1.053124in}}%
\pgfpathlineto{\pgfqpoint{0.598119in}{1.043982in}}%
\pgfpathlineto{\pgfqpoint{0.605047in}{0.985305in}}%
\pgfpathlineto{\pgfqpoint{0.608512in}{0.967166in}}%
\pgfpathlineto{\pgfqpoint{0.612669in}{0.953327in}}%
\pgfpathlineto{\pgfqpoint{0.620291in}{0.911717in}}%
\pgfpathlineto{\pgfqpoint{0.621677in}{0.913974in}}%
\pgfpathlineto{\pgfqpoint{0.624794in}{0.929484in}}%
\pgfpathlineto{\pgfqpoint{0.628605in}{0.945097in}}%
\pgfpathlineto{\pgfqpoint{0.629645in}{0.944422in}}%
\pgfpathlineto{\pgfqpoint{0.629818in}{0.943967in}}%
\pgfpathlineto{\pgfqpoint{0.631723in}{0.932931in}}%
\pgfpathlineto{\pgfqpoint{0.636227in}{0.909383in}}%
\pgfpathlineto{\pgfqpoint{0.638306in}{0.909863in}}%
\pgfpathlineto{\pgfqpoint{0.641943in}{0.915230in}}%
\pgfpathlineto{\pgfqpoint{0.643849in}{0.916695in}}%
\pgfpathlineto{\pgfqpoint{0.644368in}{0.916079in}}%
\pgfpathlineto{\pgfqpoint{0.645754in}{0.911697in}}%
\pgfpathlineto{\pgfqpoint{0.656494in}{0.854973in}}%
\pgfpathlineto{\pgfqpoint{0.662557in}{0.839154in}}%
\pgfpathlineto{\pgfqpoint{0.664635in}{0.841174in}}%
\pgfpathlineto{\pgfqpoint{0.667233in}{0.849361in}}%
\pgfpathlineto{\pgfqpoint{0.680571in}{0.893534in}}%
\pgfpathlineto{\pgfqpoint{0.683170in}{0.895353in}}%
\pgfpathlineto{\pgfqpoint{0.683343in}{0.895219in}}%
\pgfpathlineto{\pgfqpoint{0.688713in}{0.893020in}}%
\pgfpathlineto{\pgfqpoint{0.694775in}{0.897299in}}%
\pgfpathlineto{\pgfqpoint{0.696334in}{0.894914in}}%
\pgfpathlineto{\pgfqpoint{0.702051in}{0.882344in}}%
\pgfpathlineto{\pgfqpoint{0.702224in}{0.882411in}}%
\pgfpathlineto{\pgfqpoint{0.704995in}{0.887795in}}%
\pgfpathlineto{\pgfqpoint{0.711751in}{0.902095in}}%
\pgfpathlineto{\pgfqpoint{0.714523in}{0.902107in}}%
\pgfpathlineto{\pgfqpoint{0.722491in}{0.897822in}}%
\pgfpathlineto{\pgfqpoint{0.738081in}{0.901097in}}%
\pgfpathlineto{\pgfqpoint{0.745529in}{0.895314in}}%
\pgfpathlineto{\pgfqpoint{0.748993in}{0.889373in}}%
\pgfpathlineto{\pgfqpoint{0.760253in}{0.867603in}}%
\pgfpathlineto{\pgfqpoint{0.763371in}{0.865364in}}%
\pgfpathlineto{\pgfqpoint{0.766835in}{0.864389in}}%
\pgfpathlineto{\pgfqpoint{0.772725in}{0.866700in}}%
\pgfpathlineto{\pgfqpoint{0.780346in}{0.873165in}}%
\pgfpathlineto{\pgfqpoint{0.792991in}{0.903998in}}%
\pgfpathlineto{\pgfqpoint{0.794897in}{0.903073in}}%
\pgfpathlineto{\pgfqpoint{0.804944in}{0.892208in}}%
\pgfpathlineto{\pgfqpoint{0.822785in}{0.882486in}}%
\pgfpathlineto{\pgfqpoint{0.829194in}{0.888574in}}%
\pgfpathlineto{\pgfqpoint{0.833525in}{0.890340in}}%
\pgfpathlineto{\pgfqpoint{0.836643in}{0.889109in}}%
\pgfpathlineto{\pgfqpoint{0.843572in}{0.877738in}}%
\pgfpathlineto{\pgfqpoint{0.850501in}{0.868150in}}%
\pgfpathlineto{\pgfqpoint{0.853792in}{0.869268in}}%
\pgfpathlineto{\pgfqpoint{0.867476in}{0.881919in}}%
\pgfpathlineto{\pgfqpoint{0.876830in}{0.882889in}}%
\pgfpathlineto{\pgfqpoint{0.887223in}{0.877935in}}%
\pgfpathlineto{\pgfqpoint{0.892420in}{0.881242in}}%
\pgfpathlineto{\pgfqpoint{0.895711in}{0.885299in}}%
\pgfpathlineto{\pgfqpoint{0.901774in}{0.893374in}}%
\pgfpathlineto{\pgfqpoint{0.905585in}{0.893500in}}%
\pgfpathlineto{\pgfqpoint{0.909049in}{0.891138in}}%
\pgfpathlineto{\pgfqpoint{0.913553in}{0.888056in}}%
\pgfpathlineto{\pgfqpoint{0.921174in}{0.881541in}}%
\pgfpathlineto{\pgfqpoint{0.932953in}{0.864913in}}%
\pgfpathlineto{\pgfqpoint{0.943173in}{0.845848in}}%
\pgfpathlineto{\pgfqpoint{0.948543in}{0.842082in}}%
\pgfpathlineto{\pgfqpoint{0.952008in}{0.842884in}}%
\pgfpathlineto{\pgfqpoint{0.958070in}{0.850125in}}%
\pgfpathlineto{\pgfqpoint{0.962921in}{0.862850in}}%
\pgfpathlineto{\pgfqpoint{0.978510in}{0.909082in}}%
\pgfpathlineto{\pgfqpoint{0.981975in}{0.910075in}}%
\pgfpathlineto{\pgfqpoint{0.984920in}{0.906532in}}%
\pgfpathlineto{\pgfqpoint{1.006745in}{0.867894in}}%
\pgfpathlineto{\pgfqpoint{1.009863in}{0.868541in}}%
\pgfpathlineto{\pgfqpoint{1.014713in}{0.875659in}}%
\pgfpathlineto{\pgfqpoint{1.024760in}{0.896324in}}%
\pgfpathlineto{\pgfqpoint{1.031516in}{0.908773in}}%
\pgfpathlineto{\pgfqpoint{1.034634in}{0.911708in}}%
\pgfpathlineto{\pgfqpoint{1.037578in}{0.910547in}}%
\pgfpathlineto{\pgfqpoint{1.041043in}{0.904930in}}%
\pgfpathlineto{\pgfqpoint{1.048665in}{0.892060in}}%
\pgfpathlineto{\pgfqpoint{1.053861in}{0.880059in}}%
\pgfpathlineto{\pgfqpoint{1.062869in}{0.858537in}}%
\pgfpathlineto{\pgfqpoint{1.067026in}{0.854284in}}%
\pgfpathlineto{\pgfqpoint{1.077246in}{0.850695in}}%
\pgfpathlineto{\pgfqpoint{1.081576in}{0.856220in}}%
\pgfpathlineto{\pgfqpoint{1.103229in}{0.893855in}}%
\pgfpathlineto{\pgfqpoint{1.106174in}{0.894341in}}%
\pgfpathlineto{\pgfqpoint{1.113276in}{0.891388in}}%
\pgfpathlineto{\pgfqpoint{1.118299in}{0.879113in}}%
\pgfpathlineto{\pgfqpoint{1.124362in}{0.866112in}}%
\pgfpathlineto{\pgfqpoint{1.127307in}{0.864227in}}%
\pgfpathlineto{\pgfqpoint{1.130598in}{0.862023in}}%
\pgfpathlineto{\pgfqpoint{1.136141in}{0.861163in}}%
\pgfpathlineto{\pgfqpoint{1.141511in}{0.863218in}}%
\pgfpathlineto{\pgfqpoint{1.146534in}{0.871600in}}%
\pgfpathlineto{\pgfqpoint{1.154675in}{0.882199in}}%
\pgfpathlineto{\pgfqpoint{1.174769in}{0.896194in}}%
\pgfpathlineto{\pgfqpoint{1.190705in}{0.896760in}}%
\pgfpathlineto{\pgfqpoint{1.194343in}{0.890953in}}%
\pgfpathlineto{\pgfqpoint{1.201272in}{0.883875in}}%
\pgfpathlineto{\pgfqpoint{1.204390in}{0.884817in}}%
\pgfpathlineto{\pgfqpoint{1.225003in}{0.903917in}}%
\pgfpathlineto{\pgfqpoint{1.232451in}{0.898237in}}%
\pgfpathlineto{\pgfqpoint{1.237821in}{0.894214in}}%
\pgfpathlineto{\pgfqpoint{1.241805in}{0.891479in}}%
\pgfpathlineto{\pgfqpoint{1.245096in}{0.886911in}}%
\pgfpathlineto{\pgfqpoint{1.251679in}{0.875436in}}%
\pgfpathlineto{\pgfqpoint{1.259993in}{0.871972in}}%
\pgfpathlineto{\pgfqpoint{1.267442in}{0.870650in}}%
\pgfpathlineto{\pgfqpoint{1.276103in}{0.877305in}}%
\pgfpathlineto{\pgfqpoint{1.286669in}{0.890123in}}%
\pgfpathlineto{\pgfqpoint{1.290134in}{0.891615in}}%
\pgfpathlineto{\pgfqpoint{1.298622in}{0.890219in}}%
\pgfpathlineto{\pgfqpoint{1.310920in}{0.880009in}}%
\pgfpathlineto{\pgfqpoint{1.320794in}{0.869080in}}%
\pgfpathlineto{\pgfqpoint{1.323565in}{0.871035in}}%
\pgfpathlineto{\pgfqpoint{1.331360in}{0.882624in}}%
\pgfpathlineto{\pgfqpoint{1.334825in}{0.884915in}}%
\pgfpathlineto{\pgfqpoint{1.353186in}{0.885469in}}%
\pgfpathlineto{\pgfqpoint{1.362020in}{0.883582in}}%
\pgfpathlineto{\pgfqpoint{1.371201in}{0.884785in}}%
\pgfpathlineto{\pgfqpoint{1.376051in}{0.884206in}}%
\pgfpathlineto{\pgfqpoint{1.384192in}{0.889697in}}%
\pgfpathlineto{\pgfqpoint{1.390082in}{0.892755in}}%
\pgfpathlineto{\pgfqpoint{1.392680in}{0.894105in}}%
\pgfpathlineto{\pgfqpoint{1.397704in}{0.895356in}}%
\pgfpathlineto{\pgfqpoint{1.417624in}{0.869790in}}%
\pgfpathlineto{\pgfqpoint{1.422128in}{0.863470in}}%
\pgfpathlineto{\pgfqpoint{1.425938in}{0.865056in}}%
\pgfpathlineto{\pgfqpoint{1.434253in}{0.874232in}}%
\pgfpathlineto{\pgfqpoint{1.434426in}{0.872963in}}%
\pgfpathlineto{\pgfqpoint{1.435119in}{0.856367in}}%
\pgfpathlineto{\pgfqpoint{1.438930in}{0.666988in}}%
\pgfpathlineto{\pgfqpoint{1.442568in}{0.577481in}}%
\pgfpathlineto{\pgfqpoint{1.444993in}{0.566098in}}%
\pgfpathlineto{\pgfqpoint{1.445166in}{0.566118in}}%
\pgfpathlineto{\pgfqpoint{1.446032in}{0.568133in}}%
\pgfpathlineto{\pgfqpoint{1.447764in}{0.582513in}}%
\pgfpathlineto{\pgfqpoint{1.450189in}{0.635235in}}%
\pgfpathlineto{\pgfqpoint{1.453654in}{0.796087in}}%
\pgfpathlineto{\pgfqpoint{1.460236in}{1.099293in}}%
\pgfpathlineto{\pgfqpoint{1.463527in}{1.134154in}}%
\pgfpathlineto{\pgfqpoint{1.466299in}{1.139335in}}%
\pgfpathlineto{\pgfqpoint{1.466472in}{1.139262in}}%
\pgfpathlineto{\pgfqpoint{1.468377in}{1.137410in}}%
\pgfpathlineto{\pgfqpoint{1.472015in}{1.129128in}}%
\pgfpathlineto{\pgfqpoint{1.474787in}{1.111625in}}%
\pgfpathlineto{\pgfqpoint{1.477731in}{1.072112in}}%
\pgfpathlineto{\pgfqpoint{1.483101in}{0.955984in}}%
\pgfpathlineto{\pgfqpoint{1.496959in}{0.653652in}}%
\pgfpathlineto{\pgfqpoint{1.499904in}{0.645312in}}%
\pgfpathlineto{\pgfqpoint{1.501289in}{0.647694in}}%
\pgfpathlineto{\pgfqpoint{1.503888in}{0.659710in}}%
\pgfpathlineto{\pgfqpoint{1.508218in}{0.695859in}}%
\pgfpathlineto{\pgfqpoint{1.514281in}{0.777747in}}%
\pgfpathlineto{\pgfqpoint{1.526579in}{0.946050in}}%
\pgfpathlineto{\pgfqpoint{1.532815in}{0.986858in}}%
\pgfpathlineto{\pgfqpoint{1.535760in}{0.992278in}}%
\pgfpathlineto{\pgfqpoint{1.537492in}{0.989948in}}%
\pgfpathlineto{\pgfqpoint{1.541130in}{0.973981in}}%
\pgfpathlineto{\pgfqpoint{1.547712in}{0.923440in}}%
\pgfpathlineto{\pgfqpoint{1.563995in}{0.789840in}}%
\pgfpathlineto{\pgfqpoint{1.571270in}{0.764270in}}%
\pgfpathlineto{\pgfqpoint{1.574388in}{0.760807in}}%
\pgfpathlineto{\pgfqpoint{1.576987in}{0.761361in}}%
\pgfpathlineto{\pgfqpoint{1.579758in}{0.767290in}}%
\pgfpathlineto{\pgfqpoint{1.585128in}{0.790822in}}%
\pgfpathlineto{\pgfqpoint{1.606261in}{0.889313in}}%
\pgfpathlineto{\pgfqpoint{1.612323in}{0.903609in}}%
\pgfpathlineto{\pgfqpoint{1.617693in}{0.909040in}}%
\pgfpathlineto{\pgfqpoint{1.620985in}{0.907826in}}%
\pgfpathlineto{\pgfqpoint{1.624622in}{0.901867in}}%
\pgfpathlineto{\pgfqpoint{1.639692in}{0.864583in}}%
\pgfpathlineto{\pgfqpoint{1.649739in}{0.829885in}}%
\pgfpathlineto{\pgfqpoint{1.654589in}{0.822539in}}%
\pgfpathlineto{\pgfqpoint{1.660825in}{0.819127in}}%
\pgfpathlineto{\pgfqpoint{1.664463in}{0.818615in}}%
\pgfpathlineto{\pgfqpoint{1.675029in}{0.821029in}}%
\pgfpathlineto{\pgfqpoint{1.678667in}{0.823558in}}%
\pgfpathlineto{\pgfqpoint{1.685249in}{0.835700in}}%
\pgfpathlineto{\pgfqpoint{1.698587in}{0.862625in}}%
\pgfpathlineto{\pgfqpoint{1.702225in}{0.865817in}}%
\pgfpathlineto{\pgfqpoint{1.712964in}{0.874805in}}%
\pgfpathlineto{\pgfqpoint{1.722318in}{0.878467in}}%
\pgfpathlineto{\pgfqpoint{1.725436in}{0.877366in}}%
\pgfpathlineto{\pgfqpoint{1.730806in}{0.871841in}}%
\pgfpathlineto{\pgfqpoint{1.741373in}{0.856105in}}%
\pgfpathlineto{\pgfqpoint{1.756962in}{0.841607in}}%
\pgfpathlineto{\pgfqpoint{1.771859in}{0.841636in}}%
\pgfpathlineto{\pgfqpoint{1.774631in}{0.841951in}}%
\pgfpathlineto{\pgfqpoint{1.780001in}{0.843684in}}%
\pgfpathlineto{\pgfqpoint{1.785024in}{0.843941in}}%
\pgfpathlineto{\pgfqpoint{1.790394in}{0.846148in}}%
\pgfpathlineto{\pgfqpoint{1.796630in}{0.854777in}}%
\pgfpathlineto{\pgfqpoint{1.804598in}{0.862532in}}%
\pgfpathlineto{\pgfqpoint{1.812220in}{0.866160in}}%
\pgfpathlineto{\pgfqpoint{1.820188in}{0.865761in}}%
\pgfpathlineto{\pgfqpoint{1.831967in}{0.862766in}}%
\pgfpathlineto{\pgfqpoint{1.837510in}{0.863594in}}%
\pgfpathlineto{\pgfqpoint{1.844612in}{0.863750in}}%
\pgfpathlineto{\pgfqpoint{1.847903in}{0.861719in}}%
\pgfpathlineto{\pgfqpoint{1.855005in}{0.855083in}}%
\pgfpathlineto{\pgfqpoint{1.861414in}{0.851715in}}%
\pgfpathlineto{\pgfqpoint{1.867650in}{0.847674in}}%
\pgfpathlineto{\pgfqpoint{1.871115in}{0.845298in}}%
\pgfpathlineto{\pgfqpoint{1.887397in}{0.832442in}}%
\pgfpathlineto{\pgfqpoint{1.894153in}{0.829737in}}%
\pgfpathlineto{\pgfqpoint{1.897098in}{0.830854in}}%
\pgfpathlineto{\pgfqpoint{1.901601in}{0.835714in}}%
\pgfpathlineto{\pgfqpoint{1.905759in}{0.842051in}}%
\pgfpathlineto{\pgfqpoint{1.924813in}{0.872247in}}%
\pgfpathlineto{\pgfqpoint{1.935899in}{0.881005in}}%
\pgfpathlineto{\pgfqpoint{1.942308in}{0.879462in}}%
\pgfpathlineto{\pgfqpoint{1.945946in}{0.874715in}}%
\pgfpathlineto{\pgfqpoint{1.954953in}{0.862036in}}%
\pgfpathlineto{\pgfqpoint{1.967252in}{0.851375in}}%
\pgfpathlineto{\pgfqpoint{1.977299in}{0.848803in}}%
\pgfpathlineto{\pgfqpoint{1.981976in}{0.850813in}}%
\pgfpathlineto{\pgfqpoint{1.986479in}{0.855398in}}%
\pgfpathlineto{\pgfqpoint{1.991849in}{0.861037in}}%
\pgfpathlineto{\pgfqpoint{1.994794in}{0.864725in}}%
\pgfpathlineto{\pgfqpoint{2.003282in}{0.873092in}}%
\pgfpathlineto{\pgfqpoint{2.008825in}{0.875639in}}%
\pgfpathlineto{\pgfqpoint{2.011770in}{0.874636in}}%
\pgfpathlineto{\pgfqpoint{2.014714in}{0.872574in}}%
\pgfpathlineto{\pgfqpoint{2.020950in}{0.867345in}}%
\pgfpathlineto{\pgfqpoint{2.036020in}{0.851206in}}%
\pgfpathlineto{\pgfqpoint{2.045028in}{0.836843in}}%
\pgfpathlineto{\pgfqpoint{2.049705in}{0.836229in}}%
\pgfpathlineto{\pgfqpoint{2.054901in}{0.837990in}}%
\pgfpathlineto{\pgfqpoint{2.063909in}{0.843287in}}%
\pgfpathlineto{\pgfqpoint{2.067893in}{0.845541in}}%
\pgfpathlineto{\pgfqpoint{2.072916in}{0.853094in}}%
\pgfpathlineto{\pgfqpoint{2.091624in}{0.880231in}}%
\pgfpathlineto{\pgfqpoint{2.098380in}{0.881835in}}%
\pgfpathlineto{\pgfqpoint{2.102017in}{0.880676in}}%
\pgfpathlineto{\pgfqpoint{2.107041in}{0.876382in}}%
\pgfpathlineto{\pgfqpoint{2.120205in}{0.857515in}}%
\pgfpathlineto{\pgfqpoint{2.130599in}{0.840330in}}%
\pgfpathlineto{\pgfqpoint{2.139433in}{0.839352in}}%
\pgfpathlineto{\pgfqpoint{2.148614in}{0.840640in}}%
\pgfpathlineto{\pgfqpoint{2.162298in}{0.846942in}}%
\pgfpathlineto{\pgfqpoint{2.170786in}{0.855145in}}%
\pgfpathlineto{\pgfqpoint{2.186549in}{0.873928in}}%
\pgfpathlineto{\pgfqpoint{2.191053in}{0.873479in}}%
\pgfpathlineto{\pgfqpoint{2.195903in}{0.872001in}}%
\pgfpathlineto{\pgfqpoint{2.203351in}{0.871576in}}%
\pgfpathlineto{\pgfqpoint{2.211146in}{0.871963in}}%
\pgfpathlineto{\pgfqpoint{2.218595in}{0.867497in}}%
\pgfpathlineto{\pgfqpoint{2.223791in}{0.864344in}}%
\pgfpathlineto{\pgfqpoint{2.257396in}{0.831976in}}%
\pgfpathlineto{\pgfqpoint{2.260341in}{0.833889in}}%
\pgfpathlineto{\pgfqpoint{2.267616in}{0.842316in}}%
\pgfpathlineto{\pgfqpoint{2.286324in}{0.869766in}}%
\pgfpathlineto{\pgfqpoint{2.292560in}{0.878387in}}%
\pgfpathlineto{\pgfqpoint{2.298622in}{0.883095in}}%
\pgfpathlineto{\pgfqpoint{2.300355in}{0.882835in}}%
\pgfpathlineto{\pgfqpoint{2.301221in}{0.858543in}}%
\pgfpathlineto{\pgfqpoint{2.308669in}{0.475844in}}%
\pgfpathlineto{\pgfqpoint{2.308842in}{0.475963in}}%
\pgfpathlineto{\pgfqpoint{2.309708in}{0.480536in}}%
\pgfpathlineto{\pgfqpoint{2.311267in}{0.509811in}}%
\pgfpathlineto{\pgfqpoint{2.313519in}{0.620991in}}%
\pgfpathlineto{\pgfqpoint{2.323220in}{1.231708in}}%
\pgfpathlineto{\pgfqpoint{2.326857in}{1.245725in}}%
\pgfpathlineto{\pgfqpoint{2.329109in}{1.246837in}}%
\pgfpathlineto{\pgfqpoint{2.330841in}{1.244766in}}%
\pgfpathlineto{\pgfqpoint{2.333440in}{1.234605in}}%
\pgfpathlineto{\pgfqpoint{2.341754in}{1.168619in}}%
\pgfpathlineto{\pgfqpoint{2.345392in}{1.097908in}}%
\pgfpathlineto{\pgfqpoint{2.351281in}{0.912468in}}%
\pgfpathlineto{\pgfqpoint{2.359942in}{0.663327in}}%
\pgfpathlineto{\pgfqpoint{2.363407in}{0.636089in}}%
\pgfpathlineto{\pgfqpoint{2.364793in}{0.635420in}}%
\pgfpathlineto{\pgfqpoint{2.365139in}{0.635835in}}%
\pgfpathlineto{\pgfqpoint{2.367910in}{0.644288in}}%
\pgfpathlineto{\pgfqpoint{2.372241in}{0.669495in}}%
\pgfpathlineto{\pgfqpoint{2.376572in}{0.713931in}}%
\pgfpathlineto{\pgfqpoint{2.381941in}{0.808935in}}%
\pgfpathlineto{\pgfqpoint{2.390602in}{0.962377in}}%
\pgfpathlineto{\pgfqpoint{2.394933in}{0.993050in}}%
\pgfpathlineto{\pgfqpoint{2.398224in}{0.999369in}}%
\pgfpathlineto{\pgfqpoint{2.399783in}{0.998700in}}%
\pgfpathlineto{\pgfqpoint{2.401862in}{0.994418in}}%
\pgfpathlineto{\pgfqpoint{2.404980in}{0.981993in}}%
\pgfpathlineto{\pgfqpoint{2.410869in}{0.936547in}}%
\pgfpathlineto{\pgfqpoint{2.426459in}{0.801645in}}%
\pgfpathlineto{\pgfqpoint{2.432522in}{0.779628in}}%
\pgfpathlineto{\pgfqpoint{2.436333in}{0.774104in}}%
\pgfpathlineto{\pgfqpoint{2.438931in}{0.774345in}}%
\pgfpathlineto{\pgfqpoint{2.442049in}{0.778275in}}%
\pgfpathlineto{\pgfqpoint{2.445513in}{0.788100in}}%
\pgfpathlineto{\pgfqpoint{2.450537in}{0.811720in}}%
\pgfpathlineto{\pgfqpoint{2.459544in}{0.854893in}}%
\pgfpathlineto{\pgfqpoint{2.478079in}{0.915312in}}%
\pgfpathlineto{\pgfqpoint{2.482756in}{0.918392in}}%
\pgfpathlineto{\pgfqpoint{2.485527in}{0.917180in}}%
\pgfpathlineto{\pgfqpoint{2.489858in}{0.910721in}}%
\pgfpathlineto{\pgfqpoint{2.494535in}{0.897236in}}%
\pgfpathlineto{\pgfqpoint{2.502503in}{0.863268in}}%
\pgfpathlineto{\pgfqpoint{2.515148in}{0.810592in}}%
\pgfpathlineto{\pgfqpoint{2.521384in}{0.797802in}}%
\pgfpathlineto{\pgfqpoint{2.524328in}{0.796371in}}%
\pgfpathlineto{\pgfqpoint{2.526407in}{0.797860in}}%
\pgfpathlineto{\pgfqpoint{2.531950in}{0.808682in}}%
\pgfpathlineto{\pgfqpoint{2.539052in}{0.833203in}}%
\pgfpathlineto{\pgfqpoint{2.547713in}{0.869426in}}%
\pgfpathlineto{\pgfqpoint{2.554122in}{0.890207in}}%
\pgfpathlineto{\pgfqpoint{2.561571in}{0.904556in}}%
\pgfpathlineto{\pgfqpoint{2.564862in}{0.906689in}}%
\pgfpathlineto{\pgfqpoint{2.566941in}{0.905777in}}%
\pgfpathlineto{\pgfqpoint{2.569885in}{0.901344in}}%
\pgfpathlineto{\pgfqpoint{2.576987in}{0.889333in}}%
\pgfpathlineto{\pgfqpoint{2.588593in}{0.871529in}}%
\pgfpathlineto{\pgfqpoint{2.600719in}{0.843240in}}%
\pgfpathlineto{\pgfqpoint{2.606435in}{0.839519in}}%
\pgfpathlineto{\pgfqpoint{2.618560in}{0.838344in}}%
\pgfpathlineto{\pgfqpoint{2.623064in}{0.839575in}}%
\pgfpathlineto{\pgfqpoint{2.625143in}{0.840278in}}%
\pgfpathlineto{\pgfqpoint{2.635189in}{0.850062in}}%
\pgfpathlineto{\pgfqpoint{2.640732in}{0.857255in}}%
\pgfpathlineto{\pgfqpoint{2.652165in}{0.869523in}}%
\pgfpathlineto{\pgfqpoint{2.662385in}{0.883939in}}%
\pgfpathlineto{\pgfqpoint{2.664637in}{0.883715in}}%
\pgfpathlineto{\pgfqpoint{2.668275in}{0.882684in}}%
\pgfpathlineto{\pgfqpoint{2.672085in}{0.880450in}}%
\pgfpathlineto{\pgfqpoint{2.681612in}{0.871634in}}%
\pgfpathlineto{\pgfqpoint{2.688541in}{0.870918in}}%
\pgfpathlineto{\pgfqpoint{2.691832in}{0.870480in}}%
\pgfpathlineto{\pgfqpoint{2.701706in}{0.865346in}}%
\pgfpathlineto{\pgfqpoint{2.709328in}{0.853992in}}%
\pgfpathlineto{\pgfqpoint{2.716430in}{0.851098in}}%
\pgfpathlineto{\pgfqpoint{2.720587in}{0.853184in}}%
\pgfpathlineto{\pgfqpoint{2.727862in}{0.856469in}}%
\pgfpathlineto{\pgfqpoint{2.732366in}{0.855686in}}%
\pgfpathlineto{\pgfqpoint{2.735657in}{0.857242in}}%
\pgfpathlineto{\pgfqpoint{2.759042in}{0.858380in}}%
\pgfpathlineto{\pgfqpoint{2.765278in}{0.861971in}}%
\pgfpathlineto{\pgfqpoint{2.789529in}{0.878621in}}%
\pgfpathlineto{\pgfqpoint{2.797843in}{0.874327in}}%
\pgfpathlineto{\pgfqpoint{2.801654in}{0.872702in}}%
\pgfpathlineto{\pgfqpoint{2.816898in}{0.874008in}}%
\pgfpathlineto{\pgfqpoint{2.820708in}{0.873502in}}%
\pgfpathlineto{\pgfqpoint{2.824173in}{0.872165in}}%
\pgfpathlineto{\pgfqpoint{2.828330in}{0.869029in}}%
\pgfpathlineto{\pgfqpoint{2.833700in}{0.863557in}}%
\pgfpathlineto{\pgfqpoint{2.836818in}{0.859832in}}%
\pgfpathlineto{\pgfqpoint{2.840802in}{0.855929in}}%
\pgfpathlineto{\pgfqpoint{2.846345in}{0.855210in}}%
\pgfpathlineto{\pgfqpoint{2.853793in}{0.856364in}}%
\pgfpathlineto{\pgfqpoint{2.866958in}{0.868144in}}%
\pgfpathlineto{\pgfqpoint{2.877351in}{0.880487in}}%
\pgfpathlineto{\pgfqpoint{2.880989in}{0.882225in}}%
\pgfpathlineto{\pgfqpoint{2.884800in}{0.882518in}}%
\pgfpathlineto{\pgfqpoint{2.897618in}{0.875144in}}%
\pgfpathlineto{\pgfqpoint{2.903508in}{0.868312in}}%
\pgfpathlineto{\pgfqpoint{2.909744in}{0.860265in}}%
\pgfpathlineto{\pgfqpoint{2.919617in}{0.852482in}}%
\pgfpathlineto{\pgfqpoint{2.942829in}{0.852330in}}%
\pgfpathlineto{\pgfqpoint{2.949065in}{0.854164in}}%
\pgfpathlineto{\pgfqpoint{2.953049in}{0.858836in}}%
\pgfpathlineto{\pgfqpoint{2.977646in}{0.889261in}}%
\pgfpathlineto{\pgfqpoint{2.984921in}{0.883975in}}%
\pgfpathlineto{\pgfqpoint{2.988212in}{0.881678in}}%
\pgfpathlineto{\pgfqpoint{2.991504in}{0.878574in}}%
\pgfpathlineto{\pgfqpoint{2.998952in}{0.872612in}}%
\pgfpathlineto{\pgfqpoint{3.011770in}{0.866790in}}%
\pgfpathlineto{\pgfqpoint{3.022510in}{0.859995in}}%
\pgfpathlineto{\pgfqpoint{3.025801in}{0.858875in}}%
\pgfpathlineto{\pgfqpoint{3.041564in}{0.850938in}}%
\pgfpathlineto{\pgfqpoint{3.045895in}{0.853621in}}%
\pgfpathlineto{\pgfqpoint{3.055249in}{0.865818in}}%
\pgfpathlineto{\pgfqpoint{3.057501in}{0.867978in}}%
\pgfpathlineto{\pgfqpoint{3.075515in}{0.889522in}}%
\pgfpathlineto{\pgfqpoint{3.080192in}{0.888482in}}%
\pgfpathlineto{\pgfqpoint{3.096995in}{0.876875in}}%
\pgfpathlineto{\pgfqpoint{3.106002in}{0.867626in}}%
\pgfpathlineto{\pgfqpoint{3.113278in}{0.861465in}}%
\pgfpathlineto{\pgfqpoint{3.118474in}{0.857944in}}%
\pgfpathlineto{\pgfqpoint{3.126789in}{0.856574in}}%
\pgfpathlineto{\pgfqpoint{3.142725in}{0.850174in}}%
\pgfpathlineto{\pgfqpoint{3.145843in}{0.849355in}}%
\pgfpathlineto{\pgfqpoint{3.154331in}{0.851936in}}%
\pgfpathlineto{\pgfqpoint{3.162299in}{0.862056in}}%
\pgfpathlineto{\pgfqpoint{3.166456in}{0.867383in}}%
\pgfpathlineto{\pgfqpoint{3.166456in}{0.867383in}}%
\pgfusepath{stroke}%
\end{pgfscope}%
\begin{pgfscope}%
\pgfsetrectcap%
\pgfsetmiterjoin%
\pgfsetlinewidth{0.803000pt}%
\definecolor{currentstroke}{rgb}{0.000000,0.000000,0.000000}%
\pgfsetstrokecolor{currentstroke}%
\pgfsetdash{}{0pt}%
\pgfpathmoveto{\pgfqpoint{0.568671in}{0.451277in}}%
\pgfpathlineto{\pgfqpoint{0.568671in}{1.250151in}}%
\pgfusepath{stroke}%
\end{pgfscope}%
\begin{pgfscope}%
\pgfsetrectcap%
\pgfsetmiterjoin%
\pgfsetlinewidth{0.803000pt}%
\definecolor{currentstroke}{rgb}{0.000000,0.000000,0.000000}%
\pgfsetstrokecolor{currentstroke}%
\pgfsetdash{}{0pt}%
\pgfpathmoveto{\pgfqpoint{3.166976in}{0.451277in}}%
\pgfpathlineto{\pgfqpoint{3.166976in}{1.250151in}}%
\pgfusepath{stroke}%
\end{pgfscope}%
\begin{pgfscope}%
\pgfsetrectcap%
\pgfsetmiterjoin%
\pgfsetlinewidth{0.803000pt}%
\definecolor{currentstroke}{rgb}{0.000000,0.000000,0.000000}%
\pgfsetstrokecolor{currentstroke}%
\pgfsetdash{}{0pt}%
\pgfpathmoveto{\pgfqpoint{0.568671in}{0.451277in}}%
\pgfpathlineto{\pgfqpoint{3.166976in}{0.451277in}}%
\pgfusepath{stroke}%
\end{pgfscope}%
\begin{pgfscope}%
\pgfsetrectcap%
\pgfsetmiterjoin%
\pgfsetlinewidth{0.803000pt}%
\definecolor{currentstroke}{rgb}{0.000000,0.000000,0.000000}%
\pgfsetstrokecolor{currentstroke}%
\pgfsetdash{}{0pt}%
\pgfpathmoveto{\pgfqpoint{0.568671in}{1.250151in}}%
\pgfpathlineto{\pgfqpoint{3.166976in}{1.250151in}}%
\pgfusepath{stroke}%
\end{pgfscope}%
\end{pgfpicture}%
\makeatother%
\endgroup%
\\\vspace*{-1.0cm}
    %% Creator: Matplotlib, PGF backend
%%
%% To include the figure in your LaTeX document, write
%%   \input{<filename>.pgf}
%%
%% Make sure the required packages are loaded in your preamble
%%   \usepackage{pgf}
%%
%% Also ensure that all the required font packages are loaded; for instance,
%% the lmodern package is sometimes necessary when using math font.
%%   \usepackage{lmodern}
%%
%% Figures using additional raster images can only be included by \input if
%% they are in the same directory as the main LaTeX file. For loading figures
%% from other directories you can use the `import` package
%%   \usepackage{import}
%%
%% and then include the figures with
%%   \import{<path to file>}{<filename>.pgf}
%%
%% Matplotlib used the following preamble
%%   \usepackage{fontspec}
%%
\begingroup%
\makeatletter%
\begin{pgfpicture}%
\pgfpathrectangle{\pgfpointorigin}{\pgfqpoint{3.390065in}{1.695033in}}%
\pgfusepath{use as bounding box, clip}%
\begin{pgfscope}%
\pgfsetbuttcap%
\pgfsetmiterjoin%
\definecolor{currentfill}{rgb}{1.000000,1.000000,1.000000}%
\pgfsetfillcolor{currentfill}%
\pgfsetlinewidth{0.000000pt}%
\definecolor{currentstroke}{rgb}{1.000000,1.000000,1.000000}%
\pgfsetstrokecolor{currentstroke}%
\pgfsetstrokeopacity{0.000000}%
\pgfsetdash{}{0pt}%
\pgfpathmoveto{\pgfqpoint{0.000000in}{0.000000in}}%
\pgfpathlineto{\pgfqpoint{3.390065in}{0.000000in}}%
\pgfpathlineto{\pgfqpoint{3.390065in}{1.695033in}}%
\pgfpathlineto{\pgfqpoint{0.000000in}{1.695033in}}%
\pgfpathlineto{\pgfqpoint{0.000000in}{0.000000in}}%
\pgfpathclose%
\pgfusepath{fill}%
\end{pgfscope}%
\begin{pgfscope}%
\pgfsetbuttcap%
\pgfsetmiterjoin%
\definecolor{currentfill}{rgb}{1.000000,1.000000,1.000000}%
\pgfsetfillcolor{currentfill}%
\pgfsetlinewidth{0.000000pt}%
\definecolor{currentstroke}{rgb}{0.000000,0.000000,0.000000}%
\pgfsetstrokecolor{currentstroke}%
\pgfsetstrokeopacity{0.000000}%
\pgfsetdash{}{0pt}%
\pgfpathmoveto{\pgfqpoint{0.504436in}{0.451277in}}%
\pgfpathlineto{\pgfqpoint{3.166976in}{0.451277in}}%
\pgfpathlineto{\pgfqpoint{3.166976in}{1.589158in}}%
\pgfpathlineto{\pgfqpoint{0.504436in}{1.589158in}}%
\pgfpathlineto{\pgfqpoint{0.504436in}{0.451277in}}%
\pgfpathclose%
\pgfusepath{fill}%
\end{pgfscope}%
\begin{pgfscope}%
\pgfpathrectangle{\pgfqpoint{0.504436in}{0.451277in}}{\pgfqpoint{2.662540in}{1.137880in}}%
\pgfusepath{clip}%
\pgfsetrectcap%
\pgfsetroundjoin%
\pgfsetlinewidth{0.803000pt}%
\definecolor{currentstroke}{rgb}{0.690196,0.690196,0.690196}%
\pgfsetstrokecolor{currentstroke}%
\pgfsetdash{}{0pt}%
\pgfpathmoveto{\pgfqpoint{0.504436in}{0.451277in}}%
\pgfpathlineto{\pgfqpoint{0.504436in}{1.589158in}}%
\pgfusepath{stroke}%
\end{pgfscope}%
\begin{pgfscope}%
\pgfsetbuttcap%
\pgfsetroundjoin%
\definecolor{currentfill}{rgb}{0.000000,0.000000,0.000000}%
\pgfsetfillcolor{currentfill}%
\pgfsetlinewidth{0.803000pt}%
\definecolor{currentstroke}{rgb}{0.000000,0.000000,0.000000}%
\pgfsetstrokecolor{currentstroke}%
\pgfsetdash{}{0pt}%
\pgfsys@defobject{currentmarker}{\pgfqpoint{0.000000in}{-0.048611in}}{\pgfqpoint{0.000000in}{0.000000in}}{%
\pgfpathmoveto{\pgfqpoint{0.000000in}{0.000000in}}%
\pgfpathlineto{\pgfqpoint{0.000000in}{-0.048611in}}%
\pgfusepath{stroke,fill}%
}%
\begin{pgfscope}%
\pgfsys@transformshift{0.504436in}{0.451277in}%
\pgfsys@useobject{currentmarker}{}%
\end{pgfscope}%
\end{pgfscope}%
\begin{pgfscope}%
\definecolor{textcolor}{rgb}{0.000000,0.000000,0.000000}%
\pgfsetstrokecolor{textcolor}%
\pgfsetfillcolor{textcolor}%
\pgftext[x=0.504436in,y=0.354055in,,top]{\color{textcolor}\rmfamily\fontsize{9.000000}{10.800000}\selectfont \(\displaystyle {0}\)}%
\end{pgfscope}%
\begin{pgfscope}%
\pgfpathrectangle{\pgfqpoint{0.504436in}{0.451277in}}{\pgfqpoint{2.662540in}{1.137880in}}%
\pgfusepath{clip}%
\pgfsetrectcap%
\pgfsetroundjoin%
\pgfsetlinewidth{0.803000pt}%
\definecolor{currentstroke}{rgb}{0.690196,0.690196,0.690196}%
\pgfsetstrokecolor{currentstroke}%
\pgfsetdash{}{0pt}%
\pgfpathmoveto{\pgfqpoint{0.948192in}{0.451277in}}%
\pgfpathlineto{\pgfqpoint{0.948192in}{1.589158in}}%
\pgfusepath{stroke}%
\end{pgfscope}%
\begin{pgfscope}%
\pgfsetbuttcap%
\pgfsetroundjoin%
\definecolor{currentfill}{rgb}{0.000000,0.000000,0.000000}%
\pgfsetfillcolor{currentfill}%
\pgfsetlinewidth{0.803000pt}%
\definecolor{currentstroke}{rgb}{0.000000,0.000000,0.000000}%
\pgfsetstrokecolor{currentstroke}%
\pgfsetdash{}{0pt}%
\pgfsys@defobject{currentmarker}{\pgfqpoint{0.000000in}{-0.048611in}}{\pgfqpoint{0.000000in}{0.000000in}}{%
\pgfpathmoveto{\pgfqpoint{0.000000in}{0.000000in}}%
\pgfpathlineto{\pgfqpoint{0.000000in}{-0.048611in}}%
\pgfusepath{stroke,fill}%
}%
\begin{pgfscope}%
\pgfsys@transformshift{0.948192in}{0.451277in}%
\pgfsys@useobject{currentmarker}{}%
\end{pgfscope}%
\end{pgfscope}%
\begin{pgfscope}%
\definecolor{textcolor}{rgb}{0.000000,0.000000,0.000000}%
\pgfsetstrokecolor{textcolor}%
\pgfsetfillcolor{textcolor}%
\pgftext[x=0.948192in,y=0.354055in,,top]{\color{textcolor}\rmfamily\fontsize{9.000000}{10.800000}\selectfont \(\displaystyle {2500}\)}%
\end{pgfscope}%
\begin{pgfscope}%
\pgfpathrectangle{\pgfqpoint{0.504436in}{0.451277in}}{\pgfqpoint{2.662540in}{1.137880in}}%
\pgfusepath{clip}%
\pgfsetrectcap%
\pgfsetroundjoin%
\pgfsetlinewidth{0.803000pt}%
\definecolor{currentstroke}{rgb}{0.690196,0.690196,0.690196}%
\pgfsetstrokecolor{currentstroke}%
\pgfsetdash{}{0pt}%
\pgfpathmoveto{\pgfqpoint{1.391949in}{0.451277in}}%
\pgfpathlineto{\pgfqpoint{1.391949in}{1.589158in}}%
\pgfusepath{stroke}%
\end{pgfscope}%
\begin{pgfscope}%
\pgfsetbuttcap%
\pgfsetroundjoin%
\definecolor{currentfill}{rgb}{0.000000,0.000000,0.000000}%
\pgfsetfillcolor{currentfill}%
\pgfsetlinewidth{0.803000pt}%
\definecolor{currentstroke}{rgb}{0.000000,0.000000,0.000000}%
\pgfsetstrokecolor{currentstroke}%
\pgfsetdash{}{0pt}%
\pgfsys@defobject{currentmarker}{\pgfqpoint{0.000000in}{-0.048611in}}{\pgfqpoint{0.000000in}{0.000000in}}{%
\pgfpathmoveto{\pgfqpoint{0.000000in}{0.000000in}}%
\pgfpathlineto{\pgfqpoint{0.000000in}{-0.048611in}}%
\pgfusepath{stroke,fill}%
}%
\begin{pgfscope}%
\pgfsys@transformshift{1.391949in}{0.451277in}%
\pgfsys@useobject{currentmarker}{}%
\end{pgfscope}%
\end{pgfscope}%
\begin{pgfscope}%
\definecolor{textcolor}{rgb}{0.000000,0.000000,0.000000}%
\pgfsetstrokecolor{textcolor}%
\pgfsetfillcolor{textcolor}%
\pgftext[x=1.391949in,y=0.354055in,,top]{\color{textcolor}\rmfamily\fontsize{9.000000}{10.800000}\selectfont \(\displaystyle {5000}\)}%
\end{pgfscope}%
\begin{pgfscope}%
\pgfpathrectangle{\pgfqpoint{0.504436in}{0.451277in}}{\pgfqpoint{2.662540in}{1.137880in}}%
\pgfusepath{clip}%
\pgfsetrectcap%
\pgfsetroundjoin%
\pgfsetlinewidth{0.803000pt}%
\definecolor{currentstroke}{rgb}{0.690196,0.690196,0.690196}%
\pgfsetstrokecolor{currentstroke}%
\pgfsetdash{}{0pt}%
\pgfpathmoveto{\pgfqpoint{1.835706in}{0.451277in}}%
\pgfpathlineto{\pgfqpoint{1.835706in}{1.589158in}}%
\pgfusepath{stroke}%
\end{pgfscope}%
\begin{pgfscope}%
\pgfsetbuttcap%
\pgfsetroundjoin%
\definecolor{currentfill}{rgb}{0.000000,0.000000,0.000000}%
\pgfsetfillcolor{currentfill}%
\pgfsetlinewidth{0.803000pt}%
\definecolor{currentstroke}{rgb}{0.000000,0.000000,0.000000}%
\pgfsetstrokecolor{currentstroke}%
\pgfsetdash{}{0pt}%
\pgfsys@defobject{currentmarker}{\pgfqpoint{0.000000in}{-0.048611in}}{\pgfqpoint{0.000000in}{0.000000in}}{%
\pgfpathmoveto{\pgfqpoint{0.000000in}{0.000000in}}%
\pgfpathlineto{\pgfqpoint{0.000000in}{-0.048611in}}%
\pgfusepath{stroke,fill}%
}%
\begin{pgfscope}%
\pgfsys@transformshift{1.835706in}{0.451277in}%
\pgfsys@useobject{currentmarker}{}%
\end{pgfscope}%
\end{pgfscope}%
\begin{pgfscope}%
\definecolor{textcolor}{rgb}{0.000000,0.000000,0.000000}%
\pgfsetstrokecolor{textcolor}%
\pgfsetfillcolor{textcolor}%
\pgftext[x=1.835706in,y=0.354055in,,top]{\color{textcolor}\rmfamily\fontsize{9.000000}{10.800000}\selectfont \(\displaystyle {7500}\)}%
\end{pgfscope}%
\begin{pgfscope}%
\pgfpathrectangle{\pgfqpoint{0.504436in}{0.451277in}}{\pgfqpoint{2.662540in}{1.137880in}}%
\pgfusepath{clip}%
\pgfsetrectcap%
\pgfsetroundjoin%
\pgfsetlinewidth{0.803000pt}%
\definecolor{currentstroke}{rgb}{0.690196,0.690196,0.690196}%
\pgfsetstrokecolor{currentstroke}%
\pgfsetdash{}{0pt}%
\pgfpathmoveto{\pgfqpoint{2.279462in}{0.451277in}}%
\pgfpathlineto{\pgfqpoint{2.279462in}{1.589158in}}%
\pgfusepath{stroke}%
\end{pgfscope}%
\begin{pgfscope}%
\pgfsetbuttcap%
\pgfsetroundjoin%
\definecolor{currentfill}{rgb}{0.000000,0.000000,0.000000}%
\pgfsetfillcolor{currentfill}%
\pgfsetlinewidth{0.803000pt}%
\definecolor{currentstroke}{rgb}{0.000000,0.000000,0.000000}%
\pgfsetstrokecolor{currentstroke}%
\pgfsetdash{}{0pt}%
\pgfsys@defobject{currentmarker}{\pgfqpoint{0.000000in}{-0.048611in}}{\pgfqpoint{0.000000in}{0.000000in}}{%
\pgfpathmoveto{\pgfqpoint{0.000000in}{0.000000in}}%
\pgfpathlineto{\pgfqpoint{0.000000in}{-0.048611in}}%
\pgfusepath{stroke,fill}%
}%
\begin{pgfscope}%
\pgfsys@transformshift{2.279462in}{0.451277in}%
\pgfsys@useobject{currentmarker}{}%
\end{pgfscope}%
\end{pgfscope}%
\begin{pgfscope}%
\definecolor{textcolor}{rgb}{0.000000,0.000000,0.000000}%
\pgfsetstrokecolor{textcolor}%
\pgfsetfillcolor{textcolor}%
\pgftext[x=2.279462in,y=0.354055in,,top]{\color{textcolor}\rmfamily\fontsize{9.000000}{10.800000}\selectfont \(\displaystyle {10000}\)}%
\end{pgfscope}%
\begin{pgfscope}%
\pgfpathrectangle{\pgfqpoint{0.504436in}{0.451277in}}{\pgfqpoint{2.662540in}{1.137880in}}%
\pgfusepath{clip}%
\pgfsetrectcap%
\pgfsetroundjoin%
\pgfsetlinewidth{0.803000pt}%
\definecolor{currentstroke}{rgb}{0.690196,0.690196,0.690196}%
\pgfsetstrokecolor{currentstroke}%
\pgfsetdash{}{0pt}%
\pgfpathmoveto{\pgfqpoint{2.723219in}{0.451277in}}%
\pgfpathlineto{\pgfqpoint{2.723219in}{1.589158in}}%
\pgfusepath{stroke}%
\end{pgfscope}%
\begin{pgfscope}%
\pgfsetbuttcap%
\pgfsetroundjoin%
\definecolor{currentfill}{rgb}{0.000000,0.000000,0.000000}%
\pgfsetfillcolor{currentfill}%
\pgfsetlinewidth{0.803000pt}%
\definecolor{currentstroke}{rgb}{0.000000,0.000000,0.000000}%
\pgfsetstrokecolor{currentstroke}%
\pgfsetdash{}{0pt}%
\pgfsys@defobject{currentmarker}{\pgfqpoint{0.000000in}{-0.048611in}}{\pgfqpoint{0.000000in}{0.000000in}}{%
\pgfpathmoveto{\pgfqpoint{0.000000in}{0.000000in}}%
\pgfpathlineto{\pgfqpoint{0.000000in}{-0.048611in}}%
\pgfusepath{stroke,fill}%
}%
\begin{pgfscope}%
\pgfsys@transformshift{2.723219in}{0.451277in}%
\pgfsys@useobject{currentmarker}{}%
\end{pgfscope}%
\end{pgfscope}%
\begin{pgfscope}%
\definecolor{textcolor}{rgb}{0.000000,0.000000,0.000000}%
\pgfsetstrokecolor{textcolor}%
\pgfsetfillcolor{textcolor}%
\pgftext[x=2.723219in,y=0.354055in,,top]{\color{textcolor}\rmfamily\fontsize{9.000000}{10.800000}\selectfont \(\displaystyle {12500}\)}%
\end{pgfscope}%
\begin{pgfscope}%
\pgfpathrectangle{\pgfqpoint{0.504436in}{0.451277in}}{\pgfqpoint{2.662540in}{1.137880in}}%
\pgfusepath{clip}%
\pgfsetrectcap%
\pgfsetroundjoin%
\pgfsetlinewidth{0.803000pt}%
\definecolor{currentstroke}{rgb}{0.690196,0.690196,0.690196}%
\pgfsetstrokecolor{currentstroke}%
\pgfsetdash{}{0pt}%
\pgfpathmoveto{\pgfqpoint{3.166976in}{0.451277in}}%
\pgfpathlineto{\pgfqpoint{3.166976in}{1.589158in}}%
\pgfusepath{stroke}%
\end{pgfscope}%
\begin{pgfscope}%
\pgfsetbuttcap%
\pgfsetroundjoin%
\definecolor{currentfill}{rgb}{0.000000,0.000000,0.000000}%
\pgfsetfillcolor{currentfill}%
\pgfsetlinewidth{0.803000pt}%
\definecolor{currentstroke}{rgb}{0.000000,0.000000,0.000000}%
\pgfsetstrokecolor{currentstroke}%
\pgfsetdash{}{0pt}%
\pgfsys@defobject{currentmarker}{\pgfqpoint{0.000000in}{-0.048611in}}{\pgfqpoint{0.000000in}{0.000000in}}{%
\pgfpathmoveto{\pgfqpoint{0.000000in}{0.000000in}}%
\pgfpathlineto{\pgfqpoint{0.000000in}{-0.048611in}}%
\pgfusepath{stroke,fill}%
}%
\begin{pgfscope}%
\pgfsys@transformshift{3.166976in}{0.451277in}%
\pgfsys@useobject{currentmarker}{}%
\end{pgfscope}%
\end{pgfscope}%
\begin{pgfscope}%
\definecolor{textcolor}{rgb}{0.000000,0.000000,0.000000}%
\pgfsetstrokecolor{textcolor}%
\pgfsetfillcolor{textcolor}%
\pgftext[x=3.166976in,y=0.354055in,,top]{\color{textcolor}\rmfamily\fontsize{9.000000}{10.800000}\selectfont \(\displaystyle {15000}\)}%
\end{pgfscope}%
\begin{pgfscope}%
\definecolor{textcolor}{rgb}{0.000000,0.000000,0.000000}%
\pgfsetstrokecolor{textcolor}%
\pgfsetfillcolor{textcolor}%
\pgftext[x=1.835706in,y=0.187500in,,top]{\color{textcolor}\rmfamily\fontsize{9.000000}{10.800000}\selectfont Time [frames]}%
\end{pgfscope}%
\begin{pgfscope}%
\pgfpathrectangle{\pgfqpoint{0.504436in}{0.451277in}}{\pgfqpoint{2.662540in}{1.137880in}}%
\pgfusepath{clip}%
\pgfsetrectcap%
\pgfsetroundjoin%
\pgfsetlinewidth{0.803000pt}%
\definecolor{currentstroke}{rgb}{0.690196,0.690196,0.690196}%
\pgfsetstrokecolor{currentstroke}%
\pgfsetdash{}{0pt}%
\pgfpathmoveto{\pgfqpoint{0.504436in}{0.451277in}}%
\pgfpathlineto{\pgfqpoint{3.166976in}{0.451277in}}%
\pgfusepath{stroke}%
\end{pgfscope}%
\begin{pgfscope}%
\pgfsetbuttcap%
\pgfsetroundjoin%
\definecolor{currentfill}{rgb}{0.000000,0.000000,0.000000}%
\pgfsetfillcolor{currentfill}%
\pgfsetlinewidth{0.803000pt}%
\definecolor{currentstroke}{rgb}{0.000000,0.000000,0.000000}%
\pgfsetstrokecolor{currentstroke}%
\pgfsetdash{}{0pt}%
\pgfsys@defobject{currentmarker}{\pgfqpoint{-0.048611in}{0.000000in}}{\pgfqpoint{-0.000000in}{0.000000in}}{%
\pgfpathmoveto{\pgfqpoint{-0.000000in}{0.000000in}}%
\pgfpathlineto{\pgfqpoint{-0.048611in}{0.000000in}}%
\pgfusepath{stroke,fill}%
}%
\begin{pgfscope}%
\pgfsys@transformshift{0.504436in}{0.451277in}%
\pgfsys@useobject{currentmarker}{}%
\end{pgfscope}%
\end{pgfscope}%
\begin{pgfscope}%
\definecolor{textcolor}{rgb}{0.000000,0.000000,0.000000}%
\pgfsetstrokecolor{textcolor}%
\pgfsetfillcolor{textcolor}%
\pgftext[x=0.178820in, y=0.407902in, left, base]{\color{textcolor}\rmfamily\fontsize{9.000000}{10.800000}\selectfont \(\displaystyle {0.00}\)}%
\end{pgfscope}%
\begin{pgfscope}%
\pgfpathrectangle{\pgfqpoint{0.504436in}{0.451277in}}{\pgfqpoint{2.662540in}{1.137880in}}%
\pgfusepath{clip}%
\pgfsetrectcap%
\pgfsetroundjoin%
\pgfsetlinewidth{0.803000pt}%
\definecolor{currentstroke}{rgb}{0.690196,0.690196,0.690196}%
\pgfsetstrokecolor{currentstroke}%
\pgfsetdash{}{0pt}%
\pgfpathmoveto{\pgfqpoint{0.504436in}{0.735747in}}%
\pgfpathlineto{\pgfqpoint{3.166976in}{0.735747in}}%
\pgfusepath{stroke}%
\end{pgfscope}%
\begin{pgfscope}%
\pgfsetbuttcap%
\pgfsetroundjoin%
\definecolor{currentfill}{rgb}{0.000000,0.000000,0.000000}%
\pgfsetfillcolor{currentfill}%
\pgfsetlinewidth{0.803000pt}%
\definecolor{currentstroke}{rgb}{0.000000,0.000000,0.000000}%
\pgfsetstrokecolor{currentstroke}%
\pgfsetdash{}{0pt}%
\pgfsys@defobject{currentmarker}{\pgfqpoint{-0.048611in}{0.000000in}}{\pgfqpoint{-0.000000in}{0.000000in}}{%
\pgfpathmoveto{\pgfqpoint{-0.000000in}{0.000000in}}%
\pgfpathlineto{\pgfqpoint{-0.048611in}{0.000000in}}%
\pgfusepath{stroke,fill}%
}%
\begin{pgfscope}%
\pgfsys@transformshift{0.504436in}{0.735747in}%
\pgfsys@useobject{currentmarker}{}%
\end{pgfscope}%
\end{pgfscope}%
\begin{pgfscope}%
\definecolor{textcolor}{rgb}{0.000000,0.000000,0.000000}%
\pgfsetstrokecolor{textcolor}%
\pgfsetfillcolor{textcolor}%
\pgftext[x=0.178820in, y=0.692372in, left, base]{\color{textcolor}\rmfamily\fontsize{9.000000}{10.800000}\selectfont \(\displaystyle {0.25}\)}%
\end{pgfscope}%
\begin{pgfscope}%
\pgfpathrectangle{\pgfqpoint{0.504436in}{0.451277in}}{\pgfqpoint{2.662540in}{1.137880in}}%
\pgfusepath{clip}%
\pgfsetrectcap%
\pgfsetroundjoin%
\pgfsetlinewidth{0.803000pt}%
\definecolor{currentstroke}{rgb}{0.690196,0.690196,0.690196}%
\pgfsetstrokecolor{currentstroke}%
\pgfsetdash{}{0pt}%
\pgfpathmoveto{\pgfqpoint{0.504436in}{1.020217in}}%
\pgfpathlineto{\pgfqpoint{3.166976in}{1.020217in}}%
\pgfusepath{stroke}%
\end{pgfscope}%
\begin{pgfscope}%
\pgfsetbuttcap%
\pgfsetroundjoin%
\definecolor{currentfill}{rgb}{0.000000,0.000000,0.000000}%
\pgfsetfillcolor{currentfill}%
\pgfsetlinewidth{0.803000pt}%
\definecolor{currentstroke}{rgb}{0.000000,0.000000,0.000000}%
\pgfsetstrokecolor{currentstroke}%
\pgfsetdash{}{0pt}%
\pgfsys@defobject{currentmarker}{\pgfqpoint{-0.048611in}{0.000000in}}{\pgfqpoint{-0.000000in}{0.000000in}}{%
\pgfpathmoveto{\pgfqpoint{-0.000000in}{0.000000in}}%
\pgfpathlineto{\pgfqpoint{-0.048611in}{0.000000in}}%
\pgfusepath{stroke,fill}%
}%
\begin{pgfscope}%
\pgfsys@transformshift{0.504436in}{1.020217in}%
\pgfsys@useobject{currentmarker}{}%
\end{pgfscope}%
\end{pgfscope}%
\begin{pgfscope}%
\definecolor{textcolor}{rgb}{0.000000,0.000000,0.000000}%
\pgfsetstrokecolor{textcolor}%
\pgfsetfillcolor{textcolor}%
\pgftext[x=0.178820in, y=0.976842in, left, base]{\color{textcolor}\rmfamily\fontsize{9.000000}{10.800000}\selectfont \(\displaystyle {0.50}\)}%
\end{pgfscope}%
\begin{pgfscope}%
\pgfpathrectangle{\pgfqpoint{0.504436in}{0.451277in}}{\pgfqpoint{2.662540in}{1.137880in}}%
\pgfusepath{clip}%
\pgfsetrectcap%
\pgfsetroundjoin%
\pgfsetlinewidth{0.803000pt}%
\definecolor{currentstroke}{rgb}{0.690196,0.690196,0.690196}%
\pgfsetstrokecolor{currentstroke}%
\pgfsetdash{}{0pt}%
\pgfpathmoveto{\pgfqpoint{0.504436in}{1.304687in}}%
\pgfpathlineto{\pgfqpoint{3.166976in}{1.304687in}}%
\pgfusepath{stroke}%
\end{pgfscope}%
\begin{pgfscope}%
\pgfsetbuttcap%
\pgfsetroundjoin%
\definecolor{currentfill}{rgb}{0.000000,0.000000,0.000000}%
\pgfsetfillcolor{currentfill}%
\pgfsetlinewidth{0.803000pt}%
\definecolor{currentstroke}{rgb}{0.000000,0.000000,0.000000}%
\pgfsetstrokecolor{currentstroke}%
\pgfsetdash{}{0pt}%
\pgfsys@defobject{currentmarker}{\pgfqpoint{-0.048611in}{0.000000in}}{\pgfqpoint{-0.000000in}{0.000000in}}{%
\pgfpathmoveto{\pgfqpoint{-0.000000in}{0.000000in}}%
\pgfpathlineto{\pgfqpoint{-0.048611in}{0.000000in}}%
\pgfusepath{stroke,fill}%
}%
\begin{pgfscope}%
\pgfsys@transformshift{0.504436in}{1.304687in}%
\pgfsys@useobject{currentmarker}{}%
\end{pgfscope}%
\end{pgfscope}%
\begin{pgfscope}%
\definecolor{textcolor}{rgb}{0.000000,0.000000,0.000000}%
\pgfsetstrokecolor{textcolor}%
\pgfsetfillcolor{textcolor}%
\pgftext[x=0.178820in, y=1.261312in, left, base]{\color{textcolor}\rmfamily\fontsize{9.000000}{10.800000}\selectfont \(\displaystyle {0.75}\)}%
\end{pgfscope}%
\begin{pgfscope}%
\pgfpathrectangle{\pgfqpoint{0.504436in}{0.451277in}}{\pgfqpoint{2.662540in}{1.137880in}}%
\pgfusepath{clip}%
\pgfsetrectcap%
\pgfsetroundjoin%
\pgfsetlinewidth{0.803000pt}%
\definecolor{currentstroke}{rgb}{0.690196,0.690196,0.690196}%
\pgfsetstrokecolor{currentstroke}%
\pgfsetdash{}{0pt}%
\pgfpathmoveto{\pgfqpoint{0.504436in}{1.589158in}}%
\pgfpathlineto{\pgfqpoint{3.166976in}{1.589158in}}%
\pgfusepath{stroke}%
\end{pgfscope}%
\begin{pgfscope}%
\pgfsetbuttcap%
\pgfsetroundjoin%
\definecolor{currentfill}{rgb}{0.000000,0.000000,0.000000}%
\pgfsetfillcolor{currentfill}%
\pgfsetlinewidth{0.803000pt}%
\definecolor{currentstroke}{rgb}{0.000000,0.000000,0.000000}%
\pgfsetstrokecolor{currentstroke}%
\pgfsetdash{}{0pt}%
\pgfsys@defobject{currentmarker}{\pgfqpoint{-0.048611in}{0.000000in}}{\pgfqpoint{-0.000000in}{0.000000in}}{%
\pgfpathmoveto{\pgfqpoint{-0.000000in}{0.000000in}}%
\pgfpathlineto{\pgfqpoint{-0.048611in}{0.000000in}}%
\pgfusepath{stroke,fill}%
}%
\begin{pgfscope}%
\pgfsys@transformshift{0.504436in}{1.589158in}%
\pgfsys@useobject{currentmarker}{}%
\end{pgfscope}%
\end{pgfscope}%
\begin{pgfscope}%
\definecolor{textcolor}{rgb}{0.000000,0.000000,0.000000}%
\pgfsetstrokecolor{textcolor}%
\pgfsetfillcolor{textcolor}%
\pgftext[x=0.178820in, y=1.545783in, left, base]{\color{textcolor}\rmfamily\fontsize{9.000000}{10.800000}\selectfont \(\displaystyle {1.00}\)}%
\end{pgfscope}%
\begin{pgfscope}%
\definecolor{textcolor}{rgb}{0.000000,0.000000,0.000000}%
\pgfsetstrokecolor{textcolor}%
\pgfsetfillcolor{textcolor}%
\pgftext[x=0.123264in,y=1.020217in,,bottom,rotate=90.000000]{\color{textcolor}\rmfamily\fontsize{9.000000}{10.800000}\selectfont \(\displaystyle \|\mathbf{h}_i\|\) [1]}%
\end{pgfscope}%
\begin{pgfscope}%
\pgfpathrectangle{\pgfqpoint{0.504436in}{0.451277in}}{\pgfqpoint{2.662540in}{1.137880in}}%
\pgfusepath{clip}%
\pgfsetbuttcap%
\pgfsetroundjoin%
\pgfsetlinewidth{0.752812pt}%
\definecolor{currentstroke}{rgb}{0.000000,0.000000,0.000000}%
\pgfsetstrokecolor{currentstroke}%
\pgfsetdash{{2.775000pt}{1.200000pt}}{0.000000pt}%
\pgfpathmoveto{\pgfqpoint{1.391949in}{0.451277in}}%
\pgfpathlineto{\pgfqpoint{1.391949in}{1.589158in}}%
\pgfusepath{stroke}%
\end{pgfscope}%
\begin{pgfscope}%
\pgfpathrectangle{\pgfqpoint{0.504436in}{0.451277in}}{\pgfqpoint{2.662540in}{1.137880in}}%
\pgfusepath{clip}%
\pgfsetbuttcap%
\pgfsetroundjoin%
\pgfsetlinewidth{0.752812pt}%
\definecolor{currentstroke}{rgb}{0.000000,0.000000,0.000000}%
\pgfsetstrokecolor{currentstroke}%
\pgfsetdash{{2.775000pt}{1.200000pt}}{0.000000pt}%
\pgfpathmoveto{\pgfqpoint{2.279462in}{0.451277in}}%
\pgfpathlineto{\pgfqpoint{2.279462in}{1.589158in}}%
\pgfusepath{stroke}%
\end{pgfscope}%
\begin{pgfscope}%
\pgfpathrectangle{\pgfqpoint{0.504436in}{0.451277in}}{\pgfqpoint{2.662540in}{1.137880in}}%
\pgfusepath{clip}%
\pgfsetrectcap%
\pgfsetroundjoin%
\pgfsetlinewidth{1.505625pt}%
\definecolor{currentstroke}{rgb}{0.121569,0.466667,0.705882}%
\pgfsetstrokecolor{currentstroke}%
\pgfsetdash{}{0pt}%
\pgfpathmoveto{\pgfqpoint{0.504436in}{0.609275in}}%
\pgfpathlineto{\pgfqpoint{0.504613in}{0.529414in}}%
\pgfpathlineto{\pgfqpoint{0.505146in}{0.741343in}}%
\pgfpathlineto{\pgfqpoint{0.507276in}{1.485725in}}%
\pgfpathlineto{\pgfqpoint{0.507453in}{1.484658in}}%
\pgfpathlineto{\pgfqpoint{0.511713in}{1.430778in}}%
\pgfpathlineto{\pgfqpoint{0.517393in}{1.462839in}}%
\pgfpathlineto{\pgfqpoint{0.520233in}{1.468624in}}%
\pgfpathlineto{\pgfqpoint{0.524316in}{1.479939in}}%
\pgfpathlineto{\pgfqpoint{0.525381in}{1.477984in}}%
\pgfpathlineto{\pgfqpoint{0.529464in}{1.458204in}}%
\pgfpathlineto{\pgfqpoint{0.530529in}{1.460662in}}%
\pgfpathlineto{\pgfqpoint{0.532481in}{1.467384in}}%
\pgfpathlineto{\pgfqpoint{0.533191in}{1.466260in}}%
\pgfpathlineto{\pgfqpoint{0.534611in}{1.459667in}}%
\pgfpathlineto{\pgfqpoint{0.540469in}{1.404879in}}%
\pgfpathlineto{\pgfqpoint{0.544374in}{1.371215in}}%
\pgfpathlineto{\pgfqpoint{0.549521in}{1.356357in}}%
\pgfpathlineto{\pgfqpoint{0.552006in}{1.330646in}}%
\pgfpathlineto{\pgfqpoint{0.555201in}{1.314513in}}%
\pgfpathlineto{\pgfqpoint{0.557331in}{1.310598in}}%
\pgfpathlineto{\pgfqpoint{0.557686in}{1.310888in}}%
\pgfpathlineto{\pgfqpoint{0.559994in}{1.317834in}}%
\pgfpathlineto{\pgfqpoint{0.563367in}{1.338485in}}%
\pgfpathlineto{\pgfqpoint{0.566917in}{1.352703in}}%
\pgfpathlineto{\pgfqpoint{0.567804in}{1.351781in}}%
\pgfpathlineto{\pgfqpoint{0.569934in}{1.338401in}}%
\pgfpathlineto{\pgfqpoint{0.573662in}{1.321450in}}%
\pgfpathlineto{\pgfqpoint{0.575437in}{1.322003in}}%
\pgfpathlineto{\pgfqpoint{0.577567in}{1.324227in}}%
\pgfpathlineto{\pgfqpoint{0.581472in}{1.328983in}}%
\pgfpathlineto{\pgfqpoint{0.581827in}{1.328640in}}%
\pgfpathlineto{\pgfqpoint{0.583424in}{1.324533in}}%
\pgfpathlineto{\pgfqpoint{0.585732in}{1.302932in}}%
\pgfpathlineto{\pgfqpoint{0.589282in}{1.278097in}}%
\pgfpathlineto{\pgfqpoint{0.599400in}{1.245968in}}%
\pgfpathlineto{\pgfqpoint{0.602062in}{1.245699in}}%
\pgfpathlineto{\pgfqpoint{0.604015in}{1.249863in}}%
\pgfpathlineto{\pgfqpoint{0.608630in}{1.267990in}}%
\pgfpathlineto{\pgfqpoint{0.619102in}{1.300540in}}%
\pgfpathlineto{\pgfqpoint{0.622652in}{1.304298in}}%
\pgfpathlineto{\pgfqpoint{0.627623in}{1.305199in}}%
\pgfpathlineto{\pgfqpoint{0.633303in}{1.312608in}}%
\pgfpathlineto{\pgfqpoint{0.634900in}{1.311774in}}%
\pgfpathlineto{\pgfqpoint{0.637208in}{1.305592in}}%
\pgfpathlineto{\pgfqpoint{0.640580in}{1.298914in}}%
\pgfpathlineto{\pgfqpoint{0.642000in}{1.299763in}}%
\pgfpathlineto{\pgfqpoint{0.644840in}{1.305748in}}%
\pgfpathlineto{\pgfqpoint{0.650165in}{1.317597in}}%
\pgfpathlineto{\pgfqpoint{0.654070in}{1.318931in}}%
\pgfpathlineto{\pgfqpoint{0.662058in}{1.314716in}}%
\pgfpathlineto{\pgfqpoint{0.664188in}{1.314483in}}%
\pgfpathlineto{\pgfqpoint{0.666673in}{1.315478in}}%
\pgfpathlineto{\pgfqpoint{0.675726in}{1.319714in}}%
\pgfpathlineto{\pgfqpoint{0.680518in}{1.316818in}}%
\pgfpathlineto{\pgfqpoint{0.683891in}{1.315070in}}%
\pgfpathlineto{\pgfqpoint{0.687796in}{1.311191in}}%
\pgfpathlineto{\pgfqpoint{0.692766in}{1.298624in}}%
\pgfpathlineto{\pgfqpoint{0.697381in}{1.287964in}}%
\pgfpathlineto{\pgfqpoint{0.707321in}{1.281590in}}%
\pgfpathlineto{\pgfqpoint{0.720989in}{1.288514in}}%
\pgfpathlineto{\pgfqpoint{0.735189in}{1.322876in}}%
\pgfpathlineto{\pgfqpoint{0.736077in}{1.322074in}}%
\pgfpathlineto{\pgfqpoint{0.741757in}{1.313362in}}%
\pgfpathlineto{\pgfqpoint{0.744952in}{1.309676in}}%
\pgfpathlineto{\pgfqpoint{0.750099in}{1.307206in}}%
\pgfpathlineto{\pgfqpoint{0.754537in}{1.303923in}}%
\pgfpathlineto{\pgfqpoint{0.765365in}{1.299575in}}%
\pgfpathlineto{\pgfqpoint{0.775837in}{1.309079in}}%
\pgfpathlineto{\pgfqpoint{0.779742in}{1.307633in}}%
\pgfpathlineto{\pgfqpoint{0.785422in}{1.298477in}}%
\pgfpathlineto{\pgfqpoint{0.789505in}{1.291279in}}%
\pgfpathlineto{\pgfqpoint{0.796783in}{1.285580in}}%
\pgfpathlineto{\pgfqpoint{0.815065in}{1.299694in}}%
\pgfpathlineto{\pgfqpoint{0.821810in}{1.297133in}}%
\pgfpathlineto{\pgfqpoint{0.828556in}{1.291945in}}%
\pgfpathlineto{\pgfqpoint{0.837253in}{1.295397in}}%
\pgfpathlineto{\pgfqpoint{0.844176in}{1.305260in}}%
\pgfpathlineto{\pgfqpoint{0.847548in}{1.307341in}}%
\pgfpathlineto{\pgfqpoint{0.850743in}{1.305891in}}%
\pgfpathlineto{\pgfqpoint{0.856779in}{1.301667in}}%
\pgfpathlineto{\pgfqpoint{0.865121in}{1.297008in}}%
\pgfpathlineto{\pgfqpoint{0.871511in}{1.288949in}}%
\pgfpathlineto{\pgfqpoint{0.876659in}{1.281548in}}%
\pgfpathlineto{\pgfqpoint{0.889972in}{1.255908in}}%
\pgfpathlineto{\pgfqpoint{0.897427in}{1.253816in}}%
\pgfpathlineto{\pgfqpoint{0.903994in}{1.261412in}}%
\pgfpathlineto{\pgfqpoint{0.910917in}{1.281319in}}%
\pgfpathlineto{\pgfqpoint{0.921389in}{1.319642in}}%
\pgfpathlineto{\pgfqpoint{0.925117in}{1.324065in}}%
\pgfpathlineto{\pgfqpoint{0.928312in}{1.325060in}}%
\pgfpathlineto{\pgfqpoint{0.933282in}{1.319318in}}%
\pgfpathlineto{\pgfqpoint{0.936832in}{1.313413in}}%
\pgfpathlineto{\pgfqpoint{0.951387in}{1.283156in}}%
\pgfpathlineto{\pgfqpoint{0.957600in}{1.283995in}}%
\pgfpathlineto{\pgfqpoint{0.960440in}{1.287296in}}%
\pgfpathlineto{\pgfqpoint{0.978013in}{1.321166in}}%
\pgfpathlineto{\pgfqpoint{0.981740in}{1.326156in}}%
\pgfpathlineto{\pgfqpoint{0.984758in}{1.326305in}}%
\pgfpathlineto{\pgfqpoint{0.987243in}{1.323811in}}%
\pgfpathlineto{\pgfqpoint{0.994521in}{1.312152in}}%
\pgfpathlineto{\pgfqpoint{1.001621in}{1.296187in}}%
\pgfpathlineto{\pgfqpoint{1.010141in}{1.274242in}}%
\pgfpathlineto{\pgfqpoint{1.019193in}{1.266425in}}%
\pgfpathlineto{\pgfqpoint{1.022921in}{1.265019in}}%
\pgfpathlineto{\pgfqpoint{1.026471in}{1.264242in}}%
\pgfpathlineto{\pgfqpoint{1.028601in}{1.265991in}}%
\pgfpathlineto{\pgfqpoint{1.038896in}{1.286137in}}%
\pgfpathlineto{\pgfqpoint{1.048126in}{1.301764in}}%
\pgfpathlineto{\pgfqpoint{1.055759in}{1.309370in}}%
\pgfpathlineto{\pgfqpoint{1.062682in}{1.306418in}}%
\pgfpathlineto{\pgfqpoint{1.064989in}{1.301381in}}%
\pgfpathlineto{\pgfqpoint{1.075107in}{1.278871in}}%
\pgfpathlineto{\pgfqpoint{1.078657in}{1.277121in}}%
\pgfpathlineto{\pgfqpoint{1.086822in}{1.275147in}}%
\pgfpathlineto{\pgfqpoint{1.092857in}{1.278225in}}%
\pgfpathlineto{\pgfqpoint{1.105460in}{1.297507in}}%
\pgfpathlineto{\pgfqpoint{1.109720in}{1.300897in}}%
\pgfpathlineto{\pgfqpoint{1.115577in}{1.304577in}}%
\pgfpathlineto{\pgfqpoint{1.120015in}{1.309173in}}%
\pgfpathlineto{\pgfqpoint{1.124098in}{1.310046in}}%
\pgfpathlineto{\pgfqpoint{1.127293in}{1.309767in}}%
\pgfpathlineto{\pgfqpoint{1.140605in}{1.310794in}}%
\pgfpathlineto{\pgfqpoint{1.143090in}{1.308669in}}%
\pgfpathlineto{\pgfqpoint{1.150368in}{1.298079in}}%
\pgfpathlineto{\pgfqpoint{1.154805in}{1.296282in}}%
\pgfpathlineto{\pgfqpoint{1.169183in}{1.312496in}}%
\pgfpathlineto{\pgfqpoint{1.178236in}{1.317035in}}%
\pgfpathlineto{\pgfqpoint{1.180366in}{1.315498in}}%
\pgfpathlineto{\pgfqpoint{1.188176in}{1.310448in}}%
\pgfpathlineto{\pgfqpoint{1.195276in}{1.305781in}}%
\pgfpathlineto{\pgfqpoint{1.199536in}{1.297532in}}%
\pgfpathlineto{\pgfqpoint{1.204151in}{1.289049in}}%
\pgfpathlineto{\pgfqpoint{1.207879in}{1.286820in}}%
\pgfpathlineto{\pgfqpoint{1.219594in}{1.282254in}}%
\pgfpathlineto{\pgfqpoint{1.226872in}{1.289056in}}%
\pgfpathlineto{\pgfqpoint{1.231487in}{1.294119in}}%
\pgfpathlineto{\pgfqpoint{1.242669in}{1.304483in}}%
\pgfpathlineto{\pgfqpoint{1.245864in}{1.304511in}}%
\pgfpathlineto{\pgfqpoint{1.253497in}{1.302535in}}%
\pgfpathlineto{\pgfqpoint{1.269650in}{1.285461in}}%
\pgfpathlineto{\pgfqpoint{1.273910in}{1.280503in}}%
\pgfpathlineto{\pgfqpoint{1.276750in}{1.281146in}}%
\pgfpathlineto{\pgfqpoint{1.281365in}{1.286956in}}%
\pgfpathlineto{\pgfqpoint{1.289530in}{1.299185in}}%
\pgfpathlineto{\pgfqpoint{1.310298in}{1.299503in}}%
\pgfpathlineto{\pgfqpoint{1.316865in}{1.298180in}}%
\pgfpathlineto{\pgfqpoint{1.325918in}{1.300277in}}%
\pgfpathlineto{\pgfqpoint{1.332841in}{1.298093in}}%
\pgfpathlineto{\pgfqpoint{1.338343in}{1.301526in}}%
\pgfpathlineto{\pgfqpoint{1.341893in}{1.303556in}}%
\pgfpathlineto{\pgfqpoint{1.354851in}{1.307372in}}%
\pgfpathlineto{\pgfqpoint{1.357336in}{1.304356in}}%
\pgfpathlineto{\pgfqpoint{1.362839in}{1.298317in}}%
\pgfpathlineto{\pgfqpoint{1.372246in}{1.288623in}}%
\pgfpathlineto{\pgfqpoint{1.380234in}{1.275528in}}%
\pgfpathlineto{\pgfqpoint{1.383606in}{1.277071in}}%
\pgfpathlineto{\pgfqpoint{1.391417in}{1.285054in}}%
\pgfpathlineto{\pgfqpoint{1.391594in}{1.284044in}}%
\pgfpathlineto{\pgfqpoint{1.392304in}{1.267190in}}%
\pgfpathlineto{\pgfqpoint{1.395677in}{1.086774in}}%
\pgfpathlineto{\pgfqpoint{1.399582in}{0.962951in}}%
\pgfpathlineto{\pgfqpoint{1.402599in}{0.942859in}}%
\pgfpathlineto{\pgfqpoint{1.403309in}{0.942244in}}%
\pgfpathlineto{\pgfqpoint{1.404019in}{0.943128in}}%
\pgfpathlineto{\pgfqpoint{1.405617in}{0.950678in}}%
\pgfpathlineto{\pgfqpoint{1.407747in}{0.978501in}}%
\pgfpathlineto{\pgfqpoint{1.410587in}{1.063755in}}%
\pgfpathlineto{\pgfqpoint{1.421770in}{1.461649in}}%
\pgfpathlineto{\pgfqpoint{1.424610in}{1.469647in}}%
\pgfpathlineto{\pgfqpoint{1.429580in}{1.463161in}}%
\pgfpathlineto{\pgfqpoint{1.432775in}{1.447906in}}%
\pgfpathlineto{\pgfqpoint{1.435970in}{1.410047in}}%
\pgfpathlineto{\pgfqpoint{1.441650in}{1.298037in}}%
\pgfpathlineto{\pgfqpoint{1.456205in}{1.009244in}}%
\pgfpathlineto{\pgfqpoint{1.459045in}{1.001342in}}%
\pgfpathlineto{\pgfqpoint{1.460643in}{1.003399in}}%
\pgfpathlineto{\pgfqpoint{1.463305in}{1.013982in}}%
\pgfpathlineto{\pgfqpoint{1.467920in}{1.046547in}}%
\pgfpathlineto{\pgfqpoint{1.474310in}{1.120287in}}%
\pgfpathlineto{\pgfqpoint{1.487623in}{1.273690in}}%
\pgfpathlineto{\pgfqpoint{1.493836in}{1.304943in}}%
\pgfpathlineto{\pgfqpoint{1.496321in}{1.307826in}}%
\pgfpathlineto{\pgfqpoint{1.496498in}{1.307717in}}%
\pgfpathlineto{\pgfqpoint{1.498273in}{1.304188in}}%
\pgfpathlineto{\pgfqpoint{1.502533in}{1.285982in}}%
\pgfpathlineto{\pgfqpoint{1.507681in}{1.250278in}}%
\pgfpathlineto{\pgfqpoint{1.526319in}{1.117054in}}%
\pgfpathlineto{\pgfqpoint{1.533596in}{1.098174in}}%
\pgfpathlineto{\pgfqpoint{1.537856in}{1.097091in}}%
\pgfpathlineto{\pgfqpoint{1.540874in}{1.102042in}}%
\pgfpathlineto{\pgfqpoint{1.546376in}{1.122219in}}%
\pgfpathlineto{\pgfqpoint{1.570517in}{1.214779in}}%
\pgfpathlineto{\pgfqpoint{1.579569in}{1.227931in}}%
\pgfpathlineto{\pgfqpoint{1.583829in}{1.227419in}}%
\pgfpathlineto{\pgfqpoint{1.588622in}{1.219318in}}%
\pgfpathlineto{\pgfqpoint{1.595012in}{1.205674in}}%
\pgfpathlineto{\pgfqpoint{1.602112in}{1.189052in}}%
\pgfpathlineto{\pgfqpoint{1.610810in}{1.161788in}}%
\pgfpathlineto{\pgfqpoint{1.615602in}{1.152957in}}%
\pgfpathlineto{\pgfqpoint{1.625898in}{1.146186in}}%
\pgfpathlineto{\pgfqpoint{1.631045in}{1.147483in}}%
\pgfpathlineto{\pgfqpoint{1.636015in}{1.147318in}}%
\pgfpathlineto{\pgfqpoint{1.639920in}{1.147486in}}%
\pgfpathlineto{\pgfqpoint{1.642405in}{1.149097in}}%
\pgfpathlineto{\pgfqpoint{1.647730in}{1.155554in}}%
\pgfpathlineto{\pgfqpoint{1.659623in}{1.181013in}}%
\pgfpathlineto{\pgfqpoint{1.686426in}{1.199717in}}%
\pgfpathlineto{\pgfqpoint{1.690331in}{1.198552in}}%
\pgfpathlineto{\pgfqpoint{1.717489in}{1.167006in}}%
\pgfpathlineto{\pgfqpoint{1.730624in}{1.163950in}}%
\pgfpathlineto{\pgfqpoint{1.757250in}{1.170189in}}%
\pgfpathlineto{\pgfqpoint{1.768255in}{1.183364in}}%
\pgfpathlineto{\pgfqpoint{1.774290in}{1.186803in}}%
\pgfpathlineto{\pgfqpoint{1.779260in}{1.187784in}}%
\pgfpathlineto{\pgfqpoint{1.789555in}{1.183870in}}%
\pgfpathlineto{\pgfqpoint{1.794170in}{1.180634in}}%
\pgfpathlineto{\pgfqpoint{1.798963in}{1.181983in}}%
\pgfpathlineto{\pgfqpoint{1.801625in}{1.182002in}}%
\pgfpathlineto{\pgfqpoint{1.815115in}{1.181718in}}%
\pgfpathlineto{\pgfqpoint{1.821506in}{1.175832in}}%
\pgfpathlineto{\pgfqpoint{1.852924in}{1.156523in}}%
\pgfpathlineto{\pgfqpoint{1.856474in}{1.153132in}}%
\pgfpathlineto{\pgfqpoint{1.859314in}{1.151184in}}%
\pgfpathlineto{\pgfqpoint{1.864639in}{1.149588in}}%
\pgfpathlineto{\pgfqpoint{1.868899in}{1.152371in}}%
\pgfpathlineto{\pgfqpoint{1.873869in}{1.158242in}}%
\pgfpathlineto{\pgfqpoint{1.878306in}{1.166636in}}%
\pgfpathlineto{\pgfqpoint{1.888957in}{1.182117in}}%
\pgfpathlineto{\pgfqpoint{1.890554in}{1.183662in}}%
\pgfpathlineto{\pgfqpoint{1.899074in}{1.192576in}}%
\pgfpathlineto{\pgfqpoint{1.901559in}{1.194657in}}%
\pgfpathlineto{\pgfqpoint{1.908304in}{1.198221in}}%
\pgfpathlineto{\pgfqpoint{1.913097in}{1.195964in}}%
\pgfpathlineto{\pgfqpoint{1.925522in}{1.179919in}}%
\pgfpathlineto{\pgfqpoint{1.928717in}{1.177374in}}%
\pgfpathlineto{\pgfqpoint{1.935285in}{1.171064in}}%
\pgfpathlineto{\pgfqpoint{1.940610in}{1.168975in}}%
\pgfpathlineto{\pgfqpoint{1.942385in}{1.168653in}}%
\pgfpathlineto{\pgfqpoint{1.944692in}{1.167623in}}%
\pgfpathlineto{\pgfqpoint{1.950727in}{1.168158in}}%
\pgfpathlineto{\pgfqpoint{1.959780in}{1.176009in}}%
\pgfpathlineto{\pgfqpoint{1.965815in}{1.180723in}}%
\pgfpathlineto{\pgfqpoint{1.976110in}{1.189579in}}%
\pgfpathlineto{\pgfqpoint{1.986761in}{1.189169in}}%
\pgfpathlineto{\pgfqpoint{1.990666in}{1.186852in}}%
\pgfpathlineto{\pgfqpoint{2.002913in}{1.178930in}}%
\pgfpathlineto{\pgfqpoint{2.006286in}{1.174567in}}%
\pgfpathlineto{\pgfqpoint{2.020131in}{1.155466in}}%
\pgfpathlineto{\pgfqpoint{2.028296in}{1.158183in}}%
\pgfpathlineto{\pgfqpoint{2.042851in}{1.165613in}}%
\pgfpathlineto{\pgfqpoint{2.063087in}{1.195860in}}%
\pgfpathlineto{\pgfqpoint{2.067702in}{1.199176in}}%
\pgfpathlineto{\pgfqpoint{2.078352in}{1.196744in}}%
\pgfpathlineto{\pgfqpoint{2.083322in}{1.192208in}}%
\pgfpathlineto{\pgfqpoint{2.090600in}{1.182835in}}%
\pgfpathlineto{\pgfqpoint{2.097700in}{1.170813in}}%
\pgfpathlineto{\pgfqpoint{2.104090in}{1.161454in}}%
\pgfpathlineto{\pgfqpoint{2.113497in}{1.159825in}}%
\pgfpathlineto{\pgfqpoint{2.117225in}{1.159844in}}%
\pgfpathlineto{\pgfqpoint{2.126100in}{1.161315in}}%
\pgfpathlineto{\pgfqpoint{2.133555in}{1.163884in}}%
\pgfpathlineto{\pgfqpoint{2.137815in}{1.166963in}}%
\pgfpathlineto{\pgfqpoint{2.148110in}{1.175160in}}%
\pgfpathlineto{\pgfqpoint{2.156631in}{1.185694in}}%
\pgfpathlineto{\pgfqpoint{2.161246in}{1.189080in}}%
\pgfpathlineto{\pgfqpoint{2.165861in}{1.189803in}}%
\pgfpathlineto{\pgfqpoint{2.170831in}{1.188706in}}%
\pgfpathlineto{\pgfqpoint{2.175623in}{1.187742in}}%
\pgfpathlineto{\pgfqpoint{2.180771in}{1.188776in}}%
\pgfpathlineto{\pgfqpoint{2.184676in}{1.189814in}}%
\pgfpathlineto{\pgfqpoint{2.188049in}{1.189366in}}%
\pgfpathlineto{\pgfqpoint{2.190889in}{1.188147in}}%
\pgfpathlineto{\pgfqpoint{2.199764in}{1.183758in}}%
\pgfpathlineto{\pgfqpoint{2.205976in}{1.180127in}}%
\pgfpathlineto{\pgfqpoint{2.207929in}{1.179194in}}%
\pgfpathlineto{\pgfqpoint{2.225324in}{1.162185in}}%
\pgfpathlineto{\pgfqpoint{2.231537in}{1.154207in}}%
\pgfpathlineto{\pgfqpoint{2.235797in}{1.152829in}}%
\pgfpathlineto{\pgfqpoint{2.241654in}{1.157143in}}%
\pgfpathlineto{\pgfqpoint{2.248044in}{1.165672in}}%
\pgfpathlineto{\pgfqpoint{2.278042in}{1.200152in}}%
\pgfpathlineto{\pgfqpoint{2.279107in}{1.198311in}}%
\pgfpathlineto{\pgfqpoint{2.279817in}{1.176470in}}%
\pgfpathlineto{\pgfqpoint{2.287983in}{0.784350in}}%
\pgfpathlineto{\pgfqpoint{2.288515in}{0.784384in}}%
\pgfpathlineto{\pgfqpoint{2.288870in}{0.785185in}}%
\pgfpathlineto{\pgfqpoint{2.290290in}{0.795028in}}%
\pgfpathlineto{\pgfqpoint{2.292420in}{0.838993in}}%
\pgfpathlineto{\pgfqpoint{2.295083in}{0.977099in}}%
\pgfpathlineto{\pgfqpoint{2.301473in}{1.318804in}}%
\pgfpathlineto{\pgfqpoint{2.305200in}{1.347036in}}%
\pgfpathlineto{\pgfqpoint{2.308218in}{1.352385in}}%
\pgfpathlineto{\pgfqpoint{2.310170in}{1.352705in}}%
\pgfpathlineto{\pgfqpoint{2.310348in}{1.352365in}}%
\pgfpathlineto{\pgfqpoint{2.313543in}{1.339169in}}%
\pgfpathlineto{\pgfqpoint{2.322951in}{1.277417in}}%
\pgfpathlineto{\pgfqpoint{2.327211in}{1.207253in}}%
\pgfpathlineto{\pgfqpoint{2.333778in}{1.045530in}}%
\pgfpathlineto{\pgfqpoint{2.340523in}{0.912846in}}%
\pgfpathlineto{\pgfqpoint{2.343896in}{0.891895in}}%
\pgfpathlineto{\pgfqpoint{2.345493in}{0.890428in}}%
\pgfpathlineto{\pgfqpoint{2.346026in}{0.890923in}}%
\pgfpathlineto{\pgfqpoint{2.350463in}{0.901656in}}%
\pgfpathlineto{\pgfqpoint{2.355256in}{0.922610in}}%
\pgfpathlineto{\pgfqpoint{2.359694in}{0.957425in}}%
\pgfpathlineto{\pgfqpoint{2.365374in}{1.029424in}}%
\pgfpathlineto{\pgfqpoint{2.372296in}{1.105223in}}%
\pgfpathlineto{\pgfqpoint{2.376556in}{1.125581in}}%
\pgfpathlineto{\pgfqpoint{2.380106in}{1.130208in}}%
\pgfpathlineto{\pgfqpoint{2.381704in}{1.129007in}}%
\pgfpathlineto{\pgfqpoint{2.387739in}{1.115024in}}%
\pgfpathlineto{\pgfqpoint{2.391467in}{1.096006in}}%
\pgfpathlineto{\pgfqpoint{2.410992in}{0.984409in}}%
\pgfpathlineto{\pgfqpoint{2.417915in}{0.971573in}}%
\pgfpathlineto{\pgfqpoint{2.421110in}{0.971434in}}%
\pgfpathlineto{\pgfqpoint{2.424660in}{0.973393in}}%
\pgfpathlineto{\pgfqpoint{2.427677in}{0.978136in}}%
\pgfpathlineto{\pgfqpoint{2.432470in}{0.993081in}}%
\pgfpathlineto{\pgfqpoint{2.441345in}{1.022875in}}%
\pgfpathlineto{\pgfqpoint{2.459805in}{1.062100in}}%
\pgfpathlineto{\pgfqpoint{2.464953in}{1.065258in}}%
\pgfpathlineto{\pgfqpoint{2.469568in}{1.064281in}}%
\pgfpathlineto{\pgfqpoint{2.476668in}{1.055524in}}%
\pgfpathlineto{\pgfqpoint{2.489271in}{1.019535in}}%
\pgfpathlineto{\pgfqpoint{2.500986in}{0.987650in}}%
\pgfpathlineto{\pgfqpoint{2.507908in}{0.979540in}}%
\pgfpathlineto{\pgfqpoint{2.511281in}{0.980816in}}%
\pgfpathlineto{\pgfqpoint{2.518026in}{0.990834in}}%
\pgfpathlineto{\pgfqpoint{2.524239in}{1.006307in}}%
\pgfpathlineto{\pgfqpoint{2.537374in}{1.041086in}}%
\pgfpathlineto{\pgfqpoint{2.546959in}{1.053556in}}%
\pgfpathlineto{\pgfqpoint{2.549621in}{1.054046in}}%
\pgfpathlineto{\pgfqpoint{2.555657in}{1.050075in}}%
\pgfpathlineto{\pgfqpoint{2.563467in}{1.041108in}}%
\pgfpathlineto{\pgfqpoint{2.565952in}{1.039377in}}%
\pgfpathlineto{\pgfqpoint{2.570567in}{1.035140in}}%
\pgfpathlineto{\pgfqpoint{2.578732in}{1.021896in}}%
\pgfpathlineto{\pgfqpoint{2.586365in}{1.010894in}}%
\pgfpathlineto{\pgfqpoint{2.591512in}{1.008083in}}%
\pgfpathlineto{\pgfqpoint{2.602872in}{1.004295in}}%
\pgfpathlineto{\pgfqpoint{2.608730in}{1.005554in}}%
\pgfpathlineto{\pgfqpoint{2.613522in}{1.007098in}}%
\pgfpathlineto{\pgfqpoint{2.622220in}{1.011660in}}%
\pgfpathlineto{\pgfqpoint{2.627013in}{1.014648in}}%
\pgfpathlineto{\pgfqpoint{2.636598in}{1.018809in}}%
\pgfpathlineto{\pgfqpoint{2.654703in}{1.033314in}}%
\pgfpathlineto{\pgfqpoint{2.660383in}{1.031696in}}%
\pgfpathlineto{\pgfqpoint{2.666418in}{1.028288in}}%
\pgfpathlineto{\pgfqpoint{2.669613in}{1.027295in}}%
\pgfpathlineto{\pgfqpoint{2.673163in}{1.026622in}}%
\pgfpathlineto{\pgfqpoint{2.675648in}{1.027152in}}%
\pgfpathlineto{\pgfqpoint{2.677956in}{1.027470in}}%
\pgfpathlineto{\pgfqpoint{2.681328in}{1.027473in}}%
\pgfpathlineto{\pgfqpoint{2.685589in}{1.027970in}}%
\pgfpathlineto{\pgfqpoint{2.706711in}{1.013059in}}%
\pgfpathlineto{\pgfqpoint{2.716829in}{1.016213in}}%
\pgfpathlineto{\pgfqpoint{2.722687in}{1.014901in}}%
\pgfpathlineto{\pgfqpoint{2.727834in}{1.015112in}}%
\pgfpathlineto{\pgfqpoint{2.732094in}{1.013841in}}%
\pgfpathlineto{\pgfqpoint{2.736887in}{1.014365in}}%
\pgfpathlineto{\pgfqpoint{2.740614in}{1.014254in}}%
\pgfpathlineto{\pgfqpoint{2.749845in}{1.014962in}}%
\pgfpathlineto{\pgfqpoint{2.753750in}{1.016282in}}%
\pgfpathlineto{\pgfqpoint{2.758187in}{1.017278in}}%
\pgfpathlineto{\pgfqpoint{2.762092in}{1.019165in}}%
\pgfpathlineto{\pgfqpoint{2.767772in}{1.023460in}}%
\pgfpathlineto{\pgfqpoint{2.775405in}{1.028994in}}%
\pgfpathlineto{\pgfqpoint{2.783570in}{1.028796in}}%
\pgfpathlineto{\pgfqpoint{2.791913in}{1.024290in}}%
\pgfpathlineto{\pgfqpoint{2.794753in}{1.023668in}}%
\pgfpathlineto{\pgfqpoint{2.799013in}{1.024694in}}%
\pgfpathlineto{\pgfqpoint{2.806823in}{1.026337in}}%
\pgfpathlineto{\pgfqpoint{2.820668in}{1.022744in}}%
\pgfpathlineto{\pgfqpoint{2.832916in}{1.013319in}}%
\pgfpathlineto{\pgfqpoint{2.841613in}{1.012532in}}%
\pgfpathlineto{\pgfqpoint{2.844808in}{1.011861in}}%
\pgfpathlineto{\pgfqpoint{2.852441in}{1.014180in}}%
\pgfpathlineto{\pgfqpoint{2.866819in}{1.026549in}}%
\pgfpathlineto{\pgfqpoint{2.879421in}{1.030752in}}%
\pgfpathlineto{\pgfqpoint{2.893089in}{1.026051in}}%
\pgfpathlineto{\pgfqpoint{2.903917in}{1.016526in}}%
\pgfpathlineto{\pgfqpoint{2.907289in}{1.013960in}}%
\pgfpathlineto{\pgfqpoint{2.911549in}{1.011378in}}%
\pgfpathlineto{\pgfqpoint{2.916697in}{1.010983in}}%
\pgfpathlineto{\pgfqpoint{2.921312in}{1.011160in}}%
\pgfpathlineto{\pgfqpoint{2.930365in}{1.008724in}}%
\pgfpathlineto{\pgfqpoint{2.933560in}{1.008972in}}%
\pgfpathlineto{\pgfqpoint{2.940837in}{1.010723in}}%
\pgfpathlineto{\pgfqpoint{2.948470in}{1.013797in}}%
\pgfpathlineto{\pgfqpoint{2.951665in}{1.016061in}}%
\pgfpathlineto{\pgfqpoint{2.973320in}{1.034050in}}%
\pgfpathlineto{\pgfqpoint{2.975983in}{1.032535in}}%
\pgfpathlineto{\pgfqpoint{3.005448in}{1.021529in}}%
\pgfpathlineto{\pgfqpoint{3.009531in}{1.019543in}}%
\pgfpathlineto{\pgfqpoint{3.014501in}{1.016294in}}%
\pgfpathlineto{\pgfqpoint{3.020714in}{1.014284in}}%
\pgfpathlineto{\pgfqpoint{3.025151in}{1.011918in}}%
\pgfpathlineto{\pgfqpoint{3.028879in}{1.009471in}}%
\pgfpathlineto{\pgfqpoint{3.034204in}{1.006745in}}%
\pgfpathlineto{\pgfqpoint{3.037221in}{1.006315in}}%
\pgfpathlineto{\pgfqpoint{3.048582in}{1.014039in}}%
\pgfpathlineto{\pgfqpoint{3.065977in}{1.030098in}}%
\pgfpathlineto{\pgfqpoint{3.072367in}{1.035195in}}%
\pgfpathlineto{\pgfqpoint{3.076627in}{1.033903in}}%
\pgfpathlineto{\pgfqpoint{3.098815in}{1.024385in}}%
\pgfpathlineto{\pgfqpoint{3.103252in}{1.020611in}}%
\pgfpathlineto{\pgfqpoint{3.108932in}{1.017374in}}%
\pgfpathlineto{\pgfqpoint{3.119050in}{1.012633in}}%
\pgfpathlineto{\pgfqpoint{3.122068in}{1.011801in}}%
\pgfpathlineto{\pgfqpoint{3.128103in}{1.012474in}}%
\pgfpathlineto{\pgfqpoint{3.133783in}{1.012338in}}%
\pgfpathlineto{\pgfqpoint{3.139108in}{1.011342in}}%
\pgfpathlineto{\pgfqpoint{3.144788in}{1.009072in}}%
\pgfpathlineto{\pgfqpoint{3.147806in}{1.008863in}}%
\pgfpathlineto{\pgfqpoint{3.151178in}{1.007947in}}%
\pgfpathlineto{\pgfqpoint{3.158101in}{1.011501in}}%
\pgfpathlineto{\pgfqpoint{3.165911in}{1.018540in}}%
\pgfpathlineto{\pgfqpoint{3.166443in}{1.018860in}}%
\pgfpathlineto{\pgfqpoint{3.166443in}{1.018860in}}%
\pgfusepath{stroke}%
\end{pgfscope}%
\begin{pgfscope}%
\pgfpathrectangle{\pgfqpoint{0.504436in}{0.451277in}}{\pgfqpoint{2.662540in}{1.137880in}}%
\pgfusepath{clip}%
\pgfsetrectcap%
\pgfsetroundjoin%
\pgfsetlinewidth{1.505625pt}%
\definecolor{currentstroke}{rgb}{1.000000,0.498039,0.054902}%
\pgfsetstrokecolor{currentstroke}%
\pgfsetdash{}{0pt}%
\pgfpathmoveto{\pgfqpoint{0.504436in}{0.526937in}}%
\pgfpathlineto{\pgfqpoint{0.504613in}{0.468987in}}%
\pgfpathlineto{\pgfqpoint{0.505856in}{0.511892in}}%
\pgfpathlineto{\pgfqpoint{0.506388in}{0.516564in}}%
\pgfpathlineto{\pgfqpoint{0.507098in}{0.511189in}}%
\pgfpathlineto{\pgfqpoint{0.510293in}{0.502574in}}%
\pgfpathlineto{\pgfqpoint{0.511891in}{0.501380in}}%
\pgfpathlineto{\pgfqpoint{0.512246in}{0.501749in}}%
\pgfpathlineto{\pgfqpoint{0.515263in}{0.504707in}}%
\pgfpathlineto{\pgfqpoint{0.527334in}{0.508890in}}%
\pgfpathlineto{\pgfqpoint{0.541356in}{0.506909in}}%
\pgfpathlineto{\pgfqpoint{0.548456in}{0.503647in}}%
\pgfpathlineto{\pgfqpoint{0.580052in}{0.499742in}}%
\pgfpathlineto{\pgfqpoint{0.589282in}{0.497698in}}%
\pgfpathlineto{\pgfqpoint{0.603127in}{0.495612in}}%
\pgfpathlineto{\pgfqpoint{0.612357in}{0.496306in}}%
\pgfpathlineto{\pgfqpoint{0.638628in}{0.496359in}}%
\pgfpathlineto{\pgfqpoint{0.644485in}{0.496197in}}%
\pgfpathlineto{\pgfqpoint{0.658863in}{0.496729in}}%
\pgfpathlineto{\pgfqpoint{0.672353in}{0.496822in}}%
\pgfpathlineto{\pgfqpoint{0.684956in}{0.496110in}}%
\pgfpathlineto{\pgfqpoint{0.694719in}{0.494648in}}%
\pgfpathlineto{\pgfqpoint{0.706966in}{0.493760in}}%
\pgfpathlineto{\pgfqpoint{0.730752in}{0.495578in}}%
\pgfpathlineto{\pgfqpoint{0.741402in}{0.495622in}}%
\pgfpathlineto{\pgfqpoint{0.802285in}{0.493985in}}%
\pgfpathlineto{\pgfqpoint{0.855358in}{0.495564in}}%
\pgfpathlineto{\pgfqpoint{0.897959in}{0.492371in}}%
\pgfpathlineto{\pgfqpoint{0.914467in}{0.494369in}}%
\pgfpathlineto{\pgfqpoint{0.934347in}{0.495584in}}%
\pgfpathlineto{\pgfqpoint{0.952985in}{0.493649in}}%
\pgfpathlineto{\pgfqpoint{0.968605in}{0.494281in}}%
\pgfpathlineto{\pgfqpoint{1.015466in}{0.492911in}}%
\pgfpathlineto{\pgfqpoint{1.027359in}{0.492640in}}%
\pgfpathlineto{\pgfqpoint{1.073332in}{0.493439in}}%
\pgfpathlineto{\pgfqpoint{1.086999in}{0.492876in}}%
\pgfpathlineto{\pgfqpoint{1.133683in}{0.494803in}}%
\pgfpathlineto{\pgfqpoint{1.157113in}{0.494506in}}%
\pgfpathlineto{\pgfqpoint{1.234149in}{0.493899in}}%
\pgfpathlineto{\pgfqpoint{1.269650in}{0.493902in}}%
\pgfpathlineto{\pgfqpoint{1.281897in}{0.494069in}}%
\pgfpathlineto{\pgfqpoint{1.291128in}{0.494703in}}%
\pgfpathlineto{\pgfqpoint{1.306215in}{0.494356in}}%
\pgfpathlineto{\pgfqpoint{1.349171in}{0.494621in}}%
\pgfpathlineto{\pgfqpoint{1.358756in}{0.494524in}}%
\pgfpathlineto{\pgfqpoint{1.392127in}{0.495126in}}%
\pgfpathlineto{\pgfqpoint{1.396387in}{0.508581in}}%
\pgfpathlineto{\pgfqpoint{1.401889in}{0.515341in}}%
\pgfpathlineto{\pgfqpoint{1.407037in}{0.528589in}}%
\pgfpathlineto{\pgfqpoint{1.410587in}{0.548733in}}%
\pgfpathlineto{\pgfqpoint{1.422302in}{0.631768in}}%
\pgfpathlineto{\pgfqpoint{1.427805in}{0.636186in}}%
\pgfpathlineto{\pgfqpoint{1.432065in}{0.636289in}}%
\pgfpathlineto{\pgfqpoint{1.436502in}{0.631585in}}%
\pgfpathlineto{\pgfqpoint{1.441650in}{0.615110in}}%
\pgfpathlineto{\pgfqpoint{1.447507in}{0.587715in}}%
\pgfpathlineto{\pgfqpoint{1.453365in}{0.564692in}}%
\pgfpathlineto{\pgfqpoint{1.457448in}{0.557898in}}%
\pgfpathlineto{\pgfqpoint{1.460820in}{0.557158in}}%
\pgfpathlineto{\pgfqpoint{1.465790in}{0.561108in}}%
\pgfpathlineto{\pgfqpoint{1.471115in}{0.570345in}}%
\pgfpathlineto{\pgfqpoint{1.494368in}{0.618547in}}%
\pgfpathlineto{\pgfqpoint{1.498628in}{0.619699in}}%
\pgfpathlineto{\pgfqpoint{1.502533in}{0.616955in}}%
\pgfpathlineto{\pgfqpoint{1.509278in}{0.607199in}}%
\pgfpathlineto{\pgfqpoint{1.524544in}{0.584660in}}%
\pgfpathlineto{\pgfqpoint{1.534839in}{0.577878in}}%
\pgfpathlineto{\pgfqpoint{1.539276in}{0.577591in}}%
\pgfpathlineto{\pgfqpoint{1.544779in}{0.580811in}}%
\pgfpathlineto{\pgfqpoint{1.554897in}{0.589923in}}%
\pgfpathlineto{\pgfqpoint{1.565369in}{0.599538in}}%
\pgfpathlineto{\pgfqpoint{1.573179in}{0.603760in}}%
\pgfpathlineto{\pgfqpoint{1.583297in}{0.605300in}}%
\pgfpathlineto{\pgfqpoint{1.606372in}{0.593637in}}%
\pgfpathlineto{\pgfqpoint{1.614182in}{0.589608in}}%
\pgfpathlineto{\pgfqpoint{1.621638in}{0.587977in}}%
\pgfpathlineto{\pgfqpoint{1.650038in}{0.592159in}}%
\pgfpathlineto{\pgfqpoint{1.666901in}{0.597326in}}%
\pgfpathlineto{\pgfqpoint{1.677728in}{0.598860in}}%
\pgfpathlineto{\pgfqpoint{1.747309in}{0.593279in}}%
\pgfpathlineto{\pgfqpoint{1.813873in}{0.598121in}}%
\pgfpathlineto{\pgfqpoint{1.857184in}{0.592958in}}%
\pgfpathlineto{\pgfqpoint{1.865349in}{0.593489in}}%
\pgfpathlineto{\pgfqpoint{1.872981in}{0.594641in}}%
\pgfpathlineto{\pgfqpoint{1.882921in}{0.596924in}}%
\pgfpathlineto{\pgfqpoint{1.912742in}{0.600480in}}%
\pgfpathlineto{\pgfqpoint{1.923215in}{0.598454in}}%
\pgfpathlineto{\pgfqpoint{1.942917in}{0.596206in}}%
\pgfpathlineto{\pgfqpoint{1.960490in}{0.597183in}}%
\pgfpathlineto{\pgfqpoint{1.981968in}{0.600322in}}%
\pgfpathlineto{\pgfqpoint{1.992263in}{0.598560in}}%
\pgfpathlineto{\pgfqpoint{2.020309in}{0.593564in}}%
\pgfpathlineto{\pgfqpoint{2.080837in}{0.600509in}}%
\pgfpathlineto{\pgfqpoint{2.086517in}{0.599249in}}%
\pgfpathlineto{\pgfqpoint{2.113142in}{0.594546in}}%
\pgfpathlineto{\pgfqpoint{2.135330in}{0.595288in}}%
\pgfpathlineto{\pgfqpoint{2.148820in}{0.598273in}}%
\pgfpathlineto{\pgfqpoint{2.153436in}{0.599372in}}%
\pgfpathlineto{\pgfqpoint{2.187871in}{0.599761in}}%
\pgfpathlineto{\pgfqpoint{2.225679in}{0.593846in}}%
\pgfpathlineto{\pgfqpoint{2.235797in}{0.592943in}}%
\pgfpathlineto{\pgfqpoint{2.245382in}{0.594277in}}%
\pgfpathlineto{\pgfqpoint{2.273782in}{0.600994in}}%
\pgfpathlineto{\pgfqpoint{2.279285in}{0.600187in}}%
\pgfpathlineto{\pgfqpoint{2.280882in}{0.562987in}}%
\pgfpathlineto{\pgfqpoint{2.284965in}{0.490864in}}%
\pgfpathlineto{\pgfqpoint{2.288338in}{0.477429in}}%
\pgfpathlineto{\pgfqpoint{2.291533in}{0.474366in}}%
\pgfpathlineto{\pgfqpoint{2.293308in}{0.475765in}}%
\pgfpathlineto{\pgfqpoint{2.295970in}{0.483044in}}%
\pgfpathlineto{\pgfqpoint{2.300763in}{0.494903in}}%
\pgfpathlineto{\pgfqpoint{2.304313in}{0.494838in}}%
\pgfpathlineto{\pgfqpoint{2.323306in}{0.490031in}}%
\pgfpathlineto{\pgfqpoint{2.335198in}{0.477056in}}%
\pgfpathlineto{\pgfqpoint{2.341411in}{0.471667in}}%
\pgfpathlineto{\pgfqpoint{2.346736in}{0.471131in}}%
\pgfpathlineto{\pgfqpoint{2.357031in}{0.473482in}}%
\pgfpathlineto{\pgfqpoint{2.364131in}{0.477229in}}%
\pgfpathlineto{\pgfqpoint{2.376379in}{0.483846in}}%
\pgfpathlineto{\pgfqpoint{2.387917in}{0.483551in}}%
\pgfpathlineto{\pgfqpoint{2.397857in}{0.480555in}}%
\pgfpathlineto{\pgfqpoint{2.414187in}{0.476420in}}%
\pgfpathlineto{\pgfqpoint{2.427322in}{0.476316in}}%
\pgfpathlineto{\pgfqpoint{2.447202in}{0.479671in}}%
\pgfpathlineto{\pgfqpoint{2.464598in}{0.481525in}}%
\pgfpathlineto{\pgfqpoint{2.488206in}{0.479197in}}%
\pgfpathlineto{\pgfqpoint{2.503826in}{0.477498in}}%
\pgfpathlineto{\pgfqpoint{2.516073in}{0.477475in}}%
\pgfpathlineto{\pgfqpoint{2.531871in}{0.479282in}}%
\pgfpathlineto{\pgfqpoint{2.548379in}{0.480968in}}%
\pgfpathlineto{\pgfqpoint{2.568792in}{0.480299in}}%
\pgfpathlineto{\pgfqpoint{2.633935in}{0.479302in}}%
\pgfpathlineto{\pgfqpoint{2.836998in}{0.479384in}}%
\pgfpathlineto{\pgfqpoint{2.847471in}{0.479489in}}%
\pgfpathlineto{\pgfqpoint{2.860961in}{0.480299in}}%
\pgfpathlineto{\pgfqpoint{2.879421in}{0.480520in}}%
\pgfpathlineto{\pgfqpoint{2.927702in}{0.479298in}}%
\pgfpathlineto{\pgfqpoint{2.937820in}{0.479323in}}%
\pgfpathlineto{\pgfqpoint{2.952730in}{0.479658in}}%
\pgfpathlineto{\pgfqpoint{2.970658in}{0.480635in}}%
\pgfpathlineto{\pgfqpoint{3.047161in}{0.479768in}}%
\pgfpathlineto{\pgfqpoint{3.073609in}{0.481147in}}%
\pgfpathlineto{\pgfqpoint{3.166443in}{0.480583in}}%
\pgfpathlineto{\pgfqpoint{3.166443in}{0.480583in}}%
\pgfusepath{stroke}%
\end{pgfscope}%
\begin{pgfscope}%
\pgfpathrectangle{\pgfqpoint{0.504436in}{0.451277in}}{\pgfqpoint{2.662540in}{1.137880in}}%
\pgfusepath{clip}%
\pgfsetrectcap%
\pgfsetroundjoin%
\pgfsetlinewidth{1.505625pt}%
\definecolor{currentstroke}{rgb}{0.172549,0.627451,0.172549}%
\pgfsetstrokecolor{currentstroke}%
\pgfsetdash{}{0pt}%
\pgfpathmoveto{\pgfqpoint{0.504436in}{0.561748in}}%
\pgfpathlineto{\pgfqpoint{0.504613in}{0.476256in}}%
\pgfpathlineto{\pgfqpoint{0.505501in}{0.639428in}}%
\pgfpathlineto{\pgfqpoint{0.506743in}{0.861999in}}%
\pgfpathlineto{\pgfqpoint{0.507453in}{0.840452in}}%
\pgfpathlineto{\pgfqpoint{0.509406in}{0.792448in}}%
\pgfpathlineto{\pgfqpoint{0.509583in}{0.802907in}}%
\pgfpathlineto{\pgfqpoint{0.510471in}{0.777396in}}%
\pgfpathlineto{\pgfqpoint{0.510648in}{0.780774in}}%
\pgfpathlineto{\pgfqpoint{0.511891in}{0.764087in}}%
\pgfpathlineto{\pgfqpoint{0.512423in}{0.766952in}}%
\pgfpathlineto{\pgfqpoint{0.513133in}{0.770774in}}%
\pgfpathlineto{\pgfqpoint{0.514021in}{0.767403in}}%
\pgfpathlineto{\pgfqpoint{0.515796in}{0.763801in}}%
\pgfpathlineto{\pgfqpoint{0.516328in}{0.764278in}}%
\pgfpathlineto{\pgfqpoint{0.524849in}{0.770899in}}%
\pgfpathlineto{\pgfqpoint{0.529819in}{0.771377in}}%
\pgfpathlineto{\pgfqpoint{0.533901in}{0.780322in}}%
\pgfpathlineto{\pgfqpoint{0.534611in}{0.779499in}}%
\pgfpathlineto{\pgfqpoint{0.540291in}{0.768730in}}%
\pgfpathlineto{\pgfqpoint{0.545084in}{0.756866in}}%
\pgfpathlineto{\pgfqpoint{0.555201in}{0.738306in}}%
\pgfpathlineto{\pgfqpoint{0.559107in}{0.730799in}}%
\pgfpathlineto{\pgfqpoint{0.561769in}{0.730835in}}%
\pgfpathlineto{\pgfqpoint{0.565142in}{0.731866in}}%
\pgfpathlineto{\pgfqpoint{0.567449in}{0.730197in}}%
\pgfpathlineto{\pgfqpoint{0.576679in}{0.718007in}}%
\pgfpathlineto{\pgfqpoint{0.583957in}{0.717948in}}%
\pgfpathlineto{\pgfqpoint{0.589459in}{0.708750in}}%
\pgfpathlineto{\pgfqpoint{0.593897in}{0.704167in}}%
\pgfpathlineto{\pgfqpoint{0.599045in}{0.698773in}}%
\pgfpathlineto{\pgfqpoint{0.602595in}{0.697075in}}%
\pgfpathlineto{\pgfqpoint{0.607920in}{0.698243in}}%
\pgfpathlineto{\pgfqpoint{0.620522in}{0.707362in}}%
\pgfpathlineto{\pgfqpoint{0.624428in}{0.705973in}}%
\pgfpathlineto{\pgfqpoint{0.629930in}{0.704440in}}%
\pgfpathlineto{\pgfqpoint{0.634368in}{0.703611in}}%
\pgfpathlineto{\pgfqpoint{0.637740in}{0.700814in}}%
\pgfpathlineto{\pgfqpoint{0.641290in}{0.698595in}}%
\pgfpathlineto{\pgfqpoint{0.644663in}{0.698974in}}%
\pgfpathlineto{\pgfqpoint{0.656200in}{0.703563in}}%
\pgfpathlineto{\pgfqpoint{0.665431in}{0.702691in}}%
\pgfpathlineto{\pgfqpoint{0.681228in}{0.701742in}}%
\pgfpathlineto{\pgfqpoint{0.689393in}{0.698869in}}%
\pgfpathlineto{\pgfqpoint{0.699866in}{0.691237in}}%
\pgfpathlineto{\pgfqpoint{0.722054in}{0.694098in}}%
\pgfpathlineto{\pgfqpoint{0.726314in}{0.697454in}}%
\pgfpathlineto{\pgfqpoint{0.732882in}{0.702505in}}%
\pgfpathlineto{\pgfqpoint{0.736609in}{0.703171in}}%
\pgfpathlineto{\pgfqpoint{0.768027in}{0.696812in}}%
\pgfpathlineto{\pgfqpoint{0.778855in}{0.698386in}}%
\pgfpathlineto{\pgfqpoint{0.798558in}{0.694619in}}%
\pgfpathlineto{\pgfqpoint{0.811160in}{0.699058in}}%
\pgfpathlineto{\pgfqpoint{0.825893in}{0.698280in}}%
\pgfpathlineto{\pgfqpoint{0.829798in}{0.698210in}}%
\pgfpathlineto{\pgfqpoint{0.853228in}{0.704321in}}%
\pgfpathlineto{\pgfqpoint{0.864411in}{0.701272in}}%
\pgfpathlineto{\pgfqpoint{0.873641in}{0.697486in}}%
\pgfpathlineto{\pgfqpoint{0.883226in}{0.693360in}}%
\pgfpathlineto{\pgfqpoint{0.890327in}{0.689973in}}%
\pgfpathlineto{\pgfqpoint{0.897249in}{0.689542in}}%
\pgfpathlineto{\pgfqpoint{0.904704in}{0.692512in}}%
\pgfpathlineto{\pgfqpoint{0.909852in}{0.696610in}}%
\pgfpathlineto{\pgfqpoint{0.928667in}{0.708639in}}%
\pgfpathlineto{\pgfqpoint{0.933105in}{0.706695in}}%
\pgfpathlineto{\pgfqpoint{0.947305in}{0.698466in}}%
\pgfpathlineto{\pgfqpoint{0.959730in}{0.697405in}}%
\pgfpathlineto{\pgfqpoint{0.983870in}{0.709001in}}%
\pgfpathlineto{\pgfqpoint{0.988840in}{0.706553in}}%
\pgfpathlineto{\pgfqpoint{0.995941in}{0.703028in}}%
\pgfpathlineto{\pgfqpoint{1.004816in}{0.697930in}}%
\pgfpathlineto{\pgfqpoint{1.014933in}{0.691767in}}%
\pgfpathlineto{\pgfqpoint{1.026649in}{0.691023in}}%
\pgfpathlineto{\pgfqpoint{1.052919in}{0.702896in}}%
\pgfpathlineto{\pgfqpoint{1.066409in}{0.701417in}}%
\pgfpathlineto{\pgfqpoint{1.076349in}{0.695997in}}%
\pgfpathlineto{\pgfqpoint{1.081852in}{0.694688in}}%
\pgfpathlineto{\pgfqpoint{1.088774in}{0.694478in}}%
\pgfpathlineto{\pgfqpoint{1.098360in}{0.696386in}}%
\pgfpathlineto{\pgfqpoint{1.110962in}{0.699405in}}%
\pgfpathlineto{\pgfqpoint{1.122855in}{0.703692in}}%
\pgfpathlineto{\pgfqpoint{1.135813in}{0.705210in}}%
\pgfpathlineto{\pgfqpoint{1.145753in}{0.704463in}}%
\pgfpathlineto{\pgfqpoint{1.152498in}{0.702418in}}%
\pgfpathlineto{\pgfqpoint{1.179123in}{0.707203in}}%
\pgfpathlineto{\pgfqpoint{1.197051in}{0.702600in}}%
\pgfpathlineto{\pgfqpoint{1.206104in}{0.699256in}}%
\pgfpathlineto{\pgfqpoint{1.221014in}{0.696601in}}%
\pgfpathlineto{\pgfqpoint{1.233617in}{0.699430in}}%
\pgfpathlineto{\pgfqpoint{1.242847in}{0.703250in}}%
\pgfpathlineto{\pgfqpoint{1.254739in}{0.703485in}}%
\pgfpathlineto{\pgfqpoint{1.269117in}{0.699535in}}%
\pgfpathlineto{\pgfqpoint{1.276217in}{0.697484in}}%
\pgfpathlineto{\pgfqpoint{1.281720in}{0.699439in}}%
\pgfpathlineto{\pgfqpoint{1.295033in}{0.701232in}}%
\pgfpathlineto{\pgfqpoint{1.311895in}{0.700698in}}%
\pgfpathlineto{\pgfqpoint{1.327516in}{0.701601in}}%
\pgfpathlineto{\pgfqpoint{1.358046in}{0.705177in}}%
\pgfpathlineto{\pgfqpoint{1.377749in}{0.696698in}}%
\pgfpathlineto{\pgfqpoint{1.383784in}{0.697243in}}%
\pgfpathlineto{\pgfqpoint{1.391772in}{0.698766in}}%
\pgfpathlineto{\pgfqpoint{1.394257in}{0.667066in}}%
\pgfpathlineto{\pgfqpoint{1.399227in}{0.604665in}}%
\pgfpathlineto{\pgfqpoint{1.402599in}{0.595235in}}%
\pgfpathlineto{\pgfqpoint{1.405084in}{0.594999in}}%
\pgfpathlineto{\pgfqpoint{1.407037in}{0.599192in}}%
\pgfpathlineto{\pgfqpoint{1.409522in}{0.614012in}}%
\pgfpathlineto{\pgfqpoint{1.413072in}{0.657545in}}%
\pgfpathlineto{\pgfqpoint{1.418929in}{0.727270in}}%
\pgfpathlineto{\pgfqpoint{1.422302in}{0.738000in}}%
\pgfpathlineto{\pgfqpoint{1.425320in}{0.740044in}}%
\pgfpathlineto{\pgfqpoint{1.429580in}{0.739213in}}%
\pgfpathlineto{\pgfqpoint{1.432242in}{0.736901in}}%
\pgfpathlineto{\pgfqpoint{1.435082in}{0.730486in}}%
\pgfpathlineto{\pgfqpoint{1.439342in}{0.711430in}}%
\pgfpathlineto{\pgfqpoint{1.447152in}{0.659811in}}%
\pgfpathlineto{\pgfqpoint{1.453720in}{0.620955in}}%
\pgfpathlineto{\pgfqpoint{1.457448in}{0.612419in}}%
\pgfpathlineto{\pgfqpoint{1.460820in}{0.611376in}}%
\pgfpathlineto{\pgfqpoint{1.463660in}{0.613912in}}%
\pgfpathlineto{\pgfqpoint{1.468630in}{0.623318in}}%
\pgfpathlineto{\pgfqpoint{1.473600in}{0.639958in}}%
\pgfpathlineto{\pgfqpoint{1.486558in}{0.686466in}}%
\pgfpathlineto{\pgfqpoint{1.492948in}{0.698419in}}%
\pgfpathlineto{\pgfqpoint{1.496853in}{0.700999in}}%
\pgfpathlineto{\pgfqpoint{1.499338in}{0.699706in}}%
\pgfpathlineto{\pgfqpoint{1.503776in}{0.693430in}}%
\pgfpathlineto{\pgfqpoint{1.511231in}{0.677166in}}%
\pgfpathlineto{\pgfqpoint{1.521171in}{0.654212in}}%
\pgfpathlineto{\pgfqpoint{1.533419in}{0.640405in}}%
\pgfpathlineto{\pgfqpoint{1.538389in}{0.639694in}}%
\pgfpathlineto{\pgfqpoint{1.541939in}{0.641393in}}%
\pgfpathlineto{\pgfqpoint{1.549749in}{0.650431in}}%
\pgfpathlineto{\pgfqpoint{1.560222in}{0.664704in}}%
\pgfpathlineto{\pgfqpoint{1.571227in}{0.674798in}}%
\pgfpathlineto{\pgfqpoint{1.580989in}{0.679001in}}%
\pgfpathlineto{\pgfqpoint{1.600870in}{0.669288in}}%
\pgfpathlineto{\pgfqpoint{1.619685in}{0.656137in}}%
\pgfpathlineto{\pgfqpoint{1.644890in}{0.656649in}}%
\pgfpathlineto{\pgfqpoint{1.665836in}{0.667973in}}%
\pgfpathlineto{\pgfqpoint{1.693349in}{0.669001in}}%
\pgfpathlineto{\pgfqpoint{1.704709in}{0.665725in}}%
\pgfpathlineto{\pgfqpoint{1.721217in}{0.661689in}}%
\pgfpathlineto{\pgfqpoint{1.731157in}{0.660335in}}%
\pgfpathlineto{\pgfqpoint{1.735949in}{0.660676in}}%
\pgfpathlineto{\pgfqpoint{1.748729in}{0.661873in}}%
\pgfpathlineto{\pgfqpoint{1.756185in}{0.662107in}}%
\pgfpathlineto{\pgfqpoint{1.766480in}{0.665371in}}%
\pgfpathlineto{\pgfqpoint{1.820086in}{0.665753in}}%
\pgfpathlineto{\pgfqpoint{1.825233in}{0.664501in}}%
\pgfpathlineto{\pgfqpoint{1.839788in}{0.662902in}}%
\pgfpathlineto{\pgfqpoint{1.856829in}{0.659919in}}%
\pgfpathlineto{\pgfqpoint{1.867301in}{0.659972in}}%
\pgfpathlineto{\pgfqpoint{1.889667in}{0.668414in}}%
\pgfpathlineto{\pgfqpoint{1.898364in}{0.670343in}}%
\pgfpathlineto{\pgfqpoint{1.917712in}{0.670424in}}%
\pgfpathlineto{\pgfqpoint{1.923570in}{0.668075in}}%
\pgfpathlineto{\pgfqpoint{1.935462in}{0.665425in}}%
\pgfpathlineto{\pgfqpoint{1.952503in}{0.664172in}}%
\pgfpathlineto{\pgfqpoint{1.969543in}{0.669084in}}%
\pgfpathlineto{\pgfqpoint{1.990666in}{0.669101in}}%
\pgfpathlineto{\pgfqpoint{1.999541in}{0.666386in}}%
\pgfpathlineto{\pgfqpoint{2.011611in}{0.662769in}}%
\pgfpathlineto{\pgfqpoint{2.016226in}{0.661344in}}%
\pgfpathlineto{\pgfqpoint{2.030959in}{0.661608in}}%
\pgfpathlineto{\pgfqpoint{2.073914in}{0.671997in}}%
\pgfpathlineto{\pgfqpoint{2.080127in}{0.670632in}}%
\pgfpathlineto{\pgfqpoint{2.091132in}{0.666739in}}%
\pgfpathlineto{\pgfqpoint{2.101072in}{0.661990in}}%
\pgfpathlineto{\pgfqpoint{2.111012in}{0.660845in}}%
\pgfpathlineto{\pgfqpoint{2.117935in}{0.660606in}}%
\pgfpathlineto{\pgfqpoint{2.122195in}{0.661286in}}%
\pgfpathlineto{\pgfqpoint{2.140655in}{0.662923in}}%
\pgfpathlineto{\pgfqpoint{2.150418in}{0.665840in}}%
\pgfpathlineto{\pgfqpoint{2.158761in}{0.669457in}}%
\pgfpathlineto{\pgfqpoint{2.163908in}{0.669637in}}%
\pgfpathlineto{\pgfqpoint{2.172073in}{0.669399in}}%
\pgfpathlineto{\pgfqpoint{2.185386in}{0.669590in}}%
\pgfpathlineto{\pgfqpoint{2.195504in}{0.667320in}}%
\pgfpathlineto{\pgfqpoint{2.205266in}{0.665747in}}%
\pgfpathlineto{\pgfqpoint{2.232779in}{0.658055in}}%
\pgfpathlineto{\pgfqpoint{2.240057in}{0.659259in}}%
\pgfpathlineto{\pgfqpoint{2.249819in}{0.661552in}}%
\pgfpathlineto{\pgfqpoint{2.266150in}{0.667585in}}%
\pgfpathlineto{\pgfqpoint{2.278930in}{0.671313in}}%
\pgfpathlineto{\pgfqpoint{2.279285in}{0.670708in}}%
\pgfpathlineto{\pgfqpoint{2.279640in}{0.672714in}}%
\pgfpathlineto{\pgfqpoint{2.280527in}{0.674857in}}%
\pgfpathlineto{\pgfqpoint{2.281060in}{0.673334in}}%
\pgfpathlineto{\pgfqpoint{2.282657in}{0.649831in}}%
\pgfpathlineto{\pgfqpoint{2.285853in}{0.620051in}}%
\pgfpathlineto{\pgfqpoint{2.286918in}{0.620942in}}%
\pgfpathlineto{\pgfqpoint{2.288870in}{0.628949in}}%
\pgfpathlineto{\pgfqpoint{2.291355in}{0.653122in}}%
\pgfpathlineto{\pgfqpoint{2.293840in}{0.713725in}}%
\pgfpathlineto{\pgfqpoint{2.298278in}{0.937297in}}%
\pgfpathlineto{\pgfqpoint{2.301828in}{1.041935in}}%
\pgfpathlineto{\pgfqpoint{2.305910in}{1.074214in}}%
\pgfpathlineto{\pgfqpoint{2.311235in}{1.094236in}}%
\pgfpathlineto{\pgfqpoint{2.313720in}{1.097159in}}%
\pgfpathlineto{\pgfqpoint{2.317270in}{1.095648in}}%
\pgfpathlineto{\pgfqpoint{2.320821in}{1.084550in}}%
\pgfpathlineto{\pgfqpoint{2.324726in}{1.051037in}}%
\pgfpathlineto{\pgfqpoint{2.329873in}{0.971340in}}%
\pgfpathlineto{\pgfqpoint{2.340523in}{0.790935in}}%
\pgfpathlineto{\pgfqpoint{2.343896in}{0.775244in}}%
\pgfpathlineto{\pgfqpoint{2.346203in}{0.774587in}}%
\pgfpathlineto{\pgfqpoint{2.348156in}{0.777529in}}%
\pgfpathlineto{\pgfqpoint{2.354546in}{0.797311in}}%
\pgfpathlineto{\pgfqpoint{2.358806in}{0.822117in}}%
\pgfpathlineto{\pgfqpoint{2.363421in}{0.866756in}}%
\pgfpathlineto{\pgfqpoint{2.373184in}{0.964091in}}%
\pgfpathlineto{\pgfqpoint{2.377799in}{0.981467in}}%
\pgfpathlineto{\pgfqpoint{2.380994in}{0.984842in}}%
\pgfpathlineto{\pgfqpoint{2.384367in}{0.983108in}}%
\pgfpathlineto{\pgfqpoint{2.386674in}{0.977895in}}%
\pgfpathlineto{\pgfqpoint{2.391644in}{0.957550in}}%
\pgfpathlineto{\pgfqpoint{2.413122in}{0.861491in}}%
\pgfpathlineto{\pgfqpoint{2.418625in}{0.854735in}}%
\pgfpathlineto{\pgfqpoint{2.421642in}{0.854142in}}%
\pgfpathlineto{\pgfqpoint{2.428032in}{0.861314in}}%
\pgfpathlineto{\pgfqpoint{2.433890in}{0.875663in}}%
\pgfpathlineto{\pgfqpoint{2.445250in}{0.903986in}}%
\pgfpathlineto{\pgfqpoint{2.453238in}{0.919606in}}%
\pgfpathlineto{\pgfqpoint{2.461758in}{0.935338in}}%
\pgfpathlineto{\pgfqpoint{2.466195in}{0.939454in}}%
\pgfpathlineto{\pgfqpoint{2.469213in}{0.939803in}}%
\pgfpathlineto{\pgfqpoint{2.474183in}{0.935721in}}%
\pgfpathlineto{\pgfqpoint{2.480040in}{0.924562in}}%
\pgfpathlineto{\pgfqpoint{2.499743in}{0.877743in}}%
\pgfpathlineto{\pgfqpoint{2.506488in}{0.869961in}}%
\pgfpathlineto{\pgfqpoint{2.510038in}{0.869301in}}%
\pgfpathlineto{\pgfqpoint{2.514831in}{0.872136in}}%
\pgfpathlineto{\pgfqpoint{2.522996in}{0.886033in}}%
\pgfpathlineto{\pgfqpoint{2.530984in}{0.903988in}}%
\pgfpathlineto{\pgfqpoint{2.538084in}{0.919160in}}%
\pgfpathlineto{\pgfqpoint{2.543586in}{0.926555in}}%
\pgfpathlineto{\pgfqpoint{2.552817in}{0.932708in}}%
\pgfpathlineto{\pgfqpoint{2.562047in}{0.925604in}}%
\pgfpathlineto{\pgfqpoint{2.567194in}{0.922304in}}%
\pgfpathlineto{\pgfqpoint{2.571099in}{0.919993in}}%
\pgfpathlineto{\pgfqpoint{2.574472in}{0.916268in}}%
\pgfpathlineto{\pgfqpoint{2.578022in}{0.910643in}}%
\pgfpathlineto{\pgfqpoint{2.591335in}{0.894521in}}%
\pgfpathlineto{\pgfqpoint{2.594707in}{0.895077in}}%
\pgfpathlineto{\pgfqpoint{2.609972in}{0.898546in}}%
\pgfpathlineto{\pgfqpoint{2.614587in}{0.900241in}}%
\pgfpathlineto{\pgfqpoint{2.618848in}{0.902407in}}%
\pgfpathlineto{\pgfqpoint{2.643875in}{0.920954in}}%
\pgfpathlineto{\pgfqpoint{2.648135in}{0.923890in}}%
\pgfpathlineto{\pgfqpoint{2.652573in}{0.924277in}}%
\pgfpathlineto{\pgfqpoint{2.661981in}{0.920797in}}%
\pgfpathlineto{\pgfqpoint{2.673163in}{0.913999in}}%
\pgfpathlineto{\pgfqpoint{2.678133in}{0.913887in}}%
\pgfpathlineto{\pgfqpoint{2.680796in}{0.914001in}}%
\pgfpathlineto{\pgfqpoint{2.684524in}{0.913613in}}%
\pgfpathlineto{\pgfqpoint{2.689671in}{0.911903in}}%
\pgfpathlineto{\pgfqpoint{2.706179in}{0.903921in}}%
\pgfpathlineto{\pgfqpoint{2.709374in}{0.904649in}}%
\pgfpathlineto{\pgfqpoint{2.715942in}{0.906812in}}%
\pgfpathlineto{\pgfqpoint{2.720202in}{0.906961in}}%
\pgfpathlineto{\pgfqpoint{2.723752in}{0.908065in}}%
\pgfpathlineto{\pgfqpoint{2.736177in}{0.909413in}}%
\pgfpathlineto{\pgfqpoint{2.739017in}{0.909167in}}%
\pgfpathlineto{\pgfqpoint{2.755170in}{0.911006in}}%
\pgfpathlineto{\pgfqpoint{2.760495in}{0.912970in}}%
\pgfpathlineto{\pgfqpoint{2.763690in}{0.913714in}}%
\pgfpathlineto{\pgfqpoint{2.768482in}{0.916777in}}%
\pgfpathlineto{\pgfqpoint{2.783215in}{0.920625in}}%
\pgfpathlineto{\pgfqpoint{2.788718in}{0.919518in}}%
\pgfpathlineto{\pgfqpoint{2.794398in}{0.917922in}}%
\pgfpathlineto{\pgfqpoint{2.803805in}{0.919594in}}%
\pgfpathlineto{\pgfqpoint{2.808598in}{0.918818in}}%
\pgfpathlineto{\pgfqpoint{2.823863in}{0.912889in}}%
\pgfpathlineto{\pgfqpoint{2.827591in}{0.910819in}}%
\pgfpathlineto{\pgfqpoint{2.830253in}{0.909572in}}%
\pgfpathlineto{\pgfqpoint{2.835756in}{0.906923in}}%
\pgfpathlineto{\pgfqpoint{2.842501in}{0.907283in}}%
\pgfpathlineto{\pgfqpoint{2.877114in}{0.924873in}}%
\pgfpathlineto{\pgfqpoint{2.894509in}{0.918014in}}%
\pgfpathlineto{\pgfqpoint{2.900012in}{0.914073in}}%
\pgfpathlineto{\pgfqpoint{2.916520in}{0.906315in}}%
\pgfpathlineto{\pgfqpoint{2.925395in}{0.906989in}}%
\pgfpathlineto{\pgfqpoint{2.928057in}{0.906503in}}%
\pgfpathlineto{\pgfqpoint{2.932495in}{0.906056in}}%
\pgfpathlineto{\pgfqpoint{2.938352in}{0.906546in}}%
\pgfpathlineto{\pgfqpoint{2.941902in}{0.906747in}}%
\pgfpathlineto{\pgfqpoint{2.963380in}{0.922250in}}%
\pgfpathlineto{\pgfqpoint{2.969770in}{0.928445in}}%
\pgfpathlineto{\pgfqpoint{2.973498in}{0.928343in}}%
\pgfpathlineto{\pgfqpoint{2.986988in}{0.922500in}}%
\pgfpathlineto{\pgfqpoint{2.994798in}{0.918122in}}%
\pgfpathlineto{\pgfqpoint{3.000478in}{0.917577in}}%
\pgfpathlineto{\pgfqpoint{3.012193in}{0.912594in}}%
\pgfpathlineto{\pgfqpoint{3.016276in}{0.911401in}}%
\pgfpathlineto{\pgfqpoint{3.028524in}{0.909648in}}%
\pgfpathlineto{\pgfqpoint{3.032961in}{0.908272in}}%
\pgfpathlineto{\pgfqpoint{3.036866in}{0.907455in}}%
\pgfpathlineto{\pgfqpoint{3.045564in}{0.910073in}}%
\pgfpathlineto{\pgfqpoint{3.051777in}{0.913581in}}%
\pgfpathlineto{\pgfqpoint{3.058522in}{0.917442in}}%
\pgfpathlineto{\pgfqpoint{3.068284in}{0.924718in}}%
\pgfpathlineto{\pgfqpoint{3.073609in}{0.927756in}}%
\pgfpathlineto{\pgfqpoint{3.081952in}{0.927634in}}%
\pgfpathlineto{\pgfqpoint{3.085147in}{0.926742in}}%
\pgfpathlineto{\pgfqpoint{3.095087in}{0.921829in}}%
\pgfpathlineto{\pgfqpoint{3.100235in}{0.918697in}}%
\pgfpathlineto{\pgfqpoint{3.104317in}{0.917576in}}%
\pgfpathlineto{\pgfqpoint{3.112127in}{0.913248in}}%
\pgfpathlineto{\pgfqpoint{3.119760in}{0.910988in}}%
\pgfpathlineto{\pgfqpoint{3.146385in}{0.903628in}}%
\pgfpathlineto{\pgfqpoint{3.163781in}{0.912369in}}%
\pgfpathlineto{\pgfqpoint{3.166443in}{0.913694in}}%
\pgfpathlineto{\pgfqpoint{3.166443in}{0.913694in}}%
\pgfusepath{stroke}%
\end{pgfscope}%
\begin{pgfscope}%
\pgfsetrectcap%
\pgfsetmiterjoin%
\pgfsetlinewidth{0.803000pt}%
\definecolor{currentstroke}{rgb}{0.000000,0.000000,0.000000}%
\pgfsetstrokecolor{currentstroke}%
\pgfsetdash{}{0pt}%
\pgfpathmoveto{\pgfqpoint{0.504436in}{0.451277in}}%
\pgfpathlineto{\pgfqpoint{0.504436in}{1.589158in}}%
\pgfusepath{stroke}%
\end{pgfscope}%
\begin{pgfscope}%
\pgfsetrectcap%
\pgfsetmiterjoin%
\pgfsetlinewidth{0.803000pt}%
\definecolor{currentstroke}{rgb}{0.000000,0.000000,0.000000}%
\pgfsetstrokecolor{currentstroke}%
\pgfsetdash{}{0pt}%
\pgfpathmoveto{\pgfqpoint{3.166976in}{0.451277in}}%
\pgfpathlineto{\pgfqpoint{3.166976in}{1.589158in}}%
\pgfusepath{stroke}%
\end{pgfscope}%
\begin{pgfscope}%
\pgfsetrectcap%
\pgfsetmiterjoin%
\pgfsetlinewidth{0.803000pt}%
\definecolor{currentstroke}{rgb}{0.000000,0.000000,0.000000}%
\pgfsetstrokecolor{currentstroke}%
\pgfsetdash{}{0pt}%
\pgfpathmoveto{\pgfqpoint{0.504436in}{0.451277in}}%
\pgfpathlineto{\pgfqpoint{3.166976in}{0.451277in}}%
\pgfusepath{stroke}%
\end{pgfscope}%
\begin{pgfscope}%
\pgfsetrectcap%
\pgfsetmiterjoin%
\pgfsetlinewidth{0.803000pt}%
\definecolor{currentstroke}{rgb}{0.000000,0.000000,0.000000}%
\pgfsetstrokecolor{currentstroke}%
\pgfsetdash{}{0pt}%
\pgfpathmoveto{\pgfqpoint{0.504436in}{1.589158in}}%
\pgfpathlineto{\pgfqpoint{3.166976in}{1.589158in}}%
\pgfusepath{stroke}%
\end{pgfscope}%
\begin{pgfscope}%
\pgfsetbuttcap%
\pgfsetmiterjoin%
\definecolor{currentfill}{rgb}{1.000000,1.000000,1.000000}%
\pgfsetfillcolor{currentfill}%
\pgfsetfillopacity{0.800000}%
\pgfsetlinewidth{1.003750pt}%
\definecolor{currentstroke}{rgb}{0.800000,0.800000,0.800000}%
\pgfsetstrokecolor{currentstroke}%
\pgfsetstrokeopacity{0.800000}%
\pgfsetdash{}{0pt}%
\pgfpathmoveto{\pgfqpoint{1.432672in}{1.365546in}}%
\pgfpathlineto{\pgfqpoint{3.098920in}{1.365546in}}%
\pgfpathquadraticcurveto{\pgfqpoint{3.118365in}{1.365546in}}{\pgfqpoint{3.118365in}{1.384991in}}%
\pgfpathlineto{\pgfqpoint{3.118365in}{1.521102in}}%
\pgfpathquadraticcurveto{\pgfqpoint{3.118365in}{1.540546in}}{\pgfqpoint{3.098920in}{1.540546in}}%
\pgfpathlineto{\pgfqpoint{1.432672in}{1.540546in}}%
\pgfpathquadraticcurveto{\pgfqpoint{1.413228in}{1.540546in}}{\pgfqpoint{1.413228in}{1.521102in}}%
\pgfpathlineto{\pgfqpoint{1.413228in}{1.384991in}}%
\pgfpathquadraticcurveto{\pgfqpoint{1.413228in}{1.365546in}}{\pgfqpoint{1.432672in}{1.365546in}}%
\pgfpathlineto{\pgfqpoint{1.432672in}{1.365546in}}%
\pgfpathclose%
\pgfusepath{stroke,fill}%
\end{pgfscope}%
\begin{pgfscope}%
\pgfsetrectcap%
\pgfsetroundjoin%
\pgfsetlinewidth{1.505625pt}%
\definecolor{currentstroke}{rgb}{0.121569,0.466667,0.705882}%
\pgfsetstrokecolor{currentstroke}%
\pgfsetdash{}{0pt}%
\pgfpathmoveto{\pgfqpoint{1.452116in}{1.462769in}}%
\pgfpathlineto{\pgfqpoint{1.549339in}{1.462769in}}%
\pgfpathlineto{\pgfqpoint{1.646561in}{1.462769in}}%
\pgfusepath{stroke}%
\end{pgfscope}%
\begin{pgfscope}%
\definecolor{textcolor}{rgb}{0.000000,0.000000,0.000000}%
\pgfsetstrokecolor{textcolor}%
\pgfsetfillcolor{textcolor}%
\pgftext[x=1.724339in,y=1.428741in,left,base]{\color{textcolor}\rmfamily\fontsize{7.000000}{8.400000}\selectfont \(\displaystyle \|\mathbf{h}_0\|\)}%
\end{pgfscope}%
\begin{pgfscope}%
\pgfsetrectcap%
\pgfsetroundjoin%
\pgfsetlinewidth{1.505625pt}%
\definecolor{currentstroke}{rgb}{1.000000,0.498039,0.054902}%
\pgfsetstrokecolor{currentstroke}%
\pgfsetdash{}{0pt}%
\pgfpathmoveto{\pgfqpoint{2.010773in}{1.462769in}}%
\pgfpathlineto{\pgfqpoint{2.107996in}{1.462769in}}%
\pgfpathlineto{\pgfqpoint{2.205218in}{1.462769in}}%
\pgfusepath{stroke}%
\end{pgfscope}%
\begin{pgfscope}%
\definecolor{textcolor}{rgb}{0.000000,0.000000,0.000000}%
\pgfsetstrokecolor{textcolor}%
\pgfsetfillcolor{textcolor}%
\pgftext[x=2.282996in,y=1.428741in,left,base]{\color{textcolor}\rmfamily\fontsize{7.000000}{8.400000}\selectfont \(\displaystyle \|\mathbf{h}_1\|\)}%
\end{pgfscope}%
\begin{pgfscope}%
\pgfsetrectcap%
\pgfsetroundjoin%
\pgfsetlinewidth{1.505625pt}%
\definecolor{currentstroke}{rgb}{0.172549,0.627451,0.172549}%
\pgfsetstrokecolor{currentstroke}%
\pgfsetdash{}{0pt}%
\pgfpathmoveto{\pgfqpoint{2.569430in}{1.462769in}}%
\pgfpathlineto{\pgfqpoint{2.666652in}{1.462769in}}%
\pgfpathlineto{\pgfqpoint{2.763875in}{1.462769in}}%
\pgfusepath{stroke}%
\end{pgfscope}%
\begin{pgfscope}%
\definecolor{textcolor}{rgb}{0.000000,0.000000,0.000000}%
\pgfsetstrokecolor{textcolor}%
\pgfsetfillcolor{textcolor}%
\pgftext[x=2.841652in,y=1.428741in,left,base]{\color{textcolor}\rmfamily\fontsize{7.000000}{8.400000}\selectfont \(\displaystyle \|\mathbf{h}_2\|\)}%
\end{pgfscope}%
\end{pgfpicture}%
\makeatother%
\endgroup%
\\\vspace*{-1.0cm}
    %% Creator: Matplotlib, PGF backend
%%
%% To include the figure in your LaTeX document, write
%%   \input{<filename>.pgf}
%%
%% Make sure the required packages are loaded in your preamble
%%   \usepackage{pgf}
%%
%% Also ensure that all the required font packages are loaded; for instance,
%% the lmodern package is sometimes necessary when using math font.
%%   \usepackage{lmodern}
%%
%% Figures using additional raster images can only be included by \input if
%% they are in the same directory as the main LaTeX file. For loading figures
%% from other directories you can use the `import` package
%%   \usepackage{import}
%%
%% and then include the figures with
%%   \import{<path to file>}{<filename>.pgf}
%%
%% Matplotlib used the following preamble
%%   \usepackage{fontspec}
%%
\begingroup%
\makeatletter%
\begin{pgfpicture}%
\pgfpathrectangle{\pgfpointorigin}{\pgfqpoint{3.390065in}{1.356026in}}%
\pgfusepath{use as bounding box, clip}%
\begin{pgfscope}%
\pgfsetbuttcap%
\pgfsetmiterjoin%
\definecolor{currentfill}{rgb}{1.000000,1.000000,1.000000}%
\pgfsetfillcolor{currentfill}%
\pgfsetlinewidth{0.000000pt}%
\definecolor{currentstroke}{rgb}{1.000000,1.000000,1.000000}%
\pgfsetstrokecolor{currentstroke}%
\pgfsetstrokeopacity{0.000000}%
\pgfsetdash{}{0pt}%
\pgfpathmoveto{\pgfqpoint{0.000000in}{0.000000in}}%
\pgfpathlineto{\pgfqpoint{3.390065in}{0.000000in}}%
\pgfpathlineto{\pgfqpoint{3.390065in}{1.356026in}}%
\pgfpathlineto{\pgfqpoint{0.000000in}{1.356026in}}%
\pgfpathlineto{\pgfqpoint{0.000000in}{0.000000in}}%
\pgfpathclose%
\pgfusepath{fill}%
\end{pgfscope}%
\begin{pgfscope}%
\pgfsetbuttcap%
\pgfsetmiterjoin%
\definecolor{currentfill}{rgb}{1.000000,1.000000,1.000000}%
\pgfsetfillcolor{currentfill}%
\pgfsetlinewidth{0.000000pt}%
\definecolor{currentstroke}{rgb}{0.000000,0.000000,0.000000}%
\pgfsetstrokecolor{currentstroke}%
\pgfsetstrokeopacity{0.000000}%
\pgfsetdash{}{0pt}%
\pgfpathmoveto{\pgfqpoint{0.568671in}{0.451277in}}%
\pgfpathlineto{\pgfqpoint{3.166976in}{0.451277in}}%
\pgfpathlineto{\pgfqpoint{3.166976in}{1.250151in}}%
\pgfpathlineto{\pgfqpoint{0.568671in}{1.250151in}}%
\pgfpathlineto{\pgfqpoint{0.568671in}{0.451277in}}%
\pgfpathclose%
\pgfusepath{fill}%
\end{pgfscope}%
\begin{pgfscope}%
\pgfpathrectangle{\pgfqpoint{0.568671in}{0.451277in}}{\pgfqpoint{2.598305in}{0.798874in}}%
\pgfusepath{clip}%
\pgfsetrectcap%
\pgfsetroundjoin%
\pgfsetlinewidth{0.803000pt}%
\definecolor{currentstroke}{rgb}{0.690196,0.690196,0.690196}%
\pgfsetstrokecolor{currentstroke}%
\pgfsetdash{}{0pt}%
\pgfpathmoveto{\pgfqpoint{0.568671in}{0.451277in}}%
\pgfpathlineto{\pgfqpoint{0.568671in}{1.250151in}}%
\pgfusepath{stroke}%
\end{pgfscope}%
\begin{pgfscope}%
\pgfsetbuttcap%
\pgfsetroundjoin%
\definecolor{currentfill}{rgb}{0.000000,0.000000,0.000000}%
\pgfsetfillcolor{currentfill}%
\pgfsetlinewidth{0.803000pt}%
\definecolor{currentstroke}{rgb}{0.000000,0.000000,0.000000}%
\pgfsetstrokecolor{currentstroke}%
\pgfsetdash{}{0pt}%
\pgfsys@defobject{currentmarker}{\pgfqpoint{0.000000in}{-0.048611in}}{\pgfqpoint{0.000000in}{0.000000in}}{%
\pgfpathmoveto{\pgfqpoint{0.000000in}{0.000000in}}%
\pgfpathlineto{\pgfqpoint{0.000000in}{-0.048611in}}%
\pgfusepath{stroke,fill}%
}%
\begin{pgfscope}%
\pgfsys@transformshift{0.568671in}{0.451277in}%
\pgfsys@useobject{currentmarker}{}%
\end{pgfscope}%
\end{pgfscope}%
\begin{pgfscope}%
\definecolor{textcolor}{rgb}{0.000000,0.000000,0.000000}%
\pgfsetstrokecolor{textcolor}%
\pgfsetfillcolor{textcolor}%
\pgftext[x=0.568671in,y=0.354055in,,top]{\color{textcolor}\rmfamily\fontsize{9.000000}{10.800000}\selectfont \(\displaystyle {0}\)}%
\end{pgfscope}%
\begin{pgfscope}%
\pgfpathrectangle{\pgfqpoint{0.568671in}{0.451277in}}{\pgfqpoint{2.598305in}{0.798874in}}%
\pgfusepath{clip}%
\pgfsetrectcap%
\pgfsetroundjoin%
\pgfsetlinewidth{0.803000pt}%
\definecolor{currentstroke}{rgb}{0.690196,0.690196,0.690196}%
\pgfsetstrokecolor{currentstroke}%
\pgfsetdash{}{0pt}%
\pgfpathmoveto{\pgfqpoint{1.001722in}{0.451277in}}%
\pgfpathlineto{\pgfqpoint{1.001722in}{1.250151in}}%
\pgfusepath{stroke}%
\end{pgfscope}%
\begin{pgfscope}%
\pgfsetbuttcap%
\pgfsetroundjoin%
\definecolor{currentfill}{rgb}{0.000000,0.000000,0.000000}%
\pgfsetfillcolor{currentfill}%
\pgfsetlinewidth{0.803000pt}%
\definecolor{currentstroke}{rgb}{0.000000,0.000000,0.000000}%
\pgfsetstrokecolor{currentstroke}%
\pgfsetdash{}{0pt}%
\pgfsys@defobject{currentmarker}{\pgfqpoint{0.000000in}{-0.048611in}}{\pgfqpoint{0.000000in}{0.000000in}}{%
\pgfpathmoveto{\pgfqpoint{0.000000in}{0.000000in}}%
\pgfpathlineto{\pgfqpoint{0.000000in}{-0.048611in}}%
\pgfusepath{stroke,fill}%
}%
\begin{pgfscope}%
\pgfsys@transformshift{1.001722in}{0.451277in}%
\pgfsys@useobject{currentmarker}{}%
\end{pgfscope}%
\end{pgfscope}%
\begin{pgfscope}%
\definecolor{textcolor}{rgb}{0.000000,0.000000,0.000000}%
\pgfsetstrokecolor{textcolor}%
\pgfsetfillcolor{textcolor}%
\pgftext[x=1.001722in,y=0.354055in,,top]{\color{textcolor}\rmfamily\fontsize{9.000000}{10.800000}\selectfont \(\displaystyle {2500}\)}%
\end{pgfscope}%
\begin{pgfscope}%
\pgfpathrectangle{\pgfqpoint{0.568671in}{0.451277in}}{\pgfqpoint{2.598305in}{0.798874in}}%
\pgfusepath{clip}%
\pgfsetrectcap%
\pgfsetroundjoin%
\pgfsetlinewidth{0.803000pt}%
\definecolor{currentstroke}{rgb}{0.690196,0.690196,0.690196}%
\pgfsetstrokecolor{currentstroke}%
\pgfsetdash{}{0pt}%
\pgfpathmoveto{\pgfqpoint{1.434773in}{0.451277in}}%
\pgfpathlineto{\pgfqpoint{1.434773in}{1.250151in}}%
\pgfusepath{stroke}%
\end{pgfscope}%
\begin{pgfscope}%
\pgfsetbuttcap%
\pgfsetroundjoin%
\definecolor{currentfill}{rgb}{0.000000,0.000000,0.000000}%
\pgfsetfillcolor{currentfill}%
\pgfsetlinewidth{0.803000pt}%
\definecolor{currentstroke}{rgb}{0.000000,0.000000,0.000000}%
\pgfsetstrokecolor{currentstroke}%
\pgfsetdash{}{0pt}%
\pgfsys@defobject{currentmarker}{\pgfqpoint{0.000000in}{-0.048611in}}{\pgfqpoint{0.000000in}{0.000000in}}{%
\pgfpathmoveto{\pgfqpoint{0.000000in}{0.000000in}}%
\pgfpathlineto{\pgfqpoint{0.000000in}{-0.048611in}}%
\pgfusepath{stroke,fill}%
}%
\begin{pgfscope}%
\pgfsys@transformshift{1.434773in}{0.451277in}%
\pgfsys@useobject{currentmarker}{}%
\end{pgfscope}%
\end{pgfscope}%
\begin{pgfscope}%
\definecolor{textcolor}{rgb}{0.000000,0.000000,0.000000}%
\pgfsetstrokecolor{textcolor}%
\pgfsetfillcolor{textcolor}%
\pgftext[x=1.434773in,y=0.354055in,,top]{\color{textcolor}\rmfamily\fontsize{9.000000}{10.800000}\selectfont \(\displaystyle {5000}\)}%
\end{pgfscope}%
\begin{pgfscope}%
\pgfpathrectangle{\pgfqpoint{0.568671in}{0.451277in}}{\pgfqpoint{2.598305in}{0.798874in}}%
\pgfusepath{clip}%
\pgfsetrectcap%
\pgfsetroundjoin%
\pgfsetlinewidth{0.803000pt}%
\definecolor{currentstroke}{rgb}{0.690196,0.690196,0.690196}%
\pgfsetstrokecolor{currentstroke}%
\pgfsetdash{}{0pt}%
\pgfpathmoveto{\pgfqpoint{1.867823in}{0.451277in}}%
\pgfpathlineto{\pgfqpoint{1.867823in}{1.250151in}}%
\pgfusepath{stroke}%
\end{pgfscope}%
\begin{pgfscope}%
\pgfsetbuttcap%
\pgfsetroundjoin%
\definecolor{currentfill}{rgb}{0.000000,0.000000,0.000000}%
\pgfsetfillcolor{currentfill}%
\pgfsetlinewidth{0.803000pt}%
\definecolor{currentstroke}{rgb}{0.000000,0.000000,0.000000}%
\pgfsetstrokecolor{currentstroke}%
\pgfsetdash{}{0pt}%
\pgfsys@defobject{currentmarker}{\pgfqpoint{0.000000in}{-0.048611in}}{\pgfqpoint{0.000000in}{0.000000in}}{%
\pgfpathmoveto{\pgfqpoint{0.000000in}{0.000000in}}%
\pgfpathlineto{\pgfqpoint{0.000000in}{-0.048611in}}%
\pgfusepath{stroke,fill}%
}%
\begin{pgfscope}%
\pgfsys@transformshift{1.867823in}{0.451277in}%
\pgfsys@useobject{currentmarker}{}%
\end{pgfscope}%
\end{pgfscope}%
\begin{pgfscope}%
\definecolor{textcolor}{rgb}{0.000000,0.000000,0.000000}%
\pgfsetstrokecolor{textcolor}%
\pgfsetfillcolor{textcolor}%
\pgftext[x=1.867823in,y=0.354055in,,top]{\color{textcolor}\rmfamily\fontsize{9.000000}{10.800000}\selectfont \(\displaystyle {7500}\)}%
\end{pgfscope}%
\begin{pgfscope}%
\pgfpathrectangle{\pgfqpoint{0.568671in}{0.451277in}}{\pgfqpoint{2.598305in}{0.798874in}}%
\pgfusepath{clip}%
\pgfsetrectcap%
\pgfsetroundjoin%
\pgfsetlinewidth{0.803000pt}%
\definecolor{currentstroke}{rgb}{0.690196,0.690196,0.690196}%
\pgfsetstrokecolor{currentstroke}%
\pgfsetdash{}{0pt}%
\pgfpathmoveto{\pgfqpoint{2.300874in}{0.451277in}}%
\pgfpathlineto{\pgfqpoint{2.300874in}{1.250151in}}%
\pgfusepath{stroke}%
\end{pgfscope}%
\begin{pgfscope}%
\pgfsetbuttcap%
\pgfsetroundjoin%
\definecolor{currentfill}{rgb}{0.000000,0.000000,0.000000}%
\pgfsetfillcolor{currentfill}%
\pgfsetlinewidth{0.803000pt}%
\definecolor{currentstroke}{rgb}{0.000000,0.000000,0.000000}%
\pgfsetstrokecolor{currentstroke}%
\pgfsetdash{}{0pt}%
\pgfsys@defobject{currentmarker}{\pgfqpoint{0.000000in}{-0.048611in}}{\pgfqpoint{0.000000in}{0.000000in}}{%
\pgfpathmoveto{\pgfqpoint{0.000000in}{0.000000in}}%
\pgfpathlineto{\pgfqpoint{0.000000in}{-0.048611in}}%
\pgfusepath{stroke,fill}%
}%
\begin{pgfscope}%
\pgfsys@transformshift{2.300874in}{0.451277in}%
\pgfsys@useobject{currentmarker}{}%
\end{pgfscope}%
\end{pgfscope}%
\begin{pgfscope}%
\definecolor{textcolor}{rgb}{0.000000,0.000000,0.000000}%
\pgfsetstrokecolor{textcolor}%
\pgfsetfillcolor{textcolor}%
\pgftext[x=2.300874in,y=0.354055in,,top]{\color{textcolor}\rmfamily\fontsize{9.000000}{10.800000}\selectfont \(\displaystyle {10000}\)}%
\end{pgfscope}%
\begin{pgfscope}%
\pgfpathrectangle{\pgfqpoint{0.568671in}{0.451277in}}{\pgfqpoint{2.598305in}{0.798874in}}%
\pgfusepath{clip}%
\pgfsetrectcap%
\pgfsetroundjoin%
\pgfsetlinewidth{0.803000pt}%
\definecolor{currentstroke}{rgb}{0.690196,0.690196,0.690196}%
\pgfsetstrokecolor{currentstroke}%
\pgfsetdash{}{0pt}%
\pgfpathmoveto{\pgfqpoint{2.733925in}{0.451277in}}%
\pgfpathlineto{\pgfqpoint{2.733925in}{1.250151in}}%
\pgfusepath{stroke}%
\end{pgfscope}%
\begin{pgfscope}%
\pgfsetbuttcap%
\pgfsetroundjoin%
\definecolor{currentfill}{rgb}{0.000000,0.000000,0.000000}%
\pgfsetfillcolor{currentfill}%
\pgfsetlinewidth{0.803000pt}%
\definecolor{currentstroke}{rgb}{0.000000,0.000000,0.000000}%
\pgfsetstrokecolor{currentstroke}%
\pgfsetdash{}{0pt}%
\pgfsys@defobject{currentmarker}{\pgfqpoint{0.000000in}{-0.048611in}}{\pgfqpoint{0.000000in}{0.000000in}}{%
\pgfpathmoveto{\pgfqpoint{0.000000in}{0.000000in}}%
\pgfpathlineto{\pgfqpoint{0.000000in}{-0.048611in}}%
\pgfusepath{stroke,fill}%
}%
\begin{pgfscope}%
\pgfsys@transformshift{2.733925in}{0.451277in}%
\pgfsys@useobject{currentmarker}{}%
\end{pgfscope}%
\end{pgfscope}%
\begin{pgfscope}%
\definecolor{textcolor}{rgb}{0.000000,0.000000,0.000000}%
\pgfsetstrokecolor{textcolor}%
\pgfsetfillcolor{textcolor}%
\pgftext[x=2.733925in,y=0.354055in,,top]{\color{textcolor}\rmfamily\fontsize{9.000000}{10.800000}\selectfont \(\displaystyle {12500}\)}%
\end{pgfscope}%
\begin{pgfscope}%
\pgfpathrectangle{\pgfqpoint{0.568671in}{0.451277in}}{\pgfqpoint{2.598305in}{0.798874in}}%
\pgfusepath{clip}%
\pgfsetrectcap%
\pgfsetroundjoin%
\pgfsetlinewidth{0.803000pt}%
\definecolor{currentstroke}{rgb}{0.690196,0.690196,0.690196}%
\pgfsetstrokecolor{currentstroke}%
\pgfsetdash{}{0pt}%
\pgfpathmoveto{\pgfqpoint{3.166976in}{0.451277in}}%
\pgfpathlineto{\pgfqpoint{3.166976in}{1.250151in}}%
\pgfusepath{stroke}%
\end{pgfscope}%
\begin{pgfscope}%
\pgfsetbuttcap%
\pgfsetroundjoin%
\definecolor{currentfill}{rgb}{0.000000,0.000000,0.000000}%
\pgfsetfillcolor{currentfill}%
\pgfsetlinewidth{0.803000pt}%
\definecolor{currentstroke}{rgb}{0.000000,0.000000,0.000000}%
\pgfsetstrokecolor{currentstroke}%
\pgfsetdash{}{0pt}%
\pgfsys@defobject{currentmarker}{\pgfqpoint{0.000000in}{-0.048611in}}{\pgfqpoint{0.000000in}{0.000000in}}{%
\pgfpathmoveto{\pgfqpoint{0.000000in}{0.000000in}}%
\pgfpathlineto{\pgfqpoint{0.000000in}{-0.048611in}}%
\pgfusepath{stroke,fill}%
}%
\begin{pgfscope}%
\pgfsys@transformshift{3.166976in}{0.451277in}%
\pgfsys@useobject{currentmarker}{}%
\end{pgfscope}%
\end{pgfscope}%
\begin{pgfscope}%
\definecolor{textcolor}{rgb}{0.000000,0.000000,0.000000}%
\pgfsetstrokecolor{textcolor}%
\pgfsetfillcolor{textcolor}%
\pgftext[x=3.166976in,y=0.354055in,,top]{\color{textcolor}\rmfamily\fontsize{9.000000}{10.800000}\selectfont \(\displaystyle {15000}\)}%
\end{pgfscope}%
\begin{pgfscope}%
\definecolor{textcolor}{rgb}{0.000000,0.000000,0.000000}%
\pgfsetstrokecolor{textcolor}%
\pgfsetfillcolor{textcolor}%
\pgftext[x=1.867823in,y=0.187500in,,top]{\color{textcolor}\rmfamily\fontsize{9.000000}{10.800000}\selectfont Time [frames]}%
\end{pgfscope}%
\begin{pgfscope}%
\pgfpathrectangle{\pgfqpoint{0.568671in}{0.451277in}}{\pgfqpoint{2.598305in}{0.798874in}}%
\pgfusepath{clip}%
\pgfsetrectcap%
\pgfsetroundjoin%
\pgfsetlinewidth{0.803000pt}%
\definecolor{currentstroke}{rgb}{0.690196,0.690196,0.690196}%
\pgfsetstrokecolor{currentstroke}%
\pgfsetdash{}{0pt}%
\pgfpathmoveto{\pgfqpoint{0.568671in}{0.451277in}}%
\pgfpathlineto{\pgfqpoint{3.166976in}{0.451277in}}%
\pgfusepath{stroke}%
\end{pgfscope}%
\begin{pgfscope}%
\pgfsetbuttcap%
\pgfsetroundjoin%
\definecolor{currentfill}{rgb}{0.000000,0.000000,0.000000}%
\pgfsetfillcolor{currentfill}%
\pgfsetlinewidth{0.803000pt}%
\definecolor{currentstroke}{rgb}{0.000000,0.000000,0.000000}%
\pgfsetstrokecolor{currentstroke}%
\pgfsetdash{}{0pt}%
\pgfsys@defobject{currentmarker}{\pgfqpoint{-0.048611in}{0.000000in}}{\pgfqpoint{-0.000000in}{0.000000in}}{%
\pgfpathmoveto{\pgfqpoint{-0.000000in}{0.000000in}}%
\pgfpathlineto{\pgfqpoint{-0.048611in}{0.000000in}}%
\pgfusepath{stroke,fill}%
}%
\begin{pgfscope}%
\pgfsys@transformshift{0.568671in}{0.451277in}%
\pgfsys@useobject{currentmarker}{}%
\end{pgfscope}%
\end{pgfscope}%
\begin{pgfscope}%
\definecolor{textcolor}{rgb}{0.000000,0.000000,0.000000}%
\pgfsetstrokecolor{textcolor}%
\pgfsetfillcolor{textcolor}%
\pgftext[x=0.243055in, y=0.407902in, left, base]{\color{textcolor}\rmfamily\fontsize{9.000000}{10.800000}\selectfont \(\displaystyle {\ensuremath{-}30}\)}%
\end{pgfscope}%
\begin{pgfscope}%
\pgfpathrectangle{\pgfqpoint{0.568671in}{0.451277in}}{\pgfqpoint{2.598305in}{0.798874in}}%
\pgfusepath{clip}%
\pgfsetrectcap%
\pgfsetroundjoin%
\pgfsetlinewidth{0.803000pt}%
\definecolor{currentstroke}{rgb}{0.690196,0.690196,0.690196}%
\pgfsetstrokecolor{currentstroke}%
\pgfsetdash{}{0pt}%
\pgfpathmoveto{\pgfqpoint{0.568671in}{0.717569in}}%
\pgfpathlineto{\pgfqpoint{3.166976in}{0.717569in}}%
\pgfusepath{stroke}%
\end{pgfscope}%
\begin{pgfscope}%
\pgfsetbuttcap%
\pgfsetroundjoin%
\definecolor{currentfill}{rgb}{0.000000,0.000000,0.000000}%
\pgfsetfillcolor{currentfill}%
\pgfsetlinewidth{0.803000pt}%
\definecolor{currentstroke}{rgb}{0.000000,0.000000,0.000000}%
\pgfsetstrokecolor{currentstroke}%
\pgfsetdash{}{0pt}%
\pgfsys@defobject{currentmarker}{\pgfqpoint{-0.048611in}{0.000000in}}{\pgfqpoint{-0.000000in}{0.000000in}}{%
\pgfpathmoveto{\pgfqpoint{-0.000000in}{0.000000in}}%
\pgfpathlineto{\pgfqpoint{-0.048611in}{0.000000in}}%
\pgfusepath{stroke,fill}%
}%
\begin{pgfscope}%
\pgfsys@transformshift{0.568671in}{0.717569in}%
\pgfsys@useobject{currentmarker}{}%
\end{pgfscope}%
\end{pgfscope}%
\begin{pgfscope}%
\definecolor{textcolor}{rgb}{0.000000,0.000000,0.000000}%
\pgfsetstrokecolor{textcolor}%
\pgfsetfillcolor{textcolor}%
\pgftext[x=0.243055in, y=0.674194in, left, base]{\color{textcolor}\rmfamily\fontsize{9.000000}{10.800000}\selectfont \(\displaystyle {\ensuremath{-}20}\)}%
\end{pgfscope}%
\begin{pgfscope}%
\pgfpathrectangle{\pgfqpoint{0.568671in}{0.451277in}}{\pgfqpoint{2.598305in}{0.798874in}}%
\pgfusepath{clip}%
\pgfsetrectcap%
\pgfsetroundjoin%
\pgfsetlinewidth{0.803000pt}%
\definecolor{currentstroke}{rgb}{0.690196,0.690196,0.690196}%
\pgfsetstrokecolor{currentstroke}%
\pgfsetdash{}{0pt}%
\pgfpathmoveto{\pgfqpoint{0.568671in}{0.983860in}}%
\pgfpathlineto{\pgfqpoint{3.166976in}{0.983860in}}%
\pgfusepath{stroke}%
\end{pgfscope}%
\begin{pgfscope}%
\pgfsetbuttcap%
\pgfsetroundjoin%
\definecolor{currentfill}{rgb}{0.000000,0.000000,0.000000}%
\pgfsetfillcolor{currentfill}%
\pgfsetlinewidth{0.803000pt}%
\definecolor{currentstroke}{rgb}{0.000000,0.000000,0.000000}%
\pgfsetstrokecolor{currentstroke}%
\pgfsetdash{}{0pt}%
\pgfsys@defobject{currentmarker}{\pgfqpoint{-0.048611in}{0.000000in}}{\pgfqpoint{-0.000000in}{0.000000in}}{%
\pgfpathmoveto{\pgfqpoint{-0.000000in}{0.000000in}}%
\pgfpathlineto{\pgfqpoint{-0.048611in}{0.000000in}}%
\pgfusepath{stroke,fill}%
}%
\begin{pgfscope}%
\pgfsys@transformshift{0.568671in}{0.983860in}%
\pgfsys@useobject{currentmarker}{}%
\end{pgfscope}%
\end{pgfscope}%
\begin{pgfscope}%
\definecolor{textcolor}{rgb}{0.000000,0.000000,0.000000}%
\pgfsetstrokecolor{textcolor}%
\pgfsetfillcolor{textcolor}%
\pgftext[x=0.243055in, y=0.940485in, left, base]{\color{textcolor}\rmfamily\fontsize{9.000000}{10.800000}\selectfont \(\displaystyle {\ensuremath{-}10}\)}%
\end{pgfscope}%
\begin{pgfscope}%
\pgfpathrectangle{\pgfqpoint{0.568671in}{0.451277in}}{\pgfqpoint{2.598305in}{0.798874in}}%
\pgfusepath{clip}%
\pgfsetrectcap%
\pgfsetroundjoin%
\pgfsetlinewidth{0.803000pt}%
\definecolor{currentstroke}{rgb}{0.690196,0.690196,0.690196}%
\pgfsetstrokecolor{currentstroke}%
\pgfsetdash{}{0pt}%
\pgfpathmoveto{\pgfqpoint{0.568671in}{1.250151in}}%
\pgfpathlineto{\pgfqpoint{3.166976in}{1.250151in}}%
\pgfusepath{stroke}%
\end{pgfscope}%
\begin{pgfscope}%
\pgfsetbuttcap%
\pgfsetroundjoin%
\definecolor{currentfill}{rgb}{0.000000,0.000000,0.000000}%
\pgfsetfillcolor{currentfill}%
\pgfsetlinewidth{0.803000pt}%
\definecolor{currentstroke}{rgb}{0.000000,0.000000,0.000000}%
\pgfsetstrokecolor{currentstroke}%
\pgfsetdash{}{0pt}%
\pgfsys@defobject{currentmarker}{\pgfqpoint{-0.048611in}{0.000000in}}{\pgfqpoint{-0.000000in}{0.000000in}}{%
\pgfpathmoveto{\pgfqpoint{-0.000000in}{0.000000in}}%
\pgfpathlineto{\pgfqpoint{-0.048611in}{0.000000in}}%
\pgfusepath{stroke,fill}%
}%
\begin{pgfscope}%
\pgfsys@transformshift{0.568671in}{1.250151in}%
\pgfsys@useobject{currentmarker}{}%
\end{pgfscope}%
\end{pgfscope}%
\begin{pgfscope}%
\definecolor{textcolor}{rgb}{0.000000,0.000000,0.000000}%
\pgfsetstrokecolor{textcolor}%
\pgfsetfillcolor{textcolor}%
\pgftext[x=0.407213in, y=1.206776in, left, base]{\color{textcolor}\rmfamily\fontsize{9.000000}{10.800000}\selectfont \(\displaystyle {0}\)}%
\end{pgfscope}%
\begin{pgfscope}%
\definecolor{textcolor}{rgb}{0.000000,0.000000,0.000000}%
\pgfsetstrokecolor{textcolor}%
\pgfsetfillcolor{textcolor}%
\pgftext[x=0.187500in,y=0.850714in,,bottom,rotate=90.000000]{\color{textcolor}\rmfamily\fontsize{9.000000}{10.800000}\selectfont NPM [dB]}%
\end{pgfscope}%
\begin{pgfscope}%
\pgfpathrectangle{\pgfqpoint{0.568671in}{0.451277in}}{\pgfqpoint{2.598305in}{0.798874in}}%
\pgfusepath{clip}%
\pgfsetbuttcap%
\pgfsetroundjoin%
\pgfsetlinewidth{0.752812pt}%
\definecolor{currentstroke}{rgb}{0.000000,0.000000,0.000000}%
\pgfsetstrokecolor{currentstroke}%
\pgfsetdash{{2.775000pt}{1.200000pt}}{0.000000pt}%
\pgfpathmoveto{\pgfqpoint{1.434773in}{0.451277in}}%
\pgfpathlineto{\pgfqpoint{1.434773in}{1.250151in}}%
\pgfusepath{stroke}%
\end{pgfscope}%
\begin{pgfscope}%
\pgfpathrectangle{\pgfqpoint{0.568671in}{0.451277in}}{\pgfqpoint{2.598305in}{0.798874in}}%
\pgfusepath{clip}%
\pgfsetbuttcap%
\pgfsetroundjoin%
\pgfsetlinewidth{0.752812pt}%
\definecolor{currentstroke}{rgb}{0.000000,0.000000,0.000000}%
\pgfsetstrokecolor{currentstroke}%
\pgfsetdash{{2.775000pt}{1.200000pt}}{0.000000pt}%
\pgfpathmoveto{\pgfqpoint{2.300874in}{0.451277in}}%
\pgfpathlineto{\pgfqpoint{2.300874in}{1.250151in}}%
\pgfusepath{stroke}%
\end{pgfscope}%
\begin{pgfscope}%
\pgfpathrectangle{\pgfqpoint{0.568671in}{0.451277in}}{\pgfqpoint{2.598305in}{0.798874in}}%
\pgfusepath{clip}%
\pgfsetrectcap%
\pgfsetroundjoin%
\pgfsetlinewidth{1.505625pt}%
\definecolor{currentstroke}{rgb}{0.000000,0.000000,0.000000}%
\pgfsetstrokecolor{currentstroke}%
\pgfsetstrokeopacity{0.250000}%
\pgfsetdash{}{0pt}%
\pgfpathmoveto{\pgfqpoint{0.649919in}{1.264040in}}%
\pgfpathlineto{\pgfqpoint{0.660305in}{1.248630in}}%
\pgfpathlineto{\pgfqpoint{0.663423in}{1.244240in}}%
\pgfpathlineto{\pgfqpoint{0.695468in}{1.195539in}}%
\pgfpathlineto{\pgfqpoint{0.697547in}{1.191936in}}%
\pgfpathlineto{\pgfqpoint{0.708287in}{1.172979in}}%
\pgfpathlineto{\pgfqpoint{0.716082in}{1.160009in}}%
\pgfpathlineto{\pgfqpoint{0.721798in}{1.148995in}}%
\pgfpathlineto{\pgfqpoint{0.727341in}{1.137863in}}%
\pgfpathlineto{\pgfqpoint{0.738254in}{1.110183in}}%
\pgfpathlineto{\pgfqpoint{0.743970in}{1.094235in}}%
\pgfpathlineto{\pgfqpoint{0.751245in}{1.077274in}}%
\pgfpathlineto{\pgfqpoint{0.771512in}{1.053856in}}%
\pgfpathlineto{\pgfqpoint{0.772898in}{1.053275in}}%
\pgfpathlineto{\pgfqpoint{0.780000in}{1.049584in}}%
\pgfpathlineto{\pgfqpoint{0.781732in}{1.052204in}}%
\pgfpathlineto{\pgfqpoint{0.782252in}{1.051557in}}%
\pgfpathlineto{\pgfqpoint{0.789181in}{1.046978in}}%
\pgfpathlineto{\pgfqpoint{0.789354in}{1.047230in}}%
\pgfpathlineto{\pgfqpoint{0.791779in}{1.048430in}}%
\pgfpathlineto{\pgfqpoint{0.793858in}{1.048705in}}%
\pgfpathlineto{\pgfqpoint{0.794897in}{1.048559in}}%
\pgfpathlineto{\pgfqpoint{0.795243in}{1.048013in}}%
\pgfpathlineto{\pgfqpoint{0.799747in}{1.045290in}}%
\pgfpathlineto{\pgfqpoint{0.804424in}{1.045427in}}%
\pgfpathlineto{\pgfqpoint{0.809621in}{1.041922in}}%
\pgfpathlineto{\pgfqpoint{0.814124in}{1.034185in}}%
\pgfpathlineto{\pgfqpoint{0.822959in}{1.029774in}}%
\pgfpathlineto{\pgfqpoint{0.825037in}{1.028074in}}%
\pgfpathlineto{\pgfqpoint{0.826596in}{1.028176in}}%
\pgfpathlineto{\pgfqpoint{0.826769in}{1.027904in}}%
\pgfpathlineto{\pgfqpoint{0.829021in}{1.027202in}}%
\pgfpathlineto{\pgfqpoint{0.830407in}{1.026343in}}%
\pgfpathlineto{\pgfqpoint{0.840800in}{1.017476in}}%
\pgfpathlineto{\pgfqpoint{0.844438in}{1.014088in}}%
\pgfpathlineto{\pgfqpoint{0.846863in}{1.009864in}}%
\pgfpathlineto{\pgfqpoint{0.848075in}{1.009667in}}%
\pgfpathlineto{\pgfqpoint{0.848422in}{1.009892in}}%
\pgfpathlineto{\pgfqpoint{0.849461in}{1.008985in}}%
\pgfpathlineto{\pgfqpoint{0.852406in}{1.006150in}}%
\pgfpathlineto{\pgfqpoint{0.853792in}{1.004496in}}%
\pgfpathlineto{\pgfqpoint{0.857949in}{0.999414in}}%
\pgfpathlineto{\pgfqpoint{0.859335in}{1.000124in}}%
\pgfpathlineto{\pgfqpoint{0.859681in}{0.999653in}}%
\pgfpathlineto{\pgfqpoint{0.869382in}{0.993878in}}%
\pgfpathlineto{\pgfqpoint{0.872500in}{0.992469in}}%
\pgfpathlineto{\pgfqpoint{0.874925in}{0.991234in}}%
\pgfpathlineto{\pgfqpoint{0.875098in}{0.991386in}}%
\pgfpathlineto{\pgfqpoint{0.876137in}{0.991234in}}%
\pgfpathlineto{\pgfqpoint{0.876484in}{0.990636in}}%
\pgfpathlineto{\pgfqpoint{0.880987in}{0.987526in}}%
\pgfpathlineto{\pgfqpoint{0.886357in}{0.983528in}}%
\pgfpathlineto{\pgfqpoint{0.893979in}{0.979722in}}%
\pgfpathlineto{\pgfqpoint{0.896404in}{0.979913in}}%
\pgfpathlineto{\pgfqpoint{0.898829in}{0.980651in}}%
\pgfpathlineto{\pgfqpoint{0.902813in}{0.983011in}}%
\pgfpathlineto{\pgfqpoint{0.906278in}{0.982847in}}%
\pgfpathlineto{\pgfqpoint{0.908529in}{0.981820in}}%
\pgfpathlineto{\pgfqpoint{0.913206in}{0.977893in}}%
\pgfpathlineto{\pgfqpoint{0.915285in}{0.975819in}}%
\pgfpathlineto{\pgfqpoint{0.919962in}{0.969519in}}%
\pgfpathlineto{\pgfqpoint{0.935898in}{0.954840in}}%
\pgfpathlineto{\pgfqpoint{0.936071in}{0.955100in}}%
\pgfpathlineto{\pgfqpoint{0.938323in}{0.956031in}}%
\pgfpathlineto{\pgfqpoint{0.941788in}{0.956720in}}%
\pgfpathlineto{\pgfqpoint{0.944732in}{0.955713in}}%
\pgfpathlineto{\pgfqpoint{0.946811in}{0.954768in}}%
\pgfpathlineto{\pgfqpoint{0.949063in}{0.953731in}}%
\pgfpathlineto{\pgfqpoint{0.950622in}{0.953169in}}%
\pgfpathlineto{\pgfqpoint{0.950968in}{0.953547in}}%
\pgfpathlineto{\pgfqpoint{0.957031in}{0.957257in}}%
\pgfpathlineto{\pgfqpoint{0.957377in}{0.956151in}}%
\pgfpathlineto{\pgfqpoint{0.959803in}{0.953296in}}%
\pgfpathlineto{\pgfqpoint{0.962747in}{0.953477in}}%
\pgfpathlineto{\pgfqpoint{0.965519in}{0.952646in}}%
\pgfpathlineto{\pgfqpoint{0.967597in}{0.952273in}}%
\pgfpathlineto{\pgfqpoint{0.969676in}{0.948289in}}%
\pgfpathlineto{\pgfqpoint{0.973487in}{0.939613in}}%
\pgfpathlineto{\pgfqpoint{0.974526in}{0.939285in}}%
\pgfpathlineto{\pgfqpoint{0.975046in}{0.940124in}}%
\pgfpathlineto{\pgfqpoint{0.975912in}{0.941055in}}%
\pgfpathlineto{\pgfqpoint{0.976432in}{0.940115in}}%
\pgfpathlineto{\pgfqpoint{0.979896in}{0.937324in}}%
\pgfpathlineto{\pgfqpoint{0.981109in}{0.936840in}}%
\pgfpathlineto{\pgfqpoint{0.981282in}{0.936430in}}%
\pgfpathlineto{\pgfqpoint{0.988557in}{0.926313in}}%
\pgfpathlineto{\pgfqpoint{0.989943in}{0.924993in}}%
\pgfpathlineto{\pgfqpoint{0.994273in}{0.916191in}}%
\pgfpathlineto{\pgfqpoint{0.995659in}{0.914823in}}%
\pgfpathlineto{\pgfqpoint{0.996525in}{0.913257in}}%
\pgfpathlineto{\pgfqpoint{0.997391in}{0.914394in}}%
\pgfpathlineto{\pgfqpoint{0.998950in}{0.912429in}}%
\pgfpathlineto{\pgfqpoint{1.003801in}{0.905621in}}%
\pgfpathlineto{\pgfqpoint{1.007438in}{0.902968in}}%
\pgfpathlineto{\pgfqpoint{1.010556in}{0.898031in}}%
\pgfpathlineto{\pgfqpoint{1.012981in}{0.896337in}}%
\pgfpathlineto{\pgfqpoint{1.016272in}{0.894024in}}%
\pgfpathlineto{\pgfqpoint{1.018697in}{0.892968in}}%
\pgfpathlineto{\pgfqpoint{1.024241in}{0.892624in}}%
\pgfpathlineto{\pgfqpoint{1.024933in}{0.893435in}}%
\pgfpathlineto{\pgfqpoint{1.025626in}{0.892205in}}%
\pgfpathlineto{\pgfqpoint{1.027705in}{0.889772in}}%
\pgfpathlineto{\pgfqpoint{1.027878in}{0.889869in}}%
\pgfpathlineto{\pgfqpoint{1.029437in}{0.889572in}}%
\pgfpathlineto{\pgfqpoint{1.030476in}{0.888434in}}%
\pgfpathlineto{\pgfqpoint{1.031343in}{0.886772in}}%
\pgfpathlineto{\pgfqpoint{1.032209in}{0.887626in}}%
\pgfpathlineto{\pgfqpoint{1.033075in}{0.887505in}}%
\pgfpathlineto{\pgfqpoint{1.033248in}{0.887143in}}%
\pgfpathlineto{\pgfqpoint{1.034287in}{0.884249in}}%
\pgfpathlineto{\pgfqpoint{1.035153in}{0.885213in}}%
\pgfpathlineto{\pgfqpoint{1.037925in}{0.881510in}}%
\pgfpathlineto{\pgfqpoint{1.038791in}{0.882258in}}%
\pgfpathlineto{\pgfqpoint{1.039657in}{0.880465in}}%
\pgfpathlineto{\pgfqpoint{1.049011in}{0.863584in}}%
\pgfpathlineto{\pgfqpoint{1.049531in}{0.864575in}}%
\pgfpathlineto{\pgfqpoint{1.050916in}{0.865625in}}%
\pgfpathlineto{\pgfqpoint{1.051436in}{0.865101in}}%
\pgfpathlineto{\pgfqpoint{1.053168in}{0.866495in}}%
\pgfpathlineto{\pgfqpoint{1.054034in}{0.865352in}}%
\pgfpathlineto{\pgfqpoint{1.056460in}{0.861386in}}%
\pgfpathlineto{\pgfqpoint{1.062522in}{0.852303in}}%
\pgfpathlineto{\pgfqpoint{1.063042in}{0.852942in}}%
\pgfpathlineto{\pgfqpoint{1.067372in}{0.853520in}}%
\pgfpathlineto{\pgfqpoint{1.068412in}{0.851165in}}%
\pgfpathlineto{\pgfqpoint{1.072049in}{0.842045in}}%
\pgfpathlineto{\pgfqpoint{1.072915in}{0.843145in}}%
\pgfpathlineto{\pgfqpoint{1.073435in}{0.842001in}}%
\pgfpathlineto{\pgfqpoint{1.075514in}{0.840519in}}%
\pgfpathlineto{\pgfqpoint{1.084175in}{0.832306in}}%
\pgfpathlineto{\pgfqpoint{1.087812in}{0.821607in}}%
\pgfpathlineto{\pgfqpoint{1.094741in}{0.816517in}}%
\pgfpathlineto{\pgfqpoint{1.101324in}{0.812988in}}%
\pgfpathlineto{\pgfqpoint{1.101670in}{0.812086in}}%
\pgfpathlineto{\pgfqpoint{1.102709in}{0.813292in}}%
\pgfpathlineto{\pgfqpoint{1.105308in}{0.813624in}}%
\pgfpathlineto{\pgfqpoint{1.115354in}{0.799405in}}%
\pgfpathlineto{\pgfqpoint{1.122110in}{0.791628in}}%
\pgfpathlineto{\pgfqpoint{1.122283in}{0.791883in}}%
\pgfpathlineto{\pgfqpoint{1.122803in}{0.790617in}}%
\pgfpathlineto{\pgfqpoint{1.123323in}{0.788899in}}%
\pgfpathlineto{\pgfqpoint{1.124535in}{0.789934in}}%
\pgfpathlineto{\pgfqpoint{1.125228in}{0.790938in}}%
\pgfpathlineto{\pgfqpoint{1.125574in}{0.789517in}}%
\pgfpathlineto{\pgfqpoint{1.126441in}{0.786150in}}%
\pgfpathlineto{\pgfqpoint{1.127307in}{0.787928in}}%
\pgfpathlineto{\pgfqpoint{1.129732in}{0.792910in}}%
\pgfpathlineto{\pgfqpoint{1.130425in}{0.791366in}}%
\pgfpathlineto{\pgfqpoint{1.132676in}{0.787913in}}%
\pgfpathlineto{\pgfqpoint{1.132850in}{0.787994in}}%
\pgfpathlineto{\pgfqpoint{1.133889in}{0.789339in}}%
\pgfpathlineto{\pgfqpoint{1.135275in}{0.791106in}}%
\pgfpathlineto{\pgfqpoint{1.135794in}{0.790282in}}%
\pgfpathlineto{\pgfqpoint{1.137700in}{0.790126in}}%
\pgfpathlineto{\pgfqpoint{1.138566in}{0.789304in}}%
\pgfpathlineto{\pgfqpoint{1.139086in}{0.790920in}}%
\pgfpathlineto{\pgfqpoint{1.143936in}{0.786738in}}%
\pgfpathlineto{\pgfqpoint{1.146188in}{0.779105in}}%
\pgfpathlineto{\pgfqpoint{1.147054in}{0.779770in}}%
\pgfpathlineto{\pgfqpoint{1.148613in}{0.783354in}}%
\pgfpathlineto{\pgfqpoint{1.149479in}{0.781666in}}%
\pgfpathlineto{\pgfqpoint{1.150172in}{0.782111in}}%
\pgfpathlineto{\pgfqpoint{1.151384in}{0.780554in}}%
\pgfpathlineto{\pgfqpoint{1.151904in}{0.779016in}}%
\pgfpathlineto{\pgfqpoint{1.155022in}{0.769834in}}%
\pgfpathlineto{\pgfqpoint{1.155195in}{0.769889in}}%
\pgfpathlineto{\pgfqpoint{1.155888in}{0.768082in}}%
\pgfpathlineto{\pgfqpoint{1.162470in}{0.747589in}}%
\pgfpathlineto{\pgfqpoint{1.162817in}{0.748628in}}%
\pgfpathlineto{\pgfqpoint{1.165588in}{0.751847in}}%
\pgfpathlineto{\pgfqpoint{1.172171in}{0.739502in}}%
\pgfpathlineto{\pgfqpoint{1.172690in}{0.736226in}}%
\pgfpathlineto{\pgfqpoint{1.175462in}{0.727564in}}%
\pgfpathlineto{\pgfqpoint{1.175808in}{0.728027in}}%
\pgfpathlineto{\pgfqpoint{1.176155in}{0.726647in}}%
\pgfpathlineto{\pgfqpoint{1.176328in}{0.726735in}}%
\pgfpathlineto{\pgfqpoint{1.179619in}{0.720897in}}%
\pgfpathlineto{\pgfqpoint{1.179792in}{0.721010in}}%
\pgfpathlineto{\pgfqpoint{1.181351in}{0.728991in}}%
\pgfpathlineto{\pgfqpoint{1.182044in}{0.727666in}}%
\pgfpathlineto{\pgfqpoint{1.183257in}{0.724019in}}%
\pgfpathlineto{\pgfqpoint{1.183776in}{0.725466in}}%
\pgfpathlineto{\pgfqpoint{1.187068in}{0.729251in}}%
\pgfpathlineto{\pgfqpoint{1.188107in}{0.730809in}}%
\pgfpathlineto{\pgfqpoint{1.188627in}{0.729147in}}%
\pgfpathlineto{\pgfqpoint{1.189146in}{0.728382in}}%
\pgfpathlineto{\pgfqpoint{1.191225in}{0.725562in}}%
\pgfpathlineto{\pgfqpoint{1.192264in}{0.724781in}}%
\pgfpathlineto{\pgfqpoint{1.191745in}{0.725789in}}%
\pgfpathlineto{\pgfqpoint{1.192957in}{0.725374in}}%
\pgfpathlineto{\pgfqpoint{1.193996in}{0.730493in}}%
\pgfpathlineto{\pgfqpoint{1.195036in}{0.726939in}}%
\pgfpathlineto{\pgfqpoint{1.199193in}{0.717378in}}%
\pgfpathlineto{\pgfqpoint{1.201791in}{0.722644in}}%
\pgfpathlineto{\pgfqpoint{1.202311in}{0.721885in}}%
\pgfpathlineto{\pgfqpoint{1.203524in}{0.715850in}}%
\pgfpathlineto{\pgfqpoint{1.207681in}{0.688030in}}%
\pgfpathlineto{\pgfqpoint{1.213224in}{0.708510in}}%
\pgfpathlineto{\pgfqpoint{1.213917in}{0.706483in}}%
\pgfpathlineto{\pgfqpoint{1.214783in}{0.708203in}}%
\pgfpathlineto{\pgfqpoint{1.215822in}{0.710096in}}%
\pgfpathlineto{\pgfqpoint{1.216515in}{0.708774in}}%
\pgfpathlineto{\pgfqpoint{1.217208in}{0.707551in}}%
\pgfpathlineto{\pgfqpoint{1.218074in}{0.708459in}}%
\pgfpathlineto{\pgfqpoint{1.218767in}{0.706555in}}%
\pgfpathlineto{\pgfqpoint{1.219460in}{0.703311in}}%
\pgfpathlineto{\pgfqpoint{1.220499in}{0.705253in}}%
\pgfpathlineto{\pgfqpoint{1.225523in}{0.717786in}}%
\pgfpathlineto{\pgfqpoint{1.226908in}{0.727163in}}%
\pgfpathlineto{\pgfqpoint{1.227601in}{0.724455in}}%
\pgfpathlineto{\pgfqpoint{1.230719in}{0.715850in}}%
\pgfpathlineto{\pgfqpoint{1.231066in}{0.716919in}}%
\pgfpathlineto{\pgfqpoint{1.231585in}{0.717999in}}%
\pgfpathlineto{\pgfqpoint{1.232105in}{0.715043in}}%
\pgfpathlineto{\pgfqpoint{1.236782in}{0.700510in}}%
\pgfpathlineto{\pgfqpoint{1.237128in}{0.699895in}}%
\pgfpathlineto{\pgfqpoint{1.237475in}{0.697642in}}%
\pgfpathlineto{\pgfqpoint{1.238341in}{0.702327in}}%
\pgfpathlineto{\pgfqpoint{1.239207in}{0.706111in}}%
\pgfpathlineto{\pgfqpoint{1.240419in}{0.704715in}}%
\pgfpathlineto{\pgfqpoint{1.240939in}{0.704987in}}%
\pgfpathlineto{\pgfqpoint{1.243191in}{0.696098in}}%
\pgfpathlineto{\pgfqpoint{1.245963in}{0.685817in}}%
\pgfpathlineto{\pgfqpoint{1.246136in}{0.686539in}}%
\pgfpathlineto{\pgfqpoint{1.247002in}{0.683826in}}%
\pgfpathlineto{\pgfqpoint{1.248388in}{0.679443in}}%
\pgfpathlineto{\pgfqpoint{1.249080in}{0.679888in}}%
\pgfpathlineto{\pgfqpoint{1.250986in}{0.690826in}}%
\pgfpathlineto{\pgfqpoint{1.254624in}{0.704567in}}%
\pgfpathlineto{\pgfqpoint{1.259300in}{0.694083in}}%
\pgfpathlineto{\pgfqpoint{1.261206in}{0.697524in}}%
\pgfpathlineto{\pgfqpoint{1.262072in}{0.700033in}}%
\pgfpathlineto{\pgfqpoint{1.262418in}{0.697309in}}%
\pgfpathlineto{\pgfqpoint{1.263977in}{0.692192in}}%
\pgfpathlineto{\pgfqpoint{1.264324in}{0.692633in}}%
\pgfpathlineto{\pgfqpoint{1.264844in}{0.691521in}}%
\pgfpathlineto{\pgfqpoint{1.265536in}{0.691871in}}%
\pgfpathlineto{\pgfqpoint{1.266056in}{0.687816in}}%
\pgfpathlineto{\pgfqpoint{1.266576in}{0.686669in}}%
\pgfpathlineto{\pgfqpoint{1.267442in}{0.688779in}}%
\pgfpathlineto{\pgfqpoint{1.269001in}{0.690971in}}%
\pgfpathlineto{\pgfqpoint{1.269694in}{0.690107in}}%
\pgfpathlineto{\pgfqpoint{1.270560in}{0.686038in}}%
\pgfpathlineto{\pgfqpoint{1.271253in}{0.689460in}}%
\pgfpathlineto{\pgfqpoint{1.272465in}{0.698068in}}%
\pgfpathlineto{\pgfqpoint{1.273158in}{0.692453in}}%
\pgfpathlineto{\pgfqpoint{1.274197in}{0.687486in}}%
\pgfpathlineto{\pgfqpoint{1.274890in}{0.689558in}}%
\pgfpathlineto{\pgfqpoint{1.275756in}{0.693343in}}%
\pgfpathlineto{\pgfqpoint{1.276449in}{0.690700in}}%
\pgfpathlineto{\pgfqpoint{1.278182in}{0.683852in}}%
\pgfpathlineto{\pgfqpoint{1.278355in}{0.684542in}}%
\pgfpathlineto{\pgfqpoint{1.279740in}{0.691031in}}%
\pgfpathlineto{\pgfqpoint{1.280607in}{0.688711in}}%
\pgfpathlineto{\pgfqpoint{1.282512in}{0.685428in}}%
\pgfpathlineto{\pgfqpoint{1.282858in}{0.686934in}}%
\pgfpathlineto{\pgfqpoint{1.284764in}{0.693085in}}%
\pgfpathlineto{\pgfqpoint{1.285284in}{0.690670in}}%
\pgfpathlineto{\pgfqpoint{1.285803in}{0.686714in}}%
\pgfpathlineto{\pgfqpoint{1.286843in}{0.688948in}}%
\pgfpathlineto{\pgfqpoint{1.287882in}{0.684850in}}%
\pgfpathlineto{\pgfqpoint{1.288748in}{0.682359in}}%
\pgfpathlineto{\pgfqpoint{1.289268in}{0.684239in}}%
\pgfpathlineto{\pgfqpoint{1.290480in}{0.690537in}}%
\pgfpathlineto{\pgfqpoint{1.291519in}{0.688562in}}%
\pgfpathlineto{\pgfqpoint{1.292386in}{0.690295in}}%
\pgfpathlineto{\pgfqpoint{1.293078in}{0.689056in}}%
\pgfpathlineto{\pgfqpoint{1.293771in}{0.685084in}}%
\pgfpathlineto{\pgfqpoint{1.297236in}{0.669069in}}%
\pgfpathlineto{\pgfqpoint{1.300873in}{0.675958in}}%
\pgfpathlineto{\pgfqpoint{1.301047in}{0.675129in}}%
\pgfpathlineto{\pgfqpoint{1.302952in}{0.670054in}}%
\pgfpathlineto{\pgfqpoint{1.303298in}{0.670395in}}%
\pgfpathlineto{\pgfqpoint{1.304511in}{0.672677in}}%
\pgfpathlineto{\pgfqpoint{1.306243in}{0.675104in}}%
\pgfpathlineto{\pgfqpoint{1.305204in}{0.671047in}}%
\pgfpathlineto{\pgfqpoint{1.306416in}{0.675044in}}%
\pgfpathlineto{\pgfqpoint{1.308842in}{0.667488in}}%
\pgfpathlineto{\pgfqpoint{1.306936in}{0.675156in}}%
\pgfpathlineto{\pgfqpoint{1.309188in}{0.668631in}}%
\pgfpathlineto{\pgfqpoint{1.311093in}{0.677787in}}%
\pgfpathlineto{\pgfqpoint{1.311267in}{0.676325in}}%
\pgfpathlineto{\pgfqpoint{1.313518in}{0.668939in}}%
\pgfpathlineto{\pgfqpoint{1.314038in}{0.671252in}}%
\pgfpathlineto{\pgfqpoint{1.314904in}{0.675777in}}%
\pgfpathlineto{\pgfqpoint{1.315770in}{0.672814in}}%
\pgfpathlineto{\pgfqpoint{1.317329in}{0.675443in}}%
\pgfpathlineto{\pgfqpoint{1.318022in}{0.674296in}}%
\pgfpathlineto{\pgfqpoint{1.319235in}{0.673433in}}%
\pgfpathlineto{\pgfqpoint{1.318715in}{0.674993in}}%
\pgfpathlineto{\pgfqpoint{1.319754in}{0.674615in}}%
\pgfpathlineto{\pgfqpoint{1.320794in}{0.676507in}}%
\pgfpathlineto{\pgfqpoint{1.321313in}{0.674765in}}%
\pgfpathlineto{\pgfqpoint{1.324085in}{0.656905in}}%
\pgfpathlineto{\pgfqpoint{1.324951in}{0.662501in}}%
\pgfpathlineto{\pgfqpoint{1.325990in}{0.669008in}}%
\pgfpathlineto{\pgfqpoint{1.326856in}{0.666636in}}%
\pgfpathlineto{\pgfqpoint{1.327030in}{0.666152in}}%
\pgfpathlineto{\pgfqpoint{1.327376in}{0.668696in}}%
\pgfpathlineto{\pgfqpoint{1.327896in}{0.668260in}}%
\pgfpathlineto{\pgfqpoint{1.330148in}{0.670584in}}%
\pgfpathlineto{\pgfqpoint{1.331880in}{0.673416in}}%
\pgfpathlineto{\pgfqpoint{1.334478in}{0.681137in}}%
\pgfpathlineto{\pgfqpoint{1.334651in}{0.681095in}}%
\pgfpathlineto{\pgfqpoint{1.336557in}{0.674375in}}%
\pgfpathlineto{\pgfqpoint{1.338809in}{0.665080in}}%
\pgfpathlineto{\pgfqpoint{1.340714in}{0.659589in}}%
\pgfpathlineto{\pgfqpoint{1.340887in}{0.659818in}}%
\pgfpathlineto{\pgfqpoint{1.342100in}{0.655235in}}%
\pgfpathlineto{\pgfqpoint{1.342966in}{0.655633in}}%
\pgfpathlineto{\pgfqpoint{1.344525in}{0.660166in}}%
\pgfpathlineto{\pgfqpoint{1.344871in}{0.657872in}}%
\pgfpathlineto{\pgfqpoint{1.349029in}{0.641354in}}%
\pgfpathlineto{\pgfqpoint{1.349202in}{0.641093in}}%
\pgfpathlineto{\pgfqpoint{1.349548in}{0.643082in}}%
\pgfpathlineto{\pgfqpoint{1.349721in}{0.642977in}}%
\pgfpathlineto{\pgfqpoint{1.349895in}{0.643520in}}%
\pgfpathlineto{\pgfqpoint{1.350414in}{0.640530in}}%
\pgfpathlineto{\pgfqpoint{1.351454in}{0.636939in}}%
\pgfpathlineto{\pgfqpoint{1.352147in}{0.638009in}}%
\pgfpathlineto{\pgfqpoint{1.353706in}{0.643606in}}%
\pgfpathlineto{\pgfqpoint{1.354225in}{0.640336in}}%
\pgfpathlineto{\pgfqpoint{1.355091in}{0.635557in}}%
\pgfpathlineto{\pgfqpoint{1.355611in}{0.640110in}}%
\pgfpathlineto{\pgfqpoint{1.356304in}{0.647332in}}%
\pgfpathlineto{\pgfqpoint{1.356997in}{0.640359in}}%
\pgfpathlineto{\pgfqpoint{1.357863in}{0.630949in}}%
\pgfpathlineto{\pgfqpoint{1.358556in}{0.637442in}}%
\pgfpathlineto{\pgfqpoint{1.360808in}{0.640352in}}%
\pgfpathlineto{\pgfqpoint{1.362020in}{0.652661in}}%
\pgfpathlineto{\pgfqpoint{1.362540in}{0.646067in}}%
\pgfpathlineto{\pgfqpoint{1.363752in}{0.642600in}}%
\pgfpathlineto{\pgfqpoint{1.364445in}{0.642956in}}%
\pgfpathlineto{\pgfqpoint{1.365311in}{0.641563in}}%
\pgfpathlineto{\pgfqpoint{1.366004in}{0.639122in}}%
\pgfpathlineto{\pgfqpoint{1.367044in}{0.640253in}}%
\pgfpathlineto{\pgfqpoint{1.368603in}{0.646148in}}%
\pgfpathlineto{\pgfqpoint{1.369295in}{0.644194in}}%
\pgfpathlineto{\pgfqpoint{1.369642in}{0.642688in}}%
\pgfpathlineto{\pgfqpoint{1.370335in}{0.646287in}}%
\pgfpathlineto{\pgfqpoint{1.371028in}{0.653155in}}%
\pgfpathlineto{\pgfqpoint{1.371894in}{0.650385in}}%
\pgfpathlineto{\pgfqpoint{1.372240in}{0.648020in}}%
\pgfpathlineto{\pgfqpoint{1.373279in}{0.651790in}}%
\pgfpathlineto{\pgfqpoint{1.375358in}{0.662452in}}%
\pgfpathlineto{\pgfqpoint{1.376224in}{0.659903in}}%
\pgfpathlineto{\pgfqpoint{1.377437in}{0.653401in}}%
\pgfpathlineto{\pgfqpoint{1.378130in}{0.656593in}}%
\pgfpathlineto{\pgfqpoint{1.379689in}{0.654902in}}%
\pgfpathlineto{\pgfqpoint{1.380035in}{0.654494in}}%
\pgfpathlineto{\pgfqpoint{1.380555in}{0.655538in}}%
\pgfpathlineto{\pgfqpoint{1.382114in}{0.662368in}}%
\pgfpathlineto{\pgfqpoint{1.382460in}{0.660088in}}%
\pgfpathlineto{\pgfqpoint{1.384885in}{0.650585in}}%
\pgfpathlineto{\pgfqpoint{1.385058in}{0.649515in}}%
\pgfpathlineto{\pgfqpoint{1.385578in}{0.653746in}}%
\pgfpathlineto{\pgfqpoint{1.385751in}{0.655714in}}%
\pgfpathlineto{\pgfqpoint{1.386617in}{0.649433in}}%
\pgfpathlineto{\pgfqpoint{1.387657in}{0.644246in}}%
\pgfpathlineto{\pgfqpoint{1.388350in}{0.646457in}}%
\pgfpathlineto{\pgfqpoint{1.388696in}{0.648457in}}%
\pgfpathlineto{\pgfqpoint{1.389909in}{0.647657in}}%
\pgfpathlineto{\pgfqpoint{1.390255in}{0.645433in}}%
\pgfpathlineto{\pgfqpoint{1.390775in}{0.649255in}}%
\pgfpathlineto{\pgfqpoint{1.392680in}{0.657450in}}%
\pgfpathlineto{\pgfqpoint{1.392853in}{0.656638in}}%
\pgfpathlineto{\pgfqpoint{1.393719in}{0.655137in}}%
\pgfpathlineto{\pgfqpoint{1.394412in}{0.649835in}}%
\pgfpathlineto{\pgfqpoint{1.395278in}{0.654076in}}%
\pgfpathlineto{\pgfqpoint{1.395798in}{0.655580in}}%
\pgfpathlineto{\pgfqpoint{1.396318in}{0.652800in}}%
\pgfpathlineto{\pgfqpoint{1.397704in}{0.644091in}}%
\pgfpathlineto{\pgfqpoint{1.398223in}{0.648288in}}%
\pgfpathlineto{\pgfqpoint{1.399089in}{0.657338in}}%
\pgfpathlineto{\pgfqpoint{1.399782in}{0.651616in}}%
\pgfpathlineto{\pgfqpoint{1.400129in}{0.648614in}}%
\pgfpathlineto{\pgfqpoint{1.401168in}{0.651582in}}%
\pgfpathlineto{\pgfqpoint{1.403247in}{0.665396in}}%
\pgfpathlineto{\pgfqpoint{1.403593in}{0.663221in}}%
\pgfpathlineto{\pgfqpoint{1.404113in}{0.659159in}}%
\pgfpathlineto{\pgfqpoint{1.404979in}{0.663394in}}%
\pgfpathlineto{\pgfqpoint{1.406711in}{0.665907in}}%
\pgfpathlineto{\pgfqpoint{1.408097in}{0.659851in}}%
\pgfpathlineto{\pgfqpoint{1.408616in}{0.661137in}}%
\pgfpathlineto{\pgfqpoint{1.408963in}{0.659855in}}%
\pgfpathlineto{\pgfqpoint{1.409483in}{0.662740in}}%
\pgfpathlineto{\pgfqpoint{1.412774in}{0.674359in}}%
\pgfpathlineto{\pgfqpoint{1.413986in}{0.669865in}}%
\pgfpathlineto{\pgfqpoint{1.414506in}{0.672362in}}%
\pgfpathlineto{\pgfqpoint{1.416238in}{0.670749in}}%
\pgfpathlineto{\pgfqpoint{1.416585in}{0.669413in}}%
\pgfpathlineto{\pgfqpoint{1.417451in}{0.671860in}}%
\pgfpathlineto{\pgfqpoint{1.419010in}{0.680202in}}%
\pgfpathlineto{\pgfqpoint{1.419703in}{0.674239in}}%
\pgfpathlineto{\pgfqpoint{1.421608in}{0.670244in}}%
\pgfpathlineto{\pgfqpoint{1.422301in}{0.667334in}}%
\pgfpathlineto{\pgfqpoint{1.423167in}{0.670029in}}%
\pgfpathlineto{\pgfqpoint{1.423340in}{0.669799in}}%
\pgfpathlineto{\pgfqpoint{1.423513in}{0.671480in}}%
\pgfpathlineto{\pgfqpoint{1.425072in}{0.677075in}}%
\pgfpathlineto{\pgfqpoint{1.425765in}{0.676084in}}%
\pgfpathlineto{\pgfqpoint{1.429576in}{0.657454in}}%
\pgfpathlineto{\pgfqpoint{1.429749in}{0.658474in}}%
\pgfpathlineto{\pgfqpoint{1.430789in}{0.655475in}}%
\pgfpathlineto{\pgfqpoint{1.431135in}{0.655165in}}%
\pgfpathlineto{\pgfqpoint{1.431308in}{0.656444in}}%
\pgfpathlineto{\pgfqpoint{1.431481in}{0.656150in}}%
\pgfpathlineto{\pgfqpoint{1.436505in}{0.693700in}}%
\pgfpathlineto{\pgfqpoint{1.439276in}{0.799316in}}%
\pgfpathlineto{\pgfqpoint{1.443434in}{0.895153in}}%
\pgfpathlineto{\pgfqpoint{1.448284in}{0.938045in}}%
\pgfpathlineto{\pgfqpoint{1.449843in}{0.940019in}}%
\pgfpathlineto{\pgfqpoint{1.450536in}{0.939510in}}%
\pgfpathlineto{\pgfqpoint{1.453480in}{0.934133in}}%
\pgfpathlineto{\pgfqpoint{1.456598in}{0.924162in}}%
\pgfpathlineto{\pgfqpoint{1.457811in}{0.922385in}}%
\pgfpathlineto{\pgfqpoint{1.458331in}{0.922874in}}%
\pgfpathlineto{\pgfqpoint{1.460236in}{0.922311in}}%
\pgfpathlineto{\pgfqpoint{1.461449in}{0.922031in}}%
\pgfpathlineto{\pgfqpoint{1.461968in}{0.922365in}}%
\pgfpathlineto{\pgfqpoint{1.463700in}{0.921258in}}%
\pgfpathlineto{\pgfqpoint{1.477038in}{0.867325in}}%
\pgfpathlineto{\pgfqpoint{1.479637in}{0.854589in}}%
\pgfpathlineto{\pgfqpoint{1.490376in}{0.805253in}}%
\pgfpathlineto{\pgfqpoint{1.508218in}{0.716549in}}%
\pgfpathlineto{\pgfqpoint{1.510816in}{0.706285in}}%
\pgfpathlineto{\pgfqpoint{1.517052in}{0.693858in}}%
\pgfpathlineto{\pgfqpoint{1.517572in}{0.697534in}}%
\pgfpathlineto{\pgfqpoint{1.519824in}{0.708090in}}%
\pgfpathlineto{\pgfqpoint{1.520344in}{0.707750in}}%
\pgfpathlineto{\pgfqpoint{1.521729in}{0.708044in}}%
\pgfpathlineto{\pgfqpoint{1.521902in}{0.708450in}}%
\pgfpathlineto{\pgfqpoint{1.522249in}{0.709507in}}%
\pgfpathlineto{\pgfqpoint{1.522942in}{0.707922in}}%
\pgfpathlineto{\pgfqpoint{1.523288in}{0.708036in}}%
\pgfpathlineto{\pgfqpoint{1.528312in}{0.689574in}}%
\pgfpathlineto{\pgfqpoint{1.528831in}{0.690594in}}%
\pgfpathlineto{\pgfqpoint{1.529871in}{0.693268in}}%
\pgfpathlineto{\pgfqpoint{1.534894in}{0.711734in}}%
\pgfpathlineto{\pgfqpoint{1.536453in}{0.716925in}}%
\pgfpathlineto{\pgfqpoint{1.537839in}{0.714874in}}%
\pgfpathlineto{\pgfqpoint{1.540264in}{0.707881in}}%
\pgfpathlineto{\pgfqpoint{1.547539in}{0.690893in}}%
\pgfpathlineto{\pgfqpoint{1.548059in}{0.691550in}}%
\pgfpathlineto{\pgfqpoint{1.549791in}{0.691061in}}%
\pgfpathlineto{\pgfqpoint{1.549964in}{0.691417in}}%
\pgfpathlineto{\pgfqpoint{1.552562in}{0.699750in}}%
\pgfpathlineto{\pgfqpoint{1.553255in}{0.698366in}}%
\pgfpathlineto{\pgfqpoint{1.554988in}{0.694624in}}%
\pgfpathlineto{\pgfqpoint{1.555161in}{0.694119in}}%
\pgfpathlineto{\pgfqpoint{1.555854in}{0.696123in}}%
\pgfpathlineto{\pgfqpoint{1.557239in}{0.700436in}}%
\pgfpathlineto{\pgfqpoint{1.557932in}{0.699551in}}%
\pgfpathlineto{\pgfqpoint{1.558452in}{0.699106in}}%
\pgfpathlineto{\pgfqpoint{1.558972in}{0.700443in}}%
\pgfpathlineto{\pgfqpoint{1.563302in}{0.711901in}}%
\pgfpathlineto{\pgfqpoint{1.563475in}{0.711423in}}%
\pgfpathlineto{\pgfqpoint{1.564861in}{0.704059in}}%
\pgfpathlineto{\pgfqpoint{1.568672in}{0.681654in}}%
\pgfpathlineto{\pgfqpoint{1.571270in}{0.680457in}}%
\pgfpathlineto{\pgfqpoint{1.571443in}{0.680877in}}%
\pgfpathlineto{\pgfqpoint{1.571617in}{0.681189in}}%
\pgfpathlineto{\pgfqpoint{1.572310in}{0.679956in}}%
\pgfpathlineto{\pgfqpoint{1.576467in}{0.667278in}}%
\pgfpathlineto{\pgfqpoint{1.577333in}{0.670626in}}%
\pgfpathlineto{\pgfqpoint{1.578026in}{0.672024in}}%
\pgfpathlineto{\pgfqpoint{1.578199in}{0.670504in}}%
\pgfpathlineto{\pgfqpoint{1.580624in}{0.663008in}}%
\pgfpathlineto{\pgfqpoint{1.582703in}{0.659844in}}%
\pgfpathlineto{\pgfqpoint{1.583049in}{0.661166in}}%
\pgfpathlineto{\pgfqpoint{1.584435in}{0.664634in}}%
\pgfpathlineto{\pgfqpoint{1.584955in}{0.663650in}}%
\pgfpathlineto{\pgfqpoint{1.586860in}{0.667016in}}%
\pgfpathlineto{\pgfqpoint{1.587380in}{0.664541in}}%
\pgfpathlineto{\pgfqpoint{1.589632in}{0.654322in}}%
\pgfpathlineto{\pgfqpoint{1.590151in}{0.655538in}}%
\pgfpathlineto{\pgfqpoint{1.590844in}{0.652447in}}%
\pgfpathlineto{\pgfqpoint{1.592923in}{0.643294in}}%
\pgfpathlineto{\pgfqpoint{1.593442in}{0.644293in}}%
\pgfpathlineto{\pgfqpoint{1.594309in}{0.646103in}}%
\pgfpathlineto{\pgfqpoint{1.594482in}{0.644764in}}%
\pgfpathlineto{\pgfqpoint{1.597253in}{0.631034in}}%
\pgfpathlineto{\pgfqpoint{1.598639in}{0.624683in}}%
\pgfpathlineto{\pgfqpoint{1.600371in}{0.618024in}}%
\pgfpathlineto{\pgfqpoint{1.600545in}{0.618932in}}%
\pgfpathlineto{\pgfqpoint{1.601064in}{0.621397in}}%
\pgfpathlineto{\pgfqpoint{1.601930in}{0.618300in}}%
\pgfpathlineto{\pgfqpoint{1.604182in}{0.615897in}}%
\pgfpathlineto{\pgfqpoint{1.605395in}{0.617236in}}%
\pgfpathlineto{\pgfqpoint{1.608166in}{0.625592in}}%
\pgfpathlineto{\pgfqpoint{1.608339in}{0.624793in}}%
\pgfpathlineto{\pgfqpoint{1.608513in}{0.623370in}}%
\pgfpathlineto{\pgfqpoint{1.609898in}{0.623839in}}%
\pgfpathlineto{\pgfqpoint{1.612497in}{0.638272in}}%
\pgfpathlineto{\pgfqpoint{1.615095in}{0.634935in}}%
\pgfpathlineto{\pgfqpoint{1.617347in}{0.628298in}}%
\pgfpathlineto{\pgfqpoint{1.617693in}{0.628664in}}%
\pgfpathlineto{\pgfqpoint{1.618386in}{0.628682in}}%
\pgfpathlineto{\pgfqpoint{1.618733in}{0.627796in}}%
\pgfpathlineto{\pgfqpoint{1.620292in}{0.621476in}}%
\pgfpathlineto{\pgfqpoint{1.620811in}{0.622318in}}%
\pgfpathlineto{\pgfqpoint{1.623236in}{0.626801in}}%
\pgfpathlineto{\pgfqpoint{1.624102in}{0.625371in}}%
\pgfpathlineto{\pgfqpoint{1.629992in}{0.607439in}}%
\pgfpathlineto{\pgfqpoint{1.630858in}{0.610256in}}%
\pgfpathlineto{\pgfqpoint{1.632763in}{0.614294in}}%
\pgfpathlineto{\pgfqpoint{1.632937in}{0.613972in}}%
\pgfpathlineto{\pgfqpoint{1.634669in}{0.610673in}}%
\pgfpathlineto{\pgfqpoint{1.635535in}{0.612349in}}%
\pgfpathlineto{\pgfqpoint{1.636055in}{0.610399in}}%
\pgfpathlineto{\pgfqpoint{1.636574in}{0.605961in}}%
\pgfpathlineto{\pgfqpoint{1.637787in}{0.608676in}}%
\pgfpathlineto{\pgfqpoint{1.638133in}{0.610219in}}%
\pgfpathlineto{\pgfqpoint{1.639346in}{0.609216in}}%
\pgfpathlineto{\pgfqpoint{1.640212in}{0.609356in}}%
\pgfpathlineto{\pgfqpoint{1.640558in}{0.609880in}}%
\pgfpathlineto{\pgfqpoint{1.643676in}{0.617571in}}%
\pgfpathlineto{\pgfqpoint{1.643850in}{0.616765in}}%
\pgfpathlineto{\pgfqpoint{1.644716in}{0.614146in}}%
\pgfpathlineto{\pgfqpoint{1.645235in}{0.616138in}}%
\pgfpathlineto{\pgfqpoint{1.646448in}{0.620354in}}%
\pgfpathlineto{\pgfqpoint{1.646968in}{0.617955in}}%
\pgfpathlineto{\pgfqpoint{1.650086in}{0.600377in}}%
\pgfpathlineto{\pgfqpoint{1.651991in}{0.597191in}}%
\pgfpathlineto{\pgfqpoint{1.652511in}{0.598146in}}%
\pgfpathlineto{\pgfqpoint{1.654589in}{0.604441in}}%
\pgfpathlineto{\pgfqpoint{1.654936in}{0.603730in}}%
\pgfpathlineto{\pgfqpoint{1.655455in}{0.604604in}}%
\pgfpathlineto{\pgfqpoint{1.656321in}{0.603153in}}%
\pgfpathlineto{\pgfqpoint{1.656495in}{0.602551in}}%
\pgfpathlineto{\pgfqpoint{1.657361in}{0.604196in}}%
\pgfpathlineto{\pgfqpoint{1.658054in}{0.607344in}}%
\pgfpathlineto{\pgfqpoint{1.658747in}{0.604122in}}%
\pgfpathlineto{\pgfqpoint{1.661518in}{0.591363in}}%
\pgfpathlineto{\pgfqpoint{1.661691in}{0.592408in}}%
\pgfpathlineto{\pgfqpoint{1.664116in}{0.599700in}}%
\pgfpathlineto{\pgfqpoint{1.664636in}{0.597716in}}%
\pgfpathlineto{\pgfqpoint{1.669140in}{0.579837in}}%
\pgfpathlineto{\pgfqpoint{1.669833in}{0.581569in}}%
\pgfpathlineto{\pgfqpoint{1.672085in}{0.593391in}}%
\pgfpathlineto{\pgfqpoint{1.672777in}{0.593229in}}%
\pgfpathlineto{\pgfqpoint{1.672951in}{0.594077in}}%
\pgfpathlineto{\pgfqpoint{1.675722in}{0.616303in}}%
\pgfpathlineto{\pgfqpoint{1.677108in}{0.615562in}}%
\pgfpathlineto{\pgfqpoint{1.680572in}{0.604271in}}%
\pgfpathlineto{\pgfqpoint{1.681785in}{0.605032in}}%
\pgfpathlineto{\pgfqpoint{1.681958in}{0.604781in}}%
\pgfpathlineto{\pgfqpoint{1.682478in}{0.606003in}}%
\pgfpathlineto{\pgfqpoint{1.685596in}{0.613134in}}%
\pgfpathlineto{\pgfqpoint{1.691832in}{0.638783in}}%
\pgfpathlineto{\pgfqpoint{1.693737in}{0.642364in}}%
\pgfpathlineto{\pgfqpoint{1.694083in}{0.641515in}}%
\pgfpathlineto{\pgfqpoint{1.695816in}{0.637401in}}%
\pgfpathlineto{\pgfqpoint{1.702052in}{0.598617in}}%
\pgfpathlineto{\pgfqpoint{1.703264in}{0.603817in}}%
\pgfpathlineto{\pgfqpoint{1.703784in}{0.600858in}}%
\pgfpathlineto{\pgfqpoint{1.706902in}{0.579859in}}%
\pgfpathlineto{\pgfqpoint{1.707248in}{0.578587in}}%
\pgfpathlineto{\pgfqpoint{1.707595in}{0.580099in}}%
\pgfpathlineto{\pgfqpoint{1.708634in}{0.579313in}}%
\pgfpathlineto{\pgfqpoint{1.711925in}{0.596673in}}%
\pgfpathlineto{\pgfqpoint{1.712445in}{0.594758in}}%
\pgfpathlineto{\pgfqpoint{1.712791in}{0.593762in}}%
\pgfpathlineto{\pgfqpoint{1.713657in}{0.595938in}}%
\pgfpathlineto{\pgfqpoint{1.714177in}{0.598072in}}%
\pgfpathlineto{\pgfqpoint{1.714697in}{0.595201in}}%
\pgfpathlineto{\pgfqpoint{1.715043in}{0.595278in}}%
\pgfpathlineto{\pgfqpoint{1.715216in}{0.595662in}}%
\pgfpathlineto{\pgfqpoint{1.715563in}{0.594536in}}%
\pgfpathlineto{\pgfqpoint{1.716256in}{0.591684in}}%
\pgfpathlineto{\pgfqpoint{1.717122in}{0.594220in}}%
\pgfpathlineto{\pgfqpoint{1.717295in}{0.593446in}}%
\pgfpathlineto{\pgfqpoint{1.717988in}{0.596001in}}%
\pgfpathlineto{\pgfqpoint{1.721106in}{0.606701in}}%
\pgfpathlineto{\pgfqpoint{1.721279in}{0.605587in}}%
\pgfpathlineto{\pgfqpoint{1.723531in}{0.596208in}}%
\pgfpathlineto{\pgfqpoint{1.724397in}{0.592794in}}%
\pgfpathlineto{\pgfqpoint{1.724917in}{0.596764in}}%
\pgfpathlineto{\pgfqpoint{1.725783in}{0.601653in}}%
\pgfpathlineto{\pgfqpoint{1.726476in}{0.599535in}}%
\pgfpathlineto{\pgfqpoint{1.727169in}{0.598367in}}%
\pgfpathlineto{\pgfqpoint{1.727861in}{0.600302in}}%
\pgfpathlineto{\pgfqpoint{1.730633in}{0.613488in}}%
\pgfpathlineto{\pgfqpoint{1.730806in}{0.613068in}}%
\pgfpathlineto{\pgfqpoint{1.731153in}{0.613333in}}%
\pgfpathlineto{\pgfqpoint{1.731326in}{0.614328in}}%
\pgfpathlineto{\pgfqpoint{1.731672in}{0.616041in}}%
\pgfpathlineto{\pgfqpoint{1.732365in}{0.613616in}}%
\pgfpathlineto{\pgfqpoint{1.732885in}{0.610568in}}%
\pgfpathlineto{\pgfqpoint{1.733751in}{0.614507in}}%
\pgfpathlineto{\pgfqpoint{1.734444in}{0.611334in}}%
\pgfpathlineto{\pgfqpoint{1.736003in}{0.605791in}}%
\pgfpathlineto{\pgfqpoint{1.736522in}{0.607683in}}%
\pgfpathlineto{\pgfqpoint{1.737562in}{0.612705in}}%
\pgfpathlineto{\pgfqpoint{1.738428in}{0.609879in}}%
\pgfpathlineto{\pgfqpoint{1.739467in}{0.609449in}}%
\pgfpathlineto{\pgfqpoint{1.739814in}{0.610890in}}%
\pgfpathlineto{\pgfqpoint{1.742932in}{0.615611in}}%
\pgfpathlineto{\pgfqpoint{1.743278in}{0.613617in}}%
\pgfpathlineto{\pgfqpoint{1.744664in}{0.605543in}}%
\pgfpathlineto{\pgfqpoint{1.746742in}{0.595794in}}%
\pgfpathlineto{\pgfqpoint{1.746916in}{0.596262in}}%
\pgfpathlineto{\pgfqpoint{1.747782in}{0.600984in}}%
\pgfpathlineto{\pgfqpoint{1.748648in}{0.597702in}}%
\pgfpathlineto{\pgfqpoint{1.750380in}{0.591889in}}%
\pgfpathlineto{\pgfqpoint{1.751419in}{0.592310in}}%
\pgfpathlineto{\pgfqpoint{1.754191in}{0.589394in}}%
\pgfpathlineto{\pgfqpoint{1.752112in}{0.592920in}}%
\pgfpathlineto{\pgfqpoint{1.754537in}{0.589767in}}%
\pgfpathlineto{\pgfqpoint{1.754711in}{0.589953in}}%
\pgfpathlineto{\pgfqpoint{1.755577in}{0.589077in}}%
\pgfpathlineto{\pgfqpoint{1.757309in}{0.586547in}}%
\pgfpathlineto{\pgfqpoint{1.756443in}{0.589469in}}%
\pgfpathlineto{\pgfqpoint{1.757655in}{0.587485in}}%
\pgfpathlineto{\pgfqpoint{1.760080in}{0.586640in}}%
\pgfpathlineto{\pgfqpoint{1.760254in}{0.587194in}}%
\pgfpathlineto{\pgfqpoint{1.761120in}{0.590037in}}%
\pgfpathlineto{\pgfqpoint{1.761466in}{0.587969in}}%
\pgfpathlineto{\pgfqpoint{1.762159in}{0.581183in}}%
\pgfpathlineto{\pgfqpoint{1.763198in}{0.585589in}}%
\pgfpathlineto{\pgfqpoint{1.763545in}{0.586706in}}%
\pgfpathlineto{\pgfqpoint{1.764064in}{0.584074in}}%
\pgfpathlineto{\pgfqpoint{1.764238in}{0.584198in}}%
\pgfpathlineto{\pgfqpoint{1.764411in}{0.583944in}}%
\pgfpathlineto{\pgfqpoint{1.764931in}{0.585739in}}%
\pgfpathlineto{\pgfqpoint{1.765277in}{0.587545in}}%
\pgfpathlineto{\pgfqpoint{1.766143in}{0.583252in}}%
\pgfpathlineto{\pgfqpoint{1.768568in}{0.570596in}}%
\pgfpathlineto{\pgfqpoint{1.769088in}{0.572378in}}%
\pgfpathlineto{\pgfqpoint{1.771513in}{0.577878in}}%
\pgfpathlineto{\pgfqpoint{1.772033in}{0.576776in}}%
\pgfpathlineto{\pgfqpoint{1.773938in}{0.572553in}}%
\pgfpathlineto{\pgfqpoint{1.774631in}{0.575873in}}%
\pgfpathlineto{\pgfqpoint{1.776710in}{0.582212in}}%
\pgfpathlineto{\pgfqpoint{1.777749in}{0.583977in}}%
\pgfpathlineto{\pgfqpoint{1.778442in}{0.583245in}}%
\pgfpathlineto{\pgfqpoint{1.778615in}{0.583053in}}%
\pgfpathlineto{\pgfqpoint{1.778788in}{0.583919in}}%
\pgfpathlineto{\pgfqpoint{1.779481in}{0.584281in}}%
\pgfpathlineto{\pgfqpoint{1.779828in}{0.585787in}}%
\pgfpathlineto{\pgfqpoint{1.780347in}{0.583492in}}%
\pgfpathlineto{\pgfqpoint{1.781040in}{0.576299in}}%
\pgfpathlineto{\pgfqpoint{1.782426in}{0.577609in}}%
\pgfpathlineto{\pgfqpoint{1.784158in}{0.585295in}}%
\pgfpathlineto{\pgfqpoint{1.785024in}{0.582025in}}%
\pgfpathlineto{\pgfqpoint{1.785197in}{0.581592in}}%
\pgfpathlineto{\pgfqpoint{1.785890in}{0.583662in}}%
\pgfpathlineto{\pgfqpoint{1.788315in}{0.593649in}}%
\pgfpathlineto{\pgfqpoint{1.788489in}{0.594318in}}%
\pgfpathlineto{\pgfqpoint{1.789008in}{0.592888in}}%
\pgfpathlineto{\pgfqpoint{1.789874in}{0.593769in}}%
\pgfpathlineto{\pgfqpoint{1.790394in}{0.591930in}}%
\pgfpathlineto{\pgfqpoint{1.792819in}{0.579359in}}%
\pgfpathlineto{\pgfqpoint{1.793512in}{0.582374in}}%
\pgfpathlineto{\pgfqpoint{1.795244in}{0.593009in}}%
\pgfpathlineto{\pgfqpoint{1.795591in}{0.591372in}}%
\pgfpathlineto{\pgfqpoint{1.796630in}{0.587947in}}%
\pgfpathlineto{\pgfqpoint{1.797323in}{0.589827in}}%
\pgfpathlineto{\pgfqpoint{1.798535in}{0.596128in}}%
\pgfpathlineto{\pgfqpoint{1.799228in}{0.592014in}}%
\pgfpathlineto{\pgfqpoint{1.801307in}{0.581310in}}%
\pgfpathlineto{\pgfqpoint{1.801827in}{0.584914in}}%
\pgfpathlineto{\pgfqpoint{1.805637in}{0.615854in}}%
\pgfpathlineto{\pgfqpoint{1.805984in}{0.616167in}}%
\pgfpathlineto{\pgfqpoint{1.806330in}{0.614244in}}%
\pgfpathlineto{\pgfqpoint{1.806850in}{0.610028in}}%
\pgfpathlineto{\pgfqpoint{1.808062in}{0.612314in}}%
\pgfpathlineto{\pgfqpoint{1.811873in}{0.618237in}}%
\pgfpathlineto{\pgfqpoint{1.812047in}{0.618622in}}%
\pgfpathlineto{\pgfqpoint{1.812393in}{0.616226in}}%
\pgfpathlineto{\pgfqpoint{1.815511in}{0.604230in}}%
\pgfpathlineto{\pgfqpoint{1.815857in}{0.605042in}}%
\pgfpathlineto{\pgfqpoint{1.816723in}{0.602744in}}%
\pgfpathlineto{\pgfqpoint{1.817936in}{0.597307in}}%
\pgfpathlineto{\pgfqpoint{1.818629in}{0.599427in}}%
\pgfpathlineto{\pgfqpoint{1.821400in}{0.610076in}}%
\pgfpathlineto{\pgfqpoint{1.822093in}{0.607190in}}%
\pgfpathlineto{\pgfqpoint{1.822613in}{0.610367in}}%
\pgfpathlineto{\pgfqpoint{1.824345in}{0.619171in}}%
\pgfpathlineto{\pgfqpoint{1.825211in}{0.617996in}}%
\pgfpathlineto{\pgfqpoint{1.825384in}{0.618096in}}%
\pgfpathlineto{\pgfqpoint{1.825731in}{0.617067in}}%
\pgfpathlineto{\pgfqpoint{1.825904in}{0.616895in}}%
\pgfpathlineto{\pgfqpoint{1.827463in}{0.614110in}}%
\pgfpathlineto{\pgfqpoint{1.827810in}{0.614745in}}%
\pgfpathlineto{\pgfqpoint{1.829888in}{0.625948in}}%
\pgfpathlineto{\pgfqpoint{1.830235in}{0.625898in}}%
\pgfpathlineto{\pgfqpoint{1.830754in}{0.622866in}}%
\pgfpathlineto{\pgfqpoint{1.831447in}{0.626755in}}%
\pgfpathlineto{\pgfqpoint{1.832833in}{0.637059in}}%
\pgfpathlineto{\pgfqpoint{1.833699in}{0.632866in}}%
\pgfpathlineto{\pgfqpoint{1.835604in}{0.620372in}}%
\pgfpathlineto{\pgfqpoint{1.838549in}{0.603451in}}%
\pgfpathlineto{\pgfqpoint{1.838896in}{0.604462in}}%
\pgfpathlineto{\pgfqpoint{1.839762in}{0.606760in}}%
\pgfpathlineto{\pgfqpoint{1.840455in}{0.604702in}}%
\pgfpathlineto{\pgfqpoint{1.841321in}{0.601717in}}%
\pgfpathlineto{\pgfqpoint{1.841840in}{0.600061in}}%
\pgfpathlineto{\pgfqpoint{1.842533in}{0.602586in}}%
\pgfpathlineto{\pgfqpoint{1.843399in}{0.605976in}}%
\pgfpathlineto{\pgfqpoint{1.843919in}{0.603323in}}%
\pgfpathlineto{\pgfqpoint{1.848250in}{0.587658in}}%
\pgfpathlineto{\pgfqpoint{1.848596in}{0.587211in}}%
\pgfpathlineto{\pgfqpoint{1.849289in}{0.588394in}}%
\pgfpathlineto{\pgfqpoint{1.849982in}{0.590926in}}%
\pgfpathlineto{\pgfqpoint{1.850675in}{0.588552in}}%
\pgfpathlineto{\pgfqpoint{1.852060in}{0.582408in}}%
\pgfpathlineto{\pgfqpoint{1.852753in}{0.586056in}}%
\pgfpathlineto{\pgfqpoint{1.852927in}{0.586700in}}%
\pgfpathlineto{\pgfqpoint{1.853793in}{0.584682in}}%
\pgfpathlineto{\pgfqpoint{1.857257in}{0.566649in}}%
\pgfpathlineto{\pgfqpoint{1.858123in}{0.568685in}}%
\pgfpathlineto{\pgfqpoint{1.858296in}{0.569063in}}%
\pgfpathlineto{\pgfqpoint{1.859162in}{0.567266in}}%
\pgfpathlineto{\pgfqpoint{1.859336in}{0.566773in}}%
\pgfpathlineto{\pgfqpoint{1.859509in}{0.568987in}}%
\pgfpathlineto{\pgfqpoint{1.860895in}{0.578787in}}%
\pgfpathlineto{\pgfqpoint{1.861414in}{0.576146in}}%
\pgfpathlineto{\pgfqpoint{1.861761in}{0.576550in}}%
\pgfpathlineto{\pgfqpoint{1.861934in}{0.575588in}}%
\pgfpathlineto{\pgfqpoint{1.862280in}{0.572593in}}%
\pgfpathlineto{\pgfqpoint{1.863493in}{0.574258in}}%
\pgfpathlineto{\pgfqpoint{1.864186in}{0.577025in}}%
\pgfpathlineto{\pgfqpoint{1.865918in}{0.586319in}}%
\pgfpathlineto{\pgfqpoint{1.866438in}{0.583429in}}%
\pgfpathlineto{\pgfqpoint{1.868690in}{0.575036in}}%
\pgfpathlineto{\pgfqpoint{1.869209in}{0.576935in}}%
\pgfpathlineto{\pgfqpoint{1.869902in}{0.573632in}}%
\pgfpathlineto{\pgfqpoint{1.872500in}{0.565632in}}%
\pgfpathlineto{\pgfqpoint{1.870768in}{0.574295in}}%
\pgfpathlineto{\pgfqpoint{1.872674in}{0.567176in}}%
\pgfpathlineto{\pgfqpoint{1.874059in}{0.577447in}}%
\pgfpathlineto{\pgfqpoint{1.874925in}{0.574883in}}%
\pgfpathlineto{\pgfqpoint{1.875099in}{0.575876in}}%
\pgfpathlineto{\pgfqpoint{1.875965in}{0.573412in}}%
\pgfpathlineto{\pgfqpoint{1.878043in}{0.569885in}}%
\pgfpathlineto{\pgfqpoint{1.879429in}{0.577202in}}%
\pgfpathlineto{\pgfqpoint{1.880122in}{0.572680in}}%
\pgfpathlineto{\pgfqpoint{1.880642in}{0.568194in}}%
\pgfpathlineto{\pgfqpoint{1.881681in}{0.570942in}}%
\pgfpathlineto{\pgfqpoint{1.882374in}{0.571967in}}%
\pgfpathlineto{\pgfqpoint{1.882720in}{0.569746in}}%
\pgfpathlineto{\pgfqpoint{1.883933in}{0.564753in}}%
\pgfpathlineto{\pgfqpoint{1.884626in}{0.567555in}}%
\pgfpathlineto{\pgfqpoint{1.886704in}{0.569897in}}%
\pgfpathlineto{\pgfqpoint{1.886878in}{0.569144in}}%
\pgfpathlineto{\pgfqpoint{1.887051in}{0.567948in}}%
\pgfpathlineto{\pgfqpoint{1.887744in}{0.571674in}}%
\pgfpathlineto{\pgfqpoint{1.890169in}{0.587212in}}%
\pgfpathlineto{\pgfqpoint{1.890515in}{0.586718in}}%
\pgfpathlineto{\pgfqpoint{1.891035in}{0.584868in}}%
\pgfpathlineto{\pgfqpoint{1.891728in}{0.587408in}}%
\pgfpathlineto{\pgfqpoint{1.894673in}{0.594754in}}%
\pgfpathlineto{\pgfqpoint{1.895019in}{0.594458in}}%
\pgfpathlineto{\pgfqpoint{1.896058in}{0.594081in}}%
\pgfpathlineto{\pgfqpoint{1.895539in}{0.595281in}}%
\pgfpathlineto{\pgfqpoint{1.896405in}{0.594563in}}%
\pgfpathlineto{\pgfqpoint{1.897791in}{0.601596in}}%
\pgfpathlineto{\pgfqpoint{1.898483in}{0.597947in}}%
\pgfpathlineto{\pgfqpoint{1.902468in}{0.580094in}}%
\pgfpathlineto{\pgfqpoint{1.902641in}{0.580813in}}%
\pgfpathlineto{\pgfqpoint{1.906278in}{0.593366in}}%
\pgfpathlineto{\pgfqpoint{1.906798in}{0.591510in}}%
\pgfpathlineto{\pgfqpoint{1.907318in}{0.588588in}}%
\pgfpathlineto{\pgfqpoint{1.908184in}{0.591286in}}%
\pgfpathlineto{\pgfqpoint{1.909396in}{0.598110in}}%
\pgfpathlineto{\pgfqpoint{1.909916in}{0.594299in}}%
\pgfpathlineto{\pgfqpoint{1.911129in}{0.589305in}}%
\pgfpathlineto{\pgfqpoint{1.911821in}{0.589668in}}%
\pgfpathlineto{\pgfqpoint{1.912688in}{0.593562in}}%
\pgfpathlineto{\pgfqpoint{1.913900in}{0.592995in}}%
\pgfpathlineto{\pgfqpoint{1.914073in}{0.593437in}}%
\pgfpathlineto{\pgfqpoint{1.914593in}{0.592242in}}%
\pgfpathlineto{\pgfqpoint{1.915113in}{0.592992in}}%
\pgfpathlineto{\pgfqpoint{1.915632in}{0.591842in}}%
\pgfpathlineto{\pgfqpoint{1.916325in}{0.594484in}}%
\pgfpathlineto{\pgfqpoint{1.917884in}{0.594002in}}%
\pgfpathlineto{\pgfqpoint{1.918057in}{0.594684in}}%
\pgfpathlineto{\pgfqpoint{1.918231in}{0.595608in}}%
\pgfpathlineto{\pgfqpoint{1.918577in}{0.591232in}}%
\pgfpathlineto{\pgfqpoint{1.921522in}{0.574221in}}%
\pgfpathlineto{\pgfqpoint{1.922041in}{0.577747in}}%
\pgfpathlineto{\pgfqpoint{1.924813in}{0.594829in}}%
\pgfpathlineto{\pgfqpoint{1.925506in}{0.593746in}}%
\pgfpathlineto{\pgfqpoint{1.929317in}{0.564738in}}%
\pgfpathlineto{\pgfqpoint{1.929490in}{0.565070in}}%
\pgfpathlineto{\pgfqpoint{1.929663in}{0.564804in}}%
\pgfpathlineto{\pgfqpoint{1.930010in}{0.566614in}}%
\pgfpathlineto{\pgfqpoint{1.931222in}{0.570461in}}%
\pgfpathlineto{\pgfqpoint{1.931569in}{0.566998in}}%
\pgfpathlineto{\pgfqpoint{1.935033in}{0.557269in}}%
\pgfpathlineto{\pgfqpoint{1.936419in}{0.558951in}}%
\pgfpathlineto{\pgfqpoint{1.935899in}{0.556947in}}%
\pgfpathlineto{\pgfqpoint{1.936592in}{0.558069in}}%
\pgfpathlineto{\pgfqpoint{1.936765in}{0.557653in}}%
\pgfpathlineto{\pgfqpoint{1.937112in}{0.559633in}}%
\pgfpathlineto{\pgfqpoint{1.938324in}{0.563846in}}%
\pgfpathlineto{\pgfqpoint{1.938671in}{0.561273in}}%
\pgfpathlineto{\pgfqpoint{1.939710in}{0.556063in}}%
\pgfpathlineto{\pgfqpoint{1.942655in}{0.533487in}}%
\pgfpathlineto{\pgfqpoint{1.943174in}{0.538561in}}%
\pgfpathlineto{\pgfqpoint{1.945426in}{0.551052in}}%
\pgfpathlineto{\pgfqpoint{1.945946in}{0.553007in}}%
\pgfpathlineto{\pgfqpoint{1.946292in}{0.550264in}}%
\pgfpathlineto{\pgfqpoint{1.947158in}{0.551385in}}%
\pgfpathlineto{\pgfqpoint{1.948371in}{0.543103in}}%
\pgfpathlineto{\pgfqpoint{1.949064in}{0.546413in}}%
\pgfpathlineto{\pgfqpoint{1.951142in}{0.557985in}}%
\pgfpathlineto{\pgfqpoint{1.951316in}{0.557728in}}%
\pgfpathlineto{\pgfqpoint{1.952182in}{0.556912in}}%
\pgfpathlineto{\pgfqpoint{1.952528in}{0.558203in}}%
\pgfpathlineto{\pgfqpoint{1.955300in}{0.572517in}}%
\pgfpathlineto{\pgfqpoint{1.955993in}{0.573872in}}%
\pgfpathlineto{\pgfqpoint{1.959977in}{0.595448in}}%
\pgfpathlineto{\pgfqpoint{1.961189in}{0.591062in}}%
\pgfpathlineto{\pgfqpoint{1.961536in}{0.593214in}}%
\pgfpathlineto{\pgfqpoint{1.964134in}{0.603426in}}%
\pgfpathlineto{\pgfqpoint{1.967772in}{0.611115in}}%
\pgfpathlineto{\pgfqpoint{1.968118in}{0.610666in}}%
\pgfpathlineto{\pgfqpoint{1.972622in}{0.587869in}}%
\pgfpathlineto{\pgfqpoint{1.973488in}{0.593195in}}%
\pgfpathlineto{\pgfqpoint{1.975566in}{0.598021in}}%
\pgfpathlineto{\pgfqpoint{1.975740in}{0.597903in}}%
\pgfpathlineto{\pgfqpoint{1.976086in}{0.596688in}}%
\pgfpathlineto{\pgfqpoint{1.979031in}{0.585942in}}%
\pgfpathlineto{\pgfqpoint{1.982495in}{0.577531in}}%
\pgfpathlineto{\pgfqpoint{1.985440in}{0.584549in}}%
\pgfpathlineto{\pgfqpoint{1.985613in}{0.583538in}}%
\pgfpathlineto{\pgfqpoint{1.986306in}{0.586630in}}%
\pgfpathlineto{\pgfqpoint{1.987519in}{0.590369in}}%
\pgfpathlineto{\pgfqpoint{1.988212in}{0.589298in}}%
\pgfpathlineto{\pgfqpoint{1.991156in}{0.578314in}}%
\pgfpathlineto{\pgfqpoint{1.994794in}{0.589300in}}%
\pgfpathlineto{\pgfqpoint{1.997392in}{0.583438in}}%
\pgfpathlineto{\pgfqpoint{1.997565in}{0.584524in}}%
\pgfpathlineto{\pgfqpoint{1.998432in}{0.580592in}}%
\pgfpathlineto{\pgfqpoint{2.000164in}{0.582915in}}%
\pgfpathlineto{\pgfqpoint{2.002416in}{0.598238in}}%
\pgfpathlineto{\pgfqpoint{2.003801in}{0.594945in}}%
\pgfpathlineto{\pgfqpoint{2.004841in}{0.595557in}}%
\pgfpathlineto{\pgfqpoint{2.006573in}{0.603001in}}%
\pgfpathlineto{\pgfqpoint{2.007266in}{0.602200in}}%
\pgfpathlineto{\pgfqpoint{2.009171in}{0.598248in}}%
\pgfpathlineto{\pgfqpoint{2.013329in}{0.585111in}}%
\pgfpathlineto{\pgfqpoint{2.013848in}{0.585814in}}%
\pgfpathlineto{\pgfqpoint{2.014368in}{0.587553in}}%
\pgfpathlineto{\pgfqpoint{2.014887in}{0.585714in}}%
\pgfpathlineto{\pgfqpoint{2.020777in}{0.539833in}}%
\pgfpathlineto{\pgfqpoint{2.021643in}{0.541761in}}%
\pgfpathlineto{\pgfqpoint{2.028225in}{0.564916in}}%
\pgfpathlineto{\pgfqpoint{2.029438in}{0.574893in}}%
\pgfpathlineto{\pgfqpoint{2.030131in}{0.571432in}}%
\pgfpathlineto{\pgfqpoint{2.034288in}{0.543916in}}%
\pgfpathlineto{\pgfqpoint{2.034808in}{0.544370in}}%
\pgfpathlineto{\pgfqpoint{2.036367in}{0.555312in}}%
\pgfpathlineto{\pgfqpoint{2.036886in}{0.553725in}}%
\pgfpathlineto{\pgfqpoint{2.037233in}{0.554275in}}%
\pgfpathlineto{\pgfqpoint{2.039312in}{0.550316in}}%
\pgfpathlineto{\pgfqpoint{2.039831in}{0.546687in}}%
\pgfpathlineto{\pgfqpoint{2.040178in}{0.550496in}}%
\pgfpathlineto{\pgfqpoint{2.040524in}{0.552277in}}%
\pgfpathlineto{\pgfqpoint{2.041390in}{0.548276in}}%
\pgfpathlineto{\pgfqpoint{2.043469in}{0.534702in}}%
\pgfpathlineto{\pgfqpoint{2.043989in}{0.536501in}}%
\pgfpathlineto{\pgfqpoint{2.045028in}{0.534729in}}%
\pgfpathlineto{\pgfqpoint{2.045374in}{0.535998in}}%
\pgfpathlineto{\pgfqpoint{2.047106in}{0.547292in}}%
\pgfpathlineto{\pgfqpoint{2.048319in}{0.543939in}}%
\pgfpathlineto{\pgfqpoint{2.053689in}{0.566844in}}%
\pgfpathlineto{\pgfqpoint{2.054035in}{0.565735in}}%
\pgfpathlineto{\pgfqpoint{2.055941in}{0.566301in}}%
\pgfpathlineto{\pgfqpoint{2.057673in}{0.574027in}}%
\pgfpathlineto{\pgfqpoint{2.058366in}{0.570387in}}%
\pgfpathlineto{\pgfqpoint{2.059405in}{0.568091in}}%
\pgfpathlineto{\pgfqpoint{2.059925in}{0.569201in}}%
\pgfpathlineto{\pgfqpoint{2.061311in}{0.570979in}}%
\pgfpathlineto{\pgfqpoint{2.060618in}{0.568504in}}%
\pgfpathlineto{\pgfqpoint{2.061484in}{0.570243in}}%
\pgfpathlineto{\pgfqpoint{2.062870in}{0.565250in}}%
\pgfpathlineto{\pgfqpoint{2.063216in}{0.565893in}}%
\pgfpathlineto{\pgfqpoint{2.063909in}{0.571398in}}%
\pgfpathlineto{\pgfqpoint{2.064775in}{0.566895in}}%
\pgfpathlineto{\pgfqpoint{2.070145in}{0.551448in}}%
\pgfpathlineto{\pgfqpoint{2.071011in}{0.551948in}}%
\pgfpathlineto{\pgfqpoint{2.072050in}{0.559021in}}%
\pgfpathlineto{\pgfqpoint{2.072916in}{0.561573in}}%
\pgfpathlineto{\pgfqpoint{2.073609in}{0.560333in}}%
\pgfpathlineto{\pgfqpoint{2.074475in}{0.558949in}}%
\pgfpathlineto{\pgfqpoint{2.074822in}{0.560936in}}%
\pgfpathlineto{\pgfqpoint{2.077420in}{0.581081in}}%
\pgfpathlineto{\pgfqpoint{2.079672in}{0.578695in}}%
\pgfpathlineto{\pgfqpoint{2.082270in}{0.581958in}}%
\pgfpathlineto{\pgfqpoint{2.083310in}{0.586459in}}%
\pgfpathlineto{\pgfqpoint{2.084002in}{0.583496in}}%
\pgfpathlineto{\pgfqpoint{2.085042in}{0.581404in}}%
\pgfpathlineto{\pgfqpoint{2.085561in}{0.582921in}}%
\pgfpathlineto{\pgfqpoint{2.088853in}{0.590454in}}%
\pgfpathlineto{\pgfqpoint{2.089199in}{0.589729in}}%
\pgfpathlineto{\pgfqpoint{2.091797in}{0.582096in}}%
\pgfpathlineto{\pgfqpoint{2.092837in}{0.580273in}}%
\pgfpathlineto{\pgfqpoint{2.093876in}{0.581608in}}%
\pgfpathlineto{\pgfqpoint{2.094396in}{0.582687in}}%
\pgfpathlineto{\pgfqpoint{2.095089in}{0.580845in}}%
\pgfpathlineto{\pgfqpoint{2.095608in}{0.577574in}}%
\pgfpathlineto{\pgfqpoint{2.097167in}{0.568834in}}%
\pgfpathlineto{\pgfqpoint{2.098033in}{0.570001in}}%
\pgfpathlineto{\pgfqpoint{2.099592in}{0.566832in}}%
\pgfpathlineto{\pgfqpoint{2.098553in}{0.570337in}}%
\pgfpathlineto{\pgfqpoint{2.100285in}{0.567184in}}%
\pgfpathlineto{\pgfqpoint{2.100805in}{0.566008in}}%
\pgfpathlineto{\pgfqpoint{2.101498in}{0.568282in}}%
\pgfpathlineto{\pgfqpoint{2.102017in}{0.569766in}}%
\pgfpathlineto{\pgfqpoint{2.103403in}{0.568941in}}%
\pgfpathlineto{\pgfqpoint{2.104962in}{0.563380in}}%
\pgfpathlineto{\pgfqpoint{2.105828in}{0.564608in}}%
\pgfpathlineto{\pgfqpoint{2.106348in}{0.567307in}}%
\pgfpathlineto{\pgfqpoint{2.108946in}{0.579790in}}%
\pgfpathlineto{\pgfqpoint{2.109119in}{0.579167in}}%
\pgfpathlineto{\pgfqpoint{2.112411in}{0.566420in}}%
\pgfpathlineto{\pgfqpoint{2.112757in}{0.567717in}}%
\pgfpathlineto{\pgfqpoint{2.112930in}{0.567385in}}%
\pgfpathlineto{\pgfqpoint{2.113623in}{0.568809in}}%
\pgfpathlineto{\pgfqpoint{2.114143in}{0.568277in}}%
\pgfpathlineto{\pgfqpoint{2.114489in}{0.569495in}}%
\pgfpathlineto{\pgfqpoint{2.118300in}{0.599741in}}%
\pgfpathlineto{\pgfqpoint{2.124016in}{0.617606in}}%
\pgfpathlineto{\pgfqpoint{2.124190in}{0.617171in}}%
\pgfpathlineto{\pgfqpoint{2.125056in}{0.614542in}}%
\pgfpathlineto{\pgfqpoint{2.125748in}{0.616247in}}%
\pgfpathlineto{\pgfqpoint{2.126788in}{0.619090in}}%
\pgfpathlineto{\pgfqpoint{2.127481in}{0.617781in}}%
\pgfpathlineto{\pgfqpoint{2.131638in}{0.593718in}}%
\pgfpathlineto{\pgfqpoint{2.133543in}{0.576093in}}%
\pgfpathlineto{\pgfqpoint{2.134756in}{0.584607in}}%
\pgfpathlineto{\pgfqpoint{2.135968in}{0.592699in}}%
\pgfpathlineto{\pgfqpoint{2.138047in}{0.606879in}}%
\pgfpathlineto{\pgfqpoint{2.138913in}{0.601403in}}%
\pgfpathlineto{\pgfqpoint{2.139606in}{0.598560in}}%
\pgfpathlineto{\pgfqpoint{2.140299in}{0.601407in}}%
\pgfpathlineto{\pgfqpoint{2.140645in}{0.603658in}}%
\pgfpathlineto{\pgfqpoint{2.141858in}{0.601529in}}%
\pgfpathlineto{\pgfqpoint{2.144283in}{0.597337in}}%
\pgfpathlineto{\pgfqpoint{2.144630in}{0.598602in}}%
\pgfpathlineto{\pgfqpoint{2.145496in}{0.597540in}}%
\pgfpathlineto{\pgfqpoint{2.146535in}{0.591698in}}%
\pgfpathlineto{\pgfqpoint{2.147055in}{0.595077in}}%
\pgfpathlineto{\pgfqpoint{2.147401in}{0.596894in}}%
\pgfpathlineto{\pgfqpoint{2.148267in}{0.594224in}}%
\pgfpathlineto{\pgfqpoint{2.148614in}{0.595199in}}%
\pgfpathlineto{\pgfqpoint{2.149480in}{0.593895in}}%
\pgfpathlineto{\pgfqpoint{2.149999in}{0.592564in}}%
\pgfpathlineto{\pgfqpoint{2.150692in}{0.594311in}}%
\pgfpathlineto{\pgfqpoint{2.150865in}{0.595210in}}%
\pgfpathlineto{\pgfqpoint{2.151385in}{0.591979in}}%
\pgfpathlineto{\pgfqpoint{2.152424in}{0.592272in}}%
\pgfpathlineto{\pgfqpoint{2.154330in}{0.577927in}}%
\pgfpathlineto{\pgfqpoint{2.154676in}{0.580260in}}%
\pgfpathlineto{\pgfqpoint{2.156755in}{0.588735in}}%
\pgfpathlineto{\pgfqpoint{2.156928in}{0.588616in}}%
\pgfpathlineto{\pgfqpoint{2.157448in}{0.589406in}}%
\pgfpathlineto{\pgfqpoint{2.159353in}{0.588247in}}%
\pgfpathlineto{\pgfqpoint{2.160219in}{0.587848in}}%
\pgfpathlineto{\pgfqpoint{2.164896in}{0.608684in}}%
\pgfpathlineto{\pgfqpoint{2.165243in}{0.611343in}}%
\pgfpathlineto{\pgfqpoint{2.165936in}{0.607110in}}%
\pgfpathlineto{\pgfqpoint{2.170093in}{0.587880in}}%
\pgfpathlineto{\pgfqpoint{2.172345in}{0.587880in}}%
\pgfpathlineto{\pgfqpoint{2.173384in}{0.586386in}}%
\pgfpathlineto{\pgfqpoint{2.174077in}{0.588875in}}%
\pgfpathlineto{\pgfqpoint{2.175809in}{0.583941in}}%
\pgfpathlineto{\pgfqpoint{2.176329in}{0.585555in}}%
\pgfpathlineto{\pgfqpoint{2.177195in}{0.587369in}}%
\pgfpathlineto{\pgfqpoint{2.177541in}{0.585394in}}%
\pgfpathlineto{\pgfqpoint{2.180486in}{0.574374in}}%
\pgfpathlineto{\pgfqpoint{2.182392in}{0.579249in}}%
\pgfpathlineto{\pgfqpoint{2.182565in}{0.578650in}}%
\pgfpathlineto{\pgfqpoint{2.184643in}{0.567607in}}%
\pgfpathlineto{\pgfqpoint{2.185163in}{0.572215in}}%
\pgfpathlineto{\pgfqpoint{2.187068in}{0.577440in}}%
\pgfpathlineto{\pgfqpoint{2.188281in}{0.575140in}}%
\pgfpathlineto{\pgfqpoint{2.189494in}{0.573384in}}%
\pgfpathlineto{\pgfqpoint{2.189840in}{0.573634in}}%
\pgfpathlineto{\pgfqpoint{2.190360in}{0.580628in}}%
\pgfpathlineto{\pgfqpoint{2.191572in}{0.579589in}}%
\pgfpathlineto{\pgfqpoint{2.192612in}{0.580814in}}%
\pgfpathlineto{\pgfqpoint{2.193478in}{0.583636in}}%
\pgfpathlineto{\pgfqpoint{2.194171in}{0.581425in}}%
\pgfpathlineto{\pgfqpoint{2.194690in}{0.580782in}}%
\pgfpathlineto{\pgfqpoint{2.195556in}{0.581672in}}%
\pgfpathlineto{\pgfqpoint{2.195730in}{0.581806in}}%
\pgfpathlineto{\pgfqpoint{2.196249in}{0.580353in}}%
\pgfpathlineto{\pgfqpoint{2.196422in}{0.580235in}}%
\pgfpathlineto{\pgfqpoint{2.196942in}{0.581501in}}%
\pgfpathlineto{\pgfqpoint{2.197115in}{0.581516in}}%
\pgfpathlineto{\pgfqpoint{2.197635in}{0.582032in}}%
\pgfpathlineto{\pgfqpoint{2.198155in}{0.580514in}}%
\pgfpathlineto{\pgfqpoint{2.199887in}{0.575648in}}%
\pgfpathlineto{\pgfqpoint{2.200233in}{0.577085in}}%
\pgfpathlineto{\pgfqpoint{2.203351in}{0.588792in}}%
\pgfpathlineto{\pgfqpoint{2.204044in}{0.587223in}}%
\pgfpathlineto{\pgfqpoint{2.206642in}{0.589134in}}%
\pgfpathlineto{\pgfqpoint{2.207162in}{0.592369in}}%
\pgfpathlineto{\pgfqpoint{2.208201in}{0.590525in}}%
\pgfpathlineto{\pgfqpoint{2.209414in}{0.581469in}}%
\pgfpathlineto{\pgfqpoint{2.210280in}{0.583076in}}%
\pgfpathlineto{\pgfqpoint{2.213398in}{0.592910in}}%
\pgfpathlineto{\pgfqpoint{2.216343in}{0.582961in}}%
\pgfpathlineto{\pgfqpoint{2.216689in}{0.585840in}}%
\pgfpathlineto{\pgfqpoint{2.216862in}{0.586619in}}%
\pgfpathlineto{\pgfqpoint{2.217555in}{0.584443in}}%
\pgfpathlineto{\pgfqpoint{2.218075in}{0.585069in}}%
\pgfpathlineto{\pgfqpoint{2.221539in}{0.592000in}}%
\pgfpathlineto{\pgfqpoint{2.224831in}{0.611504in}}%
\pgfpathlineto{\pgfqpoint{2.225004in}{0.611877in}}%
\pgfpathlineto{\pgfqpoint{2.225350in}{0.609288in}}%
\pgfpathlineto{\pgfqpoint{2.226563in}{0.600064in}}%
\pgfpathlineto{\pgfqpoint{2.227602in}{0.601498in}}%
\pgfpathlineto{\pgfqpoint{2.230720in}{0.591961in}}%
\pgfpathlineto{\pgfqpoint{2.230893in}{0.593547in}}%
\pgfpathlineto{\pgfqpoint{2.233145in}{0.604231in}}%
\pgfpathlineto{\pgfqpoint{2.233665in}{0.600699in}}%
\pgfpathlineto{\pgfqpoint{2.236783in}{0.581418in}}%
\pgfpathlineto{\pgfqpoint{2.237649in}{0.577605in}}%
\pgfpathlineto{\pgfqpoint{2.238688in}{0.579565in}}%
\pgfpathlineto{\pgfqpoint{2.239035in}{0.581430in}}%
\pgfpathlineto{\pgfqpoint{2.239727in}{0.576680in}}%
\pgfpathlineto{\pgfqpoint{2.240767in}{0.573031in}}%
\pgfpathlineto{\pgfqpoint{2.241460in}{0.574858in}}%
\pgfpathlineto{\pgfqpoint{2.242845in}{0.576282in}}%
\pgfpathlineto{\pgfqpoint{2.242499in}{0.573970in}}%
\pgfpathlineto{\pgfqpoint{2.243019in}{0.576102in}}%
\pgfpathlineto{\pgfqpoint{2.243538in}{0.572255in}}%
\pgfpathlineto{\pgfqpoint{2.244231in}{0.562553in}}%
\pgfpathlineto{\pgfqpoint{2.245097in}{0.569573in}}%
\pgfpathlineto{\pgfqpoint{2.245444in}{0.571572in}}%
\pgfpathlineto{\pgfqpoint{2.246310in}{0.567660in}}%
\pgfpathlineto{\pgfqpoint{2.246656in}{0.567678in}}%
\pgfpathlineto{\pgfqpoint{2.246829in}{0.568270in}}%
\pgfpathlineto{\pgfqpoint{2.247349in}{0.571275in}}%
\pgfpathlineto{\pgfqpoint{2.248388in}{0.568557in}}%
\pgfpathlineto{\pgfqpoint{2.249255in}{0.563625in}}%
\pgfpathlineto{\pgfqpoint{2.249947in}{0.566390in}}%
\pgfpathlineto{\pgfqpoint{2.250987in}{0.575258in}}%
\pgfpathlineto{\pgfqpoint{2.251853in}{0.573279in}}%
\pgfpathlineto{\pgfqpoint{2.252026in}{0.573831in}}%
\pgfpathlineto{\pgfqpoint{2.253065in}{0.572126in}}%
\pgfpathlineto{\pgfqpoint{2.253412in}{0.570420in}}%
\pgfpathlineto{\pgfqpoint{2.255491in}{0.556413in}}%
\pgfpathlineto{\pgfqpoint{2.255664in}{0.556820in}}%
\pgfpathlineto{\pgfqpoint{2.256703in}{0.558839in}}%
\pgfpathlineto{\pgfqpoint{2.256876in}{0.556098in}}%
\pgfpathlineto{\pgfqpoint{2.257742in}{0.553544in}}%
\pgfpathlineto{\pgfqpoint{2.258262in}{0.556334in}}%
\pgfpathlineto{\pgfqpoint{2.260514in}{0.562479in}}%
\pgfpathlineto{\pgfqpoint{2.261900in}{0.558710in}}%
\pgfpathlineto{\pgfqpoint{2.262246in}{0.560552in}}%
\pgfpathlineto{\pgfqpoint{2.262593in}{0.560895in}}%
\pgfpathlineto{\pgfqpoint{2.262939in}{0.558485in}}%
\pgfpathlineto{\pgfqpoint{2.265191in}{0.548651in}}%
\pgfpathlineto{\pgfqpoint{2.265537in}{0.550217in}}%
\pgfpathlineto{\pgfqpoint{2.270561in}{0.570395in}}%
\pgfpathlineto{\pgfqpoint{2.274025in}{0.585013in}}%
\pgfpathlineto{\pgfqpoint{2.274198in}{0.584578in}}%
\pgfpathlineto{\pgfqpoint{2.275064in}{0.581003in}}%
\pgfpathlineto{\pgfqpoint{2.275584in}{0.584736in}}%
\pgfpathlineto{\pgfqpoint{2.277836in}{0.589426in}}%
\pgfpathlineto{\pgfqpoint{2.278182in}{0.588476in}}%
\pgfpathlineto{\pgfqpoint{2.278702in}{0.591096in}}%
\pgfpathlineto{\pgfqpoint{2.279395in}{0.593461in}}%
\pgfpathlineto{\pgfqpoint{2.280261in}{0.592732in}}%
\pgfpathlineto{\pgfqpoint{2.280607in}{0.591896in}}%
\pgfpathlineto{\pgfqpoint{2.281300in}{0.593033in}}%
\pgfpathlineto{\pgfqpoint{2.282859in}{0.596937in}}%
\pgfpathlineto{\pgfqpoint{2.283033in}{0.596599in}}%
\pgfpathlineto{\pgfqpoint{2.284938in}{0.593264in}}%
\pgfpathlineto{\pgfqpoint{2.285111in}{0.593731in}}%
\pgfpathlineto{\pgfqpoint{2.288749in}{0.606380in}}%
\pgfpathlineto{\pgfqpoint{2.289268in}{0.604495in}}%
\pgfpathlineto{\pgfqpoint{2.293945in}{0.591049in}}%
\pgfpathlineto{\pgfqpoint{2.294465in}{0.590319in}}%
\pgfpathlineto{\pgfqpoint{2.294985in}{0.591165in}}%
\pgfpathlineto{\pgfqpoint{2.296544in}{0.595203in}}%
\pgfpathlineto{\pgfqpoint{2.296890in}{0.594588in}}%
\pgfpathlineto{\pgfqpoint{2.298796in}{0.583802in}}%
\pgfpathlineto{\pgfqpoint{2.299488in}{0.586880in}}%
\pgfpathlineto{\pgfqpoint{2.301047in}{0.602269in}}%
\pgfpathlineto{\pgfqpoint{2.303819in}{0.809319in}}%
\pgfpathlineto{\pgfqpoint{2.307457in}{0.969897in}}%
\pgfpathlineto{\pgfqpoint{2.310921in}{1.024412in}}%
\pgfpathlineto{\pgfqpoint{2.314732in}{1.044457in}}%
\pgfpathlineto{\pgfqpoint{2.316291in}{1.045777in}}%
\pgfpathlineto{\pgfqpoint{2.319062in}{1.047168in}}%
\pgfpathlineto{\pgfqpoint{2.319236in}{1.046798in}}%
\pgfpathlineto{\pgfqpoint{2.322873in}{1.038928in}}%
\pgfpathlineto{\pgfqpoint{2.324432in}{1.035489in}}%
\pgfpathlineto{\pgfqpoint{2.329629in}{1.020852in}}%
\pgfpathlineto{\pgfqpoint{2.330148in}{1.020125in}}%
\pgfpathlineto{\pgfqpoint{2.334479in}{1.002856in}}%
\pgfpathlineto{\pgfqpoint{2.340542in}{0.980078in}}%
\pgfpathlineto{\pgfqpoint{2.342447in}{0.982592in}}%
\pgfpathlineto{\pgfqpoint{2.344526in}{0.981357in}}%
\pgfpathlineto{\pgfqpoint{2.344872in}{0.982194in}}%
\pgfpathlineto{\pgfqpoint{2.345738in}{0.984808in}}%
\pgfpathlineto{\pgfqpoint{2.346431in}{0.982405in}}%
\pgfpathlineto{\pgfqpoint{2.350588in}{0.971054in}}%
\pgfpathlineto{\pgfqpoint{2.351801in}{0.965538in}}%
\pgfpathlineto{\pgfqpoint{2.352667in}{0.966034in}}%
\pgfpathlineto{\pgfqpoint{2.353360in}{0.966027in}}%
\pgfpathlineto{\pgfqpoint{2.353533in}{0.965283in}}%
\pgfpathlineto{\pgfqpoint{2.357517in}{0.951456in}}%
\pgfpathlineto{\pgfqpoint{2.357690in}{0.951490in}}%
\pgfpathlineto{\pgfqpoint{2.359596in}{0.949584in}}%
\pgfpathlineto{\pgfqpoint{2.364619in}{0.933996in}}%
\pgfpathlineto{\pgfqpoint{2.365659in}{0.929953in}}%
\pgfpathlineto{\pgfqpoint{2.368777in}{0.920933in}}%
\pgfpathlineto{\pgfqpoint{2.369296in}{0.919817in}}%
\pgfpathlineto{\pgfqpoint{2.371895in}{0.911754in}}%
\pgfpathlineto{\pgfqpoint{2.372241in}{0.912404in}}%
\pgfpathlineto{\pgfqpoint{2.372414in}{0.912671in}}%
\pgfpathlineto{\pgfqpoint{2.373280in}{0.911510in}}%
\pgfpathlineto{\pgfqpoint{2.379516in}{0.894722in}}%
\pgfpathlineto{\pgfqpoint{2.382634in}{0.882514in}}%
\pgfpathlineto{\pgfqpoint{2.386272in}{0.873746in}}%
\pgfpathlineto{\pgfqpoint{2.386445in}{0.874017in}}%
\pgfpathlineto{\pgfqpoint{2.388697in}{0.871311in}}%
\pgfpathlineto{\pgfqpoint{2.389563in}{0.868050in}}%
\pgfpathlineto{\pgfqpoint{2.390256in}{0.869414in}}%
\pgfpathlineto{\pgfqpoint{2.390602in}{0.870870in}}%
\pgfpathlineto{\pgfqpoint{2.391468in}{0.868501in}}%
\pgfpathlineto{\pgfqpoint{2.394067in}{0.863555in}}%
\pgfpathlineto{\pgfqpoint{2.394240in}{0.863742in}}%
\pgfpathlineto{\pgfqpoint{2.395626in}{0.866439in}}%
\pgfpathlineto{\pgfqpoint{2.396145in}{0.864861in}}%
\pgfpathlineto{\pgfqpoint{2.399610in}{0.844294in}}%
\pgfpathlineto{\pgfqpoint{2.400476in}{0.839160in}}%
\pgfpathlineto{\pgfqpoint{2.402555in}{0.834313in}}%
\pgfpathlineto{\pgfqpoint{2.403594in}{0.835359in}}%
\pgfpathlineto{\pgfqpoint{2.403940in}{0.834345in}}%
\pgfpathlineto{\pgfqpoint{2.407405in}{0.818887in}}%
\pgfpathlineto{\pgfqpoint{2.409657in}{0.818507in}}%
\pgfpathlineto{\pgfqpoint{2.410003in}{0.819175in}}%
\pgfpathlineto{\pgfqpoint{2.410696in}{0.817589in}}%
\pgfpathlineto{\pgfqpoint{2.413641in}{0.807779in}}%
\pgfpathlineto{\pgfqpoint{2.414853in}{0.809859in}}%
\pgfpathlineto{\pgfqpoint{2.417625in}{0.817182in}}%
\pgfpathlineto{\pgfqpoint{2.419010in}{0.814970in}}%
\pgfpathlineto{\pgfqpoint{2.424207in}{0.799214in}}%
\pgfpathlineto{\pgfqpoint{2.425420in}{0.802414in}}%
\pgfpathlineto{\pgfqpoint{2.426632in}{0.806873in}}%
\pgfpathlineto{\pgfqpoint{2.427152in}{0.804847in}}%
\pgfpathlineto{\pgfqpoint{2.431482in}{0.793286in}}%
\pgfpathlineto{\pgfqpoint{2.431829in}{0.793436in}}%
\pgfpathlineto{\pgfqpoint{2.432002in}{0.793877in}}%
\pgfpathlineto{\pgfqpoint{2.432522in}{0.792111in}}%
\pgfpathlineto{\pgfqpoint{2.433561in}{0.788182in}}%
\pgfpathlineto{\pgfqpoint{2.434081in}{0.791152in}}%
\pgfpathlineto{\pgfqpoint{2.436333in}{0.798753in}}%
\pgfpathlineto{\pgfqpoint{2.439104in}{0.810671in}}%
\pgfpathlineto{\pgfqpoint{2.437545in}{0.797874in}}%
\pgfpathlineto{\pgfqpoint{2.439797in}{0.809795in}}%
\pgfpathlineto{\pgfqpoint{2.442049in}{0.804623in}}%
\pgfpathlineto{\pgfqpoint{2.442395in}{0.804987in}}%
\pgfpathlineto{\pgfqpoint{2.444127in}{0.811389in}}%
\pgfpathlineto{\pgfqpoint{2.444994in}{0.814017in}}%
\pgfpathlineto{\pgfqpoint{2.445340in}{0.810943in}}%
\pgfpathlineto{\pgfqpoint{2.447245in}{0.807093in}}%
\pgfpathlineto{\pgfqpoint{2.447419in}{0.808008in}}%
\pgfpathlineto{\pgfqpoint{2.449151in}{0.816309in}}%
\pgfpathlineto{\pgfqpoint{2.449670in}{0.815077in}}%
\pgfpathlineto{\pgfqpoint{2.455214in}{0.795384in}}%
\pgfpathlineto{\pgfqpoint{2.455387in}{0.795622in}}%
\pgfpathlineto{\pgfqpoint{2.457292in}{0.800090in}}%
\pgfpathlineto{\pgfqpoint{2.457465in}{0.799222in}}%
\pgfpathlineto{\pgfqpoint{2.458331in}{0.794101in}}%
\pgfpathlineto{\pgfqpoint{2.459544in}{0.796001in}}%
\pgfpathlineto{\pgfqpoint{2.461449in}{0.792666in}}%
\pgfpathlineto{\pgfqpoint{2.470110in}{0.759210in}}%
\pgfpathlineto{\pgfqpoint{2.470803in}{0.761803in}}%
\pgfpathlineto{\pgfqpoint{2.472536in}{0.766611in}}%
\pgfpathlineto{\pgfqpoint{2.472882in}{0.764775in}}%
\pgfpathlineto{\pgfqpoint{2.475134in}{0.755891in}}%
\pgfpathlineto{\pgfqpoint{2.475480in}{0.756494in}}%
\pgfpathlineto{\pgfqpoint{2.475827in}{0.755222in}}%
\pgfpathlineto{\pgfqpoint{2.478425in}{0.738695in}}%
\pgfpathlineto{\pgfqpoint{2.478945in}{0.739431in}}%
\pgfpathlineto{\pgfqpoint{2.482409in}{0.749167in}}%
\pgfpathlineto{\pgfqpoint{2.482582in}{0.748140in}}%
\pgfpathlineto{\pgfqpoint{2.484834in}{0.741534in}}%
\pgfpathlineto{\pgfqpoint{2.485354in}{0.743003in}}%
\pgfpathlineto{\pgfqpoint{2.488299in}{0.752169in}}%
\pgfpathlineto{\pgfqpoint{2.488472in}{0.750613in}}%
\pgfpathlineto{\pgfqpoint{2.489511in}{0.754067in}}%
\pgfpathlineto{\pgfqpoint{2.491243in}{0.757274in}}%
\pgfpathlineto{\pgfqpoint{2.491763in}{0.756867in}}%
\pgfpathlineto{\pgfqpoint{2.492283in}{0.755952in}}%
\pgfpathlineto{\pgfqpoint{2.492976in}{0.757675in}}%
\pgfpathlineto{\pgfqpoint{2.494361in}{0.764497in}}%
\pgfpathlineto{\pgfqpoint{2.495054in}{0.763226in}}%
\pgfpathlineto{\pgfqpoint{2.499385in}{0.740306in}}%
\pgfpathlineto{\pgfqpoint{2.500944in}{0.743176in}}%
\pgfpathlineto{\pgfqpoint{2.504755in}{0.759810in}}%
\pgfpathlineto{\pgfqpoint{2.501637in}{0.742112in}}%
\pgfpathlineto{\pgfqpoint{2.505621in}{0.757728in}}%
\pgfpathlineto{\pgfqpoint{2.505967in}{0.757511in}}%
\pgfpathlineto{\pgfqpoint{2.506140in}{0.758438in}}%
\pgfpathlineto{\pgfqpoint{2.506487in}{0.760833in}}%
\pgfpathlineto{\pgfqpoint{2.507353in}{0.756397in}}%
\pgfpathlineto{\pgfqpoint{2.507526in}{0.755731in}}%
\pgfpathlineto{\pgfqpoint{2.508046in}{0.757939in}}%
\pgfpathlineto{\pgfqpoint{2.508565in}{0.757748in}}%
\pgfpathlineto{\pgfqpoint{2.509431in}{0.755457in}}%
\pgfpathlineto{\pgfqpoint{2.510298in}{0.758280in}}%
\pgfpathlineto{\pgfqpoint{2.510817in}{0.758951in}}%
\pgfpathlineto{\pgfqpoint{2.510990in}{0.757097in}}%
\pgfpathlineto{\pgfqpoint{2.513416in}{0.749438in}}%
\pgfpathlineto{\pgfqpoint{2.513589in}{0.750310in}}%
\pgfpathlineto{\pgfqpoint{2.514108in}{0.752686in}}%
\pgfpathlineto{\pgfqpoint{2.515321in}{0.751390in}}%
\pgfpathlineto{\pgfqpoint{2.517226in}{0.742433in}}%
\pgfpathlineto{\pgfqpoint{2.517746in}{0.738166in}}%
\pgfpathlineto{\pgfqpoint{2.518612in}{0.742609in}}%
\pgfpathlineto{\pgfqpoint{2.518785in}{0.743267in}}%
\pgfpathlineto{\pgfqpoint{2.519825in}{0.741217in}}%
\pgfpathlineto{\pgfqpoint{2.520171in}{0.739144in}}%
\pgfpathlineto{\pgfqpoint{2.520691in}{0.741925in}}%
\pgfpathlineto{\pgfqpoint{2.521210in}{0.741281in}}%
\pgfpathlineto{\pgfqpoint{2.521903in}{0.747572in}}%
\pgfpathlineto{\pgfqpoint{2.522769in}{0.743500in}}%
\pgfpathlineto{\pgfqpoint{2.522943in}{0.742190in}}%
\pgfpathlineto{\pgfqpoint{2.523462in}{0.744578in}}%
\pgfpathlineto{\pgfqpoint{2.523982in}{0.744237in}}%
\pgfpathlineto{\pgfqpoint{2.525541in}{0.749703in}}%
\pgfpathlineto{\pgfqpoint{2.525714in}{0.748406in}}%
\pgfpathlineto{\pgfqpoint{2.532816in}{0.709298in}}%
\pgfpathlineto{\pgfqpoint{2.533163in}{0.711680in}}%
\pgfpathlineto{\pgfqpoint{2.533336in}{0.712541in}}%
\pgfpathlineto{\pgfqpoint{2.533856in}{0.709366in}}%
\pgfpathlineto{\pgfqpoint{2.534548in}{0.704662in}}%
\pgfpathlineto{\pgfqpoint{2.535588in}{0.706059in}}%
\pgfpathlineto{\pgfqpoint{2.535934in}{0.707646in}}%
\pgfpathlineto{\pgfqpoint{2.536974in}{0.720518in}}%
\pgfpathlineto{\pgfqpoint{2.537840in}{0.716675in}}%
\pgfpathlineto{\pgfqpoint{2.538186in}{0.715503in}}%
\pgfpathlineto{\pgfqpoint{2.538533in}{0.717657in}}%
\pgfpathlineto{\pgfqpoint{2.539225in}{0.725230in}}%
\pgfpathlineto{\pgfqpoint{2.539918in}{0.716397in}}%
\pgfpathlineto{\pgfqpoint{2.540438in}{0.709930in}}%
\pgfpathlineto{\pgfqpoint{2.541650in}{0.710518in}}%
\pgfpathlineto{\pgfqpoint{2.546847in}{0.750723in}}%
\pgfpathlineto{\pgfqpoint{2.551004in}{0.772716in}}%
\pgfpathlineto{\pgfqpoint{2.552217in}{0.769278in}}%
\pgfpathlineto{\pgfqpoint{2.553603in}{0.755584in}}%
\pgfpathlineto{\pgfqpoint{2.554988in}{0.757451in}}%
\pgfpathlineto{\pgfqpoint{2.555681in}{0.760912in}}%
\pgfpathlineto{\pgfqpoint{2.556894in}{0.760089in}}%
\pgfpathlineto{\pgfqpoint{2.559492in}{0.766965in}}%
\pgfpathlineto{\pgfqpoint{2.559839in}{0.764890in}}%
\pgfpathlineto{\pgfqpoint{2.560012in}{0.764587in}}%
\pgfpathlineto{\pgfqpoint{2.560358in}{0.766285in}}%
\pgfpathlineto{\pgfqpoint{2.560531in}{0.765961in}}%
\pgfpathlineto{\pgfqpoint{2.561744in}{0.768421in}}%
\pgfpathlineto{\pgfqpoint{2.562090in}{0.766265in}}%
\pgfpathlineto{\pgfqpoint{2.566594in}{0.740359in}}%
\pgfpathlineto{\pgfqpoint{2.567634in}{0.739009in}}%
\pgfpathlineto{\pgfqpoint{2.567807in}{0.740385in}}%
\pgfpathlineto{\pgfqpoint{2.567980in}{0.741356in}}%
\pgfpathlineto{\pgfqpoint{2.568846in}{0.738474in}}%
\pgfpathlineto{\pgfqpoint{2.572137in}{0.724323in}}%
\pgfpathlineto{\pgfqpoint{2.573350in}{0.716119in}}%
\pgfpathlineto{\pgfqpoint{2.573696in}{0.721049in}}%
\pgfpathlineto{\pgfqpoint{2.574909in}{0.728807in}}%
\pgfpathlineto{\pgfqpoint{2.575428in}{0.725053in}}%
\pgfpathlineto{\pgfqpoint{2.577161in}{0.723034in}}%
\pgfpathlineto{\pgfqpoint{2.577507in}{0.723866in}}%
\pgfpathlineto{\pgfqpoint{2.579759in}{0.729338in}}%
\pgfpathlineto{\pgfqpoint{2.581145in}{0.724515in}}%
\pgfpathlineto{\pgfqpoint{2.581838in}{0.725973in}}%
\pgfpathlineto{\pgfqpoint{2.582184in}{0.724850in}}%
\pgfpathlineto{\pgfqpoint{2.582704in}{0.723101in}}%
\pgfpathlineto{\pgfqpoint{2.583223in}{0.725794in}}%
\pgfpathlineto{\pgfqpoint{2.583397in}{0.725396in}}%
\pgfpathlineto{\pgfqpoint{2.584782in}{0.734510in}}%
\pgfpathlineto{\pgfqpoint{2.585648in}{0.731431in}}%
\pgfpathlineto{\pgfqpoint{2.588593in}{0.720468in}}%
\pgfpathlineto{\pgfqpoint{2.589459in}{0.717554in}}%
\pgfpathlineto{\pgfqpoint{2.590672in}{0.714828in}}%
\pgfpathlineto{\pgfqpoint{2.591018in}{0.715901in}}%
\pgfpathlineto{\pgfqpoint{2.591538in}{0.721584in}}%
\pgfpathlineto{\pgfqpoint{2.592231in}{0.713955in}}%
\pgfpathlineto{\pgfqpoint{2.594309in}{0.705367in}}%
\pgfpathlineto{\pgfqpoint{2.594483in}{0.708688in}}%
\pgfpathlineto{\pgfqpoint{2.596388in}{0.714194in}}%
\pgfpathlineto{\pgfqpoint{2.597081in}{0.714499in}}%
\pgfpathlineto{\pgfqpoint{2.599506in}{0.726752in}}%
\pgfpathlineto{\pgfqpoint{2.599679in}{0.726169in}}%
\pgfpathlineto{\pgfqpoint{2.600892in}{0.723208in}}%
\pgfpathlineto{\pgfqpoint{2.602278in}{0.721380in}}%
\pgfpathlineto{\pgfqpoint{2.601758in}{0.724266in}}%
\pgfpathlineto{\pgfqpoint{2.602451in}{0.721822in}}%
\pgfpathlineto{\pgfqpoint{2.604356in}{0.738640in}}%
\pgfpathlineto{\pgfqpoint{2.605569in}{0.732074in}}%
\pgfpathlineto{\pgfqpoint{2.607647in}{0.726590in}}%
\pgfpathlineto{\pgfqpoint{2.608860in}{0.712601in}}%
\pgfpathlineto{\pgfqpoint{2.610419in}{0.715433in}}%
\pgfpathlineto{\pgfqpoint{2.610939in}{0.718982in}}%
\pgfpathlineto{\pgfqpoint{2.611631in}{0.712609in}}%
\pgfpathlineto{\pgfqpoint{2.612498in}{0.717622in}}%
\pgfpathlineto{\pgfqpoint{2.612844in}{0.720658in}}%
\pgfpathlineto{\pgfqpoint{2.613710in}{0.716422in}}%
\pgfpathlineto{\pgfqpoint{2.613883in}{0.716453in}}%
\pgfpathlineto{\pgfqpoint{2.614749in}{0.705650in}}%
\pgfpathlineto{\pgfqpoint{2.616135in}{0.708928in}}%
\pgfpathlineto{\pgfqpoint{2.616655in}{0.707863in}}%
\pgfpathlineto{\pgfqpoint{2.618387in}{0.694426in}}%
\pgfpathlineto{\pgfqpoint{2.619253in}{0.698552in}}%
\pgfpathlineto{\pgfqpoint{2.619946in}{0.695793in}}%
\pgfpathlineto{\pgfqpoint{2.620119in}{0.696232in}}%
\pgfpathlineto{\pgfqpoint{2.620812in}{0.693444in}}%
\pgfpathlineto{\pgfqpoint{2.621332in}{0.698815in}}%
\pgfpathlineto{\pgfqpoint{2.621505in}{0.699020in}}%
\pgfpathlineto{\pgfqpoint{2.621678in}{0.697229in}}%
\pgfpathlineto{\pgfqpoint{2.622025in}{0.690251in}}%
\pgfpathlineto{\pgfqpoint{2.622718in}{0.700507in}}%
\pgfpathlineto{\pgfqpoint{2.624969in}{0.719417in}}%
\pgfpathlineto{\pgfqpoint{2.625143in}{0.717602in}}%
\pgfpathlineto{\pgfqpoint{2.626009in}{0.721550in}}%
\pgfpathlineto{\pgfqpoint{2.627395in}{0.734007in}}%
\pgfpathlineto{\pgfqpoint{2.628607in}{0.733310in}}%
\pgfpathlineto{\pgfqpoint{2.630339in}{0.718142in}}%
\pgfpathlineto{\pgfqpoint{2.631898in}{0.722397in}}%
\pgfpathlineto{\pgfqpoint{2.632938in}{0.726201in}}%
\pgfpathlineto{\pgfqpoint{2.633457in}{0.722928in}}%
\pgfpathlineto{\pgfqpoint{2.633804in}{0.721332in}}%
\pgfpathlineto{\pgfqpoint{2.634670in}{0.722686in}}%
\pgfpathlineto{\pgfqpoint{2.635189in}{0.727996in}}%
\pgfpathlineto{\pgfqpoint{2.636056in}{0.722539in}}%
\pgfpathlineto{\pgfqpoint{2.636748in}{0.719696in}}%
\pgfpathlineto{\pgfqpoint{2.639174in}{0.708931in}}%
\pgfpathlineto{\pgfqpoint{2.640559in}{0.704199in}}%
\pgfpathlineto{\pgfqpoint{2.641079in}{0.706440in}}%
\pgfpathlineto{\pgfqpoint{2.641425in}{0.708307in}}%
\pgfpathlineto{\pgfqpoint{2.642291in}{0.705198in}}%
\pgfpathlineto{\pgfqpoint{2.642465in}{0.705196in}}%
\pgfpathlineto{\pgfqpoint{2.642638in}{0.705792in}}%
\pgfpathlineto{\pgfqpoint{2.643158in}{0.702511in}}%
\pgfpathlineto{\pgfqpoint{2.643331in}{0.702308in}}%
\pgfpathlineto{\pgfqpoint{2.643677in}{0.704176in}}%
\pgfpathlineto{\pgfqpoint{2.644197in}{0.703634in}}%
\pgfpathlineto{\pgfqpoint{2.645583in}{0.708645in}}%
\pgfpathlineto{\pgfqpoint{2.645929in}{0.706916in}}%
\pgfpathlineto{\pgfqpoint{2.646622in}{0.700835in}}%
\pgfpathlineto{\pgfqpoint{2.647315in}{0.706050in}}%
\pgfpathlineto{\pgfqpoint{2.649394in}{0.711975in}}%
\pgfpathlineto{\pgfqpoint{2.651126in}{0.704634in}}%
\pgfpathlineto{\pgfqpoint{2.651472in}{0.704895in}}%
\pgfpathlineto{\pgfqpoint{2.652511in}{0.698530in}}%
\pgfpathlineto{\pgfqpoint{2.653378in}{0.703953in}}%
\pgfpathlineto{\pgfqpoint{2.655976in}{0.715434in}}%
\pgfpathlineto{\pgfqpoint{2.653724in}{0.703713in}}%
\pgfpathlineto{\pgfqpoint{2.656669in}{0.714120in}}%
\pgfpathlineto{\pgfqpoint{2.657362in}{0.717307in}}%
\pgfpathlineto{\pgfqpoint{2.658401in}{0.714800in}}%
\pgfpathlineto{\pgfqpoint{2.659094in}{0.711152in}}%
\pgfpathlineto{\pgfqpoint{2.659614in}{0.716897in}}%
\pgfpathlineto{\pgfqpoint{2.659960in}{0.713346in}}%
\pgfpathlineto{\pgfqpoint{2.660306in}{0.714443in}}%
\pgfpathlineto{\pgfqpoint{2.660480in}{0.713550in}}%
\pgfpathlineto{\pgfqpoint{2.662385in}{0.717250in}}%
\pgfpathlineto{\pgfqpoint{2.662558in}{0.717778in}}%
\pgfpathlineto{\pgfqpoint{2.662731in}{0.714858in}}%
\pgfpathlineto{\pgfqpoint{2.664464in}{0.705987in}}%
\pgfpathlineto{\pgfqpoint{2.663251in}{0.716313in}}%
\pgfpathlineto{\pgfqpoint{2.664810in}{0.707041in}}%
\pgfpathlineto{\pgfqpoint{2.665330in}{0.707900in}}%
\pgfpathlineto{\pgfqpoint{2.666023in}{0.705923in}}%
\pgfpathlineto{\pgfqpoint{2.667062in}{0.703397in}}%
\pgfpathlineto{\pgfqpoint{2.667408in}{0.705659in}}%
\pgfpathlineto{\pgfqpoint{2.667755in}{0.708085in}}%
\pgfpathlineto{\pgfqpoint{2.668621in}{0.703978in}}%
\pgfpathlineto{\pgfqpoint{2.671219in}{0.688571in}}%
\pgfpathlineto{\pgfqpoint{2.671566in}{0.692687in}}%
\pgfpathlineto{\pgfqpoint{2.676069in}{0.709204in}}%
\pgfpathlineto{\pgfqpoint{2.672085in}{0.692564in}}%
\pgfpathlineto{\pgfqpoint{2.676243in}{0.707894in}}%
\pgfpathlineto{\pgfqpoint{2.677975in}{0.691500in}}%
\pgfpathlineto{\pgfqpoint{2.678841in}{0.692650in}}%
\pgfpathlineto{\pgfqpoint{2.680053in}{0.695486in}}%
\pgfpathlineto{\pgfqpoint{2.679361in}{0.690830in}}%
\pgfpathlineto{\pgfqpoint{2.680400in}{0.692455in}}%
\pgfpathlineto{\pgfqpoint{2.681439in}{0.689654in}}%
\pgfpathlineto{\pgfqpoint{2.681786in}{0.692046in}}%
\pgfpathlineto{\pgfqpoint{2.682652in}{0.697962in}}%
\pgfpathlineto{\pgfqpoint{2.683345in}{0.692937in}}%
\pgfpathlineto{\pgfqpoint{2.685770in}{0.678132in}}%
\pgfpathlineto{\pgfqpoint{2.687675in}{0.689661in}}%
\pgfpathlineto{\pgfqpoint{2.688368in}{0.701063in}}%
\pgfpathlineto{\pgfqpoint{2.689581in}{0.698247in}}%
\pgfpathlineto{\pgfqpoint{2.689754in}{0.698725in}}%
\pgfpathlineto{\pgfqpoint{2.690100in}{0.695553in}}%
\pgfpathlineto{\pgfqpoint{2.690447in}{0.692943in}}%
\pgfpathlineto{\pgfqpoint{2.690966in}{0.700461in}}%
\pgfpathlineto{\pgfqpoint{2.692525in}{0.705413in}}%
\pgfpathlineto{\pgfqpoint{2.692699in}{0.704546in}}%
\pgfpathlineto{\pgfqpoint{2.692872in}{0.704770in}}%
\pgfpathlineto{\pgfqpoint{2.693218in}{0.703099in}}%
\pgfpathlineto{\pgfqpoint{2.696163in}{0.676354in}}%
\pgfpathlineto{\pgfqpoint{2.697549in}{0.684529in}}%
\pgfpathlineto{\pgfqpoint{2.698588in}{0.687805in}}%
\pgfpathlineto{\pgfqpoint{2.698068in}{0.683811in}}%
\pgfpathlineto{\pgfqpoint{2.698935in}{0.685099in}}%
\pgfpathlineto{\pgfqpoint{2.700147in}{0.679280in}}%
\pgfpathlineto{\pgfqpoint{2.700667in}{0.679397in}}%
\pgfpathlineto{\pgfqpoint{2.701879in}{0.671863in}}%
\pgfpathlineto{\pgfqpoint{2.702399in}{0.673991in}}%
\pgfpathlineto{\pgfqpoint{2.703438in}{0.689818in}}%
\pgfpathlineto{\pgfqpoint{2.704131in}{0.683044in}}%
\pgfpathlineto{\pgfqpoint{2.704478in}{0.677675in}}%
\pgfpathlineto{\pgfqpoint{2.704997in}{0.686945in}}%
\pgfpathlineto{\pgfqpoint{2.705344in}{0.685887in}}%
\pgfpathlineto{\pgfqpoint{2.705690in}{0.685954in}}%
\pgfpathlineto{\pgfqpoint{2.705863in}{0.687421in}}%
\pgfpathlineto{\pgfqpoint{2.706556in}{0.685648in}}%
\pgfpathlineto{\pgfqpoint{2.709501in}{0.703277in}}%
\pgfpathlineto{\pgfqpoint{2.711060in}{0.706980in}}%
\pgfpathlineto{\pgfqpoint{2.712792in}{0.696414in}}%
\pgfpathlineto{\pgfqpoint{2.714178in}{0.698670in}}%
\pgfpathlineto{\pgfqpoint{2.714871in}{0.695818in}}%
\pgfpathlineto{\pgfqpoint{2.717296in}{0.709515in}}%
\pgfpathlineto{\pgfqpoint{2.718855in}{0.705257in}}%
\pgfpathlineto{\pgfqpoint{2.719028in}{0.705383in}}%
\pgfpathlineto{\pgfqpoint{2.720241in}{0.712675in}}%
\pgfpathlineto{\pgfqpoint{2.721107in}{0.709652in}}%
\pgfpathlineto{\pgfqpoint{2.721280in}{0.708334in}}%
\pgfpathlineto{\pgfqpoint{2.722146in}{0.712326in}}%
\pgfpathlineto{\pgfqpoint{2.722666in}{0.713199in}}%
\pgfpathlineto{\pgfqpoint{2.723185in}{0.711163in}}%
\pgfpathlineto{\pgfqpoint{2.724744in}{0.706621in}}%
\pgfpathlineto{\pgfqpoint{2.724918in}{0.706811in}}%
\pgfpathlineto{\pgfqpoint{2.725437in}{0.710275in}}%
\pgfpathlineto{\pgfqpoint{2.726130in}{0.706248in}}%
\pgfpathlineto{\pgfqpoint{2.726303in}{0.701459in}}%
\pgfpathlineto{\pgfqpoint{2.727516in}{0.708945in}}%
\pgfpathlineto{\pgfqpoint{2.729421in}{0.714191in}}%
\pgfpathlineto{\pgfqpoint{2.729768in}{0.713317in}}%
\pgfpathlineto{\pgfqpoint{2.731500in}{0.706564in}}%
\pgfpathlineto{\pgfqpoint{2.731673in}{0.708086in}}%
\pgfpathlineto{\pgfqpoint{2.732539in}{0.704525in}}%
\pgfpathlineto{\pgfqpoint{2.734098in}{0.699031in}}%
\pgfpathlineto{\pgfqpoint{2.734445in}{0.699491in}}%
\pgfpathlineto{\pgfqpoint{2.736004in}{0.689412in}}%
\pgfpathlineto{\pgfqpoint{2.736870in}{0.695494in}}%
\pgfpathlineto{\pgfqpoint{2.737389in}{0.697296in}}%
\pgfpathlineto{\pgfqpoint{2.737736in}{0.693557in}}%
\pgfpathlineto{\pgfqpoint{2.739988in}{0.679619in}}%
\pgfpathlineto{\pgfqpoint{2.740161in}{0.681439in}}%
\pgfpathlineto{\pgfqpoint{2.740854in}{0.675461in}}%
\pgfpathlineto{\pgfqpoint{2.741200in}{0.676752in}}%
\pgfpathlineto{\pgfqpoint{2.745531in}{0.693300in}}%
\pgfpathlineto{\pgfqpoint{2.746050in}{0.689731in}}%
\pgfpathlineto{\pgfqpoint{2.747956in}{0.682394in}}%
\pgfpathlineto{\pgfqpoint{2.748476in}{0.678912in}}%
\pgfpathlineto{\pgfqpoint{2.749168in}{0.683520in}}%
\pgfpathlineto{\pgfqpoint{2.751247in}{0.693039in}}%
\pgfpathlineto{\pgfqpoint{2.753326in}{0.704349in}}%
\pgfpathlineto{\pgfqpoint{2.754019in}{0.696955in}}%
\pgfpathlineto{\pgfqpoint{2.756097in}{0.687663in}}%
\pgfpathlineto{\pgfqpoint{2.758349in}{0.675951in}}%
\pgfpathlineto{\pgfqpoint{2.758522in}{0.677954in}}%
\pgfpathlineto{\pgfqpoint{2.760428in}{0.689758in}}%
\pgfpathlineto{\pgfqpoint{2.760947in}{0.688075in}}%
\pgfpathlineto{\pgfqpoint{2.761640in}{0.686204in}}%
\pgfpathlineto{\pgfqpoint{2.761987in}{0.688336in}}%
\pgfpathlineto{\pgfqpoint{2.764412in}{0.713886in}}%
\pgfpathlineto{\pgfqpoint{2.764758in}{0.712016in}}%
\pgfpathlineto{\pgfqpoint{2.765451in}{0.712112in}}%
\pgfpathlineto{\pgfqpoint{2.767010in}{0.704616in}}%
\pgfpathlineto{\pgfqpoint{2.768742in}{0.707663in}}%
\pgfpathlineto{\pgfqpoint{2.767876in}{0.704052in}}%
\pgfpathlineto{\pgfqpoint{2.769089in}{0.706465in}}%
\pgfpathlineto{\pgfqpoint{2.769782in}{0.701766in}}%
\pgfpathlineto{\pgfqpoint{2.770648in}{0.705179in}}%
\pgfpathlineto{\pgfqpoint{2.770994in}{0.704335in}}%
\pgfpathlineto{\pgfqpoint{2.771341in}{0.706067in}}%
\pgfpathlineto{\pgfqpoint{2.771687in}{0.705776in}}%
\pgfpathlineto{\pgfqpoint{2.772900in}{0.714603in}}%
\pgfpathlineto{\pgfqpoint{2.773246in}{0.709052in}}%
\pgfpathlineto{\pgfqpoint{2.773592in}{0.705421in}}%
\pgfpathlineto{\pgfqpoint{2.774285in}{0.713932in}}%
\pgfpathlineto{\pgfqpoint{2.774459in}{0.713813in}}%
\pgfpathlineto{\pgfqpoint{2.774632in}{0.715110in}}%
\pgfpathlineto{\pgfqpoint{2.775325in}{0.717588in}}%
\pgfpathlineto{\pgfqpoint{2.776191in}{0.716765in}}%
\pgfpathlineto{\pgfqpoint{2.776364in}{0.714015in}}%
\pgfpathlineto{\pgfqpoint{2.777577in}{0.717449in}}%
\pgfpathlineto{\pgfqpoint{2.780695in}{0.731841in}}%
\pgfpathlineto{\pgfqpoint{2.781387in}{0.734554in}}%
\pgfpathlineto{\pgfqpoint{2.782427in}{0.738187in}}%
\pgfpathlineto{\pgfqpoint{2.782946in}{0.735484in}}%
\pgfpathlineto{\pgfqpoint{2.783812in}{0.728119in}}%
\pgfpathlineto{\pgfqpoint{2.784852in}{0.732116in}}%
\pgfpathlineto{\pgfqpoint{2.785025in}{0.731985in}}%
\pgfpathlineto{\pgfqpoint{2.785198in}{0.733191in}}%
\pgfpathlineto{\pgfqpoint{2.785371in}{0.733232in}}%
\pgfpathlineto{\pgfqpoint{2.785545in}{0.734013in}}%
\pgfpathlineto{\pgfqpoint{2.785891in}{0.731697in}}%
\pgfpathlineto{\pgfqpoint{2.786238in}{0.727718in}}%
\pgfpathlineto{\pgfqpoint{2.787104in}{0.734944in}}%
\pgfpathlineto{\pgfqpoint{2.787450in}{0.736287in}}%
\pgfpathlineto{\pgfqpoint{2.788316in}{0.734694in}}%
\pgfpathlineto{\pgfqpoint{2.796284in}{0.685657in}}%
\pgfpathlineto{\pgfqpoint{2.796977in}{0.685827in}}%
\pgfpathlineto{\pgfqpoint{2.797150in}{0.687080in}}%
\pgfpathlineto{\pgfqpoint{2.798017in}{0.682689in}}%
\pgfpathlineto{\pgfqpoint{2.799402in}{0.672490in}}%
\pgfpathlineto{\pgfqpoint{2.800095in}{0.677946in}}%
\pgfpathlineto{\pgfqpoint{2.801134in}{0.679717in}}%
\pgfpathlineto{\pgfqpoint{2.801654in}{0.678461in}}%
\pgfpathlineto{\pgfqpoint{2.801827in}{0.677904in}}%
\pgfpathlineto{\pgfqpoint{2.802347in}{0.680790in}}%
\pgfpathlineto{\pgfqpoint{2.802867in}{0.682037in}}%
\pgfpathlineto{\pgfqpoint{2.803040in}{0.680605in}}%
\pgfpathlineto{\pgfqpoint{2.803560in}{0.675365in}}%
\pgfpathlineto{\pgfqpoint{2.804426in}{0.681114in}}%
\pgfpathlineto{\pgfqpoint{2.804599in}{0.680880in}}%
\pgfpathlineto{\pgfqpoint{2.804945in}{0.682079in}}%
\pgfpathlineto{\pgfqpoint{2.805292in}{0.681460in}}%
\pgfpathlineto{\pgfqpoint{2.806158in}{0.683480in}}%
\pgfpathlineto{\pgfqpoint{2.805811in}{0.680808in}}%
\pgfpathlineto{\pgfqpoint{2.806678in}{0.681156in}}%
\pgfpathlineto{\pgfqpoint{2.807197in}{0.677742in}}%
\pgfpathlineto{\pgfqpoint{2.808063in}{0.679998in}}%
\pgfpathlineto{\pgfqpoint{2.809622in}{0.684224in}}%
\pgfpathlineto{\pgfqpoint{2.808929in}{0.678552in}}%
\pgfpathlineto{\pgfqpoint{2.809796in}{0.683205in}}%
\pgfpathlineto{\pgfqpoint{2.811181in}{0.668700in}}%
\pgfpathlineto{\pgfqpoint{2.811528in}{0.673289in}}%
\pgfpathlineto{\pgfqpoint{2.812221in}{0.682139in}}%
\pgfpathlineto{\pgfqpoint{2.812913in}{0.673852in}}%
\pgfpathlineto{\pgfqpoint{2.816031in}{0.658824in}}%
\pgfpathlineto{\pgfqpoint{2.813433in}{0.675454in}}%
\pgfpathlineto{\pgfqpoint{2.816378in}{0.660401in}}%
\pgfpathlineto{\pgfqpoint{2.820362in}{0.690789in}}%
\pgfpathlineto{\pgfqpoint{2.821228in}{0.695164in}}%
\pgfpathlineto{\pgfqpoint{2.821748in}{0.690758in}}%
\pgfpathlineto{\pgfqpoint{2.821921in}{0.690124in}}%
\pgfpathlineto{\pgfqpoint{2.822441in}{0.693798in}}%
\pgfpathlineto{\pgfqpoint{2.824519in}{0.697468in}}%
\pgfpathlineto{\pgfqpoint{2.824866in}{0.698401in}}%
\pgfpathlineto{\pgfqpoint{2.825905in}{0.687949in}}%
\pgfpathlineto{\pgfqpoint{2.826425in}{0.693469in}}%
\pgfpathlineto{\pgfqpoint{2.826944in}{0.700001in}}%
\pgfpathlineto{\pgfqpoint{2.828157in}{0.698691in}}%
\pgfpathlineto{\pgfqpoint{2.829023in}{0.698047in}}%
\pgfpathlineto{\pgfqpoint{2.828503in}{0.699054in}}%
\pgfpathlineto{\pgfqpoint{2.829543in}{0.699260in}}%
\pgfpathlineto{\pgfqpoint{2.830582in}{0.701533in}}%
\pgfpathlineto{\pgfqpoint{2.830928in}{0.699304in}}%
\pgfpathlineto{\pgfqpoint{2.831794in}{0.695673in}}%
\pgfpathlineto{\pgfqpoint{2.832314in}{0.699821in}}%
\pgfpathlineto{\pgfqpoint{2.833180in}{0.699303in}}%
\pgfpathlineto{\pgfqpoint{2.835432in}{0.682086in}}%
\pgfpathlineto{\pgfqpoint{2.833873in}{0.700994in}}%
\pgfpathlineto{\pgfqpoint{2.835952in}{0.682816in}}%
\pgfpathlineto{\pgfqpoint{2.836645in}{0.685418in}}%
\pgfpathlineto{\pgfqpoint{2.836991in}{0.680590in}}%
\pgfpathlineto{\pgfqpoint{2.837164in}{0.679112in}}%
\pgfpathlineto{\pgfqpoint{2.837857in}{0.686243in}}%
\pgfpathlineto{\pgfqpoint{2.840629in}{0.695408in}}%
\pgfpathlineto{\pgfqpoint{2.841495in}{0.702436in}}%
\pgfpathlineto{\pgfqpoint{2.842014in}{0.695981in}}%
\pgfpathlineto{\pgfqpoint{2.844440in}{0.674424in}}%
\pgfpathlineto{\pgfqpoint{2.844786in}{0.676170in}}%
\pgfpathlineto{\pgfqpoint{2.845825in}{0.684948in}}%
\pgfpathlineto{\pgfqpoint{2.846865in}{0.682042in}}%
\pgfpathlineto{\pgfqpoint{2.847211in}{0.680550in}}%
\pgfpathlineto{\pgfqpoint{2.848077in}{0.682974in}}%
\pgfpathlineto{\pgfqpoint{2.848250in}{0.682083in}}%
\pgfpathlineto{\pgfqpoint{2.849636in}{0.691073in}}%
\pgfpathlineto{\pgfqpoint{2.849983in}{0.687523in}}%
\pgfpathlineto{\pgfqpoint{2.850156in}{0.686439in}}%
\pgfpathlineto{\pgfqpoint{2.851022in}{0.690629in}}%
\pgfpathlineto{\pgfqpoint{2.851195in}{0.691661in}}%
\pgfpathlineto{\pgfqpoint{2.851715in}{0.686107in}}%
\pgfpathlineto{\pgfqpoint{2.855872in}{0.657180in}}%
\pgfpathlineto{\pgfqpoint{2.856219in}{0.660381in}}%
\pgfpathlineto{\pgfqpoint{2.859510in}{0.686907in}}%
\pgfpathlineto{\pgfqpoint{2.859856in}{0.682417in}}%
\pgfpathlineto{\pgfqpoint{2.860722in}{0.678671in}}%
\pgfpathlineto{\pgfqpoint{2.861242in}{0.683908in}}%
\pgfpathlineto{\pgfqpoint{2.861762in}{0.684253in}}%
\pgfpathlineto{\pgfqpoint{2.864360in}{0.662246in}}%
\pgfpathlineto{\pgfqpoint{2.864880in}{0.659532in}}%
\pgfpathlineto{\pgfqpoint{2.865226in}{0.657164in}}%
\pgfpathlineto{\pgfqpoint{2.866265in}{0.661517in}}%
\pgfpathlineto{\pgfqpoint{2.869557in}{0.680806in}}%
\pgfpathlineto{\pgfqpoint{2.869903in}{0.680493in}}%
\pgfpathlineto{\pgfqpoint{2.870076in}{0.677638in}}%
\pgfpathlineto{\pgfqpoint{2.870942in}{0.683341in}}%
\pgfpathlineto{\pgfqpoint{2.871462in}{0.687891in}}%
\pgfpathlineto{\pgfqpoint{2.871982in}{0.680573in}}%
\pgfpathlineto{\pgfqpoint{2.872328in}{0.683215in}}%
\pgfpathlineto{\pgfqpoint{2.872848in}{0.677648in}}%
\pgfpathlineto{\pgfqpoint{2.873021in}{0.675766in}}%
\pgfpathlineto{\pgfqpoint{2.874060in}{0.679427in}}%
\pgfpathlineto{\pgfqpoint{2.876139in}{0.689323in}}%
\pgfpathlineto{\pgfqpoint{2.876832in}{0.686791in}}%
\pgfpathlineto{\pgfqpoint{2.879084in}{0.674894in}}%
\pgfpathlineto{\pgfqpoint{2.879257in}{0.675835in}}%
\pgfpathlineto{\pgfqpoint{2.880816in}{0.673343in}}%
\pgfpathlineto{\pgfqpoint{2.879603in}{0.676944in}}%
\pgfpathlineto{\pgfqpoint{2.880989in}{0.675402in}}%
\pgfpathlineto{\pgfqpoint{2.881336in}{0.677796in}}%
\pgfpathlineto{\pgfqpoint{2.881682in}{0.673753in}}%
\pgfpathlineto{\pgfqpoint{2.882548in}{0.676879in}}%
\pgfpathlineto{\pgfqpoint{2.882721in}{0.675797in}}%
\pgfpathlineto{\pgfqpoint{2.883241in}{0.678107in}}%
\pgfpathlineto{\pgfqpoint{2.883934in}{0.677688in}}%
\pgfpathlineto{\pgfqpoint{2.885320in}{0.689155in}}%
\pgfpathlineto{\pgfqpoint{2.886186in}{0.688672in}}%
\pgfpathlineto{\pgfqpoint{2.886705in}{0.686161in}}%
\pgfpathlineto{\pgfqpoint{2.887052in}{0.681911in}}%
\pgfpathlineto{\pgfqpoint{2.887745in}{0.690097in}}%
\pgfpathlineto{\pgfqpoint{2.887918in}{0.689827in}}%
\pgfpathlineto{\pgfqpoint{2.888091in}{0.691771in}}%
\pgfpathlineto{\pgfqpoint{2.888784in}{0.685994in}}%
\pgfpathlineto{\pgfqpoint{2.889477in}{0.686511in}}%
\pgfpathlineto{\pgfqpoint{2.890689in}{0.681383in}}%
\pgfpathlineto{\pgfqpoint{2.890863in}{0.681159in}}%
\pgfpathlineto{\pgfqpoint{2.891209in}{0.682247in}}%
\pgfpathlineto{\pgfqpoint{2.893114in}{0.688772in}}%
\pgfpathlineto{\pgfqpoint{2.893461in}{0.687098in}}%
\pgfpathlineto{\pgfqpoint{2.898311in}{0.665216in}}%
\pgfpathlineto{\pgfqpoint{2.898831in}{0.667387in}}%
\pgfpathlineto{\pgfqpoint{2.902295in}{0.680422in}}%
\pgfpathlineto{\pgfqpoint{2.902468in}{0.680384in}}%
\pgfpathlineto{\pgfqpoint{2.904893in}{0.673084in}}%
\pgfpathlineto{\pgfqpoint{2.906799in}{0.682641in}}%
\pgfpathlineto{\pgfqpoint{2.907145in}{0.679971in}}%
\pgfpathlineto{\pgfqpoint{2.909224in}{0.671073in}}%
\pgfpathlineto{\pgfqpoint{2.909397in}{0.671618in}}%
\pgfpathlineto{\pgfqpoint{2.913381in}{0.684292in}}%
\pgfpathlineto{\pgfqpoint{2.913554in}{0.683691in}}%
\pgfpathlineto{\pgfqpoint{2.916153in}{0.658190in}}%
\pgfpathlineto{\pgfqpoint{2.917192in}{0.656005in}}%
\pgfpathlineto{\pgfqpoint{2.917885in}{0.658133in}}%
\pgfpathlineto{\pgfqpoint{2.918405in}{0.664004in}}%
\pgfpathlineto{\pgfqpoint{2.919444in}{0.661011in}}%
\pgfpathlineto{\pgfqpoint{2.919964in}{0.657616in}}%
\pgfpathlineto{\pgfqpoint{2.920830in}{0.662175in}}%
\pgfpathlineto{\pgfqpoint{2.923774in}{0.669658in}}%
\pgfpathlineto{\pgfqpoint{2.921176in}{0.661868in}}%
\pgfpathlineto{\pgfqpoint{2.923948in}{0.668239in}}%
\pgfpathlineto{\pgfqpoint{2.925507in}{0.662654in}}%
\pgfpathlineto{\pgfqpoint{2.925680in}{0.663633in}}%
\pgfpathlineto{\pgfqpoint{2.927412in}{0.677724in}}%
\pgfpathlineto{\pgfqpoint{2.927759in}{0.672353in}}%
\pgfpathlineto{\pgfqpoint{2.927932in}{0.670806in}}%
\pgfpathlineto{\pgfqpoint{2.928625in}{0.676324in}}%
\pgfpathlineto{\pgfqpoint{2.928798in}{0.676196in}}%
\pgfpathlineto{\pgfqpoint{2.930010in}{0.672680in}}%
\pgfpathlineto{\pgfqpoint{2.929144in}{0.677727in}}%
\pgfpathlineto{\pgfqpoint{2.930357in}{0.676069in}}%
\pgfpathlineto{\pgfqpoint{2.932609in}{0.689446in}}%
\pgfpathlineto{\pgfqpoint{2.930877in}{0.675257in}}%
\pgfpathlineto{\pgfqpoint{2.934341in}{0.687327in}}%
\pgfpathlineto{\pgfqpoint{2.937459in}{0.659073in}}%
\pgfpathlineto{\pgfqpoint{2.937632in}{0.659864in}}%
\pgfpathlineto{\pgfqpoint{2.940404in}{0.672865in}}%
\pgfpathlineto{\pgfqpoint{2.940577in}{0.670513in}}%
\pgfpathlineto{\pgfqpoint{2.941097in}{0.675312in}}%
\pgfpathlineto{\pgfqpoint{2.941789in}{0.674888in}}%
\pgfpathlineto{\pgfqpoint{2.944734in}{0.697643in}}%
\pgfpathlineto{\pgfqpoint{2.945947in}{0.690910in}}%
\pgfpathlineto{\pgfqpoint{2.948199in}{0.676790in}}%
\pgfpathlineto{\pgfqpoint{2.948545in}{0.680330in}}%
\pgfpathlineto{\pgfqpoint{2.948891in}{0.689427in}}%
\pgfpathlineto{\pgfqpoint{2.950104in}{0.685682in}}%
\pgfpathlineto{\pgfqpoint{2.950277in}{0.685802in}}%
\pgfpathlineto{\pgfqpoint{2.950450in}{0.684401in}}%
\pgfpathlineto{\pgfqpoint{2.951836in}{0.669887in}}%
\pgfpathlineto{\pgfqpoint{2.952356in}{0.670085in}}%
\pgfpathlineto{\pgfqpoint{2.954261in}{0.690181in}}%
\pgfpathlineto{\pgfqpoint{2.956340in}{0.676001in}}%
\pgfpathlineto{\pgfqpoint{2.956513in}{0.676430in}}%
\pgfpathlineto{\pgfqpoint{2.957552in}{0.674979in}}%
\pgfpathlineto{\pgfqpoint{2.958592in}{0.673003in}}%
\pgfpathlineto{\pgfqpoint{2.958938in}{0.674935in}}%
\pgfpathlineto{\pgfqpoint{2.959631in}{0.678995in}}%
\pgfpathlineto{\pgfqpoint{2.961537in}{0.685128in}}%
\pgfpathlineto{\pgfqpoint{2.963442in}{0.692024in}}%
\pgfpathlineto{\pgfqpoint{2.961883in}{0.684515in}}%
\pgfpathlineto{\pgfqpoint{2.963615in}{0.690167in}}%
\pgfpathlineto{\pgfqpoint{2.964135in}{0.683902in}}%
\pgfpathlineto{\pgfqpoint{2.964828in}{0.690948in}}%
\pgfpathlineto{\pgfqpoint{2.965347in}{0.693869in}}%
\pgfpathlineto{\pgfqpoint{2.965867in}{0.688749in}}%
\pgfpathlineto{\pgfqpoint{2.966387in}{0.691134in}}%
\pgfpathlineto{\pgfqpoint{2.967946in}{0.696501in}}%
\pgfpathlineto{\pgfqpoint{2.969851in}{0.707869in}}%
\pgfpathlineto{\pgfqpoint{2.970024in}{0.707256in}}%
\pgfpathlineto{\pgfqpoint{2.970371in}{0.706269in}}%
\pgfpathlineto{\pgfqpoint{2.971583in}{0.701322in}}%
\pgfpathlineto{\pgfqpoint{2.971930in}{0.705606in}}%
\pgfpathlineto{\pgfqpoint{2.972623in}{0.707439in}}%
\pgfpathlineto{\pgfqpoint{2.973142in}{0.703051in}}%
\pgfpathlineto{\pgfqpoint{2.973489in}{0.701385in}}%
\pgfpathlineto{\pgfqpoint{2.974182in}{0.705066in}}%
\pgfpathlineto{\pgfqpoint{2.975567in}{0.705622in}}%
\pgfpathlineto{\pgfqpoint{2.975221in}{0.703466in}}%
\pgfpathlineto{\pgfqpoint{2.975741in}{0.705020in}}%
\pgfpathlineto{\pgfqpoint{2.977473in}{0.699796in}}%
\pgfpathlineto{\pgfqpoint{2.976607in}{0.707635in}}%
\pgfpathlineto{\pgfqpoint{2.977992in}{0.701595in}}%
\pgfpathlineto{\pgfqpoint{2.978339in}{0.701391in}}%
\pgfpathlineto{\pgfqpoint{2.978512in}{0.700341in}}%
\pgfpathlineto{\pgfqpoint{2.978859in}{0.698612in}}%
\pgfpathlineto{\pgfqpoint{2.979378in}{0.695522in}}%
\pgfpathlineto{\pgfqpoint{2.980244in}{0.700130in}}%
\pgfpathlineto{\pgfqpoint{2.981977in}{0.705819in}}%
\pgfpathlineto{\pgfqpoint{2.982150in}{0.704999in}}%
\pgfpathlineto{\pgfqpoint{2.982323in}{0.703878in}}%
\pgfpathlineto{\pgfqpoint{2.983189in}{0.706228in}}%
\pgfpathlineto{\pgfqpoint{2.983362in}{0.705965in}}%
\pgfpathlineto{\pgfqpoint{2.985094in}{0.710684in}}%
\pgfpathlineto{\pgfqpoint{2.986134in}{0.708122in}}%
\pgfpathlineto{\pgfqpoint{2.987346in}{0.706069in}}%
\pgfpathlineto{\pgfqpoint{2.987866in}{0.707186in}}%
\pgfpathlineto{\pgfqpoint{2.988212in}{0.708168in}}%
\pgfpathlineto{\pgfqpoint{2.988732in}{0.706898in}}%
\pgfpathlineto{\pgfqpoint{2.989425in}{0.698542in}}%
\pgfpathlineto{\pgfqpoint{2.990464in}{0.704087in}}%
\pgfpathlineto{\pgfqpoint{2.990638in}{0.704847in}}%
\pgfpathlineto{\pgfqpoint{2.991157in}{0.702097in}}%
\pgfpathlineto{\pgfqpoint{2.991850in}{0.703827in}}%
\pgfpathlineto{\pgfqpoint{2.993929in}{0.695335in}}%
\pgfpathlineto{\pgfqpoint{2.994275in}{0.696812in}}%
\pgfpathlineto{\pgfqpoint{2.996527in}{0.706509in}}%
\pgfpathlineto{\pgfqpoint{2.996700in}{0.705254in}}%
\pgfpathlineto{\pgfqpoint{2.997220in}{0.711095in}}%
\pgfpathlineto{\pgfqpoint{2.999472in}{0.722483in}}%
\pgfpathlineto{\pgfqpoint{2.997740in}{0.707568in}}%
\pgfpathlineto{\pgfqpoint{3.000165in}{0.718935in}}%
\pgfpathlineto{\pgfqpoint{3.001724in}{0.711807in}}%
\pgfpathlineto{\pgfqpoint{3.002070in}{0.713339in}}%
\pgfpathlineto{\pgfqpoint{3.004322in}{0.734817in}}%
\pgfpathlineto{\pgfqpoint{3.005188in}{0.732031in}}%
\pgfpathlineto{\pgfqpoint{3.006054in}{0.728314in}}%
\pgfpathlineto{\pgfqpoint{3.006574in}{0.731574in}}%
\pgfpathlineto{\pgfqpoint{3.008133in}{0.740490in}}%
\pgfpathlineto{\pgfqpoint{3.008999in}{0.739700in}}%
\pgfpathlineto{\pgfqpoint{3.011424in}{0.730476in}}%
\pgfpathlineto{\pgfqpoint{3.012290in}{0.731105in}}%
\pgfpathlineto{\pgfqpoint{3.011944in}{0.730340in}}%
\pgfpathlineto{\pgfqpoint{3.012463in}{0.730527in}}%
\pgfpathlineto{\pgfqpoint{3.012810in}{0.727661in}}%
\pgfpathlineto{\pgfqpoint{3.013849in}{0.732105in}}%
\pgfpathlineto{\pgfqpoint{3.015928in}{0.741136in}}%
\pgfpathlineto{\pgfqpoint{3.016101in}{0.740905in}}%
\pgfpathlineto{\pgfqpoint{3.016621in}{0.741806in}}%
\pgfpathlineto{\pgfqpoint{3.019565in}{0.757298in}}%
\pgfpathlineto{\pgfqpoint{3.024415in}{0.731392in}}%
\pgfpathlineto{\pgfqpoint{3.025628in}{0.723902in}}%
\pgfpathlineto{\pgfqpoint{3.028053in}{0.711359in}}%
\pgfpathlineto{\pgfqpoint{3.030651in}{0.736058in}}%
\pgfpathlineto{\pgfqpoint{3.031691in}{0.733101in}}%
\pgfpathlineto{\pgfqpoint{3.032210in}{0.730233in}}%
\pgfpathlineto{\pgfqpoint{3.033250in}{0.732295in}}%
\pgfpathlineto{\pgfqpoint{3.033423in}{0.732688in}}%
\pgfpathlineto{\pgfqpoint{3.033769in}{0.731540in}}%
\pgfpathlineto{\pgfqpoint{3.034982in}{0.715021in}}%
\pgfpathlineto{\pgfqpoint{3.035848in}{0.722064in}}%
\pgfpathlineto{\pgfqpoint{3.036368in}{0.724231in}}%
\pgfpathlineto{\pgfqpoint{3.038100in}{0.730380in}}%
\pgfpathlineto{\pgfqpoint{3.039312in}{0.738146in}}%
\pgfpathlineto{\pgfqpoint{3.040179in}{0.732491in}}%
\pgfpathlineto{\pgfqpoint{3.040871in}{0.729992in}}%
\pgfpathlineto{\pgfqpoint{3.041391in}{0.733344in}}%
\pgfpathlineto{\pgfqpoint{3.042257in}{0.743341in}}%
\pgfpathlineto{\pgfqpoint{3.043296in}{0.736495in}}%
\pgfpathlineto{\pgfqpoint{3.044682in}{0.731685in}}%
\pgfpathlineto{\pgfqpoint{3.045029in}{0.734505in}}%
\pgfpathlineto{\pgfqpoint{3.045895in}{0.734643in}}%
\pgfpathlineto{\pgfqpoint{3.046068in}{0.733295in}}%
\pgfpathlineto{\pgfqpoint{3.048493in}{0.723267in}}%
\pgfpathlineto{\pgfqpoint{3.049013in}{0.723324in}}%
\pgfpathlineto{\pgfqpoint{3.049186in}{0.722494in}}%
\pgfpathlineto{\pgfqpoint{3.052131in}{0.708951in}}%
\pgfpathlineto{\pgfqpoint{3.052304in}{0.709555in}}%
\pgfpathlineto{\pgfqpoint{3.053516in}{0.711273in}}%
\pgfpathlineto{\pgfqpoint{3.052997in}{0.708914in}}%
\pgfpathlineto{\pgfqpoint{3.053690in}{0.709744in}}%
\pgfpathlineto{\pgfqpoint{3.054036in}{0.708575in}}%
\pgfpathlineto{\pgfqpoint{3.054383in}{0.711096in}}%
\pgfpathlineto{\pgfqpoint{3.055942in}{0.722775in}}%
\pgfpathlineto{\pgfqpoint{3.056461in}{0.717103in}}%
\pgfpathlineto{\pgfqpoint{3.058020in}{0.698434in}}%
\pgfpathlineto{\pgfqpoint{3.059060in}{0.701393in}}%
\pgfpathlineto{\pgfqpoint{3.059752in}{0.700600in}}%
\pgfpathlineto{\pgfqpoint{3.060792in}{0.702690in}}%
\pgfpathlineto{\pgfqpoint{3.061658in}{0.697154in}}%
\pgfpathlineto{\pgfqpoint{3.065815in}{0.716154in}}%
\pgfpathlineto{\pgfqpoint{3.066854in}{0.719894in}}%
\pgfpathlineto{\pgfqpoint{3.067201in}{0.716948in}}%
\pgfpathlineto{\pgfqpoint{3.068587in}{0.711625in}}%
\pgfpathlineto{\pgfqpoint{3.068933in}{0.713794in}}%
\pgfpathlineto{\pgfqpoint{3.069453in}{0.719650in}}%
\pgfpathlineto{\pgfqpoint{3.070146in}{0.712795in}}%
\pgfpathlineto{\pgfqpoint{3.071531in}{0.706792in}}%
\pgfpathlineto{\pgfqpoint{3.071878in}{0.708657in}}%
\pgfpathlineto{\pgfqpoint{3.073090in}{0.712299in}}%
\pgfpathlineto{\pgfqpoint{3.073264in}{0.714614in}}%
\pgfpathlineto{\pgfqpoint{3.074303in}{0.709098in}}%
\pgfpathlineto{\pgfqpoint{3.076208in}{0.707228in}}%
\pgfpathlineto{\pgfqpoint{3.075169in}{0.710507in}}%
\pgfpathlineto{\pgfqpoint{3.076382in}{0.707498in}}%
\pgfpathlineto{\pgfqpoint{3.077248in}{0.706446in}}%
\pgfpathlineto{\pgfqpoint{3.077767in}{0.704250in}}%
\pgfpathlineto{\pgfqpoint{3.078460in}{0.707937in}}%
\pgfpathlineto{\pgfqpoint{3.078807in}{0.705884in}}%
\pgfpathlineto{\pgfqpoint{3.079500in}{0.708660in}}%
\pgfpathlineto{\pgfqpoint{3.080019in}{0.703727in}}%
\pgfpathlineto{\pgfqpoint{3.080192in}{0.702933in}}%
\pgfpathlineto{\pgfqpoint{3.081059in}{0.705869in}}%
\pgfpathlineto{\pgfqpoint{3.081751in}{0.708597in}}%
\pgfpathlineto{\pgfqpoint{3.082444in}{0.706626in}}%
\pgfpathlineto{\pgfqpoint{3.084350in}{0.696850in}}%
\pgfpathlineto{\pgfqpoint{3.086255in}{0.709714in}}%
\pgfpathlineto{\pgfqpoint{3.086602in}{0.708092in}}%
\pgfpathlineto{\pgfqpoint{3.087641in}{0.710299in}}%
\pgfpathlineto{\pgfqpoint{3.088161in}{0.707984in}}%
\pgfpathlineto{\pgfqpoint{3.089373in}{0.698658in}}%
\pgfpathlineto{\pgfqpoint{3.090066in}{0.703210in}}%
\pgfpathlineto{\pgfqpoint{3.090759in}{0.705068in}}%
\pgfpathlineto{\pgfqpoint{3.090932in}{0.704005in}}%
\pgfpathlineto{\pgfqpoint{3.091798in}{0.694078in}}%
\pgfpathlineto{\pgfqpoint{3.092664in}{0.698009in}}%
\pgfpathlineto{\pgfqpoint{3.092838in}{0.699004in}}%
\pgfpathlineto{\pgfqpoint{3.093184in}{0.696160in}}%
\pgfpathlineto{\pgfqpoint{3.093704in}{0.697792in}}%
\pgfpathlineto{\pgfqpoint{3.095609in}{0.671480in}}%
\pgfpathlineto{\pgfqpoint{3.096129in}{0.675154in}}%
\pgfpathlineto{\pgfqpoint{3.096648in}{0.679595in}}%
\pgfpathlineto{\pgfqpoint{3.099766in}{0.705856in}}%
\pgfpathlineto{\pgfqpoint{3.100806in}{0.700979in}}%
\pgfpathlineto{\pgfqpoint{3.101672in}{0.694560in}}%
\pgfpathlineto{\pgfqpoint{3.102365in}{0.696008in}}%
\pgfpathlineto{\pgfqpoint{3.102884in}{0.700058in}}%
\pgfpathlineto{\pgfqpoint{3.104097in}{0.699434in}}%
\pgfpathlineto{\pgfqpoint{3.104270in}{0.699552in}}%
\pgfpathlineto{\pgfqpoint{3.105483in}{0.708585in}}%
\pgfpathlineto{\pgfqpoint{3.106175in}{0.706302in}}%
\pgfpathlineto{\pgfqpoint{3.108254in}{0.695799in}}%
\pgfpathlineto{\pgfqpoint{3.109986in}{0.686237in}}%
\pgfpathlineto{\pgfqpoint{3.110333in}{0.686750in}}%
\pgfpathlineto{\pgfqpoint{3.110679in}{0.684482in}}%
\pgfpathlineto{\pgfqpoint{3.111026in}{0.690743in}}%
\pgfpathlineto{\pgfqpoint{3.112758in}{0.701872in}}%
\pgfpathlineto{\pgfqpoint{3.112931in}{0.699769in}}%
\pgfpathlineto{\pgfqpoint{3.114144in}{0.687319in}}%
\pgfpathlineto{\pgfqpoint{3.114836in}{0.693925in}}%
\pgfpathlineto{\pgfqpoint{3.116915in}{0.681552in}}%
\pgfpathlineto{\pgfqpoint{3.117608in}{0.684262in}}%
\pgfpathlineto{\pgfqpoint{3.118474in}{0.688392in}}%
\pgfpathlineto{\pgfqpoint{3.118994in}{0.691847in}}%
\pgfpathlineto{\pgfqpoint{3.120033in}{0.689558in}}%
\pgfpathlineto{\pgfqpoint{3.120380in}{0.690821in}}%
\pgfpathlineto{\pgfqpoint{3.120553in}{0.687156in}}%
\pgfpathlineto{\pgfqpoint{3.121072in}{0.681274in}}%
\pgfpathlineto{\pgfqpoint{3.121765in}{0.687512in}}%
\pgfpathlineto{\pgfqpoint{3.122112in}{0.689399in}}%
\pgfpathlineto{\pgfqpoint{3.122458in}{0.683241in}}%
\pgfpathlineto{\pgfqpoint{3.123151in}{0.678455in}}%
\pgfpathlineto{\pgfqpoint{3.124017in}{0.679750in}}%
\pgfpathlineto{\pgfqpoint{3.124364in}{0.680865in}}%
\pgfpathlineto{\pgfqpoint{3.124710in}{0.677765in}}%
\pgfpathlineto{\pgfqpoint{3.125230in}{0.679731in}}%
\pgfpathlineto{\pgfqpoint{3.125749in}{0.677032in}}%
\pgfpathlineto{\pgfqpoint{3.126442in}{0.681511in}}%
\pgfpathlineto{\pgfqpoint{3.128001in}{0.698282in}}%
\pgfpathlineto{\pgfqpoint{3.128348in}{0.695436in}}%
\pgfpathlineto{\pgfqpoint{3.129214in}{0.685793in}}%
\pgfpathlineto{\pgfqpoint{3.130253in}{0.692789in}}%
\pgfpathlineto{\pgfqpoint{3.130773in}{0.696975in}}%
\pgfpathlineto{\pgfqpoint{3.131639in}{0.693186in}}%
\pgfpathlineto{\pgfqpoint{3.133891in}{0.684954in}}%
\pgfpathlineto{\pgfqpoint{3.137009in}{0.694282in}}%
\pgfpathlineto{\pgfqpoint{3.137182in}{0.695184in}}%
\pgfpathlineto{\pgfqpoint{3.137875in}{0.692186in}}%
\pgfpathlineto{\pgfqpoint{3.140127in}{0.663664in}}%
\pgfpathlineto{\pgfqpoint{3.141859in}{0.672653in}}%
\pgfpathlineto{\pgfqpoint{3.142205in}{0.671553in}}%
\pgfpathlineto{\pgfqpoint{3.142552in}{0.673864in}}%
\pgfpathlineto{\pgfqpoint{3.145150in}{0.686116in}}%
\pgfpathlineto{\pgfqpoint{3.146189in}{0.682285in}}%
\pgfpathlineto{\pgfqpoint{3.146536in}{0.686081in}}%
\pgfpathlineto{\pgfqpoint{3.146882in}{0.688922in}}%
\pgfpathlineto{\pgfqpoint{3.147055in}{0.685516in}}%
\pgfpathlineto{\pgfqpoint{3.147748in}{0.686702in}}%
\pgfpathlineto{\pgfqpoint{3.148268in}{0.680589in}}%
\pgfpathlineto{\pgfqpoint{3.148961in}{0.689058in}}%
\pgfpathlineto{\pgfqpoint{3.149654in}{0.694146in}}%
\pgfpathlineto{\pgfqpoint{3.150520in}{0.690987in}}%
\pgfpathlineto{\pgfqpoint{3.152425in}{0.679490in}}%
\pgfpathlineto{\pgfqpoint{3.152599in}{0.681686in}}%
\pgfpathlineto{\pgfqpoint{3.152772in}{0.682098in}}%
\pgfpathlineto{\pgfqpoint{3.152945in}{0.678841in}}%
\pgfpathlineto{\pgfqpoint{3.154157in}{0.665788in}}%
\pgfpathlineto{\pgfqpoint{3.154850in}{0.668532in}}%
\pgfpathlineto{\pgfqpoint{3.155543in}{0.671886in}}%
\pgfpathlineto{\pgfqpoint{3.156409in}{0.669296in}}%
\pgfpathlineto{\pgfqpoint{3.158142in}{0.665518in}}%
\pgfpathlineto{\pgfqpoint{3.158315in}{0.665154in}}%
\pgfpathlineto{\pgfqpoint{3.158488in}{0.666870in}}%
\pgfpathlineto{\pgfqpoint{3.159354in}{0.671911in}}%
\pgfpathlineto{\pgfqpoint{3.160047in}{0.667165in}}%
\pgfpathlineto{\pgfqpoint{3.160740in}{0.670293in}}%
\pgfpathlineto{\pgfqpoint{3.162819in}{0.679393in}}%
\pgfpathlineto{\pgfqpoint{3.162992in}{0.678234in}}%
\pgfpathlineto{\pgfqpoint{3.163511in}{0.682077in}}%
\pgfpathlineto{\pgfqpoint{3.164204in}{0.679939in}}%
\pgfpathlineto{\pgfqpoint{3.166456in}{0.694735in}}%
\pgfpathlineto{\pgfqpoint{3.166456in}{0.694735in}}%
\pgfusepath{stroke}%
\end{pgfscope}%
\begin{pgfscope}%
\pgfpathrectangle{\pgfqpoint{0.568671in}{0.451277in}}{\pgfqpoint{2.598305in}{0.798874in}}%
\pgfusepath{clip}%
\pgfsetrectcap%
\pgfsetroundjoin%
\pgfsetlinewidth{1.505625pt}%
\definecolor{currentstroke}{rgb}{0.000000,0.000000,0.000000}%
\pgfsetstrokecolor{currentstroke}%
\pgfsetdash{}{0pt}%
\pgfpathmoveto{\pgfqpoint{0.649100in}{1.264040in}}%
\pgfpathlineto{\pgfqpoint{0.666194in}{1.238613in}}%
\pgfpathlineto{\pgfqpoint{0.676241in}{1.224650in}}%
\pgfpathlineto{\pgfqpoint{0.694602in}{1.195483in}}%
\pgfpathlineto{\pgfqpoint{0.697201in}{1.191123in}}%
\pgfpathlineto{\pgfqpoint{0.731325in}{1.124242in}}%
\pgfpathlineto{\pgfqpoint{0.750726in}{1.077080in}}%
\pgfpathlineto{\pgfqpoint{0.754363in}{1.072809in}}%
\pgfpathlineto{\pgfqpoint{0.757654in}{1.069090in}}%
\pgfpathlineto{\pgfqpoint{0.759387in}{1.067777in}}%
\pgfpathlineto{\pgfqpoint{0.766315in}{1.057082in}}%
\pgfpathlineto{\pgfqpoint{0.769260in}{1.053215in}}%
\pgfpathlineto{\pgfqpoint{0.769607in}{1.053394in}}%
\pgfpathlineto{\pgfqpoint{0.772205in}{1.053110in}}%
\pgfpathlineto{\pgfqpoint{0.776535in}{1.050893in}}%
\pgfpathlineto{\pgfqpoint{0.780173in}{1.051144in}}%
\pgfpathlineto{\pgfqpoint{0.782079in}{1.051070in}}%
\pgfpathlineto{\pgfqpoint{0.790047in}{1.048506in}}%
\pgfpathlineto{\pgfqpoint{0.794377in}{1.048045in}}%
\pgfpathlineto{\pgfqpoint{0.798361in}{1.045092in}}%
\pgfpathlineto{\pgfqpoint{0.804597in}{1.044624in}}%
\pgfpathlineto{\pgfqpoint{0.818108in}{1.031950in}}%
\pgfpathlineto{\pgfqpoint{0.821053in}{1.031142in}}%
\pgfpathlineto{\pgfqpoint{0.826769in}{1.027371in}}%
\pgfpathlineto{\pgfqpoint{0.831446in}{1.023763in}}%
\pgfpathlineto{\pgfqpoint{0.835084in}{1.019825in}}%
\pgfpathlineto{\pgfqpoint{0.838029in}{1.018764in}}%
\pgfpathlineto{\pgfqpoint{0.841493in}{1.017459in}}%
\pgfpathlineto{\pgfqpoint{0.842706in}{1.016608in}}%
\pgfpathlineto{\pgfqpoint{0.847209in}{1.009663in}}%
\pgfpathlineto{\pgfqpoint{0.848942in}{1.008507in}}%
\pgfpathlineto{\pgfqpoint{0.856563in}{1.000015in}}%
\pgfpathlineto{\pgfqpoint{0.868342in}{0.993942in}}%
\pgfpathlineto{\pgfqpoint{0.871114in}{0.992858in}}%
\pgfpathlineto{\pgfqpoint{0.878909in}{0.987719in}}%
\pgfpathlineto{\pgfqpoint{0.886011in}{0.982863in}}%
\pgfpathlineto{\pgfqpoint{0.890861in}{0.978707in}}%
\pgfpathlineto{\pgfqpoint{0.898309in}{0.980997in}}%
\pgfpathlineto{\pgfqpoint{0.901774in}{0.982838in}}%
\pgfpathlineto{\pgfqpoint{0.907317in}{0.982048in}}%
\pgfpathlineto{\pgfqpoint{0.913899in}{0.976353in}}%
\pgfpathlineto{\pgfqpoint{0.927410in}{0.959873in}}%
\pgfpathlineto{\pgfqpoint{0.929489in}{0.958189in}}%
\pgfpathlineto{\pgfqpoint{0.933646in}{0.955327in}}%
\pgfpathlineto{\pgfqpoint{0.935898in}{0.955625in}}%
\pgfpathlineto{\pgfqpoint{0.943347in}{0.956620in}}%
\pgfpathlineto{\pgfqpoint{0.947157in}{0.954509in}}%
\pgfpathlineto{\pgfqpoint{0.951315in}{0.954199in}}%
\pgfpathlineto{\pgfqpoint{0.955299in}{0.957094in}}%
\pgfpathlineto{\pgfqpoint{0.956511in}{0.956207in}}%
\pgfpathlineto{\pgfqpoint{0.959456in}{0.953573in}}%
\pgfpathlineto{\pgfqpoint{0.963267in}{0.953883in}}%
\pgfpathlineto{\pgfqpoint{0.967078in}{0.951804in}}%
\pgfpathlineto{\pgfqpoint{0.971062in}{0.943720in}}%
\pgfpathlineto{\pgfqpoint{0.973487in}{0.939499in}}%
\pgfpathlineto{\pgfqpoint{0.974007in}{0.939908in}}%
\pgfpathlineto{\pgfqpoint{0.975566in}{0.940180in}}%
\pgfpathlineto{\pgfqpoint{0.975912in}{0.939714in}}%
\pgfpathlineto{\pgfqpoint{0.991155in}{0.919710in}}%
\pgfpathlineto{\pgfqpoint{0.994100in}{0.915843in}}%
\pgfpathlineto{\pgfqpoint{1.013674in}{0.895813in}}%
\pgfpathlineto{\pgfqpoint{1.015406in}{0.894323in}}%
\pgfpathlineto{\pgfqpoint{1.018178in}{0.892710in}}%
\pgfpathlineto{\pgfqpoint{1.021296in}{0.891675in}}%
\pgfpathlineto{\pgfqpoint{1.024933in}{0.891760in}}%
\pgfpathlineto{\pgfqpoint{1.027532in}{0.889800in}}%
\pgfpathlineto{\pgfqpoint{1.029784in}{0.888047in}}%
\pgfpathlineto{\pgfqpoint{1.033768in}{0.884687in}}%
\pgfpathlineto{\pgfqpoint{1.035500in}{0.882812in}}%
\pgfpathlineto{\pgfqpoint{1.037578in}{0.881888in}}%
\pgfpathlineto{\pgfqpoint{1.039137in}{0.880353in}}%
\pgfpathlineto{\pgfqpoint{1.048491in}{0.864150in}}%
\pgfpathlineto{\pgfqpoint{1.049011in}{0.864607in}}%
\pgfpathlineto{\pgfqpoint{1.052649in}{0.865983in}}%
\pgfpathlineto{\pgfqpoint{1.058365in}{0.858314in}}%
\pgfpathlineto{\pgfqpoint{1.061829in}{0.852624in}}%
\pgfpathlineto{\pgfqpoint{1.062176in}{0.852838in}}%
\pgfpathlineto{\pgfqpoint{1.066333in}{0.853930in}}%
\pgfpathlineto{\pgfqpoint{1.067892in}{0.850148in}}%
\pgfpathlineto{\pgfqpoint{1.071530in}{0.842625in}}%
\pgfpathlineto{\pgfqpoint{1.073782in}{0.840988in}}%
\pgfpathlineto{\pgfqpoint{1.080537in}{0.836869in}}%
\pgfpathlineto{\pgfqpoint{1.085387in}{0.824763in}}%
\pgfpathlineto{\pgfqpoint{1.088505in}{0.820485in}}%
\pgfpathlineto{\pgfqpoint{1.092663in}{0.817703in}}%
\pgfpathlineto{\pgfqpoint{1.097859in}{0.814565in}}%
\pgfpathlineto{\pgfqpoint{1.100111in}{0.813896in}}%
\pgfpathlineto{\pgfqpoint{1.101670in}{0.812927in}}%
\pgfpathlineto{\pgfqpoint{1.102190in}{0.813355in}}%
\pgfpathlineto{\pgfqpoint{1.104615in}{0.813022in}}%
\pgfpathlineto{\pgfqpoint{1.108426in}{0.807338in}}%
\pgfpathlineto{\pgfqpoint{1.110851in}{0.803886in}}%
\pgfpathlineto{\pgfqpoint{1.123149in}{0.789597in}}%
\pgfpathlineto{\pgfqpoint{1.124362in}{0.789911in}}%
\pgfpathlineto{\pgfqpoint{1.124708in}{0.789301in}}%
\pgfpathlineto{\pgfqpoint{1.125921in}{0.787096in}}%
\pgfpathlineto{\pgfqpoint{1.126614in}{0.788016in}}%
\pgfpathlineto{\pgfqpoint{1.128866in}{0.792288in}}%
\pgfpathlineto{\pgfqpoint{1.129558in}{0.791445in}}%
\pgfpathlineto{\pgfqpoint{1.131984in}{0.788182in}}%
\pgfpathlineto{\pgfqpoint{1.132503in}{0.788546in}}%
\pgfpathlineto{\pgfqpoint{1.135794in}{0.790013in}}%
\pgfpathlineto{\pgfqpoint{1.142377in}{0.788278in}}%
\pgfpathlineto{\pgfqpoint{1.145668in}{0.779625in}}%
\pgfpathlineto{\pgfqpoint{1.147227in}{0.782047in}}%
\pgfpathlineto{\pgfqpoint{1.148093in}{0.782672in}}%
\pgfpathlineto{\pgfqpoint{1.148959in}{0.782005in}}%
\pgfpathlineto{\pgfqpoint{1.151384in}{0.777721in}}%
\pgfpathlineto{\pgfqpoint{1.158659in}{0.755229in}}%
\pgfpathlineto{\pgfqpoint{1.159526in}{0.753987in}}%
\pgfpathlineto{\pgfqpoint{1.161604in}{0.748437in}}%
\pgfpathlineto{\pgfqpoint{1.162470in}{0.748813in}}%
\pgfpathlineto{\pgfqpoint{1.165242in}{0.750650in}}%
\pgfpathlineto{\pgfqpoint{1.167667in}{0.746354in}}%
\pgfpathlineto{\pgfqpoint{1.170612in}{0.740727in}}%
\pgfpathlineto{\pgfqpoint{1.171824in}{0.736778in}}%
\pgfpathlineto{\pgfqpoint{1.175982in}{0.725584in}}%
\pgfpathlineto{\pgfqpoint{1.178580in}{0.722685in}}%
\pgfpathlineto{\pgfqpoint{1.178753in}{0.722943in}}%
\pgfpathlineto{\pgfqpoint{1.180832in}{0.727958in}}%
\pgfpathlineto{\pgfqpoint{1.181871in}{0.725770in}}%
\pgfpathlineto{\pgfqpoint{1.182910in}{0.725246in}}%
\pgfpathlineto{\pgfqpoint{1.183430in}{0.726001in}}%
\pgfpathlineto{\pgfqpoint{1.187068in}{0.730018in}}%
\pgfpathlineto{\pgfqpoint{1.188107in}{0.728765in}}%
\pgfpathlineto{\pgfqpoint{1.191398in}{0.725310in}}%
\pgfpathlineto{\pgfqpoint{1.192437in}{0.727202in}}%
\pgfpathlineto{\pgfqpoint{1.193477in}{0.728531in}}%
\pgfpathlineto{\pgfqpoint{1.193996in}{0.727581in}}%
\pgfpathlineto{\pgfqpoint{1.198154in}{0.718308in}}%
\pgfpathlineto{\pgfqpoint{1.198673in}{0.718517in}}%
\pgfpathlineto{\pgfqpoint{1.200059in}{0.720916in}}%
\pgfpathlineto{\pgfqpoint{1.201098in}{0.721949in}}%
\pgfpathlineto{\pgfqpoint{1.201791in}{0.720895in}}%
\pgfpathlineto{\pgfqpoint{1.203697in}{0.707448in}}%
\pgfpathlineto{\pgfqpoint{1.206642in}{0.689928in}}%
\pgfpathlineto{\pgfqpoint{1.207161in}{0.690775in}}%
\pgfpathlineto{\pgfqpoint{1.212358in}{0.707169in}}%
\pgfpathlineto{\pgfqpoint{1.217208in}{0.708022in}}%
\pgfpathlineto{\pgfqpoint{1.219113in}{0.704255in}}%
\pgfpathlineto{\pgfqpoint{1.219979in}{0.706593in}}%
\pgfpathlineto{\pgfqpoint{1.222924in}{0.711084in}}%
\pgfpathlineto{\pgfqpoint{1.224656in}{0.717866in}}%
\pgfpathlineto{\pgfqpoint{1.226215in}{0.725306in}}%
\pgfpathlineto{\pgfqpoint{1.226908in}{0.723700in}}%
\pgfpathlineto{\pgfqpoint{1.229680in}{0.716584in}}%
\pgfpathlineto{\pgfqpoint{1.229853in}{0.716665in}}%
\pgfpathlineto{\pgfqpoint{1.230546in}{0.716667in}}%
\pgfpathlineto{\pgfqpoint{1.231066in}{0.715768in}}%
\pgfpathlineto{\pgfqpoint{1.234184in}{0.705366in}}%
\pgfpathlineto{\pgfqpoint{1.236609in}{0.699564in}}%
\pgfpathlineto{\pgfqpoint{1.237475in}{0.701661in}}%
\pgfpathlineto{\pgfqpoint{1.238860in}{0.705048in}}%
\pgfpathlineto{\pgfqpoint{1.239553in}{0.704558in}}%
\pgfpathlineto{\pgfqpoint{1.240766in}{0.702314in}}%
\pgfpathlineto{\pgfqpoint{1.247695in}{0.680345in}}%
\pgfpathlineto{\pgfqpoint{1.248734in}{0.682465in}}%
\pgfpathlineto{\pgfqpoint{1.254104in}{0.703046in}}%
\pgfpathlineto{\pgfqpoint{1.255143in}{0.701848in}}%
\pgfpathlineto{\pgfqpoint{1.256529in}{0.699497in}}%
\pgfpathlineto{\pgfqpoint{1.258954in}{0.694726in}}%
\pgfpathlineto{\pgfqpoint{1.259300in}{0.695054in}}%
\pgfpathlineto{\pgfqpoint{1.261033in}{0.698334in}}%
\pgfpathlineto{\pgfqpoint{1.261726in}{0.696857in}}%
\pgfpathlineto{\pgfqpoint{1.266056in}{0.687724in}}%
\pgfpathlineto{\pgfqpoint{1.268135in}{0.690538in}}%
\pgfpathlineto{\pgfqpoint{1.268828in}{0.689533in}}%
\pgfpathlineto{\pgfqpoint{1.269867in}{0.688226in}}%
\pgfpathlineto{\pgfqpoint{1.270387in}{0.689196in}}%
\pgfpathlineto{\pgfqpoint{1.271599in}{0.694656in}}%
\pgfpathlineto{\pgfqpoint{1.272465in}{0.692690in}}%
\pgfpathlineto{\pgfqpoint{1.273505in}{0.689137in}}%
\pgfpathlineto{\pgfqpoint{1.274197in}{0.690290in}}%
\pgfpathlineto{\pgfqpoint{1.274890in}{0.691387in}}%
\pgfpathlineto{\pgfqpoint{1.275583in}{0.689900in}}%
\pgfpathlineto{\pgfqpoint{1.277315in}{0.685818in}}%
\pgfpathlineto{\pgfqpoint{1.278008in}{0.686916in}}%
\pgfpathlineto{\pgfqpoint{1.279221in}{0.689644in}}%
\pgfpathlineto{\pgfqpoint{1.279914in}{0.688132in}}%
\pgfpathlineto{\pgfqpoint{1.281299in}{0.686397in}}%
\pgfpathlineto{\pgfqpoint{1.281992in}{0.686981in}}%
\pgfpathlineto{\pgfqpoint{1.283725in}{0.692124in}}%
\pgfpathlineto{\pgfqpoint{1.284591in}{0.689632in}}%
\pgfpathlineto{\pgfqpoint{1.287882in}{0.683527in}}%
\pgfpathlineto{\pgfqpoint{1.288055in}{0.683734in}}%
\pgfpathlineto{\pgfqpoint{1.291519in}{0.689462in}}%
\pgfpathlineto{\pgfqpoint{1.292212in}{0.688574in}}%
\pgfpathlineto{\pgfqpoint{1.294118in}{0.676319in}}%
\pgfpathlineto{\pgfqpoint{1.296543in}{0.669443in}}%
\pgfpathlineto{\pgfqpoint{1.296716in}{0.669547in}}%
\pgfpathlineto{\pgfqpoint{1.297755in}{0.672115in}}%
\pgfpathlineto{\pgfqpoint{1.299314in}{0.675085in}}%
\pgfpathlineto{\pgfqpoint{1.299834in}{0.674515in}}%
\pgfpathlineto{\pgfqpoint{1.302952in}{0.671283in}}%
\pgfpathlineto{\pgfqpoint{1.303818in}{0.672035in}}%
\pgfpathlineto{\pgfqpoint{1.306763in}{0.674012in}}%
\pgfpathlineto{\pgfqpoint{1.307109in}{0.673128in}}%
\pgfpathlineto{\pgfqpoint{1.308322in}{0.670761in}}%
\pgfpathlineto{\pgfqpoint{1.308842in}{0.672107in}}%
\pgfpathlineto{\pgfqpoint{1.310054in}{0.676305in}}%
\pgfpathlineto{\pgfqpoint{1.310920in}{0.674889in}}%
\pgfpathlineto{\pgfqpoint{1.312826in}{0.671152in}}%
\pgfpathlineto{\pgfqpoint{1.313518in}{0.672601in}}%
\pgfpathlineto{\pgfqpoint{1.315597in}{0.674130in}}%
\pgfpathlineto{\pgfqpoint{1.317329in}{0.674490in}}%
\pgfpathlineto{\pgfqpoint{1.317503in}{0.674361in}}%
\pgfpathlineto{\pgfqpoint{1.319062in}{0.674908in}}%
\pgfpathlineto{\pgfqpoint{1.319928in}{0.675530in}}%
\pgfpathlineto{\pgfqpoint{1.320274in}{0.674598in}}%
\pgfpathlineto{\pgfqpoint{1.323392in}{0.660047in}}%
\pgfpathlineto{\pgfqpoint{1.324431in}{0.663950in}}%
\pgfpathlineto{\pgfqpoint{1.326683in}{0.667652in}}%
\pgfpathlineto{\pgfqpoint{1.329108in}{0.670049in}}%
\pgfpathlineto{\pgfqpoint{1.334132in}{0.680021in}}%
\pgfpathlineto{\pgfqpoint{1.334998in}{0.677832in}}%
\pgfpathlineto{\pgfqpoint{1.340887in}{0.656914in}}%
\pgfpathlineto{\pgfqpoint{1.341927in}{0.656308in}}%
\pgfpathlineto{\pgfqpoint{1.342446in}{0.657334in}}%
\pgfpathlineto{\pgfqpoint{1.343486in}{0.658793in}}%
\pgfpathlineto{\pgfqpoint{1.344178in}{0.657228in}}%
\pgfpathlineto{\pgfqpoint{1.349895in}{0.639497in}}%
\pgfpathlineto{\pgfqpoint{1.350761in}{0.637636in}}%
\pgfpathlineto{\pgfqpoint{1.351454in}{0.638735in}}%
\pgfpathlineto{\pgfqpoint{1.352839in}{0.641998in}}%
\pgfpathlineto{\pgfqpoint{1.353359in}{0.640633in}}%
\pgfpathlineto{\pgfqpoint{1.354052in}{0.638175in}}%
\pgfpathlineto{\pgfqpoint{1.354745in}{0.640180in}}%
\pgfpathlineto{\pgfqpoint{1.355611in}{0.643242in}}%
\pgfpathlineto{\pgfqpoint{1.356131in}{0.640855in}}%
\pgfpathlineto{\pgfqpoint{1.357170in}{0.634858in}}%
\pgfpathlineto{\pgfqpoint{1.358036in}{0.636646in}}%
\pgfpathlineto{\pgfqpoint{1.361154in}{0.647995in}}%
\pgfpathlineto{\pgfqpoint{1.362020in}{0.645803in}}%
\pgfpathlineto{\pgfqpoint{1.364099in}{0.642160in}}%
\pgfpathlineto{\pgfqpoint{1.365658in}{0.640028in}}%
\pgfpathlineto{\pgfqpoint{1.366177in}{0.641170in}}%
\pgfpathlineto{\pgfqpoint{1.374665in}{0.661080in}}%
\pgfpathlineto{\pgfqpoint{1.375705in}{0.657991in}}%
\pgfpathlineto{\pgfqpoint{1.377090in}{0.655367in}}%
\pgfpathlineto{\pgfqpoint{1.377610in}{0.655921in}}%
\pgfpathlineto{\pgfqpoint{1.381594in}{0.659250in}}%
\pgfpathlineto{\pgfqpoint{1.382114in}{0.657438in}}%
\pgfpathlineto{\pgfqpoint{1.384019in}{0.652075in}}%
\pgfpathlineto{\pgfqpoint{1.384539in}{0.652555in}}%
\pgfpathlineto{\pgfqpoint{1.385578in}{0.651500in}}%
\pgfpathlineto{\pgfqpoint{1.386964in}{0.645686in}}%
\pgfpathlineto{\pgfqpoint{1.387830in}{0.647016in}}%
\pgfpathlineto{\pgfqpoint{1.390775in}{0.655025in}}%
\pgfpathlineto{\pgfqpoint{1.391641in}{0.656819in}}%
\pgfpathlineto{\pgfqpoint{1.392334in}{0.656170in}}%
\pgfpathlineto{\pgfqpoint{1.393893in}{0.652324in}}%
\pgfpathlineto{\pgfqpoint{1.394586in}{0.653680in}}%
\pgfpathlineto{\pgfqpoint{1.394932in}{0.653883in}}%
\pgfpathlineto{\pgfqpoint{1.395452in}{0.652639in}}%
\pgfpathlineto{\pgfqpoint{1.396664in}{0.646652in}}%
\pgfpathlineto{\pgfqpoint{1.397530in}{0.650018in}}%
\pgfpathlineto{\pgfqpoint{1.398396in}{0.653958in}}%
\pgfpathlineto{\pgfqpoint{1.399263in}{0.651442in}}%
\pgfpathlineto{\pgfqpoint{1.399782in}{0.650389in}}%
\pgfpathlineto{\pgfqpoint{1.400302in}{0.652739in}}%
\pgfpathlineto{\pgfqpoint{1.402034in}{0.663970in}}%
\pgfpathlineto{\pgfqpoint{1.402900in}{0.662340in}}%
\pgfpathlineto{\pgfqpoint{1.403766in}{0.661670in}}%
\pgfpathlineto{\pgfqpoint{1.404286in}{0.662861in}}%
\pgfpathlineto{\pgfqpoint{1.405325in}{0.664332in}}%
\pgfpathlineto{\pgfqpoint{1.406018in}{0.663536in}}%
\pgfpathlineto{\pgfqpoint{1.407750in}{0.660482in}}%
\pgfpathlineto{\pgfqpoint{1.408443in}{0.662140in}}%
\pgfpathlineto{\pgfqpoint{1.412081in}{0.672203in}}%
\pgfpathlineto{\pgfqpoint{1.413120in}{0.670925in}}%
\pgfpathlineto{\pgfqpoint{1.416931in}{0.674137in}}%
\pgfpathlineto{\pgfqpoint{1.418144in}{0.678227in}}%
\pgfpathlineto{\pgfqpoint{1.418836in}{0.676258in}}%
\pgfpathlineto{\pgfqpoint{1.421781in}{0.668379in}}%
\pgfpathlineto{\pgfqpoint{1.422128in}{0.669092in}}%
\pgfpathlineto{\pgfqpoint{1.424379in}{0.675922in}}%
\pgfpathlineto{\pgfqpoint{1.425072in}{0.674267in}}%
\pgfpathlineto{\pgfqpoint{1.427324in}{0.661942in}}%
\pgfpathlineto{\pgfqpoint{1.429576in}{0.656958in}}%
\pgfpathlineto{\pgfqpoint{1.430615in}{0.656324in}}%
\pgfpathlineto{\pgfqpoint{1.430962in}{0.657151in}}%
\pgfpathlineto{\pgfqpoint{1.434773in}{0.677865in}}%
\pgfpathlineto{\pgfqpoint{1.436505in}{0.719990in}}%
\pgfpathlineto{\pgfqpoint{1.443607in}{0.907529in}}%
\pgfpathlineto{\pgfqpoint{1.447591in}{0.938432in}}%
\pgfpathlineto{\pgfqpoint{1.449496in}{0.939569in}}%
\pgfpathlineto{\pgfqpoint{1.449843in}{0.939221in}}%
\pgfpathlineto{\pgfqpoint{1.453827in}{0.930730in}}%
\pgfpathlineto{\pgfqpoint{1.457118in}{0.922632in}}%
\pgfpathlineto{\pgfqpoint{1.457638in}{0.922854in}}%
\pgfpathlineto{\pgfqpoint{1.459716in}{0.922202in}}%
\pgfpathlineto{\pgfqpoint{1.463354in}{0.919167in}}%
\pgfpathlineto{\pgfqpoint{1.513241in}{0.702114in}}%
\pgfpathlineto{\pgfqpoint{1.514974in}{0.697672in}}%
\pgfpathlineto{\pgfqpoint{1.516013in}{0.695249in}}%
\pgfpathlineto{\pgfqpoint{1.516533in}{0.696555in}}%
\pgfpathlineto{\pgfqpoint{1.520690in}{0.708018in}}%
\pgfpathlineto{\pgfqpoint{1.521902in}{0.708501in}}%
\pgfpathlineto{\pgfqpoint{1.522249in}{0.707895in}}%
\pgfpathlineto{\pgfqpoint{1.525713in}{0.694935in}}%
\pgfpathlineto{\pgfqpoint{1.527619in}{0.690195in}}%
\pgfpathlineto{\pgfqpoint{1.528138in}{0.690776in}}%
\pgfpathlineto{\pgfqpoint{1.531083in}{0.700433in}}%
\pgfpathlineto{\pgfqpoint{1.536107in}{0.715886in}}%
\pgfpathlineto{\pgfqpoint{1.536280in}{0.715798in}}%
\pgfpathlineto{\pgfqpoint{1.538358in}{0.711468in}}%
\pgfpathlineto{\pgfqpoint{1.546500in}{0.691167in}}%
\pgfpathlineto{\pgfqpoint{1.549791in}{0.693545in}}%
\pgfpathlineto{\pgfqpoint{1.551870in}{0.698996in}}%
\pgfpathlineto{\pgfqpoint{1.552562in}{0.698062in}}%
\pgfpathlineto{\pgfqpoint{1.554295in}{0.694905in}}%
\pgfpathlineto{\pgfqpoint{1.554988in}{0.696054in}}%
\pgfpathlineto{\pgfqpoint{1.561224in}{0.709536in}}%
\pgfpathlineto{\pgfqpoint{1.562609in}{0.710913in}}%
\pgfpathlineto{\pgfqpoint{1.563129in}{0.709982in}}%
\pgfpathlineto{\pgfqpoint{1.565208in}{0.694361in}}%
\pgfpathlineto{\pgfqpoint{1.568326in}{0.681364in}}%
\pgfpathlineto{\pgfqpoint{1.570577in}{0.680756in}}%
\pgfpathlineto{\pgfqpoint{1.571963in}{0.678708in}}%
\pgfpathlineto{\pgfqpoint{1.575254in}{0.668289in}}%
\pgfpathlineto{\pgfqpoint{1.576294in}{0.669535in}}%
\pgfpathlineto{\pgfqpoint{1.576987in}{0.670394in}}%
\pgfpathlineto{\pgfqpoint{1.577679in}{0.669119in}}%
\pgfpathlineto{\pgfqpoint{1.581837in}{0.661294in}}%
\pgfpathlineto{\pgfqpoint{1.584955in}{0.665286in}}%
\pgfpathlineto{\pgfqpoint{1.585648in}{0.666290in}}%
\pgfpathlineto{\pgfqpoint{1.586340in}{0.665075in}}%
\pgfpathlineto{\pgfqpoint{1.599159in}{0.619173in}}%
\pgfpathlineto{\pgfqpoint{1.600025in}{0.619840in}}%
\pgfpathlineto{\pgfqpoint{1.601064in}{0.618893in}}%
\pgfpathlineto{\pgfqpoint{1.603662in}{0.616344in}}%
\pgfpathlineto{\pgfqpoint{1.604529in}{0.617677in}}%
\pgfpathlineto{\pgfqpoint{1.607820in}{0.624486in}}%
\pgfpathlineto{\pgfqpoint{1.608166in}{0.624218in}}%
\pgfpathlineto{\pgfqpoint{1.608513in}{0.624176in}}%
\pgfpathlineto{\pgfqpoint{1.608859in}{0.625029in}}%
\pgfpathlineto{\pgfqpoint{1.612497in}{0.637904in}}%
\pgfpathlineto{\pgfqpoint{1.613709in}{0.636522in}}%
\pgfpathlineto{\pgfqpoint{1.619772in}{0.622154in}}%
\pgfpathlineto{\pgfqpoint{1.620811in}{0.623435in}}%
\pgfpathlineto{\pgfqpoint{1.622717in}{0.625824in}}%
\pgfpathlineto{\pgfqpoint{1.623410in}{0.624547in}}%
\pgfpathlineto{\pgfqpoint{1.629126in}{0.608166in}}%
\pgfpathlineto{\pgfqpoint{1.629819in}{0.609251in}}%
\pgfpathlineto{\pgfqpoint{1.632590in}{0.613476in}}%
\pgfpathlineto{\pgfqpoint{1.633110in}{0.612586in}}%
\pgfpathlineto{\pgfqpoint{1.636401in}{0.607407in}}%
\pgfpathlineto{\pgfqpoint{1.637440in}{0.609454in}}%
\pgfpathlineto{\pgfqpoint{1.645582in}{0.619194in}}%
\pgfpathlineto{\pgfqpoint{1.645928in}{0.618109in}}%
\pgfpathlineto{\pgfqpoint{1.651125in}{0.598331in}}%
\pgfpathlineto{\pgfqpoint{1.651818in}{0.599207in}}%
\pgfpathlineto{\pgfqpoint{1.654416in}{0.604298in}}%
\pgfpathlineto{\pgfqpoint{1.655282in}{0.603684in}}%
\pgfpathlineto{\pgfqpoint{1.656148in}{0.603652in}}%
\pgfpathlineto{\pgfqpoint{1.656495in}{0.604375in}}%
\pgfpathlineto{\pgfqpoint{1.657361in}{0.605435in}}%
\pgfpathlineto{\pgfqpoint{1.657880in}{0.604128in}}%
\pgfpathlineto{\pgfqpoint{1.660825in}{0.593245in}}%
\pgfpathlineto{\pgfqpoint{1.661518in}{0.594945in}}%
\pgfpathlineto{\pgfqpoint{1.663077in}{0.598531in}}%
\pgfpathlineto{\pgfqpoint{1.663597in}{0.597980in}}%
\pgfpathlineto{\pgfqpoint{1.666715in}{0.585429in}}%
\pgfpathlineto{\pgfqpoint{1.668274in}{0.581286in}}%
\pgfpathlineto{\pgfqpoint{1.668793in}{0.582345in}}%
\pgfpathlineto{\pgfqpoint{1.673643in}{0.608086in}}%
\pgfpathlineto{\pgfqpoint{1.675549in}{0.615473in}}%
\pgfpathlineto{\pgfqpoint{1.675895in}{0.615231in}}%
\pgfpathlineto{\pgfqpoint{1.677454in}{0.611780in}}%
\pgfpathlineto{\pgfqpoint{1.680572in}{0.605380in}}%
\pgfpathlineto{\pgfqpoint{1.682304in}{0.608443in}}%
\pgfpathlineto{\pgfqpoint{1.693217in}{0.641607in}}%
\pgfpathlineto{\pgfqpoint{1.694257in}{0.640034in}}%
\pgfpathlineto{\pgfqpoint{1.695816in}{0.631354in}}%
\pgfpathlineto{\pgfqpoint{1.700666in}{0.601981in}}%
\pgfpathlineto{\pgfqpoint{1.701532in}{0.600880in}}%
\pgfpathlineto{\pgfqpoint{1.702225in}{0.601889in}}%
\pgfpathlineto{\pgfqpoint{1.702918in}{0.600133in}}%
\pgfpathlineto{\pgfqpoint{1.707075in}{0.579408in}}%
\pgfpathlineto{\pgfqpoint{1.707941in}{0.580949in}}%
\pgfpathlineto{\pgfqpoint{1.713657in}{0.596378in}}%
\pgfpathlineto{\pgfqpoint{1.714350in}{0.595081in}}%
\pgfpathlineto{\pgfqpoint{1.715736in}{0.593028in}}%
\pgfpathlineto{\pgfqpoint{1.716256in}{0.593481in}}%
\pgfpathlineto{\pgfqpoint{1.717988in}{0.601642in}}%
\pgfpathlineto{\pgfqpoint{1.719547in}{0.605526in}}%
\pgfpathlineto{\pgfqpoint{1.720067in}{0.604882in}}%
\pgfpathlineto{\pgfqpoint{1.723358in}{0.594793in}}%
\pgfpathlineto{\pgfqpoint{1.724051in}{0.596449in}}%
\pgfpathlineto{\pgfqpoint{1.725263in}{0.600142in}}%
\pgfpathlineto{\pgfqpoint{1.726129in}{0.599211in}}%
\pgfpathlineto{\pgfqpoint{1.726822in}{0.600231in}}%
\pgfpathlineto{\pgfqpoint{1.730806in}{0.614431in}}%
\pgfpathlineto{\pgfqpoint{1.731326in}{0.613851in}}%
\pgfpathlineto{\pgfqpoint{1.735310in}{0.607478in}}%
\pgfpathlineto{\pgfqpoint{1.735656in}{0.608047in}}%
\pgfpathlineto{\pgfqpoint{1.737042in}{0.611191in}}%
\pgfpathlineto{\pgfqpoint{1.737908in}{0.609977in}}%
\pgfpathlineto{\pgfqpoint{1.738774in}{0.610246in}}%
\pgfpathlineto{\pgfqpoint{1.739121in}{0.610683in}}%
\pgfpathlineto{\pgfqpoint{1.742239in}{0.613591in}}%
\pgfpathlineto{\pgfqpoint{1.742585in}{0.612777in}}%
\pgfpathlineto{\pgfqpoint{1.745876in}{0.597612in}}%
\pgfpathlineto{\pgfqpoint{1.747435in}{0.598998in}}%
\pgfpathlineto{\pgfqpoint{1.754191in}{0.589404in}}%
\pgfpathlineto{\pgfqpoint{1.761120in}{0.585216in}}%
\pgfpathlineto{\pgfqpoint{1.761813in}{0.584055in}}%
\pgfpathlineto{\pgfqpoint{1.762679in}{0.585361in}}%
\pgfpathlineto{\pgfqpoint{1.765623in}{0.582349in}}%
\pgfpathlineto{\pgfqpoint{1.768222in}{0.572413in}}%
\pgfpathlineto{\pgfqpoint{1.768741in}{0.573114in}}%
\pgfpathlineto{\pgfqpoint{1.771167in}{0.576798in}}%
\pgfpathlineto{\pgfqpoint{1.771686in}{0.575990in}}%
\pgfpathlineto{\pgfqpoint{1.772899in}{0.573623in}}%
\pgfpathlineto{\pgfqpoint{1.773765in}{0.574883in}}%
\pgfpathlineto{\pgfqpoint{1.778788in}{0.584347in}}%
\pgfpathlineto{\pgfqpoint{1.779828in}{0.581148in}}%
\pgfpathlineto{\pgfqpoint{1.781040in}{0.578188in}}%
\pgfpathlineto{\pgfqpoint{1.781733in}{0.579121in}}%
\pgfpathlineto{\pgfqpoint{1.788489in}{0.593555in}}%
\pgfpathlineto{\pgfqpoint{1.788662in}{0.593419in}}%
\pgfpathlineto{\pgfqpoint{1.790394in}{0.589076in}}%
\pgfpathlineto{\pgfqpoint{1.792126in}{0.581205in}}%
\pgfpathlineto{\pgfqpoint{1.792819in}{0.584587in}}%
\pgfpathlineto{\pgfqpoint{1.794205in}{0.591985in}}%
\pgfpathlineto{\pgfqpoint{1.795071in}{0.590237in}}%
\pgfpathlineto{\pgfqpoint{1.795937in}{0.588888in}}%
\pgfpathlineto{\pgfqpoint{1.796457in}{0.590242in}}%
\pgfpathlineto{\pgfqpoint{1.797669in}{0.594101in}}%
\pgfpathlineto{\pgfqpoint{1.798362in}{0.592642in}}%
\pgfpathlineto{\pgfqpoint{1.800268in}{0.583375in}}%
\pgfpathlineto{\pgfqpoint{1.801134in}{0.587181in}}%
\pgfpathlineto{\pgfqpoint{1.804944in}{0.614836in}}%
\pgfpathlineto{\pgfqpoint{1.805291in}{0.614266in}}%
\pgfpathlineto{\pgfqpoint{1.806677in}{0.611289in}}%
\pgfpathlineto{\pgfqpoint{1.807370in}{0.612494in}}%
\pgfpathlineto{\pgfqpoint{1.811007in}{0.616912in}}%
\pgfpathlineto{\pgfqpoint{1.811180in}{0.616777in}}%
\pgfpathlineto{\pgfqpoint{1.812220in}{0.613125in}}%
\pgfpathlineto{\pgfqpoint{1.816897in}{0.598519in}}%
\pgfpathlineto{\pgfqpoint{1.817243in}{0.598430in}}%
\pgfpathlineto{\pgfqpoint{1.817590in}{0.599441in}}%
\pgfpathlineto{\pgfqpoint{1.823825in}{0.618248in}}%
\pgfpathlineto{\pgfqpoint{1.824865in}{0.617136in}}%
\pgfpathlineto{\pgfqpoint{1.826424in}{0.614771in}}%
\pgfpathlineto{\pgfqpoint{1.826943in}{0.616289in}}%
\pgfpathlineto{\pgfqpoint{1.830235in}{0.625372in}}%
\pgfpathlineto{\pgfqpoint{1.831447in}{0.632496in}}%
\pgfpathlineto{\pgfqpoint{1.832313in}{0.635109in}}%
\pgfpathlineto{\pgfqpoint{1.832833in}{0.633456in}}%
\pgfpathlineto{\pgfqpoint{1.837856in}{0.604530in}}%
\pgfpathlineto{\pgfqpoint{1.839069in}{0.605501in}}%
\pgfpathlineto{\pgfqpoint{1.839935in}{0.603878in}}%
\pgfpathlineto{\pgfqpoint{1.841148in}{0.601522in}}%
\pgfpathlineto{\pgfqpoint{1.841667in}{0.602445in}}%
\pgfpathlineto{\pgfqpoint{1.842533in}{0.604172in}}%
\pgfpathlineto{\pgfqpoint{1.843053in}{0.602911in}}%
\pgfpathlineto{\pgfqpoint{1.847557in}{0.587768in}}%
\pgfpathlineto{\pgfqpoint{1.848250in}{0.588360in}}%
\pgfpathlineto{\pgfqpoint{1.849462in}{0.589347in}}%
\pgfpathlineto{\pgfqpoint{1.849809in}{0.588769in}}%
\pgfpathlineto{\pgfqpoint{1.851541in}{0.584586in}}%
\pgfpathlineto{\pgfqpoint{1.852234in}{0.585558in}}%
\pgfpathlineto{\pgfqpoint{1.852927in}{0.584125in}}%
\pgfpathlineto{\pgfqpoint{1.856391in}{0.567651in}}%
\pgfpathlineto{\pgfqpoint{1.857430in}{0.568092in}}%
\pgfpathlineto{\pgfqpoint{1.858470in}{0.569909in}}%
\pgfpathlineto{\pgfqpoint{1.860202in}{0.577270in}}%
\pgfpathlineto{\pgfqpoint{1.861068in}{0.575252in}}%
\pgfpathlineto{\pgfqpoint{1.862280in}{0.573888in}}%
\pgfpathlineto{\pgfqpoint{1.862800in}{0.575094in}}%
\pgfpathlineto{\pgfqpoint{1.865225in}{0.583198in}}%
\pgfpathlineto{\pgfqpoint{1.866091in}{0.579856in}}%
\pgfpathlineto{\pgfqpoint{1.868690in}{0.575131in}}%
\pgfpathlineto{\pgfqpoint{1.870422in}{0.571684in}}%
\pgfpathlineto{\pgfqpoint{1.871461in}{0.568419in}}%
\pgfpathlineto{\pgfqpoint{1.872154in}{0.570336in}}%
\pgfpathlineto{\pgfqpoint{1.873713in}{0.576014in}}%
\pgfpathlineto{\pgfqpoint{1.874406in}{0.574996in}}%
\pgfpathlineto{\pgfqpoint{1.877004in}{0.570815in}}%
\pgfpathlineto{\pgfqpoint{1.877524in}{0.571520in}}%
\pgfpathlineto{\pgfqpoint{1.878736in}{0.574360in}}%
\pgfpathlineto{\pgfqpoint{1.879256in}{0.572943in}}%
\pgfpathlineto{\pgfqpoint{1.880295in}{0.569782in}}%
\pgfpathlineto{\pgfqpoint{1.881161in}{0.570818in}}%
\pgfpathlineto{\pgfqpoint{1.882028in}{0.569116in}}%
\pgfpathlineto{\pgfqpoint{1.883240in}{0.566068in}}%
\pgfpathlineto{\pgfqpoint{1.883933in}{0.567260in}}%
\pgfpathlineto{\pgfqpoint{1.888437in}{0.582075in}}%
\pgfpathlineto{\pgfqpoint{1.890862in}{0.587258in}}%
\pgfpathlineto{\pgfqpoint{1.894673in}{0.594585in}}%
\pgfpathlineto{\pgfqpoint{1.895712in}{0.595897in}}%
\pgfpathlineto{\pgfqpoint{1.897098in}{0.599705in}}%
\pgfpathlineto{\pgfqpoint{1.897791in}{0.598391in}}%
\pgfpathlineto{\pgfqpoint{1.900389in}{0.585689in}}%
\pgfpathlineto{\pgfqpoint{1.902294in}{0.581108in}}%
\pgfpathlineto{\pgfqpoint{1.902814in}{0.581793in}}%
\pgfpathlineto{\pgfqpoint{1.904546in}{0.589127in}}%
\pgfpathlineto{\pgfqpoint{1.905585in}{0.591424in}}%
\pgfpathlineto{\pgfqpoint{1.906278in}{0.590775in}}%
\pgfpathlineto{\pgfqpoint{1.906798in}{0.590409in}}%
\pgfpathlineto{\pgfqpoint{1.907318in}{0.591842in}}%
\pgfpathlineto{\pgfqpoint{1.908357in}{0.596042in}}%
\pgfpathlineto{\pgfqpoint{1.909223in}{0.594206in}}%
\pgfpathlineto{\pgfqpoint{1.910609in}{0.590334in}}%
\pgfpathlineto{\pgfqpoint{1.911302in}{0.591397in}}%
\pgfpathlineto{\pgfqpoint{1.913900in}{0.592516in}}%
\pgfpathlineto{\pgfqpoint{1.915286in}{0.593348in}}%
\pgfpathlineto{\pgfqpoint{1.916845in}{0.594555in}}%
\pgfpathlineto{\pgfqpoint{1.917191in}{0.593664in}}%
\pgfpathlineto{\pgfqpoint{1.920656in}{0.576844in}}%
\pgfpathlineto{\pgfqpoint{1.921695in}{0.580840in}}%
\pgfpathlineto{\pgfqpoint{1.924293in}{0.593099in}}%
\pgfpathlineto{\pgfqpoint{1.924813in}{0.591625in}}%
\pgfpathlineto{\pgfqpoint{1.930876in}{0.566501in}}%
\pgfpathlineto{\pgfqpoint{1.934513in}{0.557765in}}%
\pgfpathlineto{\pgfqpoint{1.934686in}{0.557845in}}%
\pgfpathlineto{\pgfqpoint{1.936072in}{0.559485in}}%
\pgfpathlineto{\pgfqpoint{1.937285in}{0.562384in}}%
\pgfpathlineto{\pgfqpoint{1.937978in}{0.560929in}}%
\pgfpathlineto{\pgfqpoint{1.939883in}{0.548549in}}%
\pgfpathlineto{\pgfqpoint{1.941789in}{0.537521in}}%
\pgfpathlineto{\pgfqpoint{1.942481in}{0.540748in}}%
\pgfpathlineto{\pgfqpoint{1.945253in}{0.551223in}}%
\pgfpathlineto{\pgfqpoint{1.945773in}{0.551140in}}%
\pgfpathlineto{\pgfqpoint{1.946119in}{0.549975in}}%
\pgfpathlineto{\pgfqpoint{1.947505in}{0.545114in}}%
\pgfpathlineto{\pgfqpoint{1.948198in}{0.546678in}}%
\pgfpathlineto{\pgfqpoint{1.956685in}{0.586380in}}%
\pgfpathlineto{\pgfqpoint{1.958937in}{0.594393in}}%
\pgfpathlineto{\pgfqpoint{1.959457in}{0.593882in}}%
\pgfpathlineto{\pgfqpoint{1.960150in}{0.592979in}}%
\pgfpathlineto{\pgfqpoint{1.960670in}{0.594196in}}%
\pgfpathlineto{\pgfqpoint{1.966559in}{0.610355in}}%
\pgfpathlineto{\pgfqpoint{1.967425in}{0.608951in}}%
\pgfpathlineto{\pgfqpoint{1.970890in}{0.596363in}}%
\pgfpathlineto{\pgfqpoint{1.971929in}{0.590260in}}%
\pgfpathlineto{\pgfqpoint{1.972622in}{0.592663in}}%
\pgfpathlineto{\pgfqpoint{1.974354in}{0.597105in}}%
\pgfpathlineto{\pgfqpoint{1.974874in}{0.596662in}}%
\pgfpathlineto{\pgfqpoint{1.975913in}{0.593205in}}%
\pgfpathlineto{\pgfqpoint{1.979724in}{0.580124in}}%
\pgfpathlineto{\pgfqpoint{1.981456in}{0.578568in}}%
\pgfpathlineto{\pgfqpoint{1.981802in}{0.578870in}}%
\pgfpathlineto{\pgfqpoint{1.987172in}{0.589074in}}%
\pgfpathlineto{\pgfqpoint{1.987692in}{0.588026in}}%
\pgfpathlineto{\pgfqpoint{1.990117in}{0.579729in}}%
\pgfpathlineto{\pgfqpoint{1.990983in}{0.581408in}}%
\pgfpathlineto{\pgfqpoint{1.994274in}{0.588161in}}%
\pgfpathlineto{\pgfqpoint{1.995660in}{0.586672in}}%
\pgfpathlineto{\pgfqpoint{1.998432in}{0.580941in}}%
\pgfpathlineto{\pgfqpoint{1.998951in}{0.581931in}}%
\pgfpathlineto{\pgfqpoint{2.000857in}{0.594240in}}%
\pgfpathlineto{\pgfqpoint{2.001896in}{0.596965in}}%
\pgfpathlineto{\pgfqpoint{2.002589in}{0.595949in}}%
\pgfpathlineto{\pgfqpoint{2.003282in}{0.595155in}}%
\pgfpathlineto{\pgfqpoint{2.003975in}{0.596369in}}%
\pgfpathlineto{\pgfqpoint{2.006053in}{0.602451in}}%
\pgfpathlineto{\pgfqpoint{2.006746in}{0.601438in}}%
\pgfpathlineto{\pgfqpoint{2.009344in}{0.592555in}}%
\pgfpathlineto{\pgfqpoint{2.011943in}{0.585449in}}%
\pgfpathlineto{\pgfqpoint{2.014368in}{0.583338in}}%
\pgfpathlineto{\pgfqpoint{2.020431in}{0.540878in}}%
\pgfpathlineto{\pgfqpoint{2.021470in}{0.543195in}}%
\pgfpathlineto{\pgfqpoint{2.028745in}{0.573185in}}%
\pgfpathlineto{\pgfqpoint{2.029784in}{0.567437in}}%
\pgfpathlineto{\pgfqpoint{2.033422in}{0.545480in}}%
\pgfpathlineto{\pgfqpoint{2.033595in}{0.545629in}}%
\pgfpathlineto{\pgfqpoint{2.036020in}{0.553924in}}%
\pgfpathlineto{\pgfqpoint{2.037753in}{0.551741in}}%
\pgfpathlineto{\pgfqpoint{2.040524in}{0.547487in}}%
\pgfpathlineto{\pgfqpoint{2.043469in}{0.535344in}}%
\pgfpathlineto{\pgfqpoint{2.044681in}{0.536884in}}%
\pgfpathlineto{\pgfqpoint{2.049705in}{0.554978in}}%
\pgfpathlineto{\pgfqpoint{2.049878in}{0.554911in}}%
\pgfpathlineto{\pgfqpoint{2.050744in}{0.556670in}}%
\pgfpathlineto{\pgfqpoint{2.052650in}{0.566349in}}%
\pgfpathlineto{\pgfqpoint{2.054035in}{0.565499in}}%
\pgfpathlineto{\pgfqpoint{2.054901in}{0.566414in}}%
\pgfpathlineto{\pgfqpoint{2.056634in}{0.572446in}}%
\pgfpathlineto{\pgfqpoint{2.057673in}{0.570992in}}%
\pgfpathlineto{\pgfqpoint{2.059232in}{0.568862in}}%
\pgfpathlineto{\pgfqpoint{2.059578in}{0.569145in}}%
\pgfpathlineto{\pgfqpoint{2.060271in}{0.569439in}}%
\pgfpathlineto{\pgfqpoint{2.060964in}{0.568461in}}%
\pgfpathlineto{\pgfqpoint{2.061830in}{0.566486in}}%
\pgfpathlineto{\pgfqpoint{2.062696in}{0.568143in}}%
\pgfpathlineto{\pgfqpoint{2.063389in}{0.568853in}}%
\pgfpathlineto{\pgfqpoint{2.063909in}{0.567189in}}%
\pgfpathlineto{\pgfqpoint{2.067373in}{0.555186in}}%
\pgfpathlineto{\pgfqpoint{2.069972in}{0.552140in}}%
\pgfpathlineto{\pgfqpoint{2.070318in}{0.552925in}}%
\pgfpathlineto{\pgfqpoint{2.073782in}{0.560731in}}%
\pgfpathlineto{\pgfqpoint{2.075341in}{0.571460in}}%
\pgfpathlineto{\pgfqpoint{2.077074in}{0.579050in}}%
\pgfpathlineto{\pgfqpoint{2.077593in}{0.578060in}}%
\pgfpathlineto{\pgfqpoint{2.078286in}{0.577701in}}%
\pgfpathlineto{\pgfqpoint{2.078806in}{0.578597in}}%
\pgfpathlineto{\pgfqpoint{2.082617in}{0.584704in}}%
\pgfpathlineto{\pgfqpoint{2.083136in}{0.584156in}}%
\pgfpathlineto{\pgfqpoint{2.084176in}{0.582293in}}%
\pgfpathlineto{\pgfqpoint{2.084869in}{0.583415in}}%
\pgfpathlineto{\pgfqpoint{2.088160in}{0.589347in}}%
\pgfpathlineto{\pgfqpoint{2.088333in}{0.589174in}}%
\pgfpathlineto{\pgfqpoint{2.092317in}{0.580951in}}%
\pgfpathlineto{\pgfqpoint{2.093530in}{0.581849in}}%
\pgfpathlineto{\pgfqpoint{2.094222in}{0.580582in}}%
\pgfpathlineto{\pgfqpoint{2.098553in}{0.568180in}}%
\pgfpathlineto{\pgfqpoint{2.099592in}{0.567011in}}%
\pgfpathlineto{\pgfqpoint{2.100458in}{0.567834in}}%
\pgfpathlineto{\pgfqpoint{2.102017in}{0.568758in}}%
\pgfpathlineto{\pgfqpoint{2.102364in}{0.568245in}}%
\pgfpathlineto{\pgfqpoint{2.104442in}{0.564463in}}%
\pgfpathlineto{\pgfqpoint{2.104962in}{0.565386in}}%
\pgfpathlineto{\pgfqpoint{2.108253in}{0.577411in}}%
\pgfpathlineto{\pgfqpoint{2.108946in}{0.575236in}}%
\pgfpathlineto{\pgfqpoint{2.111371in}{0.567012in}}%
\pgfpathlineto{\pgfqpoint{2.111718in}{0.567129in}}%
\pgfpathlineto{\pgfqpoint{2.113450in}{0.570265in}}%
\pgfpathlineto{\pgfqpoint{2.123150in}{0.616510in}}%
\pgfpathlineto{\pgfqpoint{2.123323in}{0.616418in}}%
\pgfpathlineto{\pgfqpoint{2.124536in}{0.615585in}}%
\pgfpathlineto{\pgfqpoint{2.125056in}{0.616735in}}%
\pgfpathlineto{\pgfqpoint{2.125922in}{0.618299in}}%
\pgfpathlineto{\pgfqpoint{2.126441in}{0.617227in}}%
\pgfpathlineto{\pgfqpoint{2.131638in}{0.587010in}}%
\pgfpathlineto{\pgfqpoint{2.132851in}{0.579157in}}%
\pgfpathlineto{\pgfqpoint{2.133543in}{0.581185in}}%
\pgfpathlineto{\pgfqpoint{2.137354in}{0.604489in}}%
\pgfpathlineto{\pgfqpoint{2.138047in}{0.602178in}}%
\pgfpathlineto{\pgfqpoint{2.138913in}{0.600204in}}%
\pgfpathlineto{\pgfqpoint{2.139606in}{0.601505in}}%
\pgfpathlineto{\pgfqpoint{2.140472in}{0.602443in}}%
\pgfpathlineto{\pgfqpoint{2.140992in}{0.601323in}}%
\pgfpathlineto{\pgfqpoint{2.143763in}{0.597987in}}%
\pgfpathlineto{\pgfqpoint{2.144630in}{0.596961in}}%
\pgfpathlineto{\pgfqpoint{2.146188in}{0.594370in}}%
\pgfpathlineto{\pgfqpoint{2.146708in}{0.594894in}}%
\pgfpathlineto{\pgfqpoint{2.147228in}{0.595232in}}%
\pgfpathlineto{\pgfqpoint{2.148094in}{0.594264in}}%
\pgfpathlineto{\pgfqpoint{2.151732in}{0.589259in}}%
\pgfpathlineto{\pgfqpoint{2.153291in}{0.580137in}}%
\pgfpathlineto{\pgfqpoint{2.154330in}{0.583714in}}%
\pgfpathlineto{\pgfqpoint{2.156755in}{0.588814in}}%
\pgfpathlineto{\pgfqpoint{2.159873in}{0.590298in}}%
\pgfpathlineto{\pgfqpoint{2.164723in}{0.608334in}}%
\pgfpathlineto{\pgfqpoint{2.164896in}{0.608133in}}%
\pgfpathlineto{\pgfqpoint{2.166628in}{0.601700in}}%
\pgfpathlineto{\pgfqpoint{2.170439in}{0.586695in}}%
\pgfpathlineto{\pgfqpoint{2.170959in}{0.587099in}}%
\pgfpathlineto{\pgfqpoint{2.173211in}{0.587950in}}%
\pgfpathlineto{\pgfqpoint{2.176848in}{0.583925in}}%
\pgfpathlineto{\pgfqpoint{2.179620in}{0.575210in}}%
\pgfpathlineto{\pgfqpoint{2.180313in}{0.575839in}}%
\pgfpathlineto{\pgfqpoint{2.181872in}{0.577725in}}%
\pgfpathlineto{\pgfqpoint{2.182392in}{0.576659in}}%
\pgfpathlineto{\pgfqpoint{2.183951in}{0.571158in}}%
\pgfpathlineto{\pgfqpoint{2.184643in}{0.572829in}}%
\pgfpathlineto{\pgfqpoint{2.186376in}{0.576827in}}%
\pgfpathlineto{\pgfqpoint{2.186895in}{0.576386in}}%
\pgfpathlineto{\pgfqpoint{2.188281in}{0.574199in}}%
\pgfpathlineto{\pgfqpoint{2.188974in}{0.575969in}}%
\pgfpathlineto{\pgfqpoint{2.192785in}{0.582699in}}%
\pgfpathlineto{\pgfqpoint{2.196422in}{0.581305in}}%
\pgfpathlineto{\pgfqpoint{2.197462in}{0.580152in}}%
\pgfpathlineto{\pgfqpoint{2.199367in}{0.577628in}}%
\pgfpathlineto{\pgfqpoint{2.199887in}{0.578497in}}%
\pgfpathlineto{\pgfqpoint{2.203178in}{0.587693in}}%
\pgfpathlineto{\pgfqpoint{2.204044in}{0.587279in}}%
\pgfpathlineto{\pgfqpoint{2.205603in}{0.588969in}}%
\pgfpathlineto{\pgfqpoint{2.206642in}{0.591217in}}%
\pgfpathlineto{\pgfqpoint{2.207335in}{0.589455in}}%
\pgfpathlineto{\pgfqpoint{2.208894in}{0.582554in}}%
\pgfpathlineto{\pgfqpoint{2.209760in}{0.583903in}}%
\pgfpathlineto{\pgfqpoint{2.212532in}{0.591686in}}%
\pgfpathlineto{\pgfqpoint{2.213398in}{0.589537in}}%
\pgfpathlineto{\pgfqpoint{2.214957in}{0.583943in}}%
\pgfpathlineto{\pgfqpoint{2.215823in}{0.584732in}}%
\pgfpathlineto{\pgfqpoint{2.219980in}{0.588145in}}%
\pgfpathlineto{\pgfqpoint{2.223791in}{0.610498in}}%
\pgfpathlineto{\pgfqpoint{2.225004in}{0.604954in}}%
\pgfpathlineto{\pgfqpoint{2.227948in}{0.595722in}}%
\pgfpathlineto{\pgfqpoint{2.229334in}{0.593694in}}%
\pgfpathlineto{\pgfqpoint{2.230027in}{0.594399in}}%
\pgfpathlineto{\pgfqpoint{2.231413in}{0.600279in}}%
\pgfpathlineto{\pgfqpoint{2.232106in}{0.602347in}}%
\pgfpathlineto{\pgfqpoint{2.232799in}{0.600413in}}%
\pgfpathlineto{\pgfqpoint{2.237129in}{0.578560in}}%
\pgfpathlineto{\pgfqpoint{2.237822in}{0.579558in}}%
\pgfpathlineto{\pgfqpoint{2.238688in}{0.578200in}}%
\pgfpathlineto{\pgfqpoint{2.239901in}{0.574190in}}%
\pgfpathlineto{\pgfqpoint{2.241286in}{0.575117in}}%
\pgfpathlineto{\pgfqpoint{2.241806in}{0.575143in}}%
\pgfpathlineto{\pgfqpoint{2.242153in}{0.574244in}}%
\pgfpathlineto{\pgfqpoint{2.243712in}{0.566611in}}%
\pgfpathlineto{\pgfqpoint{2.244578in}{0.569128in}}%
\pgfpathlineto{\pgfqpoint{2.246656in}{0.569632in}}%
\pgfpathlineto{\pgfqpoint{2.247176in}{0.569516in}}%
\pgfpathlineto{\pgfqpoint{2.247349in}{0.569005in}}%
\pgfpathlineto{\pgfqpoint{2.248388in}{0.566138in}}%
\pgfpathlineto{\pgfqpoint{2.249081in}{0.567560in}}%
\pgfpathlineto{\pgfqpoint{2.250814in}{0.573734in}}%
\pgfpathlineto{\pgfqpoint{2.251333in}{0.573051in}}%
\pgfpathlineto{\pgfqpoint{2.252199in}{0.571370in}}%
\pgfpathlineto{\pgfqpoint{2.256703in}{0.555204in}}%
\pgfpathlineto{\pgfqpoint{2.257049in}{0.555247in}}%
\pgfpathlineto{\pgfqpoint{2.257396in}{0.556473in}}%
\pgfpathlineto{\pgfqpoint{2.260167in}{0.560722in}}%
\pgfpathlineto{\pgfqpoint{2.260341in}{0.560592in}}%
\pgfpathlineto{\pgfqpoint{2.262246in}{0.558252in}}%
\pgfpathlineto{\pgfqpoint{2.264671in}{0.550855in}}%
\pgfpathlineto{\pgfqpoint{2.265191in}{0.552693in}}%
\pgfpathlineto{\pgfqpoint{2.270734in}{0.575643in}}%
\pgfpathlineto{\pgfqpoint{2.273159in}{0.583920in}}%
\pgfpathlineto{\pgfqpoint{2.274025in}{0.582847in}}%
\pgfpathlineto{\pgfqpoint{2.274891in}{0.584332in}}%
\pgfpathlineto{\pgfqpoint{2.279048in}{0.592804in}}%
\pgfpathlineto{\pgfqpoint{2.280607in}{0.593737in}}%
\pgfpathlineto{\pgfqpoint{2.282513in}{0.596264in}}%
\pgfpathlineto{\pgfqpoint{2.283033in}{0.595516in}}%
\pgfpathlineto{\pgfqpoint{2.284245in}{0.594456in}}%
\pgfpathlineto{\pgfqpoint{2.284765in}{0.595098in}}%
\pgfpathlineto{\pgfqpoint{2.287190in}{0.602208in}}%
\pgfpathlineto{\pgfqpoint{2.288056in}{0.604571in}}%
\pgfpathlineto{\pgfqpoint{2.288749in}{0.602973in}}%
\pgfpathlineto{\pgfqpoint{2.292733in}{0.591555in}}%
\pgfpathlineto{\pgfqpoint{2.293599in}{0.591091in}}%
\pgfpathlineto{\pgfqpoint{2.294119in}{0.592029in}}%
\pgfpathlineto{\pgfqpoint{2.295504in}{0.594499in}}%
\pgfpathlineto{\pgfqpoint{2.296024in}{0.593337in}}%
\pgfpathlineto{\pgfqpoint{2.298449in}{0.585874in}}%
\pgfpathlineto{\pgfqpoint{2.299142in}{0.587680in}}%
\pgfpathlineto{\pgfqpoint{2.300008in}{0.600076in}}%
\pgfpathlineto{\pgfqpoint{2.301740in}{0.711724in}}%
\pgfpathlineto{\pgfqpoint{2.306417in}{0.962380in}}%
\pgfpathlineto{\pgfqpoint{2.310228in}{1.025023in}}%
\pgfpathlineto{\pgfqpoint{2.314385in}{1.044442in}}%
\pgfpathlineto{\pgfqpoint{2.318023in}{1.047019in}}%
\pgfpathlineto{\pgfqpoint{2.318543in}{1.045963in}}%
\pgfpathlineto{\pgfqpoint{2.337424in}{0.988621in}}%
\pgfpathlineto{\pgfqpoint{2.340022in}{0.980779in}}%
\pgfpathlineto{\pgfqpoint{2.340542in}{0.981230in}}%
\pgfpathlineto{\pgfqpoint{2.341927in}{0.981793in}}%
\pgfpathlineto{\pgfqpoint{2.342447in}{0.981366in}}%
\pgfpathlineto{\pgfqpoint{2.343486in}{0.981391in}}%
\pgfpathlineto{\pgfqpoint{2.343833in}{0.981959in}}%
\pgfpathlineto{\pgfqpoint{2.345219in}{0.983376in}}%
\pgfpathlineto{\pgfqpoint{2.345738in}{0.982488in}}%
\pgfpathlineto{\pgfqpoint{2.350415in}{0.967715in}}%
\pgfpathlineto{\pgfqpoint{2.355265in}{0.954980in}}%
\pgfpathlineto{\pgfqpoint{2.360635in}{0.944704in}}%
\pgfpathlineto{\pgfqpoint{2.363926in}{0.933414in}}%
\pgfpathlineto{\pgfqpoint{2.373800in}{0.909333in}}%
\pgfpathlineto{\pgfqpoint{2.376745in}{0.904921in}}%
\pgfpathlineto{\pgfqpoint{2.378997in}{0.892790in}}%
\pgfpathlineto{\pgfqpoint{2.382115in}{0.881604in}}%
\pgfpathlineto{\pgfqpoint{2.386099in}{0.874024in}}%
\pgfpathlineto{\pgfqpoint{2.387311in}{0.873638in}}%
\pgfpathlineto{\pgfqpoint{2.387484in}{0.873119in}}%
\pgfpathlineto{\pgfqpoint{2.388870in}{0.869071in}}%
\pgfpathlineto{\pgfqpoint{2.390083in}{0.869611in}}%
\pgfpathlineto{\pgfqpoint{2.391988in}{0.865469in}}%
\pgfpathlineto{\pgfqpoint{2.393547in}{0.864285in}}%
\pgfpathlineto{\pgfqpoint{2.394067in}{0.864909in}}%
\pgfpathlineto{\pgfqpoint{2.394586in}{0.865430in}}%
\pgfpathlineto{\pgfqpoint{2.395106in}{0.864468in}}%
\pgfpathlineto{\pgfqpoint{2.397012in}{0.853139in}}%
\pgfpathlineto{\pgfqpoint{2.401688in}{0.835156in}}%
\pgfpathlineto{\pgfqpoint{2.403421in}{0.832180in}}%
\pgfpathlineto{\pgfqpoint{2.407578in}{0.818375in}}%
\pgfpathlineto{\pgfqpoint{2.410349in}{0.816244in}}%
\pgfpathlineto{\pgfqpoint{2.413294in}{0.808992in}}%
\pgfpathlineto{\pgfqpoint{2.413987in}{0.809837in}}%
\pgfpathlineto{\pgfqpoint{2.417105in}{0.816008in}}%
\pgfpathlineto{\pgfqpoint{2.418318in}{0.814276in}}%
\pgfpathlineto{\pgfqpoint{2.423687in}{0.800381in}}%
\pgfpathlineto{\pgfqpoint{2.424727in}{0.802914in}}%
\pgfpathlineto{\pgfqpoint{2.426113in}{0.805529in}}%
\pgfpathlineto{\pgfqpoint{2.426632in}{0.804787in}}%
\pgfpathlineto{\pgfqpoint{2.432522in}{0.789960in}}%
\pgfpathlineto{\pgfqpoint{2.433388in}{0.792431in}}%
\pgfpathlineto{\pgfqpoint{2.436333in}{0.799350in}}%
\pgfpathlineto{\pgfqpoint{2.437718in}{0.806003in}}%
\pgfpathlineto{\pgfqpoint{2.438758in}{0.809535in}}%
\pgfpathlineto{\pgfqpoint{2.439450in}{0.807851in}}%
\pgfpathlineto{\pgfqpoint{2.441009in}{0.804881in}}%
\pgfpathlineto{\pgfqpoint{2.441529in}{0.805549in}}%
\pgfpathlineto{\pgfqpoint{2.443954in}{0.812165in}}%
\pgfpathlineto{\pgfqpoint{2.444994in}{0.810385in}}%
\pgfpathlineto{\pgfqpoint{2.446033in}{0.808608in}}%
\pgfpathlineto{\pgfqpoint{2.446726in}{0.809219in}}%
\pgfpathlineto{\pgfqpoint{2.448631in}{0.815140in}}%
\pgfpathlineto{\pgfqpoint{2.449497in}{0.813084in}}%
\pgfpathlineto{\pgfqpoint{2.454001in}{0.795797in}}%
\pgfpathlineto{\pgfqpoint{2.454867in}{0.797047in}}%
\pgfpathlineto{\pgfqpoint{2.456080in}{0.799447in}}%
\pgfpathlineto{\pgfqpoint{2.456773in}{0.798042in}}%
\pgfpathlineto{\pgfqpoint{2.459544in}{0.794985in}}%
\pgfpathlineto{\pgfqpoint{2.461449in}{0.788936in}}%
\pgfpathlineto{\pgfqpoint{2.469244in}{0.760932in}}%
\pgfpathlineto{\pgfqpoint{2.470110in}{0.762020in}}%
\pgfpathlineto{\pgfqpoint{2.471323in}{0.765533in}}%
\pgfpathlineto{\pgfqpoint{2.472189in}{0.764182in}}%
\pgfpathlineto{\pgfqpoint{2.478252in}{0.740379in}}%
\pgfpathlineto{\pgfqpoint{2.479811in}{0.742226in}}%
\pgfpathlineto{\pgfqpoint{2.481543in}{0.747576in}}%
\pgfpathlineto{\pgfqpoint{2.482756in}{0.745888in}}%
\pgfpathlineto{\pgfqpoint{2.484141in}{0.743594in}}%
\pgfpathlineto{\pgfqpoint{2.484834in}{0.744632in}}%
\pgfpathlineto{\pgfqpoint{2.488125in}{0.752656in}}%
\pgfpathlineto{\pgfqpoint{2.493668in}{0.763552in}}%
\pgfpathlineto{\pgfqpoint{2.494361in}{0.761404in}}%
\pgfpathlineto{\pgfqpoint{2.499211in}{0.741514in}}%
\pgfpathlineto{\pgfqpoint{2.499385in}{0.741701in}}%
\pgfpathlineto{\pgfqpoint{2.501983in}{0.750035in}}%
\pgfpathlineto{\pgfqpoint{2.504581in}{0.758432in}}%
\pgfpathlineto{\pgfqpoint{2.506487in}{0.757913in}}%
\pgfpathlineto{\pgfqpoint{2.508739in}{0.756918in}}%
\pgfpathlineto{\pgfqpoint{2.509778in}{0.757328in}}%
\pgfpathlineto{\pgfqpoint{2.510298in}{0.756272in}}%
\pgfpathlineto{\pgfqpoint{2.512376in}{0.750476in}}%
\pgfpathlineto{\pgfqpoint{2.513416in}{0.751261in}}%
\pgfpathlineto{\pgfqpoint{2.514108in}{0.751390in}}%
\pgfpathlineto{\pgfqpoint{2.514455in}{0.750587in}}%
\pgfpathlineto{\pgfqpoint{2.516187in}{0.743541in}}%
\pgfpathlineto{\pgfqpoint{2.517226in}{0.740635in}}%
\pgfpathlineto{\pgfqpoint{2.517919in}{0.741714in}}%
\pgfpathlineto{\pgfqpoint{2.518266in}{0.742308in}}%
\pgfpathlineto{\pgfqpoint{2.519305in}{0.741261in}}%
\pgfpathlineto{\pgfqpoint{2.519825in}{0.741029in}}%
\pgfpathlineto{\pgfqpoint{2.520171in}{0.742176in}}%
\pgfpathlineto{\pgfqpoint{2.521557in}{0.745236in}}%
\pgfpathlineto{\pgfqpoint{2.522077in}{0.744498in}}%
\pgfpathlineto{\pgfqpoint{2.522596in}{0.744023in}}%
\pgfpathlineto{\pgfqpoint{2.523116in}{0.745071in}}%
\pgfpathlineto{\pgfqpoint{2.524328in}{0.748180in}}%
\pgfpathlineto{\pgfqpoint{2.525021in}{0.746650in}}%
\pgfpathlineto{\pgfqpoint{2.532643in}{0.709774in}}%
\pgfpathlineto{\pgfqpoint{2.534202in}{0.705807in}}%
\pgfpathlineto{\pgfqpoint{2.534895in}{0.708227in}}%
\pgfpathlineto{\pgfqpoint{2.538359in}{0.721411in}}%
\pgfpathlineto{\pgfqpoint{2.539572in}{0.714098in}}%
\pgfpathlineto{\pgfqpoint{2.540091in}{0.711549in}}%
\pgfpathlineto{\pgfqpoint{2.540958in}{0.714639in}}%
\pgfpathlineto{\pgfqpoint{2.549965in}{0.770843in}}%
\pgfpathlineto{\pgfqpoint{2.550311in}{0.770499in}}%
\pgfpathlineto{\pgfqpoint{2.551524in}{0.766673in}}%
\pgfpathlineto{\pgfqpoint{2.553429in}{0.756699in}}%
\pgfpathlineto{\pgfqpoint{2.554469in}{0.758661in}}%
\pgfpathlineto{\pgfqpoint{2.560531in}{0.767310in}}%
\pgfpathlineto{\pgfqpoint{2.561051in}{0.766261in}}%
\pgfpathlineto{\pgfqpoint{2.562783in}{0.756391in}}%
\pgfpathlineto{\pgfqpoint{2.566248in}{0.739961in}}%
\pgfpathlineto{\pgfqpoint{2.568153in}{0.737245in}}%
\pgfpathlineto{\pgfqpoint{2.572310in}{0.720815in}}%
\pgfpathlineto{\pgfqpoint{2.572657in}{0.721465in}}%
\pgfpathlineto{\pgfqpoint{2.574043in}{0.727096in}}%
\pgfpathlineto{\pgfqpoint{2.575428in}{0.724196in}}%
\pgfpathlineto{\pgfqpoint{2.576468in}{0.723994in}}%
\pgfpathlineto{\pgfqpoint{2.576814in}{0.724499in}}%
\pgfpathlineto{\pgfqpoint{2.579239in}{0.727635in}}%
\pgfpathlineto{\pgfqpoint{2.579759in}{0.726807in}}%
\pgfpathlineto{\pgfqpoint{2.581838in}{0.724733in}}%
\pgfpathlineto{\pgfqpoint{2.582011in}{0.724807in}}%
\pgfpathlineto{\pgfqpoint{2.582877in}{0.728548in}}%
\pgfpathlineto{\pgfqpoint{2.584089in}{0.733216in}}%
\pgfpathlineto{\pgfqpoint{2.584782in}{0.731157in}}%
\pgfpathlineto{\pgfqpoint{2.589459in}{0.715935in}}%
\pgfpathlineto{\pgfqpoint{2.589632in}{0.716032in}}%
\pgfpathlineto{\pgfqpoint{2.590499in}{0.717577in}}%
\pgfpathlineto{\pgfqpoint{2.591191in}{0.716484in}}%
\pgfpathlineto{\pgfqpoint{2.593443in}{0.708647in}}%
\pgfpathlineto{\pgfqpoint{2.594136in}{0.710168in}}%
\pgfpathlineto{\pgfqpoint{2.599160in}{0.725643in}}%
\pgfpathlineto{\pgfqpoint{2.600026in}{0.724147in}}%
\pgfpathlineto{\pgfqpoint{2.601065in}{0.722722in}}%
\pgfpathlineto{\pgfqpoint{2.601585in}{0.724383in}}%
\pgfpathlineto{\pgfqpoint{2.603837in}{0.735407in}}%
\pgfpathlineto{\pgfqpoint{2.604876in}{0.731837in}}%
\pgfpathlineto{\pgfqpoint{2.608687in}{0.714724in}}%
\pgfpathlineto{\pgfqpoint{2.610246in}{0.715940in}}%
\pgfpathlineto{\pgfqpoint{2.611112in}{0.715603in}}%
\pgfpathlineto{\pgfqpoint{2.611458in}{0.716468in}}%
\pgfpathlineto{\pgfqpoint{2.612324in}{0.718309in}}%
\pgfpathlineto{\pgfqpoint{2.612844in}{0.716510in}}%
\pgfpathlineto{\pgfqpoint{2.614403in}{0.707309in}}%
\pgfpathlineto{\pgfqpoint{2.615269in}{0.708119in}}%
\pgfpathlineto{\pgfqpoint{2.616135in}{0.706291in}}%
\pgfpathlineto{\pgfqpoint{2.619253in}{0.695821in}}%
\pgfpathlineto{\pgfqpoint{2.620639in}{0.695104in}}%
\pgfpathlineto{\pgfqpoint{2.620985in}{0.695708in}}%
\pgfpathlineto{\pgfqpoint{2.622025in}{0.701622in}}%
\pgfpathlineto{\pgfqpoint{2.624969in}{0.720655in}}%
\pgfpathlineto{\pgfqpoint{2.627221in}{0.733218in}}%
\pgfpathlineto{\pgfqpoint{2.628087in}{0.729525in}}%
\pgfpathlineto{\pgfqpoint{2.629993in}{0.720224in}}%
\pgfpathlineto{\pgfqpoint{2.630859in}{0.721984in}}%
\pgfpathlineto{\pgfqpoint{2.631898in}{0.724484in}}%
\pgfpathlineto{\pgfqpoint{2.632764in}{0.723370in}}%
\pgfpathlineto{\pgfqpoint{2.633284in}{0.722610in}}%
\pgfpathlineto{\pgfqpoint{2.633977in}{0.724308in}}%
\pgfpathlineto{\pgfqpoint{2.634843in}{0.724965in}}%
\pgfpathlineto{\pgfqpoint{2.635189in}{0.723896in}}%
\pgfpathlineto{\pgfqpoint{2.639520in}{0.706350in}}%
\pgfpathlineto{\pgfqpoint{2.641599in}{0.705383in}}%
\pgfpathlineto{\pgfqpoint{2.642811in}{0.703706in}}%
\pgfpathlineto{\pgfqpoint{2.643504in}{0.704320in}}%
\pgfpathlineto{\pgfqpoint{2.644543in}{0.706169in}}%
\pgfpathlineto{\pgfqpoint{2.645583in}{0.704705in}}%
\pgfpathlineto{\pgfqpoint{2.645929in}{0.704447in}}%
\pgfpathlineto{\pgfqpoint{2.646622in}{0.705770in}}%
\pgfpathlineto{\pgfqpoint{2.648701in}{0.709869in}}%
\pgfpathlineto{\pgfqpoint{2.649047in}{0.709697in}}%
\pgfpathlineto{\pgfqpoint{2.650433in}{0.705796in}}%
\pgfpathlineto{\pgfqpoint{2.651819in}{0.701150in}}%
\pgfpathlineto{\pgfqpoint{2.652338in}{0.702353in}}%
\pgfpathlineto{\pgfqpoint{2.656842in}{0.716071in}}%
\pgfpathlineto{\pgfqpoint{2.657188in}{0.715610in}}%
\pgfpathlineto{\pgfqpoint{2.658574in}{0.713199in}}%
\pgfpathlineto{\pgfqpoint{2.659267in}{0.714354in}}%
\pgfpathlineto{\pgfqpoint{2.660999in}{0.715272in}}%
\pgfpathlineto{\pgfqpoint{2.663424in}{0.709724in}}%
\pgfpathlineto{\pgfqpoint{2.665849in}{0.704810in}}%
\pgfpathlineto{\pgfqpoint{2.667235in}{0.705974in}}%
\pgfpathlineto{\pgfqpoint{2.667582in}{0.704711in}}%
\pgfpathlineto{\pgfqpoint{2.670700in}{0.691484in}}%
\pgfpathlineto{\pgfqpoint{2.671046in}{0.692109in}}%
\pgfpathlineto{\pgfqpoint{2.674684in}{0.707346in}}%
\pgfpathlineto{\pgfqpoint{2.675377in}{0.705672in}}%
\pgfpathlineto{\pgfqpoint{2.679880in}{0.692262in}}%
\pgfpathlineto{\pgfqpoint{2.680400in}{0.691713in}}%
\pgfpathlineto{\pgfqpoint{2.680920in}{0.692932in}}%
\pgfpathlineto{\pgfqpoint{2.681786in}{0.695128in}}%
\pgfpathlineto{\pgfqpoint{2.682479in}{0.693332in}}%
\pgfpathlineto{\pgfqpoint{2.685077in}{0.680233in}}%
\pgfpathlineto{\pgfqpoint{2.686116in}{0.683099in}}%
\pgfpathlineto{\pgfqpoint{2.688195in}{0.699208in}}%
\pgfpathlineto{\pgfqpoint{2.689407in}{0.696611in}}%
\pgfpathlineto{\pgfqpoint{2.691659in}{0.703616in}}%
\pgfpathlineto{\pgfqpoint{2.692699in}{0.702211in}}%
\pgfpathlineto{\pgfqpoint{2.693738in}{0.697779in}}%
\pgfpathlineto{\pgfqpoint{2.695643in}{0.680104in}}%
\pgfpathlineto{\pgfqpoint{2.696856in}{0.684563in}}%
\pgfpathlineto{\pgfqpoint{2.697376in}{0.685625in}}%
\pgfpathlineto{\pgfqpoint{2.698242in}{0.683907in}}%
\pgfpathlineto{\pgfqpoint{2.700840in}{0.674347in}}%
\pgfpathlineto{\pgfqpoint{2.701360in}{0.676128in}}%
\pgfpathlineto{\pgfqpoint{2.702745in}{0.686199in}}%
\pgfpathlineto{\pgfqpoint{2.703785in}{0.683493in}}%
\pgfpathlineto{\pgfqpoint{2.704131in}{0.683442in}}%
\pgfpathlineto{\pgfqpoint{2.704478in}{0.684653in}}%
\pgfpathlineto{\pgfqpoint{2.707076in}{0.690535in}}%
\pgfpathlineto{\pgfqpoint{2.709674in}{0.705612in}}%
\pgfpathlineto{\pgfqpoint{2.711060in}{0.702395in}}%
\pgfpathlineto{\pgfqpoint{2.712619in}{0.698823in}}%
\pgfpathlineto{\pgfqpoint{2.713139in}{0.698990in}}%
\pgfpathlineto{\pgfqpoint{2.714005in}{0.698229in}}%
\pgfpathlineto{\pgfqpoint{2.714351in}{0.699354in}}%
\pgfpathlineto{\pgfqpoint{2.719375in}{0.711140in}}%
\pgfpathlineto{\pgfqpoint{2.721973in}{0.711562in}}%
\pgfpathlineto{\pgfqpoint{2.725957in}{0.705817in}}%
\pgfpathlineto{\pgfqpoint{2.726303in}{0.706504in}}%
\pgfpathlineto{\pgfqpoint{2.728209in}{0.712382in}}%
\pgfpathlineto{\pgfqpoint{2.728728in}{0.711647in}}%
\pgfpathlineto{\pgfqpoint{2.732020in}{0.702068in}}%
\pgfpathlineto{\pgfqpoint{2.735311in}{0.692412in}}%
\pgfpathlineto{\pgfqpoint{2.736177in}{0.694220in}}%
\pgfpathlineto{\pgfqpoint{2.736697in}{0.694267in}}%
\pgfpathlineto{\pgfqpoint{2.736870in}{0.693615in}}%
\pgfpathlineto{\pgfqpoint{2.740507in}{0.677642in}}%
\pgfpathlineto{\pgfqpoint{2.741200in}{0.679227in}}%
\pgfpathlineto{\pgfqpoint{2.743106in}{0.687739in}}%
\pgfpathlineto{\pgfqpoint{2.744318in}{0.691473in}}%
\pgfpathlineto{\pgfqpoint{2.745011in}{0.689818in}}%
\pgfpathlineto{\pgfqpoint{2.747609in}{0.681343in}}%
\pgfpathlineto{\pgfqpoint{2.748476in}{0.683841in}}%
\pgfpathlineto{\pgfqpoint{2.752460in}{0.700254in}}%
\pgfpathlineto{\pgfqpoint{2.752806in}{0.699273in}}%
\pgfpathlineto{\pgfqpoint{2.757483in}{0.680072in}}%
\pgfpathlineto{\pgfqpoint{2.757656in}{0.680118in}}%
\pgfpathlineto{\pgfqpoint{2.759042in}{0.686435in}}%
\pgfpathlineto{\pgfqpoint{2.762853in}{0.707997in}}%
\pgfpathlineto{\pgfqpoint{2.764065in}{0.712083in}}%
\pgfpathlineto{\pgfqpoint{2.764585in}{0.710812in}}%
\pgfpathlineto{\pgfqpoint{2.766837in}{0.705429in}}%
\pgfpathlineto{\pgfqpoint{2.767703in}{0.705767in}}%
\pgfpathlineto{\pgfqpoint{2.769782in}{0.704577in}}%
\pgfpathlineto{\pgfqpoint{2.770301in}{0.705577in}}%
\pgfpathlineto{\pgfqpoint{2.778443in}{0.725388in}}%
\pgfpathlineto{\pgfqpoint{2.781561in}{0.736279in}}%
\pgfpathlineto{\pgfqpoint{2.781734in}{0.735967in}}%
\pgfpathlineto{\pgfqpoint{2.783466in}{0.730251in}}%
\pgfpathlineto{\pgfqpoint{2.784679in}{0.731784in}}%
\pgfpathlineto{\pgfqpoint{2.785198in}{0.731283in}}%
\pgfpathlineto{\pgfqpoint{2.785891in}{0.732080in}}%
\pgfpathlineto{\pgfqpoint{2.786930in}{0.735369in}}%
\pgfpathlineto{\pgfqpoint{2.787450in}{0.733385in}}%
\pgfpathlineto{\pgfqpoint{2.796284in}{0.685728in}}%
\pgfpathlineto{\pgfqpoint{2.797497in}{0.681312in}}%
\pgfpathlineto{\pgfqpoint{2.798883in}{0.676253in}}%
\pgfpathlineto{\pgfqpoint{2.799576in}{0.677879in}}%
\pgfpathlineto{\pgfqpoint{2.801654in}{0.679875in}}%
\pgfpathlineto{\pgfqpoint{2.803213in}{0.678679in}}%
\pgfpathlineto{\pgfqpoint{2.803560in}{0.679549in}}%
\pgfpathlineto{\pgfqpoint{2.804945in}{0.682285in}}%
\pgfpathlineto{\pgfqpoint{2.805638in}{0.681217in}}%
\pgfpathlineto{\pgfqpoint{2.807544in}{0.679963in}}%
\pgfpathlineto{\pgfqpoint{2.807717in}{0.680175in}}%
\pgfpathlineto{\pgfqpoint{2.808237in}{0.681386in}}%
\pgfpathlineto{\pgfqpoint{2.808929in}{0.679897in}}%
\pgfpathlineto{\pgfqpoint{2.810142in}{0.673020in}}%
\pgfpathlineto{\pgfqpoint{2.811354in}{0.677044in}}%
\pgfpathlineto{\pgfqpoint{2.811701in}{0.677439in}}%
\pgfpathlineto{\pgfqpoint{2.812221in}{0.676035in}}%
\pgfpathlineto{\pgfqpoint{2.814992in}{0.661050in}}%
\pgfpathlineto{\pgfqpoint{2.815512in}{0.663099in}}%
\pgfpathlineto{\pgfqpoint{2.821228in}{0.693104in}}%
\pgfpathlineto{\pgfqpoint{2.821748in}{0.692987in}}%
\pgfpathlineto{\pgfqpoint{2.823480in}{0.696230in}}%
\pgfpathlineto{\pgfqpoint{2.824000in}{0.695564in}}%
\pgfpathlineto{\pgfqpoint{2.825039in}{0.692131in}}%
\pgfpathlineto{\pgfqpoint{2.825732in}{0.694525in}}%
\pgfpathlineto{\pgfqpoint{2.827984in}{0.698605in}}%
\pgfpathlineto{\pgfqpoint{2.830062in}{0.698839in}}%
\pgfpathlineto{\pgfqpoint{2.830409in}{0.698074in}}%
\pgfpathlineto{\pgfqpoint{2.831102in}{0.697514in}}%
\pgfpathlineto{\pgfqpoint{2.831794in}{0.698370in}}%
\pgfpathlineto{\pgfqpoint{2.832487in}{0.699065in}}%
\pgfpathlineto{\pgfqpoint{2.833180in}{0.697705in}}%
\pgfpathlineto{\pgfqpoint{2.835952in}{0.682694in}}%
\pgfpathlineto{\pgfqpoint{2.837164in}{0.685082in}}%
\pgfpathlineto{\pgfqpoint{2.840629in}{0.698422in}}%
\pgfpathlineto{\pgfqpoint{2.841668in}{0.693086in}}%
\pgfpathlineto{\pgfqpoint{2.844093in}{0.679211in}}%
\pgfpathlineto{\pgfqpoint{2.844613in}{0.681301in}}%
\pgfpathlineto{\pgfqpoint{2.845479in}{0.683263in}}%
\pgfpathlineto{\pgfqpoint{2.846172in}{0.682057in}}%
\pgfpathlineto{\pgfqpoint{2.846691in}{0.681803in}}%
\pgfpathlineto{\pgfqpoint{2.847038in}{0.682496in}}%
\pgfpathlineto{\pgfqpoint{2.849809in}{0.688954in}}%
\pgfpathlineto{\pgfqpoint{2.849983in}{0.688726in}}%
\pgfpathlineto{\pgfqpoint{2.851022in}{0.686168in}}%
\pgfpathlineto{\pgfqpoint{2.855179in}{0.661462in}}%
\pgfpathlineto{\pgfqpoint{2.855352in}{0.661790in}}%
\pgfpathlineto{\pgfqpoint{2.860376in}{0.681604in}}%
\pgfpathlineto{\pgfqpoint{2.861069in}{0.681149in}}%
\pgfpathlineto{\pgfqpoint{2.862454in}{0.676678in}}%
\pgfpathlineto{\pgfqpoint{2.864706in}{0.659159in}}%
\pgfpathlineto{\pgfqpoint{2.865746in}{0.662778in}}%
\pgfpathlineto{\pgfqpoint{2.870942in}{0.684204in}}%
\pgfpathlineto{\pgfqpoint{2.871116in}{0.683917in}}%
\pgfpathlineto{\pgfqpoint{2.872674in}{0.678454in}}%
\pgfpathlineto{\pgfqpoint{2.873367in}{0.680227in}}%
\pgfpathlineto{\pgfqpoint{2.875273in}{0.687082in}}%
\pgfpathlineto{\pgfqpoint{2.875966in}{0.685702in}}%
\pgfpathlineto{\pgfqpoint{2.879430in}{0.674534in}}%
\pgfpathlineto{\pgfqpoint{2.880123in}{0.674730in}}%
\pgfpathlineto{\pgfqpoint{2.882721in}{0.678026in}}%
\pgfpathlineto{\pgfqpoint{2.884973in}{0.688202in}}%
\pgfpathlineto{\pgfqpoint{2.886186in}{0.686171in}}%
\pgfpathlineto{\pgfqpoint{2.887745in}{0.688388in}}%
\pgfpathlineto{\pgfqpoint{2.888438in}{0.686239in}}%
\pgfpathlineto{\pgfqpoint{2.890170in}{0.682701in}}%
\pgfpathlineto{\pgfqpoint{2.890516in}{0.683244in}}%
\pgfpathlineto{\pgfqpoint{2.892075in}{0.687453in}}%
\pgfpathlineto{\pgfqpoint{2.892941in}{0.684913in}}%
\pgfpathlineto{\pgfqpoint{2.897272in}{0.666919in}}%
\pgfpathlineto{\pgfqpoint{2.898311in}{0.668186in}}%
\pgfpathlineto{\pgfqpoint{2.901256in}{0.678429in}}%
\pgfpathlineto{\pgfqpoint{2.901949in}{0.677986in}}%
\pgfpathlineto{\pgfqpoint{2.903681in}{0.674198in}}%
\pgfpathlineto{\pgfqpoint{2.904720in}{0.675624in}}%
\pgfpathlineto{\pgfqpoint{2.905933in}{0.678822in}}%
\pgfpathlineto{\pgfqpoint{2.906799in}{0.676669in}}%
\pgfpathlineto{\pgfqpoint{2.908185in}{0.672113in}}%
\pgfpathlineto{\pgfqpoint{2.908704in}{0.673488in}}%
\pgfpathlineto{\pgfqpoint{2.912342in}{0.680528in}}%
\pgfpathlineto{\pgfqpoint{2.912862in}{0.680688in}}%
\pgfpathlineto{\pgfqpoint{2.913208in}{0.679764in}}%
\pgfpathlineto{\pgfqpoint{2.916326in}{0.657286in}}%
\pgfpathlineto{\pgfqpoint{2.917712in}{0.661456in}}%
\pgfpathlineto{\pgfqpoint{2.918058in}{0.661944in}}%
\pgfpathlineto{\pgfqpoint{2.918751in}{0.660462in}}%
\pgfpathlineto{\pgfqpoint{2.919617in}{0.660201in}}%
\pgfpathlineto{\pgfqpoint{2.919790in}{0.660678in}}%
\pgfpathlineto{\pgfqpoint{2.922562in}{0.666148in}}%
\pgfpathlineto{\pgfqpoint{2.923601in}{0.666636in}}%
\pgfpathlineto{\pgfqpoint{2.923948in}{0.665469in}}%
\pgfpathlineto{\pgfqpoint{2.924294in}{0.665006in}}%
\pgfpathlineto{\pgfqpoint{2.924987in}{0.666921in}}%
\pgfpathlineto{\pgfqpoint{2.928971in}{0.675195in}}%
\pgfpathlineto{\pgfqpoint{2.930010in}{0.676904in}}%
\pgfpathlineto{\pgfqpoint{2.932262in}{0.687914in}}%
\pgfpathlineto{\pgfqpoint{2.933128in}{0.686465in}}%
\pgfpathlineto{\pgfqpoint{2.935034in}{0.675343in}}%
\pgfpathlineto{\pgfqpoint{2.937112in}{0.661745in}}%
\pgfpathlineto{\pgfqpoint{2.937979in}{0.665259in}}%
\pgfpathlineto{\pgfqpoint{2.943695in}{0.694323in}}%
\pgfpathlineto{\pgfqpoint{2.944561in}{0.693336in}}%
\pgfpathlineto{\pgfqpoint{2.946986in}{0.681179in}}%
\pgfpathlineto{\pgfqpoint{2.948025in}{0.683614in}}%
\pgfpathlineto{\pgfqpoint{2.948718in}{0.686759in}}%
\pgfpathlineto{\pgfqpoint{2.949411in}{0.683841in}}%
\pgfpathlineto{\pgfqpoint{2.950970in}{0.672833in}}%
\pgfpathlineto{\pgfqpoint{2.951663in}{0.677434in}}%
\pgfpathlineto{\pgfqpoint{2.952875in}{0.687342in}}%
\pgfpathlineto{\pgfqpoint{2.954088in}{0.685146in}}%
\pgfpathlineto{\pgfqpoint{2.957033in}{0.674263in}}%
\pgfpathlineto{\pgfqpoint{2.957899in}{0.674450in}}%
\pgfpathlineto{\pgfqpoint{2.958072in}{0.674946in}}%
\pgfpathlineto{\pgfqpoint{2.962229in}{0.690205in}}%
\pgfpathlineto{\pgfqpoint{2.962576in}{0.689603in}}%
\pgfpathlineto{\pgfqpoint{2.963442in}{0.688014in}}%
\pgfpathlineto{\pgfqpoint{2.964135in}{0.689327in}}%
\pgfpathlineto{\pgfqpoint{2.969158in}{0.706228in}}%
\pgfpathlineto{\pgfqpoint{2.971410in}{0.705221in}}%
\pgfpathlineto{\pgfqpoint{2.971930in}{0.705302in}}%
\pgfpathlineto{\pgfqpoint{2.972449in}{0.704264in}}%
\pgfpathlineto{\pgfqpoint{2.972969in}{0.703666in}}%
\pgfpathlineto{\pgfqpoint{2.973835in}{0.704614in}}%
\pgfpathlineto{\pgfqpoint{2.975394in}{0.705148in}}%
\pgfpathlineto{\pgfqpoint{2.975567in}{0.704940in}}%
\pgfpathlineto{\pgfqpoint{2.978685in}{0.698032in}}%
\pgfpathlineto{\pgfqpoint{2.979032in}{0.698655in}}%
\pgfpathlineto{\pgfqpoint{2.982323in}{0.705363in}}%
\pgfpathlineto{\pgfqpoint{2.984575in}{0.709090in}}%
\pgfpathlineto{\pgfqpoint{2.985961in}{0.707415in}}%
\pgfpathlineto{\pgfqpoint{2.987693in}{0.706103in}}%
\pgfpathlineto{\pgfqpoint{2.989079in}{0.702738in}}%
\pgfpathlineto{\pgfqpoint{2.989598in}{0.703577in}}%
\pgfpathlineto{\pgfqpoint{2.991157in}{0.702314in}}%
\pgfpathlineto{\pgfqpoint{2.992197in}{0.700731in}}%
\pgfpathlineto{\pgfqpoint{2.993582in}{0.698529in}}%
\pgfpathlineto{\pgfqpoint{2.993929in}{0.699579in}}%
\pgfpathlineto{\pgfqpoint{2.998779in}{0.719396in}}%
\pgfpathlineto{\pgfqpoint{2.999299in}{0.718706in}}%
\pgfpathlineto{\pgfqpoint{3.000858in}{0.714500in}}%
\pgfpathlineto{\pgfqpoint{3.001550in}{0.716901in}}%
\pgfpathlineto{\pgfqpoint{3.003802in}{0.733647in}}%
\pgfpathlineto{\pgfqpoint{3.005015in}{0.730462in}}%
\pgfpathlineto{\pgfqpoint{3.005881in}{0.731923in}}%
\pgfpathlineto{\pgfqpoint{3.007786in}{0.739367in}}%
\pgfpathlineto{\pgfqpoint{3.008479in}{0.737939in}}%
\pgfpathlineto{\pgfqpoint{3.011944in}{0.729585in}}%
\pgfpathlineto{\pgfqpoint{3.012463in}{0.730134in}}%
\pgfpathlineto{\pgfqpoint{3.017140in}{0.749857in}}%
\pgfpathlineto{\pgfqpoint{3.018699in}{0.755703in}}%
\pgfpathlineto{\pgfqpoint{3.019219in}{0.754228in}}%
\pgfpathlineto{\pgfqpoint{3.027533in}{0.714472in}}%
\pgfpathlineto{\pgfqpoint{3.028226in}{0.718587in}}%
\pgfpathlineto{\pgfqpoint{3.030132in}{0.734130in}}%
\pgfpathlineto{\pgfqpoint{3.030998in}{0.732681in}}%
\pgfpathlineto{\pgfqpoint{3.033076in}{0.727541in}}%
\pgfpathlineto{\pgfqpoint{3.034289in}{0.718187in}}%
\pgfpathlineto{\pgfqpoint{3.035155in}{0.721754in}}%
\pgfpathlineto{\pgfqpoint{3.038446in}{0.734817in}}%
\pgfpathlineto{\pgfqpoint{3.039312in}{0.733285in}}%
\pgfpathlineto{\pgfqpoint{3.040005in}{0.731953in}}%
\pgfpathlineto{\pgfqpoint{3.040525in}{0.733897in}}%
\pgfpathlineto{\pgfqpoint{3.041738in}{0.739754in}}%
\pgfpathlineto{\pgfqpoint{3.042430in}{0.737959in}}%
\pgfpathlineto{\pgfqpoint{3.044855in}{0.733475in}}%
\pgfpathlineto{\pgfqpoint{3.046068in}{0.730291in}}%
\pgfpathlineto{\pgfqpoint{3.051784in}{0.709868in}}%
\pgfpathlineto{\pgfqpoint{3.052304in}{0.710038in}}%
\pgfpathlineto{\pgfqpoint{3.052824in}{0.709903in}}%
\pgfpathlineto{\pgfqpoint{3.053170in}{0.710789in}}%
\pgfpathlineto{\pgfqpoint{3.054383in}{0.718279in}}%
\pgfpathlineto{\pgfqpoint{3.054902in}{0.720497in}}%
\pgfpathlineto{\pgfqpoint{3.055595in}{0.717419in}}%
\pgfpathlineto{\pgfqpoint{3.057674in}{0.700458in}}%
\pgfpathlineto{\pgfqpoint{3.058713in}{0.700601in}}%
\pgfpathlineto{\pgfqpoint{3.060965in}{0.699361in}}%
\pgfpathlineto{\pgfqpoint{3.062178in}{0.701367in}}%
\pgfpathlineto{\pgfqpoint{3.063563in}{0.707336in}}%
\pgfpathlineto{\pgfqpoint{3.065815in}{0.718100in}}%
\pgfpathlineto{\pgfqpoint{3.066508in}{0.716281in}}%
\pgfpathlineto{\pgfqpoint{3.067547in}{0.713377in}}%
\pgfpathlineto{\pgfqpoint{3.068413in}{0.715485in}}%
\pgfpathlineto{\pgfqpoint{3.069280in}{0.714360in}}%
\pgfpathlineto{\pgfqpoint{3.071185in}{0.709437in}}%
\pgfpathlineto{\pgfqpoint{3.071705in}{0.710660in}}%
\pgfpathlineto{\pgfqpoint{3.072398in}{0.711535in}}%
\pgfpathlineto{\pgfqpoint{3.073090in}{0.710562in}}%
\pgfpathlineto{\pgfqpoint{3.077421in}{0.706105in}}%
\pgfpathlineto{\pgfqpoint{3.078287in}{0.707380in}}%
\pgfpathlineto{\pgfqpoint{3.078980in}{0.705961in}}%
\pgfpathlineto{\pgfqpoint{3.079846in}{0.704863in}}%
\pgfpathlineto{\pgfqpoint{3.080366in}{0.706169in}}%
\pgfpathlineto{\pgfqpoint{3.081059in}{0.707324in}}%
\pgfpathlineto{\pgfqpoint{3.081578in}{0.706196in}}%
\pgfpathlineto{\pgfqpoint{3.083137in}{0.699346in}}%
\pgfpathlineto{\pgfqpoint{3.084003in}{0.702017in}}%
\pgfpathlineto{\pgfqpoint{3.086082in}{0.709204in}}%
\pgfpathlineto{\pgfqpoint{3.086602in}{0.708963in}}%
\pgfpathlineto{\pgfqpoint{3.087814in}{0.705730in}}%
\pgfpathlineto{\pgfqpoint{3.091279in}{0.696396in}}%
\pgfpathlineto{\pgfqpoint{3.092664in}{0.695776in}}%
\pgfpathlineto{\pgfqpoint{3.094916in}{0.674851in}}%
\pgfpathlineto{\pgfqpoint{3.096129in}{0.681915in}}%
\pgfpathlineto{\pgfqpoint{3.099247in}{0.704312in}}%
\pgfpathlineto{\pgfqpoint{3.099940in}{0.700884in}}%
\pgfpathlineto{\pgfqpoint{3.100979in}{0.695974in}}%
\pgfpathlineto{\pgfqpoint{3.102018in}{0.697498in}}%
\pgfpathlineto{\pgfqpoint{3.104097in}{0.704838in}}%
\pgfpathlineto{\pgfqpoint{3.104790in}{0.707076in}}%
\pgfpathlineto{\pgfqpoint{3.105656in}{0.705265in}}%
\pgfpathlineto{\pgfqpoint{3.109293in}{0.686693in}}%
\pgfpathlineto{\pgfqpoint{3.110160in}{0.689836in}}%
\pgfpathlineto{\pgfqpoint{3.111719in}{0.697851in}}%
\pgfpathlineto{\pgfqpoint{3.112585in}{0.694180in}}%
\pgfpathlineto{\pgfqpoint{3.115703in}{0.682948in}}%
\pgfpathlineto{\pgfqpoint{3.117608in}{0.687489in}}%
\pgfpathlineto{\pgfqpoint{3.118821in}{0.689918in}}%
\pgfpathlineto{\pgfqpoint{3.119340in}{0.688265in}}%
\pgfpathlineto{\pgfqpoint{3.122631in}{0.680158in}}%
\pgfpathlineto{\pgfqpoint{3.124190in}{0.678801in}}%
\pgfpathlineto{\pgfqpoint{3.124710in}{0.678149in}}%
\pgfpathlineto{\pgfqpoint{3.125056in}{0.679288in}}%
\pgfpathlineto{\pgfqpoint{3.126962in}{0.696880in}}%
\pgfpathlineto{\pgfqpoint{3.128174in}{0.690887in}}%
\pgfpathlineto{\pgfqpoint{3.128867in}{0.689498in}}%
\pgfpathlineto{\pgfqpoint{3.129387in}{0.691943in}}%
\pgfpathlineto{\pgfqpoint{3.130253in}{0.694635in}}%
\pgfpathlineto{\pgfqpoint{3.130946in}{0.692425in}}%
\pgfpathlineto{\pgfqpoint{3.133544in}{0.686611in}}%
\pgfpathlineto{\pgfqpoint{3.134757in}{0.688807in}}%
\pgfpathlineto{\pgfqpoint{3.136316in}{0.693227in}}%
\pgfpathlineto{\pgfqpoint{3.137009in}{0.691984in}}%
\pgfpathlineto{\pgfqpoint{3.138221in}{0.682485in}}%
\pgfpathlineto{\pgfqpoint{3.139780in}{0.667555in}}%
\pgfpathlineto{\pgfqpoint{3.140473in}{0.670401in}}%
\pgfpathlineto{\pgfqpoint{3.146016in}{0.685704in}}%
\pgfpathlineto{\pgfqpoint{3.146363in}{0.686054in}}%
\pgfpathlineto{\pgfqpoint{3.146882in}{0.684747in}}%
\pgfpathlineto{\pgfqpoint{3.147055in}{0.684346in}}%
\pgfpathlineto{\pgfqpoint{3.147748in}{0.686130in}}%
\pgfpathlineto{\pgfqpoint{3.149134in}{0.692588in}}%
\pgfpathlineto{\pgfqpoint{3.149827in}{0.688971in}}%
\pgfpathlineto{\pgfqpoint{3.153465in}{0.667447in}}%
\pgfpathlineto{\pgfqpoint{3.154331in}{0.669195in}}%
\pgfpathlineto{\pgfqpoint{3.154850in}{0.670229in}}%
\pgfpathlineto{\pgfqpoint{3.155890in}{0.668994in}}%
\pgfpathlineto{\pgfqpoint{3.157102in}{0.667526in}}%
\pgfpathlineto{\pgfqpoint{3.157795in}{0.668465in}}%
\pgfpathlineto{\pgfqpoint{3.162126in}{0.678565in}}%
\pgfpathlineto{\pgfqpoint{3.164897in}{0.686391in}}%
\pgfpathlineto{\pgfqpoint{3.164897in}{0.686391in}}%
\pgfusepath{stroke}%
\end{pgfscope}%
\begin{pgfscope}%
\pgfpathrectangle{\pgfqpoint{0.568671in}{0.451277in}}{\pgfqpoint{2.598305in}{0.798874in}}%
\pgfusepath{clip}%
\pgfsetrectcap%
\pgfsetroundjoin%
\pgfsetlinewidth{1.505625pt}%
\definecolor{currentstroke}{rgb}{0.121569,0.466667,0.705882}%
\pgfsetstrokecolor{currentstroke}%
\pgfsetstrokeopacity{0.250000}%
\pgfsetdash{}{0pt}%
\pgfpathmoveto{\pgfqpoint{0.651825in}{1.264040in}}%
\pgfpathlineto{\pgfqpoint{0.657187in}{1.255808in}}%
\pgfpathlineto{\pgfqpoint{0.663076in}{1.246299in}}%
\pgfpathlineto{\pgfqpoint{0.672257in}{1.229344in}}%
\pgfpathlineto{\pgfqpoint{0.683343in}{1.199714in}}%
\pgfpathlineto{\pgfqpoint{0.687327in}{1.191171in}}%
\pgfpathlineto{\pgfqpoint{0.693217in}{1.178143in}}%
\pgfpathlineto{\pgfqpoint{0.711058in}{1.142390in}}%
\pgfpathlineto{\pgfqpoint{0.730805in}{1.110956in}}%
\pgfpathlineto{\pgfqpoint{0.732538in}{1.107670in}}%
\pgfpathlineto{\pgfqpoint{0.736002in}{1.103305in}}%
\pgfpathlineto{\pgfqpoint{0.737214in}{1.102034in}}%
\pgfpathlineto{\pgfqpoint{0.741025in}{1.096196in}}%
\pgfpathlineto{\pgfqpoint{0.745356in}{1.091019in}}%
\pgfpathlineto{\pgfqpoint{0.752285in}{1.073106in}}%
\pgfpathlineto{\pgfqpoint{0.765796in}{1.057695in}}%
\pgfpathlineto{\pgfqpoint{0.770126in}{1.051593in}}%
\pgfpathlineto{\pgfqpoint{0.774457in}{1.047908in}}%
\pgfpathlineto{\pgfqpoint{0.781905in}{1.040426in}}%
\pgfpathlineto{\pgfqpoint{0.783984in}{1.038872in}}%
\pgfpathlineto{\pgfqpoint{0.790047in}{1.028325in}}%
\pgfpathlineto{\pgfqpoint{0.793338in}{1.027919in}}%
\pgfpathlineto{\pgfqpoint{0.801133in}{1.022962in}}%
\pgfpathlineto{\pgfqpoint{0.807195in}{1.020223in}}%
\pgfpathlineto{\pgfqpoint{0.808928in}{1.020656in}}%
\pgfpathlineto{\pgfqpoint{0.810140in}{1.018544in}}%
\pgfpathlineto{\pgfqpoint{0.811699in}{1.016741in}}%
\pgfpathlineto{\pgfqpoint{0.812046in}{1.017220in}}%
\pgfpathlineto{\pgfqpoint{0.812739in}{1.017023in}}%
\pgfpathlineto{\pgfqpoint{0.813085in}{1.016232in}}%
\pgfpathlineto{\pgfqpoint{0.816896in}{1.010547in}}%
\pgfpathlineto{\pgfqpoint{0.818455in}{1.011246in}}%
\pgfpathlineto{\pgfqpoint{0.820360in}{1.012524in}}%
\pgfpathlineto{\pgfqpoint{0.820880in}{1.011693in}}%
\pgfpathlineto{\pgfqpoint{0.823132in}{1.010410in}}%
\pgfpathlineto{\pgfqpoint{0.826076in}{1.007870in}}%
\pgfpathlineto{\pgfqpoint{0.829714in}{1.007027in}}%
\pgfpathlineto{\pgfqpoint{0.830580in}{1.007078in}}%
\pgfpathlineto{\pgfqpoint{0.831100in}{1.006460in}}%
\pgfpathlineto{\pgfqpoint{0.835777in}{1.001301in}}%
\pgfpathlineto{\pgfqpoint{0.840107in}{0.999524in}}%
\pgfpathlineto{\pgfqpoint{0.840800in}{0.999413in}}%
\pgfpathlineto{\pgfqpoint{0.841320in}{0.998583in}}%
\pgfpathlineto{\pgfqpoint{0.844784in}{0.994192in}}%
\pgfpathlineto{\pgfqpoint{0.847383in}{0.990836in}}%
\pgfpathlineto{\pgfqpoint{0.852579in}{0.990816in}}%
\pgfpathlineto{\pgfqpoint{0.853272in}{0.991076in}}%
\pgfpathlineto{\pgfqpoint{0.853792in}{0.990222in}}%
\pgfpathlineto{\pgfqpoint{0.858642in}{0.982410in}}%
\pgfpathlineto{\pgfqpoint{0.862280in}{0.979924in}}%
\pgfpathlineto{\pgfqpoint{0.862972in}{0.979423in}}%
\pgfpathlineto{\pgfqpoint{0.863146in}{0.978965in}}%
\pgfpathlineto{\pgfqpoint{0.865917in}{0.976127in}}%
\pgfpathlineto{\pgfqpoint{0.869035in}{0.975515in}}%
\pgfpathlineto{\pgfqpoint{0.870767in}{0.978239in}}%
\pgfpathlineto{\pgfqpoint{0.871460in}{0.976696in}}%
\pgfpathlineto{\pgfqpoint{0.873366in}{0.975664in}}%
\pgfpathlineto{\pgfqpoint{0.884105in}{0.973663in}}%
\pgfpathlineto{\pgfqpoint{0.887570in}{0.970220in}}%
\pgfpathlineto{\pgfqpoint{0.888782in}{0.967651in}}%
\pgfpathlineto{\pgfqpoint{0.892247in}{0.961287in}}%
\pgfpathlineto{\pgfqpoint{0.892420in}{0.961442in}}%
\pgfpathlineto{\pgfqpoint{0.893459in}{0.962563in}}%
\pgfpathlineto{\pgfqpoint{0.893979in}{0.961058in}}%
\pgfpathlineto{\pgfqpoint{0.896924in}{0.958047in}}%
\pgfpathlineto{\pgfqpoint{0.898136in}{0.957808in}}%
\pgfpathlineto{\pgfqpoint{0.898309in}{0.957442in}}%
\pgfpathlineto{\pgfqpoint{0.900042in}{0.955874in}}%
\pgfpathlineto{\pgfqpoint{0.900561in}{0.956263in}}%
\pgfpathlineto{\pgfqpoint{0.903506in}{0.954432in}}%
\pgfpathlineto{\pgfqpoint{0.903852in}{0.955408in}}%
\pgfpathlineto{\pgfqpoint{0.905065in}{0.957275in}}%
\pgfpathlineto{\pgfqpoint{0.905758in}{0.956264in}}%
\pgfpathlineto{\pgfqpoint{0.908183in}{0.954087in}}%
\pgfpathlineto{\pgfqpoint{0.908356in}{0.954226in}}%
\pgfpathlineto{\pgfqpoint{0.909742in}{0.955062in}}%
\pgfpathlineto{\pgfqpoint{0.910781in}{0.955313in}}%
\pgfpathlineto{\pgfqpoint{0.911301in}{0.954826in}}%
\pgfpathlineto{\pgfqpoint{0.915285in}{0.950621in}}%
\pgfpathlineto{\pgfqpoint{0.917883in}{0.945354in}}%
\pgfpathlineto{\pgfqpoint{0.918056in}{0.945494in}}%
\pgfpathlineto{\pgfqpoint{0.922041in}{0.944612in}}%
\pgfpathlineto{\pgfqpoint{0.924639in}{0.941291in}}%
\pgfpathlineto{\pgfqpoint{0.924985in}{0.941617in}}%
\pgfpathlineto{\pgfqpoint{0.925505in}{0.942298in}}%
\pgfpathlineto{\pgfqpoint{0.926371in}{0.941144in}}%
\pgfpathlineto{\pgfqpoint{0.930009in}{0.932498in}}%
\pgfpathlineto{\pgfqpoint{0.932087in}{0.928343in}}%
\pgfpathlineto{\pgfqpoint{0.932434in}{0.928585in}}%
\pgfpathlineto{\pgfqpoint{0.933300in}{0.927266in}}%
\pgfpathlineto{\pgfqpoint{0.939189in}{0.915689in}}%
\pgfpathlineto{\pgfqpoint{0.944559in}{0.908838in}}%
\pgfpathlineto{\pgfqpoint{0.947504in}{0.904860in}}%
\pgfpathlineto{\pgfqpoint{0.949409in}{0.905824in}}%
\pgfpathlineto{\pgfqpoint{0.949756in}{0.904850in}}%
\pgfpathlineto{\pgfqpoint{0.954779in}{0.887618in}}%
\pgfpathlineto{\pgfqpoint{0.958936in}{0.872464in}}%
\pgfpathlineto{\pgfqpoint{0.960149in}{0.870192in}}%
\pgfpathlineto{\pgfqpoint{0.961188in}{0.867163in}}%
\pgfpathlineto{\pgfqpoint{0.962228in}{0.867821in}}%
\pgfpathlineto{\pgfqpoint{0.962401in}{0.868057in}}%
\pgfpathlineto{\pgfqpoint{0.963094in}{0.866520in}}%
\pgfpathlineto{\pgfqpoint{0.965172in}{0.861891in}}%
\pgfpathlineto{\pgfqpoint{0.965346in}{0.862064in}}%
\pgfpathlineto{\pgfqpoint{0.966039in}{0.862069in}}%
\pgfpathlineto{\pgfqpoint{0.966731in}{0.861405in}}%
\pgfpathlineto{\pgfqpoint{0.968117in}{0.859899in}}%
\pgfpathlineto{\pgfqpoint{0.969849in}{0.859466in}}%
\pgfpathlineto{\pgfqpoint{0.973833in}{0.853903in}}%
\pgfpathlineto{\pgfqpoint{0.976951in}{0.846730in}}%
\pgfpathlineto{\pgfqpoint{0.979203in}{0.846149in}}%
\pgfpathlineto{\pgfqpoint{0.981628in}{0.841156in}}%
\pgfpathlineto{\pgfqpoint{0.984573in}{0.840125in}}%
\pgfpathlineto{\pgfqpoint{0.988384in}{0.831625in}}%
\pgfpathlineto{\pgfqpoint{0.988904in}{0.831964in}}%
\pgfpathlineto{\pgfqpoint{0.989770in}{0.831041in}}%
\pgfpathlineto{\pgfqpoint{0.990463in}{0.830098in}}%
\pgfpathlineto{\pgfqpoint{0.990982in}{0.831301in}}%
\pgfpathlineto{\pgfqpoint{0.993234in}{0.834770in}}%
\pgfpathlineto{\pgfqpoint{0.994100in}{0.832432in}}%
\pgfpathlineto{\pgfqpoint{0.997218in}{0.823508in}}%
\pgfpathlineto{\pgfqpoint{1.002415in}{0.805180in}}%
\pgfpathlineto{\pgfqpoint{1.002588in}{0.805411in}}%
\pgfpathlineto{\pgfqpoint{1.007092in}{0.813855in}}%
\pgfpathlineto{\pgfqpoint{1.007438in}{0.813420in}}%
\pgfpathlineto{\pgfqpoint{1.009170in}{0.814649in}}%
\pgfpathlineto{\pgfqpoint{1.009517in}{0.813392in}}%
\pgfpathlineto{\pgfqpoint{1.012462in}{0.811028in}}%
\pgfpathlineto{\pgfqpoint{1.014887in}{0.810428in}}%
\pgfpathlineto{\pgfqpoint{1.017831in}{0.807118in}}%
\pgfpathlineto{\pgfqpoint{1.020256in}{0.805745in}}%
\pgfpathlineto{\pgfqpoint{1.021469in}{0.808110in}}%
\pgfpathlineto{\pgfqpoint{1.023374in}{0.812420in}}%
\pgfpathlineto{\pgfqpoint{1.023721in}{0.811876in}}%
\pgfpathlineto{\pgfqpoint{1.027532in}{0.804738in}}%
\pgfpathlineto{\pgfqpoint{1.034461in}{0.788521in}}%
\pgfpathlineto{\pgfqpoint{1.035153in}{0.790787in}}%
\pgfpathlineto{\pgfqpoint{1.035846in}{0.792731in}}%
\pgfpathlineto{\pgfqpoint{1.036366in}{0.790214in}}%
\pgfpathlineto{\pgfqpoint{1.037405in}{0.788345in}}%
\pgfpathlineto{\pgfqpoint{1.037925in}{0.789779in}}%
\pgfpathlineto{\pgfqpoint{1.038964in}{0.791433in}}%
\pgfpathlineto{\pgfqpoint{1.039311in}{0.790140in}}%
\pgfpathlineto{\pgfqpoint{1.041909in}{0.781601in}}%
\pgfpathlineto{\pgfqpoint{1.042082in}{0.781658in}}%
\pgfpathlineto{\pgfqpoint{1.043988in}{0.781931in}}%
\pgfpathlineto{\pgfqpoint{1.044161in}{0.781491in}}%
\pgfpathlineto{\pgfqpoint{1.045200in}{0.780884in}}%
\pgfpathlineto{\pgfqpoint{1.045547in}{0.781560in}}%
\pgfpathlineto{\pgfqpoint{1.046586in}{0.783741in}}%
\pgfpathlineto{\pgfqpoint{1.047279in}{0.781737in}}%
\pgfpathlineto{\pgfqpoint{1.051436in}{0.769736in}}%
\pgfpathlineto{\pgfqpoint{1.052475in}{0.770997in}}%
\pgfpathlineto{\pgfqpoint{1.053515in}{0.772690in}}%
\pgfpathlineto{\pgfqpoint{1.054034in}{0.771226in}}%
\pgfpathlineto{\pgfqpoint{1.057326in}{0.764167in}}%
\pgfpathlineto{\pgfqpoint{1.057499in}{0.764224in}}%
\pgfpathlineto{\pgfqpoint{1.058538in}{0.765822in}}%
\pgfpathlineto{\pgfqpoint{1.059058in}{0.765052in}}%
\pgfpathlineto{\pgfqpoint{1.062869in}{0.750704in}}%
\pgfpathlineto{\pgfqpoint{1.063388in}{0.752166in}}%
\pgfpathlineto{\pgfqpoint{1.067892in}{0.764097in}}%
\pgfpathlineto{\pgfqpoint{1.068238in}{0.763486in}}%
\pgfpathlineto{\pgfqpoint{1.068412in}{0.763747in}}%
\pgfpathlineto{\pgfqpoint{1.068758in}{0.762817in}}%
\pgfpathlineto{\pgfqpoint{1.070837in}{0.753508in}}%
\pgfpathlineto{\pgfqpoint{1.071183in}{0.754840in}}%
\pgfpathlineto{\pgfqpoint{1.072049in}{0.752738in}}%
\pgfpathlineto{\pgfqpoint{1.074301in}{0.743743in}}%
\pgfpathlineto{\pgfqpoint{1.074821in}{0.743968in}}%
\pgfpathlineto{\pgfqpoint{1.077766in}{0.737840in}}%
\pgfpathlineto{\pgfqpoint{1.077939in}{0.737930in}}%
\pgfpathlineto{\pgfqpoint{1.079151in}{0.738544in}}%
\pgfpathlineto{\pgfqpoint{1.079325in}{0.737905in}}%
\pgfpathlineto{\pgfqpoint{1.081230in}{0.733582in}}%
\pgfpathlineto{\pgfqpoint{1.081403in}{0.733840in}}%
\pgfpathlineto{\pgfqpoint{1.083135in}{0.738008in}}%
\pgfpathlineto{\pgfqpoint{1.083828in}{0.741895in}}%
\pgfpathlineto{\pgfqpoint{1.084868in}{0.740327in}}%
\pgfpathlineto{\pgfqpoint{1.085561in}{0.742060in}}%
\pgfpathlineto{\pgfqpoint{1.085734in}{0.742781in}}%
\pgfpathlineto{\pgfqpoint{1.086600in}{0.740236in}}%
\pgfpathlineto{\pgfqpoint{1.087812in}{0.736510in}}%
\pgfpathlineto{\pgfqpoint{1.088159in}{0.738603in}}%
\pgfpathlineto{\pgfqpoint{1.089198in}{0.741028in}}%
\pgfpathlineto{\pgfqpoint{1.089891in}{0.739397in}}%
\pgfpathlineto{\pgfqpoint{1.093529in}{0.733159in}}%
\pgfpathlineto{\pgfqpoint{1.094222in}{0.734306in}}%
\pgfpathlineto{\pgfqpoint{1.095607in}{0.731851in}}%
\pgfpathlineto{\pgfqpoint{1.096300in}{0.734096in}}%
\pgfpathlineto{\pgfqpoint{1.098725in}{0.741773in}}%
\pgfpathlineto{\pgfqpoint{1.099591in}{0.740252in}}%
\pgfpathlineto{\pgfqpoint{1.099765in}{0.740077in}}%
\pgfpathlineto{\pgfqpoint{1.100111in}{0.741767in}}%
\pgfpathlineto{\pgfqpoint{1.101670in}{0.742470in}}%
\pgfpathlineto{\pgfqpoint{1.101843in}{0.742363in}}%
\pgfpathlineto{\pgfqpoint{1.105308in}{0.731748in}}%
\pgfpathlineto{\pgfqpoint{1.106347in}{0.727383in}}%
\pgfpathlineto{\pgfqpoint{1.107040in}{0.728896in}}%
\pgfpathlineto{\pgfqpoint{1.107213in}{0.729520in}}%
\pgfpathlineto{\pgfqpoint{1.107560in}{0.726922in}}%
\pgfpathlineto{\pgfqpoint{1.109638in}{0.718181in}}%
\pgfpathlineto{\pgfqpoint{1.110331in}{0.716658in}}%
\pgfpathlineto{\pgfqpoint{1.111024in}{0.717609in}}%
\pgfpathlineto{\pgfqpoint{1.112756in}{0.727539in}}%
\pgfpathlineto{\pgfqpoint{1.113449in}{0.723530in}}%
\pgfpathlineto{\pgfqpoint{1.114142in}{0.721509in}}%
\pgfpathlineto{\pgfqpoint{1.114662in}{0.723476in}}%
\pgfpathlineto{\pgfqpoint{1.116221in}{0.729903in}}%
\pgfpathlineto{\pgfqpoint{1.116740in}{0.728533in}}%
\pgfpathlineto{\pgfqpoint{1.117087in}{0.727555in}}%
\pgfpathlineto{\pgfqpoint{1.119165in}{0.719499in}}%
\pgfpathlineto{\pgfqpoint{1.119512in}{0.720910in}}%
\pgfpathlineto{\pgfqpoint{1.120378in}{0.726070in}}%
\pgfpathlineto{\pgfqpoint{1.121071in}{0.721662in}}%
\pgfpathlineto{\pgfqpoint{1.121764in}{0.722288in}}%
\pgfpathlineto{\pgfqpoint{1.122110in}{0.720650in}}%
\pgfpathlineto{\pgfqpoint{1.125228in}{0.708467in}}%
\pgfpathlineto{\pgfqpoint{1.125921in}{0.709136in}}%
\pgfpathlineto{\pgfqpoint{1.126787in}{0.702907in}}%
\pgfpathlineto{\pgfqpoint{1.127653in}{0.698443in}}%
\pgfpathlineto{\pgfqpoint{1.128346in}{0.700494in}}%
\pgfpathlineto{\pgfqpoint{1.128866in}{0.702100in}}%
\pgfpathlineto{\pgfqpoint{1.129905in}{0.700395in}}%
\pgfpathlineto{\pgfqpoint{1.130251in}{0.699781in}}%
\pgfpathlineto{\pgfqpoint{1.132157in}{0.694649in}}%
\pgfpathlineto{\pgfqpoint{1.134062in}{0.698988in}}%
\pgfpathlineto{\pgfqpoint{1.134235in}{0.697742in}}%
\pgfpathlineto{\pgfqpoint{1.137007in}{0.690041in}}%
\pgfpathlineto{\pgfqpoint{1.137353in}{0.690162in}}%
\pgfpathlineto{\pgfqpoint{1.138739in}{0.689346in}}%
\pgfpathlineto{\pgfqpoint{1.140645in}{0.686142in}}%
\pgfpathlineto{\pgfqpoint{1.141684in}{0.681920in}}%
\pgfpathlineto{\pgfqpoint{1.142896in}{0.682668in}}%
\pgfpathlineto{\pgfqpoint{1.144629in}{0.685164in}}%
\pgfpathlineto{\pgfqpoint{1.146188in}{0.694252in}}%
\pgfpathlineto{\pgfqpoint{1.148093in}{0.692375in}}%
\pgfpathlineto{\pgfqpoint{1.148266in}{0.692447in}}%
\pgfpathlineto{\pgfqpoint{1.148613in}{0.691165in}}%
\pgfpathlineto{\pgfqpoint{1.151384in}{0.687215in}}%
\pgfpathlineto{\pgfqpoint{1.153463in}{0.690538in}}%
\pgfpathlineto{\pgfqpoint{1.154329in}{0.688987in}}%
\pgfpathlineto{\pgfqpoint{1.154849in}{0.682930in}}%
\pgfpathlineto{\pgfqpoint{1.156927in}{0.672576in}}%
\pgfpathlineto{\pgfqpoint{1.157967in}{0.667682in}}%
\pgfpathlineto{\pgfqpoint{1.159006in}{0.661004in}}%
\pgfpathlineto{\pgfqpoint{1.159872in}{0.664497in}}%
\pgfpathlineto{\pgfqpoint{1.160565in}{0.666596in}}%
\pgfpathlineto{\pgfqpoint{1.160738in}{0.667625in}}%
\pgfpathlineto{\pgfqpoint{1.161431in}{0.663872in}}%
\pgfpathlineto{\pgfqpoint{1.163163in}{0.654286in}}%
\pgfpathlineto{\pgfqpoint{1.164029in}{0.657654in}}%
\pgfpathlineto{\pgfqpoint{1.164722in}{0.653637in}}%
\pgfpathlineto{\pgfqpoint{1.165415in}{0.646993in}}%
\pgfpathlineto{\pgfqpoint{1.166281in}{0.652763in}}%
\pgfpathlineto{\pgfqpoint{1.167667in}{0.646339in}}%
\pgfpathlineto{\pgfqpoint{1.168187in}{0.650290in}}%
\pgfpathlineto{\pgfqpoint{1.170612in}{0.662201in}}%
\pgfpathlineto{\pgfqpoint{1.170958in}{0.661668in}}%
\pgfpathlineto{\pgfqpoint{1.171651in}{0.663341in}}%
\pgfpathlineto{\pgfqpoint{1.174076in}{0.676853in}}%
\pgfpathlineto{\pgfqpoint{1.174249in}{0.677167in}}%
\pgfpathlineto{\pgfqpoint{1.174596in}{0.675989in}}%
\pgfpathlineto{\pgfqpoint{1.174942in}{0.676161in}}%
\pgfpathlineto{\pgfqpoint{1.175982in}{0.671671in}}%
\pgfpathlineto{\pgfqpoint{1.176674in}{0.674942in}}%
\pgfpathlineto{\pgfqpoint{1.178233in}{0.688684in}}%
\pgfpathlineto{\pgfqpoint{1.178753in}{0.687672in}}%
\pgfpathlineto{\pgfqpoint{1.179966in}{0.685187in}}%
\pgfpathlineto{\pgfqpoint{1.180832in}{0.686509in}}%
\pgfpathlineto{\pgfqpoint{1.182217in}{0.694148in}}%
\pgfpathlineto{\pgfqpoint{1.182737in}{0.690521in}}%
\pgfpathlineto{\pgfqpoint{1.184643in}{0.677204in}}%
\pgfpathlineto{\pgfqpoint{1.185162in}{0.678795in}}%
\pgfpathlineto{\pgfqpoint{1.187934in}{0.682954in}}%
\pgfpathlineto{\pgfqpoint{1.188453in}{0.681770in}}%
\pgfpathlineto{\pgfqpoint{1.188800in}{0.684141in}}%
\pgfpathlineto{\pgfqpoint{1.189839in}{0.689348in}}%
\pgfpathlineto{\pgfqpoint{1.191052in}{0.696576in}}%
\pgfpathlineto{\pgfqpoint{1.191745in}{0.693764in}}%
\pgfpathlineto{\pgfqpoint{1.194343in}{0.686172in}}%
\pgfpathlineto{\pgfqpoint{1.195036in}{0.687749in}}%
\pgfpathlineto{\pgfqpoint{1.195729in}{0.689170in}}%
\pgfpathlineto{\pgfqpoint{1.196768in}{0.687677in}}%
\pgfpathlineto{\pgfqpoint{1.197288in}{0.687127in}}%
\pgfpathlineto{\pgfqpoint{1.197461in}{0.688254in}}%
\pgfpathlineto{\pgfqpoint{1.198154in}{0.689861in}}%
\pgfpathlineto{\pgfqpoint{1.198847in}{0.687436in}}%
\pgfpathlineto{\pgfqpoint{1.201618in}{0.675146in}}%
\pgfpathlineto{\pgfqpoint{1.204736in}{0.660021in}}%
\pgfpathlineto{\pgfqpoint{1.205083in}{0.661213in}}%
\pgfpathlineto{\pgfqpoint{1.205602in}{0.664326in}}%
\pgfpathlineto{\pgfqpoint{1.206468in}{0.660782in}}%
\pgfpathlineto{\pgfqpoint{1.208720in}{0.656613in}}%
\pgfpathlineto{\pgfqpoint{1.212011in}{0.662188in}}%
\pgfpathlineto{\pgfqpoint{1.212185in}{0.661389in}}%
\pgfpathlineto{\pgfqpoint{1.213570in}{0.657824in}}%
\pgfpathlineto{\pgfqpoint{1.214090in}{0.659571in}}%
\pgfpathlineto{\pgfqpoint{1.214956in}{0.661125in}}%
\pgfpathlineto{\pgfqpoint{1.215303in}{0.658863in}}%
\pgfpathlineto{\pgfqpoint{1.215822in}{0.657560in}}%
\pgfpathlineto{\pgfqpoint{1.216342in}{0.660060in}}%
\pgfpathlineto{\pgfqpoint{1.217901in}{0.671325in}}%
\pgfpathlineto{\pgfqpoint{1.218594in}{0.667369in}}%
\pgfpathlineto{\pgfqpoint{1.219460in}{0.661581in}}%
\pgfpathlineto{\pgfqpoint{1.220326in}{0.663280in}}%
\pgfpathlineto{\pgfqpoint{1.221192in}{0.664234in}}%
\pgfpathlineto{\pgfqpoint{1.221538in}{0.662960in}}%
\pgfpathlineto{\pgfqpoint{1.224137in}{0.640166in}}%
\pgfpathlineto{\pgfqpoint{1.226042in}{0.631236in}}%
\pgfpathlineto{\pgfqpoint{1.226215in}{0.631442in}}%
\pgfpathlineto{\pgfqpoint{1.231066in}{0.650886in}}%
\pgfpathlineto{\pgfqpoint{1.231412in}{0.650420in}}%
\pgfpathlineto{\pgfqpoint{1.237475in}{0.672322in}}%
\pgfpathlineto{\pgfqpoint{1.240593in}{0.682429in}}%
\pgfpathlineto{\pgfqpoint{1.240766in}{0.681788in}}%
\pgfpathlineto{\pgfqpoint{1.243884in}{0.652355in}}%
\pgfpathlineto{\pgfqpoint{1.244577in}{0.649016in}}%
\pgfpathlineto{\pgfqpoint{1.245270in}{0.653407in}}%
\pgfpathlineto{\pgfqpoint{1.249773in}{0.680727in}}%
\pgfpathlineto{\pgfqpoint{1.251332in}{0.684291in}}%
\pgfpathlineto{\pgfqpoint{1.252025in}{0.681864in}}%
\pgfpathlineto{\pgfqpoint{1.252372in}{0.681014in}}%
\pgfpathlineto{\pgfqpoint{1.252718in}{0.682408in}}%
\pgfpathlineto{\pgfqpoint{1.253065in}{0.683798in}}%
\pgfpathlineto{\pgfqpoint{1.253757in}{0.681186in}}%
\pgfpathlineto{\pgfqpoint{1.253931in}{0.681422in}}%
\pgfpathlineto{\pgfqpoint{1.254624in}{0.677602in}}%
\pgfpathlineto{\pgfqpoint{1.255316in}{0.680375in}}%
\pgfpathlineto{\pgfqpoint{1.256356in}{0.681899in}}%
\pgfpathlineto{\pgfqpoint{1.256875in}{0.680985in}}%
\pgfpathlineto{\pgfqpoint{1.257049in}{0.681276in}}%
\pgfpathlineto{\pgfqpoint{1.257395in}{0.679442in}}%
\pgfpathlineto{\pgfqpoint{1.257742in}{0.679541in}}%
\pgfpathlineto{\pgfqpoint{1.259300in}{0.675201in}}%
\pgfpathlineto{\pgfqpoint{1.259820in}{0.677871in}}%
\pgfpathlineto{\pgfqpoint{1.263111in}{0.700383in}}%
\pgfpathlineto{\pgfqpoint{1.263631in}{0.701164in}}%
\pgfpathlineto{\pgfqpoint{1.264497in}{0.699914in}}%
\pgfpathlineto{\pgfqpoint{1.267962in}{0.682363in}}%
\pgfpathlineto{\pgfqpoint{1.268135in}{0.682455in}}%
\pgfpathlineto{\pgfqpoint{1.271253in}{0.693356in}}%
\pgfpathlineto{\pgfqpoint{1.272119in}{0.690259in}}%
\pgfpathlineto{\pgfqpoint{1.272292in}{0.690341in}}%
\pgfpathlineto{\pgfqpoint{1.272465in}{0.689353in}}%
\pgfpathlineto{\pgfqpoint{1.273331in}{0.682803in}}%
\pgfpathlineto{\pgfqpoint{1.274197in}{0.687061in}}%
\pgfpathlineto{\pgfqpoint{1.276623in}{0.697103in}}%
\pgfpathlineto{\pgfqpoint{1.276796in}{0.696684in}}%
\pgfpathlineto{\pgfqpoint{1.277835in}{0.690167in}}%
\pgfpathlineto{\pgfqpoint{1.278701in}{0.692341in}}%
\pgfpathlineto{\pgfqpoint{1.278874in}{0.692830in}}%
\pgfpathlineto{\pgfqpoint{1.279567in}{0.690368in}}%
\pgfpathlineto{\pgfqpoint{1.280260in}{0.688004in}}%
\pgfpathlineto{\pgfqpoint{1.281646in}{0.679135in}}%
\pgfpathlineto{\pgfqpoint{1.283032in}{0.682706in}}%
\pgfpathlineto{\pgfqpoint{1.283205in}{0.683095in}}%
\pgfpathlineto{\pgfqpoint{1.283898in}{0.681160in}}%
\pgfpathlineto{\pgfqpoint{1.284071in}{0.680665in}}%
\pgfpathlineto{\pgfqpoint{1.284417in}{0.683359in}}%
\pgfpathlineto{\pgfqpoint{1.285110in}{0.691613in}}%
\pgfpathlineto{\pgfqpoint{1.286150in}{0.685819in}}%
\pgfpathlineto{\pgfqpoint{1.286496in}{0.686769in}}%
\pgfpathlineto{\pgfqpoint{1.286843in}{0.684733in}}%
\pgfpathlineto{\pgfqpoint{1.288748in}{0.677744in}}%
\pgfpathlineto{\pgfqpoint{1.290827in}{0.663367in}}%
\pgfpathlineto{\pgfqpoint{1.292039in}{0.660759in}}%
\pgfpathlineto{\pgfqpoint{1.295504in}{0.637597in}}%
\pgfpathlineto{\pgfqpoint{1.300527in}{0.662069in}}%
\pgfpathlineto{\pgfqpoint{1.301047in}{0.660400in}}%
\pgfpathlineto{\pgfqpoint{1.301739in}{0.658395in}}%
\pgfpathlineto{\pgfqpoint{1.302086in}{0.661277in}}%
\pgfpathlineto{\pgfqpoint{1.304684in}{0.667671in}}%
\pgfpathlineto{\pgfqpoint{1.305377in}{0.665385in}}%
\pgfpathlineto{\pgfqpoint{1.306763in}{0.665896in}}%
\pgfpathlineto{\pgfqpoint{1.306936in}{0.666368in}}%
\pgfpathlineto{\pgfqpoint{1.307802in}{0.665286in}}%
\pgfpathlineto{\pgfqpoint{1.308495in}{0.661050in}}%
\pgfpathlineto{\pgfqpoint{1.309188in}{0.665691in}}%
\pgfpathlineto{\pgfqpoint{1.310920in}{0.672776in}}%
\pgfpathlineto{\pgfqpoint{1.311786in}{0.671919in}}%
\pgfpathlineto{\pgfqpoint{1.313345in}{0.668012in}}%
\pgfpathlineto{\pgfqpoint{1.314038in}{0.670894in}}%
\pgfpathlineto{\pgfqpoint{1.319062in}{0.689967in}}%
\pgfpathlineto{\pgfqpoint{1.319235in}{0.689240in}}%
\pgfpathlineto{\pgfqpoint{1.320620in}{0.684167in}}%
\pgfpathlineto{\pgfqpoint{1.321140in}{0.684584in}}%
\pgfpathlineto{\pgfqpoint{1.324085in}{0.673731in}}%
\pgfpathlineto{\pgfqpoint{1.324951in}{0.676157in}}%
\pgfpathlineto{\pgfqpoint{1.325124in}{0.676576in}}%
\pgfpathlineto{\pgfqpoint{1.325817in}{0.675006in}}%
\pgfpathlineto{\pgfqpoint{1.328415in}{0.670335in}}%
\pgfpathlineto{\pgfqpoint{1.328589in}{0.670501in}}%
\pgfpathlineto{\pgfqpoint{1.329282in}{0.670239in}}%
\pgfpathlineto{\pgfqpoint{1.329628in}{0.671412in}}%
\pgfpathlineto{\pgfqpoint{1.329974in}{0.672215in}}%
\pgfpathlineto{\pgfqpoint{1.330494in}{0.669627in}}%
\pgfpathlineto{\pgfqpoint{1.330840in}{0.668765in}}%
\pgfpathlineto{\pgfqpoint{1.331707in}{0.670048in}}%
\pgfpathlineto{\pgfqpoint{1.332226in}{0.672474in}}%
\pgfpathlineto{\pgfqpoint{1.333266in}{0.670606in}}%
\pgfpathlineto{\pgfqpoint{1.335691in}{0.673046in}}%
\pgfpathlineto{\pgfqpoint{1.337076in}{0.667270in}}%
\pgfpathlineto{\pgfqpoint{1.337769in}{0.669980in}}%
\pgfpathlineto{\pgfqpoint{1.338809in}{0.669561in}}%
\pgfpathlineto{\pgfqpoint{1.340714in}{0.661835in}}%
\pgfpathlineto{\pgfqpoint{1.340887in}{0.662243in}}%
\pgfpathlineto{\pgfqpoint{1.342446in}{0.668895in}}%
\pgfpathlineto{\pgfqpoint{1.343486in}{0.667307in}}%
\pgfpathlineto{\pgfqpoint{1.344352in}{0.665284in}}%
\pgfpathlineto{\pgfqpoint{1.344871in}{0.666966in}}%
\pgfpathlineto{\pgfqpoint{1.345045in}{0.667506in}}%
\pgfpathlineto{\pgfqpoint{1.345391in}{0.665031in}}%
\pgfpathlineto{\pgfqpoint{1.347643in}{0.654329in}}%
\pgfpathlineto{\pgfqpoint{1.348163in}{0.656578in}}%
\pgfpathlineto{\pgfqpoint{1.349721in}{0.663964in}}%
\pgfpathlineto{\pgfqpoint{1.350241in}{0.661834in}}%
\pgfpathlineto{\pgfqpoint{1.352666in}{0.649810in}}%
\pgfpathlineto{\pgfqpoint{1.352839in}{0.650391in}}%
\pgfpathlineto{\pgfqpoint{1.353532in}{0.654704in}}%
\pgfpathlineto{\pgfqpoint{1.354398in}{0.651895in}}%
\pgfpathlineto{\pgfqpoint{1.355265in}{0.654218in}}%
\pgfpathlineto{\pgfqpoint{1.356131in}{0.652134in}}%
\pgfpathlineto{\pgfqpoint{1.356824in}{0.651487in}}%
\pgfpathlineto{\pgfqpoint{1.357516in}{0.652369in}}%
\pgfpathlineto{\pgfqpoint{1.358209in}{0.656206in}}%
\pgfpathlineto{\pgfqpoint{1.359249in}{0.660944in}}%
\pgfpathlineto{\pgfqpoint{1.359941in}{0.658879in}}%
\pgfpathlineto{\pgfqpoint{1.360288in}{0.657782in}}%
\pgfpathlineto{\pgfqpoint{1.360808in}{0.660440in}}%
\pgfpathlineto{\pgfqpoint{1.361674in}{0.667094in}}%
\pgfpathlineto{\pgfqpoint{1.362193in}{0.660959in}}%
\pgfpathlineto{\pgfqpoint{1.362367in}{0.659486in}}%
\pgfpathlineto{\pgfqpoint{1.363579in}{0.661826in}}%
\pgfpathlineto{\pgfqpoint{1.364099in}{0.664653in}}%
\pgfpathlineto{\pgfqpoint{1.364792in}{0.660025in}}%
\pgfpathlineto{\pgfqpoint{1.366870in}{0.641576in}}%
\pgfpathlineto{\pgfqpoint{1.367563in}{0.643447in}}%
\pgfpathlineto{\pgfqpoint{1.367736in}{0.642798in}}%
\pgfpathlineto{\pgfqpoint{1.368256in}{0.645111in}}%
\pgfpathlineto{\pgfqpoint{1.368776in}{0.643548in}}%
\pgfpathlineto{\pgfqpoint{1.375358in}{0.677491in}}%
\pgfpathlineto{\pgfqpoint{1.375878in}{0.676409in}}%
\pgfpathlineto{\pgfqpoint{1.380035in}{0.659726in}}%
\pgfpathlineto{\pgfqpoint{1.381421in}{0.655176in}}%
\pgfpathlineto{\pgfqpoint{1.381594in}{0.656413in}}%
\pgfpathlineto{\pgfqpoint{1.383846in}{0.661300in}}%
\pgfpathlineto{\pgfqpoint{1.384539in}{0.662408in}}%
\pgfpathlineto{\pgfqpoint{1.384712in}{0.661799in}}%
\pgfpathlineto{\pgfqpoint{1.387310in}{0.648026in}}%
\pgfpathlineto{\pgfqpoint{1.387484in}{0.648699in}}%
\pgfpathlineto{\pgfqpoint{1.392160in}{0.668888in}}%
\pgfpathlineto{\pgfqpoint{1.393373in}{0.667646in}}%
\pgfpathlineto{\pgfqpoint{1.394412in}{0.660488in}}%
\pgfpathlineto{\pgfqpoint{1.395452in}{0.664294in}}%
\pgfpathlineto{\pgfqpoint{1.395798in}{0.665297in}}%
\pgfpathlineto{\pgfqpoint{1.396145in}{0.663922in}}%
\pgfpathlineto{\pgfqpoint{1.397357in}{0.656290in}}%
\pgfpathlineto{\pgfqpoint{1.397877in}{0.659320in}}%
\pgfpathlineto{\pgfqpoint{1.401514in}{0.690829in}}%
\pgfpathlineto{\pgfqpoint{1.403073in}{0.692488in}}%
\pgfpathlineto{\pgfqpoint{1.403247in}{0.691105in}}%
\pgfpathlineto{\pgfqpoint{1.405152in}{0.684934in}}%
\pgfpathlineto{\pgfqpoint{1.405845in}{0.687024in}}%
\pgfpathlineto{\pgfqpoint{1.407057in}{0.694176in}}%
\pgfpathlineto{\pgfqpoint{1.407750in}{0.692845in}}%
\pgfpathlineto{\pgfqpoint{1.408616in}{0.692676in}}%
\pgfpathlineto{\pgfqpoint{1.408790in}{0.692088in}}%
\pgfpathlineto{\pgfqpoint{1.409829in}{0.686119in}}%
\pgfpathlineto{\pgfqpoint{1.410868in}{0.687569in}}%
\pgfpathlineto{\pgfqpoint{1.412774in}{0.691171in}}%
\pgfpathlineto{\pgfqpoint{1.413467in}{0.689982in}}%
\pgfpathlineto{\pgfqpoint{1.418144in}{0.674678in}}%
\pgfpathlineto{\pgfqpoint{1.418317in}{0.675135in}}%
\pgfpathlineto{\pgfqpoint{1.419010in}{0.683187in}}%
\pgfpathlineto{\pgfqpoint{1.420222in}{0.680761in}}%
\pgfpathlineto{\pgfqpoint{1.422301in}{0.671171in}}%
\pgfpathlineto{\pgfqpoint{1.422647in}{0.672184in}}%
\pgfpathlineto{\pgfqpoint{1.423687in}{0.673824in}}%
\pgfpathlineto{\pgfqpoint{1.424553in}{0.676896in}}%
\pgfpathlineto{\pgfqpoint{1.425246in}{0.675065in}}%
\pgfpathlineto{\pgfqpoint{1.428017in}{0.665976in}}%
\pgfpathlineto{\pgfqpoint{1.430096in}{0.661689in}}%
\pgfpathlineto{\pgfqpoint{1.430269in}{0.662015in}}%
\pgfpathlineto{\pgfqpoint{1.430789in}{0.659864in}}%
\pgfpathlineto{\pgfqpoint{1.431481in}{0.660947in}}%
\pgfpathlineto{\pgfqpoint{1.434080in}{0.671768in}}%
\pgfpathlineto{\pgfqpoint{1.434426in}{0.670456in}}%
\pgfpathlineto{\pgfqpoint{1.434773in}{0.667097in}}%
\pgfpathlineto{\pgfqpoint{1.435292in}{0.678959in}}%
\pgfpathlineto{\pgfqpoint{1.436158in}{0.676314in}}%
\pgfpathlineto{\pgfqpoint{1.436851in}{0.702516in}}%
\pgfpathlineto{\pgfqpoint{1.440489in}{0.802318in}}%
\pgfpathlineto{\pgfqpoint{1.447764in}{0.916832in}}%
\pgfpathlineto{\pgfqpoint{1.450016in}{0.918610in}}%
\pgfpathlineto{\pgfqpoint{1.450363in}{0.918399in}}%
\pgfpathlineto{\pgfqpoint{1.452614in}{0.920087in}}%
\pgfpathlineto{\pgfqpoint{1.455906in}{0.945331in}}%
\pgfpathlineto{\pgfqpoint{1.457465in}{0.941016in}}%
\pgfpathlineto{\pgfqpoint{1.458331in}{0.937138in}}%
\pgfpathlineto{\pgfqpoint{1.459370in}{0.939349in}}%
\pgfpathlineto{\pgfqpoint{1.461102in}{0.940085in}}%
\pgfpathlineto{\pgfqpoint{1.461449in}{0.939360in}}%
\pgfpathlineto{\pgfqpoint{1.466818in}{0.920172in}}%
\pgfpathlineto{\pgfqpoint{1.470110in}{0.897729in}}%
\pgfpathlineto{\pgfqpoint{1.479810in}{0.812946in}}%
\pgfpathlineto{\pgfqpoint{1.486046in}{0.759306in}}%
\pgfpathlineto{\pgfqpoint{1.492628in}{0.717908in}}%
\pgfpathlineto{\pgfqpoint{1.502155in}{0.666892in}}%
\pgfpathlineto{\pgfqpoint{1.502848in}{0.668111in}}%
\pgfpathlineto{\pgfqpoint{1.507525in}{0.659157in}}%
\pgfpathlineto{\pgfqpoint{1.508218in}{0.662250in}}%
\pgfpathlineto{\pgfqpoint{1.509257in}{0.666654in}}%
\pgfpathlineto{\pgfqpoint{1.510470in}{0.665996in}}%
\pgfpathlineto{\pgfqpoint{1.511856in}{0.668205in}}%
\pgfpathlineto{\pgfqpoint{1.514454in}{0.677892in}}%
\pgfpathlineto{\pgfqpoint{1.514800in}{0.679098in}}%
\pgfpathlineto{\pgfqpoint{1.515840in}{0.677397in}}%
\pgfpathlineto{\pgfqpoint{1.516186in}{0.676230in}}%
\pgfpathlineto{\pgfqpoint{1.516879in}{0.677712in}}%
\pgfpathlineto{\pgfqpoint{1.521036in}{0.705096in}}%
\pgfpathlineto{\pgfqpoint{1.521383in}{0.704117in}}%
\pgfpathlineto{\pgfqpoint{1.525194in}{0.683907in}}%
\pgfpathlineto{\pgfqpoint{1.527965in}{0.659296in}}%
\pgfpathlineto{\pgfqpoint{1.528831in}{0.664254in}}%
\pgfpathlineto{\pgfqpoint{1.531083in}{0.670620in}}%
\pgfpathlineto{\pgfqpoint{1.531430in}{0.670093in}}%
\pgfpathlineto{\pgfqpoint{1.534894in}{0.657854in}}%
\pgfpathlineto{\pgfqpoint{1.535067in}{0.658327in}}%
\pgfpathlineto{\pgfqpoint{1.536799in}{0.663686in}}%
\pgfpathlineto{\pgfqpoint{1.537319in}{0.661544in}}%
\pgfpathlineto{\pgfqpoint{1.543209in}{0.643182in}}%
\pgfpathlineto{\pgfqpoint{1.543382in}{0.643698in}}%
\pgfpathlineto{\pgfqpoint{1.545287in}{0.650866in}}%
\pgfpathlineto{\pgfqpoint{1.545807in}{0.653299in}}%
\pgfpathlineto{\pgfqpoint{1.546673in}{0.650221in}}%
\pgfpathlineto{\pgfqpoint{1.548925in}{0.644074in}}%
\pgfpathlineto{\pgfqpoint{1.554988in}{0.613673in}}%
\pgfpathlineto{\pgfqpoint{1.555334in}{0.615506in}}%
\pgfpathlineto{\pgfqpoint{1.556547in}{0.619211in}}%
\pgfpathlineto{\pgfqpoint{1.557413in}{0.618885in}}%
\pgfpathlineto{\pgfqpoint{1.559145in}{0.622424in}}%
\pgfpathlineto{\pgfqpoint{1.559838in}{0.619857in}}%
\pgfpathlineto{\pgfqpoint{1.560184in}{0.618485in}}%
\pgfpathlineto{\pgfqpoint{1.560877in}{0.622090in}}%
\pgfpathlineto{\pgfqpoint{1.563822in}{0.630958in}}%
\pgfpathlineto{\pgfqpoint{1.565381in}{0.633858in}}%
\pgfpathlineto{\pgfqpoint{1.568672in}{0.642471in}}%
\pgfpathlineto{\pgfqpoint{1.569365in}{0.639095in}}%
\pgfpathlineto{\pgfqpoint{1.570231in}{0.642432in}}%
\pgfpathlineto{\pgfqpoint{1.570404in}{0.642153in}}%
\pgfpathlineto{\pgfqpoint{1.571097in}{0.643837in}}%
\pgfpathlineto{\pgfqpoint{1.571963in}{0.647050in}}%
\pgfpathlineto{\pgfqpoint{1.572829in}{0.644932in}}%
\pgfpathlineto{\pgfqpoint{1.573002in}{0.644834in}}%
\pgfpathlineto{\pgfqpoint{1.573349in}{0.645788in}}%
\pgfpathlineto{\pgfqpoint{1.573869in}{0.648011in}}%
\pgfpathlineto{\pgfqpoint{1.574561in}{0.645369in}}%
\pgfpathlineto{\pgfqpoint{1.574735in}{0.645468in}}%
\pgfpathlineto{\pgfqpoint{1.575601in}{0.647709in}}%
\pgfpathlineto{\pgfqpoint{1.578026in}{0.658735in}}%
\pgfpathlineto{\pgfqpoint{1.579065in}{0.654504in}}%
\pgfpathlineto{\pgfqpoint{1.579758in}{0.652283in}}%
\pgfpathlineto{\pgfqpoint{1.580797in}{0.653643in}}%
\pgfpathlineto{\pgfqpoint{1.581144in}{0.654424in}}%
\pgfpathlineto{\pgfqpoint{1.581663in}{0.651676in}}%
\pgfpathlineto{\pgfqpoint{1.584955in}{0.641513in}}%
\pgfpathlineto{\pgfqpoint{1.586687in}{0.638439in}}%
\pgfpathlineto{\pgfqpoint{1.589458in}{0.633558in}}%
\pgfpathlineto{\pgfqpoint{1.590151in}{0.634134in}}%
\pgfpathlineto{\pgfqpoint{1.590844in}{0.635244in}}%
\pgfpathlineto{\pgfqpoint{1.591537in}{0.633575in}}%
\pgfpathlineto{\pgfqpoint{1.592057in}{0.632696in}}%
\pgfpathlineto{\pgfqpoint{1.593616in}{0.624457in}}%
\pgfpathlineto{\pgfqpoint{1.594309in}{0.627645in}}%
\pgfpathlineto{\pgfqpoint{1.596734in}{0.635163in}}%
\pgfpathlineto{\pgfqpoint{1.597427in}{0.636309in}}%
\pgfpathlineto{\pgfqpoint{1.597773in}{0.634858in}}%
\pgfpathlineto{\pgfqpoint{1.600025in}{0.622677in}}%
\pgfpathlineto{\pgfqpoint{1.600718in}{0.623901in}}%
\pgfpathlineto{\pgfqpoint{1.602103in}{0.619533in}}%
\pgfpathlineto{\pgfqpoint{1.604355in}{0.604456in}}%
\pgfpathlineto{\pgfqpoint{1.605221in}{0.608406in}}%
\pgfpathlineto{\pgfqpoint{1.607127in}{0.612440in}}%
\pgfpathlineto{\pgfqpoint{1.607473in}{0.611203in}}%
\pgfpathlineto{\pgfqpoint{1.608339in}{0.613080in}}%
\pgfpathlineto{\pgfqpoint{1.608686in}{0.614883in}}%
\pgfpathlineto{\pgfqpoint{1.609552in}{0.611844in}}%
\pgfpathlineto{\pgfqpoint{1.610072in}{0.608865in}}%
\pgfpathlineto{\pgfqpoint{1.610418in}{0.614007in}}%
\pgfpathlineto{\pgfqpoint{1.610591in}{0.613823in}}%
\pgfpathlineto{\pgfqpoint{1.613363in}{0.621836in}}%
\pgfpathlineto{\pgfqpoint{1.613536in}{0.621790in}}%
\pgfpathlineto{\pgfqpoint{1.614056in}{0.622064in}}%
\pgfpathlineto{\pgfqpoint{1.614402in}{0.623072in}}%
\pgfpathlineto{\pgfqpoint{1.614749in}{0.623525in}}%
\pgfpathlineto{\pgfqpoint{1.614922in}{0.622767in}}%
\pgfpathlineto{\pgfqpoint{1.616308in}{0.612738in}}%
\pgfpathlineto{\pgfqpoint{1.617174in}{0.613769in}}%
\pgfpathlineto{\pgfqpoint{1.618213in}{0.608161in}}%
\pgfpathlineto{\pgfqpoint{1.622370in}{0.587374in}}%
\pgfpathlineto{\pgfqpoint{1.623063in}{0.590675in}}%
\pgfpathlineto{\pgfqpoint{1.623410in}{0.592104in}}%
\pgfpathlineto{\pgfqpoint{1.624276in}{0.590591in}}%
\pgfpathlineto{\pgfqpoint{1.627047in}{0.579477in}}%
\pgfpathlineto{\pgfqpoint{1.627913in}{0.577447in}}%
\pgfpathlineto{\pgfqpoint{1.628433in}{0.579647in}}%
\pgfpathlineto{\pgfqpoint{1.630685in}{0.593549in}}%
\pgfpathlineto{\pgfqpoint{1.633110in}{0.602249in}}%
\pgfpathlineto{\pgfqpoint{1.633456in}{0.604103in}}%
\pgfpathlineto{\pgfqpoint{1.634149in}{0.601039in}}%
\pgfpathlineto{\pgfqpoint{1.634322in}{0.600231in}}%
\pgfpathlineto{\pgfqpoint{1.635189in}{0.603364in}}%
\pgfpathlineto{\pgfqpoint{1.635881in}{0.605250in}}%
\pgfpathlineto{\pgfqpoint{1.636401in}{0.607875in}}%
\pgfpathlineto{\pgfqpoint{1.637440in}{0.605990in}}%
\pgfpathlineto{\pgfqpoint{1.637614in}{0.605581in}}%
\pgfpathlineto{\pgfqpoint{1.637960in}{0.607484in}}%
\pgfpathlineto{\pgfqpoint{1.638653in}{0.611009in}}%
\pgfpathlineto{\pgfqpoint{1.639519in}{0.608394in}}%
\pgfpathlineto{\pgfqpoint{1.642464in}{0.592415in}}%
\pgfpathlineto{\pgfqpoint{1.642637in}{0.592668in}}%
\pgfpathlineto{\pgfqpoint{1.644889in}{0.587563in}}%
\pgfpathlineto{\pgfqpoint{1.645582in}{0.588025in}}%
\pgfpathlineto{\pgfqpoint{1.645755in}{0.588902in}}%
\pgfpathlineto{\pgfqpoint{1.646448in}{0.584540in}}%
\pgfpathlineto{\pgfqpoint{1.646794in}{0.582191in}}%
\pgfpathlineto{\pgfqpoint{1.647834in}{0.585051in}}%
\pgfpathlineto{\pgfqpoint{1.649393in}{0.589905in}}%
\pgfpathlineto{\pgfqpoint{1.649912in}{0.586611in}}%
\pgfpathlineto{\pgfqpoint{1.652337in}{0.581116in}}%
\pgfpathlineto{\pgfqpoint{1.652857in}{0.583187in}}%
\pgfpathlineto{\pgfqpoint{1.654416in}{0.592368in}}%
\pgfpathlineto{\pgfqpoint{1.655455in}{0.591032in}}%
\pgfpathlineto{\pgfqpoint{1.658227in}{0.585462in}}%
\pgfpathlineto{\pgfqpoint{1.658400in}{0.584941in}}%
\pgfpathlineto{\pgfqpoint{1.659093in}{0.587304in}}%
\pgfpathlineto{\pgfqpoint{1.659613in}{0.585892in}}%
\pgfpathlineto{\pgfqpoint{1.661691in}{0.590161in}}%
\pgfpathlineto{\pgfqpoint{1.660479in}{0.585571in}}%
\pgfpathlineto{\pgfqpoint{1.662038in}{0.589340in}}%
\pgfpathlineto{\pgfqpoint{1.662384in}{0.588263in}}%
\pgfpathlineto{\pgfqpoint{1.662904in}{0.590797in}}%
\pgfpathlineto{\pgfqpoint{1.663423in}{0.590182in}}%
\pgfpathlineto{\pgfqpoint{1.665849in}{0.590556in}}%
\pgfpathlineto{\pgfqpoint{1.666541in}{0.596344in}}%
\pgfpathlineto{\pgfqpoint{1.667234in}{0.591775in}}%
\pgfpathlineto{\pgfqpoint{1.668967in}{0.580702in}}%
\pgfpathlineto{\pgfqpoint{1.669659in}{0.583967in}}%
\pgfpathlineto{\pgfqpoint{1.670179in}{0.586766in}}%
\pgfpathlineto{\pgfqpoint{1.670699in}{0.583226in}}%
\pgfpathlineto{\pgfqpoint{1.671565in}{0.576366in}}%
\pgfpathlineto{\pgfqpoint{1.672258in}{0.582673in}}%
\pgfpathlineto{\pgfqpoint{1.674683in}{0.594334in}}%
\pgfpathlineto{\pgfqpoint{1.674856in}{0.594246in}}%
\pgfpathlineto{\pgfqpoint{1.676761in}{0.590206in}}%
\pgfpathlineto{\pgfqpoint{1.676935in}{0.590375in}}%
\pgfpathlineto{\pgfqpoint{1.677801in}{0.594748in}}%
\pgfpathlineto{\pgfqpoint{1.680053in}{0.603413in}}%
\pgfpathlineto{\pgfqpoint{1.680226in}{0.603016in}}%
\pgfpathlineto{\pgfqpoint{1.680746in}{0.602304in}}%
\pgfpathlineto{\pgfqpoint{1.684730in}{0.584605in}}%
\pgfpathlineto{\pgfqpoint{1.685249in}{0.585603in}}%
\pgfpathlineto{\pgfqpoint{1.685769in}{0.587935in}}%
\pgfpathlineto{\pgfqpoint{1.688887in}{0.619231in}}%
\pgfpathlineto{\pgfqpoint{1.689060in}{0.618041in}}%
\pgfpathlineto{\pgfqpoint{1.690099in}{0.610690in}}%
\pgfpathlineto{\pgfqpoint{1.690966in}{0.615229in}}%
\pgfpathlineto{\pgfqpoint{1.692351in}{0.620890in}}%
\pgfpathlineto{\pgfqpoint{1.693044in}{0.618106in}}%
\pgfpathlineto{\pgfqpoint{1.693737in}{0.618686in}}%
\pgfpathlineto{\pgfqpoint{1.694950in}{0.610420in}}%
\pgfpathlineto{\pgfqpoint{1.698068in}{0.590736in}}%
\pgfpathlineto{\pgfqpoint{1.698414in}{0.592348in}}%
\pgfpathlineto{\pgfqpoint{1.700146in}{0.608158in}}%
\pgfpathlineto{\pgfqpoint{1.701186in}{0.603387in}}%
\pgfpathlineto{\pgfqpoint{1.703264in}{0.593674in}}%
\pgfpathlineto{\pgfqpoint{1.703957in}{0.596528in}}%
\pgfpathlineto{\pgfqpoint{1.704130in}{0.596784in}}%
\pgfpathlineto{\pgfqpoint{1.704477in}{0.594877in}}%
\pgfpathlineto{\pgfqpoint{1.707941in}{0.575384in}}%
\pgfpathlineto{\pgfqpoint{1.708114in}{0.575217in}}%
\pgfpathlineto{\pgfqpoint{1.708461in}{0.575932in}}%
\pgfpathlineto{\pgfqpoint{1.709673in}{0.574793in}}%
\pgfpathlineto{\pgfqpoint{1.710193in}{0.577160in}}%
\pgfpathlineto{\pgfqpoint{1.714350in}{0.589957in}}%
\pgfpathlineto{\pgfqpoint{1.714523in}{0.589882in}}%
\pgfpathlineto{\pgfqpoint{1.715216in}{0.586058in}}%
\pgfpathlineto{\pgfqpoint{1.715909in}{0.578606in}}%
\pgfpathlineto{\pgfqpoint{1.716949in}{0.582840in}}%
\pgfpathlineto{\pgfqpoint{1.718161in}{0.584485in}}%
\pgfpathlineto{\pgfqpoint{1.718508in}{0.582853in}}%
\pgfpathlineto{\pgfqpoint{1.720759in}{0.581078in}}%
\pgfpathlineto{\pgfqpoint{1.720933in}{0.579482in}}%
\pgfpathlineto{\pgfqpoint{1.721799in}{0.584439in}}%
\pgfpathlineto{\pgfqpoint{1.721972in}{0.584863in}}%
\pgfpathlineto{\pgfqpoint{1.722665in}{0.582998in}}%
\pgfpathlineto{\pgfqpoint{1.725263in}{0.574488in}}%
\pgfpathlineto{\pgfqpoint{1.725956in}{0.575121in}}%
\pgfpathlineto{\pgfqpoint{1.727169in}{0.581418in}}%
\pgfpathlineto{\pgfqpoint{1.728554in}{0.592083in}}%
\pgfpathlineto{\pgfqpoint{1.729767in}{0.591664in}}%
\pgfpathlineto{\pgfqpoint{1.730460in}{0.595804in}}%
\pgfpathlineto{\pgfqpoint{1.731153in}{0.591657in}}%
\pgfpathlineto{\pgfqpoint{1.732019in}{0.590172in}}%
\pgfpathlineto{\pgfqpoint{1.731499in}{0.592960in}}%
\pgfpathlineto{\pgfqpoint{1.732192in}{0.591059in}}%
\pgfpathlineto{\pgfqpoint{1.733058in}{0.597268in}}%
\pgfpathlineto{\pgfqpoint{1.733578in}{0.590470in}}%
\pgfpathlineto{\pgfqpoint{1.735137in}{0.580222in}}%
\pgfpathlineto{\pgfqpoint{1.735656in}{0.580839in}}%
\pgfpathlineto{\pgfqpoint{1.736003in}{0.580751in}}%
\pgfpathlineto{\pgfqpoint{1.736522in}{0.581368in}}%
\pgfpathlineto{\pgfqpoint{1.741026in}{0.599334in}}%
\pgfpathlineto{\pgfqpoint{1.741892in}{0.598784in}}%
\pgfpathlineto{\pgfqpoint{1.742239in}{0.598263in}}%
\pgfpathlineto{\pgfqpoint{1.742758in}{0.599788in}}%
\pgfpathlineto{\pgfqpoint{1.743278in}{0.599263in}}%
\pgfpathlineto{\pgfqpoint{1.744664in}{0.603526in}}%
\pgfpathlineto{\pgfqpoint{1.745703in}{0.600698in}}%
\pgfpathlineto{\pgfqpoint{1.746569in}{0.599059in}}%
\pgfpathlineto{\pgfqpoint{1.747089in}{0.600173in}}%
\pgfpathlineto{\pgfqpoint{1.749514in}{0.605106in}}%
\pgfpathlineto{\pgfqpoint{1.751939in}{0.607745in}}%
\pgfpathlineto{\pgfqpoint{1.752112in}{0.607355in}}%
\pgfpathlineto{\pgfqpoint{1.755577in}{0.592800in}}%
\pgfpathlineto{\pgfqpoint{1.755923in}{0.594613in}}%
\pgfpathlineto{\pgfqpoint{1.756270in}{0.596949in}}%
\pgfpathlineto{\pgfqpoint{1.757136in}{0.593773in}}%
\pgfpathlineto{\pgfqpoint{1.757655in}{0.591517in}}%
\pgfpathlineto{\pgfqpoint{1.758175in}{0.595159in}}%
\pgfpathlineto{\pgfqpoint{1.760080in}{0.600297in}}%
\pgfpathlineto{\pgfqpoint{1.760427in}{0.599649in}}%
\pgfpathlineto{\pgfqpoint{1.760600in}{0.599282in}}%
\pgfpathlineto{\pgfqpoint{1.760947in}{0.601601in}}%
\pgfpathlineto{\pgfqpoint{1.763025in}{0.610348in}}%
\pgfpathlineto{\pgfqpoint{1.763198in}{0.608615in}}%
\pgfpathlineto{\pgfqpoint{1.765797in}{0.591607in}}%
\pgfpathlineto{\pgfqpoint{1.765970in}{0.591730in}}%
\pgfpathlineto{\pgfqpoint{1.767529in}{0.581323in}}%
\pgfpathlineto{\pgfqpoint{1.769261in}{0.583759in}}%
\pgfpathlineto{\pgfqpoint{1.769434in}{0.583874in}}%
\pgfpathlineto{\pgfqpoint{1.769608in}{0.582798in}}%
\pgfpathlineto{\pgfqpoint{1.772033in}{0.575243in}}%
\pgfpathlineto{\pgfqpoint{1.772726in}{0.577093in}}%
\pgfpathlineto{\pgfqpoint{1.773765in}{0.576141in}}%
\pgfpathlineto{\pgfqpoint{1.775151in}{0.577158in}}%
\pgfpathlineto{\pgfqpoint{1.775670in}{0.573928in}}%
\pgfpathlineto{\pgfqpoint{1.776190in}{0.579961in}}%
\pgfpathlineto{\pgfqpoint{1.778788in}{0.594415in}}%
\pgfpathlineto{\pgfqpoint{1.778961in}{0.594377in}}%
\pgfpathlineto{\pgfqpoint{1.781040in}{0.581095in}}%
\pgfpathlineto{\pgfqpoint{1.781560in}{0.584747in}}%
\pgfpathlineto{\pgfqpoint{1.782946in}{0.590196in}}%
\pgfpathlineto{\pgfqpoint{1.783638in}{0.590115in}}%
\pgfpathlineto{\pgfqpoint{1.783812in}{0.590667in}}%
\pgfpathlineto{\pgfqpoint{1.784331in}{0.588371in}}%
\pgfpathlineto{\pgfqpoint{1.786063in}{0.580581in}}%
\pgfpathlineto{\pgfqpoint{1.786410in}{0.582038in}}%
\pgfpathlineto{\pgfqpoint{1.786583in}{0.582741in}}%
\pgfpathlineto{\pgfqpoint{1.787103in}{0.578967in}}%
\pgfpathlineto{\pgfqpoint{1.787276in}{0.578199in}}%
\pgfpathlineto{\pgfqpoint{1.788315in}{0.580121in}}%
\pgfpathlineto{\pgfqpoint{1.789181in}{0.582049in}}%
\pgfpathlineto{\pgfqpoint{1.789874in}{0.580284in}}%
\pgfpathlineto{\pgfqpoint{1.790567in}{0.578256in}}%
\pgfpathlineto{\pgfqpoint{1.791260in}{0.580655in}}%
\pgfpathlineto{\pgfqpoint{1.792992in}{0.582711in}}%
\pgfpathlineto{\pgfqpoint{1.795417in}{0.580846in}}%
\pgfpathlineto{\pgfqpoint{1.797496in}{0.569478in}}%
\pgfpathlineto{\pgfqpoint{1.797842in}{0.573524in}}%
\pgfpathlineto{\pgfqpoint{1.799748in}{0.584521in}}%
\pgfpathlineto{\pgfqpoint{1.800094in}{0.583722in}}%
\pgfpathlineto{\pgfqpoint{1.801653in}{0.575345in}}%
\pgfpathlineto{\pgfqpoint{1.802000in}{0.579172in}}%
\pgfpathlineto{\pgfqpoint{1.803212in}{0.586695in}}%
\pgfpathlineto{\pgfqpoint{1.803905in}{0.584783in}}%
\pgfpathlineto{\pgfqpoint{1.804771in}{0.578716in}}%
\pgfpathlineto{\pgfqpoint{1.805984in}{0.581208in}}%
\pgfpathlineto{\pgfqpoint{1.808236in}{0.588882in}}%
\pgfpathlineto{\pgfqpoint{1.808409in}{0.588842in}}%
\pgfpathlineto{\pgfqpoint{1.808755in}{0.589664in}}%
\pgfpathlineto{\pgfqpoint{1.808929in}{0.588234in}}%
\pgfpathlineto{\pgfqpoint{1.809621in}{0.585147in}}%
\pgfpathlineto{\pgfqpoint{1.810314in}{0.587627in}}%
\pgfpathlineto{\pgfqpoint{1.811180in}{0.588693in}}%
\pgfpathlineto{\pgfqpoint{1.811527in}{0.586762in}}%
\pgfpathlineto{\pgfqpoint{1.812047in}{0.583430in}}%
\pgfpathlineto{\pgfqpoint{1.812913in}{0.587940in}}%
\pgfpathlineto{\pgfqpoint{1.813432in}{0.589891in}}%
\pgfpathlineto{\pgfqpoint{1.814472in}{0.587916in}}%
\pgfpathlineto{\pgfqpoint{1.814818in}{0.587547in}}%
\pgfpathlineto{\pgfqpoint{1.815338in}{0.588934in}}%
\pgfpathlineto{\pgfqpoint{1.815511in}{0.588854in}}%
\pgfpathlineto{\pgfqpoint{1.817936in}{0.593038in}}%
\pgfpathlineto{\pgfqpoint{1.818109in}{0.592497in}}%
\pgfpathlineto{\pgfqpoint{1.818802in}{0.594481in}}%
\pgfpathlineto{\pgfqpoint{1.821920in}{0.603167in}}%
\pgfpathlineto{\pgfqpoint{1.822093in}{0.603009in}}%
\pgfpathlineto{\pgfqpoint{1.822267in}{0.603832in}}%
\pgfpathlineto{\pgfqpoint{1.824692in}{0.613033in}}%
\pgfpathlineto{\pgfqpoint{1.824865in}{0.612970in}}%
\pgfpathlineto{\pgfqpoint{1.825558in}{0.611641in}}%
\pgfpathlineto{\pgfqpoint{1.826077in}{0.609726in}}%
\pgfpathlineto{\pgfqpoint{1.826943in}{0.611418in}}%
\pgfpathlineto{\pgfqpoint{1.829195in}{0.618806in}}%
\pgfpathlineto{\pgfqpoint{1.830061in}{0.615839in}}%
\pgfpathlineto{\pgfqpoint{1.839242in}{0.586973in}}%
\pgfpathlineto{\pgfqpoint{1.839762in}{0.587828in}}%
\pgfpathlineto{\pgfqpoint{1.840455in}{0.593683in}}%
\pgfpathlineto{\pgfqpoint{1.841494in}{0.589383in}}%
\pgfpathlineto{\pgfqpoint{1.842533in}{0.589407in}}%
\pgfpathlineto{\pgfqpoint{1.842707in}{0.590292in}}%
\pgfpathlineto{\pgfqpoint{1.843053in}{0.592653in}}%
\pgfpathlineto{\pgfqpoint{1.843399in}{0.588403in}}%
\pgfpathlineto{\pgfqpoint{1.845998in}{0.566607in}}%
\pgfpathlineto{\pgfqpoint{1.846171in}{0.566750in}}%
\pgfpathlineto{\pgfqpoint{1.847557in}{0.574485in}}%
\pgfpathlineto{\pgfqpoint{1.848250in}{0.578118in}}%
\pgfpathlineto{\pgfqpoint{1.848942in}{0.575111in}}%
\pgfpathlineto{\pgfqpoint{1.849116in}{0.574442in}}%
\pgfpathlineto{\pgfqpoint{1.849462in}{0.578512in}}%
\pgfpathlineto{\pgfqpoint{1.852060in}{0.590376in}}%
\pgfpathlineto{\pgfqpoint{1.852580in}{0.588469in}}%
\pgfpathlineto{\pgfqpoint{1.857084in}{0.552347in}}%
\pgfpathlineto{\pgfqpoint{1.857777in}{0.556620in}}%
\pgfpathlineto{\pgfqpoint{1.858470in}{0.559633in}}%
\pgfpathlineto{\pgfqpoint{1.859682in}{0.559000in}}%
\pgfpathlineto{\pgfqpoint{1.861761in}{0.560744in}}%
\pgfpathlineto{\pgfqpoint{1.861934in}{0.558992in}}%
\pgfpathlineto{\pgfqpoint{1.862280in}{0.556700in}}%
\pgfpathlineto{\pgfqpoint{1.862973in}{0.562325in}}%
\pgfpathlineto{\pgfqpoint{1.863839in}{0.561129in}}%
\pgfpathlineto{\pgfqpoint{1.864879in}{0.567559in}}%
\pgfpathlineto{\pgfqpoint{1.865918in}{0.573600in}}%
\pgfpathlineto{\pgfqpoint{1.866611in}{0.570061in}}%
\pgfpathlineto{\pgfqpoint{1.866784in}{0.569228in}}%
\pgfpathlineto{\pgfqpoint{1.867304in}{0.572028in}}%
\pgfpathlineto{\pgfqpoint{1.869382in}{0.580543in}}%
\pgfpathlineto{\pgfqpoint{1.871115in}{0.584207in}}%
\pgfpathlineto{\pgfqpoint{1.871634in}{0.586154in}}%
\pgfpathlineto{\pgfqpoint{1.872154in}{0.582717in}}%
\pgfpathlineto{\pgfqpoint{1.872327in}{0.582921in}}%
\pgfpathlineto{\pgfqpoint{1.874752in}{0.574863in}}%
\pgfpathlineto{\pgfqpoint{1.875099in}{0.573829in}}%
\pgfpathlineto{\pgfqpoint{1.876138in}{0.570872in}}%
\pgfpathlineto{\pgfqpoint{1.876831in}{0.572434in}}%
\pgfpathlineto{\pgfqpoint{1.877351in}{0.576618in}}%
\pgfpathlineto{\pgfqpoint{1.878217in}{0.572068in}}%
\pgfpathlineto{\pgfqpoint{1.878390in}{0.572499in}}%
\pgfpathlineto{\pgfqpoint{1.879083in}{0.573436in}}%
\pgfpathlineto{\pgfqpoint{1.881681in}{0.584684in}}%
\pgfpathlineto{\pgfqpoint{1.882201in}{0.583599in}}%
\pgfpathlineto{\pgfqpoint{1.882547in}{0.581971in}}%
\pgfpathlineto{\pgfqpoint{1.885665in}{0.564379in}}%
\pgfpathlineto{\pgfqpoint{1.886185in}{0.566343in}}%
\pgfpathlineto{\pgfqpoint{1.886878in}{0.562765in}}%
\pgfpathlineto{\pgfqpoint{1.888090in}{0.557694in}}%
\pgfpathlineto{\pgfqpoint{1.888610in}{0.560260in}}%
\pgfpathlineto{\pgfqpoint{1.889476in}{0.567667in}}%
\pgfpathlineto{\pgfqpoint{1.890689in}{0.567103in}}%
\pgfpathlineto{\pgfqpoint{1.892940in}{0.560840in}}%
\pgfpathlineto{\pgfqpoint{1.893287in}{0.562733in}}%
\pgfpathlineto{\pgfqpoint{1.895019in}{0.565510in}}%
\pgfpathlineto{\pgfqpoint{1.894499in}{0.561788in}}%
\pgfpathlineto{\pgfqpoint{1.895192in}{0.565400in}}%
\pgfpathlineto{\pgfqpoint{1.895885in}{0.570961in}}%
\pgfpathlineto{\pgfqpoint{1.897964in}{0.578798in}}%
\pgfpathlineto{\pgfqpoint{1.898137in}{0.578589in}}%
\pgfpathlineto{\pgfqpoint{1.901082in}{0.566278in}}%
\pgfpathlineto{\pgfqpoint{1.902121in}{0.568646in}}%
\pgfpathlineto{\pgfqpoint{1.902294in}{0.569706in}}%
\pgfpathlineto{\pgfqpoint{1.902987in}{0.565171in}}%
\pgfpathlineto{\pgfqpoint{1.905759in}{0.558842in}}%
\pgfpathlineto{\pgfqpoint{1.905932in}{0.559577in}}%
\pgfpathlineto{\pgfqpoint{1.906625in}{0.558960in}}%
\pgfpathlineto{\pgfqpoint{1.908357in}{0.571396in}}%
\pgfpathlineto{\pgfqpoint{1.911302in}{0.586531in}}%
\pgfpathlineto{\pgfqpoint{1.912168in}{0.589724in}}%
\pgfpathlineto{\pgfqpoint{1.912688in}{0.593177in}}%
\pgfpathlineto{\pgfqpoint{1.913554in}{0.588636in}}%
\pgfpathlineto{\pgfqpoint{1.915286in}{0.586261in}}%
\pgfpathlineto{\pgfqpoint{1.916152in}{0.592066in}}%
\pgfpathlineto{\pgfqpoint{1.918404in}{0.589892in}}%
\pgfpathlineto{\pgfqpoint{1.918923in}{0.591041in}}%
\pgfpathlineto{\pgfqpoint{1.919790in}{0.595174in}}%
\pgfpathlineto{\pgfqpoint{1.920482in}{0.592631in}}%
\pgfpathlineto{\pgfqpoint{1.922041in}{0.588969in}}%
\pgfpathlineto{\pgfqpoint{1.924813in}{0.612513in}}%
\pgfpathlineto{\pgfqpoint{1.926025in}{0.607076in}}%
\pgfpathlineto{\pgfqpoint{1.935033in}{0.570782in}}%
\pgfpathlineto{\pgfqpoint{1.935553in}{0.572759in}}%
\pgfpathlineto{\pgfqpoint{1.938844in}{0.584660in}}%
\pgfpathlineto{\pgfqpoint{1.943867in}{0.559307in}}%
\pgfpathlineto{\pgfqpoint{1.945253in}{0.559956in}}%
\pgfpathlineto{\pgfqpoint{1.945946in}{0.562538in}}%
\pgfpathlineto{\pgfqpoint{1.946639in}{0.560029in}}%
\pgfpathlineto{\pgfqpoint{1.947851in}{0.550835in}}%
\pgfpathlineto{\pgfqpoint{1.948371in}{0.544397in}}%
\pgfpathlineto{\pgfqpoint{1.949583in}{0.546202in}}%
\pgfpathlineto{\pgfqpoint{1.954087in}{0.571911in}}%
\pgfpathlineto{\pgfqpoint{1.954953in}{0.569301in}}%
\pgfpathlineto{\pgfqpoint{1.955473in}{0.571755in}}%
\pgfpathlineto{\pgfqpoint{1.958244in}{0.588198in}}%
\pgfpathlineto{\pgfqpoint{1.961536in}{0.582383in}}%
\pgfpathlineto{\pgfqpoint{1.961709in}{0.583174in}}%
\pgfpathlineto{\pgfqpoint{1.965693in}{0.605927in}}%
\pgfpathlineto{\pgfqpoint{1.966213in}{0.605117in}}%
\pgfpathlineto{\pgfqpoint{1.966732in}{0.606820in}}%
\pgfpathlineto{\pgfqpoint{1.967425in}{0.609376in}}%
\pgfpathlineto{\pgfqpoint{1.968291in}{0.616344in}}%
\pgfpathlineto{\pgfqpoint{1.968984in}{0.610351in}}%
\pgfpathlineto{\pgfqpoint{1.970890in}{0.604457in}}%
\pgfpathlineto{\pgfqpoint{1.971236in}{0.605068in}}%
\pgfpathlineto{\pgfqpoint{1.971756in}{0.603452in}}%
\pgfpathlineto{\pgfqpoint{1.972968in}{0.600568in}}%
\pgfpathlineto{\pgfqpoint{1.973488in}{0.601600in}}%
\pgfpathlineto{\pgfqpoint{1.974008in}{0.604083in}}%
\pgfpathlineto{\pgfqpoint{1.974874in}{0.601460in}}%
\pgfpathlineto{\pgfqpoint{1.979031in}{0.574899in}}%
\pgfpathlineto{\pgfqpoint{1.979897in}{0.566211in}}%
\pgfpathlineto{\pgfqpoint{1.981283in}{0.568192in}}%
\pgfpathlineto{\pgfqpoint{1.982495in}{0.569855in}}%
\pgfpathlineto{\pgfqpoint{1.985786in}{0.580581in}}%
\pgfpathlineto{\pgfqpoint{1.986306in}{0.578624in}}%
\pgfpathlineto{\pgfqpoint{1.987692in}{0.575301in}}%
\pgfpathlineto{\pgfqpoint{1.988385in}{0.576626in}}%
\pgfpathlineto{\pgfqpoint{1.989771in}{0.580441in}}%
\pgfpathlineto{\pgfqpoint{1.990290in}{0.578172in}}%
\pgfpathlineto{\pgfqpoint{1.991156in}{0.576797in}}%
\pgfpathlineto{\pgfqpoint{1.991503in}{0.578457in}}%
\pgfpathlineto{\pgfqpoint{1.992022in}{0.577344in}}%
\pgfpathlineto{\pgfqpoint{1.992889in}{0.579190in}}%
\pgfpathlineto{\pgfqpoint{1.994274in}{0.586159in}}%
\pgfpathlineto{\pgfqpoint{1.994794in}{0.584250in}}%
\pgfpathlineto{\pgfqpoint{1.996353in}{0.581013in}}%
\pgfpathlineto{\pgfqpoint{1.996526in}{0.581060in}}%
\pgfpathlineto{\pgfqpoint{1.997565in}{0.577016in}}%
\pgfpathlineto{\pgfqpoint{1.998258in}{0.579610in}}%
\pgfpathlineto{\pgfqpoint{2.000164in}{0.586916in}}%
\pgfpathlineto{\pgfqpoint{2.000683in}{0.585036in}}%
\pgfpathlineto{\pgfqpoint{2.005880in}{0.577041in}}%
\pgfpathlineto{\pgfqpoint{2.006226in}{0.577679in}}%
\pgfpathlineto{\pgfqpoint{2.007093in}{0.577232in}}%
\pgfpathlineto{\pgfqpoint{2.007266in}{0.576471in}}%
\pgfpathlineto{\pgfqpoint{2.008305in}{0.574260in}}%
\pgfpathlineto{\pgfqpoint{2.009171in}{0.575744in}}%
\pgfpathlineto{\pgfqpoint{2.010557in}{0.577390in}}%
\pgfpathlineto{\pgfqpoint{2.009864in}{0.574580in}}%
\pgfpathlineto{\pgfqpoint{2.010730in}{0.576990in}}%
\pgfpathlineto{\pgfqpoint{2.011077in}{0.575235in}}%
\pgfpathlineto{\pgfqpoint{2.011770in}{0.579062in}}%
\pgfpathlineto{\pgfqpoint{2.011943in}{0.579898in}}%
\pgfpathlineto{\pgfqpoint{2.012982in}{0.578408in}}%
\pgfpathlineto{\pgfqpoint{2.013329in}{0.577045in}}%
\pgfpathlineto{\pgfqpoint{2.014195in}{0.579378in}}%
\pgfpathlineto{\pgfqpoint{2.014714in}{0.580734in}}%
\pgfpathlineto{\pgfqpoint{2.014887in}{0.579289in}}%
\pgfpathlineto{\pgfqpoint{2.016446in}{0.569298in}}%
\pgfpathlineto{\pgfqpoint{2.016966in}{0.569776in}}%
\pgfpathlineto{\pgfqpoint{2.021297in}{0.538471in}}%
\pgfpathlineto{\pgfqpoint{2.022163in}{0.548284in}}%
\pgfpathlineto{\pgfqpoint{2.022682in}{0.550922in}}%
\pgfpathlineto{\pgfqpoint{2.023375in}{0.546546in}}%
\pgfpathlineto{\pgfqpoint{2.025281in}{0.541632in}}%
\pgfpathlineto{\pgfqpoint{2.026320in}{0.547144in}}%
\pgfpathlineto{\pgfqpoint{2.027013in}{0.543997in}}%
\pgfpathlineto{\pgfqpoint{2.027706in}{0.542114in}}%
\pgfpathlineto{\pgfqpoint{2.028225in}{0.544518in}}%
\pgfpathlineto{\pgfqpoint{2.029438in}{0.557537in}}%
\pgfpathlineto{\pgfqpoint{2.030131in}{0.566113in}}%
\pgfpathlineto{\pgfqpoint{2.030997in}{0.559175in}}%
\pgfpathlineto{\pgfqpoint{2.033942in}{0.541438in}}%
\pgfpathlineto{\pgfqpoint{2.035327in}{0.548210in}}%
\pgfpathlineto{\pgfqpoint{2.035847in}{0.545380in}}%
\pgfpathlineto{\pgfqpoint{2.037579in}{0.542234in}}%
\pgfpathlineto{\pgfqpoint{2.037753in}{0.542277in}}%
\pgfpathlineto{\pgfqpoint{2.038619in}{0.548879in}}%
\pgfpathlineto{\pgfqpoint{2.039658in}{0.545223in}}%
\pgfpathlineto{\pgfqpoint{2.039831in}{0.544863in}}%
\pgfpathlineto{\pgfqpoint{2.040004in}{0.546408in}}%
\pgfpathlineto{\pgfqpoint{2.040871in}{0.551801in}}%
\pgfpathlineto{\pgfqpoint{2.041737in}{0.549616in}}%
\pgfpathlineto{\pgfqpoint{2.043122in}{0.544410in}}%
\pgfpathlineto{\pgfqpoint{2.043815in}{0.547132in}}%
\pgfpathlineto{\pgfqpoint{2.045894in}{0.551876in}}%
\pgfpathlineto{\pgfqpoint{2.047626in}{0.561659in}}%
\pgfpathlineto{\pgfqpoint{2.048839in}{0.559555in}}%
\pgfpathlineto{\pgfqpoint{2.050224in}{0.553396in}}%
\pgfpathlineto{\pgfqpoint{2.050571in}{0.555924in}}%
\pgfpathlineto{\pgfqpoint{2.052130in}{0.563090in}}%
\pgfpathlineto{\pgfqpoint{2.052650in}{0.562664in}}%
\pgfpathlineto{\pgfqpoint{2.052996in}{0.560804in}}%
\pgfpathlineto{\pgfqpoint{2.053689in}{0.564863in}}%
\pgfpathlineto{\pgfqpoint{2.055075in}{0.566157in}}%
\pgfpathlineto{\pgfqpoint{2.058539in}{0.575687in}}%
\pgfpathlineto{\pgfqpoint{2.059059in}{0.572435in}}%
\pgfpathlineto{\pgfqpoint{2.060964in}{0.570889in}}%
\pgfpathlineto{\pgfqpoint{2.062696in}{0.566442in}}%
\pgfpathlineto{\pgfqpoint{2.063043in}{0.568739in}}%
\pgfpathlineto{\pgfqpoint{2.063562in}{0.573336in}}%
\pgfpathlineto{\pgfqpoint{2.064429in}{0.567572in}}%
\pgfpathlineto{\pgfqpoint{2.066680in}{0.548569in}}%
\pgfpathlineto{\pgfqpoint{2.067546in}{0.552488in}}%
\pgfpathlineto{\pgfqpoint{2.067893in}{0.553205in}}%
\pgfpathlineto{\pgfqpoint{2.068239in}{0.552240in}}%
\pgfpathlineto{\pgfqpoint{2.070145in}{0.556387in}}%
\pgfpathlineto{\pgfqpoint{2.070491in}{0.558808in}}%
\pgfpathlineto{\pgfqpoint{2.071531in}{0.556768in}}%
\pgfpathlineto{\pgfqpoint{2.073263in}{0.544695in}}%
\pgfpathlineto{\pgfqpoint{2.074649in}{0.540876in}}%
\pgfpathlineto{\pgfqpoint{2.075168in}{0.541772in}}%
\pgfpathlineto{\pgfqpoint{2.075688in}{0.546240in}}%
\pgfpathlineto{\pgfqpoint{2.078459in}{0.569391in}}%
\pgfpathlineto{\pgfqpoint{2.078806in}{0.566961in}}%
\pgfpathlineto{\pgfqpoint{2.080538in}{0.558248in}}%
\pgfpathlineto{\pgfqpoint{2.081058in}{0.560660in}}%
\pgfpathlineto{\pgfqpoint{2.083310in}{0.566373in}}%
\pgfpathlineto{\pgfqpoint{2.084002in}{0.565430in}}%
\pgfpathlineto{\pgfqpoint{2.084349in}{0.565948in}}%
\pgfpathlineto{\pgfqpoint{2.087986in}{0.555680in}}%
\pgfpathlineto{\pgfqpoint{2.088333in}{0.557523in}}%
\pgfpathlineto{\pgfqpoint{2.088506in}{0.557890in}}%
\pgfpathlineto{\pgfqpoint{2.089199in}{0.556275in}}%
\pgfpathlineto{\pgfqpoint{2.089892in}{0.558177in}}%
\pgfpathlineto{\pgfqpoint{2.090758in}{0.555184in}}%
\pgfpathlineto{\pgfqpoint{2.094742in}{0.570521in}}%
\pgfpathlineto{\pgfqpoint{2.096128in}{0.563916in}}%
\pgfpathlineto{\pgfqpoint{2.097687in}{0.560692in}}%
\pgfpathlineto{\pgfqpoint{2.097860in}{0.562275in}}%
\pgfpathlineto{\pgfqpoint{2.099419in}{0.570139in}}%
\pgfpathlineto{\pgfqpoint{2.099939in}{0.566488in}}%
\pgfpathlineto{\pgfqpoint{2.101671in}{0.573689in}}%
\pgfpathlineto{\pgfqpoint{2.103403in}{0.580989in}}%
\pgfpathlineto{\pgfqpoint{2.103750in}{0.579685in}}%
\pgfpathlineto{\pgfqpoint{2.105828in}{0.573148in}}%
\pgfpathlineto{\pgfqpoint{2.106001in}{0.571929in}}%
\pgfpathlineto{\pgfqpoint{2.107387in}{0.572696in}}%
\pgfpathlineto{\pgfqpoint{2.108946in}{0.582485in}}%
\pgfpathlineto{\pgfqpoint{2.109985in}{0.577072in}}%
\pgfpathlineto{\pgfqpoint{2.112064in}{0.563017in}}%
\pgfpathlineto{\pgfqpoint{2.112411in}{0.565070in}}%
\pgfpathlineto{\pgfqpoint{2.112584in}{0.565829in}}%
\pgfpathlineto{\pgfqpoint{2.113103in}{0.563075in}}%
\pgfpathlineto{\pgfqpoint{2.113450in}{0.559647in}}%
\pgfpathlineto{\pgfqpoint{2.114489in}{0.562567in}}%
\pgfpathlineto{\pgfqpoint{2.115702in}{0.570401in}}%
\pgfpathlineto{\pgfqpoint{2.116221in}{0.568729in}}%
\pgfpathlineto{\pgfqpoint{2.117261in}{0.565179in}}%
\pgfpathlineto{\pgfqpoint{2.117780in}{0.568343in}}%
\pgfpathlineto{\pgfqpoint{2.125402in}{0.619460in}}%
\pgfpathlineto{\pgfqpoint{2.125922in}{0.619290in}}%
\pgfpathlineto{\pgfqpoint{2.127134in}{0.620412in}}%
\pgfpathlineto{\pgfqpoint{2.127481in}{0.619294in}}%
\pgfpathlineto{\pgfqpoint{2.130945in}{0.607331in}}%
\pgfpathlineto{\pgfqpoint{2.131118in}{0.607813in}}%
\pgfpathlineto{\pgfqpoint{2.132851in}{0.612290in}}%
\pgfpathlineto{\pgfqpoint{2.133543in}{0.610475in}}%
\pgfpathlineto{\pgfqpoint{2.134756in}{0.609492in}}%
\pgfpathlineto{\pgfqpoint{2.134929in}{0.608669in}}%
\pgfpathlineto{\pgfqpoint{2.135449in}{0.613099in}}%
\pgfpathlineto{\pgfqpoint{2.136142in}{0.614384in}}%
\pgfpathlineto{\pgfqpoint{2.138047in}{0.624201in}}%
\pgfpathlineto{\pgfqpoint{2.138567in}{0.622861in}}%
\pgfpathlineto{\pgfqpoint{2.141338in}{0.605181in}}%
\pgfpathlineto{\pgfqpoint{2.145149in}{0.610395in}}%
\pgfpathlineto{\pgfqpoint{2.147401in}{0.599275in}}%
\pgfpathlineto{\pgfqpoint{2.147921in}{0.601943in}}%
\pgfpathlineto{\pgfqpoint{2.148440in}{0.606981in}}%
\pgfpathlineto{\pgfqpoint{2.149480in}{0.602819in}}%
\pgfpathlineto{\pgfqpoint{2.151558in}{0.592835in}}%
\pgfpathlineto{\pgfqpoint{2.153464in}{0.578725in}}%
\pgfpathlineto{\pgfqpoint{2.154330in}{0.581313in}}%
\pgfpathlineto{\pgfqpoint{2.156408in}{0.589121in}}%
\pgfpathlineto{\pgfqpoint{2.157101in}{0.588272in}}%
\pgfpathlineto{\pgfqpoint{2.157967in}{0.585845in}}%
\pgfpathlineto{\pgfqpoint{2.158660in}{0.587653in}}%
\pgfpathlineto{\pgfqpoint{2.159353in}{0.589055in}}%
\pgfpathlineto{\pgfqpoint{2.161952in}{0.606097in}}%
\pgfpathlineto{\pgfqpoint{2.162818in}{0.603899in}}%
\pgfpathlineto{\pgfqpoint{2.163684in}{0.602412in}}%
\pgfpathlineto{\pgfqpoint{2.163857in}{0.604307in}}%
\pgfpathlineto{\pgfqpoint{2.165070in}{0.616080in}}%
\pgfpathlineto{\pgfqpoint{2.165762in}{0.611106in}}%
\pgfpathlineto{\pgfqpoint{2.169920in}{0.595878in}}%
\pgfpathlineto{\pgfqpoint{2.171305in}{0.593015in}}%
\pgfpathlineto{\pgfqpoint{2.171998in}{0.595556in}}%
\pgfpathlineto{\pgfqpoint{2.173211in}{0.599898in}}%
\pgfpathlineto{\pgfqpoint{2.173904in}{0.598783in}}%
\pgfpathlineto{\pgfqpoint{2.174770in}{0.601684in}}%
\pgfpathlineto{\pgfqpoint{2.174250in}{0.598610in}}%
\pgfpathlineto{\pgfqpoint{2.175809in}{0.599445in}}%
\pgfpathlineto{\pgfqpoint{2.176156in}{0.597738in}}%
\pgfpathlineto{\pgfqpoint{2.177022in}{0.601200in}}%
\pgfpathlineto{\pgfqpoint{2.177195in}{0.601459in}}%
\pgfpathlineto{\pgfqpoint{2.177888in}{0.600052in}}%
\pgfpathlineto{\pgfqpoint{2.179447in}{0.599003in}}%
\pgfpathlineto{\pgfqpoint{2.179620in}{0.599176in}}%
\pgfpathlineto{\pgfqpoint{2.180313in}{0.598007in}}%
\pgfpathlineto{\pgfqpoint{2.180659in}{0.597323in}}%
\pgfpathlineto{\pgfqpoint{2.181525in}{0.598714in}}%
\pgfpathlineto{\pgfqpoint{2.182218in}{0.599305in}}%
\pgfpathlineto{\pgfqpoint{2.183084in}{0.598714in}}%
\pgfpathlineto{\pgfqpoint{2.184124in}{0.596906in}}%
\pgfpathlineto{\pgfqpoint{2.186549in}{0.591680in}}%
\pgfpathlineto{\pgfqpoint{2.187415in}{0.591008in}}%
\pgfpathlineto{\pgfqpoint{2.187761in}{0.592453in}}%
\pgfpathlineto{\pgfqpoint{2.189320in}{0.595695in}}%
\pgfpathlineto{\pgfqpoint{2.189840in}{0.594822in}}%
\pgfpathlineto{\pgfqpoint{2.191226in}{0.591341in}}%
\pgfpathlineto{\pgfqpoint{2.191745in}{0.592642in}}%
\pgfpathlineto{\pgfqpoint{2.196249in}{0.600278in}}%
\pgfpathlineto{\pgfqpoint{2.196942in}{0.598202in}}%
\pgfpathlineto{\pgfqpoint{2.200406in}{0.594570in}}%
\pgfpathlineto{\pgfqpoint{2.197462in}{0.599084in}}%
\pgfpathlineto{\pgfqpoint{2.200753in}{0.596428in}}%
\pgfpathlineto{\pgfqpoint{2.201965in}{0.601249in}}%
\pgfpathlineto{\pgfqpoint{2.202658in}{0.600766in}}%
\pgfpathlineto{\pgfqpoint{2.205083in}{0.604478in}}%
\pgfpathlineto{\pgfqpoint{2.205257in}{0.604211in}}%
\pgfpathlineto{\pgfqpoint{2.206469in}{0.599741in}}%
\pgfpathlineto{\pgfqpoint{2.207508in}{0.602196in}}%
\pgfpathlineto{\pgfqpoint{2.207682in}{0.602554in}}%
\pgfpathlineto{\pgfqpoint{2.208201in}{0.600939in}}%
\pgfpathlineto{\pgfqpoint{2.209414in}{0.597427in}}%
\pgfpathlineto{\pgfqpoint{2.209934in}{0.597809in}}%
\pgfpathlineto{\pgfqpoint{2.210626in}{0.598035in}}%
\pgfpathlineto{\pgfqpoint{2.210280in}{0.597545in}}%
\pgfpathlineto{\pgfqpoint{2.211146in}{0.597066in}}%
\pgfpathlineto{\pgfqpoint{2.211319in}{0.596528in}}%
\pgfpathlineto{\pgfqpoint{2.211839in}{0.598544in}}%
\pgfpathlineto{\pgfqpoint{2.212532in}{0.603474in}}%
\pgfpathlineto{\pgfqpoint{2.213398in}{0.598655in}}%
\pgfpathlineto{\pgfqpoint{2.214957in}{0.590584in}}%
\pgfpathlineto{\pgfqpoint{2.215650in}{0.592347in}}%
\pgfpathlineto{\pgfqpoint{2.217728in}{0.591111in}}%
\pgfpathlineto{\pgfqpoint{2.219461in}{0.590230in}}%
\pgfpathlineto{\pgfqpoint{2.223964in}{0.608484in}}%
\pgfpathlineto{\pgfqpoint{2.220154in}{0.589994in}}%
\pgfpathlineto{\pgfqpoint{2.225177in}{0.603758in}}%
\pgfpathlineto{\pgfqpoint{2.228468in}{0.586952in}}%
\pgfpathlineto{\pgfqpoint{2.228815in}{0.587376in}}%
\pgfpathlineto{\pgfqpoint{2.229681in}{0.591870in}}%
\pgfpathlineto{\pgfqpoint{2.230720in}{0.590731in}}%
\pgfpathlineto{\pgfqpoint{2.231066in}{0.592039in}}%
\pgfpathlineto{\pgfqpoint{2.233145in}{0.602185in}}%
\pgfpathlineto{\pgfqpoint{2.234011in}{0.598969in}}%
\pgfpathlineto{\pgfqpoint{2.235917in}{0.592502in}}%
\pgfpathlineto{\pgfqpoint{2.241460in}{0.575784in}}%
\pgfpathlineto{\pgfqpoint{2.241633in}{0.576162in}}%
\pgfpathlineto{\pgfqpoint{2.242672in}{0.577572in}}%
\pgfpathlineto{\pgfqpoint{2.243192in}{0.576087in}}%
\pgfpathlineto{\pgfqpoint{2.245617in}{0.576764in}}%
\pgfpathlineto{\pgfqpoint{2.245790in}{0.576337in}}%
\pgfpathlineto{\pgfqpoint{2.247349in}{0.570215in}}%
\pgfpathlineto{\pgfqpoint{2.248215in}{0.570775in}}%
\pgfpathlineto{\pgfqpoint{2.248388in}{0.572073in}}%
\pgfpathlineto{\pgfqpoint{2.249255in}{0.568049in}}%
\pgfpathlineto{\pgfqpoint{2.249947in}{0.566284in}}%
\pgfpathlineto{\pgfqpoint{2.250121in}{0.567190in}}%
\pgfpathlineto{\pgfqpoint{2.251160in}{0.575244in}}%
\pgfpathlineto{\pgfqpoint{2.252026in}{0.573544in}}%
\pgfpathlineto{\pgfqpoint{2.252373in}{0.572061in}}%
\pgfpathlineto{\pgfqpoint{2.253239in}{0.568718in}}%
\pgfpathlineto{\pgfqpoint{2.254624in}{0.569995in}}%
\pgfpathlineto{\pgfqpoint{2.255317in}{0.570528in}}%
\pgfpathlineto{\pgfqpoint{2.255664in}{0.568792in}}%
\pgfpathlineto{\pgfqpoint{2.257223in}{0.562221in}}%
\pgfpathlineto{\pgfqpoint{2.257742in}{0.565080in}}%
\pgfpathlineto{\pgfqpoint{2.260167in}{0.572988in}}%
\pgfpathlineto{\pgfqpoint{2.261380in}{0.574202in}}%
\pgfpathlineto{\pgfqpoint{2.261553in}{0.573212in}}%
\pgfpathlineto{\pgfqpoint{2.263285in}{0.567370in}}%
\pgfpathlineto{\pgfqpoint{2.263805in}{0.567751in}}%
\pgfpathlineto{\pgfqpoint{2.263978in}{0.568061in}}%
\pgfpathlineto{\pgfqpoint{2.264671in}{0.566161in}}%
\pgfpathlineto{\pgfqpoint{2.266577in}{0.556897in}}%
\pgfpathlineto{\pgfqpoint{2.266923in}{0.558730in}}%
\pgfpathlineto{\pgfqpoint{2.268655in}{0.563638in}}%
\pgfpathlineto{\pgfqpoint{2.268828in}{0.563354in}}%
\pgfpathlineto{\pgfqpoint{2.269348in}{0.561042in}}%
\pgfpathlineto{\pgfqpoint{2.270214in}{0.562442in}}%
\pgfpathlineto{\pgfqpoint{2.274025in}{0.580068in}}%
\pgfpathlineto{\pgfqpoint{2.274198in}{0.579556in}}%
\pgfpathlineto{\pgfqpoint{2.275064in}{0.575731in}}%
\pgfpathlineto{\pgfqpoint{2.275584in}{0.579874in}}%
\pgfpathlineto{\pgfqpoint{2.275931in}{0.581678in}}%
\pgfpathlineto{\pgfqpoint{2.276797in}{0.578891in}}%
\pgfpathlineto{\pgfqpoint{2.276970in}{0.579185in}}%
\pgfpathlineto{\pgfqpoint{2.279568in}{0.591207in}}%
\pgfpathlineto{\pgfqpoint{2.280088in}{0.593255in}}%
\pgfpathlineto{\pgfqpoint{2.281127in}{0.592474in}}%
\pgfpathlineto{\pgfqpoint{2.281820in}{0.593942in}}%
\pgfpathlineto{\pgfqpoint{2.283033in}{0.597616in}}%
\pgfpathlineto{\pgfqpoint{2.283379in}{0.595770in}}%
\pgfpathlineto{\pgfqpoint{2.285284in}{0.592520in}}%
\pgfpathlineto{\pgfqpoint{2.286151in}{0.593016in}}%
\pgfpathlineto{\pgfqpoint{2.287017in}{0.590779in}}%
\pgfpathlineto{\pgfqpoint{2.288749in}{0.595203in}}%
\pgfpathlineto{\pgfqpoint{2.289095in}{0.592789in}}%
\pgfpathlineto{\pgfqpoint{2.292386in}{0.587426in}}%
\pgfpathlineto{\pgfqpoint{2.289615in}{0.593400in}}%
\pgfpathlineto{\pgfqpoint{2.292560in}{0.587626in}}%
\pgfpathlineto{\pgfqpoint{2.294638in}{0.584710in}}%
\pgfpathlineto{\pgfqpoint{2.295678in}{0.578771in}}%
\pgfpathlineto{\pgfqpoint{2.296544in}{0.581888in}}%
\pgfpathlineto{\pgfqpoint{2.297063in}{0.583975in}}%
\pgfpathlineto{\pgfqpoint{2.297583in}{0.587420in}}%
\pgfpathlineto{\pgfqpoint{2.298103in}{0.583536in}}%
\pgfpathlineto{\pgfqpoint{2.300701in}{0.572472in}}%
\pgfpathlineto{\pgfqpoint{2.300874in}{0.572994in}}%
\pgfpathlineto{\pgfqpoint{2.302780in}{0.699129in}}%
\pgfpathlineto{\pgfqpoint{2.305551in}{0.828287in}}%
\pgfpathlineto{\pgfqpoint{2.311614in}{1.015795in}}%
\pgfpathlineto{\pgfqpoint{2.316811in}{1.077730in}}%
\pgfpathlineto{\pgfqpoint{2.319755in}{1.094116in}}%
\pgfpathlineto{\pgfqpoint{2.320275in}{1.093802in}}%
\pgfpathlineto{\pgfqpoint{2.324952in}{1.090049in}}%
\pgfpathlineto{\pgfqpoint{2.328936in}{1.079804in}}%
\pgfpathlineto{\pgfqpoint{2.347817in}{1.032004in}}%
\pgfpathlineto{\pgfqpoint{2.354573in}{0.993538in}}%
\pgfpathlineto{\pgfqpoint{2.368777in}{0.945633in}}%
\pgfpathlineto{\pgfqpoint{2.370162in}{0.942018in}}%
\pgfpathlineto{\pgfqpoint{2.371548in}{0.938232in}}%
\pgfpathlineto{\pgfqpoint{2.371895in}{0.938996in}}%
\pgfpathlineto{\pgfqpoint{2.373280in}{0.940965in}}%
\pgfpathlineto{\pgfqpoint{2.373973in}{0.940476in}}%
\pgfpathlineto{\pgfqpoint{2.379689in}{0.927124in}}%
\pgfpathlineto{\pgfqpoint{2.382115in}{0.919826in}}%
\pgfpathlineto{\pgfqpoint{2.382461in}{0.919988in}}%
\pgfpathlineto{\pgfqpoint{2.384020in}{0.917890in}}%
\pgfpathlineto{\pgfqpoint{2.384540in}{0.918492in}}%
\pgfpathlineto{\pgfqpoint{2.389736in}{0.932270in}}%
\pgfpathlineto{\pgfqpoint{2.390602in}{0.934180in}}%
\pgfpathlineto{\pgfqpoint{2.391642in}{0.933079in}}%
\pgfpathlineto{\pgfqpoint{2.392681in}{0.934670in}}%
\pgfpathlineto{\pgfqpoint{2.392854in}{0.935030in}}%
\pgfpathlineto{\pgfqpoint{2.393201in}{0.934362in}}%
\pgfpathlineto{\pgfqpoint{2.393894in}{0.934261in}}%
\pgfpathlineto{\pgfqpoint{2.394760in}{0.931968in}}%
\pgfpathlineto{\pgfqpoint{2.395972in}{0.933135in}}%
\pgfpathlineto{\pgfqpoint{2.397878in}{0.925246in}}%
\pgfpathlineto{\pgfqpoint{2.401169in}{0.913499in}}%
\pgfpathlineto{\pgfqpoint{2.404287in}{0.899296in}}%
\pgfpathlineto{\pgfqpoint{2.407924in}{0.884405in}}%
\pgfpathlineto{\pgfqpoint{2.408444in}{0.884837in}}%
\pgfpathlineto{\pgfqpoint{2.409137in}{0.884139in}}%
\pgfpathlineto{\pgfqpoint{2.415373in}{0.859664in}}%
\pgfpathlineto{\pgfqpoint{2.415546in}{0.859866in}}%
\pgfpathlineto{\pgfqpoint{2.416585in}{0.862125in}}%
\pgfpathlineto{\pgfqpoint{2.417105in}{0.860401in}}%
\pgfpathlineto{\pgfqpoint{2.420743in}{0.837247in}}%
\pgfpathlineto{\pgfqpoint{2.421609in}{0.838886in}}%
\pgfpathlineto{\pgfqpoint{2.424727in}{0.823711in}}%
\pgfpathlineto{\pgfqpoint{2.428711in}{0.807781in}}%
\pgfpathlineto{\pgfqpoint{2.435986in}{0.785710in}}%
\pgfpathlineto{\pgfqpoint{2.436679in}{0.786134in}}%
\pgfpathlineto{\pgfqpoint{2.437025in}{0.784607in}}%
\pgfpathlineto{\pgfqpoint{2.437545in}{0.780671in}}%
\pgfpathlineto{\pgfqpoint{2.438411in}{0.784300in}}%
\pgfpathlineto{\pgfqpoint{2.439104in}{0.788907in}}%
\pgfpathlineto{\pgfqpoint{2.439970in}{0.785152in}}%
\pgfpathlineto{\pgfqpoint{2.442742in}{0.777623in}}%
\pgfpathlineto{\pgfqpoint{2.445167in}{0.788754in}}%
\pgfpathlineto{\pgfqpoint{2.445860in}{0.785422in}}%
\pgfpathlineto{\pgfqpoint{2.447245in}{0.778623in}}%
\pgfpathlineto{\pgfqpoint{2.447938in}{0.780703in}}%
\pgfpathlineto{\pgfqpoint{2.448978in}{0.787294in}}%
\pgfpathlineto{\pgfqpoint{2.449670in}{0.783050in}}%
\pgfpathlineto{\pgfqpoint{2.451922in}{0.777887in}}%
\pgfpathlineto{\pgfqpoint{2.452096in}{0.778584in}}%
\pgfpathlineto{\pgfqpoint{2.456599in}{0.792797in}}%
\pgfpathlineto{\pgfqpoint{2.457465in}{0.789864in}}%
\pgfpathlineto{\pgfqpoint{2.459198in}{0.786099in}}%
\pgfpathlineto{\pgfqpoint{2.459371in}{0.786587in}}%
\pgfpathlineto{\pgfqpoint{2.459544in}{0.786521in}}%
\pgfpathlineto{\pgfqpoint{2.459717in}{0.787326in}}%
\pgfpathlineto{\pgfqpoint{2.460757in}{0.788954in}}%
\pgfpathlineto{\pgfqpoint{2.461276in}{0.787355in}}%
\pgfpathlineto{\pgfqpoint{2.461623in}{0.788418in}}%
\pgfpathlineto{\pgfqpoint{2.462142in}{0.786698in}}%
\pgfpathlineto{\pgfqpoint{2.464741in}{0.781780in}}%
\pgfpathlineto{\pgfqpoint{2.464914in}{0.782858in}}%
\pgfpathlineto{\pgfqpoint{2.465087in}{0.784306in}}%
\pgfpathlineto{\pgfqpoint{2.466126in}{0.780252in}}%
\pgfpathlineto{\pgfqpoint{2.468551in}{0.772891in}}%
\pgfpathlineto{\pgfqpoint{2.470110in}{0.767772in}}%
\pgfpathlineto{\pgfqpoint{2.470630in}{0.769205in}}%
\pgfpathlineto{\pgfqpoint{2.472016in}{0.773690in}}%
\pgfpathlineto{\pgfqpoint{2.472536in}{0.772068in}}%
\pgfpathlineto{\pgfqpoint{2.472882in}{0.772593in}}%
\pgfpathlineto{\pgfqpoint{2.473402in}{0.771351in}}%
\pgfpathlineto{\pgfqpoint{2.474614in}{0.766674in}}%
\pgfpathlineto{\pgfqpoint{2.475307in}{0.768316in}}%
\pgfpathlineto{\pgfqpoint{2.476520in}{0.771104in}}%
\pgfpathlineto{\pgfqpoint{2.477039in}{0.768480in}}%
\pgfpathlineto{\pgfqpoint{2.477732in}{0.770756in}}%
\pgfpathlineto{\pgfqpoint{2.479118in}{0.765707in}}%
\pgfpathlineto{\pgfqpoint{2.481716in}{0.773199in}}%
\pgfpathlineto{\pgfqpoint{2.482063in}{0.777201in}}%
\pgfpathlineto{\pgfqpoint{2.483102in}{0.773865in}}%
\pgfpathlineto{\pgfqpoint{2.484315in}{0.767571in}}%
\pgfpathlineto{\pgfqpoint{2.485354in}{0.768182in}}%
\pgfpathlineto{\pgfqpoint{2.486220in}{0.769068in}}%
\pgfpathlineto{\pgfqpoint{2.488818in}{0.754671in}}%
\pgfpathlineto{\pgfqpoint{2.490377in}{0.756314in}}%
\pgfpathlineto{\pgfqpoint{2.492976in}{0.748678in}}%
\pgfpathlineto{\pgfqpoint{2.490897in}{0.757414in}}%
\pgfpathlineto{\pgfqpoint{2.494188in}{0.750799in}}%
\pgfpathlineto{\pgfqpoint{2.495227in}{0.751581in}}%
\pgfpathlineto{\pgfqpoint{2.495401in}{0.751070in}}%
\pgfpathlineto{\pgfqpoint{2.497306in}{0.745544in}}%
\pgfpathlineto{\pgfqpoint{2.497479in}{0.746263in}}%
\pgfpathlineto{\pgfqpoint{2.497826in}{0.746184in}}%
\pgfpathlineto{\pgfqpoint{2.499211in}{0.740446in}}%
\pgfpathlineto{\pgfqpoint{2.499558in}{0.742341in}}%
\pgfpathlineto{\pgfqpoint{2.501637in}{0.746166in}}%
\pgfpathlineto{\pgfqpoint{2.501983in}{0.745706in}}%
\pgfpathlineto{\pgfqpoint{2.502849in}{0.736855in}}%
\pgfpathlineto{\pgfqpoint{2.504235in}{0.742505in}}%
\pgfpathlineto{\pgfqpoint{2.505794in}{0.751584in}}%
\pgfpathlineto{\pgfqpoint{2.506487in}{0.749297in}}%
\pgfpathlineto{\pgfqpoint{2.506660in}{0.748737in}}%
\pgfpathlineto{\pgfqpoint{2.507180in}{0.751035in}}%
\pgfpathlineto{\pgfqpoint{2.507699in}{0.752692in}}%
\pgfpathlineto{\pgfqpoint{2.508739in}{0.752349in}}%
\pgfpathlineto{\pgfqpoint{2.508912in}{0.752199in}}%
\pgfpathlineto{\pgfqpoint{2.509085in}{0.753375in}}%
\pgfpathlineto{\pgfqpoint{2.509605in}{0.756241in}}%
\pgfpathlineto{\pgfqpoint{2.510471in}{0.753421in}}%
\pgfpathlineto{\pgfqpoint{2.513935in}{0.744362in}}%
\pgfpathlineto{\pgfqpoint{2.510990in}{0.753636in}}%
\pgfpathlineto{\pgfqpoint{2.514628in}{0.746388in}}%
\pgfpathlineto{\pgfqpoint{2.514975in}{0.748373in}}%
\pgfpathlineto{\pgfqpoint{2.515841in}{0.746733in}}%
\pgfpathlineto{\pgfqpoint{2.517746in}{0.741970in}}%
\pgfpathlineto{\pgfqpoint{2.517919in}{0.742608in}}%
\pgfpathlineto{\pgfqpoint{2.518785in}{0.740348in}}%
\pgfpathlineto{\pgfqpoint{2.518959in}{0.740635in}}%
\pgfpathlineto{\pgfqpoint{2.520864in}{0.748855in}}%
\pgfpathlineto{\pgfqpoint{2.521557in}{0.750746in}}%
\pgfpathlineto{\pgfqpoint{2.522250in}{0.748228in}}%
\pgfpathlineto{\pgfqpoint{2.527100in}{0.729679in}}%
\pgfpathlineto{\pgfqpoint{2.528139in}{0.734453in}}%
\pgfpathlineto{\pgfqpoint{2.529179in}{0.742277in}}%
\pgfpathlineto{\pgfqpoint{2.530564in}{0.753110in}}%
\pgfpathlineto{\pgfqpoint{2.531084in}{0.749611in}}%
\pgfpathlineto{\pgfqpoint{2.534029in}{0.734938in}}%
\pgfpathlineto{\pgfqpoint{2.534722in}{0.736445in}}%
\pgfpathlineto{\pgfqpoint{2.535241in}{0.733692in}}%
\pgfpathlineto{\pgfqpoint{2.535761in}{0.733082in}}%
\pgfpathlineto{\pgfqpoint{2.536281in}{0.736000in}}%
\pgfpathlineto{\pgfqpoint{2.537840in}{0.743872in}}%
\pgfpathlineto{\pgfqpoint{2.538359in}{0.741211in}}%
\pgfpathlineto{\pgfqpoint{2.538879in}{0.742514in}}%
\pgfpathlineto{\pgfqpoint{2.539745in}{0.736127in}}%
\pgfpathlineto{\pgfqpoint{2.541997in}{0.721803in}}%
\pgfpathlineto{\pgfqpoint{2.542343in}{0.721436in}}%
\pgfpathlineto{\pgfqpoint{2.542690in}{0.722889in}}%
\pgfpathlineto{\pgfqpoint{2.551004in}{0.763009in}}%
\pgfpathlineto{\pgfqpoint{2.553603in}{0.745126in}}%
\pgfpathlineto{\pgfqpoint{2.553949in}{0.747701in}}%
\pgfpathlineto{\pgfqpoint{2.554469in}{0.746318in}}%
\pgfpathlineto{\pgfqpoint{2.555162in}{0.748923in}}%
\pgfpathlineto{\pgfqpoint{2.558280in}{0.757923in}}%
\pgfpathlineto{\pgfqpoint{2.558453in}{0.757729in}}%
\pgfpathlineto{\pgfqpoint{2.559319in}{0.758637in}}%
\pgfpathlineto{\pgfqpoint{2.559665in}{0.757215in}}%
\pgfpathlineto{\pgfqpoint{2.560012in}{0.755627in}}%
\pgfpathlineto{\pgfqpoint{2.560705in}{0.759417in}}%
\pgfpathlineto{\pgfqpoint{2.561398in}{0.762026in}}%
\pgfpathlineto{\pgfqpoint{2.561917in}{0.759032in}}%
\pgfpathlineto{\pgfqpoint{2.562090in}{0.759279in}}%
\pgfpathlineto{\pgfqpoint{2.563823in}{0.755094in}}%
\pgfpathlineto{\pgfqpoint{2.565035in}{0.746625in}}%
\pgfpathlineto{\pgfqpoint{2.565728in}{0.751919in}}%
\pgfpathlineto{\pgfqpoint{2.567460in}{0.755473in}}%
\pgfpathlineto{\pgfqpoint{2.568326in}{0.760042in}}%
\pgfpathlineto{\pgfqpoint{2.569192in}{0.756343in}}%
\pgfpathlineto{\pgfqpoint{2.573350in}{0.735196in}}%
\pgfpathlineto{\pgfqpoint{2.573523in}{0.735306in}}%
\pgfpathlineto{\pgfqpoint{2.577680in}{0.749277in}}%
\pgfpathlineto{\pgfqpoint{2.577854in}{0.749100in}}%
\pgfpathlineto{\pgfqpoint{2.578200in}{0.750146in}}%
\pgfpathlineto{\pgfqpoint{2.578720in}{0.750216in}}%
\pgfpathlineto{\pgfqpoint{2.579759in}{0.751653in}}%
\pgfpathlineto{\pgfqpoint{2.580279in}{0.749852in}}%
\pgfpathlineto{\pgfqpoint{2.582704in}{0.745167in}}%
\pgfpathlineto{\pgfqpoint{2.584782in}{0.752847in}}%
\pgfpathlineto{\pgfqpoint{2.585475in}{0.750621in}}%
\pgfpathlineto{\pgfqpoint{2.588766in}{0.734218in}}%
\pgfpathlineto{\pgfqpoint{2.589459in}{0.733922in}}%
\pgfpathlineto{\pgfqpoint{2.589632in}{0.733046in}}%
\pgfpathlineto{\pgfqpoint{2.591538in}{0.726511in}}%
\pgfpathlineto{\pgfqpoint{2.593790in}{0.709966in}}%
\pgfpathlineto{\pgfqpoint{2.594136in}{0.713171in}}%
\pgfpathlineto{\pgfqpoint{2.596042in}{0.722881in}}%
\pgfpathlineto{\pgfqpoint{2.596561in}{0.722086in}}%
\pgfpathlineto{\pgfqpoint{2.597774in}{0.723001in}}%
\pgfpathlineto{\pgfqpoint{2.598986in}{0.730331in}}%
\pgfpathlineto{\pgfqpoint{2.600199in}{0.726970in}}%
\pgfpathlineto{\pgfqpoint{2.600545in}{0.724536in}}%
\pgfpathlineto{\pgfqpoint{2.601585in}{0.727671in}}%
\pgfpathlineto{\pgfqpoint{2.602104in}{0.725531in}}%
\pgfpathlineto{\pgfqpoint{2.602451in}{0.727289in}}%
\pgfpathlineto{\pgfqpoint{2.604183in}{0.735222in}}%
\pgfpathlineto{\pgfqpoint{2.604356in}{0.735016in}}%
\pgfpathlineto{\pgfqpoint{2.606088in}{0.723879in}}%
\pgfpathlineto{\pgfqpoint{2.606435in}{0.724198in}}%
\pgfpathlineto{\pgfqpoint{2.607647in}{0.729256in}}%
\pgfpathlineto{\pgfqpoint{2.608167in}{0.726744in}}%
\pgfpathlineto{\pgfqpoint{2.609380in}{0.716922in}}%
\pgfpathlineto{\pgfqpoint{2.610246in}{0.719820in}}%
\pgfpathlineto{\pgfqpoint{2.612671in}{0.724709in}}%
\pgfpathlineto{\pgfqpoint{2.612844in}{0.724593in}}%
\pgfpathlineto{\pgfqpoint{2.615096in}{0.719470in}}%
\pgfpathlineto{\pgfqpoint{2.615442in}{0.720407in}}%
\pgfpathlineto{\pgfqpoint{2.616482in}{0.721388in}}%
\pgfpathlineto{\pgfqpoint{2.616655in}{0.720088in}}%
\pgfpathlineto{\pgfqpoint{2.618214in}{0.716491in}}%
\pgfpathlineto{\pgfqpoint{2.618387in}{0.717603in}}%
\pgfpathlineto{\pgfqpoint{2.618907in}{0.719437in}}%
\pgfpathlineto{\pgfqpoint{2.619426in}{0.716629in}}%
\pgfpathlineto{\pgfqpoint{2.622198in}{0.702221in}}%
\pgfpathlineto{\pgfqpoint{2.622718in}{0.704711in}}%
\pgfpathlineto{\pgfqpoint{2.624623in}{0.708283in}}%
\pgfpathlineto{\pgfqpoint{2.624796in}{0.708250in}}%
\pgfpathlineto{\pgfqpoint{2.627221in}{0.714202in}}%
\pgfpathlineto{\pgfqpoint{2.625489in}{0.706749in}}%
\pgfpathlineto{\pgfqpoint{2.627395in}{0.713001in}}%
\pgfpathlineto{\pgfqpoint{2.629300in}{0.704125in}}%
\pgfpathlineto{\pgfqpoint{2.629646in}{0.700731in}}%
\pgfpathlineto{\pgfqpoint{2.630686in}{0.706224in}}%
\pgfpathlineto{\pgfqpoint{2.631898in}{0.704066in}}%
\pgfpathlineto{\pgfqpoint{2.632071in}{0.703115in}}%
\pgfpathlineto{\pgfqpoint{2.632764in}{0.707657in}}%
\pgfpathlineto{\pgfqpoint{2.633284in}{0.708708in}}%
\pgfpathlineto{\pgfqpoint{2.633457in}{0.706554in}}%
\pgfpathlineto{\pgfqpoint{2.634670in}{0.693351in}}%
\pgfpathlineto{\pgfqpoint{2.635189in}{0.699782in}}%
\pgfpathlineto{\pgfqpoint{2.635882in}{0.703045in}}%
\pgfpathlineto{\pgfqpoint{2.636402in}{0.698796in}}%
\pgfpathlineto{\pgfqpoint{2.636575in}{0.698857in}}%
\pgfpathlineto{\pgfqpoint{2.637615in}{0.702287in}}%
\pgfpathlineto{\pgfqpoint{2.638654in}{0.701000in}}%
\pgfpathlineto{\pgfqpoint{2.643504in}{0.688244in}}%
\pgfpathlineto{\pgfqpoint{2.643850in}{0.688728in}}%
\pgfpathlineto{\pgfqpoint{2.644024in}{0.686932in}}%
\pgfpathlineto{\pgfqpoint{2.644717in}{0.684973in}}%
\pgfpathlineto{\pgfqpoint{2.645063in}{0.687316in}}%
\pgfpathlineto{\pgfqpoint{2.649220in}{0.702681in}}%
\pgfpathlineto{\pgfqpoint{2.650086in}{0.700383in}}%
\pgfpathlineto{\pgfqpoint{2.650606in}{0.703444in}}%
\pgfpathlineto{\pgfqpoint{2.650779in}{0.703501in}}%
\pgfpathlineto{\pgfqpoint{2.651645in}{0.705686in}}%
\pgfpathlineto{\pgfqpoint{2.652511in}{0.700066in}}%
\pgfpathlineto{\pgfqpoint{2.652685in}{0.700017in}}%
\pgfpathlineto{\pgfqpoint{2.654590in}{0.706433in}}%
\pgfpathlineto{\pgfqpoint{2.654763in}{0.707190in}}%
\pgfpathlineto{\pgfqpoint{2.655283in}{0.702851in}}%
\pgfpathlineto{\pgfqpoint{2.655456in}{0.701659in}}%
\pgfpathlineto{\pgfqpoint{2.656496in}{0.704985in}}%
\pgfpathlineto{\pgfqpoint{2.656669in}{0.703948in}}%
\pgfpathlineto{\pgfqpoint{2.658055in}{0.706604in}}%
\pgfpathlineto{\pgfqpoint{2.657535in}{0.702932in}}%
\pgfpathlineto{\pgfqpoint{2.658228in}{0.704536in}}%
\pgfpathlineto{\pgfqpoint{2.658921in}{0.698463in}}%
\pgfpathlineto{\pgfqpoint{2.659960in}{0.701819in}}%
\pgfpathlineto{\pgfqpoint{2.660133in}{0.703060in}}%
\pgfpathlineto{\pgfqpoint{2.660999in}{0.699233in}}%
\pgfpathlineto{\pgfqpoint{2.666716in}{0.681875in}}%
\pgfpathlineto{\pgfqpoint{2.668448in}{0.693515in}}%
\pgfpathlineto{\pgfqpoint{2.668621in}{0.692298in}}%
\pgfpathlineto{\pgfqpoint{2.669660in}{0.679953in}}%
\pgfpathlineto{\pgfqpoint{2.670353in}{0.689015in}}%
\pgfpathlineto{\pgfqpoint{2.671739in}{0.691260in}}%
\pgfpathlineto{\pgfqpoint{2.671392in}{0.687978in}}%
\pgfpathlineto{\pgfqpoint{2.671912in}{0.690850in}}%
\pgfpathlineto{\pgfqpoint{2.672085in}{0.688356in}}%
\pgfpathlineto{\pgfqpoint{2.673298in}{0.691927in}}%
\pgfpathlineto{\pgfqpoint{2.673471in}{0.691842in}}%
\pgfpathlineto{\pgfqpoint{2.674857in}{0.699692in}}%
\pgfpathlineto{\pgfqpoint{2.675377in}{0.696104in}}%
\pgfpathlineto{\pgfqpoint{2.677628in}{0.701497in}}%
\pgfpathlineto{\pgfqpoint{2.678321in}{0.699831in}}%
\pgfpathlineto{\pgfqpoint{2.680573in}{0.689441in}}%
\pgfpathlineto{\pgfqpoint{2.680746in}{0.689530in}}%
\pgfpathlineto{\pgfqpoint{2.680920in}{0.689175in}}%
\pgfpathlineto{\pgfqpoint{2.683518in}{0.678454in}}%
\pgfpathlineto{\pgfqpoint{2.684557in}{0.674674in}}%
\pgfpathlineto{\pgfqpoint{2.684904in}{0.677034in}}%
\pgfpathlineto{\pgfqpoint{2.685250in}{0.680732in}}%
\pgfpathlineto{\pgfqpoint{2.686289in}{0.677816in}}%
\pgfpathlineto{\pgfqpoint{2.687329in}{0.668927in}}%
\pgfpathlineto{\pgfqpoint{2.687848in}{0.674595in}}%
\pgfpathlineto{\pgfqpoint{2.688715in}{0.690312in}}%
\pgfpathlineto{\pgfqpoint{2.689754in}{0.684613in}}%
\pgfpathlineto{\pgfqpoint{2.690793in}{0.690955in}}%
\pgfpathlineto{\pgfqpoint{2.692699in}{0.703158in}}%
\pgfpathlineto{\pgfqpoint{2.692872in}{0.702390in}}%
\pgfpathlineto{\pgfqpoint{2.695470in}{0.693289in}}%
\pgfpathlineto{\pgfqpoint{2.693738in}{0.702567in}}%
\pgfpathlineto{\pgfqpoint{2.695643in}{0.694187in}}%
\pgfpathlineto{\pgfqpoint{2.696163in}{0.693029in}}%
\pgfpathlineto{\pgfqpoint{2.698761in}{0.686077in}}%
\pgfpathlineto{\pgfqpoint{2.700147in}{0.686556in}}%
\pgfpathlineto{\pgfqpoint{2.700320in}{0.687831in}}%
\pgfpathlineto{\pgfqpoint{2.701013in}{0.681771in}}%
\pgfpathlineto{\pgfqpoint{2.702399in}{0.676760in}}%
\pgfpathlineto{\pgfqpoint{2.701533in}{0.682579in}}%
\pgfpathlineto{\pgfqpoint{2.702572in}{0.681127in}}%
\pgfpathlineto{\pgfqpoint{2.703265in}{0.688197in}}%
\pgfpathlineto{\pgfqpoint{2.704131in}{0.682641in}}%
\pgfpathlineto{\pgfqpoint{2.704304in}{0.678218in}}%
\pgfpathlineto{\pgfqpoint{2.705344in}{0.685431in}}%
\pgfpathlineto{\pgfqpoint{2.705517in}{0.684958in}}%
\pgfpathlineto{\pgfqpoint{2.707076in}{0.691931in}}%
\pgfpathlineto{\pgfqpoint{2.708115in}{0.690612in}}%
\pgfpathlineto{\pgfqpoint{2.708808in}{0.688230in}}%
\pgfpathlineto{\pgfqpoint{2.709328in}{0.691143in}}%
\pgfpathlineto{\pgfqpoint{2.709501in}{0.690098in}}%
\pgfpathlineto{\pgfqpoint{2.711060in}{0.706466in}}%
\pgfpathlineto{\pgfqpoint{2.711753in}{0.707810in}}%
\pgfpathlineto{\pgfqpoint{2.712099in}{0.705735in}}%
\pgfpathlineto{\pgfqpoint{2.713485in}{0.697446in}}%
\pgfpathlineto{\pgfqpoint{2.714178in}{0.699907in}}%
\pgfpathlineto{\pgfqpoint{2.716083in}{0.706292in}}%
\pgfpathlineto{\pgfqpoint{2.716257in}{0.704151in}}%
\pgfpathlineto{\pgfqpoint{2.717123in}{0.709861in}}%
\pgfpathlineto{\pgfqpoint{2.717642in}{0.714359in}}%
\pgfpathlineto{\pgfqpoint{2.718855in}{0.711802in}}%
\pgfpathlineto{\pgfqpoint{2.721107in}{0.704317in}}%
\pgfpathlineto{\pgfqpoint{2.721626in}{0.705981in}}%
\pgfpathlineto{\pgfqpoint{2.723185in}{0.718170in}}%
\pgfpathlineto{\pgfqpoint{2.723532in}{0.715244in}}%
\pgfpathlineto{\pgfqpoint{2.724225in}{0.712599in}}%
\pgfpathlineto{\pgfqpoint{2.726303in}{0.703645in}}%
\pgfpathlineto{\pgfqpoint{2.726996in}{0.706675in}}%
\pgfpathlineto{\pgfqpoint{2.728728in}{0.710634in}}%
\pgfpathlineto{\pgfqpoint{2.729768in}{0.714676in}}%
\pgfpathlineto{\pgfqpoint{2.730287in}{0.710462in}}%
\pgfpathlineto{\pgfqpoint{2.731153in}{0.711134in}}%
\pgfpathlineto{\pgfqpoint{2.732366in}{0.704064in}}%
\pgfpathlineto{\pgfqpoint{2.732539in}{0.704387in}}%
\pgfpathlineto{\pgfqpoint{2.732886in}{0.702344in}}%
\pgfpathlineto{\pgfqpoint{2.736350in}{0.687284in}}%
\pgfpathlineto{\pgfqpoint{2.736523in}{0.687701in}}%
\pgfpathlineto{\pgfqpoint{2.737043in}{0.687900in}}%
\pgfpathlineto{\pgfqpoint{2.737216in}{0.688761in}}%
\pgfpathlineto{\pgfqpoint{2.737736in}{0.689074in}}%
\pgfpathlineto{\pgfqpoint{2.738429in}{0.685313in}}%
\pgfpathlineto{\pgfqpoint{2.739468in}{0.685497in}}%
\pgfpathlineto{\pgfqpoint{2.739641in}{0.684914in}}%
\pgfpathlineto{\pgfqpoint{2.740161in}{0.687337in}}%
\pgfpathlineto{\pgfqpoint{2.740854in}{0.685642in}}%
\pgfpathlineto{\pgfqpoint{2.743279in}{0.700624in}}%
\pgfpathlineto{\pgfqpoint{2.745184in}{0.707637in}}%
\pgfpathlineto{\pgfqpoint{2.745358in}{0.707102in}}%
\pgfpathlineto{\pgfqpoint{2.747263in}{0.699682in}}%
\pgfpathlineto{\pgfqpoint{2.746397in}{0.707897in}}%
\pgfpathlineto{\pgfqpoint{2.747956in}{0.703299in}}%
\pgfpathlineto{\pgfqpoint{2.748302in}{0.706369in}}%
\pgfpathlineto{\pgfqpoint{2.748995in}{0.696713in}}%
\pgfpathlineto{\pgfqpoint{2.749168in}{0.694905in}}%
\pgfpathlineto{\pgfqpoint{2.749861in}{0.701409in}}%
\pgfpathlineto{\pgfqpoint{2.751420in}{0.711635in}}%
\pgfpathlineto{\pgfqpoint{2.751593in}{0.710862in}}%
\pgfpathlineto{\pgfqpoint{2.752979in}{0.707542in}}%
\pgfpathlineto{\pgfqpoint{2.753326in}{0.707898in}}%
\pgfpathlineto{\pgfqpoint{2.757483in}{0.677439in}}%
\pgfpathlineto{\pgfqpoint{2.758176in}{0.679633in}}%
\pgfpathlineto{\pgfqpoint{2.760774in}{0.699876in}}%
\pgfpathlineto{\pgfqpoint{2.761121in}{0.699132in}}%
\pgfpathlineto{\pgfqpoint{2.761813in}{0.695775in}}%
\pgfpathlineto{\pgfqpoint{2.762853in}{0.697118in}}%
\pgfpathlineto{\pgfqpoint{2.763892in}{0.702856in}}%
\pgfpathlineto{\pgfqpoint{2.764758in}{0.699458in}}%
\pgfpathlineto{\pgfqpoint{2.766490in}{0.693507in}}%
\pgfpathlineto{\pgfqpoint{2.769089in}{0.706024in}}%
\pgfpathlineto{\pgfqpoint{2.769435in}{0.707162in}}%
\pgfpathlineto{\pgfqpoint{2.769782in}{0.703474in}}%
\pgfpathlineto{\pgfqpoint{2.769955in}{0.703276in}}%
\pgfpathlineto{\pgfqpoint{2.770301in}{0.707419in}}%
\pgfpathlineto{\pgfqpoint{2.771167in}{0.698783in}}%
\pgfpathlineto{\pgfqpoint{2.771687in}{0.697130in}}%
\pgfpathlineto{\pgfqpoint{2.772553in}{0.698856in}}%
\pgfpathlineto{\pgfqpoint{2.772900in}{0.702604in}}%
\pgfpathlineto{\pgfqpoint{2.773592in}{0.697601in}}%
\pgfpathlineto{\pgfqpoint{2.774285in}{0.701161in}}%
\pgfpathlineto{\pgfqpoint{2.774805in}{0.699353in}}%
\pgfpathlineto{\pgfqpoint{2.775498in}{0.703770in}}%
\pgfpathlineto{\pgfqpoint{2.776364in}{0.697145in}}%
\pgfpathlineto{\pgfqpoint{2.778962in}{0.710369in}}%
\pgfpathlineto{\pgfqpoint{2.779655in}{0.707311in}}%
\pgfpathlineto{\pgfqpoint{2.779828in}{0.707096in}}%
\pgfpathlineto{\pgfqpoint{2.780002in}{0.704404in}}%
\pgfpathlineto{\pgfqpoint{2.781041in}{0.709112in}}%
\pgfpathlineto{\pgfqpoint{2.782600in}{0.712503in}}%
\pgfpathlineto{\pgfqpoint{2.782080in}{0.707610in}}%
\pgfpathlineto{\pgfqpoint{2.782946in}{0.711593in}}%
\pgfpathlineto{\pgfqpoint{2.783986in}{0.706893in}}%
\pgfpathlineto{\pgfqpoint{2.783293in}{0.712273in}}%
\pgfpathlineto{\pgfqpoint{2.785025in}{0.709225in}}%
\pgfpathlineto{\pgfqpoint{2.785545in}{0.712413in}}%
\pgfpathlineto{\pgfqpoint{2.786238in}{0.706480in}}%
\pgfpathlineto{\pgfqpoint{2.786584in}{0.709393in}}%
\pgfpathlineto{\pgfqpoint{2.786757in}{0.709403in}}%
\pgfpathlineto{\pgfqpoint{2.788143in}{0.698802in}}%
\pgfpathlineto{\pgfqpoint{2.788663in}{0.700954in}}%
\pgfpathlineto{\pgfqpoint{2.790222in}{0.712799in}}%
\pgfpathlineto{\pgfqpoint{2.790395in}{0.710270in}}%
\pgfpathlineto{\pgfqpoint{2.791434in}{0.703237in}}%
\pgfpathlineto{\pgfqpoint{2.792300in}{0.705540in}}%
\pgfpathlineto{\pgfqpoint{2.793513in}{0.710973in}}%
\pgfpathlineto{\pgfqpoint{2.794206in}{0.708998in}}%
\pgfpathlineto{\pgfqpoint{2.798536in}{0.680414in}}%
\pgfpathlineto{\pgfqpoint{2.798709in}{0.681048in}}%
\pgfpathlineto{\pgfqpoint{2.799576in}{0.677763in}}%
\pgfpathlineto{\pgfqpoint{2.799922in}{0.680312in}}%
\pgfpathlineto{\pgfqpoint{2.801827in}{0.686219in}}%
\pgfpathlineto{\pgfqpoint{2.802001in}{0.685660in}}%
\pgfpathlineto{\pgfqpoint{2.802174in}{0.687843in}}%
\pgfpathlineto{\pgfqpoint{2.803213in}{0.690588in}}%
\pgfpathlineto{\pgfqpoint{2.803560in}{0.688519in}}%
\pgfpathlineto{\pgfqpoint{2.804079in}{0.686941in}}%
\pgfpathlineto{\pgfqpoint{2.804599in}{0.690166in}}%
\pgfpathlineto{\pgfqpoint{2.807544in}{0.701218in}}%
\pgfpathlineto{\pgfqpoint{2.808063in}{0.698102in}}%
\pgfpathlineto{\pgfqpoint{2.811354in}{0.689281in}}%
\pgfpathlineto{\pgfqpoint{2.813260in}{0.692999in}}%
\pgfpathlineto{\pgfqpoint{2.813433in}{0.692868in}}%
\pgfpathlineto{\pgfqpoint{2.817071in}{0.675735in}}%
\pgfpathlineto{\pgfqpoint{2.818110in}{0.679686in}}%
\pgfpathlineto{\pgfqpoint{2.821574in}{0.698272in}}%
\pgfpathlineto{\pgfqpoint{2.818976in}{0.679397in}}%
\pgfpathlineto{\pgfqpoint{2.821921in}{0.693208in}}%
\pgfpathlineto{\pgfqpoint{2.822267in}{0.691741in}}%
\pgfpathlineto{\pgfqpoint{2.823133in}{0.695591in}}%
\pgfpathlineto{\pgfqpoint{2.823480in}{0.698283in}}%
\pgfpathlineto{\pgfqpoint{2.824519in}{0.695123in}}%
\pgfpathlineto{\pgfqpoint{2.824692in}{0.695862in}}%
\pgfpathlineto{\pgfqpoint{2.825559in}{0.697057in}}%
\pgfpathlineto{\pgfqpoint{2.825732in}{0.694853in}}%
\pgfpathlineto{\pgfqpoint{2.825905in}{0.694020in}}%
\pgfpathlineto{\pgfqpoint{2.826251in}{0.697025in}}%
\pgfpathlineto{\pgfqpoint{2.826944in}{0.705276in}}%
\pgfpathlineto{\pgfqpoint{2.828330in}{0.704292in}}%
\pgfpathlineto{\pgfqpoint{2.830582in}{0.709315in}}%
\pgfpathlineto{\pgfqpoint{2.831275in}{0.704912in}}%
\pgfpathlineto{\pgfqpoint{2.831794in}{0.708496in}}%
\pgfpathlineto{\pgfqpoint{2.832487in}{0.716383in}}%
\pgfpathlineto{\pgfqpoint{2.833873in}{0.715999in}}%
\pgfpathlineto{\pgfqpoint{2.834566in}{0.720026in}}%
\pgfpathlineto{\pgfqpoint{2.835086in}{0.717067in}}%
\pgfpathlineto{\pgfqpoint{2.837164in}{0.696397in}}%
\pgfpathlineto{\pgfqpoint{2.837684in}{0.700212in}}%
\pgfpathlineto{\pgfqpoint{2.838204in}{0.700813in}}%
\pgfpathlineto{\pgfqpoint{2.839589in}{0.704881in}}%
\pgfpathlineto{\pgfqpoint{2.839763in}{0.702631in}}%
\pgfpathlineto{\pgfqpoint{2.840109in}{0.700988in}}%
\pgfpathlineto{\pgfqpoint{2.840802in}{0.705651in}}%
\pgfpathlineto{\pgfqpoint{2.841668in}{0.710289in}}%
\pgfpathlineto{\pgfqpoint{2.842014in}{0.703550in}}%
\pgfpathlineto{\pgfqpoint{2.842534in}{0.697831in}}%
\pgfpathlineto{\pgfqpoint{2.843747in}{0.700666in}}%
\pgfpathlineto{\pgfqpoint{2.843920in}{0.701281in}}%
\pgfpathlineto{\pgfqpoint{2.844093in}{0.698586in}}%
\pgfpathlineto{\pgfqpoint{2.846345in}{0.681923in}}%
\pgfpathlineto{\pgfqpoint{2.847731in}{0.675894in}}%
\pgfpathlineto{\pgfqpoint{2.848250in}{0.677493in}}%
\pgfpathlineto{\pgfqpoint{2.848424in}{0.677431in}}%
\pgfpathlineto{\pgfqpoint{2.848597in}{0.678349in}}%
\pgfpathlineto{\pgfqpoint{2.851368in}{0.692273in}}%
\pgfpathlineto{\pgfqpoint{2.851542in}{0.689712in}}%
\pgfpathlineto{\pgfqpoint{2.852234in}{0.688230in}}%
\pgfpathlineto{\pgfqpoint{2.853101in}{0.674673in}}%
\pgfpathlineto{\pgfqpoint{2.855006in}{0.664863in}}%
\pgfpathlineto{\pgfqpoint{2.855872in}{0.660706in}}%
\pgfpathlineto{\pgfqpoint{2.856219in}{0.667439in}}%
\pgfpathlineto{\pgfqpoint{2.858644in}{0.683040in}}%
\pgfpathlineto{\pgfqpoint{2.858817in}{0.682311in}}%
\pgfpathlineto{\pgfqpoint{2.858990in}{0.682326in}}%
\pgfpathlineto{\pgfqpoint{2.859510in}{0.687867in}}%
\pgfpathlineto{\pgfqpoint{2.860029in}{0.679144in}}%
\pgfpathlineto{\pgfqpoint{2.860376in}{0.680046in}}%
\pgfpathlineto{\pgfqpoint{2.860896in}{0.674756in}}%
\pgfpathlineto{\pgfqpoint{2.861588in}{0.679907in}}%
\pgfpathlineto{\pgfqpoint{2.862801in}{0.688223in}}%
\pgfpathlineto{\pgfqpoint{2.863667in}{0.685789in}}%
\pgfpathlineto{\pgfqpoint{2.865746in}{0.676586in}}%
\pgfpathlineto{\pgfqpoint{2.868171in}{0.691081in}}%
\pgfpathlineto{\pgfqpoint{2.868690in}{0.689739in}}%
\pgfpathlineto{\pgfqpoint{2.869903in}{0.689240in}}%
\pgfpathlineto{\pgfqpoint{2.871116in}{0.697269in}}%
\pgfpathlineto{\pgfqpoint{2.871462in}{0.693486in}}%
\pgfpathlineto{\pgfqpoint{2.872155in}{0.697641in}}%
\pgfpathlineto{\pgfqpoint{2.872674in}{0.696668in}}%
\pgfpathlineto{\pgfqpoint{2.873021in}{0.697633in}}%
\pgfpathlineto{\pgfqpoint{2.873714in}{0.695436in}}%
\pgfpathlineto{\pgfqpoint{2.873887in}{0.694198in}}%
\pgfpathlineto{\pgfqpoint{2.874407in}{0.699470in}}%
\pgfpathlineto{\pgfqpoint{2.875273in}{0.705663in}}%
\pgfpathlineto{\pgfqpoint{2.876139in}{0.704688in}}%
\pgfpathlineto{\pgfqpoint{2.882375in}{0.678483in}}%
\pgfpathlineto{\pgfqpoint{2.882721in}{0.679485in}}%
\pgfpathlineto{\pgfqpoint{2.885146in}{0.699241in}}%
\pgfpathlineto{\pgfqpoint{2.886532in}{0.694529in}}%
\pgfpathlineto{\pgfqpoint{2.886879in}{0.690339in}}%
\pgfpathlineto{\pgfqpoint{2.887745in}{0.696436in}}%
\pgfpathlineto{\pgfqpoint{2.887918in}{0.696076in}}%
\pgfpathlineto{\pgfqpoint{2.888091in}{0.697032in}}%
\pgfpathlineto{\pgfqpoint{2.888611in}{0.693383in}}%
\pgfpathlineto{\pgfqpoint{2.890516in}{0.686889in}}%
\pgfpathlineto{\pgfqpoint{2.890863in}{0.685916in}}%
\pgfpathlineto{\pgfqpoint{2.891382in}{0.688620in}}%
\pgfpathlineto{\pgfqpoint{2.892941in}{0.692959in}}%
\pgfpathlineto{\pgfqpoint{2.893288in}{0.692408in}}%
\pgfpathlineto{\pgfqpoint{2.893634in}{0.688238in}}%
\pgfpathlineto{\pgfqpoint{2.893807in}{0.688960in}}%
\pgfpathlineto{\pgfqpoint{2.894327in}{0.681620in}}%
\pgfpathlineto{\pgfqpoint{2.895540in}{0.683974in}}%
\pgfpathlineto{\pgfqpoint{2.895886in}{0.686436in}}%
\pgfpathlineto{\pgfqpoint{2.896232in}{0.682466in}}%
\pgfpathlineto{\pgfqpoint{2.898484in}{0.668819in}}%
\pgfpathlineto{\pgfqpoint{2.900736in}{0.681777in}}%
\pgfpathlineto{\pgfqpoint{2.902295in}{0.677233in}}%
\pgfpathlineto{\pgfqpoint{2.902468in}{0.678169in}}%
\pgfpathlineto{\pgfqpoint{2.902988in}{0.680363in}}%
\pgfpathlineto{\pgfqpoint{2.903508in}{0.675364in}}%
\pgfpathlineto{\pgfqpoint{2.903681in}{0.673390in}}%
\pgfpathlineto{\pgfqpoint{2.904547in}{0.679685in}}%
\pgfpathlineto{\pgfqpoint{2.905933in}{0.687201in}}%
\pgfpathlineto{\pgfqpoint{2.906626in}{0.682994in}}%
\pgfpathlineto{\pgfqpoint{2.908704in}{0.674458in}}%
\pgfpathlineto{\pgfqpoint{2.908878in}{0.674784in}}%
\pgfpathlineto{\pgfqpoint{2.909397in}{0.673071in}}%
\pgfpathlineto{\pgfqpoint{2.909744in}{0.669068in}}%
\pgfpathlineto{\pgfqpoint{2.910263in}{0.678712in}}%
\pgfpathlineto{\pgfqpoint{2.910783in}{0.672410in}}%
\pgfpathlineto{\pgfqpoint{2.913554in}{0.688567in}}%
\pgfpathlineto{\pgfqpoint{2.913901in}{0.684340in}}%
\pgfpathlineto{\pgfqpoint{2.914074in}{0.685216in}}%
\pgfpathlineto{\pgfqpoint{2.914594in}{0.683099in}}%
\pgfpathlineto{\pgfqpoint{2.917019in}{0.667897in}}%
\pgfpathlineto{\pgfqpoint{2.917365in}{0.665209in}}%
\pgfpathlineto{\pgfqpoint{2.918058in}{0.670874in}}%
\pgfpathlineto{\pgfqpoint{2.918405in}{0.674153in}}%
\pgfpathlineto{\pgfqpoint{2.919444in}{0.671487in}}%
\pgfpathlineto{\pgfqpoint{2.919790in}{0.668987in}}%
\pgfpathlineto{\pgfqpoint{2.920483in}{0.672269in}}%
\pgfpathlineto{\pgfqpoint{2.920657in}{0.672102in}}%
\pgfpathlineto{\pgfqpoint{2.921696in}{0.679013in}}%
\pgfpathlineto{\pgfqpoint{2.922042in}{0.674131in}}%
\pgfpathlineto{\pgfqpoint{2.922389in}{0.669922in}}%
\pgfpathlineto{\pgfqpoint{2.923255in}{0.675021in}}%
\pgfpathlineto{\pgfqpoint{2.923601in}{0.673780in}}%
\pgfpathlineto{\pgfqpoint{2.924641in}{0.677565in}}%
\pgfpathlineto{\pgfqpoint{2.923948in}{0.672582in}}%
\pgfpathlineto{\pgfqpoint{2.924987in}{0.675016in}}%
\pgfpathlineto{\pgfqpoint{2.925160in}{0.673069in}}%
\pgfpathlineto{\pgfqpoint{2.925853in}{0.678336in}}%
\pgfpathlineto{\pgfqpoint{2.926373in}{0.675532in}}%
\pgfpathlineto{\pgfqpoint{2.928798in}{0.686978in}}%
\pgfpathlineto{\pgfqpoint{2.929318in}{0.684758in}}%
\pgfpathlineto{\pgfqpoint{2.931223in}{0.674442in}}%
\pgfpathlineto{\pgfqpoint{2.931569in}{0.676453in}}%
\pgfpathlineto{\pgfqpoint{2.932609in}{0.680411in}}%
\pgfpathlineto{\pgfqpoint{2.933302in}{0.679374in}}%
\pgfpathlineto{\pgfqpoint{2.933475in}{0.679023in}}%
\pgfpathlineto{\pgfqpoint{2.933821in}{0.681505in}}%
\pgfpathlineto{\pgfqpoint{2.934168in}{0.682603in}}%
\pgfpathlineto{\pgfqpoint{2.934341in}{0.684571in}}%
\pgfpathlineto{\pgfqpoint{2.935034in}{0.676611in}}%
\pgfpathlineto{\pgfqpoint{2.935553in}{0.674518in}}%
\pgfpathlineto{\pgfqpoint{2.936246in}{0.676828in}}%
\pgfpathlineto{\pgfqpoint{2.936593in}{0.675725in}}%
\pgfpathlineto{\pgfqpoint{2.937459in}{0.673404in}}%
\pgfpathlineto{\pgfqpoint{2.937979in}{0.676241in}}%
\pgfpathlineto{\pgfqpoint{2.938325in}{0.674788in}}%
\pgfpathlineto{\pgfqpoint{2.942655in}{0.690618in}}%
\pgfpathlineto{\pgfqpoint{2.943522in}{0.695651in}}%
\pgfpathlineto{\pgfqpoint{2.944388in}{0.692680in}}%
\pgfpathlineto{\pgfqpoint{2.944907in}{0.695450in}}%
\pgfpathlineto{\pgfqpoint{2.945947in}{0.692644in}}%
\pgfpathlineto{\pgfqpoint{2.946466in}{0.693323in}}%
\pgfpathlineto{\pgfqpoint{2.946640in}{0.692387in}}%
\pgfpathlineto{\pgfqpoint{2.948545in}{0.686454in}}%
\pgfpathlineto{\pgfqpoint{2.948891in}{0.688389in}}%
\pgfpathlineto{\pgfqpoint{2.949758in}{0.683554in}}%
\pgfpathlineto{\pgfqpoint{2.950104in}{0.684704in}}%
\pgfpathlineto{\pgfqpoint{2.950277in}{0.682401in}}%
\pgfpathlineto{\pgfqpoint{2.950624in}{0.682403in}}%
\pgfpathlineto{\pgfqpoint{2.952183in}{0.659902in}}%
\pgfpathlineto{\pgfqpoint{2.952702in}{0.671569in}}%
\pgfpathlineto{\pgfqpoint{2.953568in}{0.683727in}}%
\pgfpathlineto{\pgfqpoint{2.954434in}{0.681031in}}%
\pgfpathlineto{\pgfqpoint{2.954954in}{0.680020in}}%
\pgfpathlineto{\pgfqpoint{2.955647in}{0.684381in}}%
\pgfpathlineto{\pgfqpoint{2.957033in}{0.667026in}}%
\pgfpathlineto{\pgfqpoint{2.957726in}{0.668336in}}%
\pgfpathlineto{\pgfqpoint{2.958245in}{0.668321in}}%
\pgfpathlineto{\pgfqpoint{2.958419in}{0.669095in}}%
\pgfpathlineto{\pgfqpoint{2.958592in}{0.668990in}}%
\pgfpathlineto{\pgfqpoint{2.960151in}{0.678618in}}%
\pgfpathlineto{\pgfqpoint{2.960497in}{0.676986in}}%
\pgfpathlineto{\pgfqpoint{2.963615in}{0.698203in}}%
\pgfpathlineto{\pgfqpoint{2.964654in}{0.693990in}}%
\pgfpathlineto{\pgfqpoint{2.967080in}{0.685037in}}%
\pgfpathlineto{\pgfqpoint{2.967253in}{0.685105in}}%
\pgfpathlineto{\pgfqpoint{2.967426in}{0.684908in}}%
\pgfpathlineto{\pgfqpoint{2.967599in}{0.685812in}}%
\pgfpathlineto{\pgfqpoint{2.970371in}{0.696391in}}%
\pgfpathlineto{\pgfqpoint{2.970544in}{0.695526in}}%
\pgfpathlineto{\pgfqpoint{2.971064in}{0.696081in}}%
\pgfpathlineto{\pgfqpoint{2.972449in}{0.691218in}}%
\pgfpathlineto{\pgfqpoint{2.974528in}{0.700296in}}%
\pgfpathlineto{\pgfqpoint{2.973315in}{0.687911in}}%
\pgfpathlineto{\pgfqpoint{2.974874in}{0.698483in}}%
\pgfpathlineto{\pgfqpoint{2.976087in}{0.692846in}}%
\pgfpathlineto{\pgfqpoint{2.976433in}{0.694337in}}%
\pgfpathlineto{\pgfqpoint{2.977473in}{0.703519in}}%
\pgfpathlineto{\pgfqpoint{2.978339in}{0.699392in}}%
\pgfpathlineto{\pgfqpoint{2.979378in}{0.693417in}}%
\pgfpathlineto{\pgfqpoint{2.980071in}{0.698952in}}%
\pgfpathlineto{\pgfqpoint{2.980418in}{0.700181in}}%
\pgfpathlineto{\pgfqpoint{2.981110in}{0.706929in}}%
\pgfpathlineto{\pgfqpoint{2.981630in}{0.699145in}}%
\pgfpathlineto{\pgfqpoint{2.982323in}{0.696757in}}%
\pgfpathlineto{\pgfqpoint{2.982669in}{0.699688in}}%
\pgfpathlineto{\pgfqpoint{2.985094in}{0.708775in}}%
\pgfpathlineto{\pgfqpoint{2.989079in}{0.694023in}}%
\pgfpathlineto{\pgfqpoint{2.989252in}{0.697139in}}%
\pgfpathlineto{\pgfqpoint{2.989598in}{0.697572in}}%
\pgfpathlineto{\pgfqpoint{2.989945in}{0.695056in}}%
\pgfpathlineto{\pgfqpoint{2.992023in}{0.684772in}}%
\pgfpathlineto{\pgfqpoint{2.993236in}{0.689544in}}%
\pgfpathlineto{\pgfqpoint{2.993409in}{0.691001in}}%
\pgfpathlineto{\pgfqpoint{2.994102in}{0.684122in}}%
\pgfpathlineto{\pgfqpoint{2.995488in}{0.692091in}}%
\pgfpathlineto{\pgfqpoint{2.997220in}{0.696376in}}%
\pgfpathlineto{\pgfqpoint{2.997393in}{0.695149in}}%
\pgfpathlineto{\pgfqpoint{2.997740in}{0.691539in}}%
\pgfpathlineto{\pgfqpoint{2.998432in}{0.698157in}}%
\pgfpathlineto{\pgfqpoint{3.000858in}{0.708711in}}%
\pgfpathlineto{\pgfqpoint{3.001204in}{0.710276in}}%
\pgfpathlineto{\pgfqpoint{3.001897in}{0.706311in}}%
\pgfpathlineto{\pgfqpoint{3.002070in}{0.704335in}}%
\pgfpathlineto{\pgfqpoint{3.002936in}{0.711590in}}%
\pgfpathlineto{\pgfqpoint{3.003802in}{0.719229in}}%
\pgfpathlineto{\pgfqpoint{3.004668in}{0.718638in}}%
\pgfpathlineto{\pgfqpoint{3.005361in}{0.723444in}}%
\pgfpathlineto{\pgfqpoint{3.006227in}{0.719987in}}%
\pgfpathlineto{\pgfqpoint{3.006574in}{0.718558in}}%
\pgfpathlineto{\pgfqpoint{3.006920in}{0.722366in}}%
\pgfpathlineto{\pgfqpoint{3.008479in}{0.736043in}}%
\pgfpathlineto{\pgfqpoint{3.008999in}{0.733240in}}%
\pgfpathlineto{\pgfqpoint{3.011597in}{0.727262in}}%
\pgfpathlineto{\pgfqpoint{3.012117in}{0.724852in}}%
\pgfpathlineto{\pgfqpoint{3.012636in}{0.728295in}}%
\pgfpathlineto{\pgfqpoint{3.013156in}{0.731561in}}%
\pgfpathlineto{\pgfqpoint{3.014195in}{0.729662in}}%
\pgfpathlineto{\pgfqpoint{3.016274in}{0.726856in}}%
\pgfpathlineto{\pgfqpoint{3.016967in}{0.726253in}}%
\pgfpathlineto{\pgfqpoint{3.017313in}{0.727950in}}%
\pgfpathlineto{\pgfqpoint{3.020778in}{0.745901in}}%
\pgfpathlineto{\pgfqpoint{3.021990in}{0.740996in}}%
\pgfpathlineto{\pgfqpoint{3.022164in}{0.739919in}}%
\pgfpathlineto{\pgfqpoint{3.023030in}{0.741232in}}%
\pgfpathlineto{\pgfqpoint{3.023376in}{0.741165in}}%
\pgfpathlineto{\pgfqpoint{3.024069in}{0.743071in}}%
\pgfpathlineto{\pgfqpoint{3.024935in}{0.741843in}}%
\pgfpathlineto{\pgfqpoint{3.029092in}{0.722692in}}%
\pgfpathlineto{\pgfqpoint{3.029439in}{0.719357in}}%
\pgfpathlineto{\pgfqpoint{3.030305in}{0.724601in}}%
\pgfpathlineto{\pgfqpoint{3.033423in}{0.739172in}}%
\pgfpathlineto{\pgfqpoint{3.033769in}{0.736909in}}%
\pgfpathlineto{\pgfqpoint{3.037927in}{0.721688in}}%
\pgfpathlineto{\pgfqpoint{3.038793in}{0.728108in}}%
\pgfpathlineto{\pgfqpoint{3.039486in}{0.722903in}}%
\pgfpathlineto{\pgfqpoint{3.039832in}{0.719190in}}%
\pgfpathlineto{\pgfqpoint{3.040698in}{0.725111in}}%
\pgfpathlineto{\pgfqpoint{3.040871in}{0.724988in}}%
\pgfpathlineto{\pgfqpoint{3.041218in}{0.726561in}}%
\pgfpathlineto{\pgfqpoint{3.042604in}{0.734114in}}%
\pgfpathlineto{\pgfqpoint{3.042777in}{0.736399in}}%
\pgfpathlineto{\pgfqpoint{3.043643in}{0.729686in}}%
\pgfpathlineto{\pgfqpoint{3.045375in}{0.726945in}}%
\pgfpathlineto{\pgfqpoint{3.047800in}{0.717452in}}%
\pgfpathlineto{\pgfqpoint{3.048320in}{0.715002in}}%
\pgfpathlineto{\pgfqpoint{3.048840in}{0.718918in}}%
\pgfpathlineto{\pgfqpoint{3.049706in}{0.716945in}}%
\pgfpathlineto{\pgfqpoint{3.050572in}{0.720319in}}%
\pgfpathlineto{\pgfqpoint{3.052997in}{0.707472in}}%
\pgfpathlineto{\pgfqpoint{3.053170in}{0.707694in}}%
\pgfpathlineto{\pgfqpoint{3.053343in}{0.705879in}}%
\pgfpathlineto{\pgfqpoint{3.054729in}{0.701048in}}%
\pgfpathlineto{\pgfqpoint{3.055249in}{0.701350in}}%
\pgfpathlineto{\pgfqpoint{3.055595in}{0.700338in}}%
\pgfpathlineto{\pgfqpoint{3.055768in}{0.699012in}}%
\pgfpathlineto{\pgfqpoint{3.056808in}{0.701785in}}%
\pgfpathlineto{\pgfqpoint{3.056981in}{0.702718in}}%
\pgfpathlineto{\pgfqpoint{3.057847in}{0.699038in}}%
\pgfpathlineto{\pgfqpoint{3.059752in}{0.696770in}}%
\pgfpathlineto{\pgfqpoint{3.060272in}{0.699595in}}%
\pgfpathlineto{\pgfqpoint{3.060965in}{0.696819in}}%
\pgfpathlineto{\pgfqpoint{3.062178in}{0.692221in}}%
\pgfpathlineto{\pgfqpoint{3.062697in}{0.694038in}}%
\pgfpathlineto{\pgfqpoint{3.063044in}{0.694796in}}%
\pgfpathlineto{\pgfqpoint{3.063390in}{0.691853in}}%
\pgfpathlineto{\pgfqpoint{3.063563in}{0.690597in}}%
\pgfpathlineto{\pgfqpoint{3.064083in}{0.697275in}}%
\pgfpathlineto{\pgfqpoint{3.065988in}{0.713147in}}%
\pgfpathlineto{\pgfqpoint{3.066162in}{0.711968in}}%
\pgfpathlineto{\pgfqpoint{3.066681in}{0.713634in}}%
\pgfpathlineto{\pgfqpoint{3.067028in}{0.710568in}}%
\pgfpathlineto{\pgfqpoint{3.068587in}{0.684518in}}%
\pgfpathlineto{\pgfqpoint{3.069106in}{0.691154in}}%
\pgfpathlineto{\pgfqpoint{3.069453in}{0.696355in}}%
\pgfpathlineto{\pgfqpoint{3.070146in}{0.686837in}}%
\pgfpathlineto{\pgfqpoint{3.071531in}{0.674679in}}%
\pgfpathlineto{\pgfqpoint{3.072051in}{0.677702in}}%
\pgfpathlineto{\pgfqpoint{3.074130in}{0.690495in}}%
\pgfpathlineto{\pgfqpoint{3.075515in}{0.702462in}}%
\pgfpathlineto{\pgfqpoint{3.076728in}{0.699723in}}%
\pgfpathlineto{\pgfqpoint{3.077594in}{0.693319in}}%
\pgfpathlineto{\pgfqpoint{3.078287in}{0.698291in}}%
\pgfpathlineto{\pgfqpoint{3.078633in}{0.701380in}}%
\pgfpathlineto{\pgfqpoint{3.079500in}{0.697438in}}%
\pgfpathlineto{\pgfqpoint{3.079673in}{0.698187in}}%
\pgfpathlineto{\pgfqpoint{3.079846in}{0.697104in}}%
\pgfpathlineto{\pgfqpoint{3.080712in}{0.700970in}}%
\pgfpathlineto{\pgfqpoint{3.082271in}{0.705233in}}%
\pgfpathlineto{\pgfqpoint{3.081232in}{0.699642in}}%
\pgfpathlineto{\pgfqpoint{3.082791in}{0.704290in}}%
\pgfpathlineto{\pgfqpoint{3.083657in}{0.693659in}}%
\pgfpathlineto{\pgfqpoint{3.084869in}{0.698916in}}%
\pgfpathlineto{\pgfqpoint{3.085909in}{0.699912in}}%
\pgfpathlineto{\pgfqpoint{3.085389in}{0.697896in}}%
\pgfpathlineto{\pgfqpoint{3.086082in}{0.699777in}}%
\pgfpathlineto{\pgfqpoint{3.088334in}{0.690021in}}%
\pgfpathlineto{\pgfqpoint{3.090932in}{0.702502in}}%
\pgfpathlineto{\pgfqpoint{3.089373in}{0.689084in}}%
\pgfpathlineto{\pgfqpoint{3.091452in}{0.701406in}}%
\pgfpathlineto{\pgfqpoint{3.092318in}{0.692482in}}%
\pgfpathlineto{\pgfqpoint{3.093184in}{0.695565in}}%
\pgfpathlineto{\pgfqpoint{3.094223in}{0.702191in}}%
\pgfpathlineto{\pgfqpoint{3.094743in}{0.695511in}}%
\pgfpathlineto{\pgfqpoint{3.096129in}{0.678618in}}%
\pgfpathlineto{\pgfqpoint{3.097168in}{0.680730in}}%
\pgfpathlineto{\pgfqpoint{3.099593in}{0.696945in}}%
\pgfpathlineto{\pgfqpoint{3.099766in}{0.695574in}}%
\pgfpathlineto{\pgfqpoint{3.100806in}{0.694867in}}%
\pgfpathlineto{\pgfqpoint{3.100286in}{0.695946in}}%
\pgfpathlineto{\pgfqpoint{3.100979in}{0.695529in}}%
\pgfpathlineto{\pgfqpoint{3.101325in}{0.697855in}}%
\pgfpathlineto{\pgfqpoint{3.101845in}{0.691364in}}%
\pgfpathlineto{\pgfqpoint{3.102538in}{0.696769in}}%
\pgfpathlineto{\pgfqpoint{3.105656in}{0.705272in}}%
\pgfpathlineto{\pgfqpoint{3.105829in}{0.703707in}}%
\pgfpathlineto{\pgfqpoint{3.109467in}{0.685384in}}%
\pgfpathlineto{\pgfqpoint{3.109986in}{0.686786in}}%
\pgfpathlineto{\pgfqpoint{3.112758in}{0.699429in}}%
\pgfpathlineto{\pgfqpoint{3.116915in}{0.680740in}}%
\pgfpathlineto{\pgfqpoint{3.119340in}{0.694566in}}%
\pgfpathlineto{\pgfqpoint{3.120033in}{0.693812in}}%
\pgfpathlineto{\pgfqpoint{3.121246in}{0.690840in}}%
\pgfpathlineto{\pgfqpoint{3.121592in}{0.693447in}}%
\pgfpathlineto{\pgfqpoint{3.121939in}{0.694338in}}%
\pgfpathlineto{\pgfqpoint{3.122285in}{0.693119in}}%
\pgfpathlineto{\pgfqpoint{3.123844in}{0.672929in}}%
\pgfpathlineto{\pgfqpoint{3.125056in}{0.675339in}}%
\pgfpathlineto{\pgfqpoint{3.127828in}{0.693205in}}%
\pgfpathlineto{\pgfqpoint{3.128001in}{0.693974in}}%
\pgfpathlineto{\pgfqpoint{3.128694in}{0.690317in}}%
\pgfpathlineto{\pgfqpoint{3.129907in}{0.692119in}}%
\pgfpathlineto{\pgfqpoint{3.129214in}{0.688940in}}%
\pgfpathlineto{\pgfqpoint{3.130080in}{0.691182in}}%
\pgfpathlineto{\pgfqpoint{3.131985in}{0.683882in}}%
\pgfpathlineto{\pgfqpoint{3.132332in}{0.684901in}}%
\pgfpathlineto{\pgfqpoint{3.133371in}{0.689687in}}%
\pgfpathlineto{\pgfqpoint{3.133891in}{0.687264in}}%
\pgfpathlineto{\pgfqpoint{3.134757in}{0.677140in}}%
\pgfpathlineto{\pgfqpoint{3.135450in}{0.682496in}}%
\pgfpathlineto{\pgfqpoint{3.137182in}{0.693694in}}%
\pgfpathlineto{\pgfqpoint{3.137702in}{0.692345in}}%
\pgfpathlineto{\pgfqpoint{3.139261in}{0.677575in}}%
\pgfpathlineto{\pgfqpoint{3.140127in}{0.668582in}}%
\pgfpathlineto{\pgfqpoint{3.140993in}{0.674666in}}%
\pgfpathlineto{\pgfqpoint{3.145843in}{0.697743in}}%
\pgfpathlineto{\pgfqpoint{3.146016in}{0.697611in}}%
\pgfpathlineto{\pgfqpoint{3.148095in}{0.684382in}}%
\pgfpathlineto{\pgfqpoint{3.148268in}{0.684644in}}%
\pgfpathlineto{\pgfqpoint{3.149827in}{0.698269in}}%
\pgfpathlineto{\pgfqpoint{3.150347in}{0.695342in}}%
\pgfpathlineto{\pgfqpoint{3.151213in}{0.679517in}}%
\pgfpathlineto{\pgfqpoint{3.152425in}{0.682210in}}%
\pgfpathlineto{\pgfqpoint{3.155197in}{0.665728in}}%
\pgfpathlineto{\pgfqpoint{3.155543in}{0.670802in}}%
\pgfpathlineto{\pgfqpoint{3.156236in}{0.675329in}}%
\pgfpathlineto{\pgfqpoint{3.156756in}{0.670564in}}%
\pgfpathlineto{\pgfqpoint{3.156929in}{0.668059in}}%
\pgfpathlineto{\pgfqpoint{3.157968in}{0.672724in}}%
\pgfpathlineto{\pgfqpoint{3.158661in}{0.676661in}}%
\pgfpathlineto{\pgfqpoint{3.159181in}{0.671207in}}%
\pgfpathlineto{\pgfqpoint{3.160047in}{0.666998in}}%
\pgfpathlineto{\pgfqpoint{3.160567in}{0.671473in}}%
\pgfpathlineto{\pgfqpoint{3.161952in}{0.670152in}}%
\pgfpathlineto{\pgfqpoint{3.161433in}{0.672608in}}%
\pgfpathlineto{\pgfqpoint{3.162126in}{0.671167in}}%
\pgfpathlineto{\pgfqpoint{3.164551in}{0.681727in}}%
\pgfpathlineto{\pgfqpoint{3.164897in}{0.679739in}}%
\pgfpathlineto{\pgfqpoint{3.165417in}{0.682895in}}%
\pgfpathlineto{\pgfqpoint{3.165936in}{0.689067in}}%
\pgfpathlineto{\pgfqpoint{3.166283in}{0.694858in}}%
\pgfpathlineto{\pgfqpoint{3.166456in}{0.692791in}}%
\pgfusepath{stroke}%
\end{pgfscope}%
\begin{pgfscope}%
\pgfpathrectangle{\pgfqpoint{0.568671in}{0.451277in}}{\pgfqpoint{2.598305in}{0.798874in}}%
\pgfusepath{clip}%
\pgfsetrectcap%
\pgfsetroundjoin%
\pgfsetlinewidth{1.505625pt}%
\definecolor{currentstroke}{rgb}{0.121569,0.466667,0.705882}%
\pgfsetstrokecolor{currentstroke}%
\pgfsetdash{}{0pt}%
\pgfpathmoveto{\pgfqpoint{0.651109in}{1.264040in}}%
\pgfpathlineto{\pgfqpoint{0.661864in}{1.247212in}}%
\pgfpathlineto{\pgfqpoint{0.680745in}{1.206622in}}%
\pgfpathlineto{\pgfqpoint{0.686461in}{1.191396in}}%
\pgfpathlineto{\pgfqpoint{0.699626in}{1.163489in}}%
\pgfpathlineto{\pgfqpoint{0.714523in}{1.135311in}}%
\pgfpathlineto{\pgfqpoint{0.733577in}{1.104817in}}%
\pgfpathlineto{\pgfqpoint{0.744143in}{1.091528in}}%
\pgfpathlineto{\pgfqpoint{0.746568in}{1.083566in}}%
\pgfpathlineto{\pgfqpoint{0.750726in}{1.073720in}}%
\pgfpathlineto{\pgfqpoint{0.754537in}{1.068741in}}%
\pgfpathlineto{\pgfqpoint{0.781212in}{1.040315in}}%
\pgfpathlineto{\pgfqpoint{0.783638in}{1.037764in}}%
\pgfpathlineto{\pgfqpoint{0.788834in}{1.028258in}}%
\pgfpathlineto{\pgfqpoint{0.792991in}{1.027905in}}%
\pgfpathlineto{\pgfqpoint{0.795243in}{1.025590in}}%
\pgfpathlineto{\pgfqpoint{0.798015in}{1.023594in}}%
\pgfpathlineto{\pgfqpoint{0.804770in}{1.022014in}}%
\pgfpathlineto{\pgfqpoint{0.807888in}{1.020659in}}%
\pgfpathlineto{\pgfqpoint{0.809274in}{1.018802in}}%
\pgfpathlineto{\pgfqpoint{0.812219in}{1.016288in}}%
\pgfpathlineto{\pgfqpoint{0.817242in}{1.010934in}}%
\pgfpathlineto{\pgfqpoint{0.817415in}{1.011060in}}%
\pgfpathlineto{\pgfqpoint{0.820880in}{1.011730in}}%
\pgfpathlineto{\pgfqpoint{0.822612in}{1.010211in}}%
\pgfpathlineto{\pgfqpoint{0.826423in}{1.007379in}}%
\pgfpathlineto{\pgfqpoint{0.829541in}{1.006980in}}%
\pgfpathlineto{\pgfqpoint{0.831793in}{1.004657in}}%
\pgfpathlineto{\pgfqpoint{0.835950in}{1.000291in}}%
\pgfpathlineto{\pgfqpoint{0.838548in}{0.999266in}}%
\pgfpathlineto{\pgfqpoint{0.840800in}{0.998697in}}%
\pgfpathlineto{\pgfqpoint{0.843225in}{0.995934in}}%
\pgfpathlineto{\pgfqpoint{0.847556in}{0.990529in}}%
\pgfpathlineto{\pgfqpoint{0.850501in}{0.990094in}}%
\pgfpathlineto{\pgfqpoint{0.852752in}{0.990513in}}%
\pgfpathlineto{\pgfqpoint{0.856910in}{0.984210in}}%
\pgfpathlineto{\pgfqpoint{0.859854in}{0.981772in}}%
\pgfpathlineto{\pgfqpoint{0.862280in}{0.978931in}}%
\pgfpathlineto{\pgfqpoint{0.866090in}{0.975540in}}%
\pgfpathlineto{\pgfqpoint{0.868342in}{0.975860in}}%
\pgfpathlineto{\pgfqpoint{0.870074in}{0.977614in}}%
\pgfpathlineto{\pgfqpoint{0.870594in}{0.977001in}}%
\pgfpathlineto{\pgfqpoint{0.873366in}{0.975663in}}%
\pgfpathlineto{\pgfqpoint{0.877003in}{0.975035in}}%
\pgfpathlineto{\pgfqpoint{0.881334in}{0.974336in}}%
\pgfpathlineto{\pgfqpoint{0.883239in}{0.973785in}}%
\pgfpathlineto{\pgfqpoint{0.896058in}{0.958452in}}%
\pgfpathlineto{\pgfqpoint{0.903333in}{0.955898in}}%
\pgfpathlineto{\pgfqpoint{0.904372in}{0.956752in}}%
\pgfpathlineto{\pgfqpoint{0.905065in}{0.955893in}}%
\pgfpathlineto{\pgfqpoint{0.907663in}{0.954186in}}%
\pgfpathlineto{\pgfqpoint{0.913206in}{0.953787in}}%
\pgfpathlineto{\pgfqpoint{0.914939in}{0.948924in}}%
\pgfpathlineto{\pgfqpoint{0.917537in}{0.945592in}}%
\pgfpathlineto{\pgfqpoint{0.921174in}{0.944520in}}%
\pgfpathlineto{\pgfqpoint{0.927757in}{0.938381in}}%
\pgfpathlineto{\pgfqpoint{0.934859in}{0.922409in}}%
\pgfpathlineto{\pgfqpoint{0.940922in}{0.912235in}}%
\pgfpathlineto{\pgfqpoint{0.942827in}{0.910634in}}%
\pgfpathlineto{\pgfqpoint{0.947157in}{0.905260in}}%
\pgfpathlineto{\pgfqpoint{0.948890in}{0.904667in}}%
\pgfpathlineto{\pgfqpoint{0.953393in}{0.890509in}}%
\pgfpathlineto{\pgfqpoint{0.958763in}{0.871717in}}%
\pgfpathlineto{\pgfqpoint{0.964133in}{0.862471in}}%
\pgfpathlineto{\pgfqpoint{0.972794in}{0.854228in}}%
\pgfpathlineto{\pgfqpoint{0.977817in}{0.846977in}}%
\pgfpathlineto{\pgfqpoint{0.989596in}{0.830863in}}%
\pgfpathlineto{\pgfqpoint{0.990463in}{0.832129in}}%
\pgfpathlineto{\pgfqpoint{0.992541in}{0.833862in}}%
\pgfpathlineto{\pgfqpoint{0.992714in}{0.833681in}}%
\pgfpathlineto{\pgfqpoint{0.994793in}{0.827637in}}%
\pgfpathlineto{\pgfqpoint{0.997911in}{0.817832in}}%
\pgfpathlineto{\pgfqpoint{1.001549in}{0.805929in}}%
\pgfpathlineto{\pgfqpoint{1.001895in}{0.806193in}}%
\pgfpathlineto{\pgfqpoint{1.008131in}{0.814218in}}%
\pgfpathlineto{\pgfqpoint{1.008304in}{0.814098in}}%
\pgfpathlineto{\pgfqpoint{1.012808in}{0.810950in}}%
\pgfpathlineto{\pgfqpoint{1.016446in}{0.808078in}}%
\pgfpathlineto{\pgfqpoint{1.019217in}{0.805814in}}%
\pgfpathlineto{\pgfqpoint{1.019390in}{0.805923in}}%
\pgfpathlineto{\pgfqpoint{1.020776in}{0.808459in}}%
\pgfpathlineto{\pgfqpoint{1.022855in}{0.811796in}}%
\pgfpathlineto{\pgfqpoint{1.023374in}{0.811317in}}%
\pgfpathlineto{\pgfqpoint{1.036193in}{0.789355in}}%
\pgfpathlineto{\pgfqpoint{1.037059in}{0.789723in}}%
\pgfpathlineto{\pgfqpoint{1.037405in}{0.790264in}}%
\pgfpathlineto{\pgfqpoint{1.037925in}{0.790664in}}%
\pgfpathlineto{\pgfqpoint{1.038618in}{0.789417in}}%
\pgfpathlineto{\pgfqpoint{1.043295in}{0.781286in}}%
\pgfpathlineto{\pgfqpoint{1.047279in}{0.779775in}}%
\pgfpathlineto{\pgfqpoint{1.050743in}{0.770196in}}%
\pgfpathlineto{\pgfqpoint{1.051436in}{0.770654in}}%
\pgfpathlineto{\pgfqpoint{1.052649in}{0.771876in}}%
\pgfpathlineto{\pgfqpoint{1.053342in}{0.770792in}}%
\pgfpathlineto{\pgfqpoint{1.056979in}{0.764940in}}%
\pgfpathlineto{\pgfqpoint{1.058018in}{0.764756in}}%
\pgfpathlineto{\pgfqpoint{1.058192in}{0.764331in}}%
\pgfpathlineto{\pgfqpoint{1.061656in}{0.751150in}}%
\pgfpathlineto{\pgfqpoint{1.063388in}{0.754163in}}%
\pgfpathlineto{\pgfqpoint{1.067372in}{0.763203in}}%
\pgfpathlineto{\pgfqpoint{1.067892in}{0.762380in}}%
\pgfpathlineto{\pgfqpoint{1.077246in}{0.738228in}}%
\pgfpathlineto{\pgfqpoint{1.078632in}{0.736983in}}%
\pgfpathlineto{\pgfqpoint{1.080884in}{0.734235in}}%
\pgfpathlineto{\pgfqpoint{1.081230in}{0.734655in}}%
\pgfpathlineto{\pgfqpoint{1.085041in}{0.741677in}}%
\pgfpathlineto{\pgfqpoint{1.085561in}{0.740897in}}%
\pgfpathlineto{\pgfqpoint{1.086773in}{0.737823in}}%
\pgfpathlineto{\pgfqpoint{1.087639in}{0.739176in}}%
\pgfpathlineto{\pgfqpoint{1.088505in}{0.740274in}}%
\pgfpathlineto{\pgfqpoint{1.089198in}{0.739337in}}%
\pgfpathlineto{\pgfqpoint{1.094048in}{0.733323in}}%
\pgfpathlineto{\pgfqpoint{1.095088in}{0.733087in}}%
\pgfpathlineto{\pgfqpoint{1.095434in}{0.733785in}}%
\pgfpathlineto{\pgfqpoint{1.101150in}{0.741951in}}%
\pgfpathlineto{\pgfqpoint{1.101324in}{0.741821in}}%
\pgfpathlineto{\pgfqpoint{1.103056in}{0.736948in}}%
\pgfpathlineto{\pgfqpoint{1.109638in}{0.717399in}}%
\pgfpathlineto{\pgfqpoint{1.109985in}{0.717839in}}%
\pgfpathlineto{\pgfqpoint{1.111890in}{0.725474in}}%
\pgfpathlineto{\pgfqpoint{1.113103in}{0.723071in}}%
\pgfpathlineto{\pgfqpoint{1.113969in}{0.724221in}}%
\pgfpathlineto{\pgfqpoint{1.115701in}{0.728365in}}%
\pgfpathlineto{\pgfqpoint{1.116221in}{0.726999in}}%
\pgfpathlineto{\pgfqpoint{1.117953in}{0.721160in}}%
\pgfpathlineto{\pgfqpoint{1.118646in}{0.721950in}}%
\pgfpathlineto{\pgfqpoint{1.119685in}{0.724024in}}%
\pgfpathlineto{\pgfqpoint{1.120378in}{0.723113in}}%
\pgfpathlineto{\pgfqpoint{1.130251in}{0.697216in}}%
\pgfpathlineto{\pgfqpoint{1.131291in}{0.696044in}}%
\pgfpathlineto{\pgfqpoint{1.131984in}{0.696591in}}%
\pgfpathlineto{\pgfqpoint{1.133543in}{0.697487in}}%
\pgfpathlineto{\pgfqpoint{1.133889in}{0.696985in}}%
\pgfpathlineto{\pgfqpoint{1.141511in}{0.682678in}}%
\pgfpathlineto{\pgfqpoint{1.143936in}{0.686167in}}%
\pgfpathlineto{\pgfqpoint{1.146014in}{0.693058in}}%
\pgfpathlineto{\pgfqpoint{1.146534in}{0.692554in}}%
\pgfpathlineto{\pgfqpoint{1.151038in}{0.687887in}}%
\pgfpathlineto{\pgfqpoint{1.152077in}{0.689389in}}%
\pgfpathlineto{\pgfqpoint{1.152770in}{0.690140in}}%
\pgfpathlineto{\pgfqpoint{1.153290in}{0.688904in}}%
\pgfpathlineto{\pgfqpoint{1.158486in}{0.663003in}}%
\pgfpathlineto{\pgfqpoint{1.159179in}{0.664667in}}%
\pgfpathlineto{\pgfqpoint{1.159872in}{0.665793in}}%
\pgfpathlineto{\pgfqpoint{1.160392in}{0.664572in}}%
\pgfpathlineto{\pgfqpoint{1.164895in}{0.650267in}}%
\pgfpathlineto{\pgfqpoint{1.165069in}{0.650306in}}%
\pgfpathlineto{\pgfqpoint{1.165935in}{0.650867in}}%
\pgfpathlineto{\pgfqpoint{1.166454in}{0.649617in}}%
\pgfpathlineto{\pgfqpoint{1.166801in}{0.649407in}}%
\pgfpathlineto{\pgfqpoint{1.167321in}{0.650667in}}%
\pgfpathlineto{\pgfqpoint{1.173210in}{0.675327in}}%
\pgfpathlineto{\pgfqpoint{1.174076in}{0.675424in}}%
\pgfpathlineto{\pgfqpoint{1.174423in}{0.674480in}}%
\pgfpathlineto{\pgfqpoint{1.175115in}{0.673265in}}%
\pgfpathlineto{\pgfqpoint{1.175635in}{0.675469in}}%
\pgfpathlineto{\pgfqpoint{1.177714in}{0.687507in}}%
\pgfpathlineto{\pgfqpoint{1.178753in}{0.686272in}}%
\pgfpathlineto{\pgfqpoint{1.179446in}{0.685918in}}%
\pgfpathlineto{\pgfqpoint{1.179966in}{0.687141in}}%
\pgfpathlineto{\pgfqpoint{1.181351in}{0.691844in}}%
\pgfpathlineto{\pgfqpoint{1.181871in}{0.689888in}}%
\pgfpathlineto{\pgfqpoint{1.183950in}{0.678531in}}%
\pgfpathlineto{\pgfqpoint{1.184816in}{0.679754in}}%
\pgfpathlineto{\pgfqpoint{1.189319in}{0.691612in}}%
\pgfpathlineto{\pgfqpoint{1.190359in}{0.694914in}}%
\pgfpathlineto{\pgfqpoint{1.190878in}{0.693362in}}%
\pgfpathlineto{\pgfqpoint{1.193130in}{0.688614in}}%
\pgfpathlineto{\pgfqpoint{1.196595in}{0.688393in}}%
\pgfpathlineto{\pgfqpoint{1.197288in}{0.688950in}}%
\pgfpathlineto{\pgfqpoint{1.197807in}{0.688080in}}%
\pgfpathlineto{\pgfqpoint{1.200406in}{0.676172in}}%
\pgfpathlineto{\pgfqpoint{1.203697in}{0.660835in}}%
\pgfpathlineto{\pgfqpoint{1.205256in}{0.662318in}}%
\pgfpathlineto{\pgfqpoint{1.208720in}{0.657762in}}%
\pgfpathlineto{\pgfqpoint{1.209240in}{0.658184in}}%
\pgfpathlineto{\pgfqpoint{1.210452in}{0.660063in}}%
\pgfpathlineto{\pgfqpoint{1.211145in}{0.660768in}}%
\pgfpathlineto{\pgfqpoint{1.211838in}{0.659817in}}%
\pgfpathlineto{\pgfqpoint{1.212877in}{0.658588in}}%
\pgfpathlineto{\pgfqpoint{1.213570in}{0.659600in}}%
\pgfpathlineto{\pgfqpoint{1.215822in}{0.662124in}}%
\pgfpathlineto{\pgfqpoint{1.217208in}{0.668646in}}%
\pgfpathlineto{\pgfqpoint{1.217901in}{0.666080in}}%
\pgfpathlineto{\pgfqpoint{1.218940in}{0.662395in}}%
\pgfpathlineto{\pgfqpoint{1.219806in}{0.663227in}}%
\pgfpathlineto{\pgfqpoint{1.220672in}{0.662780in}}%
\pgfpathlineto{\pgfqpoint{1.220846in}{0.662466in}}%
\pgfpathlineto{\pgfqpoint{1.222231in}{0.653771in}}%
\pgfpathlineto{\pgfqpoint{1.225523in}{0.632909in}}%
\pgfpathlineto{\pgfqpoint{1.226215in}{0.634264in}}%
\pgfpathlineto{\pgfqpoint{1.231585in}{0.654166in}}%
\pgfpathlineto{\pgfqpoint{1.234184in}{0.661269in}}%
\pgfpathlineto{\pgfqpoint{1.239207in}{0.682135in}}%
\pgfpathlineto{\pgfqpoint{1.239900in}{0.678858in}}%
\pgfpathlineto{\pgfqpoint{1.243711in}{0.650926in}}%
\pgfpathlineto{\pgfqpoint{1.244404in}{0.652848in}}%
\pgfpathlineto{\pgfqpoint{1.250813in}{0.682595in}}%
\pgfpathlineto{\pgfqpoint{1.251332in}{0.682465in}}%
\pgfpathlineto{\pgfqpoint{1.253065in}{0.680600in}}%
\pgfpathlineto{\pgfqpoint{1.254104in}{0.679613in}}%
\pgfpathlineto{\pgfqpoint{1.254624in}{0.680301in}}%
\pgfpathlineto{\pgfqpoint{1.255490in}{0.681419in}}%
\pgfpathlineto{\pgfqpoint{1.256356in}{0.680534in}}%
\pgfpathlineto{\pgfqpoint{1.258261in}{0.676408in}}%
\pgfpathlineto{\pgfqpoint{1.258954in}{0.678236in}}%
\pgfpathlineto{\pgfqpoint{1.263111in}{0.700413in}}%
\pgfpathlineto{\pgfqpoint{1.264151in}{0.697255in}}%
\pgfpathlineto{\pgfqpoint{1.267269in}{0.683502in}}%
\pgfpathlineto{\pgfqpoint{1.268308in}{0.685954in}}%
\pgfpathlineto{\pgfqpoint{1.270387in}{0.692076in}}%
\pgfpathlineto{\pgfqpoint{1.271079in}{0.691052in}}%
\pgfpathlineto{\pgfqpoint{1.272812in}{0.685065in}}%
\pgfpathlineto{\pgfqpoint{1.273505in}{0.687342in}}%
\pgfpathlineto{\pgfqpoint{1.275583in}{0.696022in}}%
\pgfpathlineto{\pgfqpoint{1.276276in}{0.694338in}}%
\pgfpathlineto{\pgfqpoint{1.281126in}{0.680281in}}%
\pgfpathlineto{\pgfqpoint{1.281819in}{0.681365in}}%
\pgfpathlineto{\pgfqpoint{1.284764in}{0.688348in}}%
\pgfpathlineto{\pgfqpoint{1.285630in}{0.685857in}}%
\pgfpathlineto{\pgfqpoint{1.288575in}{0.670605in}}%
\pgfpathlineto{\pgfqpoint{1.291346in}{0.659065in}}%
\pgfpathlineto{\pgfqpoint{1.294811in}{0.639789in}}%
\pgfpathlineto{\pgfqpoint{1.295504in}{0.641840in}}%
\pgfpathlineto{\pgfqpoint{1.299488in}{0.661489in}}%
\pgfpathlineto{\pgfqpoint{1.299834in}{0.660949in}}%
\pgfpathlineto{\pgfqpoint{1.300700in}{0.660101in}}%
\pgfpathlineto{\pgfqpoint{1.301393in}{0.661267in}}%
\pgfpathlineto{\pgfqpoint{1.304511in}{0.666864in}}%
\pgfpathlineto{\pgfqpoint{1.307975in}{0.663814in}}%
\pgfpathlineto{\pgfqpoint{1.308668in}{0.665712in}}%
\pgfpathlineto{\pgfqpoint{1.310574in}{0.672050in}}%
\pgfpathlineto{\pgfqpoint{1.311440in}{0.670628in}}%
\pgfpathlineto{\pgfqpoint{1.312306in}{0.669031in}}%
\pgfpathlineto{\pgfqpoint{1.312999in}{0.670397in}}%
\pgfpathlineto{\pgfqpoint{1.317676in}{0.688763in}}%
\pgfpathlineto{\pgfqpoint{1.318715in}{0.687295in}}%
\pgfpathlineto{\pgfqpoint{1.327376in}{0.670752in}}%
\pgfpathlineto{\pgfqpoint{1.330494in}{0.669811in}}%
\pgfpathlineto{\pgfqpoint{1.333785in}{0.672435in}}%
\pgfpathlineto{\pgfqpoint{1.334998in}{0.671360in}}%
\pgfpathlineto{\pgfqpoint{1.340194in}{0.663681in}}%
\pgfpathlineto{\pgfqpoint{1.340368in}{0.663924in}}%
\pgfpathlineto{\pgfqpoint{1.342100in}{0.667773in}}%
\pgfpathlineto{\pgfqpoint{1.343139in}{0.666376in}}%
\pgfpathlineto{\pgfqpoint{1.345564in}{0.663077in}}%
\pgfpathlineto{\pgfqpoint{1.347296in}{0.657228in}}%
\pgfpathlineto{\pgfqpoint{1.347989in}{0.659678in}}%
\pgfpathlineto{\pgfqpoint{1.349029in}{0.662149in}}%
\pgfpathlineto{\pgfqpoint{1.349548in}{0.660928in}}%
\pgfpathlineto{\pgfqpoint{1.351973in}{0.652052in}}%
\pgfpathlineto{\pgfqpoint{1.353359in}{0.653220in}}%
\pgfpathlineto{\pgfqpoint{1.356997in}{0.654351in}}%
\pgfpathlineto{\pgfqpoint{1.360634in}{0.663712in}}%
\pgfpathlineto{\pgfqpoint{1.361327in}{0.663116in}}%
\pgfpathlineto{\pgfqpoint{1.361500in}{0.662609in}}%
\pgfpathlineto{\pgfqpoint{1.362193in}{0.660951in}}%
\pgfpathlineto{\pgfqpoint{1.363059in}{0.662703in}}%
\pgfpathlineto{\pgfqpoint{1.363752in}{0.661061in}}%
\pgfpathlineto{\pgfqpoint{1.366524in}{0.642774in}}%
\pgfpathlineto{\pgfqpoint{1.367736in}{0.644318in}}%
\pgfpathlineto{\pgfqpoint{1.369295in}{0.650070in}}%
\pgfpathlineto{\pgfqpoint{1.374665in}{0.675802in}}%
\pgfpathlineto{\pgfqpoint{1.375012in}{0.675754in}}%
\pgfpathlineto{\pgfqpoint{1.375358in}{0.674961in}}%
\pgfpathlineto{\pgfqpoint{1.380208in}{0.656015in}}%
\pgfpathlineto{\pgfqpoint{1.381074in}{0.657349in}}%
\pgfpathlineto{\pgfqpoint{1.383153in}{0.661630in}}%
\pgfpathlineto{\pgfqpoint{1.383846in}{0.660427in}}%
\pgfpathlineto{\pgfqpoint{1.385058in}{0.655000in}}%
\pgfpathlineto{\pgfqpoint{1.386271in}{0.649522in}}%
\pgfpathlineto{\pgfqpoint{1.386964in}{0.651813in}}%
\pgfpathlineto{\pgfqpoint{1.391121in}{0.667557in}}%
\pgfpathlineto{\pgfqpoint{1.391814in}{0.668232in}}%
\pgfpathlineto{\pgfqpoint{1.392334in}{0.667302in}}%
\pgfpathlineto{\pgfqpoint{1.393893in}{0.662673in}}%
\pgfpathlineto{\pgfqpoint{1.394932in}{0.663765in}}%
\pgfpathlineto{\pgfqpoint{1.395971in}{0.660194in}}%
\pgfpathlineto{\pgfqpoint{1.396491in}{0.658877in}}%
\pgfpathlineto{\pgfqpoint{1.397184in}{0.661585in}}%
\pgfpathlineto{\pgfqpoint{1.401688in}{0.691757in}}%
\pgfpathlineto{\pgfqpoint{1.402207in}{0.691064in}}%
\pgfpathlineto{\pgfqpoint{1.404286in}{0.685876in}}%
\pgfpathlineto{\pgfqpoint{1.404979in}{0.687567in}}%
\pgfpathlineto{\pgfqpoint{1.406884in}{0.693056in}}%
\pgfpathlineto{\pgfqpoint{1.407577in}{0.692080in}}%
\pgfpathlineto{\pgfqpoint{1.409483in}{0.686839in}}%
\pgfpathlineto{\pgfqpoint{1.410522in}{0.688133in}}%
\pgfpathlineto{\pgfqpoint{1.412254in}{0.690252in}}%
\pgfpathlineto{\pgfqpoint{1.412600in}{0.689449in}}%
\pgfpathlineto{\pgfqpoint{1.417104in}{0.677145in}}%
\pgfpathlineto{\pgfqpoint{1.417451in}{0.677712in}}%
\pgfpathlineto{\pgfqpoint{1.418836in}{0.681494in}}%
\pgfpathlineto{\pgfqpoint{1.419703in}{0.679567in}}%
\pgfpathlineto{\pgfqpoint{1.421954in}{0.672470in}}%
\pgfpathlineto{\pgfqpoint{1.423167in}{0.674874in}}%
\pgfpathlineto{\pgfqpoint{1.424033in}{0.675622in}}%
\pgfpathlineto{\pgfqpoint{1.424553in}{0.674594in}}%
\pgfpathlineto{\pgfqpoint{1.429923in}{0.660869in}}%
\pgfpathlineto{\pgfqpoint{1.431308in}{0.662097in}}%
\pgfpathlineto{\pgfqpoint{1.434080in}{0.673684in}}%
\pgfpathlineto{\pgfqpoint{1.435292in}{0.683513in}}%
\pgfpathlineto{\pgfqpoint{1.437198in}{0.746668in}}%
\pgfpathlineto{\pgfqpoint{1.440489in}{0.818550in}}%
\pgfpathlineto{\pgfqpoint{1.446552in}{0.913603in}}%
\pgfpathlineto{\pgfqpoint{1.448804in}{0.918498in}}%
\pgfpathlineto{\pgfqpoint{1.451402in}{0.919508in}}%
\pgfpathlineto{\pgfqpoint{1.452961in}{0.933477in}}%
\pgfpathlineto{\pgfqpoint{1.454866in}{0.944983in}}%
\pgfpathlineto{\pgfqpoint{1.455386in}{0.944660in}}%
\pgfpathlineto{\pgfqpoint{1.463700in}{0.929626in}}%
\pgfpathlineto{\pgfqpoint{1.469763in}{0.893568in}}%
\pgfpathlineto{\pgfqpoint{1.480849in}{0.800357in}}%
\pgfpathlineto{\pgfqpoint{1.494360in}{0.702799in}}%
\pgfpathlineto{\pgfqpoint{1.501982in}{0.667348in}}%
\pgfpathlineto{\pgfqpoint{1.503368in}{0.665218in}}%
\pgfpathlineto{\pgfqpoint{1.506659in}{0.660822in}}%
\pgfpathlineto{\pgfqpoint{1.507006in}{0.661127in}}%
\pgfpathlineto{\pgfqpoint{1.512722in}{0.675066in}}%
\pgfpathlineto{\pgfqpoint{1.514454in}{0.678216in}}%
\pgfpathlineto{\pgfqpoint{1.514974in}{0.677620in}}%
\pgfpathlineto{\pgfqpoint{1.515493in}{0.677367in}}%
\pgfpathlineto{\pgfqpoint{1.516013in}{0.678761in}}%
\pgfpathlineto{\pgfqpoint{1.520170in}{0.703849in}}%
\pgfpathlineto{\pgfqpoint{1.521556in}{0.701082in}}%
\pgfpathlineto{\pgfqpoint{1.523808in}{0.687900in}}%
\pgfpathlineto{\pgfqpoint{1.527272in}{0.662048in}}%
\pgfpathlineto{\pgfqpoint{1.528312in}{0.665485in}}%
\pgfpathlineto{\pgfqpoint{1.529871in}{0.670384in}}%
\pgfpathlineto{\pgfqpoint{1.530564in}{0.669441in}}%
\pgfpathlineto{\pgfqpoint{1.534374in}{0.659973in}}%
\pgfpathlineto{\pgfqpoint{1.535240in}{0.661973in}}%
\pgfpathlineto{\pgfqpoint{1.535760in}{0.662777in}}%
\pgfpathlineto{\pgfqpoint{1.536453in}{0.661446in}}%
\pgfpathlineto{\pgfqpoint{1.542169in}{0.643899in}}%
\pgfpathlineto{\pgfqpoint{1.542516in}{0.644117in}}%
\pgfpathlineto{\pgfqpoint{1.543901in}{0.647789in}}%
\pgfpathlineto{\pgfqpoint{1.545287in}{0.651528in}}%
\pgfpathlineto{\pgfqpoint{1.545980in}{0.650376in}}%
\pgfpathlineto{\pgfqpoint{1.548925in}{0.639208in}}%
\pgfpathlineto{\pgfqpoint{1.554121in}{0.616484in}}%
\pgfpathlineto{\pgfqpoint{1.554468in}{0.615947in}}%
\pgfpathlineto{\pgfqpoint{1.555161in}{0.617172in}}%
\pgfpathlineto{\pgfqpoint{1.558279in}{0.621265in}}%
\pgfpathlineto{\pgfqpoint{1.559838in}{0.620904in}}%
\pgfpathlineto{\pgfqpoint{1.560184in}{0.621750in}}%
\pgfpathlineto{\pgfqpoint{1.567113in}{0.640336in}}%
\pgfpathlineto{\pgfqpoint{1.571790in}{0.645833in}}%
\pgfpathlineto{\pgfqpoint{1.574561in}{0.647147in}}%
\pgfpathlineto{\pgfqpoint{1.577333in}{0.657018in}}%
\pgfpathlineto{\pgfqpoint{1.578372in}{0.654567in}}%
\pgfpathlineto{\pgfqpoint{1.589112in}{0.634156in}}%
\pgfpathlineto{\pgfqpoint{1.589632in}{0.634455in}}%
\pgfpathlineto{\pgfqpoint{1.591017in}{0.632926in}}%
\pgfpathlineto{\pgfqpoint{1.592923in}{0.626588in}}%
\pgfpathlineto{\pgfqpoint{1.593789in}{0.628755in}}%
\pgfpathlineto{\pgfqpoint{1.596387in}{0.635068in}}%
\pgfpathlineto{\pgfqpoint{1.596907in}{0.634280in}}%
\pgfpathlineto{\pgfqpoint{1.603662in}{0.606507in}}%
\pgfpathlineto{\pgfqpoint{1.604875in}{0.609123in}}%
\pgfpathlineto{\pgfqpoint{1.607993in}{0.613310in}}%
\pgfpathlineto{\pgfqpoint{1.608339in}{0.612811in}}%
\pgfpathlineto{\pgfqpoint{1.609206in}{0.612018in}}%
\pgfpathlineto{\pgfqpoint{1.609725in}{0.612946in}}%
\pgfpathlineto{\pgfqpoint{1.613363in}{0.622514in}}%
\pgfpathlineto{\pgfqpoint{1.613709in}{0.622203in}}%
\pgfpathlineto{\pgfqpoint{1.614922in}{0.616345in}}%
\pgfpathlineto{\pgfqpoint{1.621158in}{0.589367in}}%
\pgfpathlineto{\pgfqpoint{1.622370in}{0.590429in}}%
\pgfpathlineto{\pgfqpoint{1.622890in}{0.591022in}}%
\pgfpathlineto{\pgfqpoint{1.623410in}{0.589397in}}%
\pgfpathlineto{\pgfqpoint{1.626874in}{0.578816in}}%
\pgfpathlineto{\pgfqpoint{1.627047in}{0.578855in}}%
\pgfpathlineto{\pgfqpoint{1.627913in}{0.580984in}}%
\pgfpathlineto{\pgfqpoint{1.632590in}{0.602256in}}%
\pgfpathlineto{\pgfqpoint{1.633630in}{0.601939in}}%
\pgfpathlineto{\pgfqpoint{1.634496in}{0.603319in}}%
\pgfpathlineto{\pgfqpoint{1.637787in}{0.609510in}}%
\pgfpathlineto{\pgfqpoint{1.638133in}{0.609641in}}%
\pgfpathlineto{\pgfqpoint{1.638480in}{0.608671in}}%
\pgfpathlineto{\pgfqpoint{1.644023in}{0.587880in}}%
\pgfpathlineto{\pgfqpoint{1.645062in}{0.586788in}}%
\pgfpathlineto{\pgfqpoint{1.646275in}{0.583977in}}%
\pgfpathlineto{\pgfqpoint{1.647141in}{0.585099in}}%
\pgfpathlineto{\pgfqpoint{1.648700in}{0.587888in}}%
\pgfpathlineto{\pgfqpoint{1.649219in}{0.586743in}}%
\pgfpathlineto{\pgfqpoint{1.651298in}{0.581925in}}%
\pgfpathlineto{\pgfqpoint{1.651818in}{0.583072in}}%
\pgfpathlineto{\pgfqpoint{1.654243in}{0.591236in}}%
\pgfpathlineto{\pgfqpoint{1.655282in}{0.590298in}}%
\pgfpathlineto{\pgfqpoint{1.659786in}{0.586450in}}%
\pgfpathlineto{\pgfqpoint{1.660479in}{0.587882in}}%
\pgfpathlineto{\pgfqpoint{1.663250in}{0.590311in}}%
\pgfpathlineto{\pgfqpoint{1.663423in}{0.590206in}}%
\pgfpathlineto{\pgfqpoint{1.664463in}{0.589875in}}%
\pgfpathlineto{\pgfqpoint{1.664809in}{0.590744in}}%
\pgfpathlineto{\pgfqpoint{1.665849in}{0.593358in}}%
\pgfpathlineto{\pgfqpoint{1.666541in}{0.591462in}}%
\pgfpathlineto{\pgfqpoint{1.668447in}{0.583515in}}%
\pgfpathlineto{\pgfqpoint{1.669313in}{0.584133in}}%
\pgfpathlineto{\pgfqpoint{1.670699in}{0.579105in}}%
\pgfpathlineto{\pgfqpoint{1.671565in}{0.582840in}}%
\pgfpathlineto{\pgfqpoint{1.674336in}{0.593514in}}%
\pgfpathlineto{\pgfqpoint{1.674510in}{0.593458in}}%
\pgfpathlineto{\pgfqpoint{1.676069in}{0.591050in}}%
\pgfpathlineto{\pgfqpoint{1.676588in}{0.592472in}}%
\pgfpathlineto{\pgfqpoint{1.679187in}{0.602767in}}%
\pgfpathlineto{\pgfqpoint{1.680053in}{0.600780in}}%
\pgfpathlineto{\pgfqpoint{1.683863in}{0.585626in}}%
\pgfpathlineto{\pgfqpoint{1.684556in}{0.587006in}}%
\pgfpathlineto{\pgfqpoint{1.686635in}{0.606151in}}%
\pgfpathlineto{\pgfqpoint{1.688194in}{0.615736in}}%
\pgfpathlineto{\pgfqpoint{1.688887in}{0.614112in}}%
\pgfpathlineto{\pgfqpoint{1.689407in}{0.612958in}}%
\pgfpathlineto{\pgfqpoint{1.690099in}{0.614496in}}%
\pgfpathlineto{\pgfqpoint{1.691658in}{0.619201in}}%
\pgfpathlineto{\pgfqpoint{1.692351in}{0.618665in}}%
\pgfpathlineto{\pgfqpoint{1.693737in}{0.613959in}}%
\pgfpathlineto{\pgfqpoint{1.697028in}{0.592288in}}%
\pgfpathlineto{\pgfqpoint{1.697721in}{0.594700in}}%
\pgfpathlineto{\pgfqpoint{1.699627in}{0.606003in}}%
\pgfpathlineto{\pgfqpoint{1.700319in}{0.603895in}}%
\pgfpathlineto{\pgfqpoint{1.707248in}{0.576106in}}%
\pgfpathlineto{\pgfqpoint{1.709327in}{0.576983in}}%
\pgfpathlineto{\pgfqpoint{1.713657in}{0.588741in}}%
\pgfpathlineto{\pgfqpoint{1.714177in}{0.586799in}}%
\pgfpathlineto{\pgfqpoint{1.715563in}{0.581173in}}%
\pgfpathlineto{\pgfqpoint{1.716256in}{0.582663in}}%
\pgfpathlineto{\pgfqpoint{1.717122in}{0.583709in}}%
\pgfpathlineto{\pgfqpoint{1.717815in}{0.582927in}}%
\pgfpathlineto{\pgfqpoint{1.720240in}{0.581676in}}%
\pgfpathlineto{\pgfqpoint{1.720586in}{0.582103in}}%
\pgfpathlineto{\pgfqpoint{1.721452in}{0.583384in}}%
\pgfpathlineto{\pgfqpoint{1.722145in}{0.581947in}}%
\pgfpathlineto{\pgfqpoint{1.724917in}{0.575414in}}%
\pgfpathlineto{\pgfqpoint{1.725436in}{0.576236in}}%
\pgfpathlineto{\pgfqpoint{1.729940in}{0.593556in}}%
\pgfpathlineto{\pgfqpoint{1.731153in}{0.592182in}}%
\pgfpathlineto{\pgfqpoint{1.732019in}{0.593749in}}%
\pgfpathlineto{\pgfqpoint{1.732538in}{0.592396in}}%
\pgfpathlineto{\pgfqpoint{1.734963in}{0.580983in}}%
\pgfpathlineto{\pgfqpoint{1.735830in}{0.582586in}}%
\pgfpathlineto{\pgfqpoint{1.741892in}{0.599071in}}%
\pgfpathlineto{\pgfqpoint{1.743278in}{0.601217in}}%
\pgfpathlineto{\pgfqpoint{1.744144in}{0.601889in}}%
\pgfpathlineto{\pgfqpoint{1.744664in}{0.601065in}}%
\pgfpathlineto{\pgfqpoint{1.745703in}{0.599705in}}%
\pgfpathlineto{\pgfqpoint{1.746223in}{0.600539in}}%
\pgfpathlineto{\pgfqpoint{1.750553in}{0.607256in}}%
\pgfpathlineto{\pgfqpoint{1.751593in}{0.605045in}}%
\pgfpathlineto{\pgfqpoint{1.754191in}{0.593586in}}%
\pgfpathlineto{\pgfqpoint{1.755230in}{0.594968in}}%
\pgfpathlineto{\pgfqpoint{1.755750in}{0.595219in}}%
\pgfpathlineto{\pgfqpoint{1.756443in}{0.594050in}}%
\pgfpathlineto{\pgfqpoint{1.756962in}{0.593865in}}%
\pgfpathlineto{\pgfqpoint{1.757309in}{0.594647in}}%
\pgfpathlineto{\pgfqpoint{1.761813in}{0.607644in}}%
\pgfpathlineto{\pgfqpoint{1.762159in}{0.607325in}}%
\pgfpathlineto{\pgfqpoint{1.763372in}{0.600474in}}%
\pgfpathlineto{\pgfqpoint{1.767182in}{0.582804in}}%
\pgfpathlineto{\pgfqpoint{1.767529in}{0.583103in}}%
\pgfpathlineto{\pgfqpoint{1.768049in}{0.583559in}}%
\pgfpathlineto{\pgfqpoint{1.768741in}{0.582658in}}%
\pgfpathlineto{\pgfqpoint{1.774284in}{0.576003in}}%
\pgfpathlineto{\pgfqpoint{1.774804in}{0.576892in}}%
\pgfpathlineto{\pgfqpoint{1.778095in}{0.592887in}}%
\pgfpathlineto{\pgfqpoint{1.779135in}{0.589937in}}%
\pgfpathlineto{\pgfqpoint{1.780520in}{0.584682in}}%
\pgfpathlineto{\pgfqpoint{1.781040in}{0.585971in}}%
\pgfpathlineto{\pgfqpoint{1.782599in}{0.590096in}}%
\pgfpathlineto{\pgfqpoint{1.783119in}{0.589162in}}%
\pgfpathlineto{\pgfqpoint{1.786930in}{0.579247in}}%
\pgfpathlineto{\pgfqpoint{1.787449in}{0.579886in}}%
\pgfpathlineto{\pgfqpoint{1.788489in}{0.580896in}}%
\pgfpathlineto{\pgfqpoint{1.789181in}{0.579941in}}%
\pgfpathlineto{\pgfqpoint{1.789701in}{0.579551in}}%
\pgfpathlineto{\pgfqpoint{1.790567in}{0.580628in}}%
\pgfpathlineto{\pgfqpoint{1.793339in}{0.583273in}}%
\pgfpathlineto{\pgfqpoint{1.794378in}{0.581060in}}%
\pgfpathlineto{\pgfqpoint{1.796283in}{0.571834in}}%
\pgfpathlineto{\pgfqpoint{1.797150in}{0.575135in}}%
\pgfpathlineto{\pgfqpoint{1.798882in}{0.583061in}}%
\pgfpathlineto{\pgfqpoint{1.799575in}{0.581975in}}%
\pgfpathlineto{\pgfqpoint{1.800614in}{0.578468in}}%
\pgfpathlineto{\pgfqpoint{1.801307in}{0.580219in}}%
\pgfpathlineto{\pgfqpoint{1.802519in}{0.585358in}}%
\pgfpathlineto{\pgfqpoint{1.803212in}{0.583365in}}%
\pgfpathlineto{\pgfqpoint{1.804425in}{0.580624in}}%
\pgfpathlineto{\pgfqpoint{1.804944in}{0.581674in}}%
\pgfpathlineto{\pgfqpoint{1.807716in}{0.588004in}}%
\pgfpathlineto{\pgfqpoint{1.808755in}{0.586806in}}%
\pgfpathlineto{\pgfqpoint{1.809275in}{0.586853in}}%
\pgfpathlineto{\pgfqpoint{1.809795in}{0.587786in}}%
\pgfpathlineto{\pgfqpoint{1.810314in}{0.587453in}}%
\pgfpathlineto{\pgfqpoint{1.810661in}{0.586590in}}%
\pgfpathlineto{\pgfqpoint{1.811527in}{0.585714in}}%
\pgfpathlineto{\pgfqpoint{1.812047in}{0.587036in}}%
\pgfpathlineto{\pgfqpoint{1.813086in}{0.588813in}}%
\pgfpathlineto{\pgfqpoint{1.813952in}{0.588361in}}%
\pgfpathlineto{\pgfqpoint{1.815511in}{0.590403in}}%
\pgfpathlineto{\pgfqpoint{1.818629in}{0.598474in}}%
\pgfpathlineto{\pgfqpoint{1.821227in}{0.603745in}}%
\pgfpathlineto{\pgfqpoint{1.824345in}{0.612186in}}%
\pgfpathlineto{\pgfqpoint{1.824865in}{0.611432in}}%
\pgfpathlineto{\pgfqpoint{1.825558in}{0.610689in}}%
\pgfpathlineto{\pgfqpoint{1.826251in}{0.611743in}}%
\pgfpathlineto{\pgfqpoint{1.828502in}{0.617508in}}%
\pgfpathlineto{\pgfqpoint{1.829369in}{0.615598in}}%
\pgfpathlineto{\pgfqpoint{1.838376in}{0.587569in}}%
\pgfpathlineto{\pgfqpoint{1.838722in}{0.588344in}}%
\pgfpathlineto{\pgfqpoint{1.840108in}{0.591418in}}%
\pgfpathlineto{\pgfqpoint{1.840801in}{0.590177in}}%
\pgfpathlineto{\pgfqpoint{1.843053in}{0.583516in}}%
\pgfpathlineto{\pgfqpoint{1.844958in}{0.568480in}}%
\pgfpathlineto{\pgfqpoint{1.845651in}{0.569080in}}%
\pgfpathlineto{\pgfqpoint{1.849289in}{0.581702in}}%
\pgfpathlineto{\pgfqpoint{1.851194in}{0.589135in}}%
\pgfpathlineto{\pgfqpoint{1.851714in}{0.587196in}}%
\pgfpathlineto{\pgfqpoint{1.856044in}{0.554314in}}%
\pgfpathlineto{\pgfqpoint{1.857603in}{0.558021in}}%
\pgfpathlineto{\pgfqpoint{1.859855in}{0.559349in}}%
\pgfpathlineto{\pgfqpoint{1.863493in}{0.563850in}}%
\pgfpathlineto{\pgfqpoint{1.870595in}{0.583976in}}%
\pgfpathlineto{\pgfqpoint{1.871461in}{0.582929in}}%
\pgfpathlineto{\pgfqpoint{1.875445in}{0.571945in}}%
\pgfpathlineto{\pgfqpoint{1.875965in}{0.573228in}}%
\pgfpathlineto{\pgfqpoint{1.877004in}{0.574268in}}%
\pgfpathlineto{\pgfqpoint{1.877697in}{0.573325in}}%
\pgfpathlineto{\pgfqpoint{1.878043in}{0.573290in}}%
\pgfpathlineto{\pgfqpoint{1.878390in}{0.574226in}}%
\pgfpathlineto{\pgfqpoint{1.880988in}{0.583575in}}%
\pgfpathlineto{\pgfqpoint{1.881681in}{0.581318in}}%
\pgfpathlineto{\pgfqpoint{1.887051in}{0.559544in}}%
\pgfpathlineto{\pgfqpoint{1.887397in}{0.559833in}}%
\pgfpathlineto{\pgfqpoint{1.889476in}{0.566220in}}%
\pgfpathlineto{\pgfqpoint{1.890862in}{0.564502in}}%
\pgfpathlineto{\pgfqpoint{1.892940in}{0.563038in}}%
\pgfpathlineto{\pgfqpoint{1.893114in}{0.563288in}}%
\pgfpathlineto{\pgfqpoint{1.894846in}{0.568798in}}%
\pgfpathlineto{\pgfqpoint{1.897098in}{0.578096in}}%
\pgfpathlineto{\pgfqpoint{1.897791in}{0.577868in}}%
\pgfpathlineto{\pgfqpoint{1.898830in}{0.575346in}}%
\pgfpathlineto{\pgfqpoint{1.903334in}{0.560957in}}%
\pgfpathlineto{\pgfqpoint{1.905759in}{0.559596in}}%
\pgfpathlineto{\pgfqpoint{1.906971in}{0.565583in}}%
\pgfpathlineto{\pgfqpoint{1.910955in}{0.588029in}}%
\pgfpathlineto{\pgfqpoint{1.911995in}{0.590965in}}%
\pgfpathlineto{\pgfqpoint{1.912688in}{0.589850in}}%
\pgfpathlineto{\pgfqpoint{1.914073in}{0.587288in}}%
\pgfpathlineto{\pgfqpoint{1.914593in}{0.588178in}}%
\pgfpathlineto{\pgfqpoint{1.917191in}{0.590011in}}%
\pgfpathlineto{\pgfqpoint{1.918057in}{0.591624in}}%
\pgfpathlineto{\pgfqpoint{1.919097in}{0.593660in}}%
\pgfpathlineto{\pgfqpoint{1.919790in}{0.592242in}}%
\pgfpathlineto{\pgfqpoint{1.920656in}{0.590375in}}%
\pgfpathlineto{\pgfqpoint{1.921175in}{0.592607in}}%
\pgfpathlineto{\pgfqpoint{1.924293in}{0.610985in}}%
\pgfpathlineto{\pgfqpoint{1.924986in}{0.608807in}}%
\pgfpathlineto{\pgfqpoint{1.933474in}{0.572683in}}%
\pgfpathlineto{\pgfqpoint{1.933994in}{0.571974in}}%
\pgfpathlineto{\pgfqpoint{1.934686in}{0.573076in}}%
\pgfpathlineto{\pgfqpoint{1.937804in}{0.583360in}}%
\pgfpathlineto{\pgfqpoint{1.938497in}{0.580950in}}%
\pgfpathlineto{\pgfqpoint{1.943174in}{0.560253in}}%
\pgfpathlineto{\pgfqpoint{1.945426in}{0.560809in}}%
\pgfpathlineto{\pgfqpoint{1.945599in}{0.560511in}}%
\pgfpathlineto{\pgfqpoint{1.946812in}{0.553119in}}%
\pgfpathlineto{\pgfqpoint{1.948198in}{0.546071in}}%
\pgfpathlineto{\pgfqpoint{1.948891in}{0.548729in}}%
\pgfpathlineto{\pgfqpoint{1.954087in}{0.570979in}}%
\pgfpathlineto{\pgfqpoint{1.955126in}{0.575474in}}%
\pgfpathlineto{\pgfqpoint{1.957725in}{0.587225in}}%
\pgfpathlineto{\pgfqpoint{1.958244in}{0.587129in}}%
\pgfpathlineto{\pgfqpoint{1.959803in}{0.585284in}}%
\pgfpathlineto{\pgfqpoint{1.960496in}{0.584112in}}%
\pgfpathlineto{\pgfqpoint{1.961016in}{0.585450in}}%
\pgfpathlineto{\pgfqpoint{1.967425in}{0.613073in}}%
\pgfpathlineto{\pgfqpoint{1.967945in}{0.611976in}}%
\pgfpathlineto{\pgfqpoint{1.971929in}{0.601228in}}%
\pgfpathlineto{\pgfqpoint{1.974181in}{0.600087in}}%
\pgfpathlineto{\pgfqpoint{1.977299in}{0.582293in}}%
\pgfpathlineto{\pgfqpoint{1.979724in}{0.567715in}}%
\pgfpathlineto{\pgfqpoint{1.980936in}{0.568331in}}%
\pgfpathlineto{\pgfqpoint{1.983708in}{0.575570in}}%
\pgfpathlineto{\pgfqpoint{1.984920in}{0.579174in}}%
\pgfpathlineto{\pgfqpoint{1.985786in}{0.578109in}}%
\pgfpathlineto{\pgfqpoint{1.986999in}{0.576382in}}%
\pgfpathlineto{\pgfqpoint{1.987519in}{0.576975in}}%
\pgfpathlineto{\pgfqpoint{1.988904in}{0.579355in}}%
\pgfpathlineto{\pgfqpoint{1.989597in}{0.578421in}}%
\pgfpathlineto{\pgfqpoint{1.991156in}{0.577653in}}%
\pgfpathlineto{\pgfqpoint{1.991503in}{0.578174in}}%
\pgfpathlineto{\pgfqpoint{1.993408in}{0.584802in}}%
\pgfpathlineto{\pgfqpoint{1.994794in}{0.581791in}}%
\pgfpathlineto{\pgfqpoint{1.996873in}{0.578438in}}%
\pgfpathlineto{\pgfqpoint{1.997392in}{0.579295in}}%
\pgfpathlineto{\pgfqpoint{1.999298in}{0.585776in}}%
\pgfpathlineto{\pgfqpoint{2.000164in}{0.584103in}}%
\pgfpathlineto{\pgfqpoint{2.004494in}{0.577548in}}%
\pgfpathlineto{\pgfqpoint{2.006400in}{0.576497in}}%
\pgfpathlineto{\pgfqpoint{2.008305in}{0.575227in}}%
\pgfpathlineto{\pgfqpoint{2.008478in}{0.575326in}}%
\pgfpathlineto{\pgfqpoint{2.011596in}{0.579000in}}%
\pgfpathlineto{\pgfqpoint{2.012636in}{0.578129in}}%
\pgfpathlineto{\pgfqpoint{2.014195in}{0.577985in}}%
\pgfpathlineto{\pgfqpoint{2.014368in}{0.577370in}}%
\pgfpathlineto{\pgfqpoint{2.019045in}{0.550751in}}%
\pgfpathlineto{\pgfqpoint{2.020431in}{0.542005in}}%
\pgfpathlineto{\pgfqpoint{2.021123in}{0.544890in}}%
\pgfpathlineto{\pgfqpoint{2.022163in}{0.548176in}}%
\pgfpathlineto{\pgfqpoint{2.022856in}{0.546294in}}%
\pgfpathlineto{\pgfqpoint{2.024415in}{0.543469in}}%
\pgfpathlineto{\pgfqpoint{2.024761in}{0.544158in}}%
\pgfpathlineto{\pgfqpoint{2.025627in}{0.545334in}}%
\pgfpathlineto{\pgfqpoint{2.026320in}{0.544244in}}%
\pgfpathlineto{\pgfqpoint{2.026840in}{0.543681in}}%
\pgfpathlineto{\pgfqpoint{2.027359in}{0.545075in}}%
\pgfpathlineto{\pgfqpoint{2.029438in}{0.562210in}}%
\pgfpathlineto{\pgfqpoint{2.030651in}{0.555964in}}%
\pgfpathlineto{\pgfqpoint{2.033422in}{0.543614in}}%
\pgfpathlineto{\pgfqpoint{2.033769in}{0.544420in}}%
\pgfpathlineto{\pgfqpoint{2.034461in}{0.545916in}}%
\pgfpathlineto{\pgfqpoint{2.035327in}{0.545131in}}%
\pgfpathlineto{\pgfqpoint{2.036367in}{0.543417in}}%
\pgfpathlineto{\pgfqpoint{2.037060in}{0.544493in}}%
\pgfpathlineto{\pgfqpoint{2.040351in}{0.550622in}}%
\pgfpathlineto{\pgfqpoint{2.040697in}{0.550089in}}%
\pgfpathlineto{\pgfqpoint{2.042256in}{0.545837in}}%
\pgfpathlineto{\pgfqpoint{2.043122in}{0.548024in}}%
\pgfpathlineto{\pgfqpoint{2.047280in}{0.560039in}}%
\pgfpathlineto{\pgfqpoint{2.047453in}{0.559882in}}%
\pgfpathlineto{\pgfqpoint{2.049185in}{0.554809in}}%
\pgfpathlineto{\pgfqpoint{2.049878in}{0.557266in}}%
\pgfpathlineto{\pgfqpoint{2.052476in}{0.563082in}}%
\pgfpathlineto{\pgfqpoint{2.055941in}{0.569977in}}%
\pgfpathlineto{\pgfqpoint{2.057500in}{0.574276in}}%
\pgfpathlineto{\pgfqpoint{2.058366in}{0.573061in}}%
\pgfpathlineto{\pgfqpoint{2.061657in}{0.567654in}}%
\pgfpathlineto{\pgfqpoint{2.062003in}{0.568624in}}%
\pgfpathlineto{\pgfqpoint{2.062870in}{0.570656in}}%
\pgfpathlineto{\pgfqpoint{2.063389in}{0.569268in}}%
\pgfpathlineto{\pgfqpoint{2.066161in}{0.550791in}}%
\pgfpathlineto{\pgfqpoint{2.067373in}{0.553254in}}%
\pgfpathlineto{\pgfqpoint{2.069279in}{0.556045in}}%
\pgfpathlineto{\pgfqpoint{2.070145in}{0.557600in}}%
\pgfpathlineto{\pgfqpoint{2.070664in}{0.556516in}}%
\pgfpathlineto{\pgfqpoint{2.073782in}{0.541330in}}%
\pgfpathlineto{\pgfqpoint{2.074822in}{0.545314in}}%
\pgfpathlineto{\pgfqpoint{2.077593in}{0.567382in}}%
\pgfpathlineto{\pgfqpoint{2.079152in}{0.561457in}}%
\pgfpathlineto{\pgfqpoint{2.079845in}{0.560407in}}%
\pgfpathlineto{\pgfqpoint{2.080711in}{0.561156in}}%
\pgfpathlineto{\pgfqpoint{2.083136in}{0.565786in}}%
\pgfpathlineto{\pgfqpoint{2.084002in}{0.564387in}}%
\pgfpathlineto{\pgfqpoint{2.089372in}{0.556569in}}%
\pgfpathlineto{\pgfqpoint{2.090758in}{0.560780in}}%
\pgfpathlineto{\pgfqpoint{2.093356in}{0.570065in}}%
\pgfpathlineto{\pgfqpoint{2.093530in}{0.570025in}}%
\pgfpathlineto{\pgfqpoint{2.094742in}{0.567742in}}%
\pgfpathlineto{\pgfqpoint{2.096301in}{0.561336in}}%
\pgfpathlineto{\pgfqpoint{2.096994in}{0.563071in}}%
\pgfpathlineto{\pgfqpoint{2.102364in}{0.579357in}}%
\pgfpathlineto{\pgfqpoint{2.103057in}{0.579109in}}%
\pgfpathlineto{\pgfqpoint{2.103403in}{0.578383in}}%
\pgfpathlineto{\pgfqpoint{2.106001in}{0.572662in}}%
\pgfpathlineto{\pgfqpoint{2.106694in}{0.573976in}}%
\pgfpathlineto{\pgfqpoint{2.108253in}{0.580961in}}%
\pgfpathlineto{\pgfqpoint{2.109119in}{0.577617in}}%
\pgfpathlineto{\pgfqpoint{2.112930in}{0.561309in}}%
\pgfpathlineto{\pgfqpoint{2.113623in}{0.563350in}}%
\pgfpathlineto{\pgfqpoint{2.114836in}{0.568876in}}%
\pgfpathlineto{\pgfqpoint{2.115875in}{0.567158in}}%
\pgfpathlineto{\pgfqpoint{2.116395in}{0.566855in}}%
\pgfpathlineto{\pgfqpoint{2.116741in}{0.567945in}}%
\pgfpathlineto{\pgfqpoint{2.121245in}{0.596673in}}%
\pgfpathlineto{\pgfqpoint{2.125229in}{0.619343in}}%
\pgfpathlineto{\pgfqpoint{2.126615in}{0.618666in}}%
\pgfpathlineto{\pgfqpoint{2.130252in}{0.609043in}}%
\pgfpathlineto{\pgfqpoint{2.131465in}{0.611370in}}%
\pgfpathlineto{\pgfqpoint{2.132504in}{0.611150in}}%
\pgfpathlineto{\pgfqpoint{2.132851in}{0.610609in}}%
\pgfpathlineto{\pgfqpoint{2.133717in}{0.609779in}}%
\pgfpathlineto{\pgfqpoint{2.134236in}{0.610657in}}%
\pgfpathlineto{\pgfqpoint{2.136315in}{0.619485in}}%
\pgfpathlineto{\pgfqpoint{2.137181in}{0.622543in}}%
\pgfpathlineto{\pgfqpoint{2.137874in}{0.620792in}}%
\pgfpathlineto{\pgfqpoint{2.140645in}{0.606617in}}%
\pgfpathlineto{\pgfqpoint{2.141685in}{0.608654in}}%
\pgfpathlineto{\pgfqpoint{2.143763in}{0.609660in}}%
\pgfpathlineto{\pgfqpoint{2.144630in}{0.607883in}}%
\pgfpathlineto{\pgfqpoint{2.146362in}{0.600950in}}%
\pgfpathlineto{\pgfqpoint{2.147228in}{0.603008in}}%
\pgfpathlineto{\pgfqpoint{2.148094in}{0.604841in}}%
\pgfpathlineto{\pgfqpoint{2.148787in}{0.602873in}}%
\pgfpathlineto{\pgfqpoint{2.151905in}{0.584365in}}%
\pgfpathlineto{\pgfqpoint{2.153117in}{0.580122in}}%
\pgfpathlineto{\pgfqpoint{2.153637in}{0.581690in}}%
\pgfpathlineto{\pgfqpoint{2.155542in}{0.588438in}}%
\pgfpathlineto{\pgfqpoint{2.156408in}{0.587518in}}%
\pgfpathlineto{\pgfqpoint{2.157448in}{0.586842in}}%
\pgfpathlineto{\pgfqpoint{2.157967in}{0.587718in}}%
\pgfpathlineto{\pgfqpoint{2.160566in}{0.599428in}}%
\pgfpathlineto{\pgfqpoint{2.162818in}{0.604688in}}%
\pgfpathlineto{\pgfqpoint{2.164550in}{0.612353in}}%
\pgfpathlineto{\pgfqpoint{2.165589in}{0.608922in}}%
\pgfpathlineto{\pgfqpoint{2.170613in}{0.594536in}}%
\pgfpathlineto{\pgfqpoint{2.171652in}{0.597687in}}%
\pgfpathlineto{\pgfqpoint{2.174250in}{0.600147in}}%
\pgfpathlineto{\pgfqpoint{2.181179in}{0.598799in}}%
\pgfpathlineto{\pgfqpoint{2.182565in}{0.597875in}}%
\pgfpathlineto{\pgfqpoint{2.186376in}{0.591650in}}%
\pgfpathlineto{\pgfqpoint{2.187242in}{0.593329in}}%
\pgfpathlineto{\pgfqpoint{2.188454in}{0.595168in}}%
\pgfpathlineto{\pgfqpoint{2.189147in}{0.593909in}}%
\pgfpathlineto{\pgfqpoint{2.190360in}{0.592015in}}%
\pgfpathlineto{\pgfqpoint{2.191053in}{0.593254in}}%
\pgfpathlineto{\pgfqpoint{2.196076in}{0.598982in}}%
\pgfpathlineto{\pgfqpoint{2.196422in}{0.598744in}}%
\pgfpathlineto{\pgfqpoint{2.197808in}{0.597045in}}%
\pgfpathlineto{\pgfqpoint{2.199021in}{0.594761in}}%
\pgfpathlineto{\pgfqpoint{2.199714in}{0.596268in}}%
\pgfpathlineto{\pgfqpoint{2.203005in}{0.602413in}}%
\pgfpathlineto{\pgfqpoint{2.204564in}{0.602490in}}%
\pgfpathlineto{\pgfqpoint{2.204737in}{0.602246in}}%
\pgfpathlineto{\pgfqpoint{2.205776in}{0.601057in}}%
\pgfpathlineto{\pgfqpoint{2.206642in}{0.601884in}}%
\pgfpathlineto{\pgfqpoint{2.207682in}{0.599838in}}%
\pgfpathlineto{\pgfqpoint{2.210107in}{0.597383in}}%
\pgfpathlineto{\pgfqpoint{2.210800in}{0.598584in}}%
\pgfpathlineto{\pgfqpoint{2.211839in}{0.601220in}}%
\pgfpathlineto{\pgfqpoint{2.212532in}{0.599105in}}%
\pgfpathlineto{\pgfqpoint{2.214611in}{0.591840in}}%
\pgfpathlineto{\pgfqpoint{2.215130in}{0.592628in}}%
\pgfpathlineto{\pgfqpoint{2.216170in}{0.593217in}}%
\pgfpathlineto{\pgfqpoint{2.216689in}{0.592426in}}%
\pgfpathlineto{\pgfqpoint{2.218768in}{0.590503in}}%
\pgfpathlineto{\pgfqpoint{2.219114in}{0.590742in}}%
\pgfpathlineto{\pgfqpoint{2.220154in}{0.594750in}}%
\pgfpathlineto{\pgfqpoint{2.223445in}{0.607129in}}%
\pgfpathlineto{\pgfqpoint{2.223618in}{0.606910in}}%
\pgfpathlineto{\pgfqpoint{2.224831in}{0.600734in}}%
\pgfpathlineto{\pgfqpoint{2.227602in}{0.588446in}}%
\pgfpathlineto{\pgfqpoint{2.227775in}{0.588538in}}%
\pgfpathlineto{\pgfqpoint{2.231066in}{0.595477in}}%
\pgfpathlineto{\pgfqpoint{2.232452in}{0.600766in}}%
\pgfpathlineto{\pgfqpoint{2.233145in}{0.598673in}}%
\pgfpathlineto{\pgfqpoint{2.237302in}{0.585395in}}%
\pgfpathlineto{\pgfqpoint{2.238515in}{0.583275in}}%
\pgfpathlineto{\pgfqpoint{2.241806in}{0.576798in}}%
\pgfpathlineto{\pgfqpoint{2.244751in}{0.575780in}}%
\pgfpathlineto{\pgfqpoint{2.248735in}{0.568050in}}%
\pgfpathlineto{\pgfqpoint{2.248908in}{0.568161in}}%
\pgfpathlineto{\pgfqpoint{2.250640in}{0.574243in}}%
\pgfpathlineto{\pgfqpoint{2.251680in}{0.571621in}}%
\pgfpathlineto{\pgfqpoint{2.253065in}{0.569346in}}%
\pgfpathlineto{\pgfqpoint{2.253758in}{0.569904in}}%
\pgfpathlineto{\pgfqpoint{2.254798in}{0.568887in}}%
\pgfpathlineto{\pgfqpoint{2.256530in}{0.564345in}}%
\pgfpathlineto{\pgfqpoint{2.257223in}{0.566745in}}%
\pgfpathlineto{\pgfqpoint{2.259994in}{0.573526in}}%
\pgfpathlineto{\pgfqpoint{2.261380in}{0.571598in}}%
\pgfpathlineto{\pgfqpoint{2.265711in}{0.558188in}}%
\pgfpathlineto{\pgfqpoint{2.266230in}{0.558787in}}%
\pgfpathlineto{\pgfqpoint{2.270387in}{0.566691in}}%
\pgfpathlineto{\pgfqpoint{2.274545in}{0.578245in}}%
\pgfpathlineto{\pgfqpoint{2.277836in}{0.586005in}}%
\pgfpathlineto{\pgfqpoint{2.281647in}{0.596013in}}%
\pgfpathlineto{\pgfqpoint{2.281993in}{0.596345in}}%
\pgfpathlineto{\pgfqpoint{2.282686in}{0.595183in}}%
\pgfpathlineto{\pgfqpoint{2.285631in}{0.591860in}}%
\pgfpathlineto{\pgfqpoint{2.286670in}{0.592963in}}%
\pgfpathlineto{\pgfqpoint{2.287536in}{0.594601in}}%
\pgfpathlineto{\pgfqpoint{2.288576in}{0.593488in}}%
\pgfpathlineto{\pgfqpoint{2.295158in}{0.580325in}}%
\pgfpathlineto{\pgfqpoint{2.296024in}{0.583006in}}%
\pgfpathlineto{\pgfqpoint{2.296717in}{0.584595in}}%
\pgfpathlineto{\pgfqpoint{2.297237in}{0.582935in}}%
\pgfpathlineto{\pgfqpoint{2.299315in}{0.573995in}}%
\pgfpathlineto{\pgfqpoint{2.299662in}{0.575912in}}%
\pgfpathlineto{\pgfqpoint{2.300701in}{0.621887in}}%
\pgfpathlineto{\pgfqpoint{2.306764in}{0.918314in}}%
\pgfpathlineto{\pgfqpoint{2.311267in}{1.022151in}}%
\pgfpathlineto{\pgfqpoint{2.317677in}{1.090769in}}%
\pgfpathlineto{\pgfqpoint{2.319236in}{1.093840in}}%
\pgfpathlineto{\pgfqpoint{2.319755in}{1.093646in}}%
\pgfpathlineto{\pgfqpoint{2.324259in}{1.089905in}}%
\pgfpathlineto{\pgfqpoint{2.339502in}{1.051848in}}%
\pgfpathlineto{\pgfqpoint{2.344353in}{1.042227in}}%
\pgfpathlineto{\pgfqpoint{2.347990in}{1.025599in}}%
\pgfpathlineto{\pgfqpoint{2.356998in}{0.977093in}}%
\pgfpathlineto{\pgfqpoint{2.366005in}{0.949573in}}%
\pgfpathlineto{\pgfqpoint{2.370682in}{0.938842in}}%
\pgfpathlineto{\pgfqpoint{2.371202in}{0.939179in}}%
\pgfpathlineto{\pgfqpoint{2.373627in}{0.940216in}}%
\pgfpathlineto{\pgfqpoint{2.373800in}{0.940031in}}%
\pgfpathlineto{\pgfqpoint{2.376918in}{0.933446in}}%
\pgfpathlineto{\pgfqpoint{2.383327in}{0.918561in}}%
\pgfpathlineto{\pgfqpoint{2.383847in}{0.919096in}}%
\pgfpathlineto{\pgfqpoint{2.392508in}{0.934428in}}%
\pgfpathlineto{\pgfqpoint{2.392681in}{0.934304in}}%
\pgfpathlineto{\pgfqpoint{2.396838in}{0.926738in}}%
\pgfpathlineto{\pgfqpoint{2.401688in}{0.909132in}}%
\pgfpathlineto{\pgfqpoint{2.411562in}{0.871495in}}%
\pgfpathlineto{\pgfqpoint{2.418491in}{0.844073in}}%
\pgfpathlineto{\pgfqpoint{2.420743in}{0.837961in}}%
\pgfpathlineto{\pgfqpoint{2.421609in}{0.836332in}}%
\pgfpathlineto{\pgfqpoint{2.436333in}{0.783554in}}%
\pgfpathlineto{\pgfqpoint{2.437025in}{0.782692in}}%
\pgfpathlineto{\pgfqpoint{2.437545in}{0.784222in}}%
\pgfpathlineto{\pgfqpoint{2.438584in}{0.786978in}}%
\pgfpathlineto{\pgfqpoint{2.439104in}{0.785111in}}%
\pgfpathlineto{\pgfqpoint{2.441529in}{0.778736in}}%
\pgfpathlineto{\pgfqpoint{2.442395in}{0.780807in}}%
\pgfpathlineto{\pgfqpoint{2.444127in}{0.787733in}}%
\pgfpathlineto{\pgfqpoint{2.444820in}{0.786113in}}%
\pgfpathlineto{\pgfqpoint{2.446553in}{0.779847in}}%
\pgfpathlineto{\pgfqpoint{2.447245in}{0.782109in}}%
\pgfpathlineto{\pgfqpoint{2.448285in}{0.785094in}}%
\pgfpathlineto{\pgfqpoint{2.448978in}{0.783640in}}%
\pgfpathlineto{\pgfqpoint{2.451056in}{0.778852in}}%
\pgfpathlineto{\pgfqpoint{2.451403in}{0.779372in}}%
\pgfpathlineto{\pgfqpoint{2.455733in}{0.792250in}}%
\pgfpathlineto{\pgfqpoint{2.456946in}{0.789931in}}%
\pgfpathlineto{\pgfqpoint{2.458851in}{0.787442in}}%
\pgfpathlineto{\pgfqpoint{2.459371in}{0.788002in}}%
\pgfpathlineto{\pgfqpoint{2.461103in}{0.787156in}}%
\pgfpathlineto{\pgfqpoint{2.462835in}{0.784824in}}%
\pgfpathlineto{\pgfqpoint{2.469418in}{0.769338in}}%
\pgfpathlineto{\pgfqpoint{2.469937in}{0.770048in}}%
\pgfpathlineto{\pgfqpoint{2.471496in}{0.772603in}}%
\pgfpathlineto{\pgfqpoint{2.472189in}{0.771715in}}%
\pgfpathlineto{\pgfqpoint{2.474095in}{0.767567in}}%
\pgfpathlineto{\pgfqpoint{2.474787in}{0.768875in}}%
\pgfpathlineto{\pgfqpoint{2.476346in}{0.769761in}}%
\pgfpathlineto{\pgfqpoint{2.476520in}{0.769561in}}%
\pgfpathlineto{\pgfqpoint{2.478425in}{0.767102in}}%
\pgfpathlineto{\pgfqpoint{2.479118in}{0.767779in}}%
\pgfpathlineto{\pgfqpoint{2.481716in}{0.774828in}}%
\pgfpathlineto{\pgfqpoint{2.482582in}{0.772335in}}%
\pgfpathlineto{\pgfqpoint{2.485527in}{0.766485in}}%
\pgfpathlineto{\pgfqpoint{2.492629in}{0.750259in}}%
\pgfpathlineto{\pgfqpoint{2.492976in}{0.750624in}}%
\pgfpathlineto{\pgfqpoint{2.494188in}{0.750883in}}%
\pgfpathlineto{\pgfqpoint{2.494535in}{0.750290in}}%
\pgfpathlineto{\pgfqpoint{2.498345in}{0.742025in}}%
\pgfpathlineto{\pgfqpoint{2.498692in}{0.742280in}}%
\pgfpathlineto{\pgfqpoint{2.500944in}{0.744861in}}%
\pgfpathlineto{\pgfqpoint{2.501290in}{0.743607in}}%
\pgfpathlineto{\pgfqpoint{2.502676in}{0.739154in}}%
\pgfpathlineto{\pgfqpoint{2.503196in}{0.741423in}}%
\pgfpathlineto{\pgfqpoint{2.506487in}{0.751090in}}%
\pgfpathlineto{\pgfqpoint{2.509258in}{0.754677in}}%
\pgfpathlineto{\pgfqpoint{2.509778in}{0.753980in}}%
\pgfpathlineto{\pgfqpoint{2.518266in}{0.741219in}}%
\pgfpathlineto{\pgfqpoint{2.518959in}{0.742622in}}%
\pgfpathlineto{\pgfqpoint{2.520691in}{0.749656in}}%
\pgfpathlineto{\pgfqpoint{2.521557in}{0.748366in}}%
\pgfpathlineto{\pgfqpoint{2.526754in}{0.732195in}}%
\pgfpathlineto{\pgfqpoint{2.527446in}{0.735314in}}%
\pgfpathlineto{\pgfqpoint{2.529698in}{0.751354in}}%
\pgfpathlineto{\pgfqpoint{2.530564in}{0.746288in}}%
\pgfpathlineto{\pgfqpoint{2.532470in}{0.735973in}}%
\pgfpathlineto{\pgfqpoint{2.532989in}{0.736221in}}%
\pgfpathlineto{\pgfqpoint{2.533856in}{0.735117in}}%
\pgfpathlineto{\pgfqpoint{2.534895in}{0.734679in}}%
\pgfpathlineto{\pgfqpoint{2.535241in}{0.735648in}}%
\pgfpathlineto{\pgfqpoint{2.537493in}{0.742270in}}%
\pgfpathlineto{\pgfqpoint{2.538013in}{0.741254in}}%
\pgfpathlineto{\pgfqpoint{2.539572in}{0.729328in}}%
\pgfpathlineto{\pgfqpoint{2.541304in}{0.722508in}}%
\pgfpathlineto{\pgfqpoint{2.541650in}{0.723015in}}%
\pgfpathlineto{\pgfqpoint{2.548579in}{0.755471in}}%
\pgfpathlineto{\pgfqpoint{2.549965in}{0.760778in}}%
\pgfpathlineto{\pgfqpoint{2.550658in}{0.760218in}}%
\pgfpathlineto{\pgfqpoint{2.551870in}{0.754603in}}%
\pgfpathlineto{\pgfqpoint{2.553429in}{0.746607in}}%
\pgfpathlineto{\pgfqpoint{2.554122in}{0.748182in}}%
\pgfpathlineto{\pgfqpoint{2.560358in}{0.759974in}}%
\pgfpathlineto{\pgfqpoint{2.560878in}{0.760333in}}%
\pgfpathlineto{\pgfqpoint{2.561571in}{0.759284in}}%
\pgfpathlineto{\pgfqpoint{2.562783in}{0.755955in}}%
\pgfpathlineto{\pgfqpoint{2.564169in}{0.749300in}}%
\pgfpathlineto{\pgfqpoint{2.565035in}{0.751645in}}%
\pgfpathlineto{\pgfqpoint{2.567634in}{0.758575in}}%
\pgfpathlineto{\pgfqpoint{2.568153in}{0.757553in}}%
\pgfpathlineto{\pgfqpoint{2.572657in}{0.736432in}}%
\pgfpathlineto{\pgfqpoint{2.574043in}{0.740394in}}%
\pgfpathlineto{\pgfqpoint{2.578720in}{0.750506in}}%
\pgfpathlineto{\pgfqpoint{2.578893in}{0.750399in}}%
\pgfpathlineto{\pgfqpoint{2.582184in}{0.746756in}}%
\pgfpathlineto{\pgfqpoint{2.582877in}{0.749216in}}%
\pgfpathlineto{\pgfqpoint{2.583916in}{0.751885in}}%
\pgfpathlineto{\pgfqpoint{2.584436in}{0.750436in}}%
\pgfpathlineto{\pgfqpoint{2.589113in}{0.732926in}}%
\pgfpathlineto{\pgfqpoint{2.590672in}{0.725820in}}%
\pgfpathlineto{\pgfqpoint{2.592577in}{0.712810in}}%
\pgfpathlineto{\pgfqpoint{2.593270in}{0.714861in}}%
\pgfpathlineto{\pgfqpoint{2.598120in}{0.727839in}}%
\pgfpathlineto{\pgfqpoint{2.598986in}{0.726870in}}%
\pgfpathlineto{\pgfqpoint{2.599506in}{0.725976in}}%
\pgfpathlineto{\pgfqpoint{2.600719in}{0.726598in}}%
\pgfpathlineto{\pgfqpoint{2.601585in}{0.729037in}}%
\pgfpathlineto{\pgfqpoint{2.602970in}{0.733879in}}%
\pgfpathlineto{\pgfqpoint{2.603837in}{0.731953in}}%
\pgfpathlineto{\pgfqpoint{2.605569in}{0.725647in}}%
\pgfpathlineto{\pgfqpoint{2.606781in}{0.727590in}}%
\pgfpathlineto{\pgfqpoint{2.607821in}{0.722467in}}%
\pgfpathlineto{\pgfqpoint{2.608860in}{0.718832in}}%
\pgfpathlineto{\pgfqpoint{2.609553in}{0.720865in}}%
\pgfpathlineto{\pgfqpoint{2.611458in}{0.723884in}}%
\pgfpathlineto{\pgfqpoint{2.611978in}{0.723433in}}%
\pgfpathlineto{\pgfqpoint{2.615096in}{0.720650in}}%
\pgfpathlineto{\pgfqpoint{2.616482in}{0.718784in}}%
\pgfpathlineto{\pgfqpoint{2.621678in}{0.704293in}}%
\pgfpathlineto{\pgfqpoint{2.622371in}{0.705702in}}%
\pgfpathlineto{\pgfqpoint{2.626182in}{0.712741in}}%
\pgfpathlineto{\pgfqpoint{2.627395in}{0.709530in}}%
\pgfpathlineto{\pgfqpoint{2.628954in}{0.703145in}}%
\pgfpathlineto{\pgfqpoint{2.629820in}{0.704637in}}%
\pgfpathlineto{\pgfqpoint{2.631898in}{0.706204in}}%
\pgfpathlineto{\pgfqpoint{2.632591in}{0.705277in}}%
\pgfpathlineto{\pgfqpoint{2.633977in}{0.697828in}}%
\pgfpathlineto{\pgfqpoint{2.634843in}{0.700286in}}%
\pgfpathlineto{\pgfqpoint{2.636575in}{0.700847in}}%
\pgfpathlineto{\pgfqpoint{2.637441in}{0.701112in}}%
\pgfpathlineto{\pgfqpoint{2.637788in}{0.700274in}}%
\pgfpathlineto{\pgfqpoint{2.643504in}{0.686913in}}%
\pgfpathlineto{\pgfqpoint{2.644543in}{0.688225in}}%
\pgfpathlineto{\pgfqpoint{2.650606in}{0.703931in}}%
\pgfpathlineto{\pgfqpoint{2.650952in}{0.703458in}}%
\pgfpathlineto{\pgfqpoint{2.653204in}{0.704382in}}%
\pgfpathlineto{\pgfqpoint{2.653551in}{0.705233in}}%
\pgfpathlineto{\pgfqpoint{2.654590in}{0.704215in}}%
\pgfpathlineto{\pgfqpoint{2.662558in}{0.694688in}}%
\pgfpathlineto{\pgfqpoint{2.665676in}{0.684706in}}%
\pgfpathlineto{\pgfqpoint{2.666023in}{0.685591in}}%
\pgfpathlineto{\pgfqpoint{2.667235in}{0.691330in}}%
\pgfpathlineto{\pgfqpoint{2.667928in}{0.688013in}}%
\pgfpathlineto{\pgfqpoint{2.668794in}{0.684322in}}%
\pgfpathlineto{\pgfqpoint{2.669487in}{0.686822in}}%
\pgfpathlineto{\pgfqpoint{2.671739in}{0.690560in}}%
\pgfpathlineto{\pgfqpoint{2.672432in}{0.692151in}}%
\pgfpathlineto{\pgfqpoint{2.676589in}{0.699547in}}%
\pgfpathlineto{\pgfqpoint{2.677282in}{0.699507in}}%
\pgfpathlineto{\pgfqpoint{2.677628in}{0.698678in}}%
\pgfpathlineto{\pgfqpoint{2.683345in}{0.677092in}}%
\pgfpathlineto{\pgfqpoint{2.684557in}{0.678318in}}%
\pgfpathlineto{\pgfqpoint{2.685077in}{0.678753in}}%
\pgfpathlineto{\pgfqpoint{2.685423in}{0.677201in}}%
\pgfpathlineto{\pgfqpoint{2.686289in}{0.673108in}}%
\pgfpathlineto{\pgfqpoint{2.686982in}{0.675889in}}%
\pgfpathlineto{\pgfqpoint{2.689754in}{0.689605in}}%
\pgfpathlineto{\pgfqpoint{2.692525in}{0.701739in}}%
\pgfpathlineto{\pgfqpoint{2.692699in}{0.701608in}}%
\pgfpathlineto{\pgfqpoint{2.701013in}{0.680789in}}%
\pgfpathlineto{\pgfqpoint{2.701879in}{0.683623in}}%
\pgfpathlineto{\pgfqpoint{2.702572in}{0.685159in}}%
\pgfpathlineto{\pgfqpoint{2.703265in}{0.683023in}}%
\pgfpathlineto{\pgfqpoint{2.703785in}{0.682527in}}%
\pgfpathlineto{\pgfqpoint{2.704304in}{0.683587in}}%
\pgfpathlineto{\pgfqpoint{2.707076in}{0.690936in}}%
\pgfpathlineto{\pgfqpoint{2.708115in}{0.690150in}}%
\pgfpathlineto{\pgfqpoint{2.708635in}{0.690980in}}%
\pgfpathlineto{\pgfqpoint{2.709847in}{0.700400in}}%
\pgfpathlineto{\pgfqpoint{2.710887in}{0.705933in}}%
\pgfpathlineto{\pgfqpoint{2.711580in}{0.703847in}}%
\pgfpathlineto{\pgfqpoint{2.713139in}{0.699558in}}%
\pgfpathlineto{\pgfqpoint{2.713831in}{0.700882in}}%
\pgfpathlineto{\pgfqpoint{2.717469in}{0.712272in}}%
\pgfpathlineto{\pgfqpoint{2.718682in}{0.709002in}}%
\pgfpathlineto{\pgfqpoint{2.720067in}{0.705551in}}%
\pgfpathlineto{\pgfqpoint{2.720587in}{0.707024in}}%
\pgfpathlineto{\pgfqpoint{2.722319in}{0.715926in}}%
\pgfpathlineto{\pgfqpoint{2.723359in}{0.712638in}}%
\pgfpathlineto{\pgfqpoint{2.726130in}{0.706549in}}%
\pgfpathlineto{\pgfqpoint{2.726650in}{0.708160in}}%
\pgfpathlineto{\pgfqpoint{2.728728in}{0.712247in}}%
\pgfpathlineto{\pgfqpoint{2.730114in}{0.709772in}}%
\pgfpathlineto{\pgfqpoint{2.732020in}{0.702819in}}%
\pgfpathlineto{\pgfqpoint{2.736870in}{0.687620in}}%
\pgfpathlineto{\pgfqpoint{2.739295in}{0.686016in}}%
\pgfpathlineto{\pgfqpoint{2.739641in}{0.686598in}}%
\pgfpathlineto{\pgfqpoint{2.745011in}{0.706878in}}%
\pgfpathlineto{\pgfqpoint{2.746397in}{0.703040in}}%
\pgfpathlineto{\pgfqpoint{2.748476in}{0.699138in}}%
\pgfpathlineto{\pgfqpoint{2.748995in}{0.701286in}}%
\pgfpathlineto{\pgfqpoint{2.750901in}{0.710499in}}%
\pgfpathlineto{\pgfqpoint{2.751767in}{0.709260in}}%
\pgfpathlineto{\pgfqpoint{2.753845in}{0.700460in}}%
\pgfpathlineto{\pgfqpoint{2.756444in}{0.678741in}}%
\pgfpathlineto{\pgfqpoint{2.757137in}{0.679850in}}%
\pgfpathlineto{\pgfqpoint{2.763199in}{0.700547in}}%
\pgfpathlineto{\pgfqpoint{2.763892in}{0.698488in}}%
\pgfpathlineto{\pgfqpoint{2.766317in}{0.695017in}}%
\pgfpathlineto{\pgfqpoint{2.767357in}{0.699436in}}%
\pgfpathlineto{\pgfqpoint{2.768916in}{0.705708in}}%
\pgfpathlineto{\pgfqpoint{2.769608in}{0.704004in}}%
\pgfpathlineto{\pgfqpoint{2.771167in}{0.698220in}}%
\pgfpathlineto{\pgfqpoint{2.772207in}{0.699811in}}%
\pgfpathlineto{\pgfqpoint{2.774459in}{0.700821in}}%
\pgfpathlineto{\pgfqpoint{2.774632in}{0.700658in}}%
\pgfpathlineto{\pgfqpoint{2.774978in}{0.700197in}}%
\pgfpathlineto{\pgfqpoint{2.775498in}{0.701142in}}%
\pgfpathlineto{\pgfqpoint{2.775844in}{0.701256in}}%
\pgfpathlineto{\pgfqpoint{2.777577in}{0.705605in}}%
\pgfpathlineto{\pgfqpoint{2.778616in}{0.708116in}}%
\pgfpathlineto{\pgfqpoint{2.779482in}{0.706639in}}%
\pgfpathlineto{\pgfqpoint{2.782253in}{0.710284in}}%
\pgfpathlineto{\pgfqpoint{2.782773in}{0.709524in}}%
\pgfpathlineto{\pgfqpoint{2.783466in}{0.708166in}}%
\pgfpathlineto{\pgfqpoint{2.784159in}{0.709780in}}%
\pgfpathlineto{\pgfqpoint{2.785198in}{0.709717in}}%
\pgfpathlineto{\pgfqpoint{2.785371in}{0.709363in}}%
\pgfpathlineto{\pgfqpoint{2.787277in}{0.701215in}}%
\pgfpathlineto{\pgfqpoint{2.787970in}{0.704684in}}%
\pgfpathlineto{\pgfqpoint{2.789009in}{0.710864in}}%
\pgfpathlineto{\pgfqpoint{2.789875in}{0.707665in}}%
\pgfpathlineto{\pgfqpoint{2.790914in}{0.705366in}}%
\pgfpathlineto{\pgfqpoint{2.791607in}{0.706623in}}%
\pgfpathlineto{\pgfqpoint{2.792993in}{0.708288in}}%
\pgfpathlineto{\pgfqpoint{2.793513in}{0.707214in}}%
\pgfpathlineto{\pgfqpoint{2.798363in}{0.679688in}}%
\pgfpathlineto{\pgfqpoint{2.800442in}{0.683282in}}%
\pgfpathlineto{\pgfqpoint{2.806331in}{0.699308in}}%
\pgfpathlineto{\pgfqpoint{2.807197in}{0.698137in}}%
\pgfpathlineto{\pgfqpoint{2.810488in}{0.691412in}}%
\pgfpathlineto{\pgfqpoint{2.813780in}{0.687986in}}%
\pgfpathlineto{\pgfqpoint{2.815858in}{0.677268in}}%
\pgfpathlineto{\pgfqpoint{2.817071in}{0.679223in}}%
\pgfpathlineto{\pgfqpoint{2.819496in}{0.690051in}}%
\pgfpathlineto{\pgfqpoint{2.820535in}{0.694756in}}%
\pgfpathlineto{\pgfqpoint{2.821574in}{0.694225in}}%
\pgfpathlineto{\pgfqpoint{2.821748in}{0.694012in}}%
\pgfpathlineto{\pgfqpoint{2.822441in}{0.695478in}}%
\pgfpathlineto{\pgfqpoint{2.824000in}{0.696271in}}%
\pgfpathlineto{\pgfqpoint{2.824173in}{0.696075in}}%
\pgfpathlineto{\pgfqpoint{2.824519in}{0.695806in}}%
\pgfpathlineto{\pgfqpoint{2.825039in}{0.696918in}}%
\pgfpathlineto{\pgfqpoint{2.828330in}{0.705713in}}%
\pgfpathlineto{\pgfqpoint{2.832834in}{0.715121in}}%
\pgfpathlineto{\pgfqpoint{2.833700in}{0.717408in}}%
\pgfpathlineto{\pgfqpoint{2.834393in}{0.715059in}}%
\pgfpathlineto{\pgfqpoint{2.836818in}{0.699502in}}%
\pgfpathlineto{\pgfqpoint{2.837511in}{0.701908in}}%
\pgfpathlineto{\pgfqpoint{2.840456in}{0.706721in}}%
\pgfpathlineto{\pgfqpoint{2.840629in}{0.706570in}}%
\pgfpathlineto{\pgfqpoint{2.843227in}{0.696247in}}%
\pgfpathlineto{\pgfqpoint{2.846691in}{0.678045in}}%
\pgfpathlineto{\pgfqpoint{2.847384in}{0.677563in}}%
\pgfpathlineto{\pgfqpoint{2.847731in}{0.678793in}}%
\pgfpathlineto{\pgfqpoint{2.850329in}{0.690302in}}%
\pgfpathlineto{\pgfqpoint{2.850849in}{0.689552in}}%
\pgfpathlineto{\pgfqpoint{2.851715in}{0.683362in}}%
\pgfpathlineto{\pgfqpoint{2.854660in}{0.664067in}}%
\pgfpathlineto{\pgfqpoint{2.855179in}{0.666092in}}%
\pgfpathlineto{\pgfqpoint{2.858297in}{0.684344in}}%
\pgfpathlineto{\pgfqpoint{2.858990in}{0.683193in}}%
\pgfpathlineto{\pgfqpoint{2.860029in}{0.677827in}}%
\pgfpathlineto{\pgfqpoint{2.861069in}{0.680924in}}%
\pgfpathlineto{\pgfqpoint{2.862281in}{0.686322in}}%
\pgfpathlineto{\pgfqpoint{2.862974in}{0.684518in}}%
\pgfpathlineto{\pgfqpoint{2.864706in}{0.679087in}}%
\pgfpathlineto{\pgfqpoint{2.865226in}{0.680459in}}%
\pgfpathlineto{\pgfqpoint{2.868171in}{0.691002in}}%
\pgfpathlineto{\pgfqpoint{2.868517in}{0.690876in}}%
\pgfpathlineto{\pgfqpoint{2.875100in}{0.704819in}}%
\pgfpathlineto{\pgfqpoint{2.876485in}{0.702828in}}%
\pgfpathlineto{\pgfqpoint{2.881509in}{0.680955in}}%
\pgfpathlineto{\pgfqpoint{2.882548in}{0.685616in}}%
\pgfpathlineto{\pgfqpoint{2.884627in}{0.696748in}}%
\pgfpathlineto{\pgfqpoint{2.885146in}{0.696189in}}%
\pgfpathlineto{\pgfqpoint{2.886012in}{0.693758in}}%
\pgfpathlineto{\pgfqpoint{2.886879in}{0.694467in}}%
\pgfpathlineto{\pgfqpoint{2.887225in}{0.694678in}}%
\pgfpathlineto{\pgfqpoint{2.887745in}{0.693391in}}%
\pgfpathlineto{\pgfqpoint{2.889997in}{0.687421in}}%
\pgfpathlineto{\pgfqpoint{2.890689in}{0.688832in}}%
\pgfpathlineto{\pgfqpoint{2.891902in}{0.692042in}}%
\pgfpathlineto{\pgfqpoint{2.892595in}{0.690175in}}%
\pgfpathlineto{\pgfqpoint{2.897099in}{0.672381in}}%
\pgfpathlineto{\pgfqpoint{2.897445in}{0.671866in}}%
\pgfpathlineto{\pgfqpoint{2.898138in}{0.673650in}}%
\pgfpathlineto{\pgfqpoint{2.900217in}{0.679138in}}%
\pgfpathlineto{\pgfqpoint{2.901083in}{0.677973in}}%
\pgfpathlineto{\pgfqpoint{2.901602in}{0.678447in}}%
\pgfpathlineto{\pgfqpoint{2.902295in}{0.677291in}}%
\pgfpathlineto{\pgfqpoint{2.903161in}{0.677022in}}%
\pgfpathlineto{\pgfqpoint{2.903508in}{0.678048in}}%
\pgfpathlineto{\pgfqpoint{2.905067in}{0.685095in}}%
\pgfpathlineto{\pgfqpoint{2.905760in}{0.683130in}}%
\pgfpathlineto{\pgfqpoint{2.909397in}{0.673558in}}%
\pgfpathlineto{\pgfqpoint{2.910956in}{0.677354in}}%
\pgfpathlineto{\pgfqpoint{2.913035in}{0.685455in}}%
\pgfpathlineto{\pgfqpoint{2.913728in}{0.681640in}}%
\pgfpathlineto{\pgfqpoint{2.916326in}{0.667417in}}%
\pgfpathlineto{\pgfqpoint{2.916672in}{0.667947in}}%
\pgfpathlineto{\pgfqpoint{2.918058in}{0.672239in}}%
\pgfpathlineto{\pgfqpoint{2.919098in}{0.671239in}}%
\pgfpathlineto{\pgfqpoint{2.919964in}{0.673012in}}%
\pgfpathlineto{\pgfqpoint{2.920483in}{0.674855in}}%
\pgfpathlineto{\pgfqpoint{2.921523in}{0.673960in}}%
\pgfpathlineto{\pgfqpoint{2.922389in}{0.673078in}}%
\pgfpathlineto{\pgfqpoint{2.922908in}{0.674340in}}%
\pgfpathlineto{\pgfqpoint{2.928625in}{0.685014in}}%
\pgfpathlineto{\pgfqpoint{2.929318in}{0.682583in}}%
\pgfpathlineto{\pgfqpoint{2.930703in}{0.677662in}}%
\pgfpathlineto{\pgfqpoint{2.931396in}{0.678906in}}%
\pgfpathlineto{\pgfqpoint{2.933302in}{0.681575in}}%
\pgfpathlineto{\pgfqpoint{2.933821in}{0.680473in}}%
\pgfpathlineto{\pgfqpoint{2.936766in}{0.674714in}}%
\pgfpathlineto{\pgfqpoint{2.937632in}{0.676195in}}%
\pgfpathlineto{\pgfqpoint{2.941443in}{0.688762in}}%
\pgfpathlineto{\pgfqpoint{2.943522in}{0.694120in}}%
\pgfpathlineto{\pgfqpoint{2.944041in}{0.693880in}}%
\pgfpathlineto{\pgfqpoint{2.945081in}{0.693526in}}%
\pgfpathlineto{\pgfqpoint{2.945254in}{0.693137in}}%
\pgfpathlineto{\pgfqpoint{2.949931in}{0.677987in}}%
\pgfpathlineto{\pgfqpoint{2.951143in}{0.665244in}}%
\pgfpathlineto{\pgfqpoint{2.951836in}{0.670367in}}%
\pgfpathlineto{\pgfqpoint{2.953395in}{0.681338in}}%
\pgfpathlineto{\pgfqpoint{2.954261in}{0.681156in}}%
\pgfpathlineto{\pgfqpoint{2.955127in}{0.678411in}}%
\pgfpathlineto{\pgfqpoint{2.957033in}{0.668348in}}%
\pgfpathlineto{\pgfqpoint{2.957726in}{0.670771in}}%
\pgfpathlineto{\pgfqpoint{2.962576in}{0.696717in}}%
\pgfpathlineto{\pgfqpoint{2.963788in}{0.693486in}}%
\pgfpathlineto{\pgfqpoint{2.966040in}{0.686265in}}%
\pgfpathlineto{\pgfqpoint{2.966387in}{0.686750in}}%
\pgfpathlineto{\pgfqpoint{2.969678in}{0.695451in}}%
\pgfpathlineto{\pgfqpoint{2.970198in}{0.695239in}}%
\pgfpathlineto{\pgfqpoint{2.971237in}{0.693585in}}%
\pgfpathlineto{\pgfqpoint{2.972276in}{0.690843in}}%
\pgfpathlineto{\pgfqpoint{2.972969in}{0.693318in}}%
\pgfpathlineto{\pgfqpoint{2.974182in}{0.696881in}}%
\pgfpathlineto{\pgfqpoint{2.974701in}{0.694993in}}%
\pgfpathlineto{\pgfqpoint{2.975048in}{0.694337in}}%
\pgfpathlineto{\pgfqpoint{2.975741in}{0.696273in}}%
\pgfpathlineto{\pgfqpoint{2.976953in}{0.700679in}}%
\pgfpathlineto{\pgfqpoint{2.977646in}{0.698295in}}%
\pgfpathlineto{\pgfqpoint{2.978685in}{0.696211in}}%
\pgfpathlineto{\pgfqpoint{2.979205in}{0.698210in}}%
\pgfpathlineto{\pgfqpoint{2.980071in}{0.702222in}}%
\pgfpathlineto{\pgfqpoint{2.980937in}{0.700526in}}%
\pgfpathlineto{\pgfqpoint{2.981457in}{0.699130in}}%
\pgfpathlineto{\pgfqpoint{2.982150in}{0.701294in}}%
\pgfpathlineto{\pgfqpoint{2.984748in}{0.707384in}}%
\pgfpathlineto{\pgfqpoint{2.985094in}{0.707124in}}%
\pgfpathlineto{\pgfqpoint{2.986827in}{0.702190in}}%
\pgfpathlineto{\pgfqpoint{2.990984in}{0.688099in}}%
\pgfpathlineto{\pgfqpoint{2.991504in}{0.686644in}}%
\pgfpathlineto{\pgfqpoint{2.992543in}{0.687814in}}%
\pgfpathlineto{\pgfqpoint{2.993755in}{0.687034in}}%
\pgfpathlineto{\pgfqpoint{2.994102in}{0.688192in}}%
\pgfpathlineto{\pgfqpoint{2.995834in}{0.694404in}}%
\pgfpathlineto{\pgfqpoint{2.996700in}{0.693968in}}%
\pgfpathlineto{\pgfqpoint{2.998259in}{0.701094in}}%
\pgfpathlineto{\pgfqpoint{3.000338in}{0.708388in}}%
\pgfpathlineto{\pgfqpoint{3.000858in}{0.707884in}}%
\pgfpathlineto{\pgfqpoint{3.001377in}{0.707858in}}%
\pgfpathlineto{\pgfqpoint{3.001550in}{0.708460in}}%
\pgfpathlineto{\pgfqpoint{3.005708in}{0.721474in}}%
\pgfpathlineto{\pgfqpoint{3.008306in}{0.733418in}}%
\pgfpathlineto{\pgfqpoint{3.009865in}{0.731155in}}%
\pgfpathlineto{\pgfqpoint{3.011251in}{0.727186in}}%
\pgfpathlineto{\pgfqpoint{3.012117in}{0.728754in}}%
\pgfpathlineto{\pgfqpoint{3.012810in}{0.730071in}}%
\pgfpathlineto{\pgfqpoint{3.013676in}{0.729022in}}%
\pgfpathlineto{\pgfqpoint{3.016447in}{0.728029in}}%
\pgfpathlineto{\pgfqpoint{3.016794in}{0.729289in}}%
\pgfpathlineto{\pgfqpoint{3.020258in}{0.744450in}}%
\pgfpathlineto{\pgfqpoint{3.021124in}{0.742342in}}%
\pgfpathlineto{\pgfqpoint{3.021990in}{0.740961in}}%
\pgfpathlineto{\pgfqpoint{3.022856in}{0.741999in}}%
\pgfpathlineto{\pgfqpoint{3.023723in}{0.742190in}}%
\pgfpathlineto{\pgfqpoint{3.024069in}{0.741273in}}%
\pgfpathlineto{\pgfqpoint{3.028919in}{0.722436in}}%
\pgfpathlineto{\pgfqpoint{3.029439in}{0.724924in}}%
\pgfpathlineto{\pgfqpoint{3.032210in}{0.737199in}}%
\pgfpathlineto{\pgfqpoint{3.032384in}{0.737331in}}%
\pgfpathlineto{\pgfqpoint{3.032903in}{0.735845in}}%
\pgfpathlineto{\pgfqpoint{3.036368in}{0.724836in}}%
\pgfpathlineto{\pgfqpoint{3.037061in}{0.724080in}}%
\pgfpathlineto{\pgfqpoint{3.037753in}{0.725214in}}%
\pgfpathlineto{\pgfqpoint{3.038100in}{0.725249in}}%
\pgfpathlineto{\pgfqpoint{3.038446in}{0.724354in}}%
\pgfpathlineto{\pgfqpoint{3.039312in}{0.722415in}}%
\pgfpathlineto{\pgfqpoint{3.040005in}{0.724258in}}%
\pgfpathlineto{\pgfqpoint{3.042257in}{0.732952in}}%
\pgfpathlineto{\pgfqpoint{3.042950in}{0.731131in}}%
\pgfpathlineto{\pgfqpoint{3.047454in}{0.717312in}}%
\pgfpathlineto{\pgfqpoint{3.047800in}{0.717499in}}%
\pgfpathlineto{\pgfqpoint{3.050225in}{0.716936in}}%
\pgfpathlineto{\pgfqpoint{3.054729in}{0.700501in}}%
\pgfpathlineto{\pgfqpoint{3.059752in}{0.697355in}}%
\pgfpathlineto{\pgfqpoint{3.062178in}{0.693111in}}%
\pgfpathlineto{\pgfqpoint{3.062524in}{0.693610in}}%
\pgfpathlineto{\pgfqpoint{3.065469in}{0.711164in}}%
\pgfpathlineto{\pgfqpoint{3.066335in}{0.705810in}}%
\pgfpathlineto{\pgfqpoint{3.067894in}{0.689164in}}%
\pgfpathlineto{\pgfqpoint{3.068933in}{0.690150in}}%
\pgfpathlineto{\pgfqpoint{3.069972in}{0.682186in}}%
\pgfpathlineto{\pgfqpoint{3.070839in}{0.678865in}}%
\pgfpathlineto{\pgfqpoint{3.071531in}{0.680985in}}%
\pgfpathlineto{\pgfqpoint{3.075169in}{0.699998in}}%
\pgfpathlineto{\pgfqpoint{3.075862in}{0.697954in}}%
\pgfpathlineto{\pgfqpoint{3.076901in}{0.696215in}}%
\pgfpathlineto{\pgfqpoint{3.077421in}{0.697465in}}%
\pgfpathlineto{\pgfqpoint{3.081578in}{0.703513in}}%
\pgfpathlineto{\pgfqpoint{3.082791in}{0.698026in}}%
\pgfpathlineto{\pgfqpoint{3.083310in}{0.696667in}}%
\pgfpathlineto{\pgfqpoint{3.084176in}{0.698271in}}%
\pgfpathlineto{\pgfqpoint{3.084696in}{0.698813in}}%
\pgfpathlineto{\pgfqpoint{3.085216in}{0.697286in}}%
\pgfpathlineto{\pgfqpoint{3.088161in}{0.690529in}}%
\pgfpathlineto{\pgfqpoint{3.088680in}{0.692017in}}%
\pgfpathlineto{\pgfqpoint{3.090239in}{0.700571in}}%
\pgfpathlineto{\pgfqpoint{3.091105in}{0.696805in}}%
\pgfpathlineto{\pgfqpoint{3.091798in}{0.694737in}}%
\pgfpathlineto{\pgfqpoint{3.092491in}{0.697260in}}%
\pgfpathlineto{\pgfqpoint{3.093184in}{0.699301in}}%
\pgfpathlineto{\pgfqpoint{3.093704in}{0.697215in}}%
\pgfpathlineto{\pgfqpoint{3.095609in}{0.679675in}}%
\pgfpathlineto{\pgfqpoint{3.096648in}{0.682589in}}%
\pgfpathlineto{\pgfqpoint{3.099766in}{0.695853in}}%
\pgfpathlineto{\pgfqpoint{3.099940in}{0.695814in}}%
\pgfpathlineto{\pgfqpoint{3.101499in}{0.694913in}}%
\pgfpathlineto{\pgfqpoint{3.101672in}{0.695344in}}%
\pgfpathlineto{\pgfqpoint{3.104270in}{0.703240in}}%
\pgfpathlineto{\pgfqpoint{3.104616in}{0.702832in}}%
\pgfpathlineto{\pgfqpoint{3.107561in}{0.692386in}}%
\pgfpathlineto{\pgfqpoint{3.108774in}{0.687586in}}%
\pgfpathlineto{\pgfqpoint{3.109467in}{0.689826in}}%
\pgfpathlineto{\pgfqpoint{3.111719in}{0.697711in}}%
\pgfpathlineto{\pgfqpoint{3.112065in}{0.697144in}}%
\pgfpathlineto{\pgfqpoint{3.116569in}{0.681829in}}%
\pgfpathlineto{\pgfqpoint{3.117435in}{0.684875in}}%
\pgfpathlineto{\pgfqpoint{3.119340in}{0.692778in}}%
\pgfpathlineto{\pgfqpoint{3.120206in}{0.692385in}}%
\pgfpathlineto{\pgfqpoint{3.120899in}{0.692659in}}%
\pgfpathlineto{\pgfqpoint{3.121072in}{0.692073in}}%
\pgfpathlineto{\pgfqpoint{3.123497in}{0.674727in}}%
\pgfpathlineto{\pgfqpoint{3.124710in}{0.679740in}}%
\pgfpathlineto{\pgfqpoint{3.126789in}{0.692445in}}%
\pgfpathlineto{\pgfqpoint{3.127482in}{0.691717in}}%
\pgfpathlineto{\pgfqpoint{3.131466in}{0.685082in}}%
\pgfpathlineto{\pgfqpoint{3.132159in}{0.686750in}}%
\pgfpathlineto{\pgfqpoint{3.132332in}{0.687026in}}%
\pgfpathlineto{\pgfqpoint{3.132851in}{0.685562in}}%
\pgfpathlineto{\pgfqpoint{3.134064in}{0.680600in}}%
\pgfpathlineto{\pgfqpoint{3.134757in}{0.684091in}}%
\pgfpathlineto{\pgfqpoint{3.136662in}{0.691539in}}%
\pgfpathlineto{\pgfqpoint{3.137182in}{0.690480in}}%
\pgfpathlineto{\pgfqpoint{3.139434in}{0.671517in}}%
\pgfpathlineto{\pgfqpoint{3.140993in}{0.677175in}}%
\pgfpathlineto{\pgfqpoint{3.144630in}{0.694597in}}%
\pgfpathlineto{\pgfqpoint{3.145843in}{0.693058in}}%
\pgfpathlineto{\pgfqpoint{3.147055in}{0.688600in}}%
\pgfpathlineto{\pgfqpoint{3.147922in}{0.690086in}}%
\pgfpathlineto{\pgfqpoint{3.148961in}{0.695578in}}%
\pgfpathlineto{\pgfqpoint{3.149654in}{0.691443in}}%
\pgfpathlineto{\pgfqpoint{3.152599in}{0.677823in}}%
\pgfpathlineto{\pgfqpoint{3.154157in}{0.669752in}}%
\pgfpathlineto{\pgfqpoint{3.155197in}{0.671952in}}%
\pgfpathlineto{\pgfqpoint{3.155716in}{0.672432in}}%
\pgfpathlineto{\pgfqpoint{3.156409in}{0.671128in}}%
\pgfpathlineto{\pgfqpoint{3.156929in}{0.671800in}}%
\pgfpathlineto{\pgfqpoint{3.157622in}{0.673762in}}%
\pgfpathlineto{\pgfqpoint{3.158315in}{0.672116in}}%
\pgfpathlineto{\pgfqpoint{3.159354in}{0.669406in}}%
\pgfpathlineto{\pgfqpoint{3.160047in}{0.670993in}}%
\pgfpathlineto{\pgfqpoint{3.164204in}{0.681925in}}%
\pgfpathlineto{\pgfqpoint{3.164897in}{0.686467in}}%
\pgfpathlineto{\pgfqpoint{3.164897in}{0.686467in}}%
\pgfusepath{stroke}%
\end{pgfscope}%
\begin{pgfscope}%
\pgfpathrectangle{\pgfqpoint{0.568671in}{0.451277in}}{\pgfqpoint{2.598305in}{0.798874in}}%
\pgfusepath{clip}%
\pgfsetbuttcap%
\pgfsetroundjoin%
\definecolor{currentfill}{rgb}{0.000000,0.000000,0.000000}%
\pgfsetfillcolor{currentfill}%
\pgfsetfillopacity{0.000000}%
\pgfsetlinewidth{1.003750pt}%
\definecolor{currentstroke}{rgb}{0.121569,0.466667,0.705882}%
\pgfsetstrokecolor{currentstroke}%
\pgfsetdash{}{0pt}%
\pgfsys@defobject{currentmarker}{\pgfqpoint{-0.027778in}{-0.027778in}}{\pgfqpoint{0.027778in}{0.027778in}}{%
\pgfpathmoveto{\pgfqpoint{0.000000in}{-0.027778in}}%
\pgfpathcurveto{\pgfqpoint{0.007367in}{-0.027778in}}{\pgfqpoint{0.014433in}{-0.024851in}}{\pgfqpoint{0.019642in}{-0.019642in}}%
\pgfpathcurveto{\pgfqpoint{0.024851in}{-0.014433in}}{\pgfqpoint{0.027778in}{-0.007367in}}{\pgfqpoint{0.027778in}{0.000000in}}%
\pgfpathcurveto{\pgfqpoint{0.027778in}{0.007367in}}{\pgfqpoint{0.024851in}{0.014433in}}{\pgfqpoint{0.019642in}{0.019642in}}%
\pgfpathcurveto{\pgfqpoint{0.014433in}{0.024851in}}{\pgfqpoint{0.007367in}{0.027778in}}{\pgfqpoint{0.000000in}{0.027778in}}%
\pgfpathcurveto{\pgfqpoint{-0.007367in}{0.027778in}}{\pgfqpoint{-0.014433in}{0.024851in}}{\pgfqpoint{-0.019642in}{0.019642in}}%
\pgfpathcurveto{\pgfqpoint{-0.024851in}{0.014433in}}{\pgfqpoint{-0.027778in}{0.007367in}}{\pgfqpoint{-0.027778in}{0.000000in}}%
\pgfpathcurveto{\pgfqpoint{-0.027778in}{-0.007367in}}{\pgfqpoint{-0.024851in}{-0.014433in}}{\pgfqpoint{-0.019642in}{-0.019642in}}%
\pgfpathcurveto{\pgfqpoint{-0.014433in}{-0.024851in}}{\pgfqpoint{-0.007367in}{-0.027778in}}{\pgfqpoint{0.000000in}{-0.027778in}}%
\pgfpathlineto{\pgfqpoint{0.000000in}{-0.027778in}}%
\pgfpathclose%
\pgfusepath{stroke,fill}%
}%
\begin{pgfscope}%
\pgfsys@transformshift{0.572136in}{1.365470in}%
\pgfsys@useobject{currentmarker}{}%
\end{pgfscope}%
\begin{pgfscope}%
\pgfsys@transformshift{0.658746in}{1.251514in}%
\pgfsys@useobject{currentmarker}{}%
\end{pgfscope}%
\begin{pgfscope}%
\pgfsys@transformshift{0.745356in}{1.088110in}%
\pgfsys@useobject{currentmarker}{}%
\end{pgfscope}%
\begin{pgfscope}%
\pgfsys@transformshift{0.831966in}{1.004344in}%
\pgfsys@useobject{currentmarker}{}%
\end{pgfscope}%
\begin{pgfscope}%
\pgfsys@transformshift{0.918576in}{0.945722in}%
\pgfsys@useobject{currentmarker}{}%
\end{pgfscope}%
\begin{pgfscope}%
\pgfsys@transformshift{1.005186in}{0.812831in}%
\pgfsys@useobject{currentmarker}{}%
\end{pgfscope}%
\begin{pgfscope}%
\pgfsys@transformshift{1.091796in}{0.734395in}%
\pgfsys@useobject{currentmarker}{}%
\end{pgfscope}%
\begin{pgfscope}%
\pgfsys@transformshift{1.178407in}{0.686625in}%
\pgfsys@useobject{currentmarker}{}%
\end{pgfscope}%
\begin{pgfscope}%
\pgfsys@transformshift{1.265017in}{0.693515in}%
\pgfsys@useobject{currentmarker}{}%
\end{pgfscope}%
\begin{pgfscope}%
\pgfsys@transformshift{1.351627in}{0.652303in}%
\pgfsys@useobject{currentmarker}{}%
\end{pgfscope}%
\begin{pgfscope}%
\pgfsys@transformshift{1.438237in}{0.781013in}%
\pgfsys@useobject{currentmarker}{}%
\end{pgfscope}%
\begin{pgfscope}%
\pgfsys@transformshift{1.524847in}{0.679768in}%
\pgfsys@useobject{currentmarker}{}%
\end{pgfscope}%
\begin{pgfscope}%
\pgfsys@transformshift{1.611457in}{0.620404in}%
\pgfsys@useobject{currentmarker}{}%
\end{pgfscope}%
\begin{pgfscope}%
\pgfsys@transformshift{1.698068in}{0.597305in}%
\pgfsys@useobject{currentmarker}{}%
\end{pgfscope}%
\begin{pgfscope}%
\pgfsys@transformshift{1.784678in}{0.582352in}%
\pgfsys@useobject{currentmarker}{}%
\end{pgfscope}%
\begin{pgfscope}%
\pgfsys@transformshift{1.871288in}{0.583394in}%
\pgfsys@useobject{currentmarker}{}%
\end{pgfscope}%
\begin{pgfscope}%
\pgfsys@transformshift{1.957898in}{0.587189in}%
\pgfsys@useobject{currentmarker}{}%
\end{pgfscope}%
\begin{pgfscope}%
\pgfsys@transformshift{2.044508in}{0.551141in}%
\pgfsys@useobject{currentmarker}{}%
\end{pgfscope}%
\begin{pgfscope}%
\pgfsys@transformshift{2.131118in}{0.610560in}%
\pgfsys@useobject{currentmarker}{}%
\end{pgfscope}%
\begin{pgfscope}%
\pgfsys@transformshift{2.217728in}{0.590690in}%
\pgfsys@useobject{currentmarker}{}%
\end{pgfscope}%
\begin{pgfscope}%
\pgfsys@transformshift{2.304339in}{0.811826in}%
\pgfsys@useobject{currentmarker}{}%
\end{pgfscope}%
\begin{pgfscope}%
\pgfsys@transformshift{2.390949in}{0.933621in}%
\pgfsys@useobject{currentmarker}{}%
\end{pgfscope}%
\begin{pgfscope}%
\pgfsys@transformshift{2.477559in}{0.768193in}%
\pgfsys@useobject{currentmarker}{}%
\end{pgfscope}%
\begin{pgfscope}%
\pgfsys@transformshift{2.564169in}{0.749300in}%
\pgfsys@useobject{currentmarker}{}%
\end{pgfscope}%
\begin{pgfscope}%
\pgfsys@transformshift{2.650779in}{0.703799in}%
\pgfsys@useobject{currentmarker}{}%
\end{pgfscope}%
\begin{pgfscope}%
\pgfsys@transformshift{2.737389in}{0.687053in}%
\pgfsys@useobject{currentmarker}{}%
\end{pgfscope}%
\begin{pgfscope}%
\pgfsys@transformshift{2.824000in}{0.696271in}%
\pgfsys@useobject{currentmarker}{}%
\end{pgfscope}%
\begin{pgfscope}%
\pgfsys@transformshift{2.910610in}{0.676154in}%
\pgfsys@useobject{currentmarker}{}%
\end{pgfscope}%
\begin{pgfscope}%
\pgfsys@transformshift{2.997220in}{0.695448in}%
\pgfsys@useobject{currentmarker}{}%
\end{pgfscope}%
\begin{pgfscope}%
\pgfsys@transformshift{3.083830in}{0.697609in}%
\pgfsys@useobject{currentmarker}{}%
\end{pgfscope}%
\end{pgfscope}%
\begin{pgfscope}%
\pgfsetrectcap%
\pgfsetmiterjoin%
\pgfsetlinewidth{0.803000pt}%
\definecolor{currentstroke}{rgb}{0.000000,0.000000,0.000000}%
\pgfsetstrokecolor{currentstroke}%
\pgfsetdash{}{0pt}%
\pgfpathmoveto{\pgfqpoint{0.568671in}{0.451277in}}%
\pgfpathlineto{\pgfqpoint{0.568671in}{1.250151in}}%
\pgfusepath{stroke}%
\end{pgfscope}%
\begin{pgfscope}%
\pgfsetrectcap%
\pgfsetmiterjoin%
\pgfsetlinewidth{0.803000pt}%
\definecolor{currentstroke}{rgb}{0.000000,0.000000,0.000000}%
\pgfsetstrokecolor{currentstroke}%
\pgfsetdash{}{0pt}%
\pgfpathmoveto{\pgfqpoint{3.166976in}{0.451277in}}%
\pgfpathlineto{\pgfqpoint{3.166976in}{1.250151in}}%
\pgfusepath{stroke}%
\end{pgfscope}%
\begin{pgfscope}%
\pgfsetrectcap%
\pgfsetmiterjoin%
\pgfsetlinewidth{0.803000pt}%
\definecolor{currentstroke}{rgb}{0.000000,0.000000,0.000000}%
\pgfsetstrokecolor{currentstroke}%
\pgfsetdash{}{0pt}%
\pgfpathmoveto{\pgfqpoint{0.568671in}{0.451277in}}%
\pgfpathlineto{\pgfqpoint{3.166976in}{0.451277in}}%
\pgfusepath{stroke}%
\end{pgfscope}%
\begin{pgfscope}%
\pgfsetrectcap%
\pgfsetmiterjoin%
\pgfsetlinewidth{0.803000pt}%
\definecolor{currentstroke}{rgb}{0.000000,0.000000,0.000000}%
\pgfsetstrokecolor{currentstroke}%
\pgfsetdash{}{0pt}%
\pgfpathmoveto{\pgfqpoint{0.568671in}{1.250151in}}%
\pgfpathlineto{\pgfqpoint{3.166976in}{1.250151in}}%
\pgfusepath{stroke}%
\end{pgfscope}%
\begin{pgfscope}%
\pgfsetbuttcap%
\pgfsetmiterjoin%
\definecolor{currentfill}{rgb}{1.000000,1.000000,1.000000}%
\pgfsetfillcolor{currentfill}%
\pgfsetfillopacity{0.800000}%
\pgfsetlinewidth{1.003750pt}%
\definecolor{currentstroke}{rgb}{0.800000,0.800000,0.800000}%
\pgfsetstrokecolor{currentstroke}%
\pgfsetstrokeopacity{0.800000}%
\pgfsetdash{}{0pt}%
\pgfpathmoveto{\pgfqpoint{0.903385in}{1.036802in}}%
\pgfpathlineto{\pgfqpoint{2.832262in}{1.036802in}}%
\pgfpathquadraticcurveto{\pgfqpoint{2.851706in}{1.036802in}}{\pgfqpoint{2.851706in}{1.056247in}}%
\pgfpathlineto{\pgfqpoint{2.851706in}{1.182095in}}%
\pgfpathquadraticcurveto{\pgfqpoint{2.851706in}{1.201540in}}{\pgfqpoint{2.832262in}{1.201540in}}%
\pgfpathlineto{\pgfqpoint{0.903385in}{1.201540in}}%
\pgfpathquadraticcurveto{\pgfqpoint{0.883941in}{1.201540in}}{\pgfqpoint{0.883941in}{1.182095in}}%
\pgfpathlineto{\pgfqpoint{0.883941in}{1.056247in}}%
\pgfpathquadraticcurveto{\pgfqpoint{0.883941in}{1.036802in}}{\pgfqpoint{0.903385in}{1.036802in}}%
\pgfpathlineto{\pgfqpoint{0.903385in}{1.036802in}}%
\pgfpathclose%
\pgfusepath{stroke,fill}%
\end{pgfscope}%
\begin{pgfscope}%
\pgfsetrectcap%
\pgfsetroundjoin%
\pgfsetlinewidth{1.505625pt}%
\definecolor{currentstroke}{rgb}{0.000000,0.000000,0.000000}%
\pgfsetstrokecolor{currentstroke}%
\pgfsetdash{}{0pt}%
\pgfpathmoveto{\pgfqpoint{0.922830in}{1.128623in}}%
\pgfpathlineto{\pgfqpoint{1.020052in}{1.128623in}}%
\pgfpathlineto{\pgfqpoint{1.117274in}{1.128623in}}%
\pgfusepath{stroke}%
\end{pgfscope}%
\begin{pgfscope}%
\definecolor{textcolor}{rgb}{0.000000,0.000000,0.000000}%
\pgfsetstrokecolor{textcolor}%
\pgfsetfillcolor{textcolor}%
\pgftext[x=1.195052in,y=1.094595in,left,base]{\color{textcolor}\rmfamily\fontsize{7.000000}{8.400000}\selectfont optimal}%
\end{pgfscope}%
\begin{pgfscope}%
\pgfsetrectcap%
\pgfsetroundjoin%
\pgfsetlinewidth{1.505625pt}%
\definecolor{currentstroke}{rgb}{0.121569,0.466667,0.705882}%
\pgfsetstrokecolor{currentstroke}%
\pgfsetdash{}{0pt}%
\pgfpathmoveto{\pgfqpoint{1.613107in}{1.128623in}}%
\pgfpathlineto{\pgfqpoint{1.710330in}{1.128623in}}%
\pgfpathlineto{\pgfqpoint{1.807552in}{1.128623in}}%
\pgfusepath{stroke}%
\end{pgfscope}%
\begin{pgfscope}%
\pgfsetbuttcap%
\pgfsetroundjoin%
\definecolor{currentfill}{rgb}{0.000000,0.000000,0.000000}%
\pgfsetfillcolor{currentfill}%
\pgfsetfillopacity{0.000000}%
\pgfsetlinewidth{1.003750pt}%
\definecolor{currentstroke}{rgb}{0.121569,0.466667,0.705882}%
\pgfsetstrokecolor{currentstroke}%
\pgfsetdash{}{0pt}%
\pgfsys@defobject{currentmarker}{\pgfqpoint{-0.027778in}{-0.027778in}}{\pgfqpoint{0.027778in}{0.027778in}}{%
\pgfpathmoveto{\pgfqpoint{0.000000in}{-0.027778in}}%
\pgfpathcurveto{\pgfqpoint{0.007367in}{-0.027778in}}{\pgfqpoint{0.014433in}{-0.024851in}}{\pgfqpoint{0.019642in}{-0.019642in}}%
\pgfpathcurveto{\pgfqpoint{0.024851in}{-0.014433in}}{\pgfqpoint{0.027778in}{-0.007367in}}{\pgfqpoint{0.027778in}{0.000000in}}%
\pgfpathcurveto{\pgfqpoint{0.027778in}{0.007367in}}{\pgfqpoint{0.024851in}{0.014433in}}{\pgfqpoint{0.019642in}{0.019642in}}%
\pgfpathcurveto{\pgfqpoint{0.014433in}{0.024851in}}{\pgfqpoint{0.007367in}{0.027778in}}{\pgfqpoint{0.000000in}{0.027778in}}%
\pgfpathcurveto{\pgfqpoint{-0.007367in}{0.027778in}}{\pgfqpoint{-0.014433in}{0.024851in}}{\pgfqpoint{-0.019642in}{0.019642in}}%
\pgfpathcurveto{\pgfqpoint{-0.024851in}{0.014433in}}{\pgfqpoint{-0.027778in}{0.007367in}}{\pgfqpoint{-0.027778in}{0.000000in}}%
\pgfpathcurveto{\pgfqpoint{-0.027778in}{-0.007367in}}{\pgfqpoint{-0.024851in}{-0.014433in}}{\pgfqpoint{-0.019642in}{-0.019642in}}%
\pgfpathcurveto{\pgfqpoint{-0.014433in}{-0.024851in}}{\pgfqpoint{-0.007367in}{-0.027778in}}{\pgfqpoint{0.000000in}{-0.027778in}}%
\pgfpathlineto{\pgfqpoint{0.000000in}{-0.027778in}}%
\pgfpathclose%
\pgfusepath{stroke,fill}%
}%
\begin{pgfscope}%
\pgfsys@transformshift{1.710330in}{1.128623in}%
\pgfsys@useobject{currentmarker}{}%
\end{pgfscope}%
\end{pgfscope}%
\begin{pgfscope}%
\definecolor{textcolor}{rgb}{0.000000,0.000000,0.000000}%
\pgfsetstrokecolor{textcolor}%
\pgfsetfillcolor{textcolor}%
\pgftext[x=1.885330in,y=1.094595in,left,base]{\color{textcolor}\rmfamily\fontsize{7.000000}{8.400000}\selectfont adaptive \(\displaystyle \gamma_i,\,K=1\)}%
\end{pgfscope}%
\end{pgfpicture}%
\makeatother%
\endgroup%

    \vspace*{-0.6cm}
    \caption[]{Results of 30 Monte-Carlo runs: Top: Estimated \(\|\h^n\|\) over time, shown is the mean. Middle: Estimated \(\|\hat{\h}_i^n\|\) over time, shown is the mean. Dot-dashed lines are true values. Bottom: Median NPM, comparing optimal algorithm and proposed extended algorithm. Dashed vertical lines indicate rescaling events.}
    \label{fig:simulations:NPMtimedyn}
\end{figure}

\section[]{Conclusions}
\label{sec:conclusions}
In this contribution, we propose an extension to a distributed adaptive BSI algorithm applied in sensor networks.
Based on distributed averaging, only using the information provided by neighboring nodes in the sensor network, we compute estimates of norm values of channel estimates in order to enforce a norm constraint.
By balancing the averaging with introducing new data into the recursion, we allow the algorithm to follow an adaptive updating scheme.
The mixing factor, which balances new data with recursion data, is set adaptively dependent on instantaneous channel estimate norms.
These introductions reduce inter-node transmissions for each time frame while still delivering steady-state estimation performance close to the optimal case where network-wide information is available at all nodes.
The reduction that can be achieved goes as far as only needing a single iteration and, therefore, one additional inter-neighborhood information exchange per time frame.
We show this in simulations with a white Gaussian input signal and random impulse responses drawn from the normal distribution.
\begin{todo}
    Could use the little space that's left for an illustration in the beginning or something. Or add a comparison of numbers put into \autoref{tab:transcost:table} for different setups varying \(M,N_i,K\).
\end{todo}

% Below is an example of how to insert images. Delete the ``\vspace'' line,
% uncomment the preceding line ``\centerline...'' and replace ``imageX.ps''
% with a suitable PostScript file name.
% -------------------------------------------------------------------------
% \begin{figure}[htb]

% \begin{minipage}[b]{1.0\linewidth}
%   \centering
%   \centerline{\includegraphics[width=8.5cm]{image1}}
% %  \vspace{2.0cm}
%   \centerline{(a) Result 1}\medskip
% \end{minipage}
% %
% \begin{minipage}[b]{.48\linewidth}
%   \centering
%   \centerline{\includegraphics[width=4.0cm]{image3}}
% %  \vspace{1.5cm}
%   \centerline{(b) Results 3}\medskip
% \end{minipage}
% \hill
% \begin{minipage}[b]{0.48\linewidth}
%   \centering
%   \centerline{\includegraphics[width=4.0cm]{image4}}
% %  \vspace{1.5cm}
%   \centerline{(c) Result 4}\medskip
% \end{minipage}
% %
% \caption{Example of placing a figure with experimental results.}
% \label{fig:res}
% %
% \end{figure}

% \vfill\pagebreak

% References should be produced using the bibtex program from suitable
% BiBTeX files (here: strings, refs, manuals). The IEEEbib.bst bibliography
% style file from IEEE produces unsorted bibliography list.
% -------------------------------------------------------------------------
\bibliographystyle{IEEEbib}
\bibliography{refs}

\end{document}
