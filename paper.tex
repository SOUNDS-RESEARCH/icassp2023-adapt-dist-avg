% Template for ICASSP-2021 paper; to be used with:
%          spconf.sty  - ICASSP/ICIP LaTeX style file, and
%          IEEEbib.bst - IEEE bibliography style file.
% --------------------------------------------------------------------------
\documentclass{article}
\usepackage{spconf,amsmath,graphicx}
\usepackage[dvipsnames]{xcolor}
\usepackage[]{pgf}
\usepackage{tikz}
\usetikzlibrary{shapes,arrows,snakes,backgrounds,matrix,patterns,positioning,fadings}
\usepackage[mode=buildnew]{standalone}
\usepackage{subfig}
\usepackage{upgreek}
\usepackage{nicefrac}
\usepackage{dsfont}
% \usepackage[OT1]{fontenc}
% \renewcommand*\familydefault{\sfdefault}
\usepackage{bm}
\usepackage{cancel}
\usepackage{amsbsy}
\usepackage{amssymb}
\usepackage{algorithm}
\usepackage{algpseudocode}
\usepackage{lipsum}
\usepackage{harpoon}
\usepackage[percent]{overpic}
\usepackage{comment}
\newenvironment{note}
	{\par\textcolor{Blue}{\bfseries Note:} \color{Blue}\ignorespaces}
	{\par}
\newenvironment{attention}
	{\par\textcolor{red}{\bfseries Attention:} \color{red}\ignorespaces}
	{\par}
\newenvironment{todo}
	{\par\textcolor{red}{\bfseries TODO:} \color{red}\ignorespaces}
	{\par}
\usepackage{hyperref}
\hypersetup{
	colorlinks,
	linkcolor={blue!80!black},
	citecolor={blue!80!black},
	urlcolor={blue!80!black}
}
\usepackage{algorithm}
\usepackage{algpseudocode}
\usepackage[short]{optidef}

\renewcommand{\chapterautorefname}{Ch.}
\renewcommand{\sectionautorefname}{Sec.}
\renewcommand{\subsectionautorefname}{Sec.}
\renewcommand{\figureautorefname}{Fig.}
\newcommand{\subfigureautorefname}{\figureautorefname}
\newcommand{\algorithmautorefname}{Alg.}

\newcommand{\mtxb}[1]{\bm{\mathrm{#1}}}
\newcommand{\T}{{\mathrm{T}}}
\newcommand{\herm}{{\mathrm{H}}}
\newcommand{\ev}[1]{\mathrm{E} \left\lbrace #1 \right\rbrace}

% Common variables
\newcommand{\h}{\mtxb{h}}
\newcommand{\x}{\mtxb{x}}
\newcommand{\R}{\mtxb{R}}
\newcommand{\W}{\mtxb{W}}
\newcommand{\w}{\mtxb{w}}
\newcommand{\z}{\mtxb{z}}
\newcommand{\y}{\mtxb{y}}
\newcommand{\uu}{\mtxb{u}}
\newcommand{\aRho}{\mtxb{P}}
\newcommand{\hf}{{\bm{h}}}
\newcommand{\xf}{{\bm{x}}}
\newcommand{\Rf}{{\bm{{R}}}}
\newcommand{\Wf}{{\bm{{W}}}}
\newcommand{\Df}{{\bm{{D}}}}
\newcommand{\wf}{{\bm{w}}}
\newcommand{\zf}{{\bm{z}}}
\newcommand{\uuf}{{\bm{u}}}
\newcommand{\yf}{{\bm{y}}}
\newcommand{\aRhof}{{\bm{{P}}}}
\newcommand{\I}{\mtxb{I}}
\newcommand{\Cset}{\mathcal{C}}
\newcommand{\Mset}{\mathcal{M}}
\newcommand{\Tset}{\mathcal{T}}
\newcommand{\Rset}{\mathcal{R}}


\setlength{\abovedisplayskip}{0pt}
\setlength{\belowdisplayskip}{0pt}

% Title.
% ------
\title{Distributed Adaptive Norm Estimation For Blind System\\Identification in Wireless Sensor Networks}
%
% Single address.
% ---------------
\name{M. Blochberger\(^1\), F. Elvander\(^2\), R. Ali\(^1\), J. Østergaard\(^3\), J. Jensen\(^3\),  M. Moonen\(^1\), T. van Waterschoot\(^1\)\thanks{This research work was carried out at the ESAT Laboratory of KU Leuven, in the frame of the SOUNDS European Training Network. This project has received funding from the European Union's Horizon 2020 research and innovation programme under the Marie Skłodowska-Curie grant agreement No.\,956369. This research received funding in part from the Research Foundation - Flanders (FWO) grant 12ZD622N as well as from the European Union's Horizon 2020 research and innovation program / ERC Consolidator Grant: SONORA (No.\,773268). This paper reflects only the authors' views and the Union is not liable for any use that may be made of the contained information. Source code available at \textcolor{red}{ADD URL}}}
\address{\(^1\)KU Leuven, Dept. of Electrical Engineering (ESAT), STADIUS, 3001 Leuven, Belgium\\\(^2\)Aalto University, Dept. of Signal Processing and Acoustics, 02150 Espoo, Finland\\\(^3\)Aalborg University, Dept. of Electronic Systems, 9220 Aalborg, Denmark}

\begin{document}
\ninept
%
\maketitle
%
\begin{abstract}
    Distributed signal-processing algorithms in (wireless) sensor networks often aim to decentralize processing tasks to reduce communication cost and computational complexity or avoid reliance on a single device (i.e., fusion center) for processing.
    In this contribution, we extend a distributed adaptive algorithm for blind system identification that relies on the estimation of a stacked network-wide consensus vector at each node, the computation of which requires either broadcasting or relaying of node-specific values (i.e., local vector norms) to all other nodes.
    The extended algorithm employs a distributed-averaging-based scheme to estimate the network-wide consensus norm value by only using the local vector norm provided by neighboring sensor nodes.
    We introduce an adaptive mixing factor between instantaneous and recursive estimates of these norms for adaptivity in a time-varying system.
    Simulation results show that the extension provides estimation results close to the optimal fully-connected-network or broadcasting case while reducing inter-node transmission significantly.
\end{abstract}
%
\begin{keywords}
multi-channel signal processing, distributed signal processing, wireless sensor networks, blind system identification, distributed averaging
\end{keywords}
%
\section{INTRODUCTION}
\label{sec:intro}

Distributed algorithms have been an active area of research for quite some time, with numerous control, optimization, and signal processing applications.
With the ever-growing number of smart multimedia devices in today's surroundings providing ubiquitous processing and communication capabilities, distributed audio and speech signal processing have also found their way into the spotlight.
Algorithms for distributed signal estimation \cite{5483092}, noise control and echo cancellation \cite{9670697}, as well as beamforming \cite{6663655,6329934,MARKOVICHGOLAN20154,6309434} to name a few, have been proposed.
Another important task in audio and communication applications, which however has not recived as much attention, is multi-channel system identification, i.e. estimating channel responses in the time or frequency domain.
Distributed single-input-multiple-output (SIMO) blind system identification (BSI) has been addressed with adaptive cross-relation-based (CR) algorithms \cite{yuDistributedBlindSystem2014, liuDistributedBlindIdentification2016}.
In this context, we recently introduced an adaptive CR-based algorithm \cite{blochbergerDBSI} using the alternating direction method of multipliers (ADMM) \cite{boydDistributedOptimizationStatistical2011}.

All of the aforementioned distributed BSI algorithms rely on shared information between neighboring sensor nodes within the network.
However, the CR-based BSI task necessitates a non-triviality constraint on the full system to be identified (we refer the reader to, e.g., \cite{huangAdaptiveMultichannelLeast2002,huangClassFrequencydomainAdaptive2003}), which manifests itself as one or more variables that require network-wide shared information for their computation.
The algorithm in \cite{blochbergerDBSI} relies on node-wise values being relayed throughout the network.
In contrast, to overcome the need for the network to be fully connected, \cite{yuDistributedBlindSystem2014, liuDistributedBlindIdentification2016} use an average consensus \cite{xiaoFastLinearIterations2004} approach where a secondary recursion estimates the global variable for each signal frame.
Both approaches introduce additional transmissions of variables between nodes, the number of which, depending on network and neighborhood size, can be substantial or even unfeasable.

In this paper, we extend our algorithm in \cite{blochbergerDBSI} with a distributed averaging-based \cite{xiaoFastLinearIterations2004} estimation scheme for the global variable.
This approach allows us, similarly to \cite{6334305,9914798}, to compute the global variable without the need of a fully connected network or broadcasting.
We also introduce a mixing factor to include instantaneous values into the averaging recursion, which then allows us (i) to reduce the number of secondary iterations significantly (from 50 in existing approaches, down to 1) and (ii) track time-varying systems.
Simulation results demonstrate how the extension leads to BSI performance close to an optimal case where all information is available without the need of a fully connected network or a large number of estimation iterations.

\section{DISTRIBUTED ADAPTIVE BSI IN SENSOR NETWORKS WITH ONLINE-ADMM}
\label{sec:dbsi}
In this section, we briefly outline the adaptive SIMO BSI algorithm proposed in \cite{blochbergerDBSI}, setting the scene for the distributed averaging scheme proposed herein.
We refer the reader to the publication cited above for the full derivation of the algorithm.

\subsection[]{Sensor network}
We assume a set of nodes with one sensor each, that is, each node provides one signal.
Let this set of nodes be \(\Mset \triangleq \{1,\ldots,M\}\) and \(\mathcal{E}\) be a set of edge tuples, which connect the nodes forming a sensor network.
Each edge is an unordered pair of node indices \(\{i,j\} \in \mathcal{E}\), which represents the communication link between them.
We allow \(\{i,i\} \in \mathcal{E}\), in order to simplify notation later, however, this does not represent a link but rather indicates the obvious fact that node \(i\) has information about itself.
Furthermore, let \(\Nset_i = \{j|\{i,j\} \in \mathcal{E}\}\) denote the neighborhood of node \(i\).
We can define the symmetric adjacency matrix \(\mtxb{C}\) with elements \(C_{ij} = 1\) if \(\{i,j\} \in \mathcal{E}\) and 0 otherwise.
It may be noted that in \cite{blochbergerDBSI}, the pairs \(\{i,j\} \in \mathcal{E}\) are ordered, which yields a directed graph and leads to two sets of neighborhoods (``transmit'' and ``receive'') and therefore a non-symmetric adjacency matrix.
For the sake of brevity, we limit the explanations in this paper to the undirected case.
The case of directed graphs can be extended directly from the presented results.

\subsection[]{Signal model}
We consider a SIMO system with
\begin{align}
    % \textstyle
    \mtxb{s}(n) &= [s(n),\,\ldots,\,s(n-2L+2)]^{\T},\\
    \x_i(n) &= [x_i(n),\,\ldots,\,x_i(n-L+1)]^{\T}, \quad i \in \Mset,
\end{align}
the \(2L \times 1\) input signal frame and \(M\) \(L \times 1\)  output signal frames, respectively.
Each output \(\x_i(n)\) is the convolution of \(\mtxb{s}(n)\) with the respective channel impulse response \(\h_i\) and an additive noise term \(\mtxb{v}_i(n)\), assumed to be zero-mean and uncorrelated with \(\mtxb{s}(n)\).
The signal model is
\begin{equation}
    \x_i(n) = \mtxb{H}_i \mtxb{s}(n) + \mtxb{v}_i(n),
\end{equation}
with \(\mtxb{H}_i\), the \(L \times (2L-1)\) linear convolution matrix of the \(i\)th channel using the elements of \(\h_i\) of length \(L\).
For the purpose of this paper, the length \(L\) of the impulse responses is assumed to be known.

\subsection[]{Distributed BSI with Online-ADMM}
In BSI, the cross-relation problem formulation only uses output signals \(\x_i(n)\), exploiting relative information between them, to identify the system, i.e., the acoustic or communication channels \(\h = [\h_1^\T,\,...,\,\h_M^\T]^\T\).
The solution to this problem is found by the minimization problem \cite{langtongBlindIdentificationEqualization1994,huangAdaptiveMultichannelLeast2002,huangClassFrequencydomainAdaptive2003,blochbergerDBSI}:
\begin{equation}
    \begin{aligned}
        \hat{\h} = \arg \min_{\h} \quad &\h^\herm \R \h \\
        \text{s.t. } \quad &\|\h\| = 1,
    \end{aligned}\label{eq:frequency_domain:min_prob}
\end{equation}
where \(\hat{\h} = [\hat{\h}_1^\T,\,...,\,\hat{\h}_M^\T]^\T\) is the vector of estimated channel responses and \(\R\) is the so-called cross-relation (CR) matrix.
In \cite{blochbergerDBSI}, this problem is solved by a distributed adaptive algorithm that is based on separating the problem \eqref{eq:frequency_domain:min_prob} into node-wise subproblems,
\begin{equation}
    \begin{aligned}
        \underset{\w,\,\h}{\min} \quad &\sum_{i \in \Mset} \w_i^\herm \aRhof_i \w_i\\
        \text{s.t.} \quad &(\w_i)_j = \h_{\mathcal{G}(i,j)}\quad i \in \Mset,\,j \in \Nset_i\\
        &\|\h\| = 1.
    \end{aligned}\label{eq:general_consensus_admm:min_prob}
\end{equation}
Here, each subproblem is represented by a cost function \(\w_i^\herm \aRhof_i \w_i\), with the node-local channel estimates \(\w_i\) and CR matrix \(\aRhof_i\).
Both are analogous to \(\h_i\) and \(\R\) in \eqref{eq:frequency_domain:min_prob}, however, only using a subset of channels, corresponding to \(\Nset_i \subset \Mset\).
The first constraint enforces a consensus of local estimates, and \(\mathcal{G}(i,j)\) is a map equating local estimates of nodes connected by edge \((i,j) \in \mathcal{E}\).
The general-form consensus alternating direction method of multipliers (ADMM) \cite{boydDistributedOptimizationStatistical2011} is applied in an adaptive updating scheme (Online-ADMM) \cite{wangOnlineAlternatingDirection2013,hosseiniOnlineDistributedADMM2014}.
ADMM leads to update steps for local, consensus, and dual variables.
We refer the reader to \cite{blochbergerDBSI} for details.
The algorithm's consensus variable is the system's estimate, \(\hat{\h}\).
Each node \(i\) computes the respective subvector \(\hat{\h}_i\) locally, which is the computation step of interest for this extension.
It is defined as
\begin{equation}
    \hat{\h_i}^{(n)} = \frac{\bar{\h}_i^{(n)}}{\sqrt{\sum_{j \in \Mset} \|\bar{\h}_j^{(n)}\|^2}}\label{eq:online_admm:consensus_update}
\end{equation}
for each node \(i \in \Mset\) with the local unnormalized consensus \(\bar{\h}_i^{(n)}\), a combination of neigborhood averages of local estimates and dual variables, see \cite{blochbergerDBSI}.
The computation of the denominator of \eqref{eq:online_admm:consensus_update} requires  \(\|\bar{\h}_j^{(n)}\|\) from all nodes \(j \in \Mset\), which however, is not possible without a fully-connected network or broadcasting.
In \cite{blochbergerDBSI}, we assume that partial squared norms are relayed throughout the network until all nodes have the information necessary.

In \cite{yuDistributedBlindSystem2014,liuDistributedBlindIdentification2016},
a global value is estimated by a related iterative averaging approach, combining values within neighborhoods until convergence to a network-wide average.

Subsequently, we introduce an extension to the algorithm described in \cite{blochbergerDBSI} using a fastest distributed linear averaging (FDLA) approach \cite{xiaoFastLinearIterations2004} to avoid the need of network-wide data transmission.
The approach described in \cite{yuDistributedBlindSystem2014,liuDistributedBlindIdentification2016} employs a related iterative averaging scheme within a distributed gradient descent algorithm for BSI in order to estimate a global variable, however, necessitating many iterations per frame.
To alleviate this, we further introduce an adaptive mixing factor to include instantaneous values in the recursion to separate avaraging iterations onto time frames of the adaptive algorithm.

\section{Adaptive norm estimation}
\label{sec:adaptivenormest}

\subsection[]{Distributed averaging}
First, assuming we have some unnormalized consensus variables \(\bar{\h}_i\) for \(i \in \Mset\), we look to distributively compute a sum of their squared norms, \(\sum_{j \in \Mset} \|\bar{\h}_j\|^2\), for \eqref{eq:online_admm:consensus_update} at each node \(i\).

Distributed linear combinations allow us to compute an average in a distributed manner, which makes computing the desired sum straightforward.
They generally have the form of
\begin{equation}
    \phi_i({k+1}) = \sum_{j \in \Nset_i} W_{ij} \phi_j({k}),\quad i\in \Mset,\label{eq:adaptivenormest:distlincomb}
\end{equation}
with iteration index \(k=0,...,K\).
This can be written in vector form \(\bm{\phi}({k+1}) = \W \bm{\phi}({k})\) with \(\bm{\phi}(k) = \begin{bmatrix} \phi_1(k) & \ldots & \phi_M(k) \end{bmatrix}^\T\).
We want to find the matrix \(\mtxb{W}\), which will ensure that for any initial value \(\bm{\phi}({0})\), each element of \(\bm{\phi}({k})\) converges to the average of the elements of \(\bm{\phi}({0})\) when \(k \to \infty\).
In this application, the elements of \(\bm{\phi}(0)\) are the initial node-wise squared norm values \(\|\bar{\h}_i\|^2\), which we want to compute the sum of.
This leads to \(\lim_{k \to \infty} M \phi_i(k) = \sum_{j \in \Mset} \|\bar{\h}_j\|^2\) for all \(i \in \Mset\), \cite{xiaoFastLinearIterations2004}.
Computing the desired \(\mtxb{W}\) is done by solving the fastest distributed linear averaging (FDLA) problem, introduced in \cite{xiaoFastLinearIterations2004}, in the form of
\begin{equation}
    \begin{aligned}
        \min& \quad \| \W - \bm{1}\bm{1}^\T/M\|_2\\
        \text{s.t.}& \quad \W \in \mathcal{C},\, \bm{1}^\T \W = \bm{1}^\T,\, \W \bm{1}= \bm{1}.
    \end{aligned}\label{eq:adaptivenormest:fdlaminprob}
\end{equation}
Here, the set \(\mathcal{C} = \{\W \in \mathbb{R}^{M \times M} \vert W_{ij} = 0 \text{ if } \{i,j\} \notin \mathcal{E}\}\) describes all matrices with the same sparsity pattern as the adjacency matrix \(\mtxb{C}\).
Now, the \(i\)-th row of \(\W\) contains the neigborhood weights for node \(i\) as non-zero entries.
It is to note that this has to be done only once for any fixed-topology network \(\Mset\) and \(\mathcal{E}\).
\subsection[]{Proposed adaptive estimation}
\begin{figure}[t]
    \centering
    %% Creator: Matplotlib, PGF backend
%%
%% To include the figure in your LaTeX document, write
%%   \input{<filename>.pgf}
%%
%% Make sure the required packages are loaded in your preamble
%%   \usepackage{pgf}
%%
%% Also ensure that all the required font packages are loaded; for instance,
%% the lmodern package is sometimes necessary when using math font.
%%   \usepackage{lmodern}
%%
%% Figures using additional raster images can only be included by \input if
%% they are in the same directory as the main LaTeX file. For loading figures
%% from other directories you can use the `import` package
%%   \usepackage{import}
%%
%% and then include the figures with
%%   \import{<path to file>}{<filename>.pgf}
%%
%% Matplotlib used the following preamble
%%   \usepackage{fontspec}
%%
\begingroup%
\makeatletter%
\begin{pgfpicture}%
\pgfpathrectangle{\pgfpointorigin}{\pgfqpoint{3.390065in}{2.095175in}}%
\pgfusepath{use as bounding box, clip}%
\begin{pgfscope}%
\pgfsetbuttcap%
\pgfsetmiterjoin%
\definecolor{currentfill}{rgb}{1.000000,1.000000,1.000000}%
\pgfsetfillcolor{currentfill}%
\pgfsetlinewidth{0.000000pt}%
\definecolor{currentstroke}{rgb}{1.000000,1.000000,1.000000}%
\pgfsetstrokecolor{currentstroke}%
\pgfsetstrokeopacity{0.000000}%
\pgfsetdash{}{0pt}%
\pgfpathmoveto{\pgfqpoint{0.000000in}{0.000000in}}%
\pgfpathlineto{\pgfqpoint{3.390065in}{0.000000in}}%
\pgfpathlineto{\pgfqpoint{3.390065in}{2.095175in}}%
\pgfpathlineto{\pgfqpoint{0.000000in}{2.095175in}}%
\pgfpathlineto{\pgfqpoint{0.000000in}{0.000000in}}%
\pgfpathclose%
\pgfusepath{fill}%
\end{pgfscope}%
\begin{pgfscope}%
\pgfsetbuttcap%
\pgfsetmiterjoin%
\definecolor{currentfill}{rgb}{1.000000,1.000000,1.000000}%
\pgfsetfillcolor{currentfill}%
\pgfsetlinewidth{0.000000pt}%
\definecolor{currentstroke}{rgb}{0.000000,0.000000,0.000000}%
\pgfsetstrokecolor{currentstroke}%
\pgfsetstrokeopacity{0.000000}%
\pgfsetdash{}{0pt}%
\pgfpathmoveto{\pgfqpoint{0.608025in}{0.484444in}}%
\pgfpathlineto{\pgfqpoint{3.320621in}{0.484444in}}%
\pgfpathlineto{\pgfqpoint{3.320621in}{2.025731in}}%
\pgfpathlineto{\pgfqpoint{0.608025in}{2.025731in}}%
\pgfpathlineto{\pgfqpoint{0.608025in}{0.484444in}}%
\pgfpathclose%
\pgfusepath{fill}%
\end{pgfscope}%
\begin{pgfscope}%
\pgfpathrectangle{\pgfqpoint{0.608025in}{0.484444in}}{\pgfqpoint{2.712595in}{1.541287in}}%
\pgfusepath{clip}%
\pgfsetrectcap%
\pgfsetroundjoin%
\pgfsetlinewidth{0.803000pt}%
\definecolor{currentstroke}{rgb}{0.690196,0.690196,0.690196}%
\pgfsetstrokecolor{currentstroke}%
\pgfsetdash{}{0pt}%
\pgfpathmoveto{\pgfqpoint{0.731325in}{0.484444in}}%
\pgfpathlineto{\pgfqpoint{0.731325in}{2.025731in}}%
\pgfusepath{stroke}%
\end{pgfscope}%
\begin{pgfscope}%
\pgfsetbuttcap%
\pgfsetroundjoin%
\definecolor{currentfill}{rgb}{0.000000,0.000000,0.000000}%
\pgfsetfillcolor{currentfill}%
\pgfsetlinewidth{0.803000pt}%
\definecolor{currentstroke}{rgb}{0.000000,0.000000,0.000000}%
\pgfsetstrokecolor{currentstroke}%
\pgfsetdash{}{0pt}%
\pgfsys@defobject{currentmarker}{\pgfqpoint{0.000000in}{-0.048611in}}{\pgfqpoint{0.000000in}{0.000000in}}{%
\pgfpathmoveto{\pgfqpoint{0.000000in}{0.000000in}}%
\pgfpathlineto{\pgfqpoint{0.000000in}{-0.048611in}}%
\pgfusepath{stroke,fill}%
}%
\begin{pgfscope}%
\pgfsys@transformshift{0.731325in}{0.484444in}%
\pgfsys@useobject{currentmarker}{}%
\end{pgfscope}%
\end{pgfscope}%
\begin{pgfscope}%
\definecolor{textcolor}{rgb}{0.000000,0.000000,0.000000}%
\pgfsetstrokecolor{textcolor}%
\pgfsetfillcolor{textcolor}%
\pgftext[x=0.731325in,y=0.387222in,,top]{\color{textcolor}\rmfamily\fontsize{10.000000}{12.000000}\selectfont \(\displaystyle {0}\)}%
\end{pgfscope}%
\begin{pgfscope}%
\pgfpathrectangle{\pgfqpoint{0.608025in}{0.484444in}}{\pgfqpoint{2.712595in}{1.541287in}}%
\pgfusepath{clip}%
\pgfsetrectcap%
\pgfsetroundjoin%
\pgfsetlinewidth{0.803000pt}%
\definecolor{currentstroke}{rgb}{0.690196,0.690196,0.690196}%
\pgfsetstrokecolor{currentstroke}%
\pgfsetdash{}{0pt}%
\pgfpathmoveto{\pgfqpoint{1.520823in}{0.484444in}}%
\pgfpathlineto{\pgfqpoint{1.520823in}{2.025731in}}%
\pgfusepath{stroke}%
\end{pgfscope}%
\begin{pgfscope}%
\pgfsetbuttcap%
\pgfsetroundjoin%
\definecolor{currentfill}{rgb}{0.000000,0.000000,0.000000}%
\pgfsetfillcolor{currentfill}%
\pgfsetlinewidth{0.803000pt}%
\definecolor{currentstroke}{rgb}{0.000000,0.000000,0.000000}%
\pgfsetstrokecolor{currentstroke}%
\pgfsetdash{}{0pt}%
\pgfsys@defobject{currentmarker}{\pgfqpoint{0.000000in}{-0.048611in}}{\pgfqpoint{0.000000in}{0.000000in}}{%
\pgfpathmoveto{\pgfqpoint{0.000000in}{0.000000in}}%
\pgfpathlineto{\pgfqpoint{0.000000in}{-0.048611in}}%
\pgfusepath{stroke,fill}%
}%
\begin{pgfscope}%
\pgfsys@transformshift{1.520823in}{0.484444in}%
\pgfsys@useobject{currentmarker}{}%
\end{pgfscope}%
\end{pgfscope}%
\begin{pgfscope}%
\definecolor{textcolor}{rgb}{0.000000,0.000000,0.000000}%
\pgfsetstrokecolor{textcolor}%
\pgfsetfillcolor{textcolor}%
\pgftext[x=1.520823in,y=0.387222in,,top]{\color{textcolor}\rmfamily\fontsize{10.000000}{12.000000}\selectfont \(\displaystyle {2000}\)}%
\end{pgfscope}%
\begin{pgfscope}%
\pgfpathrectangle{\pgfqpoint{0.608025in}{0.484444in}}{\pgfqpoint{2.712595in}{1.541287in}}%
\pgfusepath{clip}%
\pgfsetrectcap%
\pgfsetroundjoin%
\pgfsetlinewidth{0.803000pt}%
\definecolor{currentstroke}{rgb}{0.690196,0.690196,0.690196}%
\pgfsetstrokecolor{currentstroke}%
\pgfsetdash{}{0pt}%
\pgfpathmoveto{\pgfqpoint{2.310320in}{0.484444in}}%
\pgfpathlineto{\pgfqpoint{2.310320in}{2.025731in}}%
\pgfusepath{stroke}%
\end{pgfscope}%
\begin{pgfscope}%
\pgfsetbuttcap%
\pgfsetroundjoin%
\definecolor{currentfill}{rgb}{0.000000,0.000000,0.000000}%
\pgfsetfillcolor{currentfill}%
\pgfsetlinewidth{0.803000pt}%
\definecolor{currentstroke}{rgb}{0.000000,0.000000,0.000000}%
\pgfsetstrokecolor{currentstroke}%
\pgfsetdash{}{0pt}%
\pgfsys@defobject{currentmarker}{\pgfqpoint{0.000000in}{-0.048611in}}{\pgfqpoint{0.000000in}{0.000000in}}{%
\pgfpathmoveto{\pgfqpoint{0.000000in}{0.000000in}}%
\pgfpathlineto{\pgfqpoint{0.000000in}{-0.048611in}}%
\pgfusepath{stroke,fill}%
}%
\begin{pgfscope}%
\pgfsys@transformshift{2.310320in}{0.484444in}%
\pgfsys@useobject{currentmarker}{}%
\end{pgfscope}%
\end{pgfscope}%
\begin{pgfscope}%
\definecolor{textcolor}{rgb}{0.000000,0.000000,0.000000}%
\pgfsetstrokecolor{textcolor}%
\pgfsetfillcolor{textcolor}%
\pgftext[x=2.310320in,y=0.387222in,,top]{\color{textcolor}\rmfamily\fontsize{10.000000}{12.000000}\selectfont \(\displaystyle {4000}\)}%
\end{pgfscope}%
\begin{pgfscope}%
\pgfpathrectangle{\pgfqpoint{0.608025in}{0.484444in}}{\pgfqpoint{2.712595in}{1.541287in}}%
\pgfusepath{clip}%
\pgfsetrectcap%
\pgfsetroundjoin%
\pgfsetlinewidth{0.803000pt}%
\definecolor{currentstroke}{rgb}{0.690196,0.690196,0.690196}%
\pgfsetstrokecolor{currentstroke}%
\pgfsetdash{}{0pt}%
\pgfpathmoveto{\pgfqpoint{3.099818in}{0.484444in}}%
\pgfpathlineto{\pgfqpoint{3.099818in}{2.025731in}}%
\pgfusepath{stroke}%
\end{pgfscope}%
\begin{pgfscope}%
\pgfsetbuttcap%
\pgfsetroundjoin%
\definecolor{currentfill}{rgb}{0.000000,0.000000,0.000000}%
\pgfsetfillcolor{currentfill}%
\pgfsetlinewidth{0.803000pt}%
\definecolor{currentstroke}{rgb}{0.000000,0.000000,0.000000}%
\pgfsetstrokecolor{currentstroke}%
\pgfsetdash{}{0pt}%
\pgfsys@defobject{currentmarker}{\pgfqpoint{0.000000in}{-0.048611in}}{\pgfqpoint{0.000000in}{0.000000in}}{%
\pgfpathmoveto{\pgfqpoint{0.000000in}{0.000000in}}%
\pgfpathlineto{\pgfqpoint{0.000000in}{-0.048611in}}%
\pgfusepath{stroke,fill}%
}%
\begin{pgfscope}%
\pgfsys@transformshift{3.099818in}{0.484444in}%
\pgfsys@useobject{currentmarker}{}%
\end{pgfscope}%
\end{pgfscope}%
\begin{pgfscope}%
\definecolor{textcolor}{rgb}{0.000000,0.000000,0.000000}%
\pgfsetstrokecolor{textcolor}%
\pgfsetfillcolor{textcolor}%
\pgftext[x=3.099818in,y=0.387222in,,top]{\color{textcolor}\rmfamily\fontsize{10.000000}{12.000000}\selectfont \(\displaystyle {6000}\)}%
\end{pgfscope}%
\begin{pgfscope}%
\definecolor{textcolor}{rgb}{0.000000,0.000000,0.000000}%
\pgfsetstrokecolor{textcolor}%
\pgfsetfillcolor{textcolor}%
\pgftext[x=1.964323in,y=0.208333in,,top]{\color{textcolor}\rmfamily\fontsize{10.000000}{12.000000}\selectfont Time [frames]}%
\end{pgfscope}%
\begin{pgfscope}%
\pgfpathrectangle{\pgfqpoint{0.608025in}{0.484444in}}{\pgfqpoint{2.712595in}{1.541287in}}%
\pgfusepath{clip}%
\pgfsetrectcap%
\pgfsetroundjoin%
\pgfsetlinewidth{0.803000pt}%
\definecolor{currentstroke}{rgb}{0.690196,0.690196,0.690196}%
\pgfsetstrokecolor{currentstroke}%
\pgfsetdash{}{0pt}%
\pgfpathmoveto{\pgfqpoint{0.608025in}{0.797730in}}%
\pgfpathlineto{\pgfqpoint{3.320621in}{0.797730in}}%
\pgfusepath{stroke}%
\end{pgfscope}%
\begin{pgfscope}%
\pgfsetbuttcap%
\pgfsetroundjoin%
\definecolor{currentfill}{rgb}{0.000000,0.000000,0.000000}%
\pgfsetfillcolor{currentfill}%
\pgfsetlinewidth{0.803000pt}%
\definecolor{currentstroke}{rgb}{0.000000,0.000000,0.000000}%
\pgfsetstrokecolor{currentstroke}%
\pgfsetdash{}{0pt}%
\pgfsys@defobject{currentmarker}{\pgfqpoint{-0.048611in}{0.000000in}}{\pgfqpoint{-0.000000in}{0.000000in}}{%
\pgfpathmoveto{\pgfqpoint{-0.000000in}{0.000000in}}%
\pgfpathlineto{\pgfqpoint{-0.048611in}{0.000000in}}%
\pgfusepath{stroke,fill}%
}%
\begin{pgfscope}%
\pgfsys@transformshift{0.608025in}{0.797730in}%
\pgfsys@useobject{currentmarker}{}%
\end{pgfscope}%
\end{pgfscope}%
\begin{pgfscope}%
\definecolor{textcolor}{rgb}{0.000000,0.000000,0.000000}%
\pgfsetstrokecolor{textcolor}%
\pgfsetfillcolor{textcolor}%
\pgftext[x=0.263889in, y=0.749535in, left, base]{\color{textcolor}\rmfamily\fontsize{10.000000}{12.000000}\selectfont \(\displaystyle {\ensuremath{-}30}\)}%
\end{pgfscope}%
\begin{pgfscope}%
\pgfpathrectangle{\pgfqpoint{0.608025in}{0.484444in}}{\pgfqpoint{2.712595in}{1.541287in}}%
\pgfusepath{clip}%
\pgfsetrectcap%
\pgfsetroundjoin%
\pgfsetlinewidth{0.803000pt}%
\definecolor{currentstroke}{rgb}{0.690196,0.690196,0.690196}%
\pgfsetstrokecolor{currentstroke}%
\pgfsetdash{}{0pt}%
\pgfpathmoveto{\pgfqpoint{0.608025in}{1.184024in}}%
\pgfpathlineto{\pgfqpoint{3.320621in}{1.184024in}}%
\pgfusepath{stroke}%
\end{pgfscope}%
\begin{pgfscope}%
\pgfsetbuttcap%
\pgfsetroundjoin%
\definecolor{currentfill}{rgb}{0.000000,0.000000,0.000000}%
\pgfsetfillcolor{currentfill}%
\pgfsetlinewidth{0.803000pt}%
\definecolor{currentstroke}{rgb}{0.000000,0.000000,0.000000}%
\pgfsetstrokecolor{currentstroke}%
\pgfsetdash{}{0pt}%
\pgfsys@defobject{currentmarker}{\pgfqpoint{-0.048611in}{0.000000in}}{\pgfqpoint{-0.000000in}{0.000000in}}{%
\pgfpathmoveto{\pgfqpoint{-0.000000in}{0.000000in}}%
\pgfpathlineto{\pgfqpoint{-0.048611in}{0.000000in}}%
\pgfusepath{stroke,fill}%
}%
\begin{pgfscope}%
\pgfsys@transformshift{0.608025in}{1.184024in}%
\pgfsys@useobject{currentmarker}{}%
\end{pgfscope}%
\end{pgfscope}%
\begin{pgfscope}%
\definecolor{textcolor}{rgb}{0.000000,0.000000,0.000000}%
\pgfsetstrokecolor{textcolor}%
\pgfsetfillcolor{textcolor}%
\pgftext[x=0.263889in, y=1.135830in, left, base]{\color{textcolor}\rmfamily\fontsize{10.000000}{12.000000}\selectfont \(\displaystyle {\ensuremath{-}20}\)}%
\end{pgfscope}%
\begin{pgfscope}%
\pgfpathrectangle{\pgfqpoint{0.608025in}{0.484444in}}{\pgfqpoint{2.712595in}{1.541287in}}%
\pgfusepath{clip}%
\pgfsetrectcap%
\pgfsetroundjoin%
\pgfsetlinewidth{0.803000pt}%
\definecolor{currentstroke}{rgb}{0.690196,0.690196,0.690196}%
\pgfsetstrokecolor{currentstroke}%
\pgfsetdash{}{0pt}%
\pgfpathmoveto{\pgfqpoint{0.608025in}{1.570319in}}%
\pgfpathlineto{\pgfqpoint{3.320621in}{1.570319in}}%
\pgfusepath{stroke}%
\end{pgfscope}%
\begin{pgfscope}%
\pgfsetbuttcap%
\pgfsetroundjoin%
\definecolor{currentfill}{rgb}{0.000000,0.000000,0.000000}%
\pgfsetfillcolor{currentfill}%
\pgfsetlinewidth{0.803000pt}%
\definecolor{currentstroke}{rgb}{0.000000,0.000000,0.000000}%
\pgfsetstrokecolor{currentstroke}%
\pgfsetdash{}{0pt}%
\pgfsys@defobject{currentmarker}{\pgfqpoint{-0.048611in}{0.000000in}}{\pgfqpoint{-0.000000in}{0.000000in}}{%
\pgfpathmoveto{\pgfqpoint{-0.000000in}{0.000000in}}%
\pgfpathlineto{\pgfqpoint{-0.048611in}{0.000000in}}%
\pgfusepath{stroke,fill}%
}%
\begin{pgfscope}%
\pgfsys@transformshift{0.608025in}{1.570319in}%
\pgfsys@useobject{currentmarker}{}%
\end{pgfscope}%
\end{pgfscope}%
\begin{pgfscope}%
\definecolor{textcolor}{rgb}{0.000000,0.000000,0.000000}%
\pgfsetstrokecolor{textcolor}%
\pgfsetfillcolor{textcolor}%
\pgftext[x=0.263889in, y=1.522124in, left, base]{\color{textcolor}\rmfamily\fontsize{10.000000}{12.000000}\selectfont \(\displaystyle {\ensuremath{-}10}\)}%
\end{pgfscope}%
\begin{pgfscope}%
\pgfpathrectangle{\pgfqpoint{0.608025in}{0.484444in}}{\pgfqpoint{2.712595in}{1.541287in}}%
\pgfusepath{clip}%
\pgfsetrectcap%
\pgfsetroundjoin%
\pgfsetlinewidth{0.803000pt}%
\definecolor{currentstroke}{rgb}{0.690196,0.690196,0.690196}%
\pgfsetstrokecolor{currentstroke}%
\pgfsetdash{}{0pt}%
\pgfpathmoveto{\pgfqpoint{0.608025in}{1.956613in}}%
\pgfpathlineto{\pgfqpoint{3.320621in}{1.956613in}}%
\pgfusepath{stroke}%
\end{pgfscope}%
\begin{pgfscope}%
\pgfsetbuttcap%
\pgfsetroundjoin%
\definecolor{currentfill}{rgb}{0.000000,0.000000,0.000000}%
\pgfsetfillcolor{currentfill}%
\pgfsetlinewidth{0.803000pt}%
\definecolor{currentstroke}{rgb}{0.000000,0.000000,0.000000}%
\pgfsetstrokecolor{currentstroke}%
\pgfsetdash{}{0pt}%
\pgfsys@defobject{currentmarker}{\pgfqpoint{-0.048611in}{0.000000in}}{\pgfqpoint{-0.000000in}{0.000000in}}{%
\pgfpathmoveto{\pgfqpoint{-0.000000in}{0.000000in}}%
\pgfpathlineto{\pgfqpoint{-0.048611in}{0.000000in}}%
\pgfusepath{stroke,fill}%
}%
\begin{pgfscope}%
\pgfsys@transformshift{0.608025in}{1.956613in}%
\pgfsys@useobject{currentmarker}{}%
\end{pgfscope}%
\end{pgfscope}%
\begin{pgfscope}%
\definecolor{textcolor}{rgb}{0.000000,0.000000,0.000000}%
\pgfsetstrokecolor{textcolor}%
\pgfsetfillcolor{textcolor}%
\pgftext[x=0.441358in, y=1.908419in, left, base]{\color{textcolor}\rmfamily\fontsize{10.000000}{12.000000}\selectfont \(\displaystyle {0}\)}%
\end{pgfscope}%
\begin{pgfscope}%
\definecolor{textcolor}{rgb}{0.000000,0.000000,0.000000}%
\pgfsetstrokecolor{textcolor}%
\pgfsetfillcolor{textcolor}%
\pgftext[x=0.208333in,y=1.255087in,,bottom,rotate=90.000000]{\color{textcolor}\rmfamily\fontsize{10.000000}{12.000000}\selectfont NPM [dB]}%
\end{pgfscope}%
\begin{pgfscope}%
\pgfpathrectangle{\pgfqpoint{0.608025in}{0.484444in}}{\pgfqpoint{2.712595in}{1.541287in}}%
\pgfusepath{clip}%
\pgfsetrectcap%
\pgfsetroundjoin%
\pgfsetlinewidth{1.505625pt}%
\definecolor{currentstroke}{rgb}{0.121569,0.466667,0.705882}%
\pgfsetstrokecolor{currentstroke}%
\pgfsetdash{}{0pt}%
\pgfpathmoveto{\pgfqpoint{0.731325in}{1.955572in}}%
\pgfpathlineto{\pgfqpoint{0.735272in}{1.953805in}}%
\pgfpathlineto{\pgfqpoint{0.742773in}{1.948385in}}%
\pgfpathlineto{\pgfqpoint{0.749878in}{1.937440in}}%
\pgfpathlineto{\pgfqpoint{0.759747in}{1.913701in}}%
\pgfpathlineto{\pgfqpoint{0.789353in}{1.824407in}}%
\pgfpathlineto{\pgfqpoint{0.803169in}{1.747076in}}%
\pgfpathlineto{\pgfqpoint{0.809880in}{1.715798in}}%
\pgfpathlineto{\pgfqpoint{0.836328in}{1.613408in}}%
\pgfpathlineto{\pgfqpoint{0.851329in}{1.532205in}}%
\pgfpathlineto{\pgfqpoint{0.862382in}{1.475416in}}%
\pgfpathlineto{\pgfqpoint{0.887645in}{1.380410in}}%
\pgfpathlineto{\pgfqpoint{0.910146in}{1.302399in}}%
\pgfpathlineto{\pgfqpoint{0.928305in}{1.237981in}}%
\pgfpathlineto{\pgfqpoint{0.947647in}{1.187605in}}%
\pgfpathlineto{\pgfqpoint{0.966595in}{1.146449in}}%
\pgfpathlineto{\pgfqpoint{0.989885in}{1.061082in}}%
\pgfpathlineto{\pgfqpoint{1.020676in}{0.995645in}}%
\pgfpathlineto{\pgfqpoint{1.022255in}{0.992012in}}%
\pgfpathlineto{\pgfqpoint{1.027387in}{0.983122in}}%
\pgfpathlineto{\pgfqpoint{1.031334in}{0.977077in}}%
\pgfpathlineto{\pgfqpoint{1.036071in}{0.969265in}}%
\pgfpathlineto{\pgfqpoint{1.054229in}{0.941738in}}%
\pgfpathlineto{\pgfqpoint{1.058572in}{0.933586in}}%
\pgfpathlineto{\pgfqpoint{1.060151in}{0.929972in}}%
\pgfpathlineto{\pgfqpoint{1.067256in}{0.907828in}}%
\pgfpathlineto{\pgfqpoint{1.068440in}{0.908369in}}%
\pgfpathlineto{\pgfqpoint{1.071598in}{0.907985in}}%
\pgfpathlineto{\pgfqpoint{1.094889in}{0.855484in}}%
\pgfpathlineto{\pgfqpoint{1.099231in}{0.850076in}}%
\pgfpathlineto{\pgfqpoint{1.105547in}{0.836578in}}%
\pgfpathlineto{\pgfqpoint{1.107915in}{0.832093in}}%
\pgfpathlineto{\pgfqpoint{1.119363in}{0.803032in}}%
\pgfpathlineto{\pgfqpoint{1.125679in}{0.787975in}}%
\pgfpathlineto{\pgfqpoint{1.131600in}{0.783389in}}%
\pgfpathlineto{\pgfqpoint{1.131995in}{0.783909in}}%
\pgfpathlineto{\pgfqpoint{1.134758in}{0.786988in}}%
\pgfpathlineto{\pgfqpoint{1.135153in}{0.786608in}}%
\pgfpathlineto{\pgfqpoint{1.153311in}{0.735198in}}%
\pgfpathlineto{\pgfqpoint{1.155285in}{0.732051in}}%
\pgfpathlineto{\pgfqpoint{1.155680in}{0.732982in}}%
\pgfpathlineto{\pgfqpoint{1.156075in}{0.732455in}}%
\pgfpathlineto{\pgfqpoint{1.159627in}{0.720116in}}%
\pgfpathlineto{\pgfqpoint{1.160812in}{0.719361in}}%
\pgfpathlineto{\pgfqpoint{1.161206in}{0.720073in}}%
\pgfpathlineto{\pgfqpoint{1.165154in}{0.732769in}}%
\pgfpathlineto{\pgfqpoint{1.165943in}{0.731504in}}%
\pgfpathlineto{\pgfqpoint{1.177391in}{0.695600in}}%
\pgfpathlineto{\pgfqpoint{1.178575in}{0.692422in}}%
\pgfpathlineto{\pgfqpoint{1.178970in}{0.692697in}}%
\pgfpathlineto{\pgfqpoint{1.182918in}{0.701034in}}%
\pgfpathlineto{\pgfqpoint{1.183312in}{0.701275in}}%
\pgfpathlineto{\pgfqpoint{1.190418in}{0.678679in}}%
\pgfpathlineto{\pgfqpoint{1.192392in}{0.676611in}}%
\pgfpathlineto{\pgfqpoint{1.195550in}{0.667624in}}%
\pgfpathlineto{\pgfqpoint{1.196734in}{0.669271in}}%
\pgfpathlineto{\pgfqpoint{1.198708in}{0.667457in}}%
\pgfpathlineto{\pgfqpoint{1.200287in}{0.664693in}}%
\pgfpathlineto{\pgfqpoint{1.200681in}{0.665090in}}%
\pgfpathlineto{\pgfqpoint{1.201866in}{0.663950in}}%
\pgfpathlineto{\pgfqpoint{1.203445in}{0.661761in}}%
\pgfpathlineto{\pgfqpoint{1.203839in}{0.661994in}}%
\pgfpathlineto{\pgfqpoint{1.204629in}{0.660625in}}%
\pgfpathlineto{\pgfqpoint{1.208182in}{0.643604in}}%
\pgfpathlineto{\pgfqpoint{1.208971in}{0.645085in}}%
\pgfpathlineto{\pgfqpoint{1.211734in}{0.648280in}}%
\pgfpathlineto{\pgfqpoint{1.212918in}{0.647101in}}%
\pgfpathlineto{\pgfqpoint{1.213313in}{0.648057in}}%
\pgfpathlineto{\pgfqpoint{1.215287in}{0.651875in}}%
\pgfpathlineto{\pgfqpoint{1.215682in}{0.651349in}}%
\pgfpathlineto{\pgfqpoint{1.219629in}{0.642736in}}%
\pgfpathlineto{\pgfqpoint{1.220419in}{0.643985in}}%
\pgfpathlineto{\pgfqpoint{1.221208in}{0.645025in}}%
\pgfpathlineto{\pgfqpoint{1.221603in}{0.643450in}}%
\pgfpathlineto{\pgfqpoint{1.222787in}{0.642832in}}%
\pgfpathlineto{\pgfqpoint{1.224761in}{0.638338in}}%
\pgfpathlineto{\pgfqpoint{1.225156in}{0.640008in}}%
\pgfpathlineto{\pgfqpoint{1.225945in}{0.637987in}}%
\pgfpathlineto{\pgfqpoint{1.226735in}{0.636719in}}%
\pgfpathlineto{\pgfqpoint{1.227129in}{0.637829in}}%
\pgfpathlineto{\pgfqpoint{1.228708in}{0.645700in}}%
\pgfpathlineto{\pgfqpoint{1.231077in}{0.660910in}}%
\pgfpathlineto{\pgfqpoint{1.231472in}{0.658756in}}%
\pgfpathlineto{\pgfqpoint{1.235419in}{0.650288in}}%
\pgfpathlineto{\pgfqpoint{1.235814in}{0.650526in}}%
\pgfpathlineto{\pgfqpoint{1.236998in}{0.652134in}}%
\pgfpathlineto{\pgfqpoint{1.238182in}{0.659354in}}%
\pgfpathlineto{\pgfqpoint{1.239367in}{0.658431in}}%
\pgfpathlineto{\pgfqpoint{1.240156in}{0.655540in}}%
\pgfpathlineto{\pgfqpoint{1.241340in}{0.657587in}}%
\pgfpathlineto{\pgfqpoint{1.243314in}{0.662185in}}%
\pgfpathlineto{\pgfqpoint{1.245288in}{0.647846in}}%
\pgfpathlineto{\pgfqpoint{1.246077in}{0.648462in}}%
\pgfpathlineto{\pgfqpoint{1.246472in}{0.648322in}}%
\pgfpathlineto{\pgfqpoint{1.247656in}{0.650534in}}%
\pgfpathlineto{\pgfqpoint{1.248051in}{0.649543in}}%
\pgfpathlineto{\pgfqpoint{1.251999in}{0.647604in}}%
\pgfpathlineto{\pgfqpoint{1.252788in}{0.644454in}}%
\pgfpathlineto{\pgfqpoint{1.253578in}{0.646057in}}%
\pgfpathlineto{\pgfqpoint{1.254762in}{0.648554in}}%
\pgfpathlineto{\pgfqpoint{1.257525in}{0.656297in}}%
\pgfpathlineto{\pgfqpoint{1.259499in}{0.656350in}}%
\pgfpathlineto{\pgfqpoint{1.260683in}{0.659363in}}%
\pgfpathlineto{\pgfqpoint{1.261473in}{0.657795in}}%
\pgfpathlineto{\pgfqpoint{1.263446in}{0.653268in}}%
\pgfpathlineto{\pgfqpoint{1.264631in}{0.656098in}}%
\pgfpathlineto{\pgfqpoint{1.265420in}{0.656474in}}%
\pgfpathlineto{\pgfqpoint{1.265815in}{0.655370in}}%
\pgfpathlineto{\pgfqpoint{1.266604in}{0.654143in}}%
\pgfpathlineto{\pgfqpoint{1.266999in}{0.655648in}}%
\pgfpathlineto{\pgfqpoint{1.268973in}{0.660892in}}%
\pgfpathlineto{\pgfqpoint{1.269368in}{0.660434in}}%
\pgfpathlineto{\pgfqpoint{1.270552in}{0.658508in}}%
\pgfpathlineto{\pgfqpoint{1.273315in}{0.648887in}}%
\pgfpathlineto{\pgfqpoint{1.273710in}{0.651585in}}%
\pgfpathlineto{\pgfqpoint{1.274499in}{0.650175in}}%
\pgfpathlineto{\pgfqpoint{1.274894in}{0.648539in}}%
\pgfpathlineto{\pgfqpoint{1.275684in}{0.649379in}}%
\pgfpathlineto{\pgfqpoint{1.278052in}{0.661840in}}%
\pgfpathlineto{\pgfqpoint{1.280421in}{0.665968in}}%
\pgfpathlineto{\pgfqpoint{1.283579in}{0.649829in}}%
\pgfpathlineto{\pgfqpoint{1.284368in}{0.652129in}}%
\pgfpathlineto{\pgfqpoint{1.286737in}{0.654811in}}%
\pgfpathlineto{\pgfqpoint{1.288710in}{0.663566in}}%
\pgfpathlineto{\pgfqpoint{1.289105in}{0.662658in}}%
\pgfpathlineto{\pgfqpoint{1.295026in}{0.645520in}}%
\pgfpathlineto{\pgfqpoint{1.296210in}{0.644456in}}%
\pgfpathlineto{\pgfqpoint{1.297000in}{0.641485in}}%
\pgfpathlineto{\pgfqpoint{1.297789in}{0.641934in}}%
\pgfpathlineto{\pgfqpoint{1.298184in}{0.642889in}}%
\pgfpathlineto{\pgfqpoint{1.298579in}{0.641025in}}%
\pgfpathlineto{\pgfqpoint{1.301342in}{0.640618in}}%
\pgfpathlineto{\pgfqpoint{1.309237in}{0.666014in}}%
\pgfpathlineto{\pgfqpoint{1.309632in}{0.664664in}}%
\pgfpathlineto{\pgfqpoint{1.312395in}{0.657953in}}%
\pgfpathlineto{\pgfqpoint{1.319501in}{0.682290in}}%
\pgfpathlineto{\pgfqpoint{1.319895in}{0.681038in}}%
\pgfpathlineto{\pgfqpoint{1.320685in}{0.683087in}}%
\pgfpathlineto{\pgfqpoint{1.321080in}{0.684780in}}%
\pgfpathlineto{\pgfqpoint{1.321869in}{0.682700in}}%
\pgfpathlineto{\pgfqpoint{1.322659in}{0.681011in}}%
\pgfpathlineto{\pgfqpoint{1.324238in}{0.671607in}}%
\pgfpathlineto{\pgfqpoint{1.325422in}{0.672704in}}%
\pgfpathlineto{\pgfqpoint{1.327790in}{0.677577in}}%
\pgfpathlineto{\pgfqpoint{1.330159in}{0.680136in}}%
\pgfpathlineto{\pgfqpoint{1.338843in}{0.667882in}}%
\pgfpathlineto{\pgfqpoint{1.340422in}{0.665643in}}%
\pgfpathlineto{\pgfqpoint{1.343580in}{0.656755in}}%
\pgfpathlineto{\pgfqpoint{1.343975in}{0.652764in}}%
\pgfpathlineto{\pgfqpoint{1.344765in}{0.655080in}}%
\pgfpathlineto{\pgfqpoint{1.345949in}{0.657121in}}%
\pgfpathlineto{\pgfqpoint{1.346344in}{0.656657in}}%
\pgfpathlineto{\pgfqpoint{1.347133in}{0.653680in}}%
\pgfpathlineto{\pgfqpoint{1.347923in}{0.656361in}}%
\pgfpathlineto{\pgfqpoint{1.349107in}{0.655035in}}%
\pgfpathlineto{\pgfqpoint{1.351870in}{0.646551in}}%
\pgfpathlineto{\pgfqpoint{1.352265in}{0.646574in}}%
\pgfpathlineto{\pgfqpoint{1.358186in}{0.626404in}}%
\pgfpathlineto{\pgfqpoint{1.358581in}{0.626077in}}%
\pgfpathlineto{\pgfqpoint{1.358976in}{0.627673in}}%
\pgfpathlineto{\pgfqpoint{1.360555in}{0.633378in}}%
\pgfpathlineto{\pgfqpoint{1.361739in}{0.631806in}}%
\pgfpathlineto{\pgfqpoint{1.362134in}{0.630763in}}%
\pgfpathlineto{\pgfqpoint{1.363318in}{0.631481in}}%
\pgfpathlineto{\pgfqpoint{1.364502in}{0.632506in}}%
\pgfpathlineto{\pgfqpoint{1.364897in}{0.633218in}}%
\pgfpathlineto{\pgfqpoint{1.365686in}{0.632433in}}%
\pgfpathlineto{\pgfqpoint{1.368055in}{0.628989in}}%
\pgfpathlineto{\pgfqpoint{1.368844in}{0.632096in}}%
\pgfpathlineto{\pgfqpoint{1.369634in}{0.630399in}}%
\pgfpathlineto{\pgfqpoint{1.370029in}{0.630494in}}%
\pgfpathlineto{\pgfqpoint{1.370818in}{0.626078in}}%
\pgfpathlineto{\pgfqpoint{1.371608in}{0.628870in}}%
\pgfpathlineto{\pgfqpoint{1.375950in}{0.647687in}}%
\pgfpathlineto{\pgfqpoint{1.376739in}{0.646219in}}%
\pgfpathlineto{\pgfqpoint{1.378318in}{0.641330in}}%
\pgfpathlineto{\pgfqpoint{1.379108in}{0.642306in}}%
\pgfpathlineto{\pgfqpoint{1.380687in}{0.639748in}}%
\pgfpathlineto{\pgfqpoint{1.385029in}{0.628184in}}%
\pgfpathlineto{\pgfqpoint{1.388187in}{0.636175in}}%
\pgfpathlineto{\pgfqpoint{1.389371in}{0.635557in}}%
\pgfpathlineto{\pgfqpoint{1.392134in}{0.629524in}}%
\pgfpathlineto{\pgfqpoint{1.395687in}{0.609627in}}%
\pgfpathlineto{\pgfqpoint{1.396082in}{0.608849in}}%
\pgfpathlineto{\pgfqpoint{1.398450in}{0.616205in}}%
\pgfpathlineto{\pgfqpoint{1.405556in}{0.619224in}}%
\pgfpathlineto{\pgfqpoint{1.406740in}{0.615715in}}%
\pgfpathlineto{\pgfqpoint{1.408714in}{0.607146in}}%
\pgfpathlineto{\pgfqpoint{1.409109in}{0.607286in}}%
\pgfpathlineto{\pgfqpoint{1.413451in}{0.607236in}}%
\pgfpathlineto{\pgfqpoint{1.414635in}{0.611888in}}%
\pgfpathlineto{\pgfqpoint{1.415425in}{0.609978in}}%
\pgfpathlineto{\pgfqpoint{1.415819in}{0.609427in}}%
\pgfpathlineto{\pgfqpoint{1.416214in}{0.610990in}}%
\pgfpathlineto{\pgfqpoint{1.419372in}{0.621364in}}%
\pgfpathlineto{\pgfqpoint{1.420951in}{0.625869in}}%
\pgfpathlineto{\pgfqpoint{1.421346in}{0.625679in}}%
\pgfpathlineto{\pgfqpoint{1.422135in}{0.627231in}}%
\pgfpathlineto{\pgfqpoint{1.424109in}{0.631700in}}%
\pgfpathlineto{\pgfqpoint{1.426478in}{0.628545in}}%
\pgfpathlineto{\pgfqpoint{1.428451in}{0.612263in}}%
\pgfpathlineto{\pgfqpoint{1.428846in}{0.612355in}}%
\pgfpathlineto{\pgfqpoint{1.430030in}{0.613126in}}%
\pgfpathlineto{\pgfqpoint{1.432399in}{0.601836in}}%
\pgfpathlineto{\pgfqpoint{1.433188in}{0.604656in}}%
\pgfpathlineto{\pgfqpoint{1.436346in}{0.612447in}}%
\pgfpathlineto{\pgfqpoint{1.436741in}{0.610835in}}%
\pgfpathlineto{\pgfqpoint{1.437531in}{0.609334in}}%
\pgfpathlineto{\pgfqpoint{1.437925in}{0.610648in}}%
\pgfpathlineto{\pgfqpoint{1.439504in}{0.612480in}}%
\pgfpathlineto{\pgfqpoint{1.439899in}{0.611733in}}%
\pgfpathlineto{\pgfqpoint{1.440294in}{0.610086in}}%
\pgfpathlineto{\pgfqpoint{1.441083in}{0.613388in}}%
\pgfpathlineto{\pgfqpoint{1.443847in}{0.620246in}}%
\pgfpathlineto{\pgfqpoint{1.444636in}{0.619613in}}%
\pgfpathlineto{\pgfqpoint{1.445031in}{0.619891in}}%
\pgfpathlineto{\pgfqpoint{1.445820in}{0.618601in}}%
\pgfpathlineto{\pgfqpoint{1.447399in}{0.620149in}}%
\pgfpathlineto{\pgfqpoint{1.449768in}{0.625417in}}%
\pgfpathlineto{\pgfqpoint{1.450163in}{0.625196in}}%
\pgfpathlineto{\pgfqpoint{1.451742in}{0.618556in}}%
\pgfpathlineto{\pgfqpoint{1.452926in}{0.612249in}}%
\pgfpathlineto{\pgfqpoint{1.453715in}{0.613530in}}%
\pgfpathlineto{\pgfqpoint{1.454900in}{0.613408in}}%
\pgfpathlineto{\pgfqpoint{1.455294in}{0.612473in}}%
\pgfpathlineto{\pgfqpoint{1.456084in}{0.613766in}}%
\pgfpathlineto{\pgfqpoint{1.458058in}{0.616025in}}%
\pgfpathlineto{\pgfqpoint{1.459242in}{0.621475in}}%
\pgfpathlineto{\pgfqpoint{1.459637in}{0.619683in}}%
\pgfpathlineto{\pgfqpoint{1.462005in}{0.610438in}}%
\pgfpathlineto{\pgfqpoint{1.462794in}{0.614795in}}%
\pgfpathlineto{\pgfqpoint{1.463584in}{0.611367in}}%
\pgfpathlineto{\pgfqpoint{1.463979in}{0.610841in}}%
\pgfpathlineto{\pgfqpoint{1.464768in}{0.606070in}}%
\pgfpathlineto{\pgfqpoint{1.465558in}{0.607298in}}%
\pgfpathlineto{\pgfqpoint{1.466347in}{0.609231in}}%
\pgfpathlineto{\pgfqpoint{1.466742in}{0.607088in}}%
\pgfpathlineto{\pgfqpoint{1.469505in}{0.601915in}}%
\pgfpathlineto{\pgfqpoint{1.472663in}{0.612546in}}%
\pgfpathlineto{\pgfqpoint{1.474242in}{0.609911in}}%
\pgfpathlineto{\pgfqpoint{1.474637in}{0.611176in}}%
\pgfpathlineto{\pgfqpoint{1.477400in}{0.620061in}}%
\pgfpathlineto{\pgfqpoint{1.478190in}{0.623996in}}%
\pgfpathlineto{\pgfqpoint{1.479374in}{0.623544in}}%
\pgfpathlineto{\pgfqpoint{1.480558in}{0.622368in}}%
\pgfpathlineto{\pgfqpoint{1.480953in}{0.623300in}}%
\pgfpathlineto{\pgfqpoint{1.481742in}{0.628028in}}%
\pgfpathlineto{\pgfqpoint{1.482532in}{0.625135in}}%
\pgfpathlineto{\pgfqpoint{1.483321in}{0.623128in}}%
\pgfpathlineto{\pgfqpoint{1.484111in}{0.624702in}}%
\pgfpathlineto{\pgfqpoint{1.486874in}{0.629497in}}%
\pgfpathlineto{\pgfqpoint{1.487269in}{0.627242in}}%
\pgfpathlineto{\pgfqpoint{1.488453in}{0.621354in}}%
\pgfpathlineto{\pgfqpoint{1.489243in}{0.623008in}}%
\pgfpathlineto{\pgfqpoint{1.489637in}{0.623748in}}%
\pgfpathlineto{\pgfqpoint{1.490427in}{0.622091in}}%
\pgfpathlineto{\pgfqpoint{1.491611in}{0.617705in}}%
\pgfpathlineto{\pgfqpoint{1.492795in}{0.620058in}}%
\pgfpathlineto{\pgfqpoint{1.493585in}{0.621155in}}%
\pgfpathlineto{\pgfqpoint{1.494374in}{0.619769in}}%
\pgfpathlineto{\pgfqpoint{1.495559in}{0.619656in}}%
\pgfpathlineto{\pgfqpoint{1.499506in}{0.600560in}}%
\pgfpathlineto{\pgfqpoint{1.502269in}{0.606828in}}%
\pgfpathlineto{\pgfqpoint{1.502664in}{0.605008in}}%
\pgfpathlineto{\pgfqpoint{1.503848in}{0.606306in}}%
\pgfpathlineto{\pgfqpoint{1.505427in}{0.605197in}}%
\pgfpathlineto{\pgfqpoint{1.506612in}{0.603345in}}%
\pgfpathlineto{\pgfqpoint{1.507006in}{0.603669in}}%
\pgfpathlineto{\pgfqpoint{1.509375in}{0.602468in}}%
\pgfpathlineto{\pgfqpoint{1.511349in}{0.598785in}}%
\pgfpathlineto{\pgfqpoint{1.511743in}{0.599178in}}%
\pgfpathlineto{\pgfqpoint{1.514507in}{0.608560in}}%
\pgfpathlineto{\pgfqpoint{1.514901in}{0.607393in}}%
\pgfpathlineto{\pgfqpoint{1.515296in}{0.609133in}}%
\pgfpathlineto{\pgfqpoint{1.518454in}{0.615389in}}%
\pgfpathlineto{\pgfqpoint{1.518849in}{0.615130in}}%
\pgfpathlineto{\pgfqpoint{1.519244in}{0.616276in}}%
\pgfpathlineto{\pgfqpoint{1.520033in}{0.618180in}}%
\pgfpathlineto{\pgfqpoint{1.520428in}{0.616209in}}%
\pgfpathlineto{\pgfqpoint{1.523191in}{0.606092in}}%
\pgfpathlineto{\pgfqpoint{1.528323in}{0.587464in}}%
\pgfpathlineto{\pgfqpoint{1.529112in}{0.590074in}}%
\pgfpathlineto{\pgfqpoint{1.529902in}{0.589858in}}%
\pgfpathlineto{\pgfqpoint{1.534244in}{0.602663in}}%
\pgfpathlineto{\pgfqpoint{1.534639in}{0.602514in}}%
\pgfpathlineto{\pgfqpoint{1.536613in}{0.598498in}}%
\pgfpathlineto{\pgfqpoint{1.537007in}{0.598806in}}%
\pgfpathlineto{\pgfqpoint{1.537797in}{0.599923in}}%
\pgfpathlineto{\pgfqpoint{1.538192in}{0.598662in}}%
\pgfpathlineto{\pgfqpoint{1.538586in}{0.596807in}}%
\pgfpathlineto{\pgfqpoint{1.539376in}{0.597916in}}%
\pgfpathlineto{\pgfqpoint{1.542139in}{0.603925in}}%
\pgfpathlineto{\pgfqpoint{1.545692in}{0.593137in}}%
\pgfpathlineto{\pgfqpoint{1.550429in}{0.607570in}}%
\pgfpathlineto{\pgfqpoint{1.550823in}{0.607357in}}%
\pgfpathlineto{\pgfqpoint{1.551613in}{0.608094in}}%
\pgfpathlineto{\pgfqpoint{1.553981in}{0.601057in}}%
\pgfpathlineto{\pgfqpoint{1.554771in}{0.596679in}}%
\pgfpathlineto{\pgfqpoint{1.555560in}{0.600087in}}%
\pgfpathlineto{\pgfqpoint{1.557139in}{0.603513in}}%
\pgfpathlineto{\pgfqpoint{1.557534in}{0.603316in}}%
\pgfpathlineto{\pgfqpoint{1.558718in}{0.601974in}}%
\pgfpathlineto{\pgfqpoint{1.559508in}{0.599957in}}%
\pgfpathlineto{\pgfqpoint{1.561087in}{0.593264in}}%
\pgfpathlineto{\pgfqpoint{1.561482in}{0.594984in}}%
\pgfpathlineto{\pgfqpoint{1.562666in}{0.597619in}}%
\pgfpathlineto{\pgfqpoint{1.565034in}{0.605282in}}%
\pgfpathlineto{\pgfqpoint{1.565429in}{0.605151in}}%
\pgfpathlineto{\pgfqpoint{1.567008in}{0.605173in}}%
\pgfpathlineto{\pgfqpoint{1.570956in}{0.610890in}}%
\pgfpathlineto{\pgfqpoint{1.571350in}{0.610239in}}%
\pgfpathlineto{\pgfqpoint{1.572535in}{0.607115in}}%
\pgfpathlineto{\pgfqpoint{1.572929in}{0.608416in}}%
\pgfpathlineto{\pgfqpoint{1.574903in}{0.611653in}}%
\pgfpathlineto{\pgfqpoint{1.575693in}{0.610317in}}%
\pgfpathlineto{\pgfqpoint{1.576087in}{0.608783in}}%
\pgfpathlineto{\pgfqpoint{1.577272in}{0.609743in}}%
\pgfpathlineto{\pgfqpoint{1.580035in}{0.614140in}}%
\pgfpathlineto{\pgfqpoint{1.581219in}{0.617301in}}%
\pgfpathlineto{\pgfqpoint{1.585167in}{0.630578in}}%
\pgfpathlineto{\pgfqpoint{1.585956in}{0.629165in}}%
\pgfpathlineto{\pgfqpoint{1.595035in}{0.605721in}}%
\pgfpathlineto{\pgfqpoint{1.595825in}{0.604903in}}%
\pgfpathlineto{\pgfqpoint{1.597404in}{0.599149in}}%
\pgfpathlineto{\pgfqpoint{1.597799in}{0.600268in}}%
\pgfpathlineto{\pgfqpoint{1.599772in}{0.601445in}}%
\pgfpathlineto{\pgfqpoint{1.604509in}{0.605667in}}%
\pgfpathlineto{\pgfqpoint{1.605299in}{0.604373in}}%
\pgfpathlineto{\pgfqpoint{1.608062in}{0.595660in}}%
\pgfpathlineto{\pgfqpoint{1.608852in}{0.599639in}}%
\pgfpathlineto{\pgfqpoint{1.609246in}{0.599409in}}%
\pgfpathlineto{\pgfqpoint{1.609641in}{0.600100in}}%
\pgfpathlineto{\pgfqpoint{1.610036in}{0.600936in}}%
\pgfpathlineto{\pgfqpoint{1.610431in}{0.599956in}}%
\pgfpathlineto{\pgfqpoint{1.614773in}{0.584169in}}%
\pgfpathlineto{\pgfqpoint{1.615168in}{0.588352in}}%
\pgfpathlineto{\pgfqpoint{1.617536in}{0.595952in}}%
\pgfpathlineto{\pgfqpoint{1.618720in}{0.593047in}}%
\pgfpathlineto{\pgfqpoint{1.620694in}{0.589898in}}%
\pgfpathlineto{\pgfqpoint{1.622668in}{0.597702in}}%
\pgfpathlineto{\pgfqpoint{1.623457in}{0.595754in}}%
\pgfpathlineto{\pgfqpoint{1.624247in}{0.589270in}}%
\pgfpathlineto{\pgfqpoint{1.625431in}{0.591143in}}%
\pgfpathlineto{\pgfqpoint{1.628589in}{0.584738in}}%
\pgfpathlineto{\pgfqpoint{1.628984in}{0.585945in}}%
\pgfpathlineto{\pgfqpoint{1.630563in}{0.589045in}}%
\pgfpathlineto{\pgfqpoint{1.631352in}{0.595163in}}%
\pgfpathlineto{\pgfqpoint{1.631747in}{0.599113in}}%
\pgfpathlineto{\pgfqpoint{1.632536in}{0.593836in}}%
\pgfpathlineto{\pgfqpoint{1.635694in}{0.581694in}}%
\pgfpathlineto{\pgfqpoint{1.636089in}{0.581786in}}%
\pgfpathlineto{\pgfqpoint{1.636879in}{0.577719in}}%
\pgfpathlineto{\pgfqpoint{1.637668in}{0.578736in}}%
\pgfpathlineto{\pgfqpoint{1.640431in}{0.586735in}}%
\pgfpathlineto{\pgfqpoint{1.643195in}{0.593353in}}%
\pgfpathlineto{\pgfqpoint{1.644379in}{0.590792in}}%
\pgfpathlineto{\pgfqpoint{1.644774in}{0.591647in}}%
\pgfpathlineto{\pgfqpoint{1.645168in}{0.593316in}}%
\pgfpathlineto{\pgfqpoint{1.645958in}{0.593112in}}%
\pgfpathlineto{\pgfqpoint{1.646747in}{0.590251in}}%
\pgfpathlineto{\pgfqpoint{1.647537in}{0.592415in}}%
\pgfpathlineto{\pgfqpoint{1.650300in}{0.599209in}}%
\pgfpathlineto{\pgfqpoint{1.652669in}{0.593766in}}%
\pgfpathlineto{\pgfqpoint{1.656221in}{0.576994in}}%
\pgfpathlineto{\pgfqpoint{1.657011in}{0.577475in}}%
\pgfpathlineto{\pgfqpoint{1.657800in}{0.577540in}}%
\pgfpathlineto{\pgfqpoint{1.661748in}{0.594099in}}%
\pgfpathlineto{\pgfqpoint{1.662537in}{0.591870in}}%
\pgfpathlineto{\pgfqpoint{1.663327in}{0.590416in}}%
\pgfpathlineto{\pgfqpoint{1.665301in}{0.583839in}}%
\pgfpathlineto{\pgfqpoint{1.666090in}{0.584781in}}%
\pgfpathlineto{\pgfqpoint{1.666880in}{0.586082in}}%
\pgfpathlineto{\pgfqpoint{1.669248in}{0.592948in}}%
\pgfpathlineto{\pgfqpoint{1.669643in}{0.591656in}}%
\pgfpathlineto{\pgfqpoint{1.674380in}{0.593509in}}%
\pgfpathlineto{\pgfqpoint{1.675169in}{0.588320in}}%
\pgfpathlineto{\pgfqpoint{1.675564in}{0.591176in}}%
\pgfpathlineto{\pgfqpoint{1.676354in}{0.594681in}}%
\pgfpathlineto{\pgfqpoint{1.676748in}{0.593478in}}%
\pgfpathlineto{\pgfqpoint{1.677143in}{0.591199in}}%
\pgfpathlineto{\pgfqpoint{1.677933in}{0.592308in}}%
\pgfpathlineto{\pgfqpoint{1.679512in}{0.598378in}}%
\pgfpathlineto{\pgfqpoint{1.680696in}{0.596472in}}%
\pgfpathlineto{\pgfqpoint{1.681485in}{0.597827in}}%
\pgfpathlineto{\pgfqpoint{1.683459in}{0.599428in}}%
\pgfpathlineto{\pgfqpoint{1.687012in}{0.584372in}}%
\pgfpathlineto{\pgfqpoint{1.687801in}{0.587792in}}%
\pgfpathlineto{\pgfqpoint{1.688986in}{0.594092in}}%
\pgfpathlineto{\pgfqpoint{1.689775in}{0.592699in}}%
\pgfpathlineto{\pgfqpoint{1.690170in}{0.589500in}}%
\pgfpathlineto{\pgfqpoint{1.690959in}{0.595692in}}%
\pgfpathlineto{\pgfqpoint{1.692538in}{0.590873in}}%
\pgfpathlineto{\pgfqpoint{1.692933in}{0.593148in}}%
\pgfpathlineto{\pgfqpoint{1.695302in}{0.600199in}}%
\pgfpathlineto{\pgfqpoint{1.696881in}{0.602345in}}%
\pgfpathlineto{\pgfqpoint{1.697670in}{0.600299in}}%
\pgfpathlineto{\pgfqpoint{1.698854in}{0.602236in}}%
\pgfpathlineto{\pgfqpoint{1.702802in}{0.610837in}}%
\pgfpathlineto{\pgfqpoint{1.704776in}{0.605186in}}%
\pgfpathlineto{\pgfqpoint{1.705960in}{0.606635in}}%
\pgfpathlineto{\pgfqpoint{1.706355in}{0.607678in}}%
\pgfpathlineto{\pgfqpoint{1.707144in}{0.605522in}}%
\pgfpathlineto{\pgfqpoint{1.711091in}{0.592676in}}%
\pgfpathlineto{\pgfqpoint{1.711881in}{0.594286in}}%
\pgfpathlineto{\pgfqpoint{1.713460in}{0.600152in}}%
\pgfpathlineto{\pgfqpoint{1.717013in}{0.622400in}}%
\pgfpathlineto{\pgfqpoint{1.719381in}{0.631802in}}%
\pgfpathlineto{\pgfqpoint{1.720960in}{0.627672in}}%
\pgfpathlineto{\pgfqpoint{1.721355in}{0.626805in}}%
\pgfpathlineto{\pgfqpoint{1.722144in}{0.628658in}}%
\pgfpathlineto{\pgfqpoint{1.728066in}{0.640538in}}%
\pgfpathlineto{\pgfqpoint{1.728460in}{0.640191in}}%
\pgfpathlineto{\pgfqpoint{1.735171in}{0.621132in}}%
\pgfpathlineto{\pgfqpoint{1.735961in}{0.622349in}}%
\pgfpathlineto{\pgfqpoint{1.736355in}{0.621011in}}%
\pgfpathlineto{\pgfqpoint{1.746224in}{0.602554in}}%
\pgfpathlineto{\pgfqpoint{1.747408in}{0.597118in}}%
\pgfpathlineto{\pgfqpoint{1.747803in}{0.599339in}}%
\pgfpathlineto{\pgfqpoint{1.749382in}{0.600413in}}%
\pgfpathlineto{\pgfqpoint{1.751751in}{0.588473in}}%
\pgfpathlineto{\pgfqpoint{1.752540in}{0.591179in}}%
\pgfpathlineto{\pgfqpoint{1.753330in}{0.594303in}}%
\pgfpathlineto{\pgfqpoint{1.754119in}{0.591563in}}%
\pgfpathlineto{\pgfqpoint{1.756882in}{0.584063in}}%
\pgfpathlineto{\pgfqpoint{1.757277in}{0.584691in}}%
\pgfpathlineto{\pgfqpoint{1.758067in}{0.583362in}}%
\pgfpathlineto{\pgfqpoint{1.760040in}{0.583358in}}%
\pgfpathlineto{\pgfqpoint{1.762804in}{0.588605in}}%
\pgfpathlineto{\pgfqpoint{1.763198in}{0.587521in}}%
\pgfpathlineto{\pgfqpoint{1.763593in}{0.585461in}}%
\pgfpathlineto{\pgfqpoint{1.764383in}{0.586962in}}%
\pgfpathlineto{\pgfqpoint{1.767146in}{0.593422in}}%
\pgfpathlineto{\pgfqpoint{1.767541in}{0.593160in}}%
\pgfpathlineto{\pgfqpoint{1.770699in}{0.586118in}}%
\pgfpathlineto{\pgfqpoint{1.771093in}{0.586822in}}%
\pgfpathlineto{\pgfqpoint{1.777015in}{0.604905in}}%
\pgfpathlineto{\pgfqpoint{1.777409in}{0.603104in}}%
\pgfpathlineto{\pgfqpoint{1.779383in}{0.594101in}}%
\pgfpathlineto{\pgfqpoint{1.779778in}{0.594354in}}%
\pgfpathlineto{\pgfqpoint{1.782541in}{0.595316in}}%
\pgfpathlineto{\pgfqpoint{1.784120in}{0.591255in}}%
\pgfpathlineto{\pgfqpoint{1.784515in}{0.591450in}}%
\pgfpathlineto{\pgfqpoint{1.785304in}{0.594173in}}%
\pgfpathlineto{\pgfqpoint{1.785699in}{0.592704in}}%
\pgfpathlineto{\pgfqpoint{1.786489in}{0.587803in}}%
\pgfpathlineto{\pgfqpoint{1.787278in}{0.590125in}}%
\pgfpathlineto{\pgfqpoint{1.790041in}{0.597228in}}%
\pgfpathlineto{\pgfqpoint{1.790831in}{0.595174in}}%
\pgfpathlineto{\pgfqpoint{1.793199in}{0.590267in}}%
\pgfpathlineto{\pgfqpoint{1.794383in}{0.587612in}}%
\pgfpathlineto{\pgfqpoint{1.795568in}{0.588746in}}%
\pgfpathlineto{\pgfqpoint{1.796357in}{0.586550in}}%
\pgfpathlineto{\pgfqpoint{1.796752in}{0.585591in}}%
\pgfpathlineto{\pgfqpoint{1.797147in}{0.586356in}}%
\pgfpathlineto{\pgfqpoint{1.799515in}{0.594346in}}%
\pgfpathlineto{\pgfqpoint{1.801094in}{0.592325in}}%
\pgfpathlineto{\pgfqpoint{1.801489in}{0.592744in}}%
\pgfpathlineto{\pgfqpoint{1.801884in}{0.594071in}}%
\pgfpathlineto{\pgfqpoint{1.802673in}{0.591576in}}%
\pgfpathlineto{\pgfqpoint{1.803857in}{0.589222in}}%
\pgfpathlineto{\pgfqpoint{1.804252in}{0.590408in}}%
\pgfpathlineto{\pgfqpoint{1.806226in}{0.591700in}}%
\pgfpathlineto{\pgfqpoint{1.808989in}{0.591393in}}%
\pgfpathlineto{\pgfqpoint{1.810963in}{0.593095in}}%
\pgfpathlineto{\pgfqpoint{1.812147in}{0.584365in}}%
\pgfpathlineto{\pgfqpoint{1.812937in}{0.579864in}}%
\pgfpathlineto{\pgfqpoint{1.813726in}{0.583011in}}%
\pgfpathlineto{\pgfqpoint{1.814121in}{0.583598in}}%
\pgfpathlineto{\pgfqpoint{1.814516in}{0.581742in}}%
\pgfpathlineto{\pgfqpoint{1.815700in}{0.580871in}}%
\pgfpathlineto{\pgfqpoint{1.816095in}{0.582377in}}%
\pgfpathlineto{\pgfqpoint{1.819647in}{0.594640in}}%
\pgfpathlineto{\pgfqpoint{1.821226in}{0.598448in}}%
\pgfpathlineto{\pgfqpoint{1.821621in}{0.597216in}}%
\pgfpathlineto{\pgfqpoint{1.823595in}{0.589027in}}%
\pgfpathlineto{\pgfqpoint{1.824779in}{0.590435in}}%
\pgfpathlineto{\pgfqpoint{1.828727in}{0.597155in}}%
\pgfpathlineto{\pgfqpoint{1.829121in}{0.595727in}}%
\pgfpathlineto{\pgfqpoint{1.831885in}{0.588861in}}%
\pgfpathlineto{\pgfqpoint{1.832674in}{0.592016in}}%
\pgfpathlineto{\pgfqpoint{1.833858in}{0.591356in}}%
\pgfpathlineto{\pgfqpoint{1.834648in}{0.591671in}}%
\pgfpathlineto{\pgfqpoint{1.836622in}{0.593560in}}%
\pgfpathlineto{\pgfqpoint{1.838201in}{0.591259in}}%
\pgfpathlineto{\pgfqpoint{1.838595in}{0.592197in}}%
\pgfpathlineto{\pgfqpoint{1.839780in}{0.593283in}}%
\pgfpathlineto{\pgfqpoint{1.840174in}{0.592713in}}%
\pgfpathlineto{\pgfqpoint{1.842148in}{0.585926in}}%
\pgfpathlineto{\pgfqpoint{1.842543in}{0.586456in}}%
\pgfpathlineto{\pgfqpoint{1.846490in}{0.591146in}}%
\pgfpathlineto{\pgfqpoint{1.847280in}{0.589944in}}%
\pgfpathlineto{\pgfqpoint{1.848464in}{0.584049in}}%
\pgfpathlineto{\pgfqpoint{1.849648in}{0.578748in}}%
\pgfpathlineto{\pgfqpoint{1.850438in}{0.579854in}}%
\pgfpathlineto{\pgfqpoint{1.851622in}{0.579776in}}%
\pgfpathlineto{\pgfqpoint{1.857938in}{0.605511in}}%
\pgfpathlineto{\pgfqpoint{1.858333in}{0.603614in}}%
\pgfpathlineto{\pgfqpoint{1.861886in}{0.592339in}}%
\pgfpathlineto{\pgfqpoint{1.862280in}{0.592956in}}%
\pgfpathlineto{\pgfqpoint{1.864254in}{0.589193in}}%
\pgfpathlineto{\pgfqpoint{1.866623in}{0.597744in}}%
\pgfpathlineto{\pgfqpoint{1.869781in}{0.591794in}}%
\pgfpathlineto{\pgfqpoint{1.870570in}{0.587490in}}%
\pgfpathlineto{\pgfqpoint{1.871754in}{0.588446in}}%
\pgfpathlineto{\pgfqpoint{1.872544in}{0.590704in}}%
\pgfpathlineto{\pgfqpoint{1.872939in}{0.592174in}}%
\pgfpathlineto{\pgfqpoint{1.873728in}{0.589345in}}%
\pgfpathlineto{\pgfqpoint{1.876491in}{0.588703in}}%
\pgfpathlineto{\pgfqpoint{1.877675in}{0.579332in}}%
\pgfpathlineto{\pgfqpoint{1.878465in}{0.584362in}}%
\pgfpathlineto{\pgfqpoint{1.878860in}{0.584590in}}%
\pgfpathlineto{\pgfqpoint{1.881623in}{0.595117in}}%
\pgfpathlineto{\pgfqpoint{1.883202in}{0.592347in}}%
\pgfpathlineto{\pgfqpoint{1.883597in}{0.593441in}}%
\pgfpathlineto{\pgfqpoint{1.884386in}{0.597026in}}%
\pgfpathlineto{\pgfqpoint{1.885176in}{0.594200in}}%
\pgfpathlineto{\pgfqpoint{1.886360in}{0.590547in}}%
\pgfpathlineto{\pgfqpoint{1.887149in}{0.592405in}}%
\pgfpathlineto{\pgfqpoint{1.889913in}{0.595245in}}%
\pgfpathlineto{\pgfqpoint{1.890702in}{0.596538in}}%
\pgfpathlineto{\pgfqpoint{1.891492in}{0.595478in}}%
\pgfpathlineto{\pgfqpoint{1.894255in}{0.587865in}}%
\pgfpathlineto{\pgfqpoint{1.895834in}{0.587673in}}%
\pgfpathlineto{\pgfqpoint{1.899781in}{0.600087in}}%
\pgfpathlineto{\pgfqpoint{1.902939in}{0.596804in}}%
\pgfpathlineto{\pgfqpoint{1.905308in}{0.590305in}}%
\pgfpathlineto{\pgfqpoint{1.905703in}{0.589899in}}%
\pgfpathlineto{\pgfqpoint{1.906097in}{0.591089in}}%
\pgfpathlineto{\pgfqpoint{1.908071in}{0.592331in}}%
\pgfpathlineto{\pgfqpoint{1.909650in}{0.592776in}}%
\pgfpathlineto{\pgfqpoint{1.910045in}{0.591831in}}%
\pgfpathlineto{\pgfqpoint{1.912019in}{0.587028in}}%
\pgfpathlineto{\pgfqpoint{1.912808in}{0.588151in}}%
\pgfpathlineto{\pgfqpoint{1.914782in}{0.585825in}}%
\pgfpathlineto{\pgfqpoint{1.915966in}{0.582869in}}%
\pgfpathlineto{\pgfqpoint{1.916361in}{0.585340in}}%
\pgfpathlineto{\pgfqpoint{1.919124in}{0.587503in}}%
\pgfpathlineto{\pgfqpoint{1.922677in}{0.584368in}}%
\pgfpathlineto{\pgfqpoint{1.923466in}{0.583254in}}%
\pgfpathlineto{\pgfqpoint{1.923861in}{0.582377in}}%
\pgfpathlineto{\pgfqpoint{1.924651in}{0.583514in}}%
\pgfpathlineto{\pgfqpoint{1.925835in}{0.586429in}}%
\pgfpathlineto{\pgfqpoint{1.928203in}{0.580332in}}%
\pgfpathlineto{\pgfqpoint{1.928598in}{0.579069in}}%
\pgfpathlineto{\pgfqpoint{1.928993in}{0.580341in}}%
\pgfpathlineto{\pgfqpoint{1.931756in}{0.586124in}}%
\pgfpathlineto{\pgfqpoint{1.932151in}{0.585441in}}%
\pgfpathlineto{\pgfqpoint{1.932546in}{0.587600in}}%
\pgfpathlineto{\pgfqpoint{1.933730in}{0.591471in}}%
\pgfpathlineto{\pgfqpoint{1.934125in}{0.590452in}}%
\pgfpathlineto{\pgfqpoint{1.936098in}{0.585307in}}%
\pgfpathlineto{\pgfqpoint{1.936493in}{0.585772in}}%
\pgfpathlineto{\pgfqpoint{1.942809in}{0.581832in}}%
\pgfpathlineto{\pgfqpoint{1.943599in}{0.583926in}}%
\pgfpathlineto{\pgfqpoint{1.944388in}{0.582737in}}%
\pgfpathlineto{\pgfqpoint{1.945572in}{0.580003in}}%
\pgfpathlineto{\pgfqpoint{1.947941in}{0.570697in}}%
\pgfpathlineto{\pgfqpoint{1.951099in}{0.583283in}}%
\pgfpathlineto{\pgfqpoint{1.952283in}{0.587992in}}%
\pgfpathlineto{\pgfqpoint{1.952678in}{0.586456in}}%
\pgfpathlineto{\pgfqpoint{1.955046in}{0.580477in}}%
\pgfpathlineto{\pgfqpoint{1.955441in}{0.578169in}}%
\pgfpathlineto{\pgfqpoint{1.956231in}{0.581605in}}%
\pgfpathlineto{\pgfqpoint{1.956625in}{0.583574in}}%
\pgfpathlineto{\pgfqpoint{1.957415in}{0.582703in}}%
\pgfpathlineto{\pgfqpoint{1.962152in}{0.566782in}}%
\pgfpathlineto{\pgfqpoint{1.962546in}{0.567722in}}%
\pgfpathlineto{\pgfqpoint{1.966494in}{0.579459in}}%
\pgfpathlineto{\pgfqpoint{1.967283in}{0.575377in}}%
\pgfpathlineto{\pgfqpoint{1.969257in}{0.566211in}}%
\pgfpathlineto{\pgfqpoint{1.969652in}{0.567432in}}%
\pgfpathlineto{\pgfqpoint{1.973994in}{0.575489in}}%
\pgfpathlineto{\pgfqpoint{1.974389in}{0.574452in}}%
\pgfpathlineto{\pgfqpoint{1.975178in}{0.571423in}}%
\pgfpathlineto{\pgfqpoint{1.975968in}{0.571996in}}%
\pgfpathlineto{\pgfqpoint{1.976363in}{0.573714in}}%
\pgfpathlineto{\pgfqpoint{1.977547in}{0.572415in}}%
\pgfpathlineto{\pgfqpoint{1.981494in}{0.557826in}}%
\pgfpathlineto{\pgfqpoint{1.978731in}{0.573286in}}%
\pgfpathlineto{\pgfqpoint{1.981889in}{0.558839in}}%
\pgfpathlineto{\pgfqpoint{1.986231in}{0.591611in}}%
\pgfpathlineto{\pgfqpoint{1.987021in}{0.591069in}}%
\pgfpathlineto{\pgfqpoint{1.987416in}{0.590331in}}%
\pgfpathlineto{\pgfqpoint{1.987810in}{0.591573in}}%
\pgfpathlineto{\pgfqpoint{1.990968in}{0.597156in}}%
\pgfpathlineto{\pgfqpoint{1.992547in}{0.589497in}}%
\pgfpathlineto{\pgfqpoint{1.993337in}{0.590089in}}%
\pgfpathlineto{\pgfqpoint{1.994916in}{0.590216in}}%
\pgfpathlineto{\pgfqpoint{1.998074in}{0.604628in}}%
\pgfpathlineto{\pgfqpoint{1.998863in}{0.600169in}}%
\pgfpathlineto{\pgfqpoint{2.000048in}{0.595320in}}%
\pgfpathlineto{\pgfqpoint{2.000837in}{0.597131in}}%
\pgfpathlineto{\pgfqpoint{2.002021in}{0.597201in}}%
\pgfpathlineto{\pgfqpoint{2.002416in}{0.595010in}}%
\pgfpathlineto{\pgfqpoint{2.003206in}{0.595985in}}%
\pgfpathlineto{\pgfqpoint{2.005969in}{0.602906in}}%
\pgfpathlineto{\pgfqpoint{2.011101in}{0.608512in}}%
\pgfpathlineto{\pgfqpoint{2.011495in}{0.609747in}}%
\pgfpathlineto{\pgfqpoint{2.011890in}{0.608452in}}%
\pgfpathlineto{\pgfqpoint{2.018601in}{0.588951in}}%
\pgfpathlineto{\pgfqpoint{2.019390in}{0.590745in}}%
\pgfpathlineto{\pgfqpoint{2.020180in}{0.589023in}}%
\pgfpathlineto{\pgfqpoint{2.023338in}{0.594873in}}%
\pgfpathlineto{\pgfqpoint{2.024522in}{0.593465in}}%
\pgfpathlineto{\pgfqpoint{2.027285in}{0.576823in}}%
\pgfpathlineto{\pgfqpoint{2.028864in}{0.577702in}}%
\pgfpathlineto{\pgfqpoint{2.032812in}{0.585934in}}%
\pgfpathlineto{\pgfqpoint{2.035180in}{0.595896in}}%
\pgfpathlineto{\pgfqpoint{2.037944in}{0.597904in}}%
\pgfpathlineto{\pgfqpoint{2.038338in}{0.594807in}}%
\pgfpathlineto{\pgfqpoint{2.039128in}{0.599000in}}%
\pgfpathlineto{\pgfqpoint{2.039917in}{0.603700in}}%
\pgfpathlineto{\pgfqpoint{2.040707in}{0.602807in}}%
\pgfpathlineto{\pgfqpoint{2.041496in}{0.600651in}}%
\pgfpathlineto{\pgfqpoint{2.041891in}{0.603172in}}%
\pgfpathlineto{\pgfqpoint{2.045049in}{0.615657in}}%
\pgfpathlineto{\pgfqpoint{2.045838in}{0.619000in}}%
\pgfpathlineto{\pgfqpoint{2.046628in}{0.616547in}}%
\pgfpathlineto{\pgfqpoint{2.047023in}{0.616207in}}%
\pgfpathlineto{\pgfqpoint{2.047417in}{0.617542in}}%
\pgfpathlineto{\pgfqpoint{2.048207in}{0.617998in}}%
\pgfpathlineto{\pgfqpoint{2.048602in}{0.616926in}}%
\pgfpathlineto{\pgfqpoint{2.050970in}{0.615192in}}%
\pgfpathlineto{\pgfqpoint{2.052154in}{0.608952in}}%
\pgfpathlineto{\pgfqpoint{2.052944in}{0.610754in}}%
\pgfpathlineto{\pgfqpoint{2.053733in}{0.611105in}}%
\pgfpathlineto{\pgfqpoint{2.058470in}{0.599418in}}%
\pgfpathlineto{\pgfqpoint{2.060444in}{0.595166in}}%
\pgfpathlineto{\pgfqpoint{2.060839in}{0.597311in}}%
\pgfpathlineto{\pgfqpoint{2.061234in}{0.597857in}}%
\pgfpathlineto{\pgfqpoint{2.061628in}{0.596156in}}%
\pgfpathlineto{\pgfqpoint{2.062418in}{0.591563in}}%
\pgfpathlineto{\pgfqpoint{2.063207in}{0.593150in}}%
\pgfpathlineto{\pgfqpoint{2.065181in}{0.598167in}}%
\pgfpathlineto{\pgfqpoint{2.070313in}{0.580098in}}%
\pgfpathlineto{\pgfqpoint{2.070708in}{0.581669in}}%
\pgfpathlineto{\pgfqpoint{2.071102in}{0.583035in}}%
\pgfpathlineto{\pgfqpoint{2.072287in}{0.582176in}}%
\pgfpathlineto{\pgfqpoint{2.072681in}{0.581930in}}%
\pgfpathlineto{\pgfqpoint{2.073076in}{0.583424in}}%
\pgfpathlineto{\pgfqpoint{2.074655in}{0.588540in}}%
\pgfpathlineto{\pgfqpoint{2.075050in}{0.586483in}}%
\pgfpathlineto{\pgfqpoint{2.075839in}{0.586919in}}%
\pgfpathlineto{\pgfqpoint{2.077418in}{0.591994in}}%
\pgfpathlineto{\pgfqpoint{2.077813in}{0.591321in}}%
\pgfpathlineto{\pgfqpoint{2.080182in}{0.582439in}}%
\pgfpathlineto{\pgfqpoint{2.080971in}{0.582789in}}%
\pgfpathlineto{\pgfqpoint{2.082550in}{0.584754in}}%
\pgfpathlineto{\pgfqpoint{2.082945in}{0.583585in}}%
\pgfpathlineto{\pgfqpoint{2.083340in}{0.582170in}}%
\pgfpathlineto{\pgfqpoint{2.084129in}{0.584637in}}%
\pgfpathlineto{\pgfqpoint{2.084524in}{0.584313in}}%
\pgfpathlineto{\pgfqpoint{2.084919in}{0.585249in}}%
\pgfpathlineto{\pgfqpoint{2.085708in}{0.591198in}}%
\pgfpathlineto{\pgfqpoint{2.086498in}{0.589118in}}%
\pgfpathlineto{\pgfqpoint{2.088471in}{0.581189in}}%
\pgfpathlineto{\pgfqpoint{2.089656in}{0.584481in}}%
\pgfpathlineto{\pgfqpoint{2.091235in}{0.589086in}}%
\pgfpathlineto{\pgfqpoint{2.091629in}{0.587544in}}%
\pgfpathlineto{\pgfqpoint{2.093208in}{0.581974in}}%
\pgfpathlineto{\pgfqpoint{2.094393in}{0.583742in}}%
\pgfpathlineto{\pgfqpoint{2.095577in}{0.582092in}}%
\pgfpathlineto{\pgfqpoint{2.097156in}{0.579506in}}%
\pgfpathlineto{\pgfqpoint{2.097945in}{0.578251in}}%
\pgfpathlineto{\pgfqpoint{2.099919in}{0.572576in}}%
\pgfpathlineto{\pgfqpoint{2.100314in}{0.572672in}}%
\pgfpathlineto{\pgfqpoint{2.101893in}{0.578472in}}%
\pgfpathlineto{\pgfqpoint{2.102682in}{0.577619in}}%
\pgfpathlineto{\pgfqpoint{2.103077in}{0.575979in}}%
\pgfpathlineto{\pgfqpoint{2.103867in}{0.576874in}}%
\pgfpathlineto{\pgfqpoint{2.105446in}{0.585501in}}%
\pgfpathlineto{\pgfqpoint{2.107419in}{0.583972in}}%
\pgfpathlineto{\pgfqpoint{2.107814in}{0.582430in}}%
\pgfpathlineto{\pgfqpoint{2.108209in}{0.583930in}}%
\pgfpathlineto{\pgfqpoint{2.111367in}{0.595172in}}%
\pgfpathlineto{\pgfqpoint{2.112156in}{0.593025in}}%
\pgfpathlineto{\pgfqpoint{2.112946in}{0.595236in}}%
\pgfpathlineto{\pgfqpoint{2.114130in}{0.598781in}}%
\pgfpathlineto{\pgfqpoint{2.114525in}{0.597516in}}%
\pgfpathlineto{\pgfqpoint{2.116499in}{0.595743in}}%
\pgfpathlineto{\pgfqpoint{2.116893in}{0.597240in}}%
\pgfpathlineto{\pgfqpoint{2.118078in}{0.595870in}}%
\pgfpathlineto{\pgfqpoint{2.119657in}{0.595727in}}%
\pgfpathlineto{\pgfqpoint{2.121236in}{0.591463in}}%
\pgfpathlineto{\pgfqpoint{2.122025in}{0.592234in}}%
\pgfpathlineto{\pgfqpoint{2.122815in}{0.591904in}}%
\pgfpathlineto{\pgfqpoint{2.124788in}{0.595644in}}%
\pgfpathlineto{\pgfqpoint{2.125973in}{0.596191in}}%
\pgfpathlineto{\pgfqpoint{2.127552in}{0.600131in}}%
\pgfpathlineto{\pgfqpoint{2.129130in}{0.598828in}}%
\pgfpathlineto{\pgfqpoint{2.131499in}{0.597011in}}%
\pgfpathlineto{\pgfqpoint{2.132288in}{0.597760in}}%
\pgfpathlineto{\pgfqpoint{2.132683in}{0.597023in}}%
\pgfpathlineto{\pgfqpoint{2.137815in}{0.578283in}}%
\pgfpathlineto{\pgfqpoint{2.140183in}{0.588298in}}%
\pgfpathlineto{\pgfqpoint{2.140973in}{0.583354in}}%
\pgfpathlineto{\pgfqpoint{2.142947in}{0.574141in}}%
\pgfpathlineto{\pgfqpoint{2.143736in}{0.574771in}}%
\pgfpathlineto{\pgfqpoint{2.146499in}{0.565614in}}%
\pgfpathlineto{\pgfqpoint{2.149657in}{0.567208in}}%
\pgfpathlineto{\pgfqpoint{2.153210in}{0.575121in}}%
\pgfpathlineto{\pgfqpoint{2.154394in}{0.574490in}}%
\pgfpathlineto{\pgfqpoint{2.154789in}{0.572542in}}%
\pgfpathlineto{\pgfqpoint{2.155973in}{0.574224in}}%
\pgfpathlineto{\pgfqpoint{2.156763in}{0.575994in}}%
\pgfpathlineto{\pgfqpoint{2.157552in}{0.574082in}}%
\pgfpathlineto{\pgfqpoint{2.158737in}{0.574275in}}%
\pgfpathlineto{\pgfqpoint{2.161500in}{0.570407in}}%
\pgfpathlineto{\pgfqpoint{2.161895in}{0.570725in}}%
\pgfpathlineto{\pgfqpoint{2.170184in}{0.590681in}}%
\pgfpathlineto{\pgfqpoint{2.170974in}{0.589031in}}%
\pgfpathlineto{\pgfqpoint{2.172948in}{0.586609in}}%
\pgfpathlineto{\pgfqpoint{2.179264in}{0.614647in}}%
\pgfpathlineto{\pgfqpoint{2.180053in}{0.612411in}}%
\pgfpathlineto{\pgfqpoint{2.183606in}{0.596477in}}%
\pgfpathlineto{\pgfqpoint{2.190317in}{0.589849in}}%
\pgfpathlineto{\pgfqpoint{2.191501in}{0.591382in}}%
\pgfpathlineto{\pgfqpoint{2.193869in}{0.598562in}}%
\pgfpathlineto{\pgfqpoint{2.197027in}{0.620749in}}%
\pgfpathlineto{\pgfqpoint{2.197422in}{0.619278in}}%
\pgfpathlineto{\pgfqpoint{2.201764in}{0.610239in}}%
\pgfpathlineto{\pgfqpoint{2.202554in}{0.610708in}}%
\pgfpathlineto{\pgfqpoint{2.202949in}{0.611535in}}%
\pgfpathlineto{\pgfqpoint{2.203738in}{0.610818in}}%
\pgfpathlineto{\pgfqpoint{2.206501in}{0.602882in}}%
\pgfpathlineto{\pgfqpoint{2.206896in}{0.603030in}}%
\pgfpathlineto{\pgfqpoint{2.208870in}{0.604624in}}%
\pgfpathlineto{\pgfqpoint{2.209659in}{0.606876in}}%
\pgfpathlineto{\pgfqpoint{2.210449in}{0.605674in}}%
\pgfpathlineto{\pgfqpoint{2.214791in}{0.595405in}}%
\pgfpathlineto{\pgfqpoint{2.215186in}{0.595974in}}%
\pgfpathlineto{\pgfqpoint{2.216370in}{0.595610in}}%
\pgfpathlineto{\pgfqpoint{2.216765in}{0.594281in}}%
\pgfpathlineto{\pgfqpoint{2.217554in}{0.596374in}}%
\pgfpathlineto{\pgfqpoint{2.219923in}{0.602918in}}%
\pgfpathlineto{\pgfqpoint{2.221896in}{0.602539in}}%
\pgfpathlineto{\pgfqpoint{2.227028in}{0.618050in}}%
\pgfpathlineto{\pgfqpoint{2.227818in}{0.617188in}}%
\pgfpathlineto{\pgfqpoint{2.231765in}{0.602702in}}%
\pgfpathlineto{\pgfqpoint{2.233739in}{0.599256in}}%
\pgfpathlineto{\pgfqpoint{2.234134in}{0.600340in}}%
\pgfpathlineto{\pgfqpoint{2.234528in}{0.598971in}}%
\pgfpathlineto{\pgfqpoint{2.236107in}{0.594606in}}%
\pgfpathlineto{\pgfqpoint{2.236502in}{0.596608in}}%
\pgfpathlineto{\pgfqpoint{2.238081in}{0.603290in}}%
\pgfpathlineto{\pgfqpoint{2.239265in}{0.602775in}}%
\pgfpathlineto{\pgfqpoint{2.241239in}{0.597070in}}%
\pgfpathlineto{\pgfqpoint{2.243213in}{0.589573in}}%
\pgfpathlineto{\pgfqpoint{2.243608in}{0.590148in}}%
\pgfpathlineto{\pgfqpoint{2.246371in}{0.594787in}}%
\pgfpathlineto{\pgfqpoint{2.248739in}{0.602038in}}%
\pgfpathlineto{\pgfqpoint{2.249134in}{0.601802in}}%
\pgfpathlineto{\pgfqpoint{2.253082in}{0.591488in}}%
\pgfpathlineto{\pgfqpoint{2.255055in}{0.596071in}}%
\pgfpathlineto{\pgfqpoint{2.255450in}{0.594833in}}%
\pgfpathlineto{\pgfqpoint{2.257424in}{0.589389in}}%
\pgfpathlineto{\pgfqpoint{2.261371in}{0.594025in}}%
\pgfpathlineto{\pgfqpoint{2.262161in}{0.589886in}}%
\pgfpathlineto{\pgfqpoint{2.262950in}{0.592103in}}%
\pgfpathlineto{\pgfqpoint{2.263345in}{0.593105in}}%
\pgfpathlineto{\pgfqpoint{2.264135in}{0.591130in}}%
\pgfpathlineto{\pgfqpoint{2.264924in}{0.588874in}}%
\pgfpathlineto{\pgfqpoint{2.266898in}{0.585679in}}%
\pgfpathlineto{\pgfqpoint{2.267293in}{0.585845in}}%
\pgfpathlineto{\pgfqpoint{2.270056in}{0.595719in}}%
\pgfpathlineto{\pgfqpoint{2.270845in}{0.594082in}}%
\pgfpathlineto{\pgfqpoint{2.272030in}{0.587847in}}%
\pgfpathlineto{\pgfqpoint{2.272819in}{0.592718in}}%
\pgfpathlineto{\pgfqpoint{2.275188in}{0.596258in}}%
\pgfpathlineto{\pgfqpoint{2.277161in}{0.591406in}}%
\pgfpathlineto{\pgfqpoint{2.277556in}{0.592697in}}%
\pgfpathlineto{\pgfqpoint{2.278346in}{0.595016in}}%
\pgfpathlineto{\pgfqpoint{2.278740in}{0.592799in}}%
\pgfpathlineto{\pgfqpoint{2.281109in}{0.582756in}}%
\pgfpathlineto{\pgfqpoint{2.282688in}{0.584065in}}%
\pgfpathlineto{\pgfqpoint{2.283083in}{0.583283in}}%
\pgfpathlineto{\pgfqpoint{2.283872in}{0.580854in}}%
\pgfpathlineto{\pgfqpoint{2.284267in}{0.584153in}}%
\pgfpathlineto{\pgfqpoint{2.286635in}{0.588845in}}%
\pgfpathlineto{\pgfqpoint{2.291767in}{0.576607in}}%
\pgfpathlineto{\pgfqpoint{2.292557in}{0.577155in}}%
\pgfpathlineto{\pgfqpoint{2.293346in}{0.582204in}}%
\pgfpathlineto{\pgfqpoint{2.294136in}{0.580609in}}%
\pgfpathlineto{\pgfqpoint{2.294925in}{0.577966in}}%
\pgfpathlineto{\pgfqpoint{2.295714in}{0.580269in}}%
\pgfpathlineto{\pgfqpoint{2.298872in}{0.584283in}}%
\pgfpathlineto{\pgfqpoint{2.300846in}{0.578415in}}%
\pgfpathlineto{\pgfqpoint{2.302425in}{0.579724in}}%
\pgfpathlineto{\pgfqpoint{2.303609in}{0.584124in}}%
\pgfpathlineto{\pgfqpoint{2.304399in}{0.581934in}}%
\pgfpathlineto{\pgfqpoint{2.304794in}{0.579918in}}%
\pgfpathlineto{\pgfqpoint{2.305978in}{0.581615in}}%
\pgfpathlineto{\pgfqpoint{2.309925in}{0.571173in}}%
\pgfpathlineto{\pgfqpoint{2.310320in}{0.571521in}}%
\pgfpathlineto{\pgfqpoint{2.311504in}{0.575762in}}%
\pgfpathlineto{\pgfqpoint{2.312294in}{0.574056in}}%
\pgfpathlineto{\pgfqpoint{2.313478in}{0.572153in}}%
\pgfpathlineto{\pgfqpoint{2.313873in}{0.573838in}}%
\pgfpathlineto{\pgfqpoint{2.315847in}{0.573709in}}%
\pgfpathlineto{\pgfqpoint{2.316636in}{0.571857in}}%
\pgfpathlineto{\pgfqpoint{2.317426in}{0.573787in}}%
\pgfpathlineto{\pgfqpoint{2.318610in}{0.578289in}}%
\pgfpathlineto{\pgfqpoint{2.319005in}{0.577323in}}%
\pgfpathlineto{\pgfqpoint{2.320189in}{0.574658in}}%
\pgfpathlineto{\pgfqpoint{2.320584in}{0.575728in}}%
\pgfpathlineto{\pgfqpoint{2.322952in}{0.593054in}}%
\pgfpathlineto{\pgfqpoint{2.323347in}{0.592714in}}%
\pgfpathlineto{\pgfqpoint{2.323742in}{0.591123in}}%
\pgfpathlineto{\pgfqpoint{2.324531in}{0.593248in}}%
\pgfpathlineto{\pgfqpoint{2.324926in}{0.592267in}}%
\pgfpathlineto{\pgfqpoint{2.325715in}{0.591078in}}%
\pgfpathlineto{\pgfqpoint{2.328479in}{0.587394in}}%
\pgfpathlineto{\pgfqpoint{2.330058in}{0.586310in}}%
\pgfpathlineto{\pgfqpoint{2.330452in}{0.587509in}}%
\pgfpathlineto{\pgfqpoint{2.330847in}{0.588470in}}%
\pgfpathlineto{\pgfqpoint{2.331242in}{0.585866in}}%
\pgfpathlineto{\pgfqpoint{2.333216in}{0.578387in}}%
\pgfpathlineto{\pgfqpoint{2.333610in}{0.579189in}}%
\pgfpathlineto{\pgfqpoint{2.334400in}{0.576638in}}%
\pgfpathlineto{\pgfqpoint{2.337558in}{0.567732in}}%
\pgfpathlineto{\pgfqpoint{2.339532in}{0.570488in}}%
\pgfpathlineto{\pgfqpoint{2.340716in}{0.575629in}}%
\pgfpathlineto{\pgfqpoint{2.341505in}{0.573853in}}%
\pgfpathlineto{\pgfqpoint{2.341900in}{0.571776in}}%
\pgfpathlineto{\pgfqpoint{2.342690in}{0.573116in}}%
\pgfpathlineto{\pgfqpoint{2.343479in}{0.575822in}}%
\pgfpathlineto{\pgfqpoint{2.343874in}{0.573983in}}%
\pgfpathlineto{\pgfqpoint{2.346637in}{0.566305in}}%
\pgfpathlineto{\pgfqpoint{2.347032in}{0.567602in}}%
\pgfpathlineto{\pgfqpoint{2.349006in}{0.572813in}}%
\pgfpathlineto{\pgfqpoint{2.352164in}{0.589399in}}%
\pgfpathlineto{\pgfqpoint{2.352953in}{0.589696in}}%
\pgfpathlineto{\pgfqpoint{2.354927in}{0.586424in}}%
\pgfpathlineto{\pgfqpoint{2.355716in}{0.586294in}}%
\pgfpathlineto{\pgfqpoint{2.358085in}{0.596640in}}%
\pgfpathlineto{\pgfqpoint{2.360453in}{0.588864in}}%
\pgfpathlineto{\pgfqpoint{2.362427in}{0.577867in}}%
\pgfpathlineto{\pgfqpoint{2.362822in}{0.579679in}}%
\pgfpathlineto{\pgfqpoint{2.365190in}{0.585687in}}%
\pgfpathlineto{\pgfqpoint{2.365585in}{0.585315in}}%
\pgfpathlineto{\pgfqpoint{2.367954in}{0.579090in}}%
\pgfpathlineto{\pgfqpoint{2.370322in}{0.572428in}}%
\pgfpathlineto{\pgfqpoint{2.370717in}{0.574347in}}%
\pgfpathlineto{\pgfqpoint{2.371901in}{0.582098in}}%
\pgfpathlineto{\pgfqpoint{2.372691in}{0.581189in}}%
\pgfpathlineto{\pgfqpoint{2.373480in}{0.577928in}}%
\pgfpathlineto{\pgfqpoint{2.373875in}{0.574857in}}%
\pgfpathlineto{\pgfqpoint{2.374664in}{0.576911in}}%
\pgfpathlineto{\pgfqpoint{2.375059in}{0.577857in}}%
\pgfpathlineto{\pgfqpoint{2.375454in}{0.576981in}}%
\pgfpathlineto{\pgfqpoint{2.377822in}{0.570384in}}%
\pgfpathlineto{\pgfqpoint{2.378217in}{0.571613in}}%
\pgfpathlineto{\pgfqpoint{2.380585in}{0.574732in}}%
\pgfpathlineto{\pgfqpoint{2.381375in}{0.573494in}}%
\pgfpathlineto{\pgfqpoint{2.382164in}{0.569895in}}%
\pgfpathlineto{\pgfqpoint{2.382954in}{0.572714in}}%
\pgfpathlineto{\pgfqpoint{2.383349in}{0.573514in}}%
\pgfpathlineto{\pgfqpoint{2.383743in}{0.572569in}}%
\pgfpathlineto{\pgfqpoint{2.384533in}{0.571159in}}%
\pgfpathlineto{\pgfqpoint{2.384928in}{0.573068in}}%
\pgfpathlineto{\pgfqpoint{2.385717in}{0.574894in}}%
\pgfpathlineto{\pgfqpoint{2.386112in}{0.574608in}}%
\pgfpathlineto{\pgfqpoint{2.386507in}{0.572860in}}%
\pgfpathlineto{\pgfqpoint{2.387296in}{0.575926in}}%
\pgfpathlineto{\pgfqpoint{2.391244in}{0.585541in}}%
\pgfpathlineto{\pgfqpoint{2.391638in}{0.583991in}}%
\pgfpathlineto{\pgfqpoint{2.392428in}{0.581927in}}%
\pgfpathlineto{\pgfqpoint{2.393217in}{0.583580in}}%
\pgfpathlineto{\pgfqpoint{2.395586in}{0.597570in}}%
\pgfpathlineto{\pgfqpoint{2.396375in}{0.594605in}}%
\pgfpathlineto{\pgfqpoint{2.401112in}{0.573835in}}%
\pgfpathlineto{\pgfqpoint{2.401902in}{0.575313in}}%
\pgfpathlineto{\pgfqpoint{2.402297in}{0.574440in}}%
\pgfpathlineto{\pgfqpoint{2.404665in}{0.586176in}}%
\pgfpathlineto{\pgfqpoint{2.405060in}{0.585255in}}%
\pgfpathlineto{\pgfqpoint{2.406639in}{0.581568in}}%
\pgfpathlineto{\pgfqpoint{2.407428in}{0.581925in}}%
\pgfpathlineto{\pgfqpoint{2.407823in}{0.582413in}}%
\pgfpathlineto{\pgfqpoint{2.408218in}{0.581803in}}%
\pgfpathlineto{\pgfqpoint{2.409007in}{0.579537in}}%
\pgfpathlineto{\pgfqpoint{2.409797in}{0.580815in}}%
\pgfpathlineto{\pgfqpoint{2.410981in}{0.583726in}}%
\pgfpathlineto{\pgfqpoint{2.411376in}{0.582157in}}%
\pgfpathlineto{\pgfqpoint{2.412955in}{0.577173in}}%
\pgfpathlineto{\pgfqpoint{2.413744in}{0.577896in}}%
\pgfpathlineto{\pgfqpoint{2.414139in}{0.578266in}}%
\pgfpathlineto{\pgfqpoint{2.414534in}{0.577521in}}%
\pgfpathlineto{\pgfqpoint{2.416113in}{0.575394in}}%
\pgfpathlineto{\pgfqpoint{2.416508in}{0.575587in}}%
\pgfpathlineto{\pgfqpoint{2.417297in}{0.575120in}}%
\pgfpathlineto{\pgfqpoint{2.418087in}{0.573443in}}%
\pgfpathlineto{\pgfqpoint{2.418481in}{0.574961in}}%
\pgfpathlineto{\pgfqpoint{2.418876in}{0.575357in}}%
\pgfpathlineto{\pgfqpoint{2.419271in}{0.573947in}}%
\pgfpathlineto{\pgfqpoint{2.422429in}{0.566132in}}%
\pgfpathlineto{\pgfqpoint{2.423218in}{0.565236in}}%
\pgfpathlineto{\pgfqpoint{2.423613in}{0.566403in}}%
\pgfpathlineto{\pgfqpoint{2.424797in}{0.573412in}}%
\pgfpathlineto{\pgfqpoint{2.425192in}{0.571630in}}%
\pgfpathlineto{\pgfqpoint{2.425587in}{0.566376in}}%
\pgfpathlineto{\pgfqpoint{2.426771in}{0.569701in}}%
\pgfpathlineto{\pgfqpoint{2.427955in}{0.573340in}}%
\pgfpathlineto{\pgfqpoint{2.428350in}{0.573175in}}%
\pgfpathlineto{\pgfqpoint{2.430719in}{0.569022in}}%
\pgfpathlineto{\pgfqpoint{2.431113in}{0.570347in}}%
\pgfpathlineto{\pgfqpoint{2.433877in}{0.575017in}}%
\pgfpathlineto{\pgfqpoint{2.436245in}{0.577570in}}%
\pgfpathlineto{\pgfqpoint{2.437035in}{0.576035in}}%
\pgfpathlineto{\pgfqpoint{2.437429in}{0.577879in}}%
\pgfpathlineto{\pgfqpoint{2.440982in}{0.592930in}}%
\pgfpathlineto{\pgfqpoint{2.441377in}{0.592341in}}%
\pgfpathlineto{\pgfqpoint{2.449272in}{0.569168in}}%
\pgfpathlineto{\pgfqpoint{2.450061in}{0.570634in}}%
\pgfpathlineto{\pgfqpoint{2.450851in}{0.570011in}}%
\pgfpathlineto{\pgfqpoint{2.451246in}{0.568778in}}%
\pgfpathlineto{\pgfqpoint{2.451640in}{0.569688in}}%
\pgfpathlineto{\pgfqpoint{2.452430in}{0.573057in}}%
\pgfpathlineto{\pgfqpoint{2.453219in}{0.570443in}}%
\pgfpathlineto{\pgfqpoint{2.453614in}{0.568692in}}%
\pgfpathlineto{\pgfqpoint{2.454798in}{0.570393in}}%
\pgfpathlineto{\pgfqpoint{2.457167in}{0.567672in}}%
\pgfpathlineto{\pgfqpoint{2.457562in}{0.568882in}}%
\pgfpathlineto{\pgfqpoint{2.459535in}{0.570244in}}%
\pgfpathlineto{\pgfqpoint{2.461114in}{0.568260in}}%
\pgfpathlineto{\pgfqpoint{2.462693in}{0.572239in}}%
\pgfpathlineto{\pgfqpoint{2.463088in}{0.572085in}}%
\pgfpathlineto{\pgfqpoint{2.463877in}{0.569454in}}%
\pgfpathlineto{\pgfqpoint{2.464667in}{0.565224in}}%
\pgfpathlineto{\pgfqpoint{2.465851in}{0.565434in}}%
\pgfpathlineto{\pgfqpoint{2.469799in}{0.575887in}}%
\pgfpathlineto{\pgfqpoint{2.470588in}{0.575352in}}%
\pgfpathlineto{\pgfqpoint{2.470983in}{0.574347in}}%
\pgfpathlineto{\pgfqpoint{2.471772in}{0.575037in}}%
\pgfpathlineto{\pgfqpoint{2.473351in}{0.578020in}}%
\pgfpathlineto{\pgfqpoint{2.473746in}{0.577060in}}%
\pgfpathlineto{\pgfqpoint{2.474930in}{0.575042in}}%
\pgfpathlineto{\pgfqpoint{2.475325in}{0.576444in}}%
\pgfpathlineto{\pgfqpoint{2.475720in}{0.576638in}}%
\pgfpathlineto{\pgfqpoint{2.477299in}{0.580777in}}%
\pgfpathlineto{\pgfqpoint{2.477694in}{0.579485in}}%
\pgfpathlineto{\pgfqpoint{2.478878in}{0.574478in}}%
\pgfpathlineto{\pgfqpoint{2.479667in}{0.577298in}}%
\pgfpathlineto{\pgfqpoint{2.480062in}{0.577782in}}%
\pgfpathlineto{\pgfqpoint{2.480852in}{0.576687in}}%
\pgfpathlineto{\pgfqpoint{2.482431in}{0.572254in}}%
\pgfpathlineto{\pgfqpoint{2.483615in}{0.573222in}}%
\pgfpathlineto{\pgfqpoint{2.484799in}{0.577596in}}%
\pgfpathlineto{\pgfqpoint{2.485194in}{0.575219in}}%
\pgfpathlineto{\pgfqpoint{2.485983in}{0.573404in}}%
\pgfpathlineto{\pgfqpoint{2.486378in}{0.574375in}}%
\pgfpathlineto{\pgfqpoint{2.489536in}{0.578986in}}%
\pgfpathlineto{\pgfqpoint{2.490326in}{0.578781in}}%
\pgfpathlineto{\pgfqpoint{2.490720in}{0.579830in}}%
\pgfpathlineto{\pgfqpoint{2.493089in}{0.580422in}}%
\pgfpathlineto{\pgfqpoint{2.493878in}{0.581728in}}%
\pgfpathlineto{\pgfqpoint{2.497036in}{0.569758in}}%
\pgfpathlineto{\pgfqpoint{2.498615in}{0.572457in}}%
\pgfpathlineto{\pgfqpoint{2.499010in}{0.572056in}}%
\pgfpathlineto{\pgfqpoint{2.501379in}{0.566454in}}%
\pgfpathlineto{\pgfqpoint{2.501773in}{0.569115in}}%
\pgfpathlineto{\pgfqpoint{2.502168in}{0.569243in}}%
\pgfpathlineto{\pgfqpoint{2.504537in}{0.577501in}}%
\pgfpathlineto{\pgfqpoint{2.504931in}{0.576514in}}%
\pgfpathlineto{\pgfqpoint{2.505721in}{0.574727in}}%
\pgfpathlineto{\pgfqpoint{2.506116in}{0.574315in}}%
\pgfpathlineto{\pgfqpoint{2.506905in}{0.575473in}}%
\pgfpathlineto{\pgfqpoint{2.508089in}{0.577398in}}%
\pgfpathlineto{\pgfqpoint{2.510063in}{0.579402in}}%
\pgfpathlineto{\pgfqpoint{2.513221in}{0.574197in}}%
\pgfpathlineto{\pgfqpoint{2.514011in}{0.575228in}}%
\pgfpathlineto{\pgfqpoint{2.515984in}{0.579470in}}%
\pgfpathlineto{\pgfqpoint{2.516379in}{0.579860in}}%
\pgfpathlineto{\pgfqpoint{2.516774in}{0.578229in}}%
\pgfpathlineto{\pgfqpoint{2.517169in}{0.576175in}}%
\pgfpathlineto{\pgfqpoint{2.517958in}{0.579530in}}%
\pgfpathlineto{\pgfqpoint{2.519932in}{0.583790in}}%
\pgfpathlineto{\pgfqpoint{2.520327in}{0.583339in}}%
\pgfpathlineto{\pgfqpoint{2.522300in}{0.584931in}}%
\pgfpathlineto{\pgfqpoint{2.523090in}{0.588896in}}%
\pgfpathlineto{\pgfqpoint{2.524274in}{0.587949in}}%
\pgfpathlineto{\pgfqpoint{2.524669in}{0.588070in}}%
\pgfpathlineto{\pgfqpoint{2.525064in}{0.586769in}}%
\pgfpathlineto{\pgfqpoint{2.526248in}{0.583466in}}%
\pgfpathlineto{\pgfqpoint{2.527037in}{0.584418in}}%
\pgfpathlineto{\pgfqpoint{2.528222in}{0.588089in}}%
\pgfpathlineto{\pgfqpoint{2.529011in}{0.586340in}}%
\pgfpathlineto{\pgfqpoint{2.531774in}{0.580969in}}%
\pgfpathlineto{\pgfqpoint{2.532959in}{0.577537in}}%
\pgfpathlineto{\pgfqpoint{2.533353in}{0.579341in}}%
\pgfpathlineto{\pgfqpoint{2.533748in}{0.579923in}}%
\pgfpathlineto{\pgfqpoint{2.534143in}{0.578840in}}%
\pgfpathlineto{\pgfqpoint{2.536906in}{0.571870in}}%
\pgfpathlineto{\pgfqpoint{2.537301in}{0.572435in}}%
\pgfpathlineto{\pgfqpoint{2.540064in}{0.581038in}}%
\pgfpathlineto{\pgfqpoint{2.545196in}{0.587267in}}%
\pgfpathlineto{\pgfqpoint{2.545590in}{0.586497in}}%
\pgfpathlineto{\pgfqpoint{2.548354in}{0.583091in}}%
\pgfpathlineto{\pgfqpoint{2.549538in}{0.584695in}}%
\pgfpathlineto{\pgfqpoint{2.551906in}{0.589089in}}%
\pgfpathlineto{\pgfqpoint{2.557433in}{0.607128in}}%
\pgfpathlineto{\pgfqpoint{2.558222in}{0.603798in}}%
\pgfpathlineto{\pgfqpoint{2.561380in}{0.587773in}}%
\pgfpathlineto{\pgfqpoint{2.562565in}{0.589741in}}%
\pgfpathlineto{\pgfqpoint{2.564933in}{0.595879in}}%
\pgfpathlineto{\pgfqpoint{2.565328in}{0.595602in}}%
\pgfpathlineto{\pgfqpoint{2.569670in}{0.597879in}}%
\pgfpathlineto{\pgfqpoint{2.576776in}{0.584283in}}%
\pgfpathlineto{\pgfqpoint{2.577170in}{0.587093in}}%
\pgfpathlineto{\pgfqpoint{2.578355in}{0.591274in}}%
\pgfpathlineto{\pgfqpoint{2.579144in}{0.594901in}}%
\pgfpathlineto{\pgfqpoint{2.579934in}{0.593383in}}%
\pgfpathlineto{\pgfqpoint{2.580723in}{0.591974in}}%
\pgfpathlineto{\pgfqpoint{2.581907in}{0.589564in}}%
\pgfpathlineto{\pgfqpoint{2.583881in}{0.594446in}}%
\pgfpathlineto{\pgfqpoint{2.585460in}{0.595267in}}%
\pgfpathlineto{\pgfqpoint{2.589013in}{0.578074in}}%
\pgfpathlineto{\pgfqpoint{2.589802in}{0.580383in}}%
\pgfpathlineto{\pgfqpoint{2.591776in}{0.586815in}}%
\pgfpathlineto{\pgfqpoint{2.592171in}{0.586329in}}%
\pgfpathlineto{\pgfqpoint{2.594539in}{0.580965in}}%
\pgfpathlineto{\pgfqpoint{2.595329in}{0.582011in}}%
\pgfpathlineto{\pgfqpoint{2.605198in}{0.609983in}}%
\pgfpathlineto{\pgfqpoint{2.605592in}{0.606714in}}%
\pgfpathlineto{\pgfqpoint{2.607566in}{0.598675in}}%
\pgfpathlineto{\pgfqpoint{2.607961in}{0.598140in}}%
\pgfpathlineto{\pgfqpoint{2.608356in}{0.598953in}}%
\pgfpathlineto{\pgfqpoint{2.609935in}{0.605593in}}%
\pgfpathlineto{\pgfqpoint{2.610724in}{0.602859in}}%
\pgfpathlineto{\pgfqpoint{2.612303in}{0.593141in}}%
\pgfpathlineto{\pgfqpoint{2.613487in}{0.588835in}}%
\pgfpathlineto{\pgfqpoint{2.613882in}{0.590324in}}%
\pgfpathlineto{\pgfqpoint{2.614672in}{0.588843in}}%
\pgfpathlineto{\pgfqpoint{2.615066in}{0.590488in}}%
\pgfpathlineto{\pgfqpoint{2.617040in}{0.596301in}}%
\pgfpathlineto{\pgfqpoint{2.617435in}{0.593941in}}%
\pgfpathlineto{\pgfqpoint{2.618619in}{0.597537in}}%
\pgfpathlineto{\pgfqpoint{2.619409in}{0.599440in}}%
\pgfpathlineto{\pgfqpoint{2.619803in}{0.597566in}}%
\pgfpathlineto{\pgfqpoint{2.620593in}{0.595327in}}%
\pgfpathlineto{\pgfqpoint{2.620988in}{0.598054in}}%
\pgfpathlineto{\pgfqpoint{2.621382in}{0.597873in}}%
\pgfpathlineto{\pgfqpoint{2.621777in}{0.599020in}}%
\pgfpathlineto{\pgfqpoint{2.622961in}{0.601712in}}%
\pgfpathlineto{\pgfqpoint{2.623356in}{0.601239in}}%
\pgfpathlineto{\pgfqpoint{2.624146in}{0.599138in}}%
\pgfpathlineto{\pgfqpoint{2.624935in}{0.600399in}}%
\pgfpathlineto{\pgfqpoint{2.625725in}{0.602585in}}%
\pgfpathlineto{\pgfqpoint{2.626514in}{0.601111in}}%
\pgfpathlineto{\pgfqpoint{2.629277in}{0.595498in}}%
\pgfpathlineto{\pgfqpoint{2.632040in}{0.585233in}}%
\pgfpathlineto{\pgfqpoint{2.638751in}{0.597323in}}%
\pgfpathlineto{\pgfqpoint{2.641120in}{0.604269in}}%
\pgfpathlineto{\pgfqpoint{2.643488in}{0.597411in}}%
\pgfpathlineto{\pgfqpoint{2.645462in}{0.588448in}}%
\pgfpathlineto{\pgfqpoint{2.645857in}{0.589648in}}%
\pgfpathlineto{\pgfqpoint{2.651383in}{0.609047in}}%
\pgfpathlineto{\pgfqpoint{2.652173in}{0.606765in}}%
\pgfpathlineto{\pgfqpoint{2.658094in}{0.595835in}}%
\pgfpathlineto{\pgfqpoint{2.660068in}{0.599415in}}%
\pgfpathlineto{\pgfqpoint{2.660462in}{0.598643in}}%
\pgfpathlineto{\pgfqpoint{2.665199in}{0.595714in}}%
\pgfpathlineto{\pgfqpoint{2.666778in}{0.593502in}}%
\pgfpathlineto{\pgfqpoint{2.669936in}{0.602845in}}%
\pgfpathlineto{\pgfqpoint{2.670331in}{0.602679in}}%
\pgfpathlineto{\pgfqpoint{2.670726in}{0.604568in}}%
\pgfpathlineto{\pgfqpoint{2.671515in}{0.601699in}}%
\pgfpathlineto{\pgfqpoint{2.672305in}{0.598213in}}%
\pgfpathlineto{\pgfqpoint{2.673094in}{0.601258in}}%
\pgfpathlineto{\pgfqpoint{2.674279in}{0.607715in}}%
\pgfpathlineto{\pgfqpoint{2.675068in}{0.605893in}}%
\pgfpathlineto{\pgfqpoint{2.678621in}{0.594555in}}%
\pgfpathlineto{\pgfqpoint{2.679410in}{0.591680in}}%
\pgfpathlineto{\pgfqpoint{2.680595in}{0.592189in}}%
\pgfpathlineto{\pgfqpoint{2.681384in}{0.594588in}}%
\pgfpathlineto{\pgfqpoint{2.683753in}{0.600725in}}%
\pgfpathlineto{\pgfqpoint{2.684147in}{0.600413in}}%
\pgfpathlineto{\pgfqpoint{2.684937in}{0.601583in}}%
\pgfpathlineto{\pgfqpoint{2.685332in}{0.599339in}}%
\pgfpathlineto{\pgfqpoint{2.686911in}{0.593204in}}%
\pgfpathlineto{\pgfqpoint{2.687305in}{0.593394in}}%
\pgfpathlineto{\pgfqpoint{2.687700in}{0.590737in}}%
\pgfpathlineto{\pgfqpoint{2.688884in}{0.593174in}}%
\pgfpathlineto{\pgfqpoint{2.690069in}{0.596349in}}%
\pgfpathlineto{\pgfqpoint{2.691253in}{0.594660in}}%
\pgfpathlineto{\pgfqpoint{2.694411in}{0.588778in}}%
\pgfpathlineto{\pgfqpoint{2.694806in}{0.589014in}}%
\pgfpathlineto{\pgfqpoint{2.695595in}{0.585867in}}%
\pgfpathlineto{\pgfqpoint{2.696385in}{0.588184in}}%
\pgfpathlineto{\pgfqpoint{2.697964in}{0.592804in}}%
\pgfpathlineto{\pgfqpoint{2.698358in}{0.590744in}}%
\pgfpathlineto{\pgfqpoint{2.700727in}{0.583317in}}%
\pgfpathlineto{\pgfqpoint{2.701516in}{0.583821in}}%
\pgfpathlineto{\pgfqpoint{2.701911in}{0.583360in}}%
\pgfpathlineto{\pgfqpoint{2.702306in}{0.584996in}}%
\pgfpathlineto{\pgfqpoint{2.703885in}{0.589200in}}%
\pgfpathlineto{\pgfqpoint{2.704280in}{0.586925in}}%
\pgfpathlineto{\pgfqpoint{2.705069in}{0.586147in}}%
\pgfpathlineto{\pgfqpoint{2.705464in}{0.587538in}}%
\pgfpathlineto{\pgfqpoint{2.705859in}{0.590068in}}%
\pgfpathlineto{\pgfqpoint{2.706648in}{0.586943in}}%
\pgfpathlineto{\pgfqpoint{2.707438in}{0.583700in}}%
\pgfpathlineto{\pgfqpoint{2.708227in}{0.585604in}}%
\pgfpathlineto{\pgfqpoint{2.708622in}{0.586233in}}%
\pgfpathlineto{\pgfqpoint{2.709017in}{0.585278in}}%
\pgfpathlineto{\pgfqpoint{2.710990in}{0.578443in}}%
\pgfpathlineto{\pgfqpoint{2.711385in}{0.578918in}}%
\pgfpathlineto{\pgfqpoint{2.713359in}{0.585412in}}%
\pgfpathlineto{\pgfqpoint{2.714938in}{0.593758in}}%
\pgfpathlineto{\pgfqpoint{2.716122in}{0.591026in}}%
\pgfpathlineto{\pgfqpoint{2.717701in}{0.585895in}}%
\pgfpathlineto{\pgfqpoint{2.718490in}{0.586600in}}%
\pgfpathlineto{\pgfqpoint{2.720464in}{0.588526in}}%
\pgfpathlineto{\pgfqpoint{2.722833in}{0.585830in}}%
\pgfpathlineto{\pgfqpoint{2.724017in}{0.587967in}}%
\pgfpathlineto{\pgfqpoint{2.724806in}{0.586619in}}%
\pgfpathlineto{\pgfqpoint{2.725596in}{0.583728in}}%
\pgfpathlineto{\pgfqpoint{2.727570in}{0.578225in}}%
\pgfpathlineto{\pgfqpoint{2.730728in}{0.580482in}}%
\pgfpathlineto{\pgfqpoint{2.732701in}{0.578466in}}%
\pgfpathlineto{\pgfqpoint{2.733096in}{0.576538in}}%
\pgfpathlineto{\pgfqpoint{2.734280in}{0.576838in}}%
\pgfpathlineto{\pgfqpoint{2.739017in}{0.585302in}}%
\pgfpathlineto{\pgfqpoint{2.739412in}{0.585241in}}%
\pgfpathlineto{\pgfqpoint{2.741781in}{0.572159in}}%
\pgfpathlineto{\pgfqpoint{2.742570in}{0.574008in}}%
\pgfpathlineto{\pgfqpoint{2.744939in}{0.580521in}}%
\pgfpathlineto{\pgfqpoint{2.745333in}{0.578520in}}%
\pgfpathlineto{\pgfqpoint{2.752439in}{0.564914in}}%
\pgfpathlineto{\pgfqpoint{2.752834in}{0.565952in}}%
\pgfpathlineto{\pgfqpoint{2.756386in}{0.574049in}}%
\pgfpathlineto{\pgfqpoint{2.761123in}{0.590797in}}%
\pgfpathlineto{\pgfqpoint{2.761518in}{0.590207in}}%
\pgfpathlineto{\pgfqpoint{2.761913in}{0.590857in}}%
\pgfpathlineto{\pgfqpoint{2.762702in}{0.590224in}}%
\pgfpathlineto{\pgfqpoint{2.763492in}{0.587790in}}%
\pgfpathlineto{\pgfqpoint{2.764676in}{0.588164in}}%
\pgfpathlineto{\pgfqpoint{2.766255in}{0.587495in}}%
\pgfpathlineto{\pgfqpoint{2.767834in}{0.592973in}}%
\pgfpathlineto{\pgfqpoint{2.768624in}{0.596416in}}%
\pgfpathlineto{\pgfqpoint{2.769413in}{0.595082in}}%
\pgfpathlineto{\pgfqpoint{2.772176in}{0.591668in}}%
\pgfpathlineto{\pgfqpoint{2.773755in}{0.589542in}}%
\pgfpathlineto{\pgfqpoint{2.774150in}{0.590377in}}%
\pgfpathlineto{\pgfqpoint{2.777308in}{0.598233in}}%
\pgfpathlineto{\pgfqpoint{2.778492in}{0.597130in}}%
\pgfpathlineto{\pgfqpoint{2.779282in}{0.596678in}}%
\pgfpathlineto{\pgfqpoint{2.779677in}{0.597819in}}%
\pgfpathlineto{\pgfqpoint{2.781256in}{0.606668in}}%
\pgfpathlineto{\pgfqpoint{2.782045in}{0.602702in}}%
\pgfpathlineto{\pgfqpoint{2.782835in}{0.597055in}}%
\pgfpathlineto{\pgfqpoint{2.783624in}{0.599828in}}%
\pgfpathlineto{\pgfqpoint{2.787177in}{0.606529in}}%
\pgfpathlineto{\pgfqpoint{2.789545in}{0.606588in}}%
\pgfpathlineto{\pgfqpoint{2.789940in}{0.605301in}}%
\pgfpathlineto{\pgfqpoint{2.790730in}{0.606494in}}%
\pgfpathlineto{\pgfqpoint{2.792703in}{0.606436in}}%
\pgfpathlineto{\pgfqpoint{2.794282in}{0.608381in}}%
\pgfpathlineto{\pgfqpoint{2.794677in}{0.610206in}}%
\pgfpathlineto{\pgfqpoint{2.795466in}{0.609070in}}%
\pgfpathlineto{\pgfqpoint{2.799414in}{0.596130in}}%
\pgfpathlineto{\pgfqpoint{2.801388in}{0.588026in}}%
\pgfpathlineto{\pgfqpoint{2.801782in}{0.588224in}}%
\pgfpathlineto{\pgfqpoint{2.802177in}{0.588827in}}%
\pgfpathlineto{\pgfqpoint{2.803361in}{0.586199in}}%
\pgfpathlineto{\pgfqpoint{2.803756in}{0.586778in}}%
\pgfpathlineto{\pgfqpoint{2.807704in}{0.594186in}}%
\pgfpathlineto{\pgfqpoint{2.808098in}{0.593744in}}%
\pgfpathlineto{\pgfqpoint{2.810862in}{0.602690in}}%
\pgfpathlineto{\pgfqpoint{2.812835in}{0.606344in}}%
\pgfpathlineto{\pgfqpoint{2.813230in}{0.605565in}}%
\pgfpathlineto{\pgfqpoint{2.815204in}{0.600196in}}%
\pgfpathlineto{\pgfqpoint{2.815599in}{0.602316in}}%
\pgfpathlineto{\pgfqpoint{2.817572in}{0.607245in}}%
\pgfpathlineto{\pgfqpoint{2.819546in}{0.610737in}}%
\pgfpathlineto{\pgfqpoint{2.820730in}{0.611110in}}%
\pgfpathlineto{\pgfqpoint{2.823099in}{0.602661in}}%
\pgfpathlineto{\pgfqpoint{2.824283in}{0.597639in}}%
\pgfpathlineto{\pgfqpoint{2.825862in}{0.594354in}}%
\pgfpathlineto{\pgfqpoint{2.826652in}{0.595067in}}%
\pgfpathlineto{\pgfqpoint{2.827836in}{0.594250in}}%
\pgfpathlineto{\pgfqpoint{2.828231in}{0.595737in}}%
\pgfpathlineto{\pgfqpoint{2.829415in}{0.596095in}}%
\pgfpathlineto{\pgfqpoint{2.830994in}{0.589526in}}%
\pgfpathlineto{\pgfqpoint{2.831783in}{0.591020in}}%
\pgfpathlineto{\pgfqpoint{2.832178in}{0.593081in}}%
\pgfpathlineto{\pgfqpoint{2.832968in}{0.591705in}}%
\pgfpathlineto{\pgfqpoint{2.834547in}{0.589249in}}%
\pgfpathlineto{\pgfqpoint{2.834941in}{0.589919in}}%
\pgfpathlineto{\pgfqpoint{2.838494in}{0.600053in}}%
\pgfpathlineto{\pgfqpoint{2.838889in}{0.599814in}}%
\pgfpathlineto{\pgfqpoint{2.840073in}{0.599282in}}%
\pgfpathlineto{\pgfqpoint{2.842047in}{0.593903in}}%
\pgfpathlineto{\pgfqpoint{2.843231in}{0.594564in}}%
\pgfpathlineto{\pgfqpoint{2.844415in}{0.600697in}}%
\pgfpathlineto{\pgfqpoint{2.845205in}{0.597687in}}%
\pgfpathlineto{\pgfqpoint{2.845994in}{0.593889in}}%
\pgfpathlineto{\pgfqpoint{2.846389in}{0.595432in}}%
\pgfpathlineto{\pgfqpoint{2.848363in}{0.601085in}}%
\pgfpathlineto{\pgfqpoint{2.848758in}{0.600312in}}%
\pgfpathlineto{\pgfqpoint{2.849152in}{0.600562in}}%
\pgfpathlineto{\pgfqpoint{2.850731in}{0.593631in}}%
\pgfpathlineto{\pgfqpoint{2.851126in}{0.595146in}}%
\pgfpathlineto{\pgfqpoint{2.853100in}{0.600178in}}%
\pgfpathlineto{\pgfqpoint{2.855863in}{0.594098in}}%
\pgfpathlineto{\pgfqpoint{2.858626in}{0.599314in}}%
\pgfpathlineto{\pgfqpoint{2.860205in}{0.593637in}}%
\pgfpathlineto{\pgfqpoint{2.862179in}{0.586816in}}%
\pgfpathlineto{\pgfqpoint{2.869679in}{0.574150in}}%
\pgfpathlineto{\pgfqpoint{2.873232in}{0.595396in}}%
\pgfpathlineto{\pgfqpoint{2.874811in}{0.589856in}}%
\pgfpathlineto{\pgfqpoint{2.875206in}{0.591352in}}%
\pgfpathlineto{\pgfqpoint{2.877180in}{0.597625in}}%
\pgfpathlineto{\pgfqpoint{2.878759in}{0.603471in}}%
\pgfpathlineto{\pgfqpoint{2.879153in}{0.601266in}}%
\pgfpathlineto{\pgfqpoint{2.881127in}{0.594670in}}%
\pgfpathlineto{\pgfqpoint{2.881522in}{0.595214in}}%
\pgfpathlineto{\pgfqpoint{2.882311in}{0.599123in}}%
\pgfpathlineto{\pgfqpoint{2.883101in}{0.597305in}}%
\pgfpathlineto{\pgfqpoint{2.883890in}{0.593671in}}%
\pgfpathlineto{\pgfqpoint{2.888627in}{0.567211in}}%
\pgfpathlineto{\pgfqpoint{2.889811in}{0.569744in}}%
\pgfpathlineto{\pgfqpoint{2.892575in}{0.581707in}}%
\pgfpathlineto{\pgfqpoint{2.892969in}{0.581543in}}%
\pgfpathlineto{\pgfqpoint{2.894154in}{0.580852in}}%
\pgfpathlineto{\pgfqpoint{2.894548in}{0.581476in}}%
\pgfpathlineto{\pgfqpoint{2.895733in}{0.585342in}}%
\pgfpathlineto{\pgfqpoint{2.896917in}{0.582711in}}%
\pgfpathlineto{\pgfqpoint{2.899680in}{0.575387in}}%
\pgfpathlineto{\pgfqpoint{2.900470in}{0.576053in}}%
\pgfpathlineto{\pgfqpoint{2.902443in}{0.577605in}}%
\pgfpathlineto{\pgfqpoint{2.903628in}{0.574020in}}%
\pgfpathlineto{\pgfqpoint{2.904417in}{0.574409in}}%
\pgfpathlineto{\pgfqpoint{2.907180in}{0.586702in}}%
\pgfpathlineto{\pgfqpoint{2.908365in}{0.583012in}}%
\pgfpathlineto{\pgfqpoint{2.908759in}{0.582985in}}%
\pgfpathlineto{\pgfqpoint{2.909549in}{0.578378in}}%
\pgfpathlineto{\pgfqpoint{2.910338in}{0.581399in}}%
\pgfpathlineto{\pgfqpoint{2.913102in}{0.579579in}}%
\pgfpathlineto{\pgfqpoint{2.920602in}{0.594438in}}%
\pgfpathlineto{\pgfqpoint{2.921391in}{0.593541in}}%
\pgfpathlineto{\pgfqpoint{2.922970in}{0.593152in}}%
\pgfpathlineto{\pgfqpoint{2.925339in}{0.601696in}}%
\pgfpathlineto{\pgfqpoint{2.926128in}{0.600964in}}%
\pgfpathlineto{\pgfqpoint{2.927707in}{0.600621in}}%
\pgfpathlineto{\pgfqpoint{2.928497in}{0.597211in}}%
\pgfpathlineto{\pgfqpoint{2.929681in}{0.598526in}}%
\pgfpathlineto{\pgfqpoint{2.933629in}{0.607099in}}%
\pgfpathlineto{\pgfqpoint{2.935997in}{0.600092in}}%
\pgfpathlineto{\pgfqpoint{2.937181in}{0.600839in}}%
\pgfpathlineto{\pgfqpoint{2.937971in}{0.602397in}}%
\pgfpathlineto{\pgfqpoint{2.938366in}{0.600197in}}%
\pgfpathlineto{\pgfqpoint{2.940734in}{0.590428in}}%
\pgfpathlineto{\pgfqpoint{2.941129in}{0.590027in}}%
\pgfpathlineto{\pgfqpoint{2.943103in}{0.594082in}}%
\pgfpathlineto{\pgfqpoint{2.944287in}{0.590888in}}%
\pgfpathlineto{\pgfqpoint{2.944682in}{0.592630in}}%
\pgfpathlineto{\pgfqpoint{2.946261in}{0.598789in}}%
\pgfpathlineto{\pgfqpoint{2.947445in}{0.596963in}}%
\pgfpathlineto{\pgfqpoint{2.948234in}{0.595704in}}%
\pgfpathlineto{\pgfqpoint{2.951787in}{0.586962in}}%
\pgfpathlineto{\pgfqpoint{2.952182in}{0.588835in}}%
\pgfpathlineto{\pgfqpoint{2.953366in}{0.591044in}}%
\pgfpathlineto{\pgfqpoint{2.953761in}{0.589532in}}%
\pgfpathlineto{\pgfqpoint{2.954156in}{0.588326in}}%
\pgfpathlineto{\pgfqpoint{2.954945in}{0.590056in}}%
\pgfpathlineto{\pgfqpoint{2.963235in}{0.607835in}}%
\pgfpathlineto{\pgfqpoint{2.964814in}{0.606911in}}%
\pgfpathlineto{\pgfqpoint{2.965998in}{0.606108in}}%
\pgfpathlineto{\pgfqpoint{2.966393in}{0.606517in}}%
\pgfpathlineto{\pgfqpoint{2.968366in}{0.606111in}}%
\pgfpathlineto{\pgfqpoint{2.970340in}{0.605905in}}%
\pgfpathlineto{\pgfqpoint{2.973103in}{0.604473in}}%
\pgfpathlineto{\pgfqpoint{2.973498in}{0.605106in}}%
\pgfpathlineto{\pgfqpoint{2.974288in}{0.606763in}}%
\pgfpathlineto{\pgfqpoint{2.974682in}{0.606270in}}%
\pgfpathlineto{\pgfqpoint{2.979025in}{0.587074in}}%
\pgfpathlineto{\pgfqpoint{2.980604in}{0.579468in}}%
\pgfpathlineto{\pgfqpoint{2.981393in}{0.580409in}}%
\pgfpathlineto{\pgfqpoint{2.982972in}{0.586167in}}%
\pgfpathlineto{\pgfqpoint{2.983762in}{0.585634in}}%
\pgfpathlineto{\pgfqpoint{2.984156in}{0.585028in}}%
\pgfpathlineto{\pgfqpoint{2.984946in}{0.586600in}}%
\pgfpathlineto{\pgfqpoint{2.985735in}{0.588333in}}%
\pgfpathlineto{\pgfqpoint{2.987709in}{0.591376in}}%
\pgfpathlineto{\pgfqpoint{2.989683in}{0.593284in}}%
\pgfpathlineto{\pgfqpoint{2.990078in}{0.592780in}}%
\pgfpathlineto{\pgfqpoint{2.992051in}{0.586352in}}%
\pgfpathlineto{\pgfqpoint{2.992446in}{0.586837in}}%
\pgfpathlineto{\pgfqpoint{2.995999in}{0.591852in}}%
\pgfpathlineto{\pgfqpoint{2.997973in}{0.597800in}}%
\pgfpathlineto{\pgfqpoint{2.999552in}{0.592793in}}%
\pgfpathlineto{\pgfqpoint{3.000736in}{0.595285in}}%
\pgfpathlineto{\pgfqpoint{3.001131in}{0.595667in}}%
\pgfpathlineto{\pgfqpoint{3.001525in}{0.594658in}}%
\pgfpathlineto{\pgfqpoint{3.002315in}{0.593973in}}%
\pgfpathlineto{\pgfqpoint{3.002710in}{0.594851in}}%
\pgfpathlineto{\pgfqpoint{3.005078in}{0.599725in}}%
\pgfpathlineto{\pgfqpoint{3.005473in}{0.599184in}}%
\pgfpathlineto{\pgfqpoint{3.009815in}{0.587510in}}%
\pgfpathlineto{\pgfqpoint{3.010210in}{0.589232in}}%
\pgfpathlineto{\pgfqpoint{3.012973in}{0.596098in}}%
\pgfpathlineto{\pgfqpoint{3.016131in}{0.598731in}}%
\pgfpathlineto{\pgfqpoint{3.019289in}{0.602175in}}%
\pgfpathlineto{\pgfqpoint{3.020473in}{0.605582in}}%
\pgfpathlineto{\pgfqpoint{3.020868in}{0.603615in}}%
\pgfpathlineto{\pgfqpoint{3.025210in}{0.594538in}}%
\pgfpathlineto{\pgfqpoint{3.026000in}{0.595374in}}%
\pgfpathlineto{\pgfqpoint{3.027579in}{0.589117in}}%
\pgfpathlineto{\pgfqpoint{3.027974in}{0.591661in}}%
\pgfpathlineto{\pgfqpoint{3.029158in}{0.596097in}}%
\pgfpathlineto{\pgfqpoint{3.029553in}{0.594631in}}%
\pgfpathlineto{\pgfqpoint{3.031132in}{0.592517in}}%
\pgfpathlineto{\pgfqpoint{3.031526in}{0.592751in}}%
\pgfpathlineto{\pgfqpoint{3.032316in}{0.595275in}}%
\pgfpathlineto{\pgfqpoint{3.033105in}{0.593098in}}%
\pgfpathlineto{\pgfqpoint{3.034290in}{0.588589in}}%
\pgfpathlineto{\pgfqpoint{3.035474in}{0.591360in}}%
\pgfpathlineto{\pgfqpoint{3.038237in}{0.596666in}}%
\pgfpathlineto{\pgfqpoint{3.039027in}{0.597111in}}%
\pgfpathlineto{\pgfqpoint{3.039421in}{0.596360in}}%
\pgfpathlineto{\pgfqpoint{3.042974in}{0.591193in}}%
\pgfpathlineto{\pgfqpoint{3.044553in}{0.589327in}}%
\pgfpathlineto{\pgfqpoint{3.046527in}{0.592590in}}%
\pgfpathlineto{\pgfqpoint{3.047316in}{0.590321in}}%
\pgfpathlineto{\pgfqpoint{3.048106in}{0.591340in}}%
\pgfpathlineto{\pgfqpoint{3.048500in}{0.591455in}}%
\pgfpathlineto{\pgfqpoint{3.051658in}{0.579380in}}%
\pgfpathlineto{\pgfqpoint{3.054422in}{0.588197in}}%
\pgfpathlineto{\pgfqpoint{3.054816in}{0.586391in}}%
\pgfpathlineto{\pgfqpoint{3.056395in}{0.581714in}}%
\pgfpathlineto{\pgfqpoint{3.057580in}{0.587372in}}%
\pgfpathlineto{\pgfqpoint{3.058369in}{0.585384in}}%
\pgfpathlineto{\pgfqpoint{3.058764in}{0.584758in}}%
\pgfpathlineto{\pgfqpoint{3.059159in}{0.586866in}}%
\pgfpathlineto{\pgfqpoint{3.063106in}{0.599744in}}%
\pgfpathlineto{\pgfqpoint{3.063501in}{0.600752in}}%
\pgfpathlineto{\pgfqpoint{3.063896in}{0.599170in}}%
\pgfpathlineto{\pgfqpoint{3.067843in}{0.589554in}}%
\pgfpathlineto{\pgfqpoint{3.070606in}{0.592576in}}%
\pgfpathlineto{\pgfqpoint{3.074554in}{0.604777in}}%
\pgfpathlineto{\pgfqpoint{3.075343in}{0.603173in}}%
\pgfpathlineto{\pgfqpoint{3.080870in}{0.587601in}}%
\pgfpathlineto{\pgfqpoint{3.082844in}{0.590026in}}%
\pgfpathlineto{\pgfqpoint{3.084423in}{0.594240in}}%
\pgfpathlineto{\pgfqpoint{3.087975in}{0.585319in}}%
\pgfpathlineto{\pgfqpoint{3.088765in}{0.585695in}}%
\pgfpathlineto{\pgfqpoint{3.089160in}{0.584892in}}%
\pgfpathlineto{\pgfqpoint{3.093897in}{0.573561in}}%
\pgfpathlineto{\pgfqpoint{3.095081in}{0.575544in}}%
\pgfpathlineto{\pgfqpoint{3.098239in}{0.574183in}}%
\pgfpathlineto{\pgfqpoint{3.098634in}{0.572958in}}%
\pgfpathlineto{\pgfqpoint{3.099818in}{0.574202in}}%
\pgfpathlineto{\pgfqpoint{3.102976in}{0.571135in}}%
\pgfpathlineto{\pgfqpoint{3.100607in}{0.575148in}}%
\pgfpathlineto{\pgfqpoint{3.103371in}{0.571542in}}%
\pgfpathlineto{\pgfqpoint{3.106529in}{0.575258in}}%
\pgfpathlineto{\pgfqpoint{3.109292in}{0.582232in}}%
\pgfpathlineto{\pgfqpoint{3.109687in}{0.581681in}}%
\pgfpathlineto{\pgfqpoint{3.110476in}{0.579443in}}%
\pgfpathlineto{\pgfqpoint{3.110871in}{0.581354in}}%
\pgfpathlineto{\pgfqpoint{3.113634in}{0.585728in}}%
\pgfpathlineto{\pgfqpoint{3.117187in}{0.571919in}}%
\pgfpathlineto{\pgfqpoint{3.118766in}{0.559947in}}%
\pgfpathlineto{\pgfqpoint{3.119555in}{0.563102in}}%
\pgfpathlineto{\pgfqpoint{3.120740in}{0.561607in}}%
\pgfpathlineto{\pgfqpoint{3.121924in}{0.559788in}}%
\pgfpathlineto{\pgfqpoint{3.122319in}{0.560314in}}%
\pgfpathlineto{\pgfqpoint{3.127450in}{0.570637in}}%
\pgfpathlineto{\pgfqpoint{3.127845in}{0.570098in}}%
\pgfpathlineto{\pgfqpoint{3.129029in}{0.567157in}}%
\pgfpathlineto{\pgfqpoint{3.129819in}{0.568536in}}%
\pgfpathlineto{\pgfqpoint{3.131003in}{0.571567in}}%
\pgfpathlineto{\pgfqpoint{3.131792in}{0.570483in}}%
\pgfpathlineto{\pgfqpoint{3.132582in}{0.571729in}}%
\pgfpathlineto{\pgfqpoint{3.133371in}{0.570407in}}%
\pgfpathlineto{\pgfqpoint{3.134950in}{0.565985in}}%
\pgfpathlineto{\pgfqpoint{3.136135in}{0.567475in}}%
\pgfpathlineto{\pgfqpoint{3.136924in}{0.568777in}}%
\pgfpathlineto{\pgfqpoint{3.137714in}{0.567500in}}%
\pgfpathlineto{\pgfqpoint{3.139687in}{0.561314in}}%
\pgfpathlineto{\pgfqpoint{3.140477in}{0.563756in}}%
\pgfpathlineto{\pgfqpoint{3.142451in}{0.568154in}}%
\pgfpathlineto{\pgfqpoint{3.142845in}{0.567699in}}%
\pgfpathlineto{\pgfqpoint{3.144424in}{0.563944in}}%
\pgfpathlineto{\pgfqpoint{3.145214in}{0.562549in}}%
\pgfpathlineto{\pgfqpoint{3.145609in}{0.563667in}}%
\pgfpathlineto{\pgfqpoint{3.146003in}{0.564169in}}%
\pgfpathlineto{\pgfqpoint{3.148372in}{0.576024in}}%
\pgfpathlineto{\pgfqpoint{3.148767in}{0.574306in}}%
\pgfpathlineto{\pgfqpoint{3.151925in}{0.568286in}}%
\pgfpathlineto{\pgfqpoint{3.153109in}{0.571022in}}%
\pgfpathlineto{\pgfqpoint{3.154293in}{0.572554in}}%
\pgfpathlineto{\pgfqpoint{3.154688in}{0.570501in}}%
\pgfpathlineto{\pgfqpoint{3.157056in}{0.560622in}}%
\pgfpathlineto{\pgfqpoint{3.157451in}{0.561813in}}%
\pgfpathlineto{\pgfqpoint{3.160214in}{0.567873in}}%
\pgfpathlineto{\pgfqpoint{3.161004in}{0.571777in}}%
\pgfpathlineto{\pgfqpoint{3.162978in}{0.578443in}}%
\pgfpathlineto{\pgfqpoint{3.163372in}{0.578267in}}%
\pgfpathlineto{\pgfqpoint{3.165346in}{0.575355in}}%
\pgfpathlineto{\pgfqpoint{3.166925in}{0.582603in}}%
\pgfpathlineto{\pgfqpoint{3.167320in}{0.582274in}}%
\pgfpathlineto{\pgfqpoint{3.172452in}{0.556035in}}%
\pgfpathlineto{\pgfqpoint{3.172846in}{0.556080in}}%
\pgfpathlineto{\pgfqpoint{3.173241in}{0.554503in}}%
\pgfpathlineto{\pgfqpoint{3.173636in}{0.556657in}}%
\pgfpathlineto{\pgfqpoint{3.176399in}{0.564528in}}%
\pgfpathlineto{\pgfqpoint{3.177189in}{0.564178in}}%
\pgfpathlineto{\pgfqpoint{3.177583in}{0.565477in}}%
\pgfpathlineto{\pgfqpoint{3.181926in}{0.580003in}}%
\pgfpathlineto{\pgfqpoint{3.182320in}{0.580631in}}%
\pgfpathlineto{\pgfqpoint{3.182715in}{0.578742in}}%
\pgfpathlineto{\pgfqpoint{3.183899in}{0.575960in}}%
\pgfpathlineto{\pgfqpoint{3.184294in}{0.576334in}}%
\pgfpathlineto{\pgfqpoint{3.188242in}{0.585465in}}%
\pgfpathlineto{\pgfqpoint{3.189821in}{0.583656in}}%
\pgfpathlineto{\pgfqpoint{3.190215in}{0.584172in}}%
\pgfpathlineto{\pgfqpoint{3.194163in}{0.594027in}}%
\pgfpathlineto{\pgfqpoint{3.196926in}{0.597448in}}%
\pgfpathlineto{\pgfqpoint{3.197321in}{0.595333in}}%
\pgfpathlineto{\pgfqpoint{3.197321in}{0.595333in}}%
\pgfusepath{stroke}%
\end{pgfscope}%
\begin{pgfscope}%
\pgfpathrectangle{\pgfqpoint{0.608025in}{0.484444in}}{\pgfqpoint{2.712595in}{1.541287in}}%
\pgfusepath{clip}%
\pgfsetbuttcap%
\pgfsetmiterjoin%
\definecolor{currentfill}{rgb}{0.121569,0.466667,0.705882}%
\pgfsetfillcolor{currentfill}%
\pgfsetlinewidth{1.003750pt}%
\definecolor{currentstroke}{rgb}{0.121569,0.466667,0.705882}%
\pgfsetstrokecolor{currentstroke}%
\pgfsetdash{}{0pt}%
\pgfsys@defobject{currentmarker}{\pgfqpoint{-0.020833in}{-0.020833in}}{\pgfqpoint{0.020833in}{0.020833in}}{%
\pgfpathmoveto{\pgfqpoint{-0.020833in}{-0.020833in}}%
\pgfpathlineto{\pgfqpoint{0.020833in}{-0.020833in}}%
\pgfpathlineto{\pgfqpoint{0.020833in}{0.020833in}}%
\pgfpathlineto{\pgfqpoint{-0.020833in}{0.020833in}}%
\pgfpathlineto{\pgfqpoint{-0.020833in}{-0.020833in}}%
\pgfpathclose%
\pgfusepath{stroke,fill}%
}%
\begin{pgfscope}%
\pgfsys@transformshift{0.731720in}{1.955197in}%
\pgfsys@useobject{currentmarker}{}%
\end{pgfscope}%
\begin{pgfscope}%
\pgfsys@transformshift{0.929094in}{1.235693in}%
\pgfsys@useobject{currentmarker}{}%
\end{pgfscope}%
\begin{pgfscope}%
\pgfsys@transformshift{1.126468in}{0.787480in}%
\pgfsys@useobject{currentmarker}{}%
\end{pgfscope}%
\begin{pgfscope}%
\pgfsys@transformshift{1.323843in}{0.674458in}%
\pgfsys@useobject{currentmarker}{}%
\end{pgfscope}%
\begin{pgfscope}%
\pgfsys@transformshift{1.521217in}{0.613073in}%
\pgfsys@useobject{currentmarker}{}%
\end{pgfscope}%
\begin{pgfscope}%
\pgfsys@transformshift{1.718592in}{0.629954in}%
\pgfsys@useobject{currentmarker}{}%
\end{pgfscope}%
\begin{pgfscope}%
\pgfsys@transformshift{1.915966in}{0.582869in}%
\pgfsys@useobject{currentmarker}{}%
\end{pgfscope}%
\begin{pgfscope}%
\pgfsys@transformshift{2.113341in}{0.597703in}%
\pgfsys@useobject{currentmarker}{}%
\end{pgfscope}%
\begin{pgfscope}%
\pgfsys@transformshift{2.310715in}{0.572786in}%
\pgfsys@useobject{currentmarker}{}%
\end{pgfscope}%
\begin{pgfscope}%
\pgfsys@transformshift{2.508089in}{0.577398in}%
\pgfsys@useobject{currentmarker}{}%
\end{pgfscope}%
\begin{pgfscope}%
\pgfsys@transformshift{2.705464in}{0.587538in}%
\pgfsys@useobject{currentmarker}{}%
\end{pgfscope}%
\begin{pgfscope}%
\pgfsys@transformshift{2.902838in}{0.576392in}%
\pgfsys@useobject{currentmarker}{}%
\end{pgfscope}%
\begin{pgfscope}%
\pgfsys@transformshift{3.100213in}{0.573695in}%
\pgfsys@useobject{currentmarker}{}%
\end{pgfscope}%
\end{pgfscope}%
\begin{pgfscope}%
\pgfpathrectangle{\pgfqpoint{0.608025in}{0.484444in}}{\pgfqpoint{2.712595in}{1.541287in}}%
\pgfusepath{clip}%
\pgfsetrectcap%
\pgfsetroundjoin%
\pgfsetlinewidth{1.505625pt}%
\definecolor{currentstroke}{rgb}{1.000000,0.498039,0.054902}%
\pgfsetstrokecolor{currentstroke}%
\pgfsetdash{}{0pt}%
\pgfpathmoveto{\pgfqpoint{0.731325in}{1.955466in}}%
\pgfpathlineto{\pgfqpoint{0.743957in}{1.945714in}}%
\pgfpathlineto{\pgfqpoint{0.745931in}{1.942897in}}%
\pgfpathlineto{\pgfqpoint{0.746325in}{1.943198in}}%
\pgfpathlineto{\pgfqpoint{0.747904in}{1.940387in}}%
\pgfpathlineto{\pgfqpoint{0.756589in}{1.921045in}}%
\pgfpathlineto{\pgfqpoint{0.765668in}{1.903085in}}%
\pgfpathlineto{\pgfqpoint{0.772379in}{1.881289in}}%
\pgfpathlineto{\pgfqpoint{0.787774in}{1.822212in}}%
\pgfpathlineto{\pgfqpoint{0.793300in}{1.799004in}}%
\pgfpathlineto{\pgfqpoint{0.798827in}{1.762161in}}%
\pgfpathlineto{\pgfqpoint{0.826854in}{1.663153in}}%
\pgfpathlineto{\pgfqpoint{0.890803in}{1.344508in}}%
\pgfpathlineto{\pgfqpoint{0.928699in}{1.203963in}}%
\pgfpathlineto{\pgfqpoint{0.952384in}{1.151778in}}%
\pgfpathlineto{\pgfqpoint{0.974095in}{1.110876in}}%
\pgfpathlineto{\pgfqpoint{0.981596in}{1.102494in}}%
\pgfpathlineto{\pgfqpoint{0.989491in}{1.091134in}}%
\pgfpathlineto{\pgfqpoint{0.992649in}{1.080040in}}%
\pgfpathlineto{\pgfqpoint{1.023044in}{0.981074in}}%
\pgfpathlineto{\pgfqpoint{1.024623in}{0.978801in}}%
\pgfpathlineto{\pgfqpoint{1.028571in}{0.971650in}}%
\pgfpathlineto{\pgfqpoint{1.031334in}{0.968099in}}%
\pgfpathlineto{\pgfqpoint{1.035282in}{0.959777in}}%
\pgfpathlineto{\pgfqpoint{1.042387in}{0.943719in}}%
\pgfpathlineto{\pgfqpoint{1.047519in}{0.923919in}}%
\pgfpathlineto{\pgfqpoint{1.053045in}{0.891560in}}%
\pgfpathlineto{\pgfqpoint{1.064888in}{0.830750in}}%
\pgfpathlineto{\pgfqpoint{1.066072in}{0.829059in}}%
\pgfpathlineto{\pgfqpoint{1.068835in}{0.821665in}}%
\pgfpathlineto{\pgfqpoint{1.071598in}{0.821360in}}%
\pgfpathlineto{\pgfqpoint{1.073572in}{0.818321in}}%
\pgfpathlineto{\pgfqpoint{1.074362in}{0.817780in}}%
\pgfpathlineto{\pgfqpoint{1.074756in}{0.819168in}}%
\pgfpathlineto{\pgfqpoint{1.077125in}{0.827759in}}%
\pgfpathlineto{\pgfqpoint{1.082257in}{0.855515in}}%
\pgfpathlineto{\pgfqpoint{1.083441in}{0.854884in}}%
\pgfpathlineto{\pgfqpoint{1.088178in}{0.843965in}}%
\pgfpathlineto{\pgfqpoint{1.092520in}{0.837724in}}%
\pgfpathlineto{\pgfqpoint{1.094494in}{0.835541in}}%
\pgfpathlineto{\pgfqpoint{1.096862in}{0.834821in}}%
\pgfpathlineto{\pgfqpoint{1.100415in}{0.834212in}}%
\pgfpathlineto{\pgfqpoint{1.106336in}{0.826067in}}%
\pgfpathlineto{\pgfqpoint{1.106731in}{0.826521in}}%
\pgfpathlineto{\pgfqpoint{1.110679in}{0.834144in}}%
\pgfpathlineto{\pgfqpoint{1.111073in}{0.833943in}}%
\pgfpathlineto{\pgfqpoint{1.111468in}{0.834109in}}%
\pgfpathlineto{\pgfqpoint{1.111863in}{0.833122in}}%
\pgfpathlineto{\pgfqpoint{1.115810in}{0.829495in}}%
\pgfpathlineto{\pgfqpoint{1.116205in}{0.829947in}}%
\pgfpathlineto{\pgfqpoint{1.116995in}{0.830043in}}%
\pgfpathlineto{\pgfqpoint{1.117389in}{0.828996in}}%
\pgfpathlineto{\pgfqpoint{1.119363in}{0.824811in}}%
\pgfpathlineto{\pgfqpoint{1.123311in}{0.819877in}}%
\pgfpathlineto{\pgfqpoint{1.128047in}{0.823639in}}%
\pgfpathlineto{\pgfqpoint{1.130811in}{0.823844in}}%
\pgfpathlineto{\pgfqpoint{1.132784in}{0.826404in}}%
\pgfpathlineto{\pgfqpoint{1.135153in}{0.827377in}}%
\pgfpathlineto{\pgfqpoint{1.139495in}{0.818767in}}%
\pgfpathlineto{\pgfqpoint{1.145416in}{0.793923in}}%
\pgfpathlineto{\pgfqpoint{1.152917in}{0.739837in}}%
\pgfpathlineto{\pgfqpoint{1.154101in}{0.741849in}}%
\pgfpathlineto{\pgfqpoint{1.155285in}{0.743489in}}%
\pgfpathlineto{\pgfqpoint{1.156075in}{0.746104in}}%
\pgfpathlineto{\pgfqpoint{1.156864in}{0.743669in}}%
\pgfpathlineto{\pgfqpoint{1.159627in}{0.740830in}}%
\pgfpathlineto{\pgfqpoint{1.160022in}{0.741412in}}%
\pgfpathlineto{\pgfqpoint{1.162785in}{0.750396in}}%
\pgfpathlineto{\pgfqpoint{1.164759in}{0.749354in}}%
\pgfpathlineto{\pgfqpoint{1.169101in}{0.740034in}}%
\pgfpathlineto{\pgfqpoint{1.171470in}{0.733691in}}%
\pgfpathlineto{\pgfqpoint{1.179365in}{0.701942in}}%
\pgfpathlineto{\pgfqpoint{1.180944in}{0.700502in}}%
\pgfpathlineto{\pgfqpoint{1.181733in}{0.701516in}}%
\pgfpathlineto{\pgfqpoint{1.183312in}{0.703577in}}%
\pgfpathlineto{\pgfqpoint{1.185681in}{0.707989in}}%
\pgfpathlineto{\pgfqpoint{1.190418in}{0.719874in}}%
\pgfpathlineto{\pgfqpoint{1.191602in}{0.717357in}}%
\pgfpathlineto{\pgfqpoint{1.195155in}{0.705190in}}%
\pgfpathlineto{\pgfqpoint{1.195944in}{0.708254in}}%
\pgfpathlineto{\pgfqpoint{1.197918in}{0.710823in}}%
\pgfpathlineto{\pgfqpoint{1.200287in}{0.715194in}}%
\pgfpathlineto{\pgfqpoint{1.200681in}{0.714223in}}%
\pgfpathlineto{\pgfqpoint{1.205813in}{0.684996in}}%
\pgfpathlineto{\pgfqpoint{1.208576in}{0.669812in}}%
\pgfpathlineto{\pgfqpoint{1.216471in}{0.659216in}}%
\pgfpathlineto{\pgfqpoint{1.219629in}{0.665397in}}%
\pgfpathlineto{\pgfqpoint{1.221998in}{0.662297in}}%
\pgfpathlineto{\pgfqpoint{1.223971in}{0.672001in}}%
\pgfpathlineto{\pgfqpoint{1.224761in}{0.670971in}}%
\pgfpathlineto{\pgfqpoint{1.225550in}{0.669113in}}%
\pgfpathlineto{\pgfqpoint{1.225945in}{0.670988in}}%
\pgfpathlineto{\pgfqpoint{1.228708in}{0.685023in}}%
\pgfpathlineto{\pgfqpoint{1.231077in}{0.680503in}}%
\pgfpathlineto{\pgfqpoint{1.231472in}{0.680731in}}%
\pgfpathlineto{\pgfqpoint{1.235024in}{0.692939in}}%
\pgfpathlineto{\pgfqpoint{1.236209in}{0.689497in}}%
\pgfpathlineto{\pgfqpoint{1.237788in}{0.692803in}}%
\pgfpathlineto{\pgfqpoint{1.241340in}{0.699799in}}%
\pgfpathlineto{\pgfqpoint{1.245683in}{0.712479in}}%
\pgfpathlineto{\pgfqpoint{1.246077in}{0.712439in}}%
\pgfpathlineto{\pgfqpoint{1.248051in}{0.714682in}}%
\pgfpathlineto{\pgfqpoint{1.253972in}{0.706550in}}%
\pgfpathlineto{\pgfqpoint{1.254762in}{0.708813in}}%
\pgfpathlineto{\pgfqpoint{1.257920in}{0.712442in}}%
\pgfpathlineto{\pgfqpoint{1.265025in}{0.705882in}}%
\pgfpathlineto{\pgfqpoint{1.266210in}{0.706048in}}%
\pgfpathlineto{\pgfqpoint{1.266604in}{0.705582in}}%
\pgfpathlineto{\pgfqpoint{1.268578in}{0.705334in}}%
\pgfpathlineto{\pgfqpoint{1.270947in}{0.708699in}}%
\pgfpathlineto{\pgfqpoint{1.271341in}{0.708337in}}%
\pgfpathlineto{\pgfqpoint{1.275684in}{0.699261in}}%
\pgfpathlineto{\pgfqpoint{1.276078in}{0.700271in}}%
\pgfpathlineto{\pgfqpoint{1.276473in}{0.699892in}}%
\pgfpathlineto{\pgfqpoint{1.276868in}{0.700561in}}%
\pgfpathlineto{\pgfqpoint{1.278842in}{0.706224in}}%
\pgfpathlineto{\pgfqpoint{1.279236in}{0.705925in}}%
\pgfpathlineto{\pgfqpoint{1.281210in}{0.705715in}}%
\pgfpathlineto{\pgfqpoint{1.283579in}{0.703214in}}%
\pgfpathlineto{\pgfqpoint{1.286737in}{0.700012in}}%
\pgfpathlineto{\pgfqpoint{1.287921in}{0.702052in}}%
\pgfpathlineto{\pgfqpoint{1.288710in}{0.701057in}}%
\pgfpathlineto{\pgfqpoint{1.291868in}{0.694937in}}%
\pgfpathlineto{\pgfqpoint{1.292658in}{0.693955in}}%
\pgfpathlineto{\pgfqpoint{1.293447in}{0.694885in}}%
\pgfpathlineto{\pgfqpoint{1.294631in}{0.696323in}}%
\pgfpathlineto{\pgfqpoint{1.295026in}{0.695209in}}%
\pgfpathlineto{\pgfqpoint{1.295816in}{0.694809in}}%
\pgfpathlineto{\pgfqpoint{1.297789in}{0.687758in}}%
\pgfpathlineto{\pgfqpoint{1.298184in}{0.687959in}}%
\pgfpathlineto{\pgfqpoint{1.300158in}{0.690438in}}%
\pgfpathlineto{\pgfqpoint{1.300553in}{0.689916in}}%
\pgfpathlineto{\pgfqpoint{1.300947in}{0.688853in}}%
\pgfpathlineto{\pgfqpoint{1.302132in}{0.690052in}}%
\pgfpathlineto{\pgfqpoint{1.303316in}{0.691672in}}%
\pgfpathlineto{\pgfqpoint{1.305684in}{0.697785in}}%
\pgfpathlineto{\pgfqpoint{1.308053in}{0.689424in}}%
\pgfpathlineto{\pgfqpoint{1.309632in}{0.678283in}}%
\pgfpathlineto{\pgfqpoint{1.310421in}{0.679757in}}%
\pgfpathlineto{\pgfqpoint{1.315553in}{0.689808in}}%
\pgfpathlineto{\pgfqpoint{1.318316in}{0.693753in}}%
\pgfpathlineto{\pgfqpoint{1.320685in}{0.701474in}}%
\pgfpathlineto{\pgfqpoint{1.324238in}{0.706396in}}%
\pgfpathlineto{\pgfqpoint{1.325027in}{0.705251in}}%
\pgfpathlineto{\pgfqpoint{1.328580in}{0.701552in}}%
\pgfpathlineto{\pgfqpoint{1.332133in}{0.695111in}}%
\pgfpathlineto{\pgfqpoint{1.336870in}{0.681468in}}%
\pgfpathlineto{\pgfqpoint{1.339633in}{0.664586in}}%
\pgfpathlineto{\pgfqpoint{1.340422in}{0.664546in}}%
\pgfpathlineto{\pgfqpoint{1.344765in}{0.655469in}}%
\pgfpathlineto{\pgfqpoint{1.346738in}{0.657955in}}%
\pgfpathlineto{\pgfqpoint{1.348712in}{0.671576in}}%
\pgfpathlineto{\pgfqpoint{1.349502in}{0.668968in}}%
\pgfpathlineto{\pgfqpoint{1.351081in}{0.664634in}}%
\pgfpathlineto{\pgfqpoint{1.351475in}{0.665287in}}%
\pgfpathlineto{\pgfqpoint{1.354239in}{0.672564in}}%
\pgfpathlineto{\pgfqpoint{1.358186in}{0.673652in}}%
\pgfpathlineto{\pgfqpoint{1.360555in}{0.676653in}}%
\pgfpathlineto{\pgfqpoint{1.362528in}{0.677037in}}%
\pgfpathlineto{\pgfqpoint{1.364502in}{0.676341in}}%
\pgfpathlineto{\pgfqpoint{1.366081in}{0.681874in}}%
\pgfpathlineto{\pgfqpoint{1.366871in}{0.681203in}}%
\pgfpathlineto{\pgfqpoint{1.368055in}{0.681559in}}%
\pgfpathlineto{\pgfqpoint{1.370029in}{0.678775in}}%
\pgfpathlineto{\pgfqpoint{1.371608in}{0.678010in}}%
\pgfpathlineto{\pgfqpoint{1.372792in}{0.676769in}}%
\pgfpathlineto{\pgfqpoint{1.373581in}{0.677279in}}%
\pgfpathlineto{\pgfqpoint{1.375160in}{0.680555in}}%
\pgfpathlineto{\pgfqpoint{1.375950in}{0.678536in}}%
\pgfpathlineto{\pgfqpoint{1.379108in}{0.671366in}}%
\pgfpathlineto{\pgfqpoint{1.379502in}{0.672277in}}%
\pgfpathlineto{\pgfqpoint{1.380292in}{0.670640in}}%
\pgfpathlineto{\pgfqpoint{1.383055in}{0.662353in}}%
\pgfpathlineto{\pgfqpoint{1.383450in}{0.662642in}}%
\pgfpathlineto{\pgfqpoint{1.384634in}{0.664395in}}%
\pgfpathlineto{\pgfqpoint{1.385029in}{0.664080in}}%
\pgfpathlineto{\pgfqpoint{1.387792in}{0.659588in}}%
\pgfpathlineto{\pgfqpoint{1.390950in}{0.651933in}}%
\pgfpathlineto{\pgfqpoint{1.392134in}{0.648346in}}%
\pgfpathlineto{\pgfqpoint{1.392529in}{0.649283in}}%
\pgfpathlineto{\pgfqpoint{1.394503in}{0.650986in}}%
\pgfpathlineto{\pgfqpoint{1.395292in}{0.649898in}}%
\pgfpathlineto{\pgfqpoint{1.396082in}{0.651191in}}%
\pgfpathlineto{\pgfqpoint{1.398056in}{0.647272in}}%
\pgfpathlineto{\pgfqpoint{1.398450in}{0.649511in}}%
\pgfpathlineto{\pgfqpoint{1.398845in}{0.650579in}}%
\pgfpathlineto{\pgfqpoint{1.399240in}{0.648219in}}%
\pgfpathlineto{\pgfqpoint{1.399635in}{0.647441in}}%
\pgfpathlineto{\pgfqpoint{1.400424in}{0.649277in}}%
\pgfpathlineto{\pgfqpoint{1.403187in}{0.653533in}}%
\pgfpathlineto{\pgfqpoint{1.404766in}{0.660898in}}%
\pgfpathlineto{\pgfqpoint{1.405556in}{0.659910in}}%
\pgfpathlineto{\pgfqpoint{1.406740in}{0.657132in}}%
\pgfpathlineto{\pgfqpoint{1.410293in}{0.644828in}}%
\pgfpathlineto{\pgfqpoint{1.411082in}{0.646293in}}%
\pgfpathlineto{\pgfqpoint{1.411477in}{0.646067in}}%
\pgfpathlineto{\pgfqpoint{1.413846in}{0.641674in}}%
\pgfpathlineto{\pgfqpoint{1.417004in}{0.636263in}}%
\pgfpathlineto{\pgfqpoint{1.417398in}{0.637737in}}%
\pgfpathlineto{\pgfqpoint{1.418188in}{0.640187in}}%
\pgfpathlineto{\pgfqpoint{1.418977in}{0.638867in}}%
\pgfpathlineto{\pgfqpoint{1.419372in}{0.637563in}}%
\pgfpathlineto{\pgfqpoint{1.420162in}{0.638423in}}%
\pgfpathlineto{\pgfqpoint{1.421346in}{0.639286in}}%
\pgfpathlineto{\pgfqpoint{1.421741in}{0.639117in}}%
\pgfpathlineto{\pgfqpoint{1.423714in}{0.639243in}}%
\pgfpathlineto{\pgfqpoint{1.426872in}{0.642099in}}%
\pgfpathlineto{\pgfqpoint{1.430030in}{0.624826in}}%
\pgfpathlineto{\pgfqpoint{1.431215in}{0.625793in}}%
\pgfpathlineto{\pgfqpoint{1.431609in}{0.626934in}}%
\pgfpathlineto{\pgfqpoint{1.432399in}{0.624920in}}%
\pgfpathlineto{\pgfqpoint{1.433188in}{0.625178in}}%
\pgfpathlineto{\pgfqpoint{1.433583in}{0.623979in}}%
\pgfpathlineto{\pgfqpoint{1.434767in}{0.621857in}}%
\pgfpathlineto{\pgfqpoint{1.435557in}{0.622127in}}%
\pgfpathlineto{\pgfqpoint{1.436741in}{0.622775in}}%
\pgfpathlineto{\pgfqpoint{1.437136in}{0.621792in}}%
\pgfpathlineto{\pgfqpoint{1.439110in}{0.618289in}}%
\pgfpathlineto{\pgfqpoint{1.439504in}{0.618843in}}%
\pgfpathlineto{\pgfqpoint{1.441478in}{0.614239in}}%
\pgfpathlineto{\pgfqpoint{1.442268in}{0.616520in}}%
\pgfpathlineto{\pgfqpoint{1.448978in}{0.622998in}}%
\pgfpathlineto{\pgfqpoint{1.450952in}{0.621903in}}%
\pgfpathlineto{\pgfqpoint{1.455294in}{0.621625in}}%
\pgfpathlineto{\pgfqpoint{1.456479in}{0.619376in}}%
\pgfpathlineto{\pgfqpoint{1.458847in}{0.624526in}}%
\pgfpathlineto{\pgfqpoint{1.459637in}{0.623146in}}%
\pgfpathlineto{\pgfqpoint{1.462005in}{0.610145in}}%
\pgfpathlineto{\pgfqpoint{1.462400in}{0.610823in}}%
\pgfpathlineto{\pgfqpoint{1.462794in}{0.613509in}}%
\pgfpathlineto{\pgfqpoint{1.463979in}{0.611385in}}%
\pgfpathlineto{\pgfqpoint{1.465952in}{0.607114in}}%
\pgfpathlineto{\pgfqpoint{1.466347in}{0.607648in}}%
\pgfpathlineto{\pgfqpoint{1.471084in}{0.621285in}}%
\pgfpathlineto{\pgfqpoint{1.472268in}{0.622110in}}%
\pgfpathlineto{\pgfqpoint{1.473453in}{0.615267in}}%
\pgfpathlineto{\pgfqpoint{1.473847in}{0.615881in}}%
\pgfpathlineto{\pgfqpoint{1.475821in}{0.622488in}}%
\pgfpathlineto{\pgfqpoint{1.476216in}{0.622252in}}%
\pgfpathlineto{\pgfqpoint{1.477005in}{0.623409in}}%
\pgfpathlineto{\pgfqpoint{1.477795in}{0.626888in}}%
\pgfpathlineto{\pgfqpoint{1.478584in}{0.624999in}}%
\pgfpathlineto{\pgfqpoint{1.479374in}{0.626481in}}%
\pgfpathlineto{\pgfqpoint{1.480163in}{0.624076in}}%
\pgfpathlineto{\pgfqpoint{1.481348in}{0.625488in}}%
\pgfpathlineto{\pgfqpoint{1.481742in}{0.624066in}}%
\pgfpathlineto{\pgfqpoint{1.483716in}{0.614594in}}%
\pgfpathlineto{\pgfqpoint{1.484900in}{0.616969in}}%
\pgfpathlineto{\pgfqpoint{1.487664in}{0.620994in}}%
\pgfpathlineto{\pgfqpoint{1.492401in}{0.620822in}}%
\pgfpathlineto{\pgfqpoint{1.494769in}{0.626031in}}%
\pgfpathlineto{\pgfqpoint{1.496348in}{0.624591in}}%
\pgfpathlineto{\pgfqpoint{1.498717in}{0.621143in}}%
\pgfpathlineto{\pgfqpoint{1.501085in}{0.620183in}}%
\pgfpathlineto{\pgfqpoint{1.501480in}{0.621832in}}%
\pgfpathlineto{\pgfqpoint{1.502269in}{0.620136in}}%
\pgfpathlineto{\pgfqpoint{1.503454in}{0.617454in}}%
\pgfpathlineto{\pgfqpoint{1.504243in}{0.618404in}}%
\pgfpathlineto{\pgfqpoint{1.505033in}{0.619354in}}%
\pgfpathlineto{\pgfqpoint{1.505427in}{0.618563in}}%
\pgfpathlineto{\pgfqpoint{1.506612in}{0.616595in}}%
\pgfpathlineto{\pgfqpoint{1.507401in}{0.619781in}}%
\pgfpathlineto{\pgfqpoint{1.508191in}{0.617244in}}%
\pgfpathlineto{\pgfqpoint{1.509770in}{0.611678in}}%
\pgfpathlineto{\pgfqpoint{1.510559in}{0.612322in}}%
\pgfpathlineto{\pgfqpoint{1.512138in}{0.616214in}}%
\pgfpathlineto{\pgfqpoint{1.515691in}{0.619596in}}%
\pgfpathlineto{\pgfqpoint{1.516086in}{0.619348in}}%
\pgfpathlineto{\pgfqpoint{1.516480in}{0.620788in}}%
\pgfpathlineto{\pgfqpoint{1.518454in}{0.625063in}}%
\pgfpathlineto{\pgfqpoint{1.519638in}{0.621949in}}%
\pgfpathlineto{\pgfqpoint{1.520823in}{0.623279in}}%
\pgfpathlineto{\pgfqpoint{1.521217in}{0.623703in}}%
\pgfpathlineto{\pgfqpoint{1.521612in}{0.622100in}}%
\pgfpathlineto{\pgfqpoint{1.522796in}{0.617735in}}%
\pgfpathlineto{\pgfqpoint{1.523586in}{0.619038in}}%
\pgfpathlineto{\pgfqpoint{1.525165in}{0.616823in}}%
\pgfpathlineto{\pgfqpoint{1.525560in}{0.617434in}}%
\pgfpathlineto{\pgfqpoint{1.526349in}{0.620528in}}%
\pgfpathlineto{\pgfqpoint{1.527139in}{0.617892in}}%
\pgfpathlineto{\pgfqpoint{1.527928in}{0.618097in}}%
\pgfpathlineto{\pgfqpoint{1.529112in}{0.621826in}}%
\pgfpathlineto{\pgfqpoint{1.529507in}{0.619700in}}%
\pgfpathlineto{\pgfqpoint{1.530691in}{0.611243in}}%
\pgfpathlineto{\pgfqpoint{1.531481in}{0.611494in}}%
\pgfpathlineto{\pgfqpoint{1.532270in}{0.611838in}}%
\pgfpathlineto{\pgfqpoint{1.533455in}{0.616273in}}%
\pgfpathlineto{\pgfqpoint{1.534244in}{0.614659in}}%
\pgfpathlineto{\pgfqpoint{1.535034in}{0.613785in}}%
\pgfpathlineto{\pgfqpoint{1.535428in}{0.615236in}}%
\pgfpathlineto{\pgfqpoint{1.538981in}{0.621500in}}%
\pgfpathlineto{\pgfqpoint{1.540955in}{0.615473in}}%
\pgfpathlineto{\pgfqpoint{1.541350in}{0.615601in}}%
\pgfpathlineto{\pgfqpoint{1.544507in}{0.611778in}}%
\pgfpathlineto{\pgfqpoint{1.546481in}{0.612495in}}%
\pgfpathlineto{\pgfqpoint{1.549244in}{0.621194in}}%
\pgfpathlineto{\pgfqpoint{1.551218in}{0.619974in}}%
\pgfpathlineto{\pgfqpoint{1.551613in}{0.621933in}}%
\pgfpathlineto{\pgfqpoint{1.552797in}{0.620952in}}%
\pgfpathlineto{\pgfqpoint{1.555166in}{0.614356in}}%
\pgfpathlineto{\pgfqpoint{1.557139in}{0.620930in}}%
\pgfpathlineto{\pgfqpoint{1.557534in}{0.620457in}}%
\pgfpathlineto{\pgfqpoint{1.559113in}{0.615975in}}%
\pgfpathlineto{\pgfqpoint{1.559508in}{0.616412in}}%
\pgfpathlineto{\pgfqpoint{1.559903in}{0.617378in}}%
\pgfpathlineto{\pgfqpoint{1.560297in}{0.616150in}}%
\pgfpathlineto{\pgfqpoint{1.561482in}{0.614894in}}%
\pgfpathlineto{\pgfqpoint{1.561876in}{0.615206in}}%
\pgfpathlineto{\pgfqpoint{1.565429in}{0.618733in}}%
\pgfpathlineto{\pgfqpoint{1.565824in}{0.618247in}}%
\pgfpathlineto{\pgfqpoint{1.566613in}{0.618550in}}%
\pgfpathlineto{\pgfqpoint{1.570956in}{0.636427in}}%
\pgfpathlineto{\pgfqpoint{1.572929in}{0.633854in}}%
\pgfpathlineto{\pgfqpoint{1.573719in}{0.635237in}}%
\pgfpathlineto{\pgfqpoint{1.578061in}{0.641405in}}%
\pgfpathlineto{\pgfqpoint{1.578456in}{0.640469in}}%
\pgfpathlineto{\pgfqpoint{1.581614in}{0.632885in}}%
\pgfpathlineto{\pgfqpoint{1.582403in}{0.634673in}}%
\pgfpathlineto{\pgfqpoint{1.583193in}{0.635702in}}%
\pgfpathlineto{\pgfqpoint{1.583588in}{0.634129in}}%
\pgfpathlineto{\pgfqpoint{1.584377in}{0.632388in}}%
\pgfpathlineto{\pgfqpoint{1.585167in}{0.634229in}}%
\pgfpathlineto{\pgfqpoint{1.585956in}{0.635797in}}%
\pgfpathlineto{\pgfqpoint{1.586351in}{0.634977in}}%
\pgfpathlineto{\pgfqpoint{1.591088in}{0.618244in}}%
\pgfpathlineto{\pgfqpoint{1.591877in}{0.619627in}}%
\pgfpathlineto{\pgfqpoint{1.593062in}{0.623986in}}%
\pgfpathlineto{\pgfqpoint{1.593851in}{0.622108in}}%
\pgfpathlineto{\pgfqpoint{1.597799in}{0.604904in}}%
\pgfpathlineto{\pgfqpoint{1.599378in}{0.600265in}}%
\pgfpathlineto{\pgfqpoint{1.599772in}{0.602564in}}%
\pgfpathlineto{\pgfqpoint{1.601351in}{0.607423in}}%
\pgfpathlineto{\pgfqpoint{1.601746in}{0.605825in}}%
\pgfpathlineto{\pgfqpoint{1.602536in}{0.604882in}}%
\pgfpathlineto{\pgfqpoint{1.602930in}{0.605968in}}%
\pgfpathlineto{\pgfqpoint{1.603720in}{0.607391in}}%
\pgfpathlineto{\pgfqpoint{1.604115in}{0.605380in}}%
\pgfpathlineto{\pgfqpoint{1.604904in}{0.603824in}}%
\pgfpathlineto{\pgfqpoint{1.605299in}{0.604515in}}%
\pgfpathlineto{\pgfqpoint{1.607667in}{0.611137in}}%
\pgfpathlineto{\pgfqpoint{1.609246in}{0.607031in}}%
\pgfpathlineto{\pgfqpoint{1.610036in}{0.611075in}}%
\pgfpathlineto{\pgfqpoint{1.611220in}{0.610415in}}%
\pgfpathlineto{\pgfqpoint{1.614773in}{0.588846in}}%
\pgfpathlineto{\pgfqpoint{1.615168in}{0.588940in}}%
\pgfpathlineto{\pgfqpoint{1.617931in}{0.595574in}}%
\pgfpathlineto{\pgfqpoint{1.618326in}{0.595668in}}%
\pgfpathlineto{\pgfqpoint{1.619115in}{0.592246in}}%
\pgfpathlineto{\pgfqpoint{1.620299in}{0.593272in}}%
\pgfpathlineto{\pgfqpoint{1.622668in}{0.606973in}}%
\pgfpathlineto{\pgfqpoint{1.623457in}{0.603766in}}%
\pgfpathlineto{\pgfqpoint{1.625036in}{0.599495in}}%
\pgfpathlineto{\pgfqpoint{1.625431in}{0.600092in}}%
\pgfpathlineto{\pgfqpoint{1.626221in}{0.598195in}}%
\pgfpathlineto{\pgfqpoint{1.627010in}{0.600004in}}%
\pgfpathlineto{\pgfqpoint{1.628194in}{0.602548in}}%
\pgfpathlineto{\pgfqpoint{1.628984in}{0.601479in}}%
\pgfpathlineto{\pgfqpoint{1.630563in}{0.610318in}}%
\pgfpathlineto{\pgfqpoint{1.632142in}{0.614473in}}%
\pgfpathlineto{\pgfqpoint{1.632536in}{0.614163in}}%
\pgfpathlineto{\pgfqpoint{1.634510in}{0.604336in}}%
\pgfpathlineto{\pgfqpoint{1.635300in}{0.605978in}}%
\pgfpathlineto{\pgfqpoint{1.636484in}{0.605670in}}%
\pgfpathlineto{\pgfqpoint{1.639642in}{0.598972in}}%
\pgfpathlineto{\pgfqpoint{1.642405in}{0.598276in}}%
\pgfpathlineto{\pgfqpoint{1.643195in}{0.600555in}}%
\pgfpathlineto{\pgfqpoint{1.643984in}{0.598735in}}%
\pgfpathlineto{\pgfqpoint{1.644379in}{0.598496in}}%
\pgfpathlineto{\pgfqpoint{1.644774in}{0.599244in}}%
\pgfpathlineto{\pgfqpoint{1.645958in}{0.600438in}}%
\pgfpathlineto{\pgfqpoint{1.647142in}{0.596879in}}%
\pgfpathlineto{\pgfqpoint{1.647537in}{0.598340in}}%
\pgfpathlineto{\pgfqpoint{1.650300in}{0.607842in}}%
\pgfpathlineto{\pgfqpoint{1.650695in}{0.607609in}}%
\pgfpathlineto{\pgfqpoint{1.651879in}{0.606227in}}%
\pgfpathlineto{\pgfqpoint{1.652274in}{0.607032in}}%
\pgfpathlineto{\pgfqpoint{1.652669in}{0.607512in}}%
\pgfpathlineto{\pgfqpoint{1.657406in}{0.584620in}}%
\pgfpathlineto{\pgfqpoint{1.657800in}{0.584866in}}%
\pgfpathlineto{\pgfqpoint{1.662143in}{0.604721in}}%
\pgfpathlineto{\pgfqpoint{1.663722in}{0.603532in}}%
\pgfpathlineto{\pgfqpoint{1.666485in}{0.600181in}}%
\pgfpathlineto{\pgfqpoint{1.668459in}{0.609543in}}%
\pgfpathlineto{\pgfqpoint{1.669643in}{0.607815in}}%
\pgfpathlineto{\pgfqpoint{1.672011in}{0.609125in}}%
\pgfpathlineto{\pgfqpoint{1.673196in}{0.611359in}}%
\pgfpathlineto{\pgfqpoint{1.673590in}{0.611085in}}%
\pgfpathlineto{\pgfqpoint{1.677143in}{0.598166in}}%
\pgfpathlineto{\pgfqpoint{1.679512in}{0.610220in}}%
\pgfpathlineto{\pgfqpoint{1.681485in}{0.607851in}}%
\pgfpathlineto{\pgfqpoint{1.683064in}{0.607979in}}%
\pgfpathlineto{\pgfqpoint{1.685433in}{0.615002in}}%
\pgfpathlineto{\pgfqpoint{1.686222in}{0.614174in}}%
\pgfpathlineto{\pgfqpoint{1.687012in}{0.613124in}}%
\pgfpathlineto{\pgfqpoint{1.687407in}{0.614312in}}%
\pgfpathlineto{\pgfqpoint{1.688986in}{0.617296in}}%
\pgfpathlineto{\pgfqpoint{1.689380in}{0.615955in}}%
\pgfpathlineto{\pgfqpoint{1.690170in}{0.613222in}}%
\pgfpathlineto{\pgfqpoint{1.690565in}{0.616155in}}%
\pgfpathlineto{\pgfqpoint{1.691354in}{0.620258in}}%
\pgfpathlineto{\pgfqpoint{1.692144in}{0.617841in}}%
\pgfpathlineto{\pgfqpoint{1.694512in}{0.607608in}}%
\pgfpathlineto{\pgfqpoint{1.694907in}{0.608214in}}%
\pgfpathlineto{\pgfqpoint{1.698460in}{0.618438in}}%
\pgfpathlineto{\pgfqpoint{1.698854in}{0.617844in}}%
\pgfpathlineto{\pgfqpoint{1.700433in}{0.623014in}}%
\pgfpathlineto{\pgfqpoint{1.700828in}{0.622566in}}%
\pgfpathlineto{\pgfqpoint{1.701223in}{0.621335in}}%
\pgfpathlineto{\pgfqpoint{1.702012in}{0.623543in}}%
\pgfpathlineto{\pgfqpoint{1.704776in}{0.626441in}}%
\pgfpathlineto{\pgfqpoint{1.705170in}{0.625801in}}%
\pgfpathlineto{\pgfqpoint{1.705565in}{0.630135in}}%
\pgfpathlineto{\pgfqpoint{1.706355in}{0.626257in}}%
\pgfpathlineto{\pgfqpoint{1.710302in}{0.620369in}}%
\pgfpathlineto{\pgfqpoint{1.713065in}{0.624300in}}%
\pgfpathlineto{\pgfqpoint{1.714249in}{0.622084in}}%
\pgfpathlineto{\pgfqpoint{1.717802in}{0.637714in}}%
\pgfpathlineto{\pgfqpoint{1.719776in}{0.639750in}}%
\pgfpathlineto{\pgfqpoint{1.721750in}{0.635803in}}%
\pgfpathlineto{\pgfqpoint{1.722144in}{0.636185in}}%
\pgfpathlineto{\pgfqpoint{1.725302in}{0.641603in}}%
\pgfpathlineto{\pgfqpoint{1.725697in}{0.641361in}}%
\pgfpathlineto{\pgfqpoint{1.726487in}{0.641022in}}%
\pgfpathlineto{\pgfqpoint{1.727671in}{0.646758in}}%
\pgfpathlineto{\pgfqpoint{1.728855in}{0.645875in}}%
\pgfpathlineto{\pgfqpoint{1.731224in}{0.635234in}}%
\pgfpathlineto{\pgfqpoint{1.731618in}{0.636442in}}%
\pgfpathlineto{\pgfqpoint{1.732013in}{0.635744in}}%
\pgfpathlineto{\pgfqpoint{1.732803in}{0.636908in}}%
\pgfpathlineto{\pgfqpoint{1.733197in}{0.637136in}}%
\pgfpathlineto{\pgfqpoint{1.733592in}{0.635854in}}%
\pgfpathlineto{\pgfqpoint{1.734776in}{0.633621in}}%
\pgfpathlineto{\pgfqpoint{1.735171in}{0.634671in}}%
\pgfpathlineto{\pgfqpoint{1.735566in}{0.636010in}}%
\pgfpathlineto{\pgfqpoint{1.735961in}{0.634099in}}%
\pgfpathlineto{\pgfqpoint{1.737540in}{0.631136in}}%
\pgfpathlineto{\pgfqpoint{1.738724in}{0.633026in}}%
\pgfpathlineto{\pgfqpoint{1.739119in}{0.632464in}}%
\pgfpathlineto{\pgfqpoint{1.739908in}{0.629539in}}%
\pgfpathlineto{\pgfqpoint{1.740698in}{0.631142in}}%
\pgfpathlineto{\pgfqpoint{1.741882in}{0.629299in}}%
\pgfpathlineto{\pgfqpoint{1.744250in}{0.622691in}}%
\pgfpathlineto{\pgfqpoint{1.744645in}{0.623476in}}%
\pgfpathlineto{\pgfqpoint{1.746619in}{0.627550in}}%
\pgfpathlineto{\pgfqpoint{1.747408in}{0.625107in}}%
\pgfpathlineto{\pgfqpoint{1.748987in}{0.621253in}}%
\pgfpathlineto{\pgfqpoint{1.753724in}{0.607969in}}%
\pgfpathlineto{\pgfqpoint{1.754909in}{0.609067in}}%
\pgfpathlineto{\pgfqpoint{1.755303in}{0.607441in}}%
\pgfpathlineto{\pgfqpoint{1.756882in}{0.604788in}}%
\pgfpathlineto{\pgfqpoint{1.759646in}{0.599803in}}%
\pgfpathlineto{\pgfqpoint{1.760040in}{0.600281in}}%
\pgfpathlineto{\pgfqpoint{1.765962in}{0.617788in}}%
\pgfpathlineto{\pgfqpoint{1.766356in}{0.618146in}}%
\pgfpathlineto{\pgfqpoint{1.766751in}{0.617016in}}%
\pgfpathlineto{\pgfqpoint{1.773067in}{0.610185in}}%
\pgfpathlineto{\pgfqpoint{1.773462in}{0.611223in}}%
\pgfpathlineto{\pgfqpoint{1.774646in}{0.615855in}}%
\pgfpathlineto{\pgfqpoint{1.775041in}{0.614417in}}%
\pgfpathlineto{\pgfqpoint{1.778199in}{0.609827in}}%
\pgfpathlineto{\pgfqpoint{1.778594in}{0.610552in}}%
\pgfpathlineto{\pgfqpoint{1.780567in}{0.612804in}}%
\pgfpathlineto{\pgfqpoint{1.780962in}{0.611869in}}%
\pgfpathlineto{\pgfqpoint{1.782936in}{0.608733in}}%
\pgfpathlineto{\pgfqpoint{1.784120in}{0.609897in}}%
\pgfpathlineto{\pgfqpoint{1.784910in}{0.611064in}}%
\pgfpathlineto{\pgfqpoint{1.785699in}{0.610319in}}%
\pgfpathlineto{\pgfqpoint{1.787278in}{0.607950in}}%
\pgfpathlineto{\pgfqpoint{1.790041in}{0.619361in}}%
\pgfpathlineto{\pgfqpoint{1.792410in}{0.610292in}}%
\pgfpathlineto{\pgfqpoint{1.793989in}{0.605445in}}%
\pgfpathlineto{\pgfqpoint{1.794778in}{0.605768in}}%
\pgfpathlineto{\pgfqpoint{1.795962in}{0.603402in}}%
\pgfpathlineto{\pgfqpoint{1.798726in}{0.590510in}}%
\pgfpathlineto{\pgfqpoint{1.799120in}{0.591634in}}%
\pgfpathlineto{\pgfqpoint{1.800305in}{0.589842in}}%
\pgfpathlineto{\pgfqpoint{1.800699in}{0.591155in}}%
\pgfpathlineto{\pgfqpoint{1.801884in}{0.594964in}}%
\pgfpathlineto{\pgfqpoint{1.802673in}{0.592889in}}%
\pgfpathlineto{\pgfqpoint{1.803068in}{0.591290in}}%
\pgfpathlineto{\pgfqpoint{1.803857in}{0.593794in}}%
\pgfpathlineto{\pgfqpoint{1.807015in}{0.598380in}}%
\pgfpathlineto{\pgfqpoint{1.807805in}{0.596139in}}%
\pgfpathlineto{\pgfqpoint{1.808594in}{0.597179in}}%
\pgfpathlineto{\pgfqpoint{1.810568in}{0.602760in}}%
\pgfpathlineto{\pgfqpoint{1.814121in}{0.593671in}}%
\pgfpathlineto{\pgfqpoint{1.814516in}{0.593803in}}%
\pgfpathlineto{\pgfqpoint{1.816095in}{0.588357in}}%
\pgfpathlineto{\pgfqpoint{1.816489in}{0.589529in}}%
\pgfpathlineto{\pgfqpoint{1.820437in}{0.606891in}}%
\pgfpathlineto{\pgfqpoint{1.820832in}{0.606583in}}%
\pgfpathlineto{\pgfqpoint{1.823200in}{0.601651in}}%
\pgfpathlineto{\pgfqpoint{1.823595in}{0.603389in}}%
\pgfpathlineto{\pgfqpoint{1.823990in}{0.605278in}}%
\pgfpathlineto{\pgfqpoint{1.824779in}{0.602448in}}%
\pgfpathlineto{\pgfqpoint{1.826753in}{0.593531in}}%
\pgfpathlineto{\pgfqpoint{1.828332in}{0.595802in}}%
\pgfpathlineto{\pgfqpoint{1.831490in}{0.591147in}}%
\pgfpathlineto{\pgfqpoint{1.831885in}{0.592249in}}%
\pgfpathlineto{\pgfqpoint{1.833858in}{0.599327in}}%
\pgfpathlineto{\pgfqpoint{1.834253in}{0.598414in}}%
\pgfpathlineto{\pgfqpoint{1.837016in}{0.604926in}}%
\pgfpathlineto{\pgfqpoint{1.838201in}{0.610564in}}%
\pgfpathlineto{\pgfqpoint{1.838990in}{0.607497in}}%
\pgfpathlineto{\pgfqpoint{1.840174in}{0.608102in}}%
\pgfpathlineto{\pgfqpoint{1.840569in}{0.608523in}}%
\pgfpathlineto{\pgfqpoint{1.840964in}{0.607700in}}%
\pgfpathlineto{\pgfqpoint{1.845701in}{0.595091in}}%
\pgfpathlineto{\pgfqpoint{1.846096in}{0.595758in}}%
\pgfpathlineto{\pgfqpoint{1.846490in}{0.594016in}}%
\pgfpathlineto{\pgfqpoint{1.850438in}{0.580360in}}%
\pgfpathlineto{\pgfqpoint{1.850833in}{0.582682in}}%
\pgfpathlineto{\pgfqpoint{1.852017in}{0.580604in}}%
\pgfpathlineto{\pgfqpoint{1.852412in}{0.580738in}}%
\pgfpathlineto{\pgfqpoint{1.856754in}{0.600366in}}%
\pgfpathlineto{\pgfqpoint{1.862280in}{0.587261in}}%
\pgfpathlineto{\pgfqpoint{1.862675in}{0.588906in}}%
\pgfpathlineto{\pgfqpoint{1.865044in}{0.594407in}}%
\pgfpathlineto{\pgfqpoint{1.865438in}{0.594275in}}%
\pgfpathlineto{\pgfqpoint{1.867807in}{0.599453in}}%
\pgfpathlineto{\pgfqpoint{1.868202in}{0.598780in}}%
\pgfpathlineto{\pgfqpoint{1.870175in}{0.590126in}}%
\pgfpathlineto{\pgfqpoint{1.870570in}{0.590913in}}%
\pgfpathlineto{\pgfqpoint{1.870965in}{0.592062in}}%
\pgfpathlineto{\pgfqpoint{1.871754in}{0.590327in}}%
\pgfpathlineto{\pgfqpoint{1.872149in}{0.590820in}}%
\pgfpathlineto{\pgfqpoint{1.873728in}{0.582903in}}%
\pgfpathlineto{\pgfqpoint{1.874123in}{0.584529in}}%
\pgfpathlineto{\pgfqpoint{1.875702in}{0.595108in}}%
\pgfpathlineto{\pgfqpoint{1.876097in}{0.592257in}}%
\pgfpathlineto{\pgfqpoint{1.877281in}{0.584649in}}%
\pgfpathlineto{\pgfqpoint{1.878070in}{0.585853in}}%
\pgfpathlineto{\pgfqpoint{1.881228in}{0.595856in}}%
\pgfpathlineto{\pgfqpoint{1.882018in}{0.594810in}}%
\pgfpathlineto{\pgfqpoint{1.883597in}{0.594285in}}%
\pgfpathlineto{\pgfqpoint{1.883991in}{0.595067in}}%
\pgfpathlineto{\pgfqpoint{1.885570in}{0.596708in}}%
\pgfpathlineto{\pgfqpoint{1.886360in}{0.594945in}}%
\pgfpathlineto{\pgfqpoint{1.886755in}{0.595762in}}%
\pgfpathlineto{\pgfqpoint{1.890702in}{0.602208in}}%
\pgfpathlineto{\pgfqpoint{1.891492in}{0.601402in}}%
\pgfpathlineto{\pgfqpoint{1.895044in}{0.587364in}}%
\pgfpathlineto{\pgfqpoint{1.896623in}{0.589570in}}%
\pgfpathlineto{\pgfqpoint{1.897018in}{0.588483in}}%
\pgfpathlineto{\pgfqpoint{1.897413in}{0.588467in}}%
\pgfpathlineto{\pgfqpoint{1.901360in}{0.599138in}}%
\pgfpathlineto{\pgfqpoint{1.904913in}{0.589737in}}%
\pgfpathlineto{\pgfqpoint{1.905703in}{0.591069in}}%
\pgfpathlineto{\pgfqpoint{1.906492in}{0.594327in}}%
\pgfpathlineto{\pgfqpoint{1.906887in}{0.593574in}}%
\pgfpathlineto{\pgfqpoint{1.909650in}{0.585374in}}%
\pgfpathlineto{\pgfqpoint{1.910440in}{0.586076in}}%
\pgfpathlineto{\pgfqpoint{1.911624in}{0.591521in}}%
\pgfpathlineto{\pgfqpoint{1.912808in}{0.603335in}}%
\pgfpathlineto{\pgfqpoint{1.913598in}{0.600522in}}%
\pgfpathlineto{\pgfqpoint{1.914387in}{0.596309in}}%
\pgfpathlineto{\pgfqpoint{1.915177in}{0.599421in}}%
\pgfpathlineto{\pgfqpoint{1.915571in}{0.600685in}}%
\pgfpathlineto{\pgfqpoint{1.916361in}{0.598748in}}%
\pgfpathlineto{\pgfqpoint{1.919914in}{0.594318in}}%
\pgfpathlineto{\pgfqpoint{1.920308in}{0.594831in}}%
\pgfpathlineto{\pgfqpoint{1.922282in}{0.596095in}}%
\pgfpathlineto{\pgfqpoint{1.923466in}{0.591980in}}%
\pgfpathlineto{\pgfqpoint{1.924256in}{0.594346in}}%
\pgfpathlineto{\pgfqpoint{1.924651in}{0.594531in}}%
\pgfpathlineto{\pgfqpoint{1.927809in}{0.584520in}}%
\pgfpathlineto{\pgfqpoint{1.928203in}{0.585135in}}%
\pgfpathlineto{\pgfqpoint{1.930572in}{0.594575in}}%
\pgfpathlineto{\pgfqpoint{1.930967in}{0.594032in}}%
\pgfpathlineto{\pgfqpoint{1.931756in}{0.594717in}}%
\pgfpathlineto{\pgfqpoint{1.934519in}{0.601324in}}%
\pgfpathlineto{\pgfqpoint{1.935704in}{0.602020in}}%
\pgfpathlineto{\pgfqpoint{1.941230in}{0.619580in}}%
\pgfpathlineto{\pgfqpoint{1.942020in}{0.616544in}}%
\pgfpathlineto{\pgfqpoint{1.942809in}{0.619414in}}%
\pgfpathlineto{\pgfqpoint{1.943599in}{0.617435in}}%
\pgfpathlineto{\pgfqpoint{1.946362in}{0.604130in}}%
\pgfpathlineto{\pgfqpoint{1.946757in}{0.605181in}}%
\pgfpathlineto{\pgfqpoint{1.947546in}{0.604261in}}%
\pgfpathlineto{\pgfqpoint{1.953467in}{0.585996in}}%
\pgfpathlineto{\pgfqpoint{1.954257in}{0.586557in}}%
\pgfpathlineto{\pgfqpoint{1.956625in}{0.595255in}}%
\pgfpathlineto{\pgfqpoint{1.957415in}{0.593031in}}%
\pgfpathlineto{\pgfqpoint{1.958204in}{0.592992in}}%
\pgfpathlineto{\pgfqpoint{1.958599in}{0.591958in}}%
\pgfpathlineto{\pgfqpoint{1.960178in}{0.583475in}}%
\pgfpathlineto{\pgfqpoint{1.960573in}{0.585218in}}%
\pgfpathlineto{\pgfqpoint{1.962152in}{0.587842in}}%
\pgfpathlineto{\pgfqpoint{1.962546in}{0.586966in}}%
\pgfpathlineto{\pgfqpoint{1.964125in}{0.583074in}}%
\pgfpathlineto{\pgfqpoint{1.964520in}{0.585198in}}%
\pgfpathlineto{\pgfqpoint{1.965310in}{0.587465in}}%
\pgfpathlineto{\pgfqpoint{1.966099in}{0.586685in}}%
\pgfpathlineto{\pgfqpoint{1.966494in}{0.584738in}}%
\pgfpathlineto{\pgfqpoint{1.967283in}{0.587464in}}%
\pgfpathlineto{\pgfqpoint{1.968468in}{0.591798in}}%
\pgfpathlineto{\pgfqpoint{1.968862in}{0.589606in}}%
\pgfpathlineto{\pgfqpoint{1.971626in}{0.582275in}}%
\pgfpathlineto{\pgfqpoint{1.977152in}{0.612774in}}%
\pgfpathlineto{\pgfqpoint{1.977547in}{0.611895in}}%
\pgfpathlineto{\pgfqpoint{1.978336in}{0.610617in}}%
\pgfpathlineto{\pgfqpoint{1.980310in}{0.607775in}}%
\pgfpathlineto{\pgfqpoint{1.982284in}{0.614825in}}%
\pgfpathlineto{\pgfqpoint{1.982679in}{0.614218in}}%
\pgfpathlineto{\pgfqpoint{1.985047in}{0.610156in}}%
\pgfpathlineto{\pgfqpoint{1.985442in}{0.611070in}}%
\pgfpathlineto{\pgfqpoint{1.986231in}{0.611466in}}%
\pgfpathlineto{\pgfqpoint{1.988600in}{0.623227in}}%
\pgfpathlineto{\pgfqpoint{1.989784in}{0.619360in}}%
\pgfpathlineto{\pgfqpoint{1.991363in}{0.612171in}}%
\pgfpathlineto{\pgfqpoint{1.991758in}{0.614470in}}%
\pgfpathlineto{\pgfqpoint{1.993337in}{0.622207in}}%
\pgfpathlineto{\pgfqpoint{1.994126in}{0.621493in}}%
\pgfpathlineto{\pgfqpoint{1.994521in}{0.621065in}}%
\pgfpathlineto{\pgfqpoint{1.994916in}{0.622467in}}%
\pgfpathlineto{\pgfqpoint{1.995311in}{0.623546in}}%
\pgfpathlineto{\pgfqpoint{1.995705in}{0.622018in}}%
\pgfpathlineto{\pgfqpoint{1.997679in}{0.616549in}}%
\pgfpathlineto{\pgfqpoint{1.998863in}{0.618245in}}%
\pgfpathlineto{\pgfqpoint{2.000837in}{0.615248in}}%
\pgfpathlineto{\pgfqpoint{2.002021in}{0.617141in}}%
\pgfpathlineto{\pgfqpoint{2.002416in}{0.616131in}}%
\pgfpathlineto{\pgfqpoint{2.004785in}{0.613075in}}%
\pgfpathlineto{\pgfqpoint{2.005179in}{0.614007in}}%
\pgfpathlineto{\pgfqpoint{2.005574in}{0.614456in}}%
\pgfpathlineto{\pgfqpoint{2.007943in}{0.622512in}}%
\pgfpathlineto{\pgfqpoint{2.009916in}{0.624823in}}%
\pgfpathlineto{\pgfqpoint{2.011495in}{0.625791in}}%
\pgfpathlineto{\pgfqpoint{2.015048in}{0.615078in}}%
\pgfpathlineto{\pgfqpoint{2.017811in}{0.607043in}}%
\pgfpathlineto{\pgfqpoint{2.020180in}{0.605320in}}%
\pgfpathlineto{\pgfqpoint{2.020575in}{0.605379in}}%
\pgfpathlineto{\pgfqpoint{2.022548in}{0.601708in}}%
\pgfpathlineto{\pgfqpoint{2.024522in}{0.609137in}}%
\pgfpathlineto{\pgfqpoint{2.024917in}{0.607220in}}%
\pgfpathlineto{\pgfqpoint{2.029654in}{0.584730in}}%
\pgfpathlineto{\pgfqpoint{2.030443in}{0.587693in}}%
\pgfpathlineto{\pgfqpoint{2.033207in}{0.604310in}}%
\pgfpathlineto{\pgfqpoint{2.035970in}{0.590560in}}%
\pgfpathlineto{\pgfqpoint{2.036759in}{0.593653in}}%
\pgfpathlineto{\pgfqpoint{2.045838in}{0.636886in}}%
\pgfpathlineto{\pgfqpoint{2.046233in}{0.635046in}}%
\pgfpathlineto{\pgfqpoint{2.050181in}{0.621444in}}%
\pgfpathlineto{\pgfqpoint{2.050575in}{0.621894in}}%
\pgfpathlineto{\pgfqpoint{2.051760in}{0.615181in}}%
\pgfpathlineto{\pgfqpoint{2.052549in}{0.617160in}}%
\pgfpathlineto{\pgfqpoint{2.055707in}{0.629208in}}%
\pgfpathlineto{\pgfqpoint{2.056102in}{0.628230in}}%
\pgfpathlineto{\pgfqpoint{2.060839in}{0.612539in}}%
\pgfpathlineto{\pgfqpoint{2.061628in}{0.615396in}}%
\pgfpathlineto{\pgfqpoint{2.062813in}{0.614311in}}%
\pgfpathlineto{\pgfqpoint{2.063997in}{0.615523in}}%
\pgfpathlineto{\pgfqpoint{2.064392in}{0.616166in}}%
\pgfpathlineto{\pgfqpoint{2.064786in}{0.615142in}}%
\pgfpathlineto{\pgfqpoint{2.067944in}{0.605666in}}%
\pgfpathlineto{\pgfqpoint{2.070313in}{0.599946in}}%
\pgfpathlineto{\pgfqpoint{2.070708in}{0.600613in}}%
\pgfpathlineto{\pgfqpoint{2.071892in}{0.601619in}}%
\pgfpathlineto{\pgfqpoint{2.072681in}{0.604781in}}%
\pgfpathlineto{\pgfqpoint{2.073471in}{0.603131in}}%
\pgfpathlineto{\pgfqpoint{2.074655in}{0.604871in}}%
\pgfpathlineto{\pgfqpoint{2.077418in}{0.615860in}}%
\pgfpathlineto{\pgfqpoint{2.078208in}{0.612446in}}%
\pgfpathlineto{\pgfqpoint{2.079392in}{0.614718in}}%
\pgfpathlineto{\pgfqpoint{2.081761in}{0.626091in}}%
\pgfpathlineto{\pgfqpoint{2.082155in}{0.625948in}}%
\pgfpathlineto{\pgfqpoint{2.084919in}{0.624379in}}%
\pgfpathlineto{\pgfqpoint{2.087287in}{0.628385in}}%
\pgfpathlineto{\pgfqpoint{2.087682in}{0.628046in}}%
\pgfpathlineto{\pgfqpoint{2.088866in}{0.626366in}}%
\pgfpathlineto{\pgfqpoint{2.089261in}{0.627420in}}%
\pgfpathlineto{\pgfqpoint{2.090445in}{0.627197in}}%
\pgfpathlineto{\pgfqpoint{2.094787in}{0.612593in}}%
\pgfpathlineto{\pgfqpoint{2.095577in}{0.615236in}}%
\pgfpathlineto{\pgfqpoint{2.097945in}{0.614321in}}%
\pgfpathlineto{\pgfqpoint{2.100709in}{0.603493in}}%
\pgfpathlineto{\pgfqpoint{2.101103in}{0.605812in}}%
\pgfpathlineto{\pgfqpoint{2.104261in}{0.612791in}}%
\pgfpathlineto{\pgfqpoint{2.108998in}{0.609981in}}%
\pgfpathlineto{\pgfqpoint{2.109788in}{0.608379in}}%
\pgfpathlineto{\pgfqpoint{2.110577in}{0.609413in}}%
\pgfpathlineto{\pgfqpoint{2.112156in}{0.612944in}}%
\pgfpathlineto{\pgfqpoint{2.112551in}{0.610306in}}%
\pgfpathlineto{\pgfqpoint{2.114130in}{0.604320in}}%
\pgfpathlineto{\pgfqpoint{2.114525in}{0.604522in}}%
\pgfpathlineto{\pgfqpoint{2.114920in}{0.606324in}}%
\pgfpathlineto{\pgfqpoint{2.115709in}{0.603277in}}%
\pgfpathlineto{\pgfqpoint{2.117288in}{0.599119in}}%
\pgfpathlineto{\pgfqpoint{2.117683in}{0.599395in}}%
\pgfpathlineto{\pgfqpoint{2.118472in}{0.600634in}}%
\pgfpathlineto{\pgfqpoint{2.119657in}{0.603831in}}%
\pgfpathlineto{\pgfqpoint{2.120051in}{0.602164in}}%
\pgfpathlineto{\pgfqpoint{2.122815in}{0.597084in}}%
\pgfpathlineto{\pgfqpoint{2.123999in}{0.601597in}}%
\pgfpathlineto{\pgfqpoint{2.124788in}{0.601200in}}%
\pgfpathlineto{\pgfqpoint{2.126762in}{0.595480in}}%
\pgfpathlineto{\pgfqpoint{2.127157in}{0.597733in}}%
\pgfpathlineto{\pgfqpoint{2.129130in}{0.599827in}}%
\pgfpathlineto{\pgfqpoint{2.129920in}{0.600943in}}%
\pgfpathlineto{\pgfqpoint{2.133473in}{0.629971in}}%
\pgfpathlineto{\pgfqpoint{2.134657in}{0.638605in}}%
\pgfpathlineto{\pgfqpoint{2.135446in}{0.634729in}}%
\pgfpathlineto{\pgfqpoint{2.139789in}{0.610961in}}%
\pgfpathlineto{\pgfqpoint{2.142947in}{0.602683in}}%
\pgfpathlineto{\pgfqpoint{2.143341in}{0.603600in}}%
\pgfpathlineto{\pgfqpoint{2.146105in}{0.610057in}}%
\pgfpathlineto{\pgfqpoint{2.149263in}{0.599047in}}%
\pgfpathlineto{\pgfqpoint{2.151236in}{0.599713in}}%
\pgfpathlineto{\pgfqpoint{2.152026in}{0.600273in}}%
\pgfpathlineto{\pgfqpoint{2.154000in}{0.590084in}}%
\pgfpathlineto{\pgfqpoint{2.154394in}{0.591116in}}%
\pgfpathlineto{\pgfqpoint{2.157158in}{0.596585in}}%
\pgfpathlineto{\pgfqpoint{2.159526in}{0.597015in}}%
\pgfpathlineto{\pgfqpoint{2.161500in}{0.591083in}}%
\pgfpathlineto{\pgfqpoint{2.162289in}{0.591485in}}%
\pgfpathlineto{\pgfqpoint{2.162684in}{0.592242in}}%
\pgfpathlineto{\pgfqpoint{2.163474in}{0.591129in}}%
\pgfpathlineto{\pgfqpoint{2.165053in}{0.590866in}}%
\pgfpathlineto{\pgfqpoint{2.165447in}{0.591932in}}%
\pgfpathlineto{\pgfqpoint{2.168211in}{0.598613in}}%
\pgfpathlineto{\pgfqpoint{2.169000in}{0.599354in}}%
\pgfpathlineto{\pgfqpoint{2.169790in}{0.601806in}}%
\pgfpathlineto{\pgfqpoint{2.170184in}{0.601437in}}%
\pgfpathlineto{\pgfqpoint{2.170579in}{0.598547in}}%
\pgfpathlineto{\pgfqpoint{2.171369in}{0.602020in}}%
\pgfpathlineto{\pgfqpoint{2.176500in}{0.616237in}}%
\pgfpathlineto{\pgfqpoint{2.177685in}{0.617799in}}%
\pgfpathlineto{\pgfqpoint{2.181632in}{0.627103in}}%
\pgfpathlineto{\pgfqpoint{2.182816in}{0.630314in}}%
\pgfpathlineto{\pgfqpoint{2.183211in}{0.628897in}}%
\pgfpathlineto{\pgfqpoint{2.184395in}{0.625155in}}%
\pgfpathlineto{\pgfqpoint{2.184790in}{0.626632in}}%
\pgfpathlineto{\pgfqpoint{2.186369in}{0.633262in}}%
\pgfpathlineto{\pgfqpoint{2.187159in}{0.629064in}}%
\pgfpathlineto{\pgfqpoint{2.188343in}{0.625955in}}%
\pgfpathlineto{\pgfqpoint{2.189132in}{0.627080in}}%
\pgfpathlineto{\pgfqpoint{2.189922in}{0.628467in}}%
\pgfpathlineto{\pgfqpoint{2.195054in}{0.639061in}}%
\pgfpathlineto{\pgfqpoint{2.195843in}{0.636440in}}%
\pgfpathlineto{\pgfqpoint{2.199396in}{0.631452in}}%
\pgfpathlineto{\pgfqpoint{2.202159in}{0.621039in}}%
\pgfpathlineto{\pgfqpoint{2.206501in}{0.629055in}}%
\pgfpathlineto{\pgfqpoint{2.207686in}{0.626125in}}%
\pgfpathlineto{\pgfqpoint{2.208475in}{0.627296in}}%
\pgfpathlineto{\pgfqpoint{2.209659in}{0.626887in}}%
\pgfpathlineto{\pgfqpoint{2.210844in}{0.622498in}}%
\pgfpathlineto{\pgfqpoint{2.211633in}{0.624609in}}%
\pgfpathlineto{\pgfqpoint{2.214396in}{0.633233in}}%
\pgfpathlineto{\pgfqpoint{2.214791in}{0.632981in}}%
\pgfpathlineto{\pgfqpoint{2.215580in}{0.633378in}}%
\pgfpathlineto{\pgfqpoint{2.215975in}{0.631959in}}%
\pgfpathlineto{\pgfqpoint{2.216370in}{0.630911in}}%
\pgfpathlineto{\pgfqpoint{2.217554in}{0.631970in}}%
\pgfpathlineto{\pgfqpoint{2.219923in}{0.638775in}}%
\pgfpathlineto{\pgfqpoint{2.220317in}{0.637432in}}%
\pgfpathlineto{\pgfqpoint{2.221896in}{0.635069in}}%
\pgfpathlineto{\pgfqpoint{2.222291in}{0.635296in}}%
\pgfpathlineto{\pgfqpoint{2.223870in}{0.638250in}}%
\pgfpathlineto{\pgfqpoint{2.224660in}{0.637381in}}%
\pgfpathlineto{\pgfqpoint{2.227028in}{0.630651in}}%
\pgfpathlineto{\pgfqpoint{2.227423in}{0.630749in}}%
\pgfpathlineto{\pgfqpoint{2.228212in}{0.630305in}}%
\pgfpathlineto{\pgfqpoint{2.229002in}{0.632903in}}%
\pgfpathlineto{\pgfqpoint{2.229791in}{0.630661in}}%
\pgfpathlineto{\pgfqpoint{2.233344in}{0.629208in}}%
\pgfpathlineto{\pgfqpoint{2.237292in}{0.621751in}}%
\pgfpathlineto{\pgfqpoint{2.237686in}{0.622308in}}%
\pgfpathlineto{\pgfqpoint{2.238081in}{0.624439in}}%
\pgfpathlineto{\pgfqpoint{2.238871in}{0.622007in}}%
\pgfpathlineto{\pgfqpoint{2.244002in}{0.603093in}}%
\pgfpathlineto{\pgfqpoint{2.247160in}{0.603541in}}%
\pgfpathlineto{\pgfqpoint{2.248345in}{0.608342in}}%
\pgfpathlineto{\pgfqpoint{2.249134in}{0.607193in}}%
\pgfpathlineto{\pgfqpoint{2.251108in}{0.604153in}}%
\pgfpathlineto{\pgfqpoint{2.251503in}{0.604929in}}%
\pgfpathlineto{\pgfqpoint{2.251897in}{0.605909in}}%
\pgfpathlineto{\pgfqpoint{2.254266in}{0.596689in}}%
\pgfpathlineto{\pgfqpoint{2.255055in}{0.601011in}}%
\pgfpathlineto{\pgfqpoint{2.255845in}{0.599292in}}%
\pgfpathlineto{\pgfqpoint{2.257819in}{0.594633in}}%
\pgfpathlineto{\pgfqpoint{2.258213in}{0.596067in}}%
\pgfpathlineto{\pgfqpoint{2.261766in}{0.614486in}}%
\pgfpathlineto{\pgfqpoint{2.262950in}{0.611622in}}%
\pgfpathlineto{\pgfqpoint{2.266898in}{0.595492in}}%
\pgfpathlineto{\pgfqpoint{2.268872in}{0.589441in}}%
\pgfpathlineto{\pgfqpoint{2.269661in}{0.589768in}}%
\pgfpathlineto{\pgfqpoint{2.271635in}{0.597750in}}%
\pgfpathlineto{\pgfqpoint{2.274793in}{0.590790in}}%
\pgfpathlineto{\pgfqpoint{2.275188in}{0.591747in}}%
\pgfpathlineto{\pgfqpoint{2.277556in}{0.595728in}}%
\pgfpathlineto{\pgfqpoint{2.278740in}{0.594671in}}%
\pgfpathlineto{\pgfqpoint{2.281898in}{0.582187in}}%
\pgfpathlineto{\pgfqpoint{2.282688in}{0.583430in}}%
\pgfpathlineto{\pgfqpoint{2.285056in}{0.587620in}}%
\pgfpathlineto{\pgfqpoint{2.287030in}{0.593093in}}%
\pgfpathlineto{\pgfqpoint{2.289004in}{0.586947in}}%
\pgfpathlineto{\pgfqpoint{2.291767in}{0.583312in}}%
\pgfpathlineto{\pgfqpoint{2.292162in}{0.583771in}}%
\pgfpathlineto{\pgfqpoint{2.294136in}{0.590756in}}%
\pgfpathlineto{\pgfqpoint{2.297293in}{0.581265in}}%
\pgfpathlineto{\pgfqpoint{2.297688in}{0.581702in}}%
\pgfpathlineto{\pgfqpoint{2.300451in}{0.588201in}}%
\pgfpathlineto{\pgfqpoint{2.301241in}{0.587413in}}%
\pgfpathlineto{\pgfqpoint{2.302820in}{0.587436in}}%
\pgfpathlineto{\pgfqpoint{2.303215in}{0.587320in}}%
\pgfpathlineto{\pgfqpoint{2.307557in}{0.571568in}}%
\pgfpathlineto{\pgfqpoint{2.309531in}{0.575005in}}%
\pgfpathlineto{\pgfqpoint{2.310320in}{0.573358in}}%
\pgfpathlineto{\pgfqpoint{2.312294in}{0.575857in}}%
\pgfpathlineto{\pgfqpoint{2.312689in}{0.576225in}}%
\pgfpathlineto{\pgfqpoint{2.313083in}{0.575399in}}%
\pgfpathlineto{\pgfqpoint{2.315847in}{0.569735in}}%
\pgfpathlineto{\pgfqpoint{2.319399in}{0.586598in}}%
\pgfpathlineto{\pgfqpoint{2.320189in}{0.579670in}}%
\pgfpathlineto{\pgfqpoint{2.321768in}{0.581575in}}%
\pgfpathlineto{\pgfqpoint{2.324136in}{0.583653in}}%
\pgfpathlineto{\pgfqpoint{2.325321in}{0.585576in}}%
\pgfpathlineto{\pgfqpoint{2.325715in}{0.586917in}}%
\pgfpathlineto{\pgfqpoint{2.326900in}{0.585495in}}%
\pgfpathlineto{\pgfqpoint{2.329663in}{0.577676in}}%
\pgfpathlineto{\pgfqpoint{2.330058in}{0.577821in}}%
\pgfpathlineto{\pgfqpoint{2.332031in}{0.585076in}}%
\pgfpathlineto{\pgfqpoint{2.332821in}{0.583645in}}%
\pgfpathlineto{\pgfqpoint{2.334400in}{0.579279in}}%
\pgfpathlineto{\pgfqpoint{2.335189in}{0.581708in}}%
\pgfpathlineto{\pgfqpoint{2.336374in}{0.582116in}}%
\pgfpathlineto{\pgfqpoint{2.336768in}{0.579703in}}%
\pgfpathlineto{\pgfqpoint{2.337558in}{0.581025in}}%
\pgfpathlineto{\pgfqpoint{2.341505in}{0.593640in}}%
\pgfpathlineto{\pgfqpoint{2.343479in}{0.595026in}}%
\pgfpathlineto{\pgfqpoint{2.346637in}{0.582481in}}%
\pgfpathlineto{\pgfqpoint{2.347821in}{0.583894in}}%
\pgfpathlineto{\pgfqpoint{2.348216in}{0.584533in}}%
\pgfpathlineto{\pgfqpoint{2.348611in}{0.583318in}}%
\pgfpathlineto{\pgfqpoint{2.349795in}{0.580057in}}%
\pgfpathlineto{\pgfqpoint{2.350979in}{0.581880in}}%
\pgfpathlineto{\pgfqpoint{2.351374in}{0.582168in}}%
\pgfpathlineto{\pgfqpoint{2.353348in}{0.588952in}}%
\pgfpathlineto{\pgfqpoint{2.353743in}{0.588377in}}%
\pgfpathlineto{\pgfqpoint{2.355322in}{0.584786in}}%
\pgfpathlineto{\pgfqpoint{2.356111in}{0.582817in}}%
\pgfpathlineto{\pgfqpoint{2.356506in}{0.585070in}}%
\pgfpathlineto{\pgfqpoint{2.358085in}{0.589903in}}%
\pgfpathlineto{\pgfqpoint{2.358874in}{0.588253in}}%
\pgfpathlineto{\pgfqpoint{2.362032in}{0.579963in}}%
\pgfpathlineto{\pgfqpoint{2.363611in}{0.580940in}}%
\pgfpathlineto{\pgfqpoint{2.365980in}{0.590024in}}%
\pgfpathlineto{\pgfqpoint{2.366375in}{0.589951in}}%
\pgfpathlineto{\pgfqpoint{2.367954in}{0.591760in}}%
\pgfpathlineto{\pgfqpoint{2.368348in}{0.591469in}}%
\pgfpathlineto{\pgfqpoint{2.370717in}{0.590396in}}%
\pgfpathlineto{\pgfqpoint{2.371901in}{0.590930in}}%
\pgfpathlineto{\pgfqpoint{2.375454in}{0.579886in}}%
\pgfpathlineto{\pgfqpoint{2.375849in}{0.581171in}}%
\pgfpathlineto{\pgfqpoint{2.377428in}{0.588550in}}%
\pgfpathlineto{\pgfqpoint{2.377822in}{0.585274in}}%
\pgfpathlineto{\pgfqpoint{2.378217in}{0.584226in}}%
\pgfpathlineto{\pgfqpoint{2.379006in}{0.585552in}}%
\pgfpathlineto{\pgfqpoint{2.379796in}{0.589726in}}%
\pgfpathlineto{\pgfqpoint{2.380585in}{0.587331in}}%
\pgfpathlineto{\pgfqpoint{2.382954in}{0.572800in}}%
\pgfpathlineto{\pgfqpoint{2.384533in}{0.574239in}}%
\pgfpathlineto{\pgfqpoint{2.385717in}{0.577072in}}%
\pgfpathlineto{\pgfqpoint{2.386112in}{0.576192in}}%
\pgfpathlineto{\pgfqpoint{2.388086in}{0.570854in}}%
\pgfpathlineto{\pgfqpoint{2.388480in}{0.571000in}}%
\pgfpathlineto{\pgfqpoint{2.389665in}{0.576934in}}%
\pgfpathlineto{\pgfqpoint{2.391638in}{0.586683in}}%
\pgfpathlineto{\pgfqpoint{2.394402in}{0.584258in}}%
\pgfpathlineto{\pgfqpoint{2.395586in}{0.590772in}}%
\pgfpathlineto{\pgfqpoint{2.395981in}{0.589709in}}%
\pgfpathlineto{\pgfqpoint{2.398349in}{0.584082in}}%
\pgfpathlineto{\pgfqpoint{2.402297in}{0.587317in}}%
\pgfpathlineto{\pgfqpoint{2.402691in}{0.584609in}}%
\pgfpathlineto{\pgfqpoint{2.403481in}{0.587717in}}%
\pgfpathlineto{\pgfqpoint{2.405060in}{0.597424in}}%
\pgfpathlineto{\pgfqpoint{2.406639in}{0.606447in}}%
\pgfpathlineto{\pgfqpoint{2.407428in}{0.605530in}}%
\pgfpathlineto{\pgfqpoint{2.408218in}{0.605820in}}%
\pgfpathlineto{\pgfqpoint{2.408613in}{0.604605in}}%
\pgfpathlineto{\pgfqpoint{2.410981in}{0.595557in}}%
\pgfpathlineto{\pgfqpoint{2.411376in}{0.597479in}}%
\pgfpathlineto{\pgfqpoint{2.414534in}{0.605006in}}%
\pgfpathlineto{\pgfqpoint{2.417692in}{0.608108in}}%
\pgfpathlineto{\pgfqpoint{2.418481in}{0.605551in}}%
\pgfpathlineto{\pgfqpoint{2.420455in}{0.597632in}}%
\pgfpathlineto{\pgfqpoint{2.420850in}{0.597849in}}%
\pgfpathlineto{\pgfqpoint{2.421245in}{0.598051in}}%
\pgfpathlineto{\pgfqpoint{2.422429in}{0.602652in}}%
\pgfpathlineto{\pgfqpoint{2.422824in}{0.599572in}}%
\pgfpathlineto{\pgfqpoint{2.424008in}{0.590713in}}%
\pgfpathlineto{\pgfqpoint{2.424797in}{0.595359in}}%
\pgfpathlineto{\pgfqpoint{2.426376in}{0.602599in}}%
\pgfpathlineto{\pgfqpoint{2.426771in}{0.601896in}}%
\pgfpathlineto{\pgfqpoint{2.427955in}{0.599323in}}%
\pgfpathlineto{\pgfqpoint{2.431508in}{0.593355in}}%
\pgfpathlineto{\pgfqpoint{2.431903in}{0.593541in}}%
\pgfpathlineto{\pgfqpoint{2.433482in}{0.595289in}}%
\pgfpathlineto{\pgfqpoint{2.436245in}{0.601532in}}%
\pgfpathlineto{\pgfqpoint{2.440193in}{0.606078in}}%
\pgfpathlineto{\pgfqpoint{2.440982in}{0.601969in}}%
\pgfpathlineto{\pgfqpoint{2.441772in}{0.597353in}}%
\pgfpathlineto{\pgfqpoint{2.442561in}{0.599446in}}%
\pgfpathlineto{\pgfqpoint{2.444535in}{0.607043in}}%
\pgfpathlineto{\pgfqpoint{2.444930in}{0.606103in}}%
\pgfpathlineto{\pgfqpoint{2.449272in}{0.589577in}}%
\pgfpathlineto{\pgfqpoint{2.453614in}{0.580525in}}%
\pgfpathlineto{\pgfqpoint{2.455588in}{0.585911in}}%
\pgfpathlineto{\pgfqpoint{2.455983in}{0.585198in}}%
\pgfpathlineto{\pgfqpoint{2.457167in}{0.584153in}}%
\pgfpathlineto{\pgfqpoint{2.460325in}{0.591702in}}%
\pgfpathlineto{\pgfqpoint{2.462693in}{0.586485in}}%
\pgfpathlineto{\pgfqpoint{2.464667in}{0.589465in}}%
\pgfpathlineto{\pgfqpoint{2.465062in}{0.588702in}}%
\pgfpathlineto{\pgfqpoint{2.466641in}{0.583803in}}%
\pgfpathlineto{\pgfqpoint{2.467430in}{0.584858in}}%
\pgfpathlineto{\pgfqpoint{2.467825in}{0.584941in}}%
\pgfpathlineto{\pgfqpoint{2.470983in}{0.594111in}}%
\pgfpathlineto{\pgfqpoint{2.472562in}{0.597543in}}%
\pgfpathlineto{\pgfqpoint{2.474141in}{0.600521in}}%
\pgfpathlineto{\pgfqpoint{2.474536in}{0.598903in}}%
\pgfpathlineto{\pgfqpoint{2.474930in}{0.598462in}}%
\pgfpathlineto{\pgfqpoint{2.475325in}{0.599231in}}%
\pgfpathlineto{\pgfqpoint{2.476904in}{0.602480in}}%
\pgfpathlineto{\pgfqpoint{2.478088in}{0.597148in}}%
\pgfpathlineto{\pgfqpoint{2.478878in}{0.598436in}}%
\pgfpathlineto{\pgfqpoint{2.479667in}{0.600257in}}%
\pgfpathlineto{\pgfqpoint{2.480062in}{0.599406in}}%
\pgfpathlineto{\pgfqpoint{2.481641in}{0.593826in}}%
\pgfpathlineto{\pgfqpoint{2.482431in}{0.594757in}}%
\pgfpathlineto{\pgfqpoint{2.484010in}{0.596603in}}%
\pgfpathlineto{\pgfqpoint{2.486773in}{0.605878in}}%
\pgfpathlineto{\pgfqpoint{2.487168in}{0.604914in}}%
\pgfpathlineto{\pgfqpoint{2.487562in}{0.605723in}}%
\pgfpathlineto{\pgfqpoint{2.488747in}{0.609842in}}%
\pgfpathlineto{\pgfqpoint{2.489141in}{0.609411in}}%
\pgfpathlineto{\pgfqpoint{2.492694in}{0.598802in}}%
\pgfpathlineto{\pgfqpoint{2.493089in}{0.599550in}}%
\pgfpathlineto{\pgfqpoint{2.493484in}{0.598961in}}%
\pgfpathlineto{\pgfqpoint{2.495852in}{0.590735in}}%
\pgfpathlineto{\pgfqpoint{2.496642in}{0.591121in}}%
\pgfpathlineto{\pgfqpoint{2.498615in}{0.581661in}}%
\pgfpathlineto{\pgfqpoint{2.499010in}{0.582433in}}%
\pgfpathlineto{\pgfqpoint{2.500589in}{0.585264in}}%
\pgfpathlineto{\pgfqpoint{2.500984in}{0.583356in}}%
\pgfpathlineto{\pgfqpoint{2.501773in}{0.583628in}}%
\pgfpathlineto{\pgfqpoint{2.508089in}{0.602265in}}%
\pgfpathlineto{\pgfqpoint{2.508879in}{0.598100in}}%
\pgfpathlineto{\pgfqpoint{2.512432in}{0.587678in}}%
\pgfpathlineto{\pgfqpoint{2.514011in}{0.591774in}}%
\pgfpathlineto{\pgfqpoint{2.514800in}{0.590814in}}%
\pgfpathlineto{\pgfqpoint{2.516774in}{0.591748in}}%
\pgfpathlineto{\pgfqpoint{2.517563in}{0.596297in}}%
\pgfpathlineto{\pgfqpoint{2.517958in}{0.597238in}}%
\pgfpathlineto{\pgfqpoint{2.518748in}{0.595185in}}%
\pgfpathlineto{\pgfqpoint{2.520327in}{0.596663in}}%
\pgfpathlineto{\pgfqpoint{2.523485in}{0.611543in}}%
\pgfpathlineto{\pgfqpoint{2.525458in}{0.612286in}}%
\pgfpathlineto{\pgfqpoint{2.527037in}{0.615806in}}%
\pgfpathlineto{\pgfqpoint{2.527827in}{0.614836in}}%
\pgfpathlineto{\pgfqpoint{2.532564in}{0.603822in}}%
\pgfpathlineto{\pgfqpoint{2.533748in}{0.602726in}}%
\pgfpathlineto{\pgfqpoint{2.534538in}{0.599801in}}%
\pgfpathlineto{\pgfqpoint{2.535327in}{0.602229in}}%
\pgfpathlineto{\pgfqpoint{2.536906in}{0.601444in}}%
\pgfpathlineto{\pgfqpoint{2.539275in}{0.597814in}}%
\pgfpathlineto{\pgfqpoint{2.539669in}{0.597965in}}%
\pgfpathlineto{\pgfqpoint{2.540459in}{0.597741in}}%
\pgfpathlineto{\pgfqpoint{2.541643in}{0.603742in}}%
\pgfpathlineto{\pgfqpoint{2.542433in}{0.600966in}}%
\pgfpathlineto{\pgfqpoint{2.545196in}{0.594032in}}%
\pgfpathlineto{\pgfqpoint{2.545590in}{0.595465in}}%
\pgfpathlineto{\pgfqpoint{2.545985in}{0.593156in}}%
\pgfpathlineto{\pgfqpoint{2.547169in}{0.590532in}}%
\pgfpathlineto{\pgfqpoint{2.547959in}{0.591725in}}%
\pgfpathlineto{\pgfqpoint{2.548748in}{0.595363in}}%
\pgfpathlineto{\pgfqpoint{2.549538in}{0.602230in}}%
\pgfpathlineto{\pgfqpoint{2.550722in}{0.599939in}}%
\pgfpathlineto{\pgfqpoint{2.551512in}{0.597703in}}%
\pgfpathlineto{\pgfqpoint{2.552301in}{0.599621in}}%
\pgfpathlineto{\pgfqpoint{2.554670in}{0.605851in}}%
\pgfpathlineto{\pgfqpoint{2.555459in}{0.607841in}}%
\pgfpathlineto{\pgfqpoint{2.557038in}{0.610363in}}%
\pgfpathlineto{\pgfqpoint{2.557433in}{0.608710in}}%
\pgfpathlineto{\pgfqpoint{2.559407in}{0.604255in}}%
\pgfpathlineto{\pgfqpoint{2.559801in}{0.604793in}}%
\pgfpathlineto{\pgfqpoint{2.560591in}{0.607996in}}%
\pgfpathlineto{\pgfqpoint{2.560986in}{0.606386in}}%
\pgfpathlineto{\pgfqpoint{2.562565in}{0.604272in}}%
\pgfpathlineto{\pgfqpoint{2.565723in}{0.617112in}}%
\pgfpathlineto{\pgfqpoint{2.566512in}{0.614700in}}%
\pgfpathlineto{\pgfqpoint{2.571249in}{0.598642in}}%
\pgfpathlineto{\pgfqpoint{2.571644in}{0.599966in}}%
\pgfpathlineto{\pgfqpoint{2.573618in}{0.596267in}}%
\pgfpathlineto{\pgfqpoint{2.577170in}{0.583293in}}%
\pgfpathlineto{\pgfqpoint{2.577565in}{0.583477in}}%
\pgfpathlineto{\pgfqpoint{2.579144in}{0.600967in}}%
\pgfpathlineto{\pgfqpoint{2.579934in}{0.598852in}}%
\pgfpathlineto{\pgfqpoint{2.581513in}{0.589645in}}%
\pgfpathlineto{\pgfqpoint{2.582697in}{0.590051in}}%
\pgfpathlineto{\pgfqpoint{2.584276in}{0.595586in}}%
\pgfpathlineto{\pgfqpoint{2.585065in}{0.593662in}}%
\pgfpathlineto{\pgfqpoint{2.585460in}{0.591049in}}%
\pgfpathlineto{\pgfqpoint{2.586644in}{0.591648in}}%
\pgfpathlineto{\pgfqpoint{2.587434in}{0.589570in}}%
\pgfpathlineto{\pgfqpoint{2.590987in}{0.576136in}}%
\pgfpathlineto{\pgfqpoint{2.592171in}{0.577326in}}%
\pgfpathlineto{\pgfqpoint{2.593355in}{0.572018in}}%
\pgfpathlineto{\pgfqpoint{2.593750in}{0.575232in}}%
\pgfpathlineto{\pgfqpoint{2.597697in}{0.591245in}}%
\pgfpathlineto{\pgfqpoint{2.599671in}{0.592719in}}%
\pgfpathlineto{\pgfqpoint{2.600066in}{0.591876in}}%
\pgfpathlineto{\pgfqpoint{2.601645in}{0.589903in}}%
\pgfpathlineto{\pgfqpoint{2.603619in}{0.595250in}}%
\pgfpathlineto{\pgfqpoint{2.604013in}{0.594915in}}%
\pgfpathlineto{\pgfqpoint{2.604408in}{0.596261in}}%
\pgfpathlineto{\pgfqpoint{2.604803in}{0.597213in}}%
\pgfpathlineto{\pgfqpoint{2.605592in}{0.595587in}}%
\pgfpathlineto{\pgfqpoint{2.606777in}{0.591005in}}%
\pgfpathlineto{\pgfqpoint{2.607171in}{0.593369in}}%
\pgfpathlineto{\pgfqpoint{2.609540in}{0.598596in}}%
\pgfpathlineto{\pgfqpoint{2.609935in}{0.598698in}}%
\pgfpathlineto{\pgfqpoint{2.612303in}{0.587882in}}%
\pgfpathlineto{\pgfqpoint{2.613093in}{0.589128in}}%
\pgfpathlineto{\pgfqpoint{2.614672in}{0.583344in}}%
\pgfpathlineto{\pgfqpoint{2.615066in}{0.585142in}}%
\pgfpathlineto{\pgfqpoint{2.615856in}{0.589051in}}%
\pgfpathlineto{\pgfqpoint{2.616645in}{0.586213in}}%
\pgfpathlineto{\pgfqpoint{2.618224in}{0.579152in}}%
\pgfpathlineto{\pgfqpoint{2.619014in}{0.580937in}}%
\pgfpathlineto{\pgfqpoint{2.620988in}{0.590479in}}%
\pgfpathlineto{\pgfqpoint{2.623356in}{0.596392in}}%
\pgfpathlineto{\pgfqpoint{2.623751in}{0.595091in}}%
\pgfpathlineto{\pgfqpoint{2.624935in}{0.595578in}}%
\pgfpathlineto{\pgfqpoint{2.626119in}{0.599134in}}%
\pgfpathlineto{\pgfqpoint{2.626909in}{0.598871in}}%
\pgfpathlineto{\pgfqpoint{2.628882in}{0.594891in}}%
\pgfpathlineto{\pgfqpoint{2.630067in}{0.590690in}}%
\pgfpathlineto{\pgfqpoint{2.630461in}{0.592873in}}%
\pgfpathlineto{\pgfqpoint{2.633619in}{0.589945in}}%
\pgfpathlineto{\pgfqpoint{2.634409in}{0.586331in}}%
\pgfpathlineto{\pgfqpoint{2.635198in}{0.588840in}}%
\pgfpathlineto{\pgfqpoint{2.637962in}{0.599819in}}%
\pgfpathlineto{\pgfqpoint{2.638751in}{0.598539in}}%
\pgfpathlineto{\pgfqpoint{2.640330in}{0.593290in}}%
\pgfpathlineto{\pgfqpoint{2.640725in}{0.593926in}}%
\pgfpathlineto{\pgfqpoint{2.642304in}{0.594392in}}%
\pgfpathlineto{\pgfqpoint{2.642699in}{0.593715in}}%
\pgfpathlineto{\pgfqpoint{2.643093in}{0.595298in}}%
\pgfpathlineto{\pgfqpoint{2.644672in}{0.599493in}}%
\pgfpathlineto{\pgfqpoint{2.645067in}{0.597767in}}%
\pgfpathlineto{\pgfqpoint{2.646251in}{0.595009in}}%
\pgfpathlineto{\pgfqpoint{2.647041in}{0.590530in}}%
\pgfpathlineto{\pgfqpoint{2.647830in}{0.592670in}}%
\pgfpathlineto{\pgfqpoint{2.650988in}{0.609010in}}%
\pgfpathlineto{\pgfqpoint{2.652567in}{0.613913in}}%
\pgfpathlineto{\pgfqpoint{2.656515in}{0.595544in}}%
\pgfpathlineto{\pgfqpoint{2.658883in}{0.593162in}}%
\pgfpathlineto{\pgfqpoint{2.661252in}{0.589787in}}%
\pgfpathlineto{\pgfqpoint{2.663620in}{0.596914in}}%
\pgfpathlineto{\pgfqpoint{2.664410in}{0.595813in}}%
\pgfpathlineto{\pgfqpoint{2.665199in}{0.594342in}}%
\pgfpathlineto{\pgfqpoint{2.665989in}{0.595699in}}%
\pgfpathlineto{\pgfqpoint{2.666778in}{0.595193in}}%
\pgfpathlineto{\pgfqpoint{2.667963in}{0.589581in}}%
\pgfpathlineto{\pgfqpoint{2.668752in}{0.590763in}}%
\pgfpathlineto{\pgfqpoint{2.670331in}{0.596899in}}%
\pgfpathlineto{\pgfqpoint{2.672305in}{0.602373in}}%
\pgfpathlineto{\pgfqpoint{2.673884in}{0.609718in}}%
\pgfpathlineto{\pgfqpoint{2.674279in}{0.611561in}}%
\pgfpathlineto{\pgfqpoint{2.675068in}{0.608001in}}%
\pgfpathlineto{\pgfqpoint{2.677437in}{0.602523in}}%
\pgfpathlineto{\pgfqpoint{2.678621in}{0.600275in}}%
\pgfpathlineto{\pgfqpoint{2.679016in}{0.601713in}}%
\pgfpathlineto{\pgfqpoint{2.679805in}{0.605787in}}%
\pgfpathlineto{\pgfqpoint{2.680595in}{0.603523in}}%
\pgfpathlineto{\pgfqpoint{2.680989in}{0.603959in}}%
\pgfpathlineto{\pgfqpoint{2.681384in}{0.603452in}}%
\pgfpathlineto{\pgfqpoint{2.681779in}{0.602255in}}%
\pgfpathlineto{\pgfqpoint{2.682568in}{0.603212in}}%
\pgfpathlineto{\pgfqpoint{2.684542in}{0.607395in}}%
\pgfpathlineto{\pgfqpoint{2.684937in}{0.605694in}}%
\pgfpathlineto{\pgfqpoint{2.685726in}{0.604375in}}%
\pgfpathlineto{\pgfqpoint{2.686911in}{0.605054in}}%
\pgfpathlineto{\pgfqpoint{2.687305in}{0.605201in}}%
\pgfpathlineto{\pgfqpoint{2.688490in}{0.602448in}}%
\pgfpathlineto{\pgfqpoint{2.688884in}{0.604231in}}%
\pgfpathlineto{\pgfqpoint{2.690069in}{0.605351in}}%
\pgfpathlineto{\pgfqpoint{2.690463in}{0.604155in}}%
\pgfpathlineto{\pgfqpoint{2.691253in}{0.604664in}}%
\pgfpathlineto{\pgfqpoint{2.692832in}{0.603078in}}%
\pgfpathlineto{\pgfqpoint{2.693227in}{0.603924in}}%
\pgfpathlineto{\pgfqpoint{2.693621in}{0.602770in}}%
\pgfpathlineto{\pgfqpoint{2.695990in}{0.592420in}}%
\pgfpathlineto{\pgfqpoint{2.696385in}{0.594052in}}%
\pgfpathlineto{\pgfqpoint{2.698753in}{0.600869in}}%
\pgfpathlineto{\pgfqpoint{2.702306in}{0.591779in}}%
\pgfpathlineto{\pgfqpoint{2.702701in}{0.593294in}}%
\pgfpathlineto{\pgfqpoint{2.705464in}{0.598962in}}%
\pgfpathlineto{\pgfqpoint{2.705859in}{0.598661in}}%
\pgfpathlineto{\pgfqpoint{2.707438in}{0.598435in}}%
\pgfpathlineto{\pgfqpoint{2.709017in}{0.601600in}}%
\pgfpathlineto{\pgfqpoint{2.709806in}{0.603034in}}%
\pgfpathlineto{\pgfqpoint{2.710201in}{0.600655in}}%
\pgfpathlineto{\pgfqpoint{2.711780in}{0.590061in}}%
\pgfpathlineto{\pgfqpoint{2.712569in}{0.595301in}}%
\pgfpathlineto{\pgfqpoint{2.714938in}{0.602592in}}%
\pgfpathlineto{\pgfqpoint{2.721254in}{0.586133in}}%
\pgfpathlineto{\pgfqpoint{2.722043in}{0.587429in}}%
\pgfpathlineto{\pgfqpoint{2.722438in}{0.589162in}}%
\pgfpathlineto{\pgfqpoint{2.723227in}{0.588079in}}%
\pgfpathlineto{\pgfqpoint{2.725991in}{0.577264in}}%
\pgfpathlineto{\pgfqpoint{2.727175in}{0.578272in}}%
\pgfpathlineto{\pgfqpoint{2.728359in}{0.576125in}}%
\pgfpathlineto{\pgfqpoint{2.728754in}{0.575737in}}%
\pgfpathlineto{\pgfqpoint{2.729149in}{0.576317in}}%
\pgfpathlineto{\pgfqpoint{2.730333in}{0.584245in}}%
\pgfpathlineto{\pgfqpoint{2.731122in}{0.583259in}}%
\pgfpathlineto{\pgfqpoint{2.731517in}{0.583446in}}%
\pgfpathlineto{\pgfqpoint{2.735070in}{0.575416in}}%
\pgfpathlineto{\pgfqpoint{2.735465in}{0.575204in}}%
\pgfpathlineto{\pgfqpoint{2.735859in}{0.576377in}}%
\pgfpathlineto{\pgfqpoint{2.736649in}{0.577395in}}%
\pgfpathlineto{\pgfqpoint{2.740202in}{0.584806in}}%
\pgfpathlineto{\pgfqpoint{2.740991in}{0.579236in}}%
\pgfpathlineto{\pgfqpoint{2.741781in}{0.583062in}}%
\pgfpathlineto{\pgfqpoint{2.744149in}{0.590265in}}%
\pgfpathlineto{\pgfqpoint{2.745728in}{0.589214in}}%
\pgfpathlineto{\pgfqpoint{2.746518in}{0.588464in}}%
\pgfpathlineto{\pgfqpoint{2.746912in}{0.589259in}}%
\pgfpathlineto{\pgfqpoint{2.748886in}{0.591308in}}%
\pgfpathlineto{\pgfqpoint{2.749676in}{0.590388in}}%
\pgfpathlineto{\pgfqpoint{2.751255in}{0.586532in}}%
\pgfpathlineto{\pgfqpoint{2.751649in}{0.586838in}}%
\pgfpathlineto{\pgfqpoint{2.754807in}{0.591818in}}%
\pgfpathlineto{\pgfqpoint{2.757571in}{0.599569in}}%
\pgfpathlineto{\pgfqpoint{2.759150in}{0.604588in}}%
\pgfpathlineto{\pgfqpoint{2.759939in}{0.604041in}}%
\pgfpathlineto{\pgfqpoint{2.760334in}{0.604473in}}%
\pgfpathlineto{\pgfqpoint{2.760729in}{0.602927in}}%
\pgfpathlineto{\pgfqpoint{2.763492in}{0.595310in}}%
\pgfpathlineto{\pgfqpoint{2.763887in}{0.596518in}}%
\pgfpathlineto{\pgfqpoint{2.765071in}{0.599479in}}%
\pgfpathlineto{\pgfqpoint{2.770992in}{0.620228in}}%
\pgfpathlineto{\pgfqpoint{2.771387in}{0.618736in}}%
\pgfpathlineto{\pgfqpoint{2.772571in}{0.620825in}}%
\pgfpathlineto{\pgfqpoint{2.773361in}{0.623740in}}%
\pgfpathlineto{\pgfqpoint{2.774150in}{0.622350in}}%
\pgfpathlineto{\pgfqpoint{2.775729in}{0.616903in}}%
\pgfpathlineto{\pgfqpoint{2.776124in}{0.617922in}}%
\pgfpathlineto{\pgfqpoint{2.778492in}{0.623473in}}%
\pgfpathlineto{\pgfqpoint{2.779282in}{0.624725in}}%
\pgfpathlineto{\pgfqpoint{2.781256in}{0.628141in}}%
\pgfpathlineto{\pgfqpoint{2.784808in}{0.631113in}}%
\pgfpathlineto{\pgfqpoint{2.785993in}{0.634731in}}%
\pgfpathlineto{\pgfqpoint{2.786782in}{0.634015in}}%
\pgfpathlineto{\pgfqpoint{2.789940in}{0.628219in}}%
\pgfpathlineto{\pgfqpoint{2.791519in}{0.630424in}}%
\pgfpathlineto{\pgfqpoint{2.791914in}{0.629221in}}%
\pgfpathlineto{\pgfqpoint{2.792703in}{0.629189in}}%
\pgfpathlineto{\pgfqpoint{2.793098in}{0.630181in}}%
\pgfpathlineto{\pgfqpoint{2.795072in}{0.631172in}}%
\pgfpathlineto{\pgfqpoint{2.797045in}{0.625270in}}%
\pgfpathlineto{\pgfqpoint{2.797835in}{0.628276in}}%
\pgfpathlineto{\pgfqpoint{2.798230in}{0.626752in}}%
\pgfpathlineto{\pgfqpoint{2.799809in}{0.621511in}}%
\pgfpathlineto{\pgfqpoint{2.800203in}{0.623015in}}%
\pgfpathlineto{\pgfqpoint{2.800993in}{0.621863in}}%
\pgfpathlineto{\pgfqpoint{2.802177in}{0.619345in}}%
\pgfpathlineto{\pgfqpoint{2.806519in}{0.604540in}}%
\pgfpathlineto{\pgfqpoint{2.807704in}{0.602980in}}%
\pgfpathlineto{\pgfqpoint{2.808098in}{0.603404in}}%
\pgfpathlineto{\pgfqpoint{2.812046in}{0.614962in}}%
\pgfpathlineto{\pgfqpoint{2.813230in}{0.616057in}}%
\pgfpathlineto{\pgfqpoint{2.813625in}{0.614761in}}%
\pgfpathlineto{\pgfqpoint{2.814809in}{0.612222in}}%
\pgfpathlineto{\pgfqpoint{2.815204in}{0.612775in}}%
\pgfpathlineto{\pgfqpoint{2.816388in}{0.618027in}}%
\pgfpathlineto{\pgfqpoint{2.817967in}{0.617049in}}%
\pgfpathlineto{\pgfqpoint{2.818362in}{0.615522in}}%
\pgfpathlineto{\pgfqpoint{2.818757in}{0.617411in}}%
\pgfpathlineto{\pgfqpoint{2.820730in}{0.620941in}}%
\pgfpathlineto{\pgfqpoint{2.822704in}{0.613180in}}%
\pgfpathlineto{\pgfqpoint{2.823099in}{0.613902in}}%
\pgfpathlineto{\pgfqpoint{2.823494in}{0.612764in}}%
\pgfpathlineto{\pgfqpoint{2.824283in}{0.610979in}}%
\pgfpathlineto{\pgfqpoint{2.824678in}{0.613258in}}%
\pgfpathlineto{\pgfqpoint{2.826652in}{0.619089in}}%
\pgfpathlineto{\pgfqpoint{2.827046in}{0.618880in}}%
\pgfpathlineto{\pgfqpoint{2.828231in}{0.615318in}}%
\pgfpathlineto{\pgfqpoint{2.829415in}{0.616717in}}%
\pgfpathlineto{\pgfqpoint{2.832573in}{0.619946in}}%
\pgfpathlineto{\pgfqpoint{2.832968in}{0.619120in}}%
\pgfpathlineto{\pgfqpoint{2.833362in}{0.620000in}}%
\pgfpathlineto{\pgfqpoint{2.834547in}{0.624094in}}%
\pgfpathlineto{\pgfqpoint{2.835336in}{0.623707in}}%
\pgfpathlineto{\pgfqpoint{2.836520in}{0.620036in}}%
\pgfpathlineto{\pgfqpoint{2.837310in}{0.620730in}}%
\pgfpathlineto{\pgfqpoint{2.838889in}{0.625109in}}%
\pgfpathlineto{\pgfqpoint{2.839284in}{0.623697in}}%
\pgfpathlineto{\pgfqpoint{2.842047in}{0.610071in}}%
\pgfpathlineto{\pgfqpoint{2.842836in}{0.610450in}}%
\pgfpathlineto{\pgfqpoint{2.843626in}{0.611417in}}%
\pgfpathlineto{\pgfqpoint{2.850337in}{0.637197in}}%
\pgfpathlineto{\pgfqpoint{2.850731in}{0.635318in}}%
\pgfpathlineto{\pgfqpoint{2.852705in}{0.624642in}}%
\pgfpathlineto{\pgfqpoint{2.853495in}{0.626819in}}%
\pgfpathlineto{\pgfqpoint{2.855074in}{0.631755in}}%
\pgfpathlineto{\pgfqpoint{2.855468in}{0.630969in}}%
\pgfpathlineto{\pgfqpoint{2.856653in}{0.628413in}}%
\pgfpathlineto{\pgfqpoint{2.857837in}{0.624962in}}%
\pgfpathlineto{\pgfqpoint{2.859021in}{0.625892in}}%
\pgfpathlineto{\pgfqpoint{2.859811in}{0.625638in}}%
\pgfpathlineto{\pgfqpoint{2.861390in}{0.621902in}}%
\pgfpathlineto{\pgfqpoint{2.861784in}{0.622465in}}%
\pgfpathlineto{\pgfqpoint{2.862179in}{0.623334in}}%
\pgfpathlineto{\pgfqpoint{2.862574in}{0.621200in}}%
\pgfpathlineto{\pgfqpoint{2.862969in}{0.621527in}}%
\pgfpathlineto{\pgfqpoint{2.864153in}{0.617207in}}%
\pgfpathlineto{\pgfqpoint{2.866521in}{0.607634in}}%
\pgfpathlineto{\pgfqpoint{2.867706in}{0.613289in}}%
\pgfpathlineto{\pgfqpoint{2.868495in}{0.611724in}}%
\pgfpathlineto{\pgfqpoint{2.870469in}{0.606813in}}%
\pgfpathlineto{\pgfqpoint{2.870864in}{0.608285in}}%
\pgfpathlineto{\pgfqpoint{2.871258in}{0.608864in}}%
\pgfpathlineto{\pgfqpoint{2.871653in}{0.608081in}}%
\pgfpathlineto{\pgfqpoint{2.872837in}{0.604167in}}%
\pgfpathlineto{\pgfqpoint{2.873627in}{0.605602in}}%
\pgfpathlineto{\pgfqpoint{2.874811in}{0.611121in}}%
\pgfpathlineto{\pgfqpoint{2.876785in}{0.624395in}}%
\pgfpathlineto{\pgfqpoint{2.877180in}{0.623405in}}%
\pgfpathlineto{\pgfqpoint{2.878759in}{0.618975in}}%
\pgfpathlineto{\pgfqpoint{2.879548in}{0.620165in}}%
\pgfpathlineto{\pgfqpoint{2.884680in}{0.620269in}}%
\pgfpathlineto{\pgfqpoint{2.885864in}{0.626664in}}%
\pgfpathlineto{\pgfqpoint{2.886653in}{0.625140in}}%
\pgfpathlineto{\pgfqpoint{2.890206in}{0.613346in}}%
\pgfpathlineto{\pgfqpoint{2.893759in}{0.593828in}}%
\pgfpathlineto{\pgfqpoint{2.894548in}{0.595700in}}%
\pgfpathlineto{\pgfqpoint{2.895338in}{0.596851in}}%
\pgfpathlineto{\pgfqpoint{2.896522in}{0.600926in}}%
\pgfpathlineto{\pgfqpoint{2.896917in}{0.599525in}}%
\pgfpathlineto{\pgfqpoint{2.899285in}{0.595502in}}%
\pgfpathlineto{\pgfqpoint{2.901259in}{0.592938in}}%
\pgfpathlineto{\pgfqpoint{2.902049in}{0.591839in}}%
\pgfpathlineto{\pgfqpoint{2.902443in}{0.593625in}}%
\pgfpathlineto{\pgfqpoint{2.903233in}{0.595004in}}%
\pgfpathlineto{\pgfqpoint{2.905207in}{0.606380in}}%
\pgfpathlineto{\pgfqpoint{2.905996in}{0.605420in}}%
\pgfpathlineto{\pgfqpoint{2.907180in}{0.606432in}}%
\pgfpathlineto{\pgfqpoint{2.907970in}{0.606871in}}%
\pgfpathlineto{\pgfqpoint{2.911523in}{0.595864in}}%
\pgfpathlineto{\pgfqpoint{2.912312in}{0.598493in}}%
\pgfpathlineto{\pgfqpoint{2.913102in}{0.596548in}}%
\pgfpathlineto{\pgfqpoint{2.913891in}{0.594623in}}%
\pgfpathlineto{\pgfqpoint{2.914286in}{0.595504in}}%
\pgfpathlineto{\pgfqpoint{2.916654in}{0.600395in}}%
\pgfpathlineto{\pgfqpoint{2.918628in}{0.613547in}}%
\pgfpathlineto{\pgfqpoint{2.919812in}{0.618288in}}%
\pgfpathlineto{\pgfqpoint{2.920207in}{0.616683in}}%
\pgfpathlineto{\pgfqpoint{2.921391in}{0.616175in}}%
\pgfpathlineto{\pgfqpoint{2.921786in}{0.616892in}}%
\pgfpathlineto{\pgfqpoint{2.924155in}{0.615025in}}%
\pgfpathlineto{\pgfqpoint{2.924549in}{0.616369in}}%
\pgfpathlineto{\pgfqpoint{2.926918in}{0.627647in}}%
\pgfpathlineto{\pgfqpoint{2.927707in}{0.627472in}}%
\pgfpathlineto{\pgfqpoint{2.928892in}{0.623923in}}%
\pgfpathlineto{\pgfqpoint{2.929286in}{0.626855in}}%
\pgfpathlineto{\pgfqpoint{2.932050in}{0.635751in}}%
\pgfpathlineto{\pgfqpoint{2.932444in}{0.634751in}}%
\pgfpathlineto{\pgfqpoint{2.934813in}{0.629375in}}%
\pgfpathlineto{\pgfqpoint{2.935602in}{0.629649in}}%
\pgfpathlineto{\pgfqpoint{2.936392in}{0.629913in}}%
\pgfpathlineto{\pgfqpoint{2.938760in}{0.615782in}}%
\pgfpathlineto{\pgfqpoint{2.941524in}{0.602419in}}%
\pgfpathlineto{\pgfqpoint{2.941918in}{0.602528in}}%
\pgfpathlineto{\pgfqpoint{2.945076in}{0.606809in}}%
\pgfpathlineto{\pgfqpoint{2.945866in}{0.612385in}}%
\pgfpathlineto{\pgfqpoint{2.946655in}{0.609604in}}%
\pgfpathlineto{\pgfqpoint{2.952182in}{0.596614in}}%
\pgfpathlineto{\pgfqpoint{2.952971in}{0.598514in}}%
\pgfpathlineto{\pgfqpoint{2.954550in}{0.602513in}}%
\pgfpathlineto{\pgfqpoint{2.959287in}{0.620775in}}%
\pgfpathlineto{\pgfqpoint{2.960866in}{0.622433in}}%
\pgfpathlineto{\pgfqpoint{2.963235in}{0.627944in}}%
\pgfpathlineto{\pgfqpoint{2.964419in}{0.626318in}}%
\pgfpathlineto{\pgfqpoint{2.964814in}{0.627163in}}%
\pgfpathlineto{\pgfqpoint{2.965208in}{0.628574in}}%
\pgfpathlineto{\pgfqpoint{2.965603in}{0.627732in}}%
\pgfpathlineto{\pgfqpoint{2.967577in}{0.621613in}}%
\pgfpathlineto{\pgfqpoint{2.967972in}{0.621729in}}%
\pgfpathlineto{\pgfqpoint{2.969551in}{0.621741in}}%
\pgfpathlineto{\pgfqpoint{2.971919in}{0.612019in}}%
\pgfpathlineto{\pgfqpoint{2.972314in}{0.612593in}}%
\pgfpathlineto{\pgfqpoint{2.973103in}{0.612887in}}%
\pgfpathlineto{\pgfqpoint{2.973498in}{0.612353in}}%
\pgfpathlineto{\pgfqpoint{2.973893in}{0.611773in}}%
\pgfpathlineto{\pgfqpoint{2.974288in}{0.612510in}}%
\pgfpathlineto{\pgfqpoint{2.974682in}{0.614439in}}%
\pgfpathlineto{\pgfqpoint{2.975472in}{0.613689in}}%
\pgfpathlineto{\pgfqpoint{2.982183in}{0.587198in}}%
\pgfpathlineto{\pgfqpoint{2.982577in}{0.588652in}}%
\pgfpathlineto{\pgfqpoint{2.984156in}{0.592420in}}%
\pgfpathlineto{\pgfqpoint{2.984551in}{0.592338in}}%
\pgfpathlineto{\pgfqpoint{2.986525in}{0.596954in}}%
\pgfpathlineto{\pgfqpoint{2.986920in}{0.596261in}}%
\pgfpathlineto{\pgfqpoint{2.989683in}{0.589571in}}%
\pgfpathlineto{\pgfqpoint{2.987709in}{0.597061in}}%
\pgfpathlineto{\pgfqpoint{2.990078in}{0.589975in}}%
\pgfpathlineto{\pgfqpoint{2.992051in}{0.585130in}}%
\pgfpathlineto{\pgfqpoint{2.994025in}{0.587984in}}%
\pgfpathlineto{\pgfqpoint{2.995604in}{0.589117in}}%
\pgfpathlineto{\pgfqpoint{2.996394in}{0.592578in}}%
\pgfpathlineto{\pgfqpoint{2.998762in}{0.605105in}}%
\pgfpathlineto{\pgfqpoint{3.000341in}{0.601719in}}%
\pgfpathlineto{\pgfqpoint{3.003104in}{0.601631in}}%
\pgfpathlineto{\pgfqpoint{3.004289in}{0.604441in}}%
\pgfpathlineto{\pgfqpoint{3.004683in}{0.602846in}}%
\pgfpathlineto{\pgfqpoint{3.005078in}{0.602304in}}%
\pgfpathlineto{\pgfqpoint{3.005473in}{0.603642in}}%
\pgfpathlineto{\pgfqpoint{3.006657in}{0.605478in}}%
\pgfpathlineto{\pgfqpoint{3.007447in}{0.603918in}}%
\pgfpathlineto{\pgfqpoint{3.010605in}{0.598572in}}%
\pgfpathlineto{\pgfqpoint{3.012973in}{0.603433in}}%
\pgfpathlineto{\pgfqpoint{3.013368in}{0.601972in}}%
\pgfpathlineto{\pgfqpoint{3.014157in}{0.603617in}}%
\pgfpathlineto{\pgfqpoint{3.015736in}{0.605354in}}%
\pgfpathlineto{\pgfqpoint{3.018105in}{0.614198in}}%
\pgfpathlineto{\pgfqpoint{3.018894in}{0.610416in}}%
\pgfpathlineto{\pgfqpoint{3.020473in}{0.606237in}}%
\pgfpathlineto{\pgfqpoint{3.020868in}{0.606624in}}%
\pgfpathlineto{\pgfqpoint{3.021263in}{0.607648in}}%
\pgfpathlineto{\pgfqpoint{3.021658in}{0.605097in}}%
\pgfpathlineto{\pgfqpoint{3.025605in}{0.593956in}}%
\pgfpathlineto{\pgfqpoint{3.028368in}{0.599515in}}%
\pgfpathlineto{\pgfqpoint{3.031132in}{0.597366in}}%
\pgfpathlineto{\pgfqpoint{3.031526in}{0.598053in}}%
\pgfpathlineto{\pgfqpoint{3.033105in}{0.600926in}}%
\pgfpathlineto{\pgfqpoint{3.033500in}{0.600404in}}%
\pgfpathlineto{\pgfqpoint{3.035474in}{0.596011in}}%
\pgfpathlineto{\pgfqpoint{3.039421in}{0.603844in}}%
\pgfpathlineto{\pgfqpoint{3.041000in}{0.608696in}}%
\pgfpathlineto{\pgfqpoint{3.041395in}{0.608426in}}%
\pgfpathlineto{\pgfqpoint{3.042185in}{0.609988in}}%
\pgfpathlineto{\pgfqpoint{3.042974in}{0.612396in}}%
\pgfpathlineto{\pgfqpoint{3.043764in}{0.609881in}}%
\pgfpathlineto{\pgfqpoint{3.044553in}{0.608645in}}%
\pgfpathlineto{\pgfqpoint{3.045737in}{0.613679in}}%
\pgfpathlineto{\pgfqpoint{3.046527in}{0.612196in}}%
\pgfpathlineto{\pgfqpoint{3.048895in}{0.603765in}}%
\pgfpathlineto{\pgfqpoint{3.049290in}{0.604753in}}%
\pgfpathlineto{\pgfqpoint{3.050869in}{0.602891in}}%
\pgfpathlineto{\pgfqpoint{3.052053in}{0.596511in}}%
\pgfpathlineto{\pgfqpoint{3.052843in}{0.597288in}}%
\pgfpathlineto{\pgfqpoint{3.053632in}{0.596304in}}%
\pgfpathlineto{\pgfqpoint{3.055211in}{0.591849in}}%
\pgfpathlineto{\pgfqpoint{3.056395in}{0.588736in}}%
\pgfpathlineto{\pgfqpoint{3.056790in}{0.590945in}}%
\pgfpathlineto{\pgfqpoint{3.057580in}{0.593382in}}%
\pgfpathlineto{\pgfqpoint{3.058764in}{0.592889in}}%
\pgfpathlineto{\pgfqpoint{3.059553in}{0.596542in}}%
\pgfpathlineto{\pgfqpoint{3.062317in}{0.607283in}}%
\pgfpathlineto{\pgfqpoint{3.064685in}{0.615590in}}%
\pgfpathlineto{\pgfqpoint{3.065080in}{0.614135in}}%
\pgfpathlineto{\pgfqpoint{3.067843in}{0.603182in}}%
\pgfpathlineto{\pgfqpoint{3.068238in}{0.604144in}}%
\pgfpathlineto{\pgfqpoint{3.068633in}{0.604046in}}%
\pgfpathlineto{\pgfqpoint{3.069422in}{0.605792in}}%
\pgfpathlineto{\pgfqpoint{3.069817in}{0.605301in}}%
\pgfpathlineto{\pgfqpoint{3.071396in}{0.594936in}}%
\pgfpathlineto{\pgfqpoint{3.071791in}{0.596832in}}%
\pgfpathlineto{\pgfqpoint{3.074949in}{0.609546in}}%
\pgfpathlineto{\pgfqpoint{3.076528in}{0.607001in}}%
\pgfpathlineto{\pgfqpoint{3.079291in}{0.608581in}}%
\pgfpathlineto{\pgfqpoint{3.079686in}{0.607515in}}%
\pgfpathlineto{\pgfqpoint{3.080475in}{0.605659in}}%
\pgfpathlineto{\pgfqpoint{3.080870in}{0.606253in}}%
\pgfpathlineto{\pgfqpoint{3.081265in}{0.607653in}}%
\pgfpathlineto{\pgfqpoint{3.082054in}{0.605957in}}%
\pgfpathlineto{\pgfqpoint{3.082844in}{0.602336in}}%
\pgfpathlineto{\pgfqpoint{3.083633in}{0.603808in}}%
\pgfpathlineto{\pgfqpoint{3.084028in}{0.604831in}}%
\pgfpathlineto{\pgfqpoint{3.084423in}{0.602405in}}%
\pgfpathlineto{\pgfqpoint{3.086791in}{0.595578in}}%
\pgfpathlineto{\pgfqpoint{3.087581in}{0.596325in}}%
\pgfpathlineto{\pgfqpoint{3.088370in}{0.597647in}}%
\pgfpathlineto{\pgfqpoint{3.088765in}{0.595221in}}%
\pgfpathlineto{\pgfqpoint{3.091133in}{0.587428in}}%
\pgfpathlineto{\pgfqpoint{3.091923in}{0.589086in}}%
\pgfpathlineto{\pgfqpoint{3.093107in}{0.593327in}}%
\pgfpathlineto{\pgfqpoint{3.093502in}{0.589862in}}%
\pgfpathlineto{\pgfqpoint{3.095870in}{0.580395in}}%
\pgfpathlineto{\pgfqpoint{3.096660in}{0.581099in}}%
\pgfpathlineto{\pgfqpoint{3.101397in}{0.597110in}}%
\pgfpathlineto{\pgfqpoint{3.101792in}{0.596374in}}%
\pgfpathlineto{\pgfqpoint{3.102581in}{0.594339in}}%
\pgfpathlineto{\pgfqpoint{3.102976in}{0.596320in}}%
\pgfpathlineto{\pgfqpoint{3.103371in}{0.596805in}}%
\pgfpathlineto{\pgfqpoint{3.103765in}{0.596117in}}%
\pgfpathlineto{\pgfqpoint{3.106134in}{0.589483in}}%
\pgfpathlineto{\pgfqpoint{3.106923in}{0.591370in}}%
\pgfpathlineto{\pgfqpoint{3.108108in}{0.596588in}}%
\pgfpathlineto{\pgfqpoint{3.109292in}{0.594589in}}%
\pgfpathlineto{\pgfqpoint{3.110081in}{0.595301in}}%
\pgfpathlineto{\pgfqpoint{3.110476in}{0.592360in}}%
\pgfpathlineto{\pgfqpoint{3.111660in}{0.595112in}}%
\pgfpathlineto{\pgfqpoint{3.112055in}{0.594098in}}%
\pgfpathlineto{\pgfqpoint{3.112450in}{0.595557in}}%
\pgfpathlineto{\pgfqpoint{3.113239in}{0.595269in}}%
\pgfpathlineto{\pgfqpoint{3.114424in}{0.596300in}}%
\pgfpathlineto{\pgfqpoint{3.114818in}{0.595149in}}%
\pgfpathlineto{\pgfqpoint{3.116792in}{0.586921in}}%
\pgfpathlineto{\pgfqpoint{3.117582in}{0.587460in}}%
\pgfpathlineto{\pgfqpoint{3.119161in}{0.587637in}}%
\pgfpathlineto{\pgfqpoint{3.121134in}{0.576670in}}%
\pgfpathlineto{\pgfqpoint{3.121924in}{0.577264in}}%
\pgfpathlineto{\pgfqpoint{3.126266in}{0.585370in}}%
\pgfpathlineto{\pgfqpoint{3.126661in}{0.584220in}}%
\pgfpathlineto{\pgfqpoint{3.127450in}{0.583106in}}%
\pgfpathlineto{\pgfqpoint{3.127845in}{0.583476in}}%
\pgfpathlineto{\pgfqpoint{3.129029in}{0.587911in}}%
\pgfpathlineto{\pgfqpoint{3.130213in}{0.586916in}}%
\pgfpathlineto{\pgfqpoint{3.132187in}{0.581330in}}%
\pgfpathlineto{\pgfqpoint{3.132582in}{0.583324in}}%
\pgfpathlineto{\pgfqpoint{3.134161in}{0.587285in}}%
\pgfpathlineto{\pgfqpoint{3.134556in}{0.585905in}}%
\pgfpathlineto{\pgfqpoint{3.140477in}{0.572147in}}%
\pgfpathlineto{\pgfqpoint{3.144424in}{0.586522in}}%
\pgfpathlineto{\pgfqpoint{3.145609in}{0.589162in}}%
\pgfpathlineto{\pgfqpoint{3.146398in}{0.588024in}}%
\pgfpathlineto{\pgfqpoint{3.148372in}{0.581294in}}%
\pgfpathlineto{\pgfqpoint{3.149161in}{0.582663in}}%
\pgfpathlineto{\pgfqpoint{3.149556in}{0.583497in}}%
\pgfpathlineto{\pgfqpoint{3.149951in}{0.580989in}}%
\pgfpathlineto{\pgfqpoint{3.154293in}{0.565866in}}%
\pgfpathlineto{\pgfqpoint{3.162978in}{0.585834in}}%
\pgfpathlineto{\pgfqpoint{3.164162in}{0.583446in}}%
\pgfpathlineto{\pgfqpoint{3.164557in}{0.584298in}}%
\pgfpathlineto{\pgfqpoint{3.166925in}{0.587369in}}%
\pgfpathlineto{\pgfqpoint{3.167320in}{0.585781in}}%
\pgfpathlineto{\pgfqpoint{3.171662in}{0.574171in}}%
\pgfpathlineto{\pgfqpoint{3.172057in}{0.573562in}}%
\pgfpathlineto{\pgfqpoint{3.172452in}{0.575071in}}%
\pgfpathlineto{\pgfqpoint{3.177583in}{0.589725in}}%
\pgfpathlineto{\pgfqpoint{3.179952in}{0.591075in}}%
\pgfpathlineto{\pgfqpoint{3.180347in}{0.590054in}}%
\pgfpathlineto{\pgfqpoint{3.180741in}{0.591039in}}%
\pgfpathlineto{\pgfqpoint{3.181926in}{0.596486in}}%
\pgfpathlineto{\pgfqpoint{3.182715in}{0.594697in}}%
\pgfpathlineto{\pgfqpoint{3.183505in}{0.592070in}}%
\pgfpathlineto{\pgfqpoint{3.184294in}{0.594376in}}%
\pgfpathlineto{\pgfqpoint{3.185478in}{0.594741in}}%
\pgfpathlineto{\pgfqpoint{3.185873in}{0.593836in}}%
\pgfpathlineto{\pgfqpoint{3.187057in}{0.592292in}}%
\pgfpathlineto{\pgfqpoint{3.187847in}{0.593383in}}%
\pgfpathlineto{\pgfqpoint{3.189821in}{0.595845in}}%
\pgfpathlineto{\pgfqpoint{3.192979in}{0.607155in}}%
\pgfpathlineto{\pgfqpoint{3.193768in}{0.606600in}}%
\pgfpathlineto{\pgfqpoint{3.196137in}{0.601788in}}%
\pgfpathlineto{\pgfqpoint{3.196926in}{0.604785in}}%
\pgfpathlineto{\pgfqpoint{3.197321in}{0.606560in}}%
\pgfpathlineto{\pgfqpoint{3.197321in}{0.606560in}}%
\pgfusepath{stroke}%
\end{pgfscope}%
\begin{pgfscope}%
\pgfpathrectangle{\pgfqpoint{0.608025in}{0.484444in}}{\pgfqpoint{2.712595in}{1.541287in}}%
\pgfusepath{clip}%
\pgfsetbuttcap%
\pgfsetroundjoin%
\definecolor{currentfill}{rgb}{1.000000,0.498039,0.054902}%
\pgfsetfillcolor{currentfill}%
\pgfsetlinewidth{1.003750pt}%
\definecolor{currentstroke}{rgb}{1.000000,0.498039,0.054902}%
\pgfsetstrokecolor{currentstroke}%
\pgfsetdash{}{0pt}%
\pgfsys@defobject{currentmarker}{\pgfqpoint{-0.020833in}{-0.020833in}}{\pgfqpoint{0.020833in}{0.020833in}}{%
\pgfpathmoveto{\pgfqpoint{0.000000in}{-0.020833in}}%
\pgfpathcurveto{\pgfqpoint{0.005525in}{-0.020833in}}{\pgfqpoint{0.010825in}{-0.018638in}}{\pgfqpoint{0.014731in}{-0.014731in}}%
\pgfpathcurveto{\pgfqpoint{0.018638in}{-0.010825in}}{\pgfqpoint{0.020833in}{-0.005525in}}{\pgfqpoint{0.020833in}{0.000000in}}%
\pgfpathcurveto{\pgfqpoint{0.020833in}{0.005525in}}{\pgfqpoint{0.018638in}{0.010825in}}{\pgfqpoint{0.014731in}{0.014731in}}%
\pgfpathcurveto{\pgfqpoint{0.010825in}{0.018638in}}{\pgfqpoint{0.005525in}{0.020833in}}{\pgfqpoint{0.000000in}{0.020833in}}%
\pgfpathcurveto{\pgfqpoint{-0.005525in}{0.020833in}}{\pgfqpoint{-0.010825in}{0.018638in}}{\pgfqpoint{-0.014731in}{0.014731in}}%
\pgfpathcurveto{\pgfqpoint{-0.018638in}{0.010825in}}{\pgfqpoint{-0.020833in}{0.005525in}}{\pgfqpoint{-0.020833in}{0.000000in}}%
\pgfpathcurveto{\pgfqpoint{-0.020833in}{-0.005525in}}{\pgfqpoint{-0.018638in}{-0.010825in}}{\pgfqpoint{-0.014731in}{-0.014731in}}%
\pgfpathcurveto{\pgfqpoint{-0.010825in}{-0.018638in}}{\pgfqpoint{-0.005525in}{-0.020833in}}{\pgfqpoint{0.000000in}{-0.020833in}}%
\pgfpathlineto{\pgfqpoint{0.000000in}{-0.020833in}}%
\pgfpathclose%
\pgfusepath{stroke,fill}%
}%
\begin{pgfscope}%
\pgfsys@transformshift{0.751062in}{1.931587in}%
\pgfsys@useobject{currentmarker}{}%
\end{pgfscope}%
\begin{pgfscope}%
\pgfsys@transformshift{0.948437in}{1.159628in}%
\pgfsys@useobject{currentmarker}{}%
\end{pgfscope}%
\begin{pgfscope}%
\pgfsys@transformshift{1.145811in}{0.791139in}%
\pgfsys@useobject{currentmarker}{}%
\end{pgfscope}%
\begin{pgfscope}%
\pgfsys@transformshift{1.343186in}{0.657157in}%
\pgfsys@useobject{currentmarker}{}%
\end{pgfscope}%
\begin{pgfscope}%
\pgfsys@transformshift{1.540560in}{0.616372in}%
\pgfsys@useobject{currentmarker}{}%
\end{pgfscope}%
\begin{pgfscope}%
\pgfsys@transformshift{1.737934in}{0.633038in}%
\pgfsys@useobject{currentmarker}{}%
\end{pgfscope}%
\begin{pgfscope}%
\pgfsys@transformshift{1.935309in}{0.601178in}%
\pgfsys@useobject{currentmarker}{}%
\end{pgfscope}%
\begin{pgfscope}%
\pgfsys@transformshift{2.132683in}{0.623023in}%
\pgfsys@useobject{currentmarker}{}%
\end{pgfscope}%
\begin{pgfscope}%
\pgfsys@transformshift{2.330058in}{0.577821in}%
\pgfsys@useobject{currentmarker}{}%
\end{pgfscope}%
\begin{pgfscope}%
\pgfsys@transformshift{2.527432in}{0.615557in}%
\pgfsys@useobject{currentmarker}{}%
\end{pgfscope}%
\begin{pgfscope}%
\pgfsys@transformshift{2.724806in}{0.582474in}%
\pgfsys@useobject{currentmarker}{}%
\end{pgfscope}%
\begin{pgfscope}%
\pgfsys@transformshift{2.922181in}{0.616590in}%
\pgfsys@useobject{currentmarker}{}%
\end{pgfscope}%
\begin{pgfscope}%
\pgfsys@transformshift{3.119555in}{0.584614in}%
\pgfsys@useobject{currentmarker}{}%
\end{pgfscope}%
\end{pgfscope}%
\begin{pgfscope}%
\pgfpathrectangle{\pgfqpoint{0.608025in}{0.484444in}}{\pgfqpoint{2.712595in}{1.541287in}}%
\pgfusepath{clip}%
\pgfsetrectcap%
\pgfsetroundjoin%
\pgfsetlinewidth{1.505625pt}%
\definecolor{currentstroke}{rgb}{0.172549,0.627451,0.172549}%
\pgfsetstrokecolor{currentstroke}%
\pgfsetdash{}{0pt}%
\pgfpathmoveto{\pgfqpoint{0.731325in}{1.955466in}}%
\pgfpathlineto{\pgfqpoint{0.744746in}{1.945271in}}%
\pgfpathlineto{\pgfqpoint{0.749089in}{1.935578in}}%
\pgfpathlineto{\pgfqpoint{0.764879in}{1.883745in}}%
\pgfpathlineto{\pgfqpoint{0.778695in}{1.842266in}}%
\pgfpathlineto{\pgfqpoint{0.795669in}{1.808149in}}%
\pgfpathlineto{\pgfqpoint{0.809485in}{1.794937in}}%
\pgfpathlineto{\pgfqpoint{0.813827in}{1.786038in}}%
\pgfpathlineto{\pgfqpoint{0.814617in}{1.786939in}}%
\pgfpathlineto{\pgfqpoint{0.815801in}{1.786665in}}%
\pgfpathlineto{\pgfqpoint{0.818170in}{1.772703in}}%
\pgfpathlineto{\pgfqpoint{0.822117in}{1.734140in}}%
\pgfpathlineto{\pgfqpoint{0.826854in}{1.690475in}}%
\pgfpathlineto{\pgfqpoint{0.832775in}{1.637663in}}%
\pgfpathlineto{\pgfqpoint{0.839486in}{1.615184in}}%
\pgfpathlineto{\pgfqpoint{0.845407in}{1.583698in}}%
\pgfpathlineto{\pgfqpoint{0.853302in}{1.543774in}}%
\pgfpathlineto{\pgfqpoint{0.855671in}{1.539904in}}%
\pgfpathlineto{\pgfqpoint{0.856066in}{1.541674in}}%
\pgfpathlineto{\pgfqpoint{0.859224in}{1.548548in}}%
\pgfpathlineto{\pgfqpoint{0.870277in}{1.485003in}}%
\pgfpathlineto{\pgfqpoint{0.870671in}{1.488410in}}%
\pgfpathlineto{\pgfqpoint{0.873829in}{1.531750in}}%
\pgfpathlineto{\pgfqpoint{0.874619in}{1.531190in}}%
\pgfpathlineto{\pgfqpoint{0.875408in}{1.530426in}}%
\pgfpathlineto{\pgfqpoint{0.878961in}{1.543461in}}%
\pgfpathlineto{\pgfqpoint{0.882514in}{1.541822in}}%
\pgfpathlineto{\pgfqpoint{0.884487in}{1.532838in}}%
\pgfpathlineto{\pgfqpoint{0.887251in}{1.501577in}}%
\pgfpathlineto{\pgfqpoint{0.887645in}{1.495863in}}%
\pgfpathlineto{\pgfqpoint{0.888830in}{1.499626in}}%
\pgfpathlineto{\pgfqpoint{0.893961in}{1.514815in}}%
\pgfpathlineto{\pgfqpoint{0.894751in}{1.518885in}}%
\pgfpathlineto{\pgfqpoint{0.895540in}{1.517825in}}%
\pgfpathlineto{\pgfqpoint{0.896330in}{1.515476in}}%
\pgfpathlineto{\pgfqpoint{0.902251in}{1.413359in}}%
\pgfpathlineto{\pgfqpoint{0.902646in}{1.415108in}}%
\pgfpathlineto{\pgfqpoint{0.906988in}{1.439330in}}%
\pgfpathlineto{\pgfqpoint{0.907383in}{1.436196in}}%
\pgfpathlineto{\pgfqpoint{0.912120in}{1.370644in}}%
\pgfpathlineto{\pgfqpoint{0.914883in}{1.361779in}}%
\pgfpathlineto{\pgfqpoint{0.915673in}{1.358447in}}%
\pgfpathlineto{\pgfqpoint{0.916067in}{1.361105in}}%
\pgfpathlineto{\pgfqpoint{0.920015in}{1.376662in}}%
\pgfpathlineto{\pgfqpoint{0.920410in}{1.377409in}}%
\pgfpathlineto{\pgfqpoint{0.923962in}{1.318359in}}%
\pgfpathlineto{\pgfqpoint{0.925147in}{1.327400in}}%
\pgfpathlineto{\pgfqpoint{0.927910in}{1.368438in}}%
\pgfpathlineto{\pgfqpoint{0.928699in}{1.378005in}}%
\pgfpathlineto{\pgfqpoint{0.929489in}{1.371666in}}%
\pgfpathlineto{\pgfqpoint{0.935015in}{1.311456in}}%
\pgfpathlineto{\pgfqpoint{0.935410in}{1.316918in}}%
\pgfpathlineto{\pgfqpoint{0.938173in}{1.377714in}}%
\pgfpathlineto{\pgfqpoint{0.939752in}{1.407401in}}%
\pgfpathlineto{\pgfqpoint{0.940542in}{1.407027in}}%
\pgfpathlineto{\pgfqpoint{0.941726in}{1.404301in}}%
\pgfpathlineto{\pgfqpoint{0.942910in}{1.391538in}}%
\pgfpathlineto{\pgfqpoint{0.943700in}{1.394384in}}%
\pgfpathlineto{\pgfqpoint{0.946463in}{1.400026in}}%
\pgfpathlineto{\pgfqpoint{0.946858in}{1.399888in}}%
\pgfpathlineto{\pgfqpoint{0.948042in}{1.397993in}}%
\pgfpathlineto{\pgfqpoint{0.950411in}{1.385429in}}%
\pgfpathlineto{\pgfqpoint{0.952384in}{1.374350in}}%
\pgfpathlineto{\pgfqpoint{0.952779in}{1.375963in}}%
\pgfpathlineto{\pgfqpoint{0.954753in}{1.385428in}}%
\pgfpathlineto{\pgfqpoint{0.956332in}{1.407072in}}%
\pgfpathlineto{\pgfqpoint{0.957121in}{1.401818in}}%
\pgfpathlineto{\pgfqpoint{0.960279in}{1.369028in}}%
\pgfpathlineto{\pgfqpoint{0.966200in}{1.301631in}}%
\pgfpathlineto{\pgfqpoint{0.967385in}{1.309210in}}%
\pgfpathlineto{\pgfqpoint{0.967779in}{1.306538in}}%
\pgfpathlineto{\pgfqpoint{0.970937in}{1.277443in}}%
\pgfpathlineto{\pgfqpoint{0.972122in}{1.279973in}}%
\pgfpathlineto{\pgfqpoint{0.974095in}{1.288962in}}%
\pgfpathlineto{\pgfqpoint{0.974490in}{1.287168in}}%
\pgfpathlineto{\pgfqpoint{0.976069in}{1.280027in}}%
\pgfpathlineto{\pgfqpoint{0.976859in}{1.283418in}}%
\pgfpathlineto{\pgfqpoint{0.979622in}{1.338353in}}%
\pgfpathlineto{\pgfqpoint{0.980806in}{1.333309in}}%
\pgfpathlineto{\pgfqpoint{0.983175in}{1.265348in}}%
\pgfpathlineto{\pgfqpoint{0.984754in}{1.276358in}}%
\pgfpathlineto{\pgfqpoint{0.987912in}{1.308204in}}%
\pgfpathlineto{\pgfqpoint{0.988701in}{1.307212in}}%
\pgfpathlineto{\pgfqpoint{0.989885in}{1.302575in}}%
\pgfpathlineto{\pgfqpoint{0.991464in}{1.324385in}}%
\pgfpathlineto{\pgfqpoint{0.992649in}{1.319483in}}%
\pgfpathlineto{\pgfqpoint{0.996596in}{1.299788in}}%
\pgfpathlineto{\pgfqpoint{1.000938in}{1.279158in}}%
\pgfpathlineto{\pgfqpoint{1.002912in}{1.239459in}}%
\pgfpathlineto{\pgfqpoint{1.003702in}{1.242224in}}%
\pgfpathlineto{\pgfqpoint{1.006070in}{1.300463in}}%
\pgfpathlineto{\pgfqpoint{1.006860in}{1.288368in}}%
\pgfpathlineto{\pgfqpoint{1.008833in}{1.281119in}}%
\pgfpathlineto{\pgfqpoint{1.009228in}{1.282748in}}%
\pgfpathlineto{\pgfqpoint{1.009623in}{1.278678in}}%
\pgfpathlineto{\pgfqpoint{1.015149in}{1.225402in}}%
\pgfpathlineto{\pgfqpoint{1.017518in}{1.213328in}}%
\pgfpathlineto{\pgfqpoint{1.019492in}{1.210232in}}%
\pgfpathlineto{\pgfqpoint{1.019886in}{1.209241in}}%
\pgfpathlineto{\pgfqpoint{1.020281in}{1.212393in}}%
\pgfpathlineto{\pgfqpoint{1.025808in}{1.270485in}}%
\pgfpathlineto{\pgfqpoint{1.026992in}{1.267995in}}%
\pgfpathlineto{\pgfqpoint{1.031334in}{1.189848in}}%
\pgfpathlineto{\pgfqpoint{1.035676in}{1.135078in}}%
\pgfpathlineto{\pgfqpoint{1.036466in}{1.137173in}}%
\pgfpathlineto{\pgfqpoint{1.040808in}{1.174589in}}%
\pgfpathlineto{\pgfqpoint{1.045940in}{1.258144in}}%
\pgfpathlineto{\pgfqpoint{1.047519in}{1.270097in}}%
\pgfpathlineto{\pgfqpoint{1.049098in}{1.283253in}}%
\pgfpathlineto{\pgfqpoint{1.049492in}{1.280533in}}%
\pgfpathlineto{\pgfqpoint{1.052650in}{1.238777in}}%
\pgfpathlineto{\pgfqpoint{1.053045in}{1.239835in}}%
\pgfpathlineto{\pgfqpoint{1.054229in}{1.260332in}}%
\pgfpathlineto{\pgfqpoint{1.055019in}{1.254064in}}%
\pgfpathlineto{\pgfqpoint{1.058572in}{1.203883in}}%
\pgfpathlineto{\pgfqpoint{1.059361in}{1.213322in}}%
\pgfpathlineto{\pgfqpoint{1.060545in}{1.221636in}}%
\pgfpathlineto{\pgfqpoint{1.060940in}{1.219609in}}%
\pgfpathlineto{\pgfqpoint{1.061730in}{1.207233in}}%
\pgfpathlineto{\pgfqpoint{1.062519in}{1.211341in}}%
\pgfpathlineto{\pgfqpoint{1.066467in}{1.241190in}}%
\pgfpathlineto{\pgfqpoint{1.067256in}{1.234446in}}%
\pgfpathlineto{\pgfqpoint{1.069625in}{1.205031in}}%
\pgfpathlineto{\pgfqpoint{1.070019in}{1.209744in}}%
\pgfpathlineto{\pgfqpoint{1.077125in}{1.275297in}}%
\pgfpathlineto{\pgfqpoint{1.078309in}{1.282450in}}%
\pgfpathlineto{\pgfqpoint{1.078704in}{1.280105in}}%
\pgfpathlineto{\pgfqpoint{1.082257in}{1.226592in}}%
\pgfpathlineto{\pgfqpoint{1.085415in}{1.132346in}}%
\pgfpathlineto{\pgfqpoint{1.086599in}{1.135680in}}%
\pgfpathlineto{\pgfqpoint{1.087388in}{1.137720in}}%
\pgfpathlineto{\pgfqpoint{1.088967in}{1.158593in}}%
\pgfpathlineto{\pgfqpoint{1.089757in}{1.153171in}}%
\pgfpathlineto{\pgfqpoint{1.092125in}{1.134164in}}%
\pgfpathlineto{\pgfqpoint{1.092520in}{1.137744in}}%
\pgfpathlineto{\pgfqpoint{1.092915in}{1.144295in}}%
\pgfpathlineto{\pgfqpoint{1.093310in}{1.142681in}}%
\pgfpathlineto{\pgfqpoint{1.094494in}{1.117072in}}%
\pgfpathlineto{\pgfqpoint{1.095283in}{1.121784in}}%
\pgfpathlineto{\pgfqpoint{1.095678in}{1.126491in}}%
\pgfpathlineto{\pgfqpoint{1.096468in}{1.118051in}}%
\pgfpathlineto{\pgfqpoint{1.097257in}{1.109453in}}%
\pgfpathlineto{\pgfqpoint{1.097652in}{1.119893in}}%
\pgfpathlineto{\pgfqpoint{1.103968in}{1.197823in}}%
\pgfpathlineto{\pgfqpoint{1.104363in}{1.197924in}}%
\pgfpathlineto{\pgfqpoint{1.105152in}{1.209944in}}%
\pgfpathlineto{\pgfqpoint{1.106336in}{1.204442in}}%
\pgfpathlineto{\pgfqpoint{1.106731in}{1.203576in}}%
\pgfpathlineto{\pgfqpoint{1.107126in}{1.206261in}}%
\pgfpathlineto{\pgfqpoint{1.112258in}{1.228341in}}%
\pgfpathlineto{\pgfqpoint{1.113837in}{1.199574in}}%
\pgfpathlineto{\pgfqpoint{1.114626in}{1.204908in}}%
\pgfpathlineto{\pgfqpoint{1.116600in}{1.230463in}}%
\pgfpathlineto{\pgfqpoint{1.116995in}{1.218031in}}%
\pgfpathlineto{\pgfqpoint{1.119758in}{1.173350in}}%
\pgfpathlineto{\pgfqpoint{1.124890in}{1.138553in}}%
\pgfpathlineto{\pgfqpoint{1.125284in}{1.146112in}}%
\pgfpathlineto{\pgfqpoint{1.126863in}{1.177162in}}%
\pgfpathlineto{\pgfqpoint{1.128047in}{1.175517in}}%
\pgfpathlineto{\pgfqpoint{1.129232in}{1.167075in}}%
\pgfpathlineto{\pgfqpoint{1.133179in}{1.095382in}}%
\pgfpathlineto{\pgfqpoint{1.133969in}{1.105008in}}%
\pgfpathlineto{\pgfqpoint{1.135942in}{1.138742in}}%
\pgfpathlineto{\pgfqpoint{1.137127in}{1.132798in}}%
\pgfpathlineto{\pgfqpoint{1.137521in}{1.130125in}}%
\pgfpathlineto{\pgfqpoint{1.137916in}{1.133996in}}%
\pgfpathlineto{\pgfqpoint{1.138706in}{1.132591in}}%
\pgfpathlineto{\pgfqpoint{1.140285in}{1.146314in}}%
\pgfpathlineto{\pgfqpoint{1.140679in}{1.143616in}}%
\pgfpathlineto{\pgfqpoint{1.144627in}{1.110219in}}%
\pgfpathlineto{\pgfqpoint{1.145416in}{1.118138in}}%
\pgfpathlineto{\pgfqpoint{1.150548in}{1.166042in}}%
\pgfpathlineto{\pgfqpoint{1.150943in}{1.165033in}}%
\pgfpathlineto{\pgfqpoint{1.153311in}{1.153366in}}%
\pgfpathlineto{\pgfqpoint{1.157654in}{1.101778in}}%
\pgfpathlineto{\pgfqpoint{1.158838in}{1.110870in}}%
\pgfpathlineto{\pgfqpoint{1.159233in}{1.118317in}}%
\pgfpathlineto{\pgfqpoint{1.160022in}{1.117898in}}%
\pgfpathlineto{\pgfqpoint{1.163180in}{1.077244in}}%
\pgfpathlineto{\pgfqpoint{1.163970in}{1.078381in}}%
\pgfpathlineto{\pgfqpoint{1.165154in}{1.084077in}}%
\pgfpathlineto{\pgfqpoint{1.168707in}{1.128564in}}%
\pgfpathlineto{\pgfqpoint{1.170286in}{1.131990in}}%
\pgfpathlineto{\pgfqpoint{1.170680in}{1.130012in}}%
\pgfpathlineto{\pgfqpoint{1.171075in}{1.127992in}}%
\pgfpathlineto{\pgfqpoint{1.171470in}{1.130457in}}%
\pgfpathlineto{\pgfqpoint{1.174628in}{1.146762in}}%
\pgfpathlineto{\pgfqpoint{1.175023in}{1.145200in}}%
\pgfpathlineto{\pgfqpoint{1.176602in}{1.134723in}}%
\pgfpathlineto{\pgfqpoint{1.176996in}{1.135329in}}%
\pgfpathlineto{\pgfqpoint{1.179760in}{1.157302in}}%
\pgfpathlineto{\pgfqpoint{1.180154in}{1.159632in}}%
\pgfpathlineto{\pgfqpoint{1.180549in}{1.155989in}}%
\pgfpathlineto{\pgfqpoint{1.183312in}{1.134252in}}%
\pgfpathlineto{\pgfqpoint{1.184891in}{1.134862in}}%
\pgfpathlineto{\pgfqpoint{1.187655in}{1.096920in}}%
\pgfpathlineto{\pgfqpoint{1.188049in}{1.095206in}}%
\pgfpathlineto{\pgfqpoint{1.189234in}{1.068213in}}%
\pgfpathlineto{\pgfqpoint{1.190418in}{1.074707in}}%
\pgfpathlineto{\pgfqpoint{1.196734in}{1.006174in}}%
\pgfpathlineto{\pgfqpoint{1.197523in}{1.014287in}}%
\pgfpathlineto{\pgfqpoint{1.201076in}{1.056602in}}%
\pgfpathlineto{\pgfqpoint{1.205024in}{1.081068in}}%
\pgfpathlineto{\pgfqpoint{1.206208in}{1.097999in}}%
\pgfpathlineto{\pgfqpoint{1.207392in}{1.096744in}}%
\pgfpathlineto{\pgfqpoint{1.211734in}{1.138759in}}%
\pgfpathlineto{\pgfqpoint{1.213313in}{1.128950in}}%
\pgfpathlineto{\pgfqpoint{1.214103in}{1.131782in}}%
\pgfpathlineto{\pgfqpoint{1.217261in}{1.155280in}}%
\pgfpathlineto{\pgfqpoint{1.218445in}{1.153694in}}%
\pgfpathlineto{\pgfqpoint{1.220024in}{1.132956in}}%
\pgfpathlineto{\pgfqpoint{1.221603in}{1.105419in}}%
\pgfpathlineto{\pgfqpoint{1.221998in}{1.108222in}}%
\pgfpathlineto{\pgfqpoint{1.222392in}{1.111687in}}%
\pgfpathlineto{\pgfqpoint{1.222787in}{1.106858in}}%
\pgfpathlineto{\pgfqpoint{1.226340in}{1.070231in}}%
\pgfpathlineto{\pgfqpoint{1.226735in}{1.079109in}}%
\pgfpathlineto{\pgfqpoint{1.227129in}{1.082145in}}%
\pgfpathlineto{\pgfqpoint{1.228314in}{1.080505in}}%
\pgfpathlineto{\pgfqpoint{1.228708in}{1.080766in}}%
\pgfpathlineto{\pgfqpoint{1.229498in}{1.086325in}}%
\pgfpathlineto{\pgfqpoint{1.229893in}{1.084990in}}%
\pgfpathlineto{\pgfqpoint{1.230682in}{1.076318in}}%
\pgfpathlineto{\pgfqpoint{1.231472in}{1.079763in}}%
\pgfpathlineto{\pgfqpoint{1.232261in}{1.086505in}}%
\pgfpathlineto{\pgfqpoint{1.232656in}{1.081895in}}%
\pgfpathlineto{\pgfqpoint{1.233840in}{1.062122in}}%
\pgfpathlineto{\pgfqpoint{1.234235in}{1.067841in}}%
\pgfpathlineto{\pgfqpoint{1.236603in}{1.090745in}}%
\pgfpathlineto{\pgfqpoint{1.238577in}{1.070601in}}%
\pgfpathlineto{\pgfqpoint{1.239367in}{1.075283in}}%
\pgfpathlineto{\pgfqpoint{1.242919in}{1.097156in}}%
\pgfpathlineto{\pgfqpoint{1.243314in}{1.096914in}}%
\pgfpathlineto{\pgfqpoint{1.247656in}{1.061629in}}%
\pgfpathlineto{\pgfqpoint{1.248841in}{1.048835in}}%
\pgfpathlineto{\pgfqpoint{1.249630in}{1.051177in}}%
\pgfpathlineto{\pgfqpoint{1.250025in}{1.051718in}}%
\pgfpathlineto{\pgfqpoint{1.253183in}{0.988409in}}%
\pgfpathlineto{\pgfqpoint{1.254367in}{0.993923in}}%
\pgfpathlineto{\pgfqpoint{1.256736in}{0.998921in}}%
\pgfpathlineto{\pgfqpoint{1.257525in}{0.988371in}}%
\pgfpathlineto{\pgfqpoint{1.257920in}{0.993832in}}%
\pgfpathlineto{\pgfqpoint{1.263446in}{1.048303in}}%
\pgfpathlineto{\pgfqpoint{1.266210in}{1.070795in}}%
\pgfpathlineto{\pgfqpoint{1.268578in}{1.088444in}}%
\pgfpathlineto{\pgfqpoint{1.270157in}{1.109748in}}%
\pgfpathlineto{\pgfqpoint{1.274105in}{1.166948in}}%
\pgfpathlineto{\pgfqpoint{1.274499in}{1.170861in}}%
\pgfpathlineto{\pgfqpoint{1.275289in}{1.165251in}}%
\pgfpathlineto{\pgfqpoint{1.278447in}{1.122246in}}%
\pgfpathlineto{\pgfqpoint{1.281605in}{1.087170in}}%
\pgfpathlineto{\pgfqpoint{1.282394in}{1.085037in}}%
\pgfpathlineto{\pgfqpoint{1.283973in}{1.070737in}}%
\pgfpathlineto{\pgfqpoint{1.284368in}{1.071836in}}%
\pgfpathlineto{\pgfqpoint{1.285158in}{1.065889in}}%
\pgfpathlineto{\pgfqpoint{1.285552in}{1.073897in}}%
\pgfpathlineto{\pgfqpoint{1.288710in}{1.115202in}}%
\pgfpathlineto{\pgfqpoint{1.289105in}{1.112331in}}%
\pgfpathlineto{\pgfqpoint{1.293052in}{1.035431in}}%
\pgfpathlineto{\pgfqpoint{1.293447in}{1.039806in}}%
\pgfpathlineto{\pgfqpoint{1.297000in}{1.078639in}}%
\pgfpathlineto{\pgfqpoint{1.297789in}{1.071735in}}%
\pgfpathlineto{\pgfqpoint{1.300553in}{1.038980in}}%
\pgfpathlineto{\pgfqpoint{1.301342in}{1.039080in}}%
\pgfpathlineto{\pgfqpoint{1.301737in}{1.039758in}}%
\pgfpathlineto{\pgfqpoint{1.304500in}{1.068796in}}%
\pgfpathlineto{\pgfqpoint{1.305684in}{1.069557in}}%
\pgfpathlineto{\pgfqpoint{1.308842in}{1.035672in}}%
\pgfpathlineto{\pgfqpoint{1.309237in}{1.035332in}}%
\pgfpathlineto{\pgfqpoint{1.313579in}{1.004530in}}%
\pgfpathlineto{\pgfqpoint{1.315158in}{0.989538in}}%
\pgfpathlineto{\pgfqpoint{1.315948in}{0.965162in}}%
\pgfpathlineto{\pgfqpoint{1.316737in}{0.976843in}}%
\pgfpathlineto{\pgfqpoint{1.317132in}{0.981845in}}%
\pgfpathlineto{\pgfqpoint{1.317922in}{0.977083in}}%
\pgfpathlineto{\pgfqpoint{1.318316in}{0.977168in}}%
\pgfpathlineto{\pgfqpoint{1.318711in}{0.974082in}}%
\pgfpathlineto{\pgfqpoint{1.319106in}{0.975035in}}%
\pgfpathlineto{\pgfqpoint{1.319895in}{0.988156in}}%
\pgfpathlineto{\pgfqpoint{1.320685in}{0.983181in}}%
\pgfpathlineto{\pgfqpoint{1.322659in}{0.962521in}}%
\pgfpathlineto{\pgfqpoint{1.323053in}{0.969083in}}%
\pgfpathlineto{\pgfqpoint{1.325422in}{0.987720in}}%
\pgfpathlineto{\pgfqpoint{1.326211in}{0.985978in}}%
\pgfpathlineto{\pgfqpoint{1.328975in}{1.057858in}}%
\pgfpathlineto{\pgfqpoint{1.329369in}{1.057094in}}%
\pgfpathlineto{\pgfqpoint{1.330948in}{1.028501in}}%
\pgfpathlineto{\pgfqpoint{1.332133in}{1.000835in}}%
\pgfpathlineto{\pgfqpoint{1.332922in}{1.010920in}}%
\pgfpathlineto{\pgfqpoint{1.336080in}{1.060907in}}%
\pgfpathlineto{\pgfqpoint{1.336870in}{1.062092in}}%
\pgfpathlineto{\pgfqpoint{1.339238in}{1.080167in}}%
\pgfpathlineto{\pgfqpoint{1.341212in}{1.082172in}}%
\pgfpathlineto{\pgfqpoint{1.342001in}{1.081669in}}%
\pgfpathlineto{\pgfqpoint{1.345554in}{1.051899in}}%
\pgfpathlineto{\pgfqpoint{1.346738in}{1.065193in}}%
\pgfpathlineto{\pgfqpoint{1.347133in}{1.062252in}}%
\pgfpathlineto{\pgfqpoint{1.351475in}{0.994606in}}%
\pgfpathlineto{\pgfqpoint{1.351870in}{0.994736in}}%
\pgfpathlineto{\pgfqpoint{1.352265in}{1.000128in}}%
\pgfpathlineto{\pgfqpoint{1.352660in}{0.994770in}}%
\pgfpathlineto{\pgfqpoint{1.356212in}{0.958465in}}%
\pgfpathlineto{\pgfqpoint{1.356607in}{0.958442in}}%
\pgfpathlineto{\pgfqpoint{1.357397in}{0.969978in}}%
\pgfpathlineto{\pgfqpoint{1.358186in}{0.961058in}}%
\pgfpathlineto{\pgfqpoint{1.358581in}{0.956022in}}%
\pgfpathlineto{\pgfqpoint{1.359370in}{0.959910in}}%
\pgfpathlineto{\pgfqpoint{1.362923in}{0.986431in}}%
\pgfpathlineto{\pgfqpoint{1.364897in}{0.997345in}}%
\pgfpathlineto{\pgfqpoint{1.365292in}{0.997069in}}%
\pgfpathlineto{\pgfqpoint{1.366081in}{0.994230in}}%
\pgfpathlineto{\pgfqpoint{1.366476in}{0.998170in}}%
\pgfpathlineto{\pgfqpoint{1.366871in}{0.998620in}}%
\pgfpathlineto{\pgfqpoint{1.367265in}{0.996818in}}%
\pgfpathlineto{\pgfqpoint{1.368055in}{0.997751in}}%
\pgfpathlineto{\pgfqpoint{1.369239in}{0.984590in}}%
\pgfpathlineto{\pgfqpoint{1.373581in}{0.943935in}}%
\pgfpathlineto{\pgfqpoint{1.374371in}{0.951421in}}%
\pgfpathlineto{\pgfqpoint{1.376739in}{0.983480in}}%
\pgfpathlineto{\pgfqpoint{1.377923in}{0.974378in}}%
\pgfpathlineto{\pgfqpoint{1.379108in}{0.966806in}}%
\pgfpathlineto{\pgfqpoint{1.380687in}{0.957910in}}%
\pgfpathlineto{\pgfqpoint{1.381081in}{0.961433in}}%
\pgfpathlineto{\pgfqpoint{1.381476in}{0.963980in}}%
\pgfpathlineto{\pgfqpoint{1.381871in}{0.957619in}}%
\pgfpathlineto{\pgfqpoint{1.385818in}{0.914412in}}%
\pgfpathlineto{\pgfqpoint{1.387003in}{0.912997in}}%
\pgfpathlineto{\pgfqpoint{1.388187in}{0.938726in}}%
\pgfpathlineto{\pgfqpoint{1.389371in}{0.927657in}}%
\pgfpathlineto{\pgfqpoint{1.390950in}{0.893246in}}%
\pgfpathlineto{\pgfqpoint{1.392134in}{0.906351in}}%
\pgfpathlineto{\pgfqpoint{1.392529in}{0.905350in}}%
\pgfpathlineto{\pgfqpoint{1.402398in}{1.004619in}}%
\pgfpathlineto{\pgfqpoint{1.402793in}{1.004145in}}%
\pgfpathlineto{\pgfqpoint{1.406740in}{0.973083in}}%
\pgfpathlineto{\pgfqpoint{1.407530in}{0.976311in}}%
\pgfpathlineto{\pgfqpoint{1.407924in}{0.975978in}}%
\pgfpathlineto{\pgfqpoint{1.411082in}{0.941877in}}%
\pgfpathlineto{\pgfqpoint{1.411477in}{0.943113in}}%
\pgfpathlineto{\pgfqpoint{1.413846in}{0.947055in}}%
\pgfpathlineto{\pgfqpoint{1.414240in}{0.946838in}}%
\pgfpathlineto{\pgfqpoint{1.414635in}{0.945993in}}%
\pgfpathlineto{\pgfqpoint{1.417398in}{0.916182in}}%
\pgfpathlineto{\pgfqpoint{1.417793in}{0.918502in}}%
\pgfpathlineto{\pgfqpoint{1.420556in}{0.950020in}}%
\pgfpathlineto{\pgfqpoint{1.420951in}{0.950827in}}%
\pgfpathlineto{\pgfqpoint{1.421346in}{0.956887in}}%
\pgfpathlineto{\pgfqpoint{1.422135in}{0.947079in}}%
\pgfpathlineto{\pgfqpoint{1.423714in}{0.930851in}}%
\pgfpathlineto{\pgfqpoint{1.424899in}{0.932594in}}%
\pgfpathlineto{\pgfqpoint{1.425293in}{0.932940in}}%
\pgfpathlineto{\pgfqpoint{1.426083in}{0.921510in}}%
\pgfpathlineto{\pgfqpoint{1.426478in}{0.930699in}}%
\pgfpathlineto{\pgfqpoint{1.429241in}{0.976129in}}%
\pgfpathlineto{\pgfqpoint{1.433583in}{0.938886in}}%
\pgfpathlineto{\pgfqpoint{1.439899in}{0.978764in}}%
\pgfpathlineto{\pgfqpoint{1.441083in}{0.975997in}}%
\pgfpathlineto{\pgfqpoint{1.445820in}{0.928397in}}%
\pgfpathlineto{\pgfqpoint{1.446215in}{0.930796in}}%
\pgfpathlineto{\pgfqpoint{1.446610in}{0.931695in}}%
\pgfpathlineto{\pgfqpoint{1.448978in}{0.905220in}}%
\pgfpathlineto{\pgfqpoint{1.450557in}{0.879571in}}%
\pgfpathlineto{\pgfqpoint{1.450952in}{0.887253in}}%
\pgfpathlineto{\pgfqpoint{1.452926in}{0.939664in}}%
\pgfpathlineto{\pgfqpoint{1.453715in}{0.938293in}}%
\pgfpathlineto{\pgfqpoint{1.454110in}{0.934062in}}%
\pgfpathlineto{\pgfqpoint{1.454900in}{0.935489in}}%
\pgfpathlineto{\pgfqpoint{1.455689in}{0.942268in}}%
\pgfpathlineto{\pgfqpoint{1.456873in}{0.939829in}}%
\pgfpathlineto{\pgfqpoint{1.458058in}{0.928642in}}%
\pgfpathlineto{\pgfqpoint{1.458847in}{0.938179in}}%
\pgfpathlineto{\pgfqpoint{1.460031in}{0.941165in}}%
\pgfpathlineto{\pgfqpoint{1.464373in}{0.999222in}}%
\pgfpathlineto{\pgfqpoint{1.464768in}{0.999043in}}%
\pgfpathlineto{\pgfqpoint{1.465163in}{0.998748in}}%
\pgfpathlineto{\pgfqpoint{1.468321in}{0.951854in}}%
\pgfpathlineto{\pgfqpoint{1.468716in}{0.952942in}}%
\pgfpathlineto{\pgfqpoint{1.470689in}{0.956346in}}%
\pgfpathlineto{\pgfqpoint{1.475032in}{0.922222in}}%
\pgfpathlineto{\pgfqpoint{1.477005in}{0.901011in}}%
\pgfpathlineto{\pgfqpoint{1.478584in}{0.926125in}}%
\pgfpathlineto{\pgfqpoint{1.479769in}{0.920819in}}%
\pgfpathlineto{\pgfqpoint{1.480163in}{0.918822in}}%
\pgfpathlineto{\pgfqpoint{1.480558in}{0.921233in}}%
\pgfpathlineto{\pgfqpoint{1.484506in}{0.941459in}}%
\pgfpathlineto{\pgfqpoint{1.484900in}{0.934317in}}%
\pgfpathlineto{\pgfqpoint{1.487664in}{0.904782in}}%
\pgfpathlineto{\pgfqpoint{1.488848in}{0.895293in}}%
\pgfpathlineto{\pgfqpoint{1.489243in}{0.896054in}}%
\pgfpathlineto{\pgfqpoint{1.490032in}{0.910659in}}%
\pgfpathlineto{\pgfqpoint{1.490822in}{0.902341in}}%
\pgfpathlineto{\pgfqpoint{1.495164in}{0.879122in}}%
\pgfpathlineto{\pgfqpoint{1.495559in}{0.881010in}}%
\pgfpathlineto{\pgfqpoint{1.498322in}{0.915379in}}%
\pgfpathlineto{\pgfqpoint{1.498717in}{0.913920in}}%
\pgfpathlineto{\pgfqpoint{1.499111in}{0.912036in}}%
\pgfpathlineto{\pgfqpoint{1.499901in}{0.915446in}}%
\pgfpathlineto{\pgfqpoint{1.502664in}{0.930977in}}%
\pgfpathlineto{\pgfqpoint{1.503454in}{0.924028in}}%
\pgfpathlineto{\pgfqpoint{1.504243in}{0.916115in}}%
\pgfpathlineto{\pgfqpoint{1.505033in}{0.919092in}}%
\pgfpathlineto{\pgfqpoint{1.505822in}{0.927763in}}%
\pgfpathlineto{\pgfqpoint{1.509375in}{0.961611in}}%
\pgfpathlineto{\pgfqpoint{1.509770in}{0.963390in}}%
\pgfpathlineto{\pgfqpoint{1.510559in}{0.960081in}}%
\pgfpathlineto{\pgfqpoint{1.514901in}{0.895397in}}%
\pgfpathlineto{\pgfqpoint{1.515296in}{0.900015in}}%
\pgfpathlineto{\pgfqpoint{1.517665in}{0.953474in}}%
\pgfpathlineto{\pgfqpoint{1.518849in}{0.929663in}}%
\pgfpathlineto{\pgfqpoint{1.520033in}{0.921042in}}%
\pgfpathlineto{\pgfqpoint{1.520428in}{0.923614in}}%
\pgfpathlineto{\pgfqpoint{1.522402in}{0.950685in}}%
\pgfpathlineto{\pgfqpoint{1.523191in}{0.944195in}}%
\pgfpathlineto{\pgfqpoint{1.526349in}{0.918123in}}%
\pgfpathlineto{\pgfqpoint{1.526744in}{0.921610in}}%
\pgfpathlineto{\pgfqpoint{1.530691in}{0.971622in}}%
\pgfpathlineto{\pgfqpoint{1.535034in}{0.920438in}}%
\pgfpathlineto{\pgfqpoint{1.535428in}{0.922887in}}%
\pgfpathlineto{\pgfqpoint{1.535823in}{0.924775in}}%
\pgfpathlineto{\pgfqpoint{1.536218in}{0.923588in}}%
\pgfpathlineto{\pgfqpoint{1.538981in}{0.908443in}}%
\pgfpathlineto{\pgfqpoint{1.539376in}{0.911312in}}%
\pgfpathlineto{\pgfqpoint{1.540165in}{0.908038in}}%
\pgfpathlineto{\pgfqpoint{1.541744in}{0.901562in}}%
\pgfpathlineto{\pgfqpoint{1.544113in}{0.866355in}}%
\pgfpathlineto{\pgfqpoint{1.544902in}{0.867167in}}%
\pgfpathlineto{\pgfqpoint{1.545297in}{0.865818in}}%
\pgfpathlineto{\pgfqpoint{1.546086in}{0.868359in}}%
\pgfpathlineto{\pgfqpoint{1.547665in}{0.880643in}}%
\pgfpathlineto{\pgfqpoint{1.548060in}{0.877861in}}%
\pgfpathlineto{\pgfqpoint{1.548455in}{0.873412in}}%
\pgfpathlineto{\pgfqpoint{1.549244in}{0.875277in}}%
\pgfpathlineto{\pgfqpoint{1.550034in}{0.890870in}}%
\pgfpathlineto{\pgfqpoint{1.550823in}{0.884076in}}%
\pgfpathlineto{\pgfqpoint{1.551218in}{0.879294in}}%
\pgfpathlineto{\pgfqpoint{1.551613in}{0.886338in}}%
\pgfpathlineto{\pgfqpoint{1.554376in}{0.920192in}}%
\pgfpathlineto{\pgfqpoint{1.554771in}{0.913840in}}%
\pgfpathlineto{\pgfqpoint{1.557929in}{0.878540in}}%
\pgfpathlineto{\pgfqpoint{1.558718in}{0.874606in}}%
\pgfpathlineto{\pgfqpoint{1.559113in}{0.879108in}}%
\pgfpathlineto{\pgfqpoint{1.559508in}{0.883477in}}%
\pgfpathlineto{\pgfqpoint{1.560297in}{0.877010in}}%
\pgfpathlineto{\pgfqpoint{1.561482in}{0.867614in}}%
\pgfpathlineto{\pgfqpoint{1.562271in}{0.872207in}}%
\pgfpathlineto{\pgfqpoint{1.564640in}{0.879251in}}%
\pgfpathlineto{\pgfqpoint{1.565429in}{0.877224in}}%
\pgfpathlineto{\pgfqpoint{1.567403in}{0.913093in}}%
\pgfpathlineto{\pgfqpoint{1.568192in}{0.912856in}}%
\pgfpathlineto{\pgfqpoint{1.568587in}{0.914216in}}%
\pgfpathlineto{\pgfqpoint{1.568982in}{0.910069in}}%
\pgfpathlineto{\pgfqpoint{1.572140in}{0.890725in}}%
\pgfpathlineto{\pgfqpoint{1.576087in}{0.917446in}}%
\pgfpathlineto{\pgfqpoint{1.577272in}{0.931165in}}%
\pgfpathlineto{\pgfqpoint{1.578061in}{0.925844in}}%
\pgfpathlineto{\pgfqpoint{1.579245in}{0.931407in}}%
\pgfpathlineto{\pgfqpoint{1.582798in}{0.953950in}}%
\pgfpathlineto{\pgfqpoint{1.583588in}{0.954534in}}%
\pgfpathlineto{\pgfqpoint{1.583982in}{0.948737in}}%
\pgfpathlineto{\pgfqpoint{1.587140in}{0.922214in}}%
\pgfpathlineto{\pgfqpoint{1.588719in}{0.940090in}}%
\pgfpathlineto{\pgfqpoint{1.589114in}{0.933346in}}%
\pgfpathlineto{\pgfqpoint{1.590693in}{0.909114in}}%
\pgfpathlineto{\pgfqpoint{1.591877in}{0.913261in}}%
\pgfpathlineto{\pgfqpoint{1.592667in}{0.919327in}}%
\pgfpathlineto{\pgfqpoint{1.593062in}{0.916780in}}%
\pgfpathlineto{\pgfqpoint{1.595035in}{0.898398in}}%
\pgfpathlineto{\pgfqpoint{1.595825in}{0.903942in}}%
\pgfpathlineto{\pgfqpoint{1.596220in}{0.909630in}}%
\pgfpathlineto{\pgfqpoint{1.597009in}{0.904160in}}%
\pgfpathlineto{\pgfqpoint{1.598193in}{0.900272in}}%
\pgfpathlineto{\pgfqpoint{1.598588in}{0.901340in}}%
\pgfpathlineto{\pgfqpoint{1.600167in}{0.905618in}}%
\pgfpathlineto{\pgfqpoint{1.603720in}{0.877735in}}%
\pgfpathlineto{\pgfqpoint{1.608062in}{0.853199in}}%
\pgfpathlineto{\pgfqpoint{1.608457in}{0.857519in}}%
\pgfpathlineto{\pgfqpoint{1.608852in}{0.853357in}}%
\pgfpathlineto{\pgfqpoint{1.610036in}{0.838151in}}%
\pgfpathlineto{\pgfqpoint{1.610431in}{0.840115in}}%
\pgfpathlineto{\pgfqpoint{1.611615in}{0.871253in}}%
\pgfpathlineto{\pgfqpoint{1.612799in}{0.861605in}}%
\pgfpathlineto{\pgfqpoint{1.613194in}{0.861611in}}%
\pgfpathlineto{\pgfqpoint{1.614378in}{0.877684in}}%
\pgfpathlineto{\pgfqpoint{1.615168in}{0.868354in}}%
\pgfpathlineto{\pgfqpoint{1.617141in}{0.839289in}}%
\pgfpathlineto{\pgfqpoint{1.617536in}{0.839679in}}%
\pgfpathlineto{\pgfqpoint{1.619115in}{0.847518in}}%
\pgfpathlineto{\pgfqpoint{1.623063in}{0.903223in}}%
\pgfpathlineto{\pgfqpoint{1.623852in}{0.895812in}}%
\pgfpathlineto{\pgfqpoint{1.627405in}{0.861154in}}%
\pgfpathlineto{\pgfqpoint{1.627799in}{0.861643in}}%
\pgfpathlineto{\pgfqpoint{1.629378in}{0.890297in}}%
\pgfpathlineto{\pgfqpoint{1.629773in}{0.886388in}}%
\pgfpathlineto{\pgfqpoint{1.630168in}{0.882034in}}%
\pgfpathlineto{\pgfqpoint{1.630957in}{0.887293in}}%
\pgfpathlineto{\pgfqpoint{1.633326in}{0.924041in}}%
\pgfpathlineto{\pgfqpoint{1.633721in}{0.917800in}}%
\pgfpathlineto{\pgfqpoint{1.636879in}{0.882660in}}%
\pgfpathlineto{\pgfqpoint{1.638063in}{0.870156in}}%
\pgfpathlineto{\pgfqpoint{1.640037in}{0.846138in}}%
\pgfpathlineto{\pgfqpoint{1.640431in}{0.848954in}}%
\pgfpathlineto{\pgfqpoint{1.643195in}{0.878923in}}%
\pgfpathlineto{\pgfqpoint{1.645958in}{0.906518in}}%
\pgfpathlineto{\pgfqpoint{1.649116in}{0.864507in}}%
\pgfpathlineto{\pgfqpoint{1.649511in}{0.873099in}}%
\pgfpathlineto{\pgfqpoint{1.651484in}{0.904953in}}%
\pgfpathlineto{\pgfqpoint{1.652274in}{0.903986in}}%
\pgfpathlineto{\pgfqpoint{1.653458in}{0.896821in}}%
\pgfpathlineto{\pgfqpoint{1.657011in}{0.862040in}}%
\pgfpathlineto{\pgfqpoint{1.657800in}{0.869716in}}%
\pgfpathlineto{\pgfqpoint{1.660169in}{0.875720in}}%
\pgfpathlineto{\pgfqpoint{1.658590in}{0.869044in}}%
\pgfpathlineto{\pgfqpoint{1.660958in}{0.874787in}}%
\pgfpathlineto{\pgfqpoint{1.662932in}{0.881124in}}%
\pgfpathlineto{\pgfqpoint{1.664511in}{0.867798in}}%
\pgfpathlineto{\pgfqpoint{1.665695in}{0.871301in}}%
\pgfpathlineto{\pgfqpoint{1.666485in}{0.868651in}}%
\pgfpathlineto{\pgfqpoint{1.667669in}{0.884865in}}%
\pgfpathlineto{\pgfqpoint{1.669248in}{0.901380in}}%
\pgfpathlineto{\pgfqpoint{1.669643in}{0.897824in}}%
\pgfpathlineto{\pgfqpoint{1.673590in}{0.845147in}}%
\pgfpathlineto{\pgfqpoint{1.673985in}{0.847393in}}%
\pgfpathlineto{\pgfqpoint{1.674775in}{0.852385in}}%
\pgfpathlineto{\pgfqpoint{1.675169in}{0.845143in}}%
\pgfpathlineto{\pgfqpoint{1.676354in}{0.835381in}}%
\pgfpathlineto{\pgfqpoint{1.676748in}{0.841054in}}%
\pgfpathlineto{\pgfqpoint{1.677143in}{0.846906in}}%
\pgfpathlineto{\pgfqpoint{1.678327in}{0.841577in}}%
\pgfpathlineto{\pgfqpoint{1.678722in}{0.844347in}}%
\pgfpathlineto{\pgfqpoint{1.679512in}{0.840844in}}%
\pgfpathlineto{\pgfqpoint{1.681485in}{0.838600in}}%
\pgfpathlineto{\pgfqpoint{1.683064in}{0.857467in}}%
\pgfpathlineto{\pgfqpoint{1.684249in}{0.854720in}}%
\pgfpathlineto{\pgfqpoint{1.685433in}{0.864610in}}%
\pgfpathlineto{\pgfqpoint{1.686222in}{0.861949in}}%
\pgfpathlineto{\pgfqpoint{1.687012in}{0.856488in}}%
\pgfpathlineto{\pgfqpoint{1.687407in}{0.860216in}}%
\pgfpathlineto{\pgfqpoint{1.689775in}{0.880986in}}%
\pgfpathlineto{\pgfqpoint{1.690565in}{0.880508in}}%
\pgfpathlineto{\pgfqpoint{1.691749in}{0.876219in}}%
\pgfpathlineto{\pgfqpoint{1.696881in}{0.851722in}}%
\pgfpathlineto{\pgfqpoint{1.697275in}{0.853005in}}%
\pgfpathlineto{\pgfqpoint{1.700828in}{0.863096in}}%
\pgfpathlineto{\pgfqpoint{1.701618in}{0.862059in}}%
\pgfpathlineto{\pgfqpoint{1.702407in}{0.865246in}}%
\pgfpathlineto{\pgfqpoint{1.702802in}{0.861968in}}%
\pgfpathlineto{\pgfqpoint{1.703197in}{0.858636in}}%
\pgfpathlineto{\pgfqpoint{1.703986in}{0.863677in}}%
\pgfpathlineto{\pgfqpoint{1.708723in}{0.839211in}}%
\pgfpathlineto{\pgfqpoint{1.709513in}{0.843082in}}%
\pgfpathlineto{\pgfqpoint{1.711486in}{0.858913in}}%
\pgfpathlineto{\pgfqpoint{1.712276in}{0.864094in}}%
\pgfpathlineto{\pgfqpoint{1.713065in}{0.863137in}}%
\pgfpathlineto{\pgfqpoint{1.713855in}{0.863510in}}%
\pgfpathlineto{\pgfqpoint{1.716223in}{0.889973in}}%
\pgfpathlineto{\pgfqpoint{1.716618in}{0.887393in}}%
\pgfpathlineto{\pgfqpoint{1.717802in}{0.874619in}}%
\pgfpathlineto{\pgfqpoint{1.718592in}{0.877348in}}%
\pgfpathlineto{\pgfqpoint{1.720960in}{0.885036in}}%
\pgfpathlineto{\pgfqpoint{1.721355in}{0.883402in}}%
\pgfpathlineto{\pgfqpoint{1.724513in}{0.874026in}}%
\pgfpathlineto{\pgfqpoint{1.724908in}{0.874264in}}%
\pgfpathlineto{\pgfqpoint{1.726487in}{0.879260in}}%
\pgfpathlineto{\pgfqpoint{1.726881in}{0.875627in}}%
\pgfpathlineto{\pgfqpoint{1.727671in}{0.879219in}}%
\pgfpathlineto{\pgfqpoint{1.728855in}{0.882761in}}%
\pgfpathlineto{\pgfqpoint{1.729250in}{0.879493in}}%
\pgfpathlineto{\pgfqpoint{1.736750in}{0.807909in}}%
\pgfpathlineto{\pgfqpoint{1.737145in}{0.814094in}}%
\pgfpathlineto{\pgfqpoint{1.737934in}{0.806585in}}%
\pgfpathlineto{\pgfqpoint{1.739513in}{0.795606in}}%
\pgfpathlineto{\pgfqpoint{1.739908in}{0.797214in}}%
\pgfpathlineto{\pgfqpoint{1.740698in}{0.808630in}}%
\pgfpathlineto{\pgfqpoint{1.741092in}{0.817219in}}%
\pgfpathlineto{\pgfqpoint{1.742277in}{0.810232in}}%
\pgfpathlineto{\pgfqpoint{1.743066in}{0.808976in}}%
\pgfpathlineto{\pgfqpoint{1.743461in}{0.809702in}}%
\pgfpathlineto{\pgfqpoint{1.745829in}{0.820536in}}%
\pgfpathlineto{\pgfqpoint{1.746224in}{0.817041in}}%
\pgfpathlineto{\pgfqpoint{1.747408in}{0.820041in}}%
\pgfpathlineto{\pgfqpoint{1.751751in}{0.844799in}}%
\pgfpathlineto{\pgfqpoint{1.755303in}{0.891880in}}%
\pgfpathlineto{\pgfqpoint{1.756093in}{0.885316in}}%
\pgfpathlineto{\pgfqpoint{1.761619in}{0.846870in}}%
\pgfpathlineto{\pgfqpoint{1.762014in}{0.844974in}}%
\pgfpathlineto{\pgfqpoint{1.762409in}{0.847779in}}%
\pgfpathlineto{\pgfqpoint{1.763593in}{0.860822in}}%
\pgfpathlineto{\pgfqpoint{1.764383in}{0.855318in}}%
\pgfpathlineto{\pgfqpoint{1.767541in}{0.832899in}}%
\pgfpathlineto{\pgfqpoint{1.767935in}{0.833895in}}%
\pgfpathlineto{\pgfqpoint{1.768330in}{0.831428in}}%
\pgfpathlineto{\pgfqpoint{1.768725in}{0.829101in}}%
\pgfpathlineto{\pgfqpoint{1.769120in}{0.829198in}}%
\pgfpathlineto{\pgfqpoint{1.769514in}{0.833803in}}%
\pgfpathlineto{\pgfqpoint{1.770304in}{0.830131in}}%
\pgfpathlineto{\pgfqpoint{1.771093in}{0.826047in}}%
\pgfpathlineto{\pgfqpoint{1.771488in}{0.827820in}}%
\pgfpathlineto{\pgfqpoint{1.774251in}{0.856304in}}%
\pgfpathlineto{\pgfqpoint{1.774646in}{0.850966in}}%
\pgfpathlineto{\pgfqpoint{1.775041in}{0.848677in}}%
\pgfpathlineto{\pgfqpoint{1.775436in}{0.852569in}}%
\pgfpathlineto{\pgfqpoint{1.776225in}{0.858409in}}%
\pgfpathlineto{\pgfqpoint{1.778199in}{0.843304in}}%
\pgfpathlineto{\pgfqpoint{1.779383in}{0.860593in}}%
\pgfpathlineto{\pgfqpoint{1.780173in}{0.855990in}}%
\pgfpathlineto{\pgfqpoint{1.783725in}{0.828110in}}%
\pgfpathlineto{\pgfqpoint{1.784120in}{0.832044in}}%
\pgfpathlineto{\pgfqpoint{1.786883in}{0.859145in}}%
\pgfpathlineto{\pgfqpoint{1.787673in}{0.854143in}}%
\pgfpathlineto{\pgfqpoint{1.788068in}{0.854298in}}%
\pgfpathlineto{\pgfqpoint{1.788857in}{0.859279in}}%
\pgfpathlineto{\pgfqpoint{1.789252in}{0.857617in}}%
\pgfpathlineto{\pgfqpoint{1.789647in}{0.853871in}}%
\pgfpathlineto{\pgfqpoint{1.790831in}{0.857267in}}%
\pgfpathlineto{\pgfqpoint{1.791226in}{0.858279in}}%
\pgfpathlineto{\pgfqpoint{1.791620in}{0.856096in}}%
\pgfpathlineto{\pgfqpoint{1.795568in}{0.808923in}}%
\pgfpathlineto{\pgfqpoint{1.796357in}{0.815720in}}%
\pgfpathlineto{\pgfqpoint{1.797541in}{0.827663in}}%
\pgfpathlineto{\pgfqpoint{1.798331in}{0.825745in}}%
\pgfpathlineto{\pgfqpoint{1.799515in}{0.798582in}}%
\pgfpathlineto{\pgfqpoint{1.800305in}{0.777894in}}%
\pgfpathlineto{\pgfqpoint{1.801094in}{0.792247in}}%
\pgfpathlineto{\pgfqpoint{1.804252in}{0.804652in}}%
\pgfpathlineto{\pgfqpoint{1.805436in}{0.808828in}}%
\pgfpathlineto{\pgfqpoint{1.807015in}{0.831353in}}%
\pgfpathlineto{\pgfqpoint{1.807410in}{0.824482in}}%
\pgfpathlineto{\pgfqpoint{1.807805in}{0.820969in}}%
\pgfpathlineto{\pgfqpoint{1.808594in}{0.823626in}}%
\pgfpathlineto{\pgfqpoint{1.808989in}{0.824285in}}%
\pgfpathlineto{\pgfqpoint{1.809384in}{0.822696in}}%
\pgfpathlineto{\pgfqpoint{1.810568in}{0.811408in}}%
\pgfpathlineto{\pgfqpoint{1.811358in}{0.813410in}}%
\pgfpathlineto{\pgfqpoint{1.812147in}{0.817271in}}%
\pgfpathlineto{\pgfqpoint{1.815700in}{0.860943in}}%
\pgfpathlineto{\pgfqpoint{1.816884in}{0.862206in}}%
\pgfpathlineto{\pgfqpoint{1.818068in}{0.865809in}}%
\pgfpathlineto{\pgfqpoint{1.818463in}{0.864777in}}%
\pgfpathlineto{\pgfqpoint{1.819647in}{0.860537in}}%
\pgfpathlineto{\pgfqpoint{1.820832in}{0.869610in}}%
\pgfpathlineto{\pgfqpoint{1.821621in}{0.868368in}}%
\pgfpathlineto{\pgfqpoint{1.822805in}{0.863072in}}%
\pgfpathlineto{\pgfqpoint{1.823200in}{0.861333in}}%
\pgfpathlineto{\pgfqpoint{1.823990in}{0.861706in}}%
\pgfpathlineto{\pgfqpoint{1.824384in}{0.863451in}}%
\pgfpathlineto{\pgfqpoint{1.824779in}{0.860531in}}%
\pgfpathlineto{\pgfqpoint{1.827542in}{0.835831in}}%
\pgfpathlineto{\pgfqpoint{1.828332in}{0.842700in}}%
\pgfpathlineto{\pgfqpoint{1.828727in}{0.842891in}}%
\pgfpathlineto{\pgfqpoint{1.833069in}{0.823015in}}%
\pgfpathlineto{\pgfqpoint{1.833858in}{0.831740in}}%
\pgfpathlineto{\pgfqpoint{1.835043in}{0.845542in}}%
\pgfpathlineto{\pgfqpoint{1.835832in}{0.840557in}}%
\pgfpathlineto{\pgfqpoint{1.838201in}{0.802617in}}%
\pgfpathlineto{\pgfqpoint{1.842938in}{0.863291in}}%
\pgfpathlineto{\pgfqpoint{1.843727in}{0.859558in}}%
\pgfpathlineto{\pgfqpoint{1.848464in}{0.810447in}}%
\pgfpathlineto{\pgfqpoint{1.850438in}{0.791501in}}%
\pgfpathlineto{\pgfqpoint{1.851227in}{0.797644in}}%
\pgfpathlineto{\pgfqpoint{1.851622in}{0.797999in}}%
\pgfpathlineto{\pgfqpoint{1.853991in}{0.772109in}}%
\pgfpathlineto{\pgfqpoint{1.854385in}{0.773262in}}%
\pgfpathlineto{\pgfqpoint{1.855570in}{0.761002in}}%
\pgfpathlineto{\pgfqpoint{1.856359in}{0.750877in}}%
\pgfpathlineto{\pgfqpoint{1.857149in}{0.759468in}}%
\pgfpathlineto{\pgfqpoint{1.859122in}{0.782170in}}%
\pgfpathlineto{\pgfqpoint{1.859517in}{0.779901in}}%
\pgfpathlineto{\pgfqpoint{1.859912in}{0.782693in}}%
\pgfpathlineto{\pgfqpoint{1.860307in}{0.778731in}}%
\pgfpathlineto{\pgfqpoint{1.862280in}{0.764485in}}%
\pgfpathlineto{\pgfqpoint{1.867807in}{0.836154in}}%
\pgfpathlineto{\pgfqpoint{1.870570in}{0.828362in}}%
\pgfpathlineto{\pgfqpoint{1.872544in}{0.835382in}}%
\pgfpathlineto{\pgfqpoint{1.872939in}{0.836458in}}%
\pgfpathlineto{\pgfqpoint{1.873333in}{0.833683in}}%
\pgfpathlineto{\pgfqpoint{1.874518in}{0.826846in}}%
\pgfpathlineto{\pgfqpoint{1.875307in}{0.829581in}}%
\pgfpathlineto{\pgfqpoint{1.876097in}{0.828318in}}%
\pgfpathlineto{\pgfqpoint{1.876886in}{0.824599in}}%
\pgfpathlineto{\pgfqpoint{1.877675in}{0.827717in}}%
\pgfpathlineto{\pgfqpoint{1.878465in}{0.832022in}}%
\pgfpathlineto{\pgfqpoint{1.878860in}{0.827832in}}%
\pgfpathlineto{\pgfqpoint{1.880439in}{0.809332in}}%
\pgfpathlineto{\pgfqpoint{1.881228in}{0.815878in}}%
\pgfpathlineto{\pgfqpoint{1.883202in}{0.830928in}}%
\pgfpathlineto{\pgfqpoint{1.883597in}{0.827603in}}%
\pgfpathlineto{\pgfqpoint{1.885176in}{0.798571in}}%
\pgfpathlineto{\pgfqpoint{1.886360in}{0.802422in}}%
\pgfpathlineto{\pgfqpoint{1.888728in}{0.807319in}}%
\pgfpathlineto{\pgfqpoint{1.889123in}{0.812296in}}%
\pgfpathlineto{\pgfqpoint{1.889913in}{0.806578in}}%
\pgfpathlineto{\pgfqpoint{1.893071in}{0.761792in}}%
\pgfpathlineto{\pgfqpoint{1.893465in}{0.760591in}}%
\pgfpathlineto{\pgfqpoint{1.899387in}{0.806115in}}%
\pgfpathlineto{\pgfqpoint{1.900571in}{0.810265in}}%
\pgfpathlineto{\pgfqpoint{1.900966in}{0.808139in}}%
\pgfpathlineto{\pgfqpoint{1.902939in}{0.790326in}}%
\pgfpathlineto{\pgfqpoint{1.903729in}{0.799131in}}%
\pgfpathlineto{\pgfqpoint{1.904518in}{0.807856in}}%
\pgfpathlineto{\pgfqpoint{1.905703in}{0.805867in}}%
\pgfpathlineto{\pgfqpoint{1.906887in}{0.809539in}}%
\pgfpathlineto{\pgfqpoint{1.910045in}{0.838771in}}%
\pgfpathlineto{\pgfqpoint{1.910440in}{0.836493in}}%
\pgfpathlineto{\pgfqpoint{1.914782in}{0.794188in}}%
\pgfpathlineto{\pgfqpoint{1.916756in}{0.784808in}}%
\pgfpathlineto{\pgfqpoint{1.917150in}{0.786030in}}%
\pgfpathlineto{\pgfqpoint{1.917545in}{0.789321in}}%
\pgfpathlineto{\pgfqpoint{1.918335in}{0.783046in}}%
\pgfpathlineto{\pgfqpoint{1.919124in}{0.778685in}}%
\pgfpathlineto{\pgfqpoint{1.919519in}{0.783652in}}%
\pgfpathlineto{\pgfqpoint{1.921887in}{0.804898in}}%
\pgfpathlineto{\pgfqpoint{1.922677in}{0.806754in}}%
\pgfpathlineto{\pgfqpoint{1.923072in}{0.806135in}}%
\pgfpathlineto{\pgfqpoint{1.926624in}{0.784080in}}%
\pgfpathlineto{\pgfqpoint{1.927019in}{0.788237in}}%
\pgfpathlineto{\pgfqpoint{1.932546in}{0.851412in}}%
\pgfpathlineto{\pgfqpoint{1.932940in}{0.850406in}}%
\pgfpathlineto{\pgfqpoint{1.933730in}{0.852459in}}%
\pgfpathlineto{\pgfqpoint{1.934125in}{0.853348in}}%
\pgfpathlineto{\pgfqpoint{1.934519in}{0.851551in}}%
\pgfpathlineto{\pgfqpoint{1.935704in}{0.844444in}}%
\pgfpathlineto{\pgfqpoint{1.936493in}{0.848079in}}%
\pgfpathlineto{\pgfqpoint{1.938862in}{0.850578in}}%
\pgfpathlineto{\pgfqpoint{1.939651in}{0.849969in}}%
\pgfpathlineto{\pgfqpoint{1.940046in}{0.852009in}}%
\pgfpathlineto{\pgfqpoint{1.940441in}{0.848749in}}%
\pgfpathlineto{\pgfqpoint{1.945572in}{0.798195in}}%
\pgfpathlineto{\pgfqpoint{1.946757in}{0.797389in}}%
\pgfpathlineto{\pgfqpoint{1.951099in}{0.778390in}}%
\pgfpathlineto{\pgfqpoint{1.951888in}{0.782110in}}%
\pgfpathlineto{\pgfqpoint{1.954257in}{0.795095in}}%
\pgfpathlineto{\pgfqpoint{1.955046in}{0.803657in}}%
\pgfpathlineto{\pgfqpoint{1.955836in}{0.796733in}}%
\pgfpathlineto{\pgfqpoint{1.958599in}{0.785613in}}%
\pgfpathlineto{\pgfqpoint{1.960178in}{0.775309in}}%
\pgfpathlineto{\pgfqpoint{1.960573in}{0.776759in}}%
\pgfpathlineto{\pgfqpoint{1.962546in}{0.795550in}}%
\pgfpathlineto{\pgfqpoint{1.962941in}{0.792279in}}%
\pgfpathlineto{\pgfqpoint{1.963336in}{0.787144in}}%
\pgfpathlineto{\pgfqpoint{1.964125in}{0.793588in}}%
\pgfpathlineto{\pgfqpoint{1.964520in}{0.792766in}}%
\pgfpathlineto{\pgfqpoint{1.964915in}{0.794540in}}%
\pgfpathlineto{\pgfqpoint{1.969257in}{0.819477in}}%
\pgfpathlineto{\pgfqpoint{1.970441in}{0.817296in}}%
\pgfpathlineto{\pgfqpoint{1.971231in}{0.817154in}}%
\pgfpathlineto{\pgfqpoint{1.971626in}{0.817984in}}%
\pgfpathlineto{\pgfqpoint{1.972810in}{0.820214in}}%
\pgfpathlineto{\pgfqpoint{1.973994in}{0.808554in}}%
\pgfpathlineto{\pgfqpoint{1.974784in}{0.810845in}}%
\pgfpathlineto{\pgfqpoint{1.979126in}{0.824126in}}%
\pgfpathlineto{\pgfqpoint{1.980705in}{0.811160in}}%
\pgfpathlineto{\pgfqpoint{1.981494in}{0.816413in}}%
\pgfpathlineto{\pgfqpoint{1.982284in}{0.821841in}}%
\pgfpathlineto{\pgfqpoint{1.982679in}{0.818778in}}%
\pgfpathlineto{\pgfqpoint{1.984652in}{0.807713in}}%
\pgfpathlineto{\pgfqpoint{1.985047in}{0.809629in}}%
\pgfpathlineto{\pgfqpoint{1.987021in}{0.834116in}}%
\pgfpathlineto{\pgfqpoint{1.988205in}{0.828632in}}%
\pgfpathlineto{\pgfqpoint{1.988995in}{0.823612in}}%
\pgfpathlineto{\pgfqpoint{1.992153in}{0.858547in}}%
\pgfpathlineto{\pgfqpoint{1.992942in}{0.858836in}}%
\pgfpathlineto{\pgfqpoint{1.993337in}{0.858050in}}%
\pgfpathlineto{\pgfqpoint{1.994521in}{0.853077in}}%
\pgfpathlineto{\pgfqpoint{2.000837in}{0.808888in}}%
\pgfpathlineto{\pgfqpoint{2.002811in}{0.793439in}}%
\pgfpathlineto{\pgfqpoint{2.003206in}{0.793473in}}%
\pgfpathlineto{\pgfqpoint{2.004390in}{0.792089in}}%
\pgfpathlineto{\pgfqpoint{2.005179in}{0.790210in}}%
\pgfpathlineto{\pgfqpoint{2.005574in}{0.795320in}}%
\pgfpathlineto{\pgfqpoint{2.006364in}{0.791102in}}%
\pgfpathlineto{\pgfqpoint{2.008337in}{0.785646in}}%
\pgfpathlineto{\pgfqpoint{2.008732in}{0.787500in}}%
\pgfpathlineto{\pgfqpoint{2.011101in}{0.794978in}}%
\pgfpathlineto{\pgfqpoint{2.011890in}{0.797406in}}%
\pgfpathlineto{\pgfqpoint{2.012285in}{0.796301in}}%
\pgfpathlineto{\pgfqpoint{2.017811in}{0.748806in}}%
\pgfpathlineto{\pgfqpoint{2.018996in}{0.747998in}}%
\pgfpathlineto{\pgfqpoint{2.020180in}{0.740003in}}%
\pgfpathlineto{\pgfqpoint{2.021364in}{0.743471in}}%
\pgfpathlineto{\pgfqpoint{2.022943in}{0.746161in}}%
\pgfpathlineto{\pgfqpoint{2.024127in}{0.757008in}}%
\pgfpathlineto{\pgfqpoint{2.024522in}{0.751937in}}%
\pgfpathlineto{\pgfqpoint{2.024917in}{0.747301in}}%
\pgfpathlineto{\pgfqpoint{2.025706in}{0.755101in}}%
\pgfpathlineto{\pgfqpoint{2.030443in}{0.825754in}}%
\pgfpathlineto{\pgfqpoint{2.031233in}{0.823367in}}%
\pgfpathlineto{\pgfqpoint{2.031628in}{0.822751in}}%
\pgfpathlineto{\pgfqpoint{2.035970in}{0.781930in}}%
\pgfpathlineto{\pgfqpoint{2.036759in}{0.785838in}}%
\pgfpathlineto{\pgfqpoint{2.039523in}{0.793060in}}%
\pgfpathlineto{\pgfqpoint{2.043075in}{0.834003in}}%
\pgfpathlineto{\pgfqpoint{2.047023in}{0.807593in}}%
\pgfpathlineto{\pgfqpoint{2.047812in}{0.810393in}}%
\pgfpathlineto{\pgfqpoint{2.049391in}{0.818437in}}%
\pgfpathlineto{\pgfqpoint{2.049786in}{0.814176in}}%
\pgfpathlineto{\pgfqpoint{2.052944in}{0.787725in}}%
\pgfpathlineto{\pgfqpoint{2.053339in}{0.786896in}}%
\pgfpathlineto{\pgfqpoint{2.053733in}{0.788316in}}%
\pgfpathlineto{\pgfqpoint{2.054918in}{0.791834in}}%
\pgfpathlineto{\pgfqpoint{2.055312in}{0.790928in}}%
\pgfpathlineto{\pgfqpoint{2.056102in}{0.779909in}}%
\pgfpathlineto{\pgfqpoint{2.057286in}{0.781131in}}%
\pgfpathlineto{\pgfqpoint{2.058076in}{0.779366in}}%
\pgfpathlineto{\pgfqpoint{2.060839in}{0.768719in}}%
\pgfpathlineto{\pgfqpoint{2.062418in}{0.763642in}}%
\pgfpathlineto{\pgfqpoint{2.062813in}{0.761321in}}%
\pgfpathlineto{\pgfqpoint{2.063997in}{0.763391in}}%
\pgfpathlineto{\pgfqpoint{2.065181in}{0.772437in}}%
\pgfpathlineto{\pgfqpoint{2.065576in}{0.771411in}}%
\pgfpathlineto{\pgfqpoint{2.068339in}{0.749757in}}%
\pgfpathlineto{\pgfqpoint{2.073866in}{0.777498in}}%
\pgfpathlineto{\pgfqpoint{2.076234in}{0.785296in}}%
\pgfpathlineto{\pgfqpoint{2.076629in}{0.784484in}}%
\pgfpathlineto{\pgfqpoint{2.077024in}{0.784670in}}%
\pgfpathlineto{\pgfqpoint{2.080576in}{0.766510in}}%
\pgfpathlineto{\pgfqpoint{2.080971in}{0.769044in}}%
\pgfpathlineto{\pgfqpoint{2.083340in}{0.778388in}}%
\pgfpathlineto{\pgfqpoint{2.083734in}{0.777329in}}%
\pgfpathlineto{\pgfqpoint{2.089656in}{0.739375in}}%
\pgfpathlineto{\pgfqpoint{2.090445in}{0.742956in}}%
\pgfpathlineto{\pgfqpoint{2.093603in}{0.762156in}}%
\pgfpathlineto{\pgfqpoint{2.094393in}{0.769089in}}%
\pgfpathlineto{\pgfqpoint{2.096761in}{0.784913in}}%
\pgfpathlineto{\pgfqpoint{2.101103in}{0.800799in}}%
\pgfpathlineto{\pgfqpoint{2.101893in}{0.802156in}}%
\pgfpathlineto{\pgfqpoint{2.102288in}{0.800804in}}%
\pgfpathlineto{\pgfqpoint{2.103867in}{0.789547in}}%
\pgfpathlineto{\pgfqpoint{2.104261in}{0.792992in}}%
\pgfpathlineto{\pgfqpoint{2.105051in}{0.798960in}}%
\pgfpathlineto{\pgfqpoint{2.105840in}{0.796067in}}%
\pgfpathlineto{\pgfqpoint{2.110577in}{0.763933in}}%
\pgfpathlineto{\pgfqpoint{2.112551in}{0.745483in}}%
\pgfpathlineto{\pgfqpoint{2.112946in}{0.745512in}}%
\pgfpathlineto{\pgfqpoint{2.113735in}{0.750151in}}%
\pgfpathlineto{\pgfqpoint{2.114920in}{0.775186in}}%
\pgfpathlineto{\pgfqpoint{2.116104in}{0.771491in}}%
\pgfpathlineto{\pgfqpoint{2.117288in}{0.770912in}}%
\pgfpathlineto{\pgfqpoint{2.118472in}{0.778308in}}%
\pgfpathlineto{\pgfqpoint{2.122420in}{0.824755in}}%
\pgfpathlineto{\pgfqpoint{2.123209in}{0.821581in}}%
\pgfpathlineto{\pgfqpoint{2.128736in}{0.791594in}}%
\pgfpathlineto{\pgfqpoint{2.129130in}{0.792869in}}%
\pgfpathlineto{\pgfqpoint{2.131894in}{0.799749in}}%
\pgfpathlineto{\pgfqpoint{2.135052in}{0.805382in}}%
\pgfpathlineto{\pgfqpoint{2.137420in}{0.814331in}}%
\pgfpathlineto{\pgfqpoint{2.137815in}{0.813527in}}%
\pgfpathlineto{\pgfqpoint{2.140183in}{0.789413in}}%
\pgfpathlineto{\pgfqpoint{2.141762in}{0.762784in}}%
\pgfpathlineto{\pgfqpoint{2.142947in}{0.766110in}}%
\pgfpathlineto{\pgfqpoint{2.143736in}{0.761480in}}%
\pgfpathlineto{\pgfqpoint{2.147289in}{0.741417in}}%
\pgfpathlineto{\pgfqpoint{2.148078in}{0.743040in}}%
\pgfpathlineto{\pgfqpoint{2.149263in}{0.748299in}}%
\pgfpathlineto{\pgfqpoint{2.150447in}{0.745803in}}%
\pgfpathlineto{\pgfqpoint{2.150842in}{0.744178in}}%
\pgfpathlineto{\pgfqpoint{2.151236in}{0.747668in}}%
\pgfpathlineto{\pgfqpoint{2.151631in}{0.747569in}}%
\pgfpathlineto{\pgfqpoint{2.152421in}{0.740852in}}%
\pgfpathlineto{\pgfqpoint{2.153210in}{0.743080in}}%
\pgfpathlineto{\pgfqpoint{2.153605in}{0.743590in}}%
\pgfpathlineto{\pgfqpoint{2.154000in}{0.742615in}}%
\pgfpathlineto{\pgfqpoint{2.158342in}{0.732584in}}%
\pgfpathlineto{\pgfqpoint{2.158737in}{0.733268in}}%
\pgfpathlineto{\pgfqpoint{2.159921in}{0.732194in}}%
\pgfpathlineto{\pgfqpoint{2.164658in}{0.772075in}}%
\pgfpathlineto{\pgfqpoint{2.166632in}{0.767752in}}%
\pgfpathlineto{\pgfqpoint{2.169000in}{0.758712in}}%
\pgfpathlineto{\pgfqpoint{2.169395in}{0.761137in}}%
\pgfpathlineto{\pgfqpoint{2.170184in}{0.757072in}}%
\pgfpathlineto{\pgfqpoint{2.174527in}{0.777408in}}%
\pgfpathlineto{\pgfqpoint{2.177290in}{0.813043in}}%
\pgfpathlineto{\pgfqpoint{2.177685in}{0.811071in}}%
\pgfpathlineto{\pgfqpoint{2.178869in}{0.808087in}}%
\pgfpathlineto{\pgfqpoint{2.180053in}{0.815361in}}%
\pgfpathlineto{\pgfqpoint{2.180843in}{0.812040in}}%
\pgfpathlineto{\pgfqpoint{2.182027in}{0.803899in}}%
\pgfpathlineto{\pgfqpoint{2.185580in}{0.775406in}}%
\pgfpathlineto{\pgfqpoint{2.187948in}{0.768502in}}%
\pgfpathlineto{\pgfqpoint{2.188738in}{0.770152in}}%
\pgfpathlineto{\pgfqpoint{2.190317in}{0.777789in}}%
\pgfpathlineto{\pgfqpoint{2.191896in}{0.775558in}}%
\pgfpathlineto{\pgfqpoint{2.193475in}{0.768983in}}%
\pgfpathlineto{\pgfqpoint{2.194264in}{0.771026in}}%
\pgfpathlineto{\pgfqpoint{2.195843in}{0.777659in}}%
\pgfpathlineto{\pgfqpoint{2.197027in}{0.785668in}}%
\pgfpathlineto{\pgfqpoint{2.197817in}{0.783275in}}%
\pgfpathlineto{\pgfqpoint{2.203738in}{0.755986in}}%
\pgfpathlineto{\pgfqpoint{2.204133in}{0.756338in}}%
\pgfpathlineto{\pgfqpoint{2.204922in}{0.760091in}}%
\pgfpathlineto{\pgfqpoint{2.205712in}{0.758554in}}%
\pgfpathlineto{\pgfqpoint{2.207686in}{0.748677in}}%
\pgfpathlineto{\pgfqpoint{2.208475in}{0.752910in}}%
\pgfpathlineto{\pgfqpoint{2.210844in}{0.764862in}}%
\pgfpathlineto{\pgfqpoint{2.211238in}{0.760172in}}%
\pgfpathlineto{\pgfqpoint{2.211633in}{0.758236in}}%
\pgfpathlineto{\pgfqpoint{2.212028in}{0.760310in}}%
\pgfpathlineto{\pgfqpoint{2.217949in}{0.793937in}}%
\pgfpathlineto{\pgfqpoint{2.219528in}{0.773638in}}%
\pgfpathlineto{\pgfqpoint{2.219923in}{0.775610in}}%
\pgfpathlineto{\pgfqpoint{2.220712in}{0.785370in}}%
\pgfpathlineto{\pgfqpoint{2.221896in}{0.784097in}}%
\pgfpathlineto{\pgfqpoint{2.222686in}{0.782026in}}%
\pgfpathlineto{\pgfqpoint{2.223081in}{0.785403in}}%
\pgfpathlineto{\pgfqpoint{2.224660in}{0.794926in}}%
\pgfpathlineto{\pgfqpoint{2.225054in}{0.790157in}}%
\pgfpathlineto{\pgfqpoint{2.226239in}{0.783765in}}%
\pgfpathlineto{\pgfqpoint{2.226633in}{0.787183in}}%
\pgfpathlineto{\pgfqpoint{2.227423in}{0.787118in}}%
\pgfpathlineto{\pgfqpoint{2.227818in}{0.788410in}}%
\pgfpathlineto{\pgfqpoint{2.228212in}{0.786746in}}%
\pgfpathlineto{\pgfqpoint{2.229397in}{0.782650in}}%
\pgfpathlineto{\pgfqpoint{2.229791in}{0.783924in}}%
\pgfpathlineto{\pgfqpoint{2.230976in}{0.795059in}}%
\pgfpathlineto{\pgfqpoint{2.232160in}{0.790953in}}%
\pgfpathlineto{\pgfqpoint{2.234923in}{0.808260in}}%
\pgfpathlineto{\pgfqpoint{2.236107in}{0.804791in}}%
\pgfpathlineto{\pgfqpoint{2.247160in}{0.743219in}}%
\pgfpathlineto{\pgfqpoint{2.247950in}{0.744761in}}%
\pgfpathlineto{\pgfqpoint{2.248739in}{0.750238in}}%
\pgfpathlineto{\pgfqpoint{2.249529in}{0.749723in}}%
\pgfpathlineto{\pgfqpoint{2.251503in}{0.744714in}}%
\pgfpathlineto{\pgfqpoint{2.251897in}{0.748171in}}%
\pgfpathlineto{\pgfqpoint{2.254266in}{0.763901in}}%
\pgfpathlineto{\pgfqpoint{2.255055in}{0.761801in}}%
\pgfpathlineto{\pgfqpoint{2.258213in}{0.747392in}}%
\pgfpathlineto{\pgfqpoint{2.258608in}{0.749693in}}%
\pgfpathlineto{\pgfqpoint{2.262556in}{0.772632in}}%
\pgfpathlineto{\pgfqpoint{2.263345in}{0.769957in}}%
\pgfpathlineto{\pgfqpoint{2.264135in}{0.767090in}}%
\pgfpathlineto{\pgfqpoint{2.264924in}{0.768970in}}%
\pgfpathlineto{\pgfqpoint{2.265714in}{0.769607in}}%
\pgfpathlineto{\pgfqpoint{2.267293in}{0.773581in}}%
\pgfpathlineto{\pgfqpoint{2.267687in}{0.772442in}}%
\pgfpathlineto{\pgfqpoint{2.271240in}{0.750549in}}%
\pgfpathlineto{\pgfqpoint{2.272030in}{0.746490in}}%
\pgfpathlineto{\pgfqpoint{2.272819in}{0.748308in}}%
\pgfpathlineto{\pgfqpoint{2.274793in}{0.757411in}}%
\pgfpathlineto{\pgfqpoint{2.275582in}{0.752774in}}%
\pgfpathlineto{\pgfqpoint{2.282293in}{0.719326in}}%
\pgfpathlineto{\pgfqpoint{2.283083in}{0.720861in}}%
\pgfpathlineto{\pgfqpoint{2.284267in}{0.726379in}}%
\pgfpathlineto{\pgfqpoint{2.287425in}{0.759277in}}%
\pgfpathlineto{\pgfqpoint{2.288609in}{0.770273in}}%
\pgfpathlineto{\pgfqpoint{2.289399in}{0.769689in}}%
\pgfpathlineto{\pgfqpoint{2.289793in}{0.769751in}}%
\pgfpathlineto{\pgfqpoint{2.290583in}{0.775634in}}%
\pgfpathlineto{\pgfqpoint{2.291372in}{0.773145in}}%
\pgfpathlineto{\pgfqpoint{2.294136in}{0.767557in}}%
\pgfpathlineto{\pgfqpoint{2.295714in}{0.768819in}}%
\pgfpathlineto{\pgfqpoint{2.297293in}{0.765037in}}%
\pgfpathlineto{\pgfqpoint{2.300057in}{0.743591in}}%
\pgfpathlineto{\pgfqpoint{2.303609in}{0.716519in}}%
\pgfpathlineto{\pgfqpoint{2.304004in}{0.720290in}}%
\pgfpathlineto{\pgfqpoint{2.309925in}{0.776828in}}%
\pgfpathlineto{\pgfqpoint{2.310320in}{0.774597in}}%
\pgfpathlineto{\pgfqpoint{2.311504in}{0.765873in}}%
\pgfpathlineto{\pgfqpoint{2.312689in}{0.768872in}}%
\pgfpathlineto{\pgfqpoint{2.314268in}{0.772032in}}%
\pgfpathlineto{\pgfqpoint{2.317031in}{0.759457in}}%
\pgfpathlineto{\pgfqpoint{2.317426in}{0.758949in}}%
\pgfpathlineto{\pgfqpoint{2.319794in}{0.738180in}}%
\pgfpathlineto{\pgfqpoint{2.322952in}{0.730240in}}%
\pgfpathlineto{\pgfqpoint{2.323742in}{0.732327in}}%
\pgfpathlineto{\pgfqpoint{2.324531in}{0.747010in}}%
\pgfpathlineto{\pgfqpoint{2.325321in}{0.742689in}}%
\pgfpathlineto{\pgfqpoint{2.327689in}{0.729578in}}%
\pgfpathlineto{\pgfqpoint{2.328084in}{0.729876in}}%
\pgfpathlineto{\pgfqpoint{2.332031in}{0.742024in}}%
\pgfpathlineto{\pgfqpoint{2.332426in}{0.738914in}}%
\pgfpathlineto{\pgfqpoint{2.333610in}{0.734231in}}%
\pgfpathlineto{\pgfqpoint{2.334005in}{0.734772in}}%
\pgfpathlineto{\pgfqpoint{2.334795in}{0.736529in}}%
\pgfpathlineto{\pgfqpoint{2.335189in}{0.735383in}}%
\pgfpathlineto{\pgfqpoint{2.336374in}{0.731415in}}%
\pgfpathlineto{\pgfqpoint{2.336768in}{0.733826in}}%
\pgfpathlineto{\pgfqpoint{2.339926in}{0.737266in}}%
\pgfpathlineto{\pgfqpoint{2.341505in}{0.737270in}}%
\pgfpathlineto{\pgfqpoint{2.343479in}{0.741329in}}%
\pgfpathlineto{\pgfqpoint{2.343874in}{0.741545in}}%
\pgfpathlineto{\pgfqpoint{2.345848in}{0.760414in}}%
\pgfpathlineto{\pgfqpoint{2.349006in}{0.775357in}}%
\pgfpathlineto{\pgfqpoint{2.351769in}{0.785836in}}%
\pgfpathlineto{\pgfqpoint{2.352164in}{0.784534in}}%
\pgfpathlineto{\pgfqpoint{2.352953in}{0.782407in}}%
\pgfpathlineto{\pgfqpoint{2.353348in}{0.784276in}}%
\pgfpathlineto{\pgfqpoint{2.355716in}{0.794357in}}%
\pgfpathlineto{\pgfqpoint{2.356111in}{0.791289in}}%
\pgfpathlineto{\pgfqpoint{2.357690in}{0.771327in}}%
\pgfpathlineto{\pgfqpoint{2.359664in}{0.775634in}}%
\pgfpathlineto{\pgfqpoint{2.360453in}{0.776535in}}%
\pgfpathlineto{\pgfqpoint{2.361243in}{0.775622in}}%
\pgfpathlineto{\pgfqpoint{2.363611in}{0.769849in}}%
\pgfpathlineto{\pgfqpoint{2.364006in}{0.770023in}}%
\pgfpathlineto{\pgfqpoint{2.364401in}{0.773641in}}%
\pgfpathlineto{\pgfqpoint{2.365190in}{0.769854in}}%
\pgfpathlineto{\pgfqpoint{2.365585in}{0.770856in}}%
\pgfpathlineto{\pgfqpoint{2.365980in}{0.770771in}}%
\pgfpathlineto{\pgfqpoint{2.366375in}{0.766934in}}%
\pgfpathlineto{\pgfqpoint{2.367164in}{0.768970in}}%
\pgfpathlineto{\pgfqpoint{2.368348in}{0.780219in}}%
\pgfpathlineto{\pgfqpoint{2.369927in}{0.779889in}}%
\pgfpathlineto{\pgfqpoint{2.370717in}{0.782366in}}%
\pgfpathlineto{\pgfqpoint{2.371112in}{0.778310in}}%
\pgfpathlineto{\pgfqpoint{2.371901in}{0.781737in}}%
\pgfpathlineto{\pgfqpoint{2.375454in}{0.774150in}}%
\pgfpathlineto{\pgfqpoint{2.375849in}{0.779565in}}%
\pgfpathlineto{\pgfqpoint{2.377033in}{0.775652in}}%
\pgfpathlineto{\pgfqpoint{2.382164in}{0.756964in}}%
\pgfpathlineto{\pgfqpoint{2.384533in}{0.752839in}}%
\pgfpathlineto{\pgfqpoint{2.384928in}{0.754416in}}%
\pgfpathlineto{\pgfqpoint{2.386901in}{0.764946in}}%
\pgfpathlineto{\pgfqpoint{2.390849in}{0.720926in}}%
\pgfpathlineto{\pgfqpoint{2.391638in}{0.724342in}}%
\pgfpathlineto{\pgfqpoint{2.392033in}{0.723200in}}%
\pgfpathlineto{\pgfqpoint{2.392428in}{0.725868in}}%
\pgfpathlineto{\pgfqpoint{2.400323in}{0.775806in}}%
\pgfpathlineto{\pgfqpoint{2.400718in}{0.774964in}}%
\pgfpathlineto{\pgfqpoint{2.401112in}{0.773201in}}%
\pgfpathlineto{\pgfqpoint{2.401507in}{0.774512in}}%
\pgfpathlineto{\pgfqpoint{2.403481in}{0.781337in}}%
\pgfpathlineto{\pgfqpoint{2.405455in}{0.777471in}}%
\pgfpathlineto{\pgfqpoint{2.406639in}{0.775905in}}%
\pgfpathlineto{\pgfqpoint{2.407034in}{0.776273in}}%
\pgfpathlineto{\pgfqpoint{2.407823in}{0.778142in}}%
\pgfpathlineto{\pgfqpoint{2.410981in}{0.786656in}}%
\pgfpathlineto{\pgfqpoint{2.412560in}{0.780628in}}%
\pgfpathlineto{\pgfqpoint{2.412955in}{0.781666in}}%
\pgfpathlineto{\pgfqpoint{2.415323in}{0.788091in}}%
\pgfpathlineto{\pgfqpoint{2.420060in}{0.734660in}}%
\pgfpathlineto{\pgfqpoint{2.420455in}{0.735671in}}%
\pgfpathlineto{\pgfqpoint{2.422429in}{0.748488in}}%
\pgfpathlineto{\pgfqpoint{2.422824in}{0.747975in}}%
\pgfpathlineto{\pgfqpoint{2.424797in}{0.727058in}}%
\pgfpathlineto{\pgfqpoint{2.425192in}{0.730529in}}%
\pgfpathlineto{\pgfqpoint{2.425982in}{0.724966in}}%
\pgfpathlineto{\pgfqpoint{2.426376in}{0.725000in}}%
\pgfpathlineto{\pgfqpoint{2.427955in}{0.737824in}}%
\pgfpathlineto{\pgfqpoint{2.429140in}{0.732367in}}%
\pgfpathlineto{\pgfqpoint{2.429534in}{0.731341in}}%
\pgfpathlineto{\pgfqpoint{2.430324in}{0.732570in}}%
\pgfpathlineto{\pgfqpoint{2.431903in}{0.732960in}}%
\pgfpathlineto{\pgfqpoint{2.432692in}{0.729287in}}%
\pgfpathlineto{\pgfqpoint{2.433087in}{0.731235in}}%
\pgfpathlineto{\pgfqpoint{2.433877in}{0.736858in}}%
\pgfpathlineto{\pgfqpoint{2.434666in}{0.734374in}}%
\pgfpathlineto{\pgfqpoint{2.437429in}{0.724259in}}%
\pgfpathlineto{\pgfqpoint{2.440193in}{0.703484in}}%
\pgfpathlineto{\pgfqpoint{2.444930in}{0.678114in}}%
\pgfpathlineto{\pgfqpoint{2.445719in}{0.678860in}}%
\pgfpathlineto{\pgfqpoint{2.449667in}{0.692747in}}%
\pgfpathlineto{\pgfqpoint{2.450456in}{0.699372in}}%
\pgfpathlineto{\pgfqpoint{2.451640in}{0.697515in}}%
\pgfpathlineto{\pgfqpoint{2.452035in}{0.697170in}}%
\pgfpathlineto{\pgfqpoint{2.453219in}{0.689642in}}%
\pgfpathlineto{\pgfqpoint{2.453614in}{0.692385in}}%
\pgfpathlineto{\pgfqpoint{2.459535in}{0.744628in}}%
\pgfpathlineto{\pgfqpoint{2.460720in}{0.740605in}}%
\pgfpathlineto{\pgfqpoint{2.461509in}{0.742887in}}%
\pgfpathlineto{\pgfqpoint{2.463483in}{0.747553in}}%
\pgfpathlineto{\pgfqpoint{2.464272in}{0.746087in}}%
\pgfpathlineto{\pgfqpoint{2.465456in}{0.745355in}}%
\pgfpathlineto{\pgfqpoint{2.469404in}{0.727495in}}%
\pgfpathlineto{\pgfqpoint{2.469799in}{0.728819in}}%
\pgfpathlineto{\pgfqpoint{2.470983in}{0.730817in}}%
\pgfpathlineto{\pgfqpoint{2.471378in}{0.730623in}}%
\pgfpathlineto{\pgfqpoint{2.473746in}{0.717131in}}%
\pgfpathlineto{\pgfqpoint{2.474536in}{0.722091in}}%
\pgfpathlineto{\pgfqpoint{2.474930in}{0.726313in}}%
\pgfpathlineto{\pgfqpoint{2.475325in}{0.718898in}}%
\pgfpathlineto{\pgfqpoint{2.477694in}{0.692765in}}%
\pgfpathlineto{\pgfqpoint{2.478483in}{0.694506in}}%
\pgfpathlineto{\pgfqpoint{2.479667in}{0.700620in}}%
\pgfpathlineto{\pgfqpoint{2.482036in}{0.678902in}}%
\pgfpathlineto{\pgfqpoint{2.482825in}{0.680890in}}%
\pgfpathlineto{\pgfqpoint{2.483615in}{0.678600in}}%
\pgfpathlineto{\pgfqpoint{2.485589in}{0.669907in}}%
\pgfpathlineto{\pgfqpoint{2.485983in}{0.670041in}}%
\pgfpathlineto{\pgfqpoint{2.490720in}{0.710249in}}%
\pgfpathlineto{\pgfqpoint{2.491510in}{0.702075in}}%
\pgfpathlineto{\pgfqpoint{2.492694in}{0.691457in}}%
\pgfpathlineto{\pgfqpoint{2.493484in}{0.692507in}}%
\pgfpathlineto{\pgfqpoint{2.495457in}{0.699609in}}%
\pgfpathlineto{\pgfqpoint{2.499800in}{0.733140in}}%
\pgfpathlineto{\pgfqpoint{2.501379in}{0.736769in}}%
\pgfpathlineto{\pgfqpoint{2.501773in}{0.735815in}}%
\pgfpathlineto{\pgfqpoint{2.507695in}{0.708112in}}%
\pgfpathlineto{\pgfqpoint{2.509668in}{0.715769in}}%
\pgfpathlineto{\pgfqpoint{2.510063in}{0.714908in}}%
\pgfpathlineto{\pgfqpoint{2.510853in}{0.712590in}}%
\pgfpathlineto{\pgfqpoint{2.511642in}{0.714073in}}%
\pgfpathlineto{\pgfqpoint{2.513616in}{0.719145in}}%
\pgfpathlineto{\pgfqpoint{2.514011in}{0.718686in}}%
\pgfpathlineto{\pgfqpoint{2.514800in}{0.720433in}}%
\pgfpathlineto{\pgfqpoint{2.515590in}{0.722151in}}%
\pgfpathlineto{\pgfqpoint{2.516774in}{0.726668in}}%
\pgfpathlineto{\pgfqpoint{2.517169in}{0.724216in}}%
\pgfpathlineto{\pgfqpoint{2.518353in}{0.719369in}}%
\pgfpathlineto{\pgfqpoint{2.519142in}{0.719900in}}%
\pgfpathlineto{\pgfqpoint{2.519932in}{0.722084in}}%
\pgfpathlineto{\pgfqpoint{2.520327in}{0.721148in}}%
\pgfpathlineto{\pgfqpoint{2.521116in}{0.718210in}}%
\pgfpathlineto{\pgfqpoint{2.521511in}{0.718683in}}%
\pgfpathlineto{\pgfqpoint{2.525458in}{0.737165in}}%
\pgfpathlineto{\pgfqpoint{2.526248in}{0.735991in}}%
\pgfpathlineto{\pgfqpoint{2.529011in}{0.758760in}}%
\pgfpathlineto{\pgfqpoint{2.529801in}{0.756833in}}%
\pgfpathlineto{\pgfqpoint{2.530590in}{0.759320in}}%
\pgfpathlineto{\pgfqpoint{2.531380in}{0.761086in}}%
\pgfpathlineto{\pgfqpoint{2.532959in}{0.750343in}}%
\pgfpathlineto{\pgfqpoint{2.533748in}{0.751530in}}%
\pgfpathlineto{\pgfqpoint{2.535722in}{0.752013in}}%
\pgfpathlineto{\pgfqpoint{2.536511in}{0.752455in}}%
\pgfpathlineto{\pgfqpoint{2.537301in}{0.759932in}}%
\pgfpathlineto{\pgfqpoint{2.538090in}{0.757241in}}%
\pgfpathlineto{\pgfqpoint{2.542038in}{0.737680in}}%
\pgfpathlineto{\pgfqpoint{2.544406in}{0.732842in}}%
\pgfpathlineto{\pgfqpoint{2.545196in}{0.728844in}}%
\pgfpathlineto{\pgfqpoint{2.547564in}{0.717210in}}%
\pgfpathlineto{\pgfqpoint{2.549933in}{0.714789in}}%
\pgfpathlineto{\pgfqpoint{2.551512in}{0.716632in}}%
\pgfpathlineto{\pgfqpoint{2.551906in}{0.715409in}}%
\pgfpathlineto{\pgfqpoint{2.552696in}{0.716982in}}%
\pgfpathlineto{\pgfqpoint{2.553880in}{0.721526in}}%
\pgfpathlineto{\pgfqpoint{2.554670in}{0.720533in}}%
\pgfpathlineto{\pgfqpoint{2.555064in}{0.719499in}}%
\pgfpathlineto{\pgfqpoint{2.555854in}{0.721369in}}%
\pgfpathlineto{\pgfqpoint{2.558222in}{0.731421in}}%
\pgfpathlineto{\pgfqpoint{2.559407in}{0.737382in}}%
\pgfpathlineto{\pgfqpoint{2.560196in}{0.737241in}}%
\pgfpathlineto{\pgfqpoint{2.561380in}{0.739967in}}%
\pgfpathlineto{\pgfqpoint{2.561775in}{0.740784in}}%
\pgfpathlineto{\pgfqpoint{2.562959in}{0.739755in}}%
\pgfpathlineto{\pgfqpoint{2.564144in}{0.739415in}}%
\pgfpathlineto{\pgfqpoint{2.566117in}{0.729505in}}%
\pgfpathlineto{\pgfqpoint{2.568091in}{0.720150in}}%
\pgfpathlineto{\pgfqpoint{2.568486in}{0.720213in}}%
\pgfpathlineto{\pgfqpoint{2.569275in}{0.725486in}}%
\pgfpathlineto{\pgfqpoint{2.570460in}{0.724749in}}%
\pgfpathlineto{\pgfqpoint{2.571644in}{0.725491in}}%
\pgfpathlineto{\pgfqpoint{2.572039in}{0.725085in}}%
\pgfpathlineto{\pgfqpoint{2.573618in}{0.719568in}}%
\pgfpathlineto{\pgfqpoint{2.574407in}{0.720260in}}%
\pgfpathlineto{\pgfqpoint{2.577960in}{0.725067in}}%
\pgfpathlineto{\pgfqpoint{2.578749in}{0.724537in}}%
\pgfpathlineto{\pgfqpoint{2.579539in}{0.716628in}}%
\pgfpathlineto{\pgfqpoint{2.580328in}{0.718603in}}%
\pgfpathlineto{\pgfqpoint{2.582302in}{0.712764in}}%
\pgfpathlineto{\pgfqpoint{2.584671in}{0.727098in}}%
\pgfpathlineto{\pgfqpoint{2.587039in}{0.733946in}}%
\pgfpathlineto{\pgfqpoint{2.589013in}{0.723543in}}%
\pgfpathlineto{\pgfqpoint{2.590197in}{0.708247in}}%
\pgfpathlineto{\pgfqpoint{2.590987in}{0.715289in}}%
\pgfpathlineto{\pgfqpoint{2.595329in}{0.744576in}}%
\pgfpathlineto{\pgfqpoint{2.597697in}{0.730526in}}%
\pgfpathlineto{\pgfqpoint{2.598882in}{0.729810in}}%
\pgfpathlineto{\pgfqpoint{2.602829in}{0.750957in}}%
\pgfpathlineto{\pgfqpoint{2.603224in}{0.750853in}}%
\pgfpathlineto{\pgfqpoint{2.604408in}{0.748599in}}%
\pgfpathlineto{\pgfqpoint{2.604803in}{0.752581in}}%
\pgfpathlineto{\pgfqpoint{2.605987in}{0.748895in}}%
\pgfpathlineto{\pgfqpoint{2.608750in}{0.726876in}}%
\pgfpathlineto{\pgfqpoint{2.609145in}{0.730146in}}%
\pgfpathlineto{\pgfqpoint{2.609540in}{0.734599in}}%
\pgfpathlineto{\pgfqpoint{2.610329in}{0.727232in}}%
\pgfpathlineto{\pgfqpoint{2.613093in}{0.732757in}}%
\pgfpathlineto{\pgfqpoint{2.613487in}{0.732499in}}%
\pgfpathlineto{\pgfqpoint{2.615461in}{0.726023in}}%
\pgfpathlineto{\pgfqpoint{2.615856in}{0.727527in}}%
\pgfpathlineto{\pgfqpoint{2.617040in}{0.732679in}}%
\pgfpathlineto{\pgfqpoint{2.617830in}{0.731898in}}%
\pgfpathlineto{\pgfqpoint{2.618224in}{0.732099in}}%
\pgfpathlineto{\pgfqpoint{2.618619in}{0.731180in}}%
\pgfpathlineto{\pgfqpoint{2.620593in}{0.725207in}}%
\pgfpathlineto{\pgfqpoint{2.624540in}{0.739413in}}%
\pgfpathlineto{\pgfqpoint{2.625330in}{0.739144in}}%
\pgfpathlineto{\pgfqpoint{2.626119in}{0.739858in}}%
\pgfpathlineto{\pgfqpoint{2.626514in}{0.739312in}}%
\pgfpathlineto{\pgfqpoint{2.628093in}{0.733519in}}%
\pgfpathlineto{\pgfqpoint{2.628882in}{0.736835in}}%
\pgfpathlineto{\pgfqpoint{2.629277in}{0.738839in}}%
\pgfpathlineto{\pgfqpoint{2.629672in}{0.736578in}}%
\pgfpathlineto{\pgfqpoint{2.630067in}{0.726972in}}%
\pgfpathlineto{\pgfqpoint{2.631251in}{0.734009in}}%
\pgfpathlineto{\pgfqpoint{2.635593in}{0.749675in}}%
\pgfpathlineto{\pgfqpoint{2.637567in}{0.747116in}}%
\pgfpathlineto{\pgfqpoint{2.639935in}{0.739875in}}%
\pgfpathlineto{\pgfqpoint{2.642304in}{0.722360in}}%
\pgfpathlineto{\pgfqpoint{2.643883in}{0.723143in}}%
\pgfpathlineto{\pgfqpoint{2.644278in}{0.723414in}}%
\pgfpathlineto{\pgfqpoint{2.645067in}{0.722116in}}%
\pgfpathlineto{\pgfqpoint{2.647041in}{0.721763in}}%
\pgfpathlineto{\pgfqpoint{2.651383in}{0.706978in}}%
\pgfpathlineto{\pgfqpoint{2.653357in}{0.702916in}}%
\pgfpathlineto{\pgfqpoint{2.655331in}{0.688556in}}%
\pgfpathlineto{\pgfqpoint{2.656120in}{0.683268in}}%
\pgfpathlineto{\pgfqpoint{2.656910in}{0.687215in}}%
\pgfpathlineto{\pgfqpoint{2.657304in}{0.687506in}}%
\pgfpathlineto{\pgfqpoint{2.658489in}{0.683037in}}%
\pgfpathlineto{\pgfqpoint{2.658883in}{0.685740in}}%
\pgfpathlineto{\pgfqpoint{2.662041in}{0.696137in}}%
\pgfpathlineto{\pgfqpoint{2.662436in}{0.696003in}}%
\pgfpathlineto{\pgfqpoint{2.664015in}{0.688048in}}%
\pgfpathlineto{\pgfqpoint{2.664805in}{0.689811in}}%
\pgfpathlineto{\pgfqpoint{2.666778in}{0.689255in}}%
\pgfpathlineto{\pgfqpoint{2.667963in}{0.688253in}}%
\pgfpathlineto{\pgfqpoint{2.668752in}{0.691923in}}%
\pgfpathlineto{\pgfqpoint{2.669542in}{0.690859in}}%
\pgfpathlineto{\pgfqpoint{2.669936in}{0.691869in}}%
\pgfpathlineto{\pgfqpoint{2.673094in}{0.699298in}}%
\pgfpathlineto{\pgfqpoint{2.673489in}{0.698953in}}%
\pgfpathlineto{\pgfqpoint{2.673884in}{0.700604in}}%
\pgfpathlineto{\pgfqpoint{2.675858in}{0.704700in}}%
\pgfpathlineto{\pgfqpoint{2.678226in}{0.706448in}}%
\pgfpathlineto{\pgfqpoint{2.679016in}{0.706701in}}%
\pgfpathlineto{\pgfqpoint{2.682568in}{0.680654in}}%
\pgfpathlineto{\pgfqpoint{2.683753in}{0.683028in}}%
\pgfpathlineto{\pgfqpoint{2.685332in}{0.684946in}}%
\pgfpathlineto{\pgfqpoint{2.686911in}{0.697250in}}%
\pgfpathlineto{\pgfqpoint{2.687305in}{0.695311in}}%
\pgfpathlineto{\pgfqpoint{2.687700in}{0.693459in}}%
\pgfpathlineto{\pgfqpoint{2.688884in}{0.693943in}}%
\pgfpathlineto{\pgfqpoint{2.689279in}{0.693494in}}%
\pgfpathlineto{\pgfqpoint{2.691648in}{0.701123in}}%
\pgfpathlineto{\pgfqpoint{2.694806in}{0.711416in}}%
\pgfpathlineto{\pgfqpoint{2.695990in}{0.713232in}}%
\pgfpathlineto{\pgfqpoint{2.696385in}{0.712691in}}%
\pgfpathlineto{\pgfqpoint{2.697569in}{0.711522in}}%
\pgfpathlineto{\pgfqpoint{2.697964in}{0.712427in}}%
\pgfpathlineto{\pgfqpoint{2.701516in}{0.734321in}}%
\pgfpathlineto{\pgfqpoint{2.702701in}{0.730676in}}%
\pgfpathlineto{\pgfqpoint{2.704280in}{0.726325in}}%
\pgfpathlineto{\pgfqpoint{2.705069in}{0.728009in}}%
\pgfpathlineto{\pgfqpoint{2.707043in}{0.730374in}}%
\pgfpathlineto{\pgfqpoint{2.707438in}{0.729644in}}%
\pgfpathlineto{\pgfqpoint{2.708227in}{0.725307in}}%
\pgfpathlineto{\pgfqpoint{2.709411in}{0.713188in}}%
\pgfpathlineto{\pgfqpoint{2.710201in}{0.716175in}}%
\pgfpathlineto{\pgfqpoint{2.710596in}{0.715628in}}%
\pgfpathlineto{\pgfqpoint{2.712569in}{0.688669in}}%
\pgfpathlineto{\pgfqpoint{2.712964in}{0.690220in}}%
\pgfpathlineto{\pgfqpoint{2.715332in}{0.694524in}}%
\pgfpathlineto{\pgfqpoint{2.715727in}{0.694048in}}%
\pgfpathlineto{\pgfqpoint{2.717306in}{0.692541in}}%
\pgfpathlineto{\pgfqpoint{2.717701in}{0.693067in}}%
\pgfpathlineto{\pgfqpoint{2.719280in}{0.695676in}}%
\pgfpathlineto{\pgfqpoint{2.721254in}{0.701794in}}%
\pgfpathlineto{\pgfqpoint{2.722438in}{0.693035in}}%
\pgfpathlineto{\pgfqpoint{2.722833in}{0.695752in}}%
\pgfpathlineto{\pgfqpoint{2.724412in}{0.703966in}}%
\pgfpathlineto{\pgfqpoint{2.724806in}{0.703482in}}%
\pgfpathlineto{\pgfqpoint{2.727175in}{0.699738in}}%
\pgfpathlineto{\pgfqpoint{2.728754in}{0.693241in}}%
\pgfpathlineto{\pgfqpoint{2.729543in}{0.697956in}}%
\pgfpathlineto{\pgfqpoint{2.736649in}{0.755552in}}%
\pgfpathlineto{\pgfqpoint{2.737833in}{0.752615in}}%
\pgfpathlineto{\pgfqpoint{2.748491in}{0.689182in}}%
\pgfpathlineto{\pgfqpoint{2.749281in}{0.691163in}}%
\pgfpathlineto{\pgfqpoint{2.750860in}{0.696600in}}%
\pgfpathlineto{\pgfqpoint{2.753623in}{0.709636in}}%
\pgfpathlineto{\pgfqpoint{2.755202in}{0.706541in}}%
\pgfpathlineto{\pgfqpoint{2.755597in}{0.708677in}}%
\pgfpathlineto{\pgfqpoint{2.759544in}{0.727795in}}%
\pgfpathlineto{\pgfqpoint{2.760729in}{0.733350in}}%
\pgfpathlineto{\pgfqpoint{2.762702in}{0.738396in}}%
\pgfpathlineto{\pgfqpoint{2.763097in}{0.737736in}}%
\pgfpathlineto{\pgfqpoint{2.764281in}{0.736063in}}%
\pgfpathlineto{\pgfqpoint{2.764676in}{0.736586in}}%
\pgfpathlineto{\pgfqpoint{2.766650in}{0.744771in}}%
\pgfpathlineto{\pgfqpoint{2.767045in}{0.741598in}}%
\pgfpathlineto{\pgfqpoint{2.768229in}{0.732084in}}%
\pgfpathlineto{\pgfqpoint{2.769413in}{0.735510in}}%
\pgfpathlineto{\pgfqpoint{2.772176in}{0.743640in}}%
\pgfpathlineto{\pgfqpoint{2.777703in}{0.713549in}}%
\pgfpathlineto{\pgfqpoint{2.778492in}{0.716163in}}%
\pgfpathlineto{\pgfqpoint{2.780466in}{0.726265in}}%
\pgfpathlineto{\pgfqpoint{2.780861in}{0.725954in}}%
\pgfpathlineto{\pgfqpoint{2.782835in}{0.716129in}}%
\pgfpathlineto{\pgfqpoint{2.784019in}{0.710780in}}%
\pgfpathlineto{\pgfqpoint{2.784414in}{0.711879in}}%
\pgfpathlineto{\pgfqpoint{2.789151in}{0.736336in}}%
\pgfpathlineto{\pgfqpoint{2.791124in}{0.746336in}}%
\pgfpathlineto{\pgfqpoint{2.791914in}{0.743803in}}%
\pgfpathlineto{\pgfqpoint{2.792309in}{0.742463in}}%
\pgfpathlineto{\pgfqpoint{2.792703in}{0.745715in}}%
\pgfpathlineto{\pgfqpoint{2.793493in}{0.750158in}}%
\pgfpathlineto{\pgfqpoint{2.793888in}{0.746529in}}%
\pgfpathlineto{\pgfqpoint{2.798624in}{0.718300in}}%
\pgfpathlineto{\pgfqpoint{2.799019in}{0.719718in}}%
\pgfpathlineto{\pgfqpoint{2.802572in}{0.727633in}}%
\pgfpathlineto{\pgfqpoint{2.804940in}{0.706417in}}%
\pgfpathlineto{\pgfqpoint{2.805730in}{0.711599in}}%
\pgfpathlineto{\pgfqpoint{2.806914in}{0.718104in}}%
\pgfpathlineto{\pgfqpoint{2.807309in}{0.716063in}}%
\pgfpathlineto{\pgfqpoint{2.808493in}{0.703358in}}%
\pgfpathlineto{\pgfqpoint{2.808888in}{0.706025in}}%
\pgfpathlineto{\pgfqpoint{2.809283in}{0.712095in}}%
\pgfpathlineto{\pgfqpoint{2.810467in}{0.710020in}}%
\pgfpathlineto{\pgfqpoint{2.810862in}{0.708706in}}%
\pgfpathlineto{\pgfqpoint{2.812046in}{0.709791in}}%
\pgfpathlineto{\pgfqpoint{2.814809in}{0.715140in}}%
\pgfpathlineto{\pgfqpoint{2.818362in}{0.738079in}}%
\pgfpathlineto{\pgfqpoint{2.819151in}{0.739992in}}%
\pgfpathlineto{\pgfqpoint{2.820336in}{0.742017in}}%
\pgfpathlineto{\pgfqpoint{2.821915in}{0.734476in}}%
\pgfpathlineto{\pgfqpoint{2.822704in}{0.735697in}}%
\pgfpathlineto{\pgfqpoint{2.825862in}{0.741848in}}%
\pgfpathlineto{\pgfqpoint{2.828625in}{0.743218in}}%
\pgfpathlineto{\pgfqpoint{2.831783in}{0.724940in}}%
\pgfpathlineto{\pgfqpoint{2.835336in}{0.701361in}}%
\pgfpathlineto{\pgfqpoint{2.835731in}{0.703518in}}%
\pgfpathlineto{\pgfqpoint{2.844415in}{0.744628in}}%
\pgfpathlineto{\pgfqpoint{2.844810in}{0.743408in}}%
\pgfpathlineto{\pgfqpoint{2.845205in}{0.742822in}}%
\pgfpathlineto{\pgfqpoint{2.845600in}{0.744077in}}%
\pgfpathlineto{\pgfqpoint{2.848363in}{0.756546in}}%
\pgfpathlineto{\pgfqpoint{2.850337in}{0.772877in}}%
\pgfpathlineto{\pgfqpoint{2.851126in}{0.769993in}}%
\pgfpathlineto{\pgfqpoint{2.859811in}{0.717999in}}%
\pgfpathlineto{\pgfqpoint{2.860205in}{0.718062in}}%
\pgfpathlineto{\pgfqpoint{2.860600in}{0.718657in}}%
\pgfpathlineto{\pgfqpoint{2.862179in}{0.739874in}}%
\pgfpathlineto{\pgfqpoint{2.864153in}{0.737631in}}%
\pgfpathlineto{\pgfqpoint{2.864942in}{0.739264in}}%
\pgfpathlineto{\pgfqpoint{2.866521in}{0.743481in}}%
\pgfpathlineto{\pgfqpoint{2.866916in}{0.741666in}}%
\pgfpathlineto{\pgfqpoint{2.869285in}{0.736135in}}%
\pgfpathlineto{\pgfqpoint{2.871258in}{0.733353in}}%
\pgfpathlineto{\pgfqpoint{2.871653in}{0.734450in}}%
\pgfpathlineto{\pgfqpoint{2.874811in}{0.758275in}}%
\pgfpathlineto{\pgfqpoint{2.875601in}{0.757056in}}%
\pgfpathlineto{\pgfqpoint{2.879153in}{0.751620in}}%
\pgfpathlineto{\pgfqpoint{2.882311in}{0.732209in}}%
\pgfpathlineto{\pgfqpoint{2.883495in}{0.721350in}}%
\pgfpathlineto{\pgfqpoint{2.884285in}{0.711329in}}%
\pgfpathlineto{\pgfqpoint{2.885074in}{0.712188in}}%
\pgfpathlineto{\pgfqpoint{2.886259in}{0.714927in}}%
\pgfpathlineto{\pgfqpoint{2.886653in}{0.714463in}}%
\pgfpathlineto{\pgfqpoint{2.887048in}{0.712702in}}%
\pgfpathlineto{\pgfqpoint{2.888627in}{0.723861in}}%
\pgfpathlineto{\pgfqpoint{2.889022in}{0.723269in}}%
\pgfpathlineto{\pgfqpoint{2.893759in}{0.704701in}}%
\pgfpathlineto{\pgfqpoint{2.894548in}{0.705515in}}%
\pgfpathlineto{\pgfqpoint{2.896522in}{0.709159in}}%
\pgfpathlineto{\pgfqpoint{2.898101in}{0.715334in}}%
\pgfpathlineto{\pgfqpoint{2.899285in}{0.716747in}}%
\pgfpathlineto{\pgfqpoint{2.902838in}{0.705260in}}%
\pgfpathlineto{\pgfqpoint{2.903233in}{0.705368in}}%
\pgfpathlineto{\pgfqpoint{2.906391in}{0.729036in}}%
\pgfpathlineto{\pgfqpoint{2.906786in}{0.732781in}}%
\pgfpathlineto{\pgfqpoint{2.907575in}{0.730204in}}%
\pgfpathlineto{\pgfqpoint{2.911523in}{0.712389in}}%
\pgfpathlineto{\pgfqpoint{2.911917in}{0.713898in}}%
\pgfpathlineto{\pgfqpoint{2.912707in}{0.720744in}}%
\pgfpathlineto{\pgfqpoint{2.913496in}{0.716983in}}%
\pgfpathlineto{\pgfqpoint{2.914681in}{0.716495in}}%
\pgfpathlineto{\pgfqpoint{2.915075in}{0.717416in}}%
\pgfpathlineto{\pgfqpoint{2.916260in}{0.718255in}}%
\pgfpathlineto{\pgfqpoint{2.918628in}{0.733567in}}%
\pgfpathlineto{\pgfqpoint{2.919418in}{0.733337in}}%
\pgfpathlineto{\pgfqpoint{2.922181in}{0.728829in}}%
\pgfpathlineto{\pgfqpoint{2.922576in}{0.727908in}}%
\pgfpathlineto{\pgfqpoint{2.923760in}{0.740287in}}%
\pgfpathlineto{\pgfqpoint{2.924549in}{0.737670in}}%
\pgfpathlineto{\pgfqpoint{2.927707in}{0.728874in}}%
\pgfpathlineto{\pgfqpoint{2.930076in}{0.728451in}}%
\pgfpathlineto{\pgfqpoint{2.930865in}{0.729596in}}%
\pgfpathlineto{\pgfqpoint{2.931260in}{0.728787in}}%
\pgfpathlineto{\pgfqpoint{2.933234in}{0.721200in}}%
\pgfpathlineto{\pgfqpoint{2.935602in}{0.711357in}}%
\pgfpathlineto{\pgfqpoint{2.936787in}{0.708758in}}%
\pgfpathlineto{\pgfqpoint{2.937181in}{0.709793in}}%
\pgfpathlineto{\pgfqpoint{2.937576in}{0.717695in}}%
\pgfpathlineto{\pgfqpoint{2.938760in}{0.713109in}}%
\pgfpathlineto{\pgfqpoint{2.941129in}{0.708249in}}%
\pgfpathlineto{\pgfqpoint{2.944287in}{0.713508in}}%
\pgfpathlineto{\pgfqpoint{2.945076in}{0.716379in}}%
\pgfpathlineto{\pgfqpoint{2.945866in}{0.713651in}}%
\pgfpathlineto{\pgfqpoint{2.948234in}{0.715801in}}%
\pgfpathlineto{\pgfqpoint{2.949024in}{0.714173in}}%
\pgfpathlineto{\pgfqpoint{2.949419in}{0.714672in}}%
\pgfpathlineto{\pgfqpoint{2.952577in}{0.731131in}}%
\pgfpathlineto{\pgfqpoint{2.954156in}{0.726626in}}%
\pgfpathlineto{\pgfqpoint{2.956129in}{0.729862in}}%
\pgfpathlineto{\pgfqpoint{2.956919in}{0.728790in}}%
\pgfpathlineto{\pgfqpoint{2.958893in}{0.723977in}}%
\pgfpathlineto{\pgfqpoint{2.959682in}{0.727423in}}%
\pgfpathlineto{\pgfqpoint{2.960077in}{0.721251in}}%
\pgfpathlineto{\pgfqpoint{2.961261in}{0.725485in}}%
\pgfpathlineto{\pgfqpoint{2.962051in}{0.729511in}}%
\pgfpathlineto{\pgfqpoint{2.962840in}{0.728067in}}%
\pgfpathlineto{\pgfqpoint{2.963235in}{0.728226in}}%
\pgfpathlineto{\pgfqpoint{2.965998in}{0.736979in}}%
\pgfpathlineto{\pgfqpoint{2.969551in}{0.746614in}}%
\pgfpathlineto{\pgfqpoint{2.969945in}{0.745949in}}%
\pgfpathlineto{\pgfqpoint{2.977840in}{0.711558in}}%
\pgfpathlineto{\pgfqpoint{2.980209in}{0.729847in}}%
\pgfpathlineto{\pgfqpoint{2.981788in}{0.735191in}}%
\pgfpathlineto{\pgfqpoint{2.982183in}{0.733579in}}%
\pgfpathlineto{\pgfqpoint{2.982972in}{0.725336in}}%
\pgfpathlineto{\pgfqpoint{2.983762in}{0.728202in}}%
\pgfpathlineto{\pgfqpoint{2.988893in}{0.754276in}}%
\pgfpathlineto{\pgfqpoint{2.989683in}{0.765382in}}%
\pgfpathlineto{\pgfqpoint{2.990867in}{0.763484in}}%
\pgfpathlineto{\pgfqpoint{2.992841in}{0.759138in}}%
\pgfpathlineto{\pgfqpoint{2.998367in}{0.745651in}}%
\pgfpathlineto{\pgfqpoint{2.999157in}{0.745372in}}%
\pgfpathlineto{\pgfqpoint{2.999946in}{0.744474in}}%
\pgfpathlineto{\pgfqpoint{3.000341in}{0.745796in}}%
\pgfpathlineto{\pgfqpoint{3.002710in}{0.748673in}}%
\pgfpathlineto{\pgfqpoint{3.006262in}{0.742156in}}%
\pgfpathlineto{\pgfqpoint{3.007052in}{0.741632in}}%
\pgfpathlineto{\pgfqpoint{3.007447in}{0.744446in}}%
\pgfpathlineto{\pgfqpoint{3.007841in}{0.739940in}}%
\pgfpathlineto{\pgfqpoint{3.009815in}{0.734147in}}%
\pgfpathlineto{\pgfqpoint{3.010605in}{0.735652in}}%
\pgfpathlineto{\pgfqpoint{3.011394in}{0.736670in}}%
\pgfpathlineto{\pgfqpoint{3.011789in}{0.736413in}}%
\pgfpathlineto{\pgfqpoint{3.012578in}{0.732610in}}%
\pgfpathlineto{\pgfqpoint{3.012973in}{0.738036in}}%
\pgfpathlineto{\pgfqpoint{3.013368in}{0.737979in}}%
\pgfpathlineto{\pgfqpoint{3.016526in}{0.749822in}}%
\pgfpathlineto{\pgfqpoint{3.016921in}{0.748869in}}%
\pgfpathlineto{\pgfqpoint{3.022447in}{0.728309in}}%
\pgfpathlineto{\pgfqpoint{3.025210in}{0.708647in}}%
\pgfpathlineto{\pgfqpoint{3.025605in}{0.708722in}}%
\pgfpathlineto{\pgfqpoint{3.028368in}{0.718643in}}%
\pgfpathlineto{\pgfqpoint{3.029553in}{0.717770in}}%
\pgfpathlineto{\pgfqpoint{3.029947in}{0.717313in}}%
\pgfpathlineto{\pgfqpoint{3.030342in}{0.718390in}}%
\pgfpathlineto{\pgfqpoint{3.032316in}{0.720475in}}%
\pgfpathlineto{\pgfqpoint{3.032711in}{0.719882in}}%
\pgfpathlineto{\pgfqpoint{3.035079in}{0.707226in}}%
\pgfpathlineto{\pgfqpoint{3.035869in}{0.704638in}}%
\pgfpathlineto{\pgfqpoint{3.036658in}{0.706263in}}%
\pgfpathlineto{\pgfqpoint{3.037448in}{0.706330in}}%
\pgfpathlineto{\pgfqpoint{3.041000in}{0.734893in}}%
\pgfpathlineto{\pgfqpoint{3.042185in}{0.734346in}}%
\pgfpathlineto{\pgfqpoint{3.045737in}{0.724566in}}%
\pgfpathlineto{\pgfqpoint{3.051658in}{0.704370in}}%
\pgfpathlineto{\pgfqpoint{3.052843in}{0.704773in}}%
\pgfpathlineto{\pgfqpoint{3.055211in}{0.713492in}}%
\pgfpathlineto{\pgfqpoint{3.055606in}{0.711528in}}%
\pgfpathlineto{\pgfqpoint{3.056395in}{0.708977in}}%
\pgfpathlineto{\pgfqpoint{3.057185in}{0.711273in}}%
\pgfpathlineto{\pgfqpoint{3.059553in}{0.722024in}}%
\pgfpathlineto{\pgfqpoint{3.059948in}{0.721257in}}%
\pgfpathlineto{\pgfqpoint{3.061132in}{0.715509in}}%
\pgfpathlineto{\pgfqpoint{3.061922in}{0.719822in}}%
\pgfpathlineto{\pgfqpoint{3.064685in}{0.727287in}}%
\pgfpathlineto{\pgfqpoint{3.067448in}{0.737350in}}%
\pgfpathlineto{\pgfqpoint{3.069817in}{0.746826in}}%
\pgfpathlineto{\pgfqpoint{3.070212in}{0.746239in}}%
\pgfpathlineto{\pgfqpoint{3.082054in}{0.705736in}}%
\pgfpathlineto{\pgfqpoint{3.084028in}{0.691536in}}%
\pgfpathlineto{\pgfqpoint{3.084423in}{0.691718in}}%
\pgfpathlineto{\pgfqpoint{3.084817in}{0.690838in}}%
\pgfpathlineto{\pgfqpoint{3.086396in}{0.688715in}}%
\pgfpathlineto{\pgfqpoint{3.086791in}{0.689361in}}%
\pgfpathlineto{\pgfqpoint{3.087581in}{0.703762in}}%
\pgfpathlineto{\pgfqpoint{3.089160in}{0.715708in}}%
\pgfpathlineto{\pgfqpoint{3.089554in}{0.712803in}}%
\pgfpathlineto{\pgfqpoint{3.091923in}{0.700726in}}%
\pgfpathlineto{\pgfqpoint{3.092712in}{0.701021in}}%
\pgfpathlineto{\pgfqpoint{3.093107in}{0.700666in}}%
\pgfpathlineto{\pgfqpoint{3.093502in}{0.702292in}}%
\pgfpathlineto{\pgfqpoint{3.095476in}{0.710389in}}%
\pgfpathlineto{\pgfqpoint{3.097449in}{0.720106in}}%
\pgfpathlineto{\pgfqpoint{3.098239in}{0.718977in}}%
\pgfpathlineto{\pgfqpoint{3.101792in}{0.716710in}}%
\pgfpathlineto{\pgfqpoint{3.102186in}{0.717674in}}%
\pgfpathlineto{\pgfqpoint{3.108502in}{0.741264in}}%
\pgfpathlineto{\pgfqpoint{3.110871in}{0.710653in}}%
\pgfpathlineto{\pgfqpoint{3.111266in}{0.711337in}}%
\pgfpathlineto{\pgfqpoint{3.113634in}{0.722970in}}%
\pgfpathlineto{\pgfqpoint{3.114424in}{0.716708in}}%
\pgfpathlineto{\pgfqpoint{3.115608in}{0.705564in}}%
\pgfpathlineto{\pgfqpoint{3.116397in}{0.712693in}}%
\pgfpathlineto{\pgfqpoint{3.118766in}{0.723874in}}%
\pgfpathlineto{\pgfqpoint{3.119950in}{0.722420in}}%
\pgfpathlineto{\pgfqpoint{3.121924in}{0.717861in}}%
\pgfpathlineto{\pgfqpoint{3.122319in}{0.718198in}}%
\pgfpathlineto{\pgfqpoint{3.123108in}{0.720359in}}%
\pgfpathlineto{\pgfqpoint{3.123898in}{0.717954in}}%
\pgfpathlineto{\pgfqpoint{3.132187in}{0.689023in}}%
\pgfpathlineto{\pgfqpoint{3.134556in}{0.669817in}}%
\pgfpathlineto{\pgfqpoint{3.135345in}{0.668317in}}%
\pgfpathlineto{\pgfqpoint{3.135740in}{0.669883in}}%
\pgfpathlineto{\pgfqpoint{3.138108in}{0.678358in}}%
\pgfpathlineto{\pgfqpoint{3.138503in}{0.676009in}}%
\pgfpathlineto{\pgfqpoint{3.139293in}{0.671427in}}%
\pgfpathlineto{\pgfqpoint{3.140082in}{0.673092in}}%
\pgfpathlineto{\pgfqpoint{3.140872in}{0.674137in}}%
\pgfpathlineto{\pgfqpoint{3.146398in}{0.696478in}}%
\pgfpathlineto{\pgfqpoint{3.147582in}{0.689748in}}%
\pgfpathlineto{\pgfqpoint{3.149161in}{0.681251in}}%
\pgfpathlineto{\pgfqpoint{3.149951in}{0.682709in}}%
\pgfpathlineto{\pgfqpoint{3.151925in}{0.705255in}}%
\pgfpathlineto{\pgfqpoint{3.154293in}{0.725612in}}%
\pgfpathlineto{\pgfqpoint{3.156267in}{0.734372in}}%
\pgfpathlineto{\pgfqpoint{3.156662in}{0.733127in}}%
\pgfpathlineto{\pgfqpoint{3.159425in}{0.708646in}}%
\pgfpathlineto{\pgfqpoint{3.162188in}{0.683157in}}%
\pgfpathlineto{\pgfqpoint{3.162583in}{0.681542in}}%
\pgfpathlineto{\pgfqpoint{3.162978in}{0.683299in}}%
\pgfpathlineto{\pgfqpoint{3.165741in}{0.713679in}}%
\pgfpathlineto{\pgfqpoint{3.166925in}{0.713483in}}%
\pgfpathlineto{\pgfqpoint{3.171267in}{0.694891in}}%
\pgfpathlineto{\pgfqpoint{3.171662in}{0.696851in}}%
\pgfpathlineto{\pgfqpoint{3.173636in}{0.698441in}}%
\pgfpathlineto{\pgfqpoint{3.174425in}{0.699962in}}%
\pgfpathlineto{\pgfqpoint{3.174820in}{0.700512in}}%
\pgfpathlineto{\pgfqpoint{3.175215in}{0.698598in}}%
\pgfpathlineto{\pgfqpoint{3.176794in}{0.694376in}}%
\pgfpathlineto{\pgfqpoint{3.177189in}{0.694438in}}%
\pgfpathlineto{\pgfqpoint{3.181531in}{0.701468in}}%
\pgfpathlineto{\pgfqpoint{3.181926in}{0.702862in}}%
\pgfpathlineto{\pgfqpoint{3.182320in}{0.702098in}}%
\pgfpathlineto{\pgfqpoint{3.184294in}{0.688703in}}%
\pgfpathlineto{\pgfqpoint{3.185873in}{0.685226in}}%
\pgfpathlineto{\pgfqpoint{3.186268in}{0.686281in}}%
\pgfpathlineto{\pgfqpoint{3.188242in}{0.689338in}}%
\pgfpathlineto{\pgfqpoint{3.191005in}{0.709607in}}%
\pgfpathlineto{\pgfqpoint{3.191400in}{0.710523in}}%
\pgfpathlineto{\pgfqpoint{3.191794in}{0.708449in}}%
\pgfpathlineto{\pgfqpoint{3.192189in}{0.707820in}}%
\pgfpathlineto{\pgfqpoint{3.192979in}{0.709086in}}%
\pgfpathlineto{\pgfqpoint{3.193373in}{0.710389in}}%
\pgfpathlineto{\pgfqpoint{3.194163in}{0.709044in}}%
\pgfpathlineto{\pgfqpoint{3.196926in}{0.685001in}}%
\pgfpathlineto{\pgfqpoint{3.197321in}{0.682398in}}%
\pgfpathlineto{\pgfqpoint{3.197321in}{0.682398in}}%
\pgfusepath{stroke}%
\end{pgfscope}%
\begin{pgfscope}%
\pgfpathrectangle{\pgfqpoint{0.608025in}{0.484444in}}{\pgfqpoint{2.712595in}{1.541287in}}%
\pgfusepath{clip}%
\pgfsetbuttcap%
\pgfsetmiterjoin%
\definecolor{currentfill}{rgb}{0.172549,0.627451,0.172549}%
\pgfsetfillcolor{currentfill}%
\pgfsetlinewidth{1.003750pt}%
\definecolor{currentstroke}{rgb}{0.172549,0.627451,0.172549}%
\pgfsetstrokecolor{currentstroke}%
\pgfsetdash{}{0pt}%
\pgfsys@defobject{currentmarker}{\pgfqpoint{-0.020833in}{-0.020833in}}{\pgfqpoint{0.020833in}{0.020833in}}{%
\pgfpathmoveto{\pgfqpoint{-0.000000in}{-0.020833in}}%
\pgfpathlineto{\pgfqpoint{0.020833in}{0.020833in}}%
\pgfpathlineto{\pgfqpoint{-0.020833in}{0.020833in}}%
\pgfpathlineto{\pgfqpoint{-0.000000in}{-0.020833in}}%
\pgfpathclose%
\pgfusepath{stroke,fill}%
}%
\begin{pgfscope}%
\pgfsys@transformshift{0.770800in}{1.864433in}%
\pgfsys@useobject{currentmarker}{}%
\end{pgfscope}%
\begin{pgfscope}%
\pgfsys@transformshift{0.968174in}{1.299430in}%
\pgfsys@useobject{currentmarker}{}%
\end{pgfscope}%
\begin{pgfscope}%
\pgfsys@transformshift{1.165549in}{1.092485in}%
\pgfsys@useobject{currentmarker}{}%
\end{pgfscope}%
\begin{pgfscope}%
\pgfsys@transformshift{1.362923in}{0.986431in}%
\pgfsys@useobject{currentmarker}{}%
\end{pgfscope}%
\begin{pgfscope}%
\pgfsys@transformshift{1.560297in}{0.877010in}%
\pgfsys@useobject{currentmarker}{}%
\end{pgfscope}%
\begin{pgfscope}%
\pgfsys@transformshift{1.757672in}{0.877539in}%
\pgfsys@useobject{currentmarker}{}%
\end{pgfscope}%
\begin{pgfscope}%
\pgfsys@transformshift{1.955046in}{0.803657in}%
\pgfsys@useobject{currentmarker}{}%
\end{pgfscope}%
\begin{pgfscope}%
\pgfsys@transformshift{2.152421in}{0.740852in}%
\pgfsys@useobject{currentmarker}{}%
\end{pgfscope}%
\begin{pgfscope}%
\pgfsys@transformshift{2.349795in}{0.779793in}%
\pgfsys@useobject{currentmarker}{}%
\end{pgfscope}%
\begin{pgfscope}%
\pgfsys@transformshift{2.547169in}{0.718562in}%
\pgfsys@useobject{currentmarker}{}%
\end{pgfscope}%
\begin{pgfscope}%
\pgfsys@transformshift{2.744544in}{0.717312in}%
\pgfsys@useobject{currentmarker}{}%
\end{pgfscope}%
\begin{pgfscope}%
\pgfsys@transformshift{2.941918in}{0.709445in}%
\pgfsys@useobject{currentmarker}{}%
\end{pgfscope}%
\begin{pgfscope}%
\pgfsys@transformshift{3.139293in}{0.671427in}%
\pgfsys@useobject{currentmarker}{}%
\end{pgfscope}%
\end{pgfscope}%
\begin{pgfscope}%
\pgfpathrectangle{\pgfqpoint{0.608025in}{0.484444in}}{\pgfqpoint{2.712595in}{1.541287in}}%
\pgfusepath{clip}%
\pgfsetrectcap%
\pgfsetroundjoin%
\pgfsetlinewidth{1.505625pt}%
\definecolor{currentstroke}{rgb}{0.839216,0.152941,0.156863}%
\pgfsetstrokecolor{currentstroke}%
\pgfsetdash{}{0pt}%
\pgfpathmoveto{\pgfqpoint{0.731325in}{1.955537in}}%
\pgfpathlineto{\pgfqpoint{0.745141in}{1.944589in}}%
\pgfpathlineto{\pgfqpoint{0.749483in}{1.934221in}}%
\pgfpathlineto{\pgfqpoint{0.766852in}{1.872571in}}%
\pgfpathlineto{\pgfqpoint{0.775932in}{1.846256in}}%
\pgfpathlineto{\pgfqpoint{0.783827in}{1.827674in}}%
\pgfpathlineto{\pgfqpoint{0.794090in}{1.797213in}}%
\pgfpathlineto{\pgfqpoint{0.800801in}{1.786095in}}%
\pgfpathlineto{\pgfqpoint{0.809485in}{1.775436in}}%
\pgfpathlineto{\pgfqpoint{0.809880in}{1.775807in}}%
\pgfpathlineto{\pgfqpoint{0.813038in}{1.787914in}}%
\pgfpathlineto{\pgfqpoint{0.816196in}{1.774208in}}%
\pgfpathlineto{\pgfqpoint{0.817775in}{1.775488in}}%
\pgfpathlineto{\pgfqpoint{0.818959in}{1.777015in}}%
\pgfpathlineto{\pgfqpoint{0.819354in}{1.775962in}}%
\pgfpathlineto{\pgfqpoint{0.826854in}{1.732522in}}%
\pgfpathlineto{\pgfqpoint{0.830407in}{1.722716in}}%
\pgfpathlineto{\pgfqpoint{0.833960in}{1.684283in}}%
\pgfpathlineto{\pgfqpoint{0.839881in}{1.618009in}}%
\pgfpathlineto{\pgfqpoint{0.845802in}{1.585824in}}%
\pgfpathlineto{\pgfqpoint{0.851723in}{1.548443in}}%
\pgfpathlineto{\pgfqpoint{0.852513in}{1.550126in}}%
\pgfpathlineto{\pgfqpoint{0.858434in}{1.573434in}}%
\pgfpathlineto{\pgfqpoint{0.859224in}{1.568379in}}%
\pgfpathlineto{\pgfqpoint{0.867908in}{1.517760in}}%
\pgfpathlineto{\pgfqpoint{0.868303in}{1.519772in}}%
\pgfpathlineto{\pgfqpoint{0.872250in}{1.541925in}}%
\pgfpathlineto{\pgfqpoint{0.873435in}{1.558884in}}%
\pgfpathlineto{\pgfqpoint{0.874224in}{1.556870in}}%
\pgfpathlineto{\pgfqpoint{0.877777in}{1.543003in}}%
\pgfpathlineto{\pgfqpoint{0.878961in}{1.546039in}}%
\pgfpathlineto{\pgfqpoint{0.879750in}{1.548452in}}%
\pgfpathlineto{\pgfqpoint{0.882514in}{1.555367in}}%
\pgfpathlineto{\pgfqpoint{0.883698in}{1.553890in}}%
\pgfpathlineto{\pgfqpoint{0.885672in}{1.543318in}}%
\pgfpathlineto{\pgfqpoint{0.889224in}{1.499193in}}%
\pgfpathlineto{\pgfqpoint{0.893172in}{1.439524in}}%
\pgfpathlineto{\pgfqpoint{0.893567in}{1.442987in}}%
\pgfpathlineto{\pgfqpoint{0.897514in}{1.495570in}}%
\pgfpathlineto{\pgfqpoint{0.898304in}{1.483895in}}%
\pgfpathlineto{\pgfqpoint{0.900672in}{1.448988in}}%
\pgfpathlineto{\pgfqpoint{0.901856in}{1.451275in}}%
\pgfpathlineto{\pgfqpoint{0.903041in}{1.450513in}}%
\pgfpathlineto{\pgfqpoint{0.905014in}{1.443275in}}%
\pgfpathlineto{\pgfqpoint{0.907778in}{1.419542in}}%
\pgfpathlineto{\pgfqpoint{0.909357in}{1.397786in}}%
\pgfpathlineto{\pgfqpoint{0.910541in}{1.401420in}}%
\pgfpathlineto{\pgfqpoint{0.916857in}{1.424733in}}%
\pgfpathlineto{\pgfqpoint{0.919225in}{1.378991in}}%
\pgfpathlineto{\pgfqpoint{0.920015in}{1.383650in}}%
\pgfpathlineto{\pgfqpoint{0.923568in}{1.405014in}}%
\pgfpathlineto{\pgfqpoint{0.924752in}{1.403834in}}%
\pgfpathlineto{\pgfqpoint{0.925541in}{1.400704in}}%
\pgfpathlineto{\pgfqpoint{0.925936in}{1.403647in}}%
\pgfpathlineto{\pgfqpoint{0.929094in}{1.417675in}}%
\pgfpathlineto{\pgfqpoint{0.930278in}{1.416120in}}%
\pgfpathlineto{\pgfqpoint{0.931068in}{1.409449in}}%
\pgfpathlineto{\pgfqpoint{0.932252in}{1.411068in}}%
\pgfpathlineto{\pgfqpoint{0.933436in}{1.410379in}}%
\pgfpathlineto{\pgfqpoint{0.934226in}{1.401602in}}%
\pgfpathlineto{\pgfqpoint{0.936200in}{1.361865in}}%
\pgfpathlineto{\pgfqpoint{0.936989in}{1.370616in}}%
\pgfpathlineto{\pgfqpoint{0.940542in}{1.392710in}}%
\pgfpathlineto{\pgfqpoint{0.941331in}{1.397269in}}%
\pgfpathlineto{\pgfqpoint{0.941726in}{1.395388in}}%
\pgfpathlineto{\pgfqpoint{0.943700in}{1.374302in}}%
\pgfpathlineto{\pgfqpoint{0.944489in}{1.378097in}}%
\pgfpathlineto{\pgfqpoint{0.947647in}{1.385074in}}%
\pgfpathlineto{\pgfqpoint{0.948832in}{1.382929in}}%
\pgfpathlineto{\pgfqpoint{0.950411in}{1.375659in}}%
\pgfpathlineto{\pgfqpoint{0.951200in}{1.377755in}}%
\pgfpathlineto{\pgfqpoint{0.952779in}{1.380516in}}%
\pgfpathlineto{\pgfqpoint{0.953174in}{1.380137in}}%
\pgfpathlineto{\pgfqpoint{0.954753in}{1.374208in}}%
\pgfpathlineto{\pgfqpoint{0.955148in}{1.371689in}}%
\pgfpathlineto{\pgfqpoint{0.956332in}{1.373604in}}%
\pgfpathlineto{\pgfqpoint{0.957516in}{1.374120in}}%
\pgfpathlineto{\pgfqpoint{0.961858in}{1.331557in}}%
\pgfpathlineto{\pgfqpoint{0.963042in}{1.320835in}}%
\pgfpathlineto{\pgfqpoint{0.963832in}{1.325509in}}%
\pgfpathlineto{\pgfqpoint{0.964621in}{1.330685in}}%
\pgfpathlineto{\pgfqpoint{0.965016in}{1.325299in}}%
\pgfpathlineto{\pgfqpoint{0.968964in}{1.294077in}}%
\pgfpathlineto{\pgfqpoint{0.971332in}{1.285243in}}%
\pgfpathlineto{\pgfqpoint{0.972122in}{1.286011in}}%
\pgfpathlineto{\pgfqpoint{0.972516in}{1.285157in}}%
\pgfpathlineto{\pgfqpoint{0.974490in}{1.281038in}}%
\pgfpathlineto{\pgfqpoint{0.978832in}{1.301788in}}%
\pgfpathlineto{\pgfqpoint{0.980806in}{1.326518in}}%
\pgfpathlineto{\pgfqpoint{0.981596in}{1.324556in}}%
\pgfpathlineto{\pgfqpoint{0.983569in}{1.312381in}}%
\pgfpathlineto{\pgfqpoint{0.984359in}{1.315728in}}%
\pgfpathlineto{\pgfqpoint{0.988701in}{1.352332in}}%
\pgfpathlineto{\pgfqpoint{0.989096in}{1.345900in}}%
\pgfpathlineto{\pgfqpoint{0.991070in}{1.341785in}}%
\pgfpathlineto{\pgfqpoint{0.991859in}{1.339482in}}%
\pgfpathlineto{\pgfqpoint{0.992254in}{1.342881in}}%
\pgfpathlineto{\pgfqpoint{0.992649in}{1.349547in}}%
\pgfpathlineto{\pgfqpoint{0.993438in}{1.340488in}}%
\pgfpathlineto{\pgfqpoint{0.996201in}{1.289162in}}%
\pgfpathlineto{\pgfqpoint{0.996596in}{1.293673in}}%
\pgfpathlineto{\pgfqpoint{0.999754in}{1.315262in}}%
\pgfpathlineto{\pgfqpoint{1.003307in}{1.276620in}}%
\pgfpathlineto{\pgfqpoint{1.004096in}{1.282168in}}%
\pgfpathlineto{\pgfqpoint{1.005281in}{1.299267in}}%
\pgfpathlineto{\pgfqpoint{1.005675in}{1.308557in}}%
\pgfpathlineto{\pgfqpoint{1.006860in}{1.306986in}}%
\pgfpathlineto{\pgfqpoint{1.007254in}{1.307861in}}%
\pgfpathlineto{\pgfqpoint{1.007649in}{1.314888in}}%
\pgfpathlineto{\pgfqpoint{1.008833in}{1.312119in}}%
\pgfpathlineto{\pgfqpoint{1.009228in}{1.311246in}}%
\pgfpathlineto{\pgfqpoint{1.009623in}{1.312282in}}%
\pgfpathlineto{\pgfqpoint{1.013570in}{1.349074in}}%
\pgfpathlineto{\pgfqpoint{1.017123in}{1.291147in}}%
\pgfpathlineto{\pgfqpoint{1.019886in}{1.258554in}}%
\pgfpathlineto{\pgfqpoint{1.022255in}{1.239569in}}%
\pgfpathlineto{\pgfqpoint{1.023834in}{1.244993in}}%
\pgfpathlineto{\pgfqpoint{1.025808in}{1.258672in}}%
\pgfpathlineto{\pgfqpoint{1.026202in}{1.260837in}}%
\pgfpathlineto{\pgfqpoint{1.026597in}{1.259683in}}%
\pgfpathlineto{\pgfqpoint{1.031334in}{1.171493in}}%
\pgfpathlineto{\pgfqpoint{1.031729in}{1.172046in}}%
\pgfpathlineto{\pgfqpoint{1.032518in}{1.171640in}}%
\pgfpathlineto{\pgfqpoint{1.035676in}{1.146573in}}%
\pgfpathlineto{\pgfqpoint{1.037255in}{1.158488in}}%
\pgfpathlineto{\pgfqpoint{1.041992in}{1.197390in}}%
\pgfpathlineto{\pgfqpoint{1.043176in}{1.196895in}}%
\pgfpathlineto{\pgfqpoint{1.045150in}{1.190542in}}%
\pgfpathlineto{\pgfqpoint{1.047124in}{1.178604in}}%
\pgfpathlineto{\pgfqpoint{1.047519in}{1.186034in}}%
\pgfpathlineto{\pgfqpoint{1.053835in}{1.285036in}}%
\pgfpathlineto{\pgfqpoint{1.054229in}{1.285105in}}%
\pgfpathlineto{\pgfqpoint{1.055019in}{1.277770in}}%
\pgfpathlineto{\pgfqpoint{1.055808in}{1.281813in}}%
\pgfpathlineto{\pgfqpoint{1.060151in}{1.306488in}}%
\pgfpathlineto{\pgfqpoint{1.060940in}{1.305333in}}%
\pgfpathlineto{\pgfqpoint{1.069230in}{1.163913in}}%
\pgfpathlineto{\pgfqpoint{1.073572in}{1.226747in}}%
\pgfpathlineto{\pgfqpoint{1.073967in}{1.224077in}}%
\pgfpathlineto{\pgfqpoint{1.074362in}{1.230117in}}%
\pgfpathlineto{\pgfqpoint{1.083441in}{1.346249in}}%
\pgfpathlineto{\pgfqpoint{1.084625in}{1.343280in}}%
\pgfpathlineto{\pgfqpoint{1.090546in}{1.292827in}}%
\pgfpathlineto{\pgfqpoint{1.094889in}{1.188580in}}%
\pgfpathlineto{\pgfqpoint{1.098441in}{1.107031in}}%
\pgfpathlineto{\pgfqpoint{1.098836in}{1.108778in}}%
\pgfpathlineto{\pgfqpoint{1.100020in}{1.133567in}}%
\pgfpathlineto{\pgfqpoint{1.100810in}{1.125221in}}%
\pgfpathlineto{\pgfqpoint{1.101205in}{1.124490in}}%
\pgfpathlineto{\pgfqpoint{1.105152in}{1.172742in}}%
\pgfpathlineto{\pgfqpoint{1.106731in}{1.192479in}}%
\pgfpathlineto{\pgfqpoint{1.107126in}{1.188334in}}%
\pgfpathlineto{\pgfqpoint{1.108310in}{1.170752in}}%
\pgfpathlineto{\pgfqpoint{1.108705in}{1.178846in}}%
\pgfpathlineto{\pgfqpoint{1.112258in}{1.241949in}}%
\pgfpathlineto{\pgfqpoint{1.113442in}{1.239491in}}%
\pgfpathlineto{\pgfqpoint{1.115021in}{1.231653in}}%
\pgfpathlineto{\pgfqpoint{1.115416in}{1.235814in}}%
\pgfpathlineto{\pgfqpoint{1.117389in}{1.245765in}}%
\pgfpathlineto{\pgfqpoint{1.118179in}{1.245103in}}%
\pgfpathlineto{\pgfqpoint{1.120547in}{1.238628in}}%
\pgfpathlineto{\pgfqpoint{1.122126in}{1.227641in}}%
\pgfpathlineto{\pgfqpoint{1.126468in}{1.172513in}}%
\pgfpathlineto{\pgfqpoint{1.128047in}{1.176149in}}%
\pgfpathlineto{\pgfqpoint{1.129232in}{1.183472in}}%
\pgfpathlineto{\pgfqpoint{1.130021in}{1.179216in}}%
\pgfpathlineto{\pgfqpoint{1.131995in}{1.146297in}}%
\pgfpathlineto{\pgfqpoint{1.133179in}{1.120690in}}%
\pgfpathlineto{\pgfqpoint{1.133574in}{1.132704in}}%
\pgfpathlineto{\pgfqpoint{1.135548in}{1.182197in}}%
\pgfpathlineto{\pgfqpoint{1.136337in}{1.171894in}}%
\pgfpathlineto{\pgfqpoint{1.137916in}{1.153932in}}%
\pgfpathlineto{\pgfqpoint{1.138311in}{1.156510in}}%
\pgfpathlineto{\pgfqpoint{1.141864in}{1.195463in}}%
\pgfpathlineto{\pgfqpoint{1.144627in}{1.217411in}}%
\pgfpathlineto{\pgfqpoint{1.145022in}{1.216415in}}%
\pgfpathlineto{\pgfqpoint{1.151338in}{1.131442in}}%
\pgfpathlineto{\pgfqpoint{1.152522in}{1.141383in}}%
\pgfpathlineto{\pgfqpoint{1.152917in}{1.141687in}}%
\pgfpathlineto{\pgfqpoint{1.157654in}{1.064134in}}%
\pgfpathlineto{\pgfqpoint{1.158048in}{1.066719in}}%
\pgfpathlineto{\pgfqpoint{1.160812in}{1.098306in}}%
\pgfpathlineto{\pgfqpoint{1.161601in}{1.088686in}}%
\pgfpathlineto{\pgfqpoint{1.162391in}{1.093624in}}%
\pgfpathlineto{\pgfqpoint{1.162785in}{1.095905in}}%
\pgfpathlineto{\pgfqpoint{1.163575in}{1.081093in}}%
\pgfpathlineto{\pgfqpoint{1.163970in}{1.089422in}}%
\pgfpathlineto{\pgfqpoint{1.165154in}{1.104194in}}%
\pgfpathlineto{\pgfqpoint{1.165549in}{1.102407in}}%
\pgfpathlineto{\pgfqpoint{1.167522in}{1.075097in}}%
\pgfpathlineto{\pgfqpoint{1.168312in}{1.082123in}}%
\pgfpathlineto{\pgfqpoint{1.170680in}{1.109313in}}%
\pgfpathlineto{\pgfqpoint{1.176207in}{1.170460in}}%
\pgfpathlineto{\pgfqpoint{1.176602in}{1.165545in}}%
\pgfpathlineto{\pgfqpoint{1.177391in}{1.158408in}}%
\pgfpathlineto{\pgfqpoint{1.178181in}{1.162225in}}%
\pgfpathlineto{\pgfqpoint{1.178575in}{1.164996in}}%
\pgfpathlineto{\pgfqpoint{1.179365in}{1.161158in}}%
\pgfpathlineto{\pgfqpoint{1.182128in}{1.133133in}}%
\pgfpathlineto{\pgfqpoint{1.182918in}{1.141710in}}%
\pgfpathlineto{\pgfqpoint{1.184891in}{1.174791in}}%
\pgfpathlineto{\pgfqpoint{1.185681in}{1.171584in}}%
\pgfpathlineto{\pgfqpoint{1.186865in}{1.153332in}}%
\pgfpathlineto{\pgfqpoint{1.188444in}{1.133761in}}%
\pgfpathlineto{\pgfqpoint{1.188839in}{1.137982in}}%
\pgfpathlineto{\pgfqpoint{1.190023in}{1.134616in}}%
\pgfpathlineto{\pgfqpoint{1.191997in}{1.142114in}}%
\pgfpathlineto{\pgfqpoint{1.193576in}{1.143132in}}%
\pgfpathlineto{\pgfqpoint{1.193971in}{1.142683in}}%
\pgfpathlineto{\pgfqpoint{1.195944in}{1.135067in}}%
\pgfpathlineto{\pgfqpoint{1.200287in}{1.062895in}}%
\pgfpathlineto{\pgfqpoint{1.201866in}{1.048039in}}%
\pgfpathlineto{\pgfqpoint{1.202260in}{1.048867in}}%
\pgfpathlineto{\pgfqpoint{1.204234in}{1.067602in}}%
\pgfpathlineto{\pgfqpoint{1.206603in}{1.104415in}}%
\pgfpathlineto{\pgfqpoint{1.207392in}{1.098044in}}%
\pgfpathlineto{\pgfqpoint{1.208182in}{1.093261in}}%
\pgfpathlineto{\pgfqpoint{1.208971in}{1.096355in}}%
\pgfpathlineto{\pgfqpoint{1.209760in}{1.097595in}}%
\pgfpathlineto{\pgfqpoint{1.210550in}{1.096464in}}%
\pgfpathlineto{\pgfqpoint{1.213313in}{1.086326in}}%
\pgfpathlineto{\pgfqpoint{1.214103in}{1.081305in}}%
\pgfpathlineto{\pgfqpoint{1.214497in}{1.084152in}}%
\pgfpathlineto{\pgfqpoint{1.218050in}{1.118547in}}%
\pgfpathlineto{\pgfqpoint{1.218840in}{1.107860in}}%
\pgfpathlineto{\pgfqpoint{1.220024in}{1.111875in}}%
\pgfpathlineto{\pgfqpoint{1.220813in}{1.116256in}}%
\pgfpathlineto{\pgfqpoint{1.221208in}{1.109995in}}%
\pgfpathlineto{\pgfqpoint{1.221998in}{1.088547in}}%
\pgfpathlineto{\pgfqpoint{1.223182in}{1.090119in}}%
\pgfpathlineto{\pgfqpoint{1.223971in}{1.090330in}}%
\pgfpathlineto{\pgfqpoint{1.227919in}{1.045291in}}%
\pgfpathlineto{\pgfqpoint{1.228314in}{1.050995in}}%
\pgfpathlineto{\pgfqpoint{1.229103in}{1.061120in}}%
\pgfpathlineto{\pgfqpoint{1.232261in}{1.098913in}}%
\pgfpathlineto{\pgfqpoint{1.233051in}{1.099132in}}%
\pgfpathlineto{\pgfqpoint{1.233445in}{1.088131in}}%
\pgfpathlineto{\pgfqpoint{1.234630in}{1.090881in}}%
\pgfpathlineto{\pgfqpoint{1.235024in}{1.090600in}}%
\pgfpathlineto{\pgfqpoint{1.238182in}{1.123301in}}%
\pgfpathlineto{\pgfqpoint{1.240156in}{1.151724in}}%
\pgfpathlineto{\pgfqpoint{1.240551in}{1.150467in}}%
\pgfpathlineto{\pgfqpoint{1.243709in}{1.121477in}}%
\pgfpathlineto{\pgfqpoint{1.244104in}{1.123606in}}%
\pgfpathlineto{\pgfqpoint{1.245683in}{1.120622in}}%
\pgfpathlineto{\pgfqpoint{1.246472in}{1.120927in}}%
\pgfpathlineto{\pgfqpoint{1.247656in}{1.116120in}}%
\pgfpathlineto{\pgfqpoint{1.250814in}{1.069265in}}%
\pgfpathlineto{\pgfqpoint{1.251604in}{1.075254in}}%
\pgfpathlineto{\pgfqpoint{1.251999in}{1.069173in}}%
\pgfpathlineto{\pgfqpoint{1.256736in}{0.993510in}}%
\pgfpathlineto{\pgfqpoint{1.257130in}{0.996317in}}%
\pgfpathlineto{\pgfqpoint{1.259104in}{1.003678in}}%
\pgfpathlineto{\pgfqpoint{1.259894in}{0.998992in}}%
\pgfpathlineto{\pgfqpoint{1.263052in}{1.046922in}}%
\pgfpathlineto{\pgfqpoint{1.263841in}{1.043956in}}%
\pgfpathlineto{\pgfqpoint{1.265025in}{1.038215in}}%
\pgfpathlineto{\pgfqpoint{1.267394in}{1.057719in}}%
\pgfpathlineto{\pgfqpoint{1.270552in}{1.103064in}}%
\pgfpathlineto{\pgfqpoint{1.271341in}{1.097936in}}%
\pgfpathlineto{\pgfqpoint{1.272131in}{1.088592in}}%
\pgfpathlineto{\pgfqpoint{1.272920in}{1.094147in}}%
\pgfpathlineto{\pgfqpoint{1.274499in}{1.111190in}}%
\pgfpathlineto{\pgfqpoint{1.274894in}{1.107153in}}%
\pgfpathlineto{\pgfqpoint{1.275289in}{1.102839in}}%
\pgfpathlineto{\pgfqpoint{1.276078in}{1.108961in}}%
\pgfpathlineto{\pgfqpoint{1.276868in}{1.113614in}}%
\pgfpathlineto{\pgfqpoint{1.277263in}{1.109085in}}%
\pgfpathlineto{\pgfqpoint{1.279236in}{1.082259in}}%
\pgfpathlineto{\pgfqpoint{1.280026in}{1.088808in}}%
\pgfpathlineto{\pgfqpoint{1.280815in}{1.093574in}}%
\pgfpathlineto{\pgfqpoint{1.281210in}{1.087398in}}%
\pgfpathlineto{\pgfqpoint{1.282394in}{1.088558in}}%
\pgfpathlineto{\pgfqpoint{1.283184in}{1.091344in}}%
\pgfpathlineto{\pgfqpoint{1.283579in}{1.099079in}}%
\pgfpathlineto{\pgfqpoint{1.284368in}{1.092164in}}%
\pgfpathlineto{\pgfqpoint{1.285552in}{1.079540in}}%
\pgfpathlineto{\pgfqpoint{1.286342in}{1.086880in}}%
\pgfpathlineto{\pgfqpoint{1.289105in}{1.110613in}}%
\pgfpathlineto{\pgfqpoint{1.290289in}{1.126135in}}%
\pgfpathlineto{\pgfqpoint{1.291079in}{1.118701in}}%
\pgfpathlineto{\pgfqpoint{1.293842in}{1.045816in}}%
\pgfpathlineto{\pgfqpoint{1.295026in}{1.054711in}}%
\pgfpathlineto{\pgfqpoint{1.297395in}{1.091243in}}%
\pgfpathlineto{\pgfqpoint{1.297789in}{1.090036in}}%
\pgfpathlineto{\pgfqpoint{1.301342in}{1.060329in}}%
\pgfpathlineto{\pgfqpoint{1.302132in}{1.061664in}}%
\pgfpathlineto{\pgfqpoint{1.302921in}{1.061199in}}%
\pgfpathlineto{\pgfqpoint{1.303316in}{1.061899in}}%
\pgfpathlineto{\pgfqpoint{1.303711in}{1.067467in}}%
\pgfpathlineto{\pgfqpoint{1.304500in}{1.059108in}}%
\pgfpathlineto{\pgfqpoint{1.305290in}{1.058177in}}%
\pgfpathlineto{\pgfqpoint{1.305684in}{1.059813in}}%
\pgfpathlineto{\pgfqpoint{1.307263in}{1.083843in}}%
\pgfpathlineto{\pgfqpoint{1.308448in}{1.080902in}}%
\pgfpathlineto{\pgfqpoint{1.309237in}{1.077066in}}%
\pgfpathlineto{\pgfqpoint{1.309632in}{1.084847in}}%
\pgfpathlineto{\pgfqpoint{1.310421in}{1.075138in}}%
\pgfpathlineto{\pgfqpoint{1.310816in}{1.074790in}}%
\pgfpathlineto{\pgfqpoint{1.315158in}{1.036552in}}%
\pgfpathlineto{\pgfqpoint{1.315553in}{1.036799in}}%
\pgfpathlineto{\pgfqpoint{1.316343in}{1.041185in}}%
\pgfpathlineto{\pgfqpoint{1.316737in}{1.037642in}}%
\pgfpathlineto{\pgfqpoint{1.322264in}{0.981627in}}%
\pgfpathlineto{\pgfqpoint{1.322659in}{0.973225in}}%
\pgfpathlineto{\pgfqpoint{1.323843in}{0.973557in}}%
\pgfpathlineto{\pgfqpoint{1.324238in}{0.973317in}}%
\pgfpathlineto{\pgfqpoint{1.325422in}{0.993174in}}%
\pgfpathlineto{\pgfqpoint{1.327396in}{1.039687in}}%
\pgfpathlineto{\pgfqpoint{1.327790in}{1.033909in}}%
\pgfpathlineto{\pgfqpoint{1.328580in}{1.021693in}}%
\pgfpathlineto{\pgfqpoint{1.329369in}{1.029054in}}%
\pgfpathlineto{\pgfqpoint{1.332133in}{1.060204in}}%
\pgfpathlineto{\pgfqpoint{1.332527in}{1.056567in}}%
\pgfpathlineto{\pgfqpoint{1.335291in}{1.024972in}}%
\pgfpathlineto{\pgfqpoint{1.336080in}{1.030596in}}%
\pgfpathlineto{\pgfqpoint{1.339238in}{1.078146in}}%
\pgfpathlineto{\pgfqpoint{1.340422in}{1.091769in}}%
\pgfpathlineto{\pgfqpoint{1.341607in}{1.090495in}}%
\pgfpathlineto{\pgfqpoint{1.343580in}{1.105268in}}%
\pgfpathlineto{\pgfqpoint{1.343975in}{1.099911in}}%
\pgfpathlineto{\pgfqpoint{1.348317in}{1.043398in}}%
\pgfpathlineto{\pgfqpoint{1.351475in}{0.996806in}}%
\pgfpathlineto{\pgfqpoint{1.352265in}{0.990528in}}%
\pgfpathlineto{\pgfqpoint{1.352660in}{0.993635in}}%
\pgfpathlineto{\pgfqpoint{1.355423in}{1.024437in}}%
\pgfpathlineto{\pgfqpoint{1.357002in}{1.009985in}}%
\pgfpathlineto{\pgfqpoint{1.358581in}{0.973441in}}%
\pgfpathlineto{\pgfqpoint{1.359370in}{0.981533in}}%
\pgfpathlineto{\pgfqpoint{1.362923in}{1.009086in}}%
\pgfpathlineto{\pgfqpoint{1.363318in}{1.002211in}}%
\pgfpathlineto{\pgfqpoint{1.364107in}{1.008061in}}%
\pgfpathlineto{\pgfqpoint{1.364502in}{1.009504in}}%
\pgfpathlineto{\pgfqpoint{1.364897in}{1.006553in}}%
\pgfpathlineto{\pgfqpoint{1.366476in}{0.996719in}}%
\pgfpathlineto{\pgfqpoint{1.367265in}{0.999336in}}%
\pgfpathlineto{\pgfqpoint{1.369239in}{1.011183in}}%
\pgfpathlineto{\pgfqpoint{1.370029in}{1.005773in}}%
\pgfpathlineto{\pgfqpoint{1.370818in}{1.003612in}}%
\pgfpathlineto{\pgfqpoint{1.371213in}{1.006107in}}%
\pgfpathlineto{\pgfqpoint{1.371608in}{1.010409in}}%
\pgfpathlineto{\pgfqpoint{1.372002in}{1.009886in}}%
\pgfpathlineto{\pgfqpoint{1.373581in}{0.982329in}}%
\pgfpathlineto{\pgfqpoint{1.373976in}{0.983581in}}%
\pgfpathlineto{\pgfqpoint{1.374766in}{0.997097in}}%
\pgfpathlineto{\pgfqpoint{1.375555in}{0.986374in}}%
\pgfpathlineto{\pgfqpoint{1.376739in}{0.981852in}}%
\pgfpathlineto{\pgfqpoint{1.377529in}{0.983819in}}%
\pgfpathlineto{\pgfqpoint{1.377923in}{0.983640in}}%
\pgfpathlineto{\pgfqpoint{1.378713in}{0.992120in}}%
\pgfpathlineto{\pgfqpoint{1.379502in}{0.987376in}}%
\pgfpathlineto{\pgfqpoint{1.382266in}{0.978762in}}%
\pgfpathlineto{\pgfqpoint{1.382660in}{0.979159in}}%
\pgfpathlineto{\pgfqpoint{1.385029in}{0.998059in}}%
\pgfpathlineto{\pgfqpoint{1.388976in}{1.025001in}}%
\pgfpathlineto{\pgfqpoint{1.389766in}{1.034530in}}%
\pgfpathlineto{\pgfqpoint{1.390950in}{1.045604in}}%
\pgfpathlineto{\pgfqpoint{1.391345in}{1.040329in}}%
\pgfpathlineto{\pgfqpoint{1.392134in}{1.045285in}}%
\pgfpathlineto{\pgfqpoint{1.392529in}{1.048790in}}%
\pgfpathlineto{\pgfqpoint{1.393319in}{1.042269in}}%
\pgfpathlineto{\pgfqpoint{1.395687in}{1.007713in}}%
\pgfpathlineto{\pgfqpoint{1.399240in}{0.959048in}}%
\pgfpathlineto{\pgfqpoint{1.402003in}{0.994834in}}%
\pgfpathlineto{\pgfqpoint{1.402398in}{1.001083in}}%
\pgfpathlineto{\pgfqpoint{1.403187in}{0.997425in}}%
\pgfpathlineto{\pgfqpoint{1.408319in}{0.955935in}}%
\pgfpathlineto{\pgfqpoint{1.409109in}{0.960181in}}%
\pgfpathlineto{\pgfqpoint{1.410293in}{0.966902in}}%
\pgfpathlineto{\pgfqpoint{1.411082in}{0.963366in}}%
\pgfpathlineto{\pgfqpoint{1.411477in}{0.961315in}}%
\pgfpathlineto{\pgfqpoint{1.411872in}{0.966487in}}%
\pgfpathlineto{\pgfqpoint{1.413451in}{0.978028in}}%
\pgfpathlineto{\pgfqpoint{1.413846in}{0.977965in}}%
\pgfpathlineto{\pgfqpoint{1.414240in}{0.977731in}}%
\pgfpathlineto{\pgfqpoint{1.414635in}{0.978825in}}%
\pgfpathlineto{\pgfqpoint{1.416609in}{0.982010in}}%
\pgfpathlineto{\pgfqpoint{1.417793in}{0.978730in}}%
\pgfpathlineto{\pgfqpoint{1.418188in}{0.976034in}}%
\pgfpathlineto{\pgfqpoint{1.418583in}{0.980242in}}%
\pgfpathlineto{\pgfqpoint{1.420951in}{1.000168in}}%
\pgfpathlineto{\pgfqpoint{1.425293in}{0.942037in}}%
\pgfpathlineto{\pgfqpoint{1.425688in}{0.946446in}}%
\pgfpathlineto{\pgfqpoint{1.426478in}{0.963208in}}%
\pgfpathlineto{\pgfqpoint{1.427267in}{0.952097in}}%
\pgfpathlineto{\pgfqpoint{1.428846in}{0.974898in}}%
\pgfpathlineto{\pgfqpoint{1.430030in}{0.972251in}}%
\pgfpathlineto{\pgfqpoint{1.430820in}{0.968602in}}%
\pgfpathlineto{\pgfqpoint{1.431215in}{0.971702in}}%
\pgfpathlineto{\pgfqpoint{1.432004in}{0.977691in}}%
\pgfpathlineto{\pgfqpoint{1.432399in}{0.973975in}}%
\pgfpathlineto{\pgfqpoint{1.433583in}{0.963299in}}%
\pgfpathlineto{\pgfqpoint{1.433978in}{0.965730in}}%
\pgfpathlineto{\pgfqpoint{1.435557in}{0.994523in}}%
\pgfpathlineto{\pgfqpoint{1.436346in}{0.988317in}}%
\pgfpathlineto{\pgfqpoint{1.436741in}{0.988257in}}%
\pgfpathlineto{\pgfqpoint{1.438715in}{1.016216in}}%
\pgfpathlineto{\pgfqpoint{1.440294in}{0.996938in}}%
\pgfpathlineto{\pgfqpoint{1.440689in}{0.997768in}}%
\pgfpathlineto{\pgfqpoint{1.442268in}{1.004440in}}%
\pgfpathlineto{\pgfqpoint{1.442662in}{1.004267in}}%
\pgfpathlineto{\pgfqpoint{1.444241in}{0.998283in}}%
\pgfpathlineto{\pgfqpoint{1.445426in}{0.989718in}}%
\pgfpathlineto{\pgfqpoint{1.445820in}{0.995732in}}%
\pgfpathlineto{\pgfqpoint{1.446215in}{0.996577in}}%
\pgfpathlineto{\pgfqpoint{1.447005in}{0.994840in}}%
\pgfpathlineto{\pgfqpoint{1.449768in}{0.958449in}}%
\pgfpathlineto{\pgfqpoint{1.450557in}{0.958503in}}%
\pgfpathlineto{\pgfqpoint{1.452531in}{0.962399in}}%
\pgfpathlineto{\pgfqpoint{1.454110in}{0.944474in}}%
\pgfpathlineto{\pgfqpoint{1.454900in}{0.948582in}}%
\pgfpathlineto{\pgfqpoint{1.455294in}{0.947427in}}%
\pgfpathlineto{\pgfqpoint{1.458452in}{0.994393in}}%
\pgfpathlineto{\pgfqpoint{1.459242in}{0.982225in}}%
\pgfpathlineto{\pgfqpoint{1.460821in}{0.969811in}}%
\pgfpathlineto{\pgfqpoint{1.463189in}{0.940267in}}%
\pgfpathlineto{\pgfqpoint{1.463979in}{0.941426in}}%
\pgfpathlineto{\pgfqpoint{1.464373in}{0.941889in}}%
\pgfpathlineto{\pgfqpoint{1.465558in}{0.960677in}}%
\pgfpathlineto{\pgfqpoint{1.466347in}{0.947754in}}%
\pgfpathlineto{\pgfqpoint{1.467137in}{0.939501in}}%
\pgfpathlineto{\pgfqpoint{1.467531in}{0.945501in}}%
\pgfpathlineto{\pgfqpoint{1.469110in}{0.967666in}}%
\pgfpathlineto{\pgfqpoint{1.469505in}{0.967003in}}%
\pgfpathlineto{\pgfqpoint{1.469900in}{0.965906in}}%
\pgfpathlineto{\pgfqpoint{1.471874in}{0.980162in}}%
\pgfpathlineto{\pgfqpoint{1.477005in}{0.906422in}}%
\pgfpathlineto{\pgfqpoint{1.477795in}{0.903737in}}%
\pgfpathlineto{\pgfqpoint{1.478190in}{0.906825in}}%
\pgfpathlineto{\pgfqpoint{1.478979in}{0.915815in}}%
\pgfpathlineto{\pgfqpoint{1.482137in}{0.967674in}}%
\pgfpathlineto{\pgfqpoint{1.484900in}{0.948815in}}%
\pgfpathlineto{\pgfqpoint{1.485690in}{0.963130in}}%
\pgfpathlineto{\pgfqpoint{1.486479in}{0.949996in}}%
\pgfpathlineto{\pgfqpoint{1.487269in}{0.944592in}}%
\pgfpathlineto{\pgfqpoint{1.487664in}{0.948942in}}%
\pgfpathlineto{\pgfqpoint{1.489637in}{0.962630in}}%
\pgfpathlineto{\pgfqpoint{1.490427in}{0.961762in}}%
\pgfpathlineto{\pgfqpoint{1.494374in}{0.909515in}}%
\pgfpathlineto{\pgfqpoint{1.495164in}{0.924489in}}%
\pgfpathlineto{\pgfqpoint{1.496743in}{0.922151in}}%
\pgfpathlineto{\pgfqpoint{1.497532in}{0.913972in}}%
\pgfpathlineto{\pgfqpoint{1.497927in}{0.922316in}}%
\pgfpathlineto{\pgfqpoint{1.498322in}{0.933280in}}%
\pgfpathlineto{\pgfqpoint{1.499506in}{0.931103in}}%
\pgfpathlineto{\pgfqpoint{1.501480in}{0.923339in}}%
\pgfpathlineto{\pgfqpoint{1.502269in}{0.937072in}}%
\pgfpathlineto{\pgfqpoint{1.503059in}{0.930951in}}%
\pgfpathlineto{\pgfqpoint{1.505822in}{0.900876in}}%
\pgfpathlineto{\pgfqpoint{1.506217in}{0.904028in}}%
\pgfpathlineto{\pgfqpoint{1.509375in}{0.959180in}}%
\pgfpathlineto{\pgfqpoint{1.509770in}{0.959090in}}%
\pgfpathlineto{\pgfqpoint{1.510164in}{0.958444in}}%
\pgfpathlineto{\pgfqpoint{1.510559in}{0.959234in}}%
\pgfpathlineto{\pgfqpoint{1.511349in}{0.965618in}}%
\pgfpathlineto{\pgfqpoint{1.512138in}{0.959529in}}%
\pgfpathlineto{\pgfqpoint{1.515296in}{0.912470in}}%
\pgfpathlineto{\pgfqpoint{1.515691in}{0.915370in}}%
\pgfpathlineto{\pgfqpoint{1.516875in}{0.948405in}}%
\pgfpathlineto{\pgfqpoint{1.517665in}{0.947463in}}%
\pgfpathlineto{\pgfqpoint{1.519244in}{0.908567in}}%
\pgfpathlineto{\pgfqpoint{1.520428in}{0.910735in}}%
\pgfpathlineto{\pgfqpoint{1.523586in}{0.959274in}}%
\pgfpathlineto{\pgfqpoint{1.524375in}{0.948947in}}%
\pgfpathlineto{\pgfqpoint{1.525165in}{0.935845in}}%
\pgfpathlineto{\pgfqpoint{1.525954in}{0.942363in}}%
\pgfpathlineto{\pgfqpoint{1.526349in}{0.945386in}}%
\pgfpathlineto{\pgfqpoint{1.526744in}{0.941263in}}%
\pgfpathlineto{\pgfqpoint{1.529112in}{0.922466in}}%
\pgfpathlineto{\pgfqpoint{1.532270in}{0.955933in}}%
\pgfpathlineto{\pgfqpoint{1.532665in}{0.956736in}}%
\pgfpathlineto{\pgfqpoint{1.535823in}{0.989705in}}%
\pgfpathlineto{\pgfqpoint{1.536218in}{0.989061in}}%
\pgfpathlineto{\pgfqpoint{1.539376in}{0.960238in}}%
\pgfpathlineto{\pgfqpoint{1.539771in}{0.968279in}}%
\pgfpathlineto{\pgfqpoint{1.540165in}{0.975325in}}%
\pgfpathlineto{\pgfqpoint{1.540955in}{0.970813in}}%
\pgfpathlineto{\pgfqpoint{1.542534in}{0.953927in}}%
\pgfpathlineto{\pgfqpoint{1.542929in}{0.955036in}}%
\pgfpathlineto{\pgfqpoint{1.545297in}{0.963805in}}%
\pgfpathlineto{\pgfqpoint{1.546481in}{0.962772in}}%
\pgfpathlineto{\pgfqpoint{1.548060in}{0.953904in}}%
\pgfpathlineto{\pgfqpoint{1.549639in}{0.925020in}}%
\pgfpathlineto{\pgfqpoint{1.550823in}{0.931882in}}%
\pgfpathlineto{\pgfqpoint{1.551613in}{0.927157in}}%
\pgfpathlineto{\pgfqpoint{1.553587in}{0.887107in}}%
\pgfpathlineto{\pgfqpoint{1.554771in}{0.891550in}}%
\pgfpathlineto{\pgfqpoint{1.555955in}{0.903078in}}%
\pgfpathlineto{\pgfqpoint{1.556745in}{0.895789in}}%
\pgfpathlineto{\pgfqpoint{1.557929in}{0.889672in}}%
\pgfpathlineto{\pgfqpoint{1.559903in}{0.873448in}}%
\pgfpathlineto{\pgfqpoint{1.560692in}{0.875787in}}%
\pgfpathlineto{\pgfqpoint{1.561087in}{0.883506in}}%
\pgfpathlineto{\pgfqpoint{1.562271in}{0.879436in}}%
\pgfpathlineto{\pgfqpoint{1.563455in}{0.886041in}}%
\pgfpathlineto{\pgfqpoint{1.563850in}{0.881224in}}%
\pgfpathlineto{\pgfqpoint{1.566219in}{0.842915in}}%
\pgfpathlineto{\pgfqpoint{1.567403in}{0.850946in}}%
\pgfpathlineto{\pgfqpoint{1.570166in}{0.919113in}}%
\pgfpathlineto{\pgfqpoint{1.570956in}{0.915054in}}%
\pgfpathlineto{\pgfqpoint{1.571745in}{0.908009in}}%
\pgfpathlineto{\pgfqpoint{1.572140in}{0.910605in}}%
\pgfpathlineto{\pgfqpoint{1.573719in}{0.944311in}}%
\pgfpathlineto{\pgfqpoint{1.574508in}{0.955243in}}%
\pgfpathlineto{\pgfqpoint{1.574903in}{0.949962in}}%
\pgfpathlineto{\pgfqpoint{1.577272in}{0.929441in}}%
\pgfpathlineto{\pgfqpoint{1.582009in}{0.982837in}}%
\pgfpathlineto{\pgfqpoint{1.583193in}{0.972069in}}%
\pgfpathlineto{\pgfqpoint{1.585561in}{0.944653in}}%
\pgfpathlineto{\pgfqpoint{1.586746in}{0.946184in}}%
\pgfpathlineto{\pgfqpoint{1.587930in}{0.930151in}}%
\pgfpathlineto{\pgfqpoint{1.590298in}{0.886932in}}%
\pgfpathlineto{\pgfqpoint{1.591088in}{0.887932in}}%
\pgfpathlineto{\pgfqpoint{1.592272in}{0.896263in}}%
\pgfpathlineto{\pgfqpoint{1.593851in}{0.912210in}}%
\pgfpathlineto{\pgfqpoint{1.594641in}{0.905968in}}%
\pgfpathlineto{\pgfqpoint{1.595035in}{0.906297in}}%
\pgfpathlineto{\pgfqpoint{1.598983in}{0.938225in}}%
\pgfpathlineto{\pgfqpoint{1.600167in}{0.932891in}}%
\pgfpathlineto{\pgfqpoint{1.602141in}{0.921146in}}%
\pgfpathlineto{\pgfqpoint{1.602536in}{0.923017in}}%
\pgfpathlineto{\pgfqpoint{1.605299in}{0.934450in}}%
\pgfpathlineto{\pgfqpoint{1.606088in}{0.933398in}}%
\pgfpathlineto{\pgfqpoint{1.607667in}{0.924730in}}%
\pgfpathlineto{\pgfqpoint{1.610825in}{0.885686in}}%
\pgfpathlineto{\pgfqpoint{1.613194in}{0.861782in}}%
\pgfpathlineto{\pgfqpoint{1.614378in}{0.872715in}}%
\pgfpathlineto{\pgfqpoint{1.614773in}{0.868321in}}%
\pgfpathlineto{\pgfqpoint{1.616352in}{0.864322in}}%
\pgfpathlineto{\pgfqpoint{1.617931in}{0.838737in}}%
\pgfpathlineto{\pgfqpoint{1.619115in}{0.839765in}}%
\pgfpathlineto{\pgfqpoint{1.619905in}{0.832790in}}%
\pgfpathlineto{\pgfqpoint{1.621089in}{0.836850in}}%
\pgfpathlineto{\pgfqpoint{1.621878in}{0.837066in}}%
\pgfpathlineto{\pgfqpoint{1.625431in}{0.881810in}}%
\pgfpathlineto{\pgfqpoint{1.626615in}{0.881372in}}%
\pgfpathlineto{\pgfqpoint{1.627799in}{0.884712in}}%
\pgfpathlineto{\pgfqpoint{1.628194in}{0.881759in}}%
\pgfpathlineto{\pgfqpoint{1.629378in}{0.874621in}}%
\pgfpathlineto{\pgfqpoint{1.629773in}{0.877968in}}%
\pgfpathlineto{\pgfqpoint{1.632142in}{0.900969in}}%
\pgfpathlineto{\pgfqpoint{1.633326in}{0.930644in}}%
\pgfpathlineto{\pgfqpoint{1.634115in}{0.925575in}}%
\pgfpathlineto{\pgfqpoint{1.635300in}{0.906423in}}%
\pgfpathlineto{\pgfqpoint{1.636089in}{0.888032in}}%
\pgfpathlineto{\pgfqpoint{1.637273in}{0.888371in}}%
\pgfpathlineto{\pgfqpoint{1.639642in}{0.910685in}}%
\pgfpathlineto{\pgfqpoint{1.640826in}{0.903784in}}%
\pgfpathlineto{\pgfqpoint{1.642405in}{0.880416in}}%
\pgfpathlineto{\pgfqpoint{1.642800in}{0.886482in}}%
\pgfpathlineto{\pgfqpoint{1.643195in}{0.886125in}}%
\pgfpathlineto{\pgfqpoint{1.647537in}{0.926602in}}%
\pgfpathlineto{\pgfqpoint{1.647932in}{0.925101in}}%
\pgfpathlineto{\pgfqpoint{1.651090in}{0.877199in}}%
\pgfpathlineto{\pgfqpoint{1.651484in}{0.879274in}}%
\pgfpathlineto{\pgfqpoint{1.652669in}{0.884767in}}%
\pgfpathlineto{\pgfqpoint{1.653458in}{0.872124in}}%
\pgfpathlineto{\pgfqpoint{1.654642in}{0.875247in}}%
\pgfpathlineto{\pgfqpoint{1.655432in}{0.886232in}}%
\pgfpathlineto{\pgfqpoint{1.655827in}{0.878090in}}%
\pgfpathlineto{\pgfqpoint{1.656221in}{0.870577in}}%
\pgfpathlineto{\pgfqpoint{1.657011in}{0.881789in}}%
\pgfpathlineto{\pgfqpoint{1.658195in}{0.889516in}}%
\pgfpathlineto{\pgfqpoint{1.658590in}{0.882753in}}%
\pgfpathlineto{\pgfqpoint{1.659774in}{0.853163in}}%
\pgfpathlineto{\pgfqpoint{1.660958in}{0.856908in}}%
\pgfpathlineto{\pgfqpoint{1.662537in}{0.868346in}}%
\pgfpathlineto{\pgfqpoint{1.665695in}{0.904659in}}%
\pgfpathlineto{\pgfqpoint{1.666090in}{0.905199in}}%
\pgfpathlineto{\pgfqpoint{1.668459in}{0.921330in}}%
\pgfpathlineto{\pgfqpoint{1.669643in}{0.922800in}}%
\pgfpathlineto{\pgfqpoint{1.670038in}{0.921446in}}%
\pgfpathlineto{\pgfqpoint{1.672406in}{0.913125in}}%
\pgfpathlineto{\pgfqpoint{1.672801in}{0.913707in}}%
\pgfpathlineto{\pgfqpoint{1.675169in}{0.905558in}}%
\pgfpathlineto{\pgfqpoint{1.676748in}{0.893223in}}%
\pgfpathlineto{\pgfqpoint{1.683459in}{0.824457in}}%
\pgfpathlineto{\pgfqpoint{1.683854in}{0.824909in}}%
\pgfpathlineto{\pgfqpoint{1.685828in}{0.879084in}}%
\pgfpathlineto{\pgfqpoint{1.686222in}{0.888944in}}%
\pgfpathlineto{\pgfqpoint{1.687012in}{0.875514in}}%
\pgfpathlineto{\pgfqpoint{1.690170in}{0.825833in}}%
\pgfpathlineto{\pgfqpoint{1.690565in}{0.823550in}}%
\pgfpathlineto{\pgfqpoint{1.690959in}{0.828784in}}%
\pgfpathlineto{\pgfqpoint{1.698065in}{0.892062in}}%
\pgfpathlineto{\pgfqpoint{1.700039in}{0.866568in}}%
\pgfpathlineto{\pgfqpoint{1.701223in}{0.853007in}}%
\pgfpathlineto{\pgfqpoint{1.701618in}{0.860287in}}%
\pgfpathlineto{\pgfqpoint{1.702012in}{0.861074in}}%
\pgfpathlineto{\pgfqpoint{1.702407in}{0.860234in}}%
\pgfpathlineto{\pgfqpoint{1.703197in}{0.853965in}}%
\pgfpathlineto{\pgfqpoint{1.705565in}{0.825900in}}%
\pgfpathlineto{\pgfqpoint{1.706749in}{0.829488in}}%
\pgfpathlineto{\pgfqpoint{1.707144in}{0.830254in}}%
\pgfpathlineto{\pgfqpoint{1.707934in}{0.823180in}}%
\pgfpathlineto{\pgfqpoint{1.708328in}{0.829896in}}%
\pgfpathlineto{\pgfqpoint{1.709513in}{0.841356in}}%
\pgfpathlineto{\pgfqpoint{1.710302in}{0.841235in}}%
\pgfpathlineto{\pgfqpoint{1.711486in}{0.836086in}}%
\pgfpathlineto{\pgfqpoint{1.713065in}{0.838069in}}%
\pgfpathlineto{\pgfqpoint{1.716223in}{0.843851in}}%
\pgfpathlineto{\pgfqpoint{1.723329in}{0.899326in}}%
\pgfpathlineto{\pgfqpoint{1.724118in}{0.896758in}}%
\pgfpathlineto{\pgfqpoint{1.727671in}{0.878691in}}%
\pgfpathlineto{\pgfqpoint{1.730039in}{0.865061in}}%
\pgfpathlineto{\pgfqpoint{1.730434in}{0.865514in}}%
\pgfpathlineto{\pgfqpoint{1.731618in}{0.869154in}}%
\pgfpathlineto{\pgfqpoint{1.732013in}{0.864076in}}%
\pgfpathlineto{\pgfqpoint{1.732803in}{0.868012in}}%
\pgfpathlineto{\pgfqpoint{1.733197in}{0.868968in}}%
\pgfpathlineto{\pgfqpoint{1.733592in}{0.867705in}}%
\pgfpathlineto{\pgfqpoint{1.734776in}{0.856928in}}%
\pgfpathlineto{\pgfqpoint{1.735961in}{0.841643in}}%
\pgfpathlineto{\pgfqpoint{1.736750in}{0.847909in}}%
\pgfpathlineto{\pgfqpoint{1.738724in}{0.853717in}}%
\pgfpathlineto{\pgfqpoint{1.741487in}{0.819769in}}%
\pgfpathlineto{\pgfqpoint{1.741882in}{0.824893in}}%
\pgfpathlineto{\pgfqpoint{1.742277in}{0.828368in}}%
\pgfpathlineto{\pgfqpoint{1.743066in}{0.823397in}}%
\pgfpathlineto{\pgfqpoint{1.744250in}{0.811804in}}%
\pgfpathlineto{\pgfqpoint{1.745040in}{0.814687in}}%
\pgfpathlineto{\pgfqpoint{1.745435in}{0.818389in}}%
\pgfpathlineto{\pgfqpoint{1.745829in}{0.810854in}}%
\pgfpathlineto{\pgfqpoint{1.747014in}{0.803378in}}%
\pgfpathlineto{\pgfqpoint{1.749777in}{0.841442in}}%
\pgfpathlineto{\pgfqpoint{1.750566in}{0.838448in}}%
\pgfpathlineto{\pgfqpoint{1.750961in}{0.841711in}}%
\pgfpathlineto{\pgfqpoint{1.753724in}{0.870166in}}%
\pgfpathlineto{\pgfqpoint{1.754514in}{0.867489in}}%
\pgfpathlineto{\pgfqpoint{1.755303in}{0.858638in}}%
\pgfpathlineto{\pgfqpoint{1.756488in}{0.860187in}}%
\pgfpathlineto{\pgfqpoint{1.757277in}{0.857441in}}%
\pgfpathlineto{\pgfqpoint{1.757672in}{0.860933in}}%
\pgfpathlineto{\pgfqpoint{1.758461in}{0.859466in}}%
\pgfpathlineto{\pgfqpoint{1.760435in}{0.841691in}}%
\pgfpathlineto{\pgfqpoint{1.761619in}{0.818499in}}%
\pgfpathlineto{\pgfqpoint{1.762014in}{0.828078in}}%
\pgfpathlineto{\pgfqpoint{1.763988in}{0.846022in}}%
\pgfpathlineto{\pgfqpoint{1.764777in}{0.839722in}}%
\pgfpathlineto{\pgfqpoint{1.765172in}{0.836553in}}%
\pgfpathlineto{\pgfqpoint{1.765567in}{0.838041in}}%
\pgfpathlineto{\pgfqpoint{1.765962in}{0.844804in}}%
\pgfpathlineto{\pgfqpoint{1.766751in}{0.836679in}}%
\pgfpathlineto{\pgfqpoint{1.767541in}{0.830088in}}%
\pgfpathlineto{\pgfqpoint{1.768330in}{0.835035in}}%
\pgfpathlineto{\pgfqpoint{1.768725in}{0.836576in}}%
\pgfpathlineto{\pgfqpoint{1.769514in}{0.828992in}}%
\pgfpathlineto{\pgfqpoint{1.770304in}{0.831244in}}%
\pgfpathlineto{\pgfqpoint{1.772278in}{0.857830in}}%
\pgfpathlineto{\pgfqpoint{1.772672in}{0.857772in}}%
\pgfpathlineto{\pgfqpoint{1.773462in}{0.861609in}}%
\pgfpathlineto{\pgfqpoint{1.774251in}{0.866287in}}%
\pgfpathlineto{\pgfqpoint{1.774646in}{0.862691in}}%
\pgfpathlineto{\pgfqpoint{1.775830in}{0.846061in}}%
\pgfpathlineto{\pgfqpoint{1.776620in}{0.854792in}}%
\pgfpathlineto{\pgfqpoint{1.777804in}{0.863460in}}%
\pgfpathlineto{\pgfqpoint{1.778199in}{0.856922in}}%
\pgfpathlineto{\pgfqpoint{1.779383in}{0.845406in}}%
\pgfpathlineto{\pgfqpoint{1.780567in}{0.848541in}}%
\pgfpathlineto{\pgfqpoint{1.780962in}{0.850067in}}%
\pgfpathlineto{\pgfqpoint{1.781357in}{0.849108in}}%
\pgfpathlineto{\pgfqpoint{1.781752in}{0.846638in}}%
\pgfpathlineto{\pgfqpoint{1.782146in}{0.850159in}}%
\pgfpathlineto{\pgfqpoint{1.782541in}{0.856579in}}%
\pgfpathlineto{\pgfqpoint{1.783331in}{0.849778in}}%
\pgfpathlineto{\pgfqpoint{1.783725in}{0.849723in}}%
\pgfpathlineto{\pgfqpoint{1.784910in}{0.858240in}}%
\pgfpathlineto{\pgfqpoint{1.785304in}{0.854412in}}%
\pgfpathlineto{\pgfqpoint{1.786489in}{0.849218in}}%
\pgfpathlineto{\pgfqpoint{1.788068in}{0.871702in}}%
\pgfpathlineto{\pgfqpoint{1.788462in}{0.869138in}}%
\pgfpathlineto{\pgfqpoint{1.791620in}{0.852421in}}%
\pgfpathlineto{\pgfqpoint{1.792410in}{0.848064in}}%
\pgfpathlineto{\pgfqpoint{1.796752in}{0.800165in}}%
\pgfpathlineto{\pgfqpoint{1.797541in}{0.801921in}}%
\pgfpathlineto{\pgfqpoint{1.798726in}{0.823128in}}%
\pgfpathlineto{\pgfqpoint{1.799515in}{0.822398in}}%
\pgfpathlineto{\pgfqpoint{1.799910in}{0.817457in}}%
\pgfpathlineto{\pgfqpoint{1.800699in}{0.823552in}}%
\pgfpathlineto{\pgfqpoint{1.801094in}{0.823093in}}%
\pgfpathlineto{\pgfqpoint{1.801884in}{0.808510in}}%
\pgfpathlineto{\pgfqpoint{1.802673in}{0.813298in}}%
\pgfpathlineto{\pgfqpoint{1.805042in}{0.831086in}}%
\pgfpathlineto{\pgfqpoint{1.806226in}{0.836205in}}%
\pgfpathlineto{\pgfqpoint{1.806621in}{0.832717in}}%
\pgfpathlineto{\pgfqpoint{1.807410in}{0.824504in}}%
\pgfpathlineto{\pgfqpoint{1.808200in}{0.831659in}}%
\pgfpathlineto{\pgfqpoint{1.808594in}{0.833005in}}%
\pgfpathlineto{\pgfqpoint{1.809384in}{0.825651in}}%
\pgfpathlineto{\pgfqpoint{1.810173in}{0.831222in}}%
\pgfpathlineto{\pgfqpoint{1.812542in}{0.810217in}}%
\pgfpathlineto{\pgfqpoint{1.812937in}{0.817136in}}%
\pgfpathlineto{\pgfqpoint{1.816095in}{0.851419in}}%
\pgfpathlineto{\pgfqpoint{1.820832in}{0.890040in}}%
\pgfpathlineto{\pgfqpoint{1.822016in}{0.882957in}}%
\pgfpathlineto{\pgfqpoint{1.827542in}{0.847848in}}%
\pgfpathlineto{\pgfqpoint{1.829516in}{0.863618in}}%
\pgfpathlineto{\pgfqpoint{1.829911in}{0.862679in}}%
\pgfpathlineto{\pgfqpoint{1.834253in}{0.832466in}}%
\pgfpathlineto{\pgfqpoint{1.835832in}{0.834937in}}%
\pgfpathlineto{\pgfqpoint{1.837411in}{0.825142in}}%
\pgfpathlineto{\pgfqpoint{1.837806in}{0.830067in}}%
\pgfpathlineto{\pgfqpoint{1.838201in}{0.834321in}}%
\pgfpathlineto{\pgfqpoint{1.838595in}{0.829573in}}%
\pgfpathlineto{\pgfqpoint{1.838990in}{0.831633in}}%
\pgfpathlineto{\pgfqpoint{1.839780in}{0.826638in}}%
\pgfpathlineto{\pgfqpoint{1.840174in}{0.826825in}}%
\pgfpathlineto{\pgfqpoint{1.840964in}{0.833864in}}%
\pgfpathlineto{\pgfqpoint{1.841753in}{0.829285in}}%
\pgfpathlineto{\pgfqpoint{1.843727in}{0.806868in}}%
\pgfpathlineto{\pgfqpoint{1.844911in}{0.813003in}}%
\pgfpathlineto{\pgfqpoint{1.845306in}{0.813963in}}%
\pgfpathlineto{\pgfqpoint{1.845701in}{0.804856in}}%
\pgfpathlineto{\pgfqpoint{1.846885in}{0.811667in}}%
\pgfpathlineto{\pgfqpoint{1.849254in}{0.838691in}}%
\pgfpathlineto{\pgfqpoint{1.851622in}{0.856940in}}%
\pgfpathlineto{\pgfqpoint{1.852017in}{0.855311in}}%
\pgfpathlineto{\pgfqpoint{1.853596in}{0.843186in}}%
\pgfpathlineto{\pgfqpoint{1.856359in}{0.815200in}}%
\pgfpathlineto{\pgfqpoint{1.856754in}{0.815694in}}%
\pgfpathlineto{\pgfqpoint{1.857938in}{0.835338in}}%
\pgfpathlineto{\pgfqpoint{1.858728in}{0.827836in}}%
\pgfpathlineto{\pgfqpoint{1.859912in}{0.811462in}}%
\pgfpathlineto{\pgfqpoint{1.860307in}{0.812934in}}%
\pgfpathlineto{\pgfqpoint{1.860701in}{0.823999in}}%
\pgfpathlineto{\pgfqpoint{1.861491in}{0.808266in}}%
\pgfpathlineto{\pgfqpoint{1.863465in}{0.771829in}}%
\pgfpathlineto{\pgfqpoint{1.864254in}{0.773951in}}%
\pgfpathlineto{\pgfqpoint{1.867807in}{0.800155in}}%
\pgfpathlineto{\pgfqpoint{1.869386in}{0.786977in}}%
\pgfpathlineto{\pgfqpoint{1.869781in}{0.789775in}}%
\pgfpathlineto{\pgfqpoint{1.871754in}{0.841010in}}%
\pgfpathlineto{\pgfqpoint{1.872544in}{0.834074in}}%
\pgfpathlineto{\pgfqpoint{1.873728in}{0.816078in}}%
\pgfpathlineto{\pgfqpoint{1.874518in}{0.823889in}}%
\pgfpathlineto{\pgfqpoint{1.875702in}{0.834825in}}%
\pgfpathlineto{\pgfqpoint{1.876097in}{0.829411in}}%
\pgfpathlineto{\pgfqpoint{1.879254in}{0.769163in}}%
\pgfpathlineto{\pgfqpoint{1.879649in}{0.770845in}}%
\pgfpathlineto{\pgfqpoint{1.882412in}{0.810576in}}%
\pgfpathlineto{\pgfqpoint{1.882807in}{0.817102in}}%
\pgfpathlineto{\pgfqpoint{1.883991in}{0.813413in}}%
\pgfpathlineto{\pgfqpoint{1.885176in}{0.825991in}}%
\pgfpathlineto{\pgfqpoint{1.885570in}{0.821600in}}%
\pgfpathlineto{\pgfqpoint{1.888728in}{0.796214in}}%
\pgfpathlineto{\pgfqpoint{1.890702in}{0.788834in}}%
\pgfpathlineto{\pgfqpoint{1.891097in}{0.792133in}}%
\pgfpathlineto{\pgfqpoint{1.893465in}{0.819425in}}%
\pgfpathlineto{\pgfqpoint{1.895439in}{0.831531in}}%
\pgfpathlineto{\pgfqpoint{1.896623in}{0.833382in}}%
\pgfpathlineto{\pgfqpoint{1.905308in}{0.776712in}}%
\pgfpathlineto{\pgfqpoint{1.906097in}{0.783071in}}%
\pgfpathlineto{\pgfqpoint{1.906887in}{0.779597in}}%
\pgfpathlineto{\pgfqpoint{1.907282in}{0.782192in}}%
\pgfpathlineto{\pgfqpoint{1.909650in}{0.801772in}}%
\pgfpathlineto{\pgfqpoint{1.910834in}{0.811040in}}%
\pgfpathlineto{\pgfqpoint{1.911229in}{0.809269in}}%
\pgfpathlineto{\pgfqpoint{1.912808in}{0.794907in}}%
\pgfpathlineto{\pgfqpoint{1.913598in}{0.799507in}}%
\pgfpathlineto{\pgfqpoint{1.917545in}{0.821005in}}%
\pgfpathlineto{\pgfqpoint{1.919124in}{0.825520in}}%
\pgfpathlineto{\pgfqpoint{1.919519in}{0.824840in}}%
\pgfpathlineto{\pgfqpoint{1.921098in}{0.814052in}}%
\pgfpathlineto{\pgfqpoint{1.922282in}{0.799421in}}%
\pgfpathlineto{\pgfqpoint{1.923072in}{0.802829in}}%
\pgfpathlineto{\pgfqpoint{1.923861in}{0.804176in}}%
\pgfpathlineto{\pgfqpoint{1.924256in}{0.802501in}}%
\pgfpathlineto{\pgfqpoint{1.925045in}{0.800791in}}%
\pgfpathlineto{\pgfqpoint{1.926230in}{0.809403in}}%
\pgfpathlineto{\pgfqpoint{1.927414in}{0.781431in}}%
\pgfpathlineto{\pgfqpoint{1.928598in}{0.785440in}}%
\pgfpathlineto{\pgfqpoint{1.929388in}{0.790827in}}%
\pgfpathlineto{\pgfqpoint{1.930967in}{0.810347in}}%
\pgfpathlineto{\pgfqpoint{1.931756in}{0.802799in}}%
\pgfpathlineto{\pgfqpoint{1.934125in}{0.790568in}}%
\pgfpathlineto{\pgfqpoint{1.934914in}{0.801155in}}%
\pgfpathlineto{\pgfqpoint{1.935704in}{0.796161in}}%
\pgfpathlineto{\pgfqpoint{1.936888in}{0.785951in}}%
\pgfpathlineto{\pgfqpoint{1.937283in}{0.791705in}}%
\pgfpathlineto{\pgfqpoint{1.939256in}{0.799874in}}%
\pgfpathlineto{\pgfqpoint{1.940046in}{0.804779in}}%
\pgfpathlineto{\pgfqpoint{1.940835in}{0.807672in}}%
\pgfpathlineto{\pgfqpoint{1.941230in}{0.804451in}}%
\pgfpathlineto{\pgfqpoint{1.942020in}{0.797547in}}%
\pgfpathlineto{\pgfqpoint{1.942414in}{0.800751in}}%
\pgfpathlineto{\pgfqpoint{1.943599in}{0.813514in}}%
\pgfpathlineto{\pgfqpoint{1.943993in}{0.807623in}}%
\pgfpathlineto{\pgfqpoint{1.944783in}{0.796105in}}%
\pgfpathlineto{\pgfqpoint{1.945572in}{0.799489in}}%
\pgfpathlineto{\pgfqpoint{1.946362in}{0.797304in}}%
\pgfpathlineto{\pgfqpoint{1.947546in}{0.782219in}}%
\pgfpathlineto{\pgfqpoint{1.949520in}{0.747553in}}%
\pgfpathlineto{\pgfqpoint{1.950309in}{0.750587in}}%
\pgfpathlineto{\pgfqpoint{1.952283in}{0.785502in}}%
\pgfpathlineto{\pgfqpoint{1.953467in}{0.784217in}}%
\pgfpathlineto{\pgfqpoint{1.953862in}{0.785962in}}%
\pgfpathlineto{\pgfqpoint{1.954257in}{0.783410in}}%
\pgfpathlineto{\pgfqpoint{1.955046in}{0.777307in}}%
\pgfpathlineto{\pgfqpoint{1.955836in}{0.780409in}}%
\pgfpathlineto{\pgfqpoint{1.956625in}{0.787144in}}%
\pgfpathlineto{\pgfqpoint{1.957415in}{0.780887in}}%
\pgfpathlineto{\pgfqpoint{1.957810in}{0.776999in}}%
\pgfpathlineto{\pgfqpoint{1.958599in}{0.780801in}}%
\pgfpathlineto{\pgfqpoint{1.958994in}{0.781648in}}%
\pgfpathlineto{\pgfqpoint{1.959389in}{0.779482in}}%
\pgfpathlineto{\pgfqpoint{1.962152in}{0.768515in}}%
\pgfpathlineto{\pgfqpoint{1.964520in}{0.786824in}}%
\pgfpathlineto{\pgfqpoint{1.966099in}{0.783613in}}%
\pgfpathlineto{\pgfqpoint{1.967283in}{0.763885in}}%
\pgfpathlineto{\pgfqpoint{1.968073in}{0.769249in}}%
\pgfpathlineto{\pgfqpoint{1.970441in}{0.823892in}}%
\pgfpathlineto{\pgfqpoint{1.971231in}{0.815785in}}%
\pgfpathlineto{\pgfqpoint{1.972020in}{0.800067in}}%
\pgfpathlineto{\pgfqpoint{1.972810in}{0.805572in}}%
\pgfpathlineto{\pgfqpoint{1.973205in}{0.808737in}}%
\pgfpathlineto{\pgfqpoint{1.973599in}{0.806301in}}%
\pgfpathlineto{\pgfqpoint{1.976757in}{0.743420in}}%
\pgfpathlineto{\pgfqpoint{1.977942in}{0.755431in}}%
\pgfpathlineto{\pgfqpoint{1.978731in}{0.747666in}}%
\pgfpathlineto{\pgfqpoint{1.979521in}{0.751655in}}%
\pgfpathlineto{\pgfqpoint{1.981889in}{0.782503in}}%
\pgfpathlineto{\pgfqpoint{1.985442in}{0.824357in}}%
\pgfpathlineto{\pgfqpoint{1.985837in}{0.821999in}}%
\pgfpathlineto{\pgfqpoint{1.986626in}{0.825657in}}%
\pgfpathlineto{\pgfqpoint{1.988600in}{0.837255in}}%
\pgfpathlineto{\pgfqpoint{1.988995in}{0.835010in}}%
\pgfpathlineto{\pgfqpoint{1.991758in}{0.802518in}}%
\pgfpathlineto{\pgfqpoint{1.993732in}{0.764795in}}%
\pgfpathlineto{\pgfqpoint{1.994521in}{0.773991in}}%
\pgfpathlineto{\pgfqpoint{1.998469in}{0.801118in}}%
\pgfpathlineto{\pgfqpoint{2.000048in}{0.806213in}}%
\pgfpathlineto{\pgfqpoint{2.000442in}{0.803392in}}%
\pgfpathlineto{\pgfqpoint{2.002416in}{0.759001in}}%
\pgfpathlineto{\pgfqpoint{2.003206in}{0.771486in}}%
\pgfpathlineto{\pgfqpoint{2.005969in}{0.804431in}}%
\pgfpathlineto{\pgfqpoint{2.008337in}{0.797141in}}%
\pgfpathlineto{\pgfqpoint{2.011495in}{0.827808in}}%
\pgfpathlineto{\pgfqpoint{2.011890in}{0.826031in}}%
\pgfpathlineto{\pgfqpoint{2.012285in}{0.825884in}}%
\pgfpathlineto{\pgfqpoint{2.014653in}{0.802902in}}%
\pgfpathlineto{\pgfqpoint{2.019390in}{0.751232in}}%
\pgfpathlineto{\pgfqpoint{2.022548in}{0.732830in}}%
\pgfpathlineto{\pgfqpoint{2.024127in}{0.740301in}}%
\pgfpathlineto{\pgfqpoint{2.025312in}{0.749347in}}%
\pgfpathlineto{\pgfqpoint{2.025706in}{0.745894in}}%
\pgfpathlineto{\pgfqpoint{2.027680in}{0.723217in}}%
\pgfpathlineto{\pgfqpoint{2.028075in}{0.725271in}}%
\pgfpathlineto{\pgfqpoint{2.029654in}{0.725679in}}%
\pgfpathlineto{\pgfqpoint{2.036365in}{0.834144in}}%
\pgfpathlineto{\pgfqpoint{2.037154in}{0.829797in}}%
\pgfpathlineto{\pgfqpoint{2.040312in}{0.804773in}}%
\pgfpathlineto{\pgfqpoint{2.041891in}{0.801039in}}%
\pgfpathlineto{\pgfqpoint{2.041496in}{0.805737in}}%
\pgfpathlineto{\pgfqpoint{2.042286in}{0.802323in}}%
\pgfpathlineto{\pgfqpoint{2.043470in}{0.812283in}}%
\pgfpathlineto{\pgfqpoint{2.045444in}{0.826644in}}%
\pgfpathlineto{\pgfqpoint{2.049786in}{0.782561in}}%
\pgfpathlineto{\pgfqpoint{2.053733in}{0.745401in}}%
\pgfpathlineto{\pgfqpoint{2.054128in}{0.750021in}}%
\pgfpathlineto{\pgfqpoint{2.054918in}{0.747047in}}%
\pgfpathlineto{\pgfqpoint{2.055312in}{0.745235in}}%
\pgfpathlineto{\pgfqpoint{2.055707in}{0.749452in}}%
\pgfpathlineto{\pgfqpoint{2.056102in}{0.752879in}}%
\pgfpathlineto{\pgfqpoint{2.056497in}{0.747893in}}%
\pgfpathlineto{\pgfqpoint{2.058076in}{0.742694in}}%
\pgfpathlineto{\pgfqpoint{2.058470in}{0.744663in}}%
\pgfpathlineto{\pgfqpoint{2.058865in}{0.742989in}}%
\pgfpathlineto{\pgfqpoint{2.059655in}{0.734353in}}%
\pgfpathlineto{\pgfqpoint{2.060444in}{0.740199in}}%
\pgfpathlineto{\pgfqpoint{2.061234in}{0.744307in}}%
\pgfpathlineto{\pgfqpoint{2.062023in}{0.742591in}}%
\pgfpathlineto{\pgfqpoint{2.063997in}{0.737544in}}%
\pgfpathlineto{\pgfqpoint{2.064786in}{0.739785in}}%
\pgfpathlineto{\pgfqpoint{2.065181in}{0.742876in}}%
\pgfpathlineto{\pgfqpoint{2.065971in}{0.738254in}}%
\pgfpathlineto{\pgfqpoint{2.067155in}{0.740515in}}%
\pgfpathlineto{\pgfqpoint{2.075839in}{0.809625in}}%
\pgfpathlineto{\pgfqpoint{2.076234in}{0.807616in}}%
\pgfpathlineto{\pgfqpoint{2.076629in}{0.812100in}}%
\pgfpathlineto{\pgfqpoint{2.077418in}{0.812724in}}%
\pgfpathlineto{\pgfqpoint{2.080576in}{0.791581in}}%
\pgfpathlineto{\pgfqpoint{2.089261in}{0.705739in}}%
\pgfpathlineto{\pgfqpoint{2.089656in}{0.709462in}}%
\pgfpathlineto{\pgfqpoint{2.091629in}{0.722408in}}%
\pgfpathlineto{\pgfqpoint{2.092419in}{0.719876in}}%
\pgfpathlineto{\pgfqpoint{2.094787in}{0.736546in}}%
\pgfpathlineto{\pgfqpoint{2.097945in}{0.769374in}}%
\pgfpathlineto{\pgfqpoint{2.098340in}{0.768966in}}%
\pgfpathlineto{\pgfqpoint{2.099919in}{0.763867in}}%
\pgfpathlineto{\pgfqpoint{2.100709in}{0.756846in}}%
\pgfpathlineto{\pgfqpoint{2.101498in}{0.760564in}}%
\pgfpathlineto{\pgfqpoint{2.105051in}{0.784564in}}%
\pgfpathlineto{\pgfqpoint{2.108209in}{0.751635in}}%
\pgfpathlineto{\pgfqpoint{2.111367in}{0.732977in}}%
\pgfpathlineto{\pgfqpoint{2.111762in}{0.737748in}}%
\pgfpathlineto{\pgfqpoint{2.112156in}{0.738913in}}%
\pgfpathlineto{\pgfqpoint{2.115314in}{0.708422in}}%
\pgfpathlineto{\pgfqpoint{2.115709in}{0.709086in}}%
\pgfpathlineto{\pgfqpoint{2.119262in}{0.729918in}}%
\pgfpathlineto{\pgfqpoint{2.119657in}{0.730443in}}%
\pgfpathlineto{\pgfqpoint{2.121236in}{0.738983in}}%
\pgfpathlineto{\pgfqpoint{2.122025in}{0.738757in}}%
\pgfpathlineto{\pgfqpoint{2.122815in}{0.738058in}}%
\pgfpathlineto{\pgfqpoint{2.124788in}{0.752789in}}%
\pgfpathlineto{\pgfqpoint{2.125973in}{0.741382in}}%
\pgfpathlineto{\pgfqpoint{2.126367in}{0.747643in}}%
\pgfpathlineto{\pgfqpoint{2.129130in}{0.782416in}}%
\pgfpathlineto{\pgfqpoint{2.129525in}{0.782211in}}%
\pgfpathlineto{\pgfqpoint{2.131104in}{0.773918in}}%
\pgfpathlineto{\pgfqpoint{2.131894in}{0.775534in}}%
\pgfpathlineto{\pgfqpoint{2.135841in}{0.795722in}}%
\pgfpathlineto{\pgfqpoint{2.136631in}{0.791044in}}%
\pgfpathlineto{\pgfqpoint{2.140578in}{0.755751in}}%
\pgfpathlineto{\pgfqpoint{2.143736in}{0.738508in}}%
\pgfpathlineto{\pgfqpoint{2.147289in}{0.715211in}}%
\pgfpathlineto{\pgfqpoint{2.147684in}{0.711298in}}%
\pgfpathlineto{\pgfqpoint{2.148078in}{0.715650in}}%
\pgfpathlineto{\pgfqpoint{2.149657in}{0.728190in}}%
\pgfpathlineto{\pgfqpoint{2.150052in}{0.724852in}}%
\pgfpathlineto{\pgfqpoint{2.152421in}{0.751786in}}%
\pgfpathlineto{\pgfqpoint{2.153210in}{0.750960in}}%
\pgfpathlineto{\pgfqpoint{2.155973in}{0.723829in}}%
\pgfpathlineto{\pgfqpoint{2.156763in}{0.731160in}}%
\pgfpathlineto{\pgfqpoint{2.157158in}{0.730538in}}%
\pgfpathlineto{\pgfqpoint{2.159921in}{0.741369in}}%
\pgfpathlineto{\pgfqpoint{2.160316in}{0.740896in}}%
\pgfpathlineto{\pgfqpoint{2.163868in}{0.724362in}}%
\pgfpathlineto{\pgfqpoint{2.165053in}{0.714531in}}%
\pgfpathlineto{\pgfqpoint{2.165447in}{0.719092in}}%
\pgfpathlineto{\pgfqpoint{2.169790in}{0.752258in}}%
\pgfpathlineto{\pgfqpoint{2.170184in}{0.751593in}}%
\pgfpathlineto{\pgfqpoint{2.171763in}{0.728790in}}%
\pgfpathlineto{\pgfqpoint{2.172553in}{0.736036in}}%
\pgfpathlineto{\pgfqpoint{2.176500in}{0.761350in}}%
\pgfpathlineto{\pgfqpoint{2.178474in}{0.774979in}}%
\pgfpathlineto{\pgfqpoint{2.180448in}{0.772762in}}%
\pgfpathlineto{\pgfqpoint{2.180843in}{0.770802in}}%
\pgfpathlineto{\pgfqpoint{2.181237in}{0.774912in}}%
\pgfpathlineto{\pgfqpoint{2.182816in}{0.795625in}}%
\pgfpathlineto{\pgfqpoint{2.183211in}{0.792486in}}%
\pgfpathlineto{\pgfqpoint{2.189922in}{0.746996in}}%
\pgfpathlineto{\pgfqpoint{2.190711in}{0.744022in}}%
\pgfpathlineto{\pgfqpoint{2.191501in}{0.745490in}}%
\pgfpathlineto{\pgfqpoint{2.191896in}{0.746139in}}%
\pgfpathlineto{\pgfqpoint{2.192290in}{0.745799in}}%
\pgfpathlineto{\pgfqpoint{2.193475in}{0.740835in}}%
\pgfpathlineto{\pgfqpoint{2.193869in}{0.744318in}}%
\pgfpathlineto{\pgfqpoint{2.195843in}{0.758126in}}%
\pgfpathlineto{\pgfqpoint{2.196633in}{0.754482in}}%
\pgfpathlineto{\pgfqpoint{2.198212in}{0.746176in}}%
\pgfpathlineto{\pgfqpoint{2.198606in}{0.749668in}}%
\pgfpathlineto{\pgfqpoint{2.199396in}{0.753978in}}%
\pgfpathlineto{\pgfqpoint{2.199791in}{0.752482in}}%
\pgfpathlineto{\pgfqpoint{2.201370in}{0.738550in}}%
\pgfpathlineto{\pgfqpoint{2.202554in}{0.741636in}}%
\pgfpathlineto{\pgfqpoint{2.206501in}{0.764418in}}%
\pgfpathlineto{\pgfqpoint{2.206896in}{0.762268in}}%
\pgfpathlineto{\pgfqpoint{2.209265in}{0.755218in}}%
\pgfpathlineto{\pgfqpoint{2.210449in}{0.759953in}}%
\pgfpathlineto{\pgfqpoint{2.210844in}{0.755983in}}%
\pgfpathlineto{\pgfqpoint{2.214001in}{0.733782in}}%
\pgfpathlineto{\pgfqpoint{2.214396in}{0.735150in}}%
\pgfpathlineto{\pgfqpoint{2.217159in}{0.751932in}}%
\pgfpathlineto{\pgfqpoint{2.219133in}{0.735355in}}%
\pgfpathlineto{\pgfqpoint{2.219923in}{0.736231in}}%
\pgfpathlineto{\pgfqpoint{2.221107in}{0.730629in}}%
\pgfpathlineto{\pgfqpoint{2.221896in}{0.734245in}}%
\pgfpathlineto{\pgfqpoint{2.224265in}{0.741994in}}%
\pgfpathlineto{\pgfqpoint{2.225844in}{0.740193in}}%
\pgfpathlineto{\pgfqpoint{2.226633in}{0.736497in}}%
\pgfpathlineto{\pgfqpoint{2.227028in}{0.739823in}}%
\pgfpathlineto{\pgfqpoint{2.227423in}{0.742715in}}%
\pgfpathlineto{\pgfqpoint{2.227818in}{0.736993in}}%
\pgfpathlineto{\pgfqpoint{2.228212in}{0.732491in}}%
\pgfpathlineto{\pgfqpoint{2.229397in}{0.733128in}}%
\pgfpathlineto{\pgfqpoint{2.232555in}{0.761253in}}%
\pgfpathlineto{\pgfqpoint{2.232949in}{0.764484in}}%
\pgfpathlineto{\pgfqpoint{2.234134in}{0.762534in}}%
\pgfpathlineto{\pgfqpoint{2.238871in}{0.743356in}}%
\pgfpathlineto{\pgfqpoint{2.239265in}{0.743747in}}%
\pgfpathlineto{\pgfqpoint{2.239660in}{0.748880in}}%
\pgfpathlineto{\pgfqpoint{2.240844in}{0.746765in}}%
\pgfpathlineto{\pgfqpoint{2.244792in}{0.743222in}}%
\pgfpathlineto{\pgfqpoint{2.245187in}{0.740424in}}%
\pgfpathlineto{\pgfqpoint{2.246371in}{0.742603in}}%
\pgfpathlineto{\pgfqpoint{2.247555in}{0.748059in}}%
\pgfpathlineto{\pgfqpoint{2.247950in}{0.750625in}}%
\pgfpathlineto{\pgfqpoint{2.248739in}{0.747954in}}%
\pgfpathlineto{\pgfqpoint{2.253871in}{0.718528in}}%
\pgfpathlineto{\pgfqpoint{2.255450in}{0.712575in}}%
\pgfpathlineto{\pgfqpoint{2.255845in}{0.714353in}}%
\pgfpathlineto{\pgfqpoint{2.258608in}{0.718843in}}%
\pgfpathlineto{\pgfqpoint{2.260582in}{0.708103in}}%
\pgfpathlineto{\pgfqpoint{2.261371in}{0.709401in}}%
\pgfpathlineto{\pgfqpoint{2.267687in}{0.736927in}}%
\pgfpathlineto{\pgfqpoint{2.268082in}{0.735167in}}%
\pgfpathlineto{\pgfqpoint{2.269266in}{0.731259in}}%
\pgfpathlineto{\pgfqpoint{2.270056in}{0.732941in}}%
\pgfpathlineto{\pgfqpoint{2.273214in}{0.727748in}}%
\pgfpathlineto{\pgfqpoint{2.275582in}{0.734987in}}%
\pgfpathlineto{\pgfqpoint{2.282293in}{0.702098in}}%
\pgfpathlineto{\pgfqpoint{2.283477in}{0.692191in}}%
\pgfpathlineto{\pgfqpoint{2.284267in}{0.694889in}}%
\pgfpathlineto{\pgfqpoint{2.285056in}{0.693221in}}%
\pgfpathlineto{\pgfqpoint{2.285451in}{0.693829in}}%
\pgfpathlineto{\pgfqpoint{2.287030in}{0.717720in}}%
\pgfpathlineto{\pgfqpoint{2.287820in}{0.711199in}}%
\pgfpathlineto{\pgfqpoint{2.288609in}{0.704870in}}%
\pgfpathlineto{\pgfqpoint{2.289399in}{0.710616in}}%
\pgfpathlineto{\pgfqpoint{2.290583in}{0.724222in}}%
\pgfpathlineto{\pgfqpoint{2.293346in}{0.752455in}}%
\pgfpathlineto{\pgfqpoint{2.296899in}{0.726985in}}%
\pgfpathlineto{\pgfqpoint{2.297293in}{0.729435in}}%
\pgfpathlineto{\pgfqpoint{2.297688in}{0.728197in}}%
\pgfpathlineto{\pgfqpoint{2.299662in}{0.718401in}}%
\pgfpathlineto{\pgfqpoint{2.300057in}{0.719819in}}%
\pgfpathlineto{\pgfqpoint{2.301636in}{0.733194in}}%
\pgfpathlineto{\pgfqpoint{2.302030in}{0.728384in}}%
\pgfpathlineto{\pgfqpoint{2.307162in}{0.670690in}}%
\pgfpathlineto{\pgfqpoint{2.308346in}{0.678495in}}%
\pgfpathlineto{\pgfqpoint{2.309136in}{0.674965in}}%
\pgfpathlineto{\pgfqpoint{2.309925in}{0.671201in}}%
\pgfpathlineto{\pgfqpoint{2.310320in}{0.674638in}}%
\pgfpathlineto{\pgfqpoint{2.313478in}{0.700192in}}%
\pgfpathlineto{\pgfqpoint{2.315452in}{0.713941in}}%
\pgfpathlineto{\pgfqpoint{2.316636in}{0.712116in}}%
\pgfpathlineto{\pgfqpoint{2.320189in}{0.693455in}}%
\pgfpathlineto{\pgfqpoint{2.321373in}{0.697625in}}%
\pgfpathlineto{\pgfqpoint{2.326110in}{0.726116in}}%
\pgfpathlineto{\pgfqpoint{2.326505in}{0.722880in}}%
\pgfpathlineto{\pgfqpoint{2.328084in}{0.699667in}}%
\pgfpathlineto{\pgfqpoint{2.329268in}{0.702179in}}%
\pgfpathlineto{\pgfqpoint{2.330452in}{0.703701in}}%
\pgfpathlineto{\pgfqpoint{2.331637in}{0.715361in}}%
\pgfpathlineto{\pgfqpoint{2.332426in}{0.721115in}}%
\pgfpathlineto{\pgfqpoint{2.333216in}{0.717661in}}%
\pgfpathlineto{\pgfqpoint{2.334795in}{0.713493in}}%
\pgfpathlineto{\pgfqpoint{2.335189in}{0.715260in}}%
\pgfpathlineto{\pgfqpoint{2.335584in}{0.716983in}}%
\pgfpathlineto{\pgfqpoint{2.335979in}{0.715591in}}%
\pgfpathlineto{\pgfqpoint{2.337163in}{0.702450in}}%
\pgfpathlineto{\pgfqpoint{2.337953in}{0.705132in}}%
\pgfpathlineto{\pgfqpoint{2.340321in}{0.713840in}}%
\pgfpathlineto{\pgfqpoint{2.341900in}{0.714216in}}%
\pgfpathlineto{\pgfqpoint{2.343874in}{0.693903in}}%
\pgfpathlineto{\pgfqpoint{2.344663in}{0.702788in}}%
\pgfpathlineto{\pgfqpoint{2.345058in}{0.702250in}}%
\pgfpathlineto{\pgfqpoint{2.345848in}{0.703320in}}%
\pgfpathlineto{\pgfqpoint{2.348611in}{0.727586in}}%
\pgfpathlineto{\pgfqpoint{2.350190in}{0.719783in}}%
\pgfpathlineto{\pgfqpoint{2.351374in}{0.722226in}}%
\pgfpathlineto{\pgfqpoint{2.352164in}{0.720851in}}%
\pgfpathlineto{\pgfqpoint{2.352558in}{0.721126in}}%
\pgfpathlineto{\pgfqpoint{2.353743in}{0.729347in}}%
\pgfpathlineto{\pgfqpoint{2.355322in}{0.728243in}}%
\pgfpathlineto{\pgfqpoint{2.356901in}{0.728954in}}%
\pgfpathlineto{\pgfqpoint{2.360059in}{0.738381in}}%
\pgfpathlineto{\pgfqpoint{2.360453in}{0.737471in}}%
\pgfpathlineto{\pgfqpoint{2.361243in}{0.736306in}}%
\pgfpathlineto{\pgfqpoint{2.363217in}{0.742943in}}%
\pgfpathlineto{\pgfqpoint{2.365980in}{0.723368in}}%
\pgfpathlineto{\pgfqpoint{2.367164in}{0.717928in}}%
\pgfpathlineto{\pgfqpoint{2.368348in}{0.714486in}}%
\pgfpathlineto{\pgfqpoint{2.368743in}{0.715950in}}%
\pgfpathlineto{\pgfqpoint{2.369533in}{0.721872in}}%
\pgfpathlineto{\pgfqpoint{2.370322in}{0.720931in}}%
\pgfpathlineto{\pgfqpoint{2.373085in}{0.710676in}}%
\pgfpathlineto{\pgfqpoint{2.373480in}{0.708928in}}%
\pgfpathlineto{\pgfqpoint{2.374664in}{0.710131in}}%
\pgfpathlineto{\pgfqpoint{2.375454in}{0.708329in}}%
\pgfpathlineto{\pgfqpoint{2.375849in}{0.711007in}}%
\pgfpathlineto{\pgfqpoint{2.376638in}{0.718962in}}%
\pgfpathlineto{\pgfqpoint{2.377428in}{0.713982in}}%
\pgfpathlineto{\pgfqpoint{2.379006in}{0.702243in}}%
\pgfpathlineto{\pgfqpoint{2.380191in}{0.704989in}}%
\pgfpathlineto{\pgfqpoint{2.381375in}{0.702242in}}%
\pgfpathlineto{\pgfqpoint{2.382164in}{0.703608in}}%
\pgfpathlineto{\pgfqpoint{2.388480in}{0.746729in}}%
\pgfpathlineto{\pgfqpoint{2.388875in}{0.745972in}}%
\pgfpathlineto{\pgfqpoint{2.391244in}{0.734126in}}%
\pgfpathlineto{\pgfqpoint{2.392033in}{0.727843in}}%
\pgfpathlineto{\pgfqpoint{2.393612in}{0.712287in}}%
\pgfpathlineto{\pgfqpoint{2.394402in}{0.713117in}}%
\pgfpathlineto{\pgfqpoint{2.397560in}{0.704464in}}%
\pgfpathlineto{\pgfqpoint{2.397954in}{0.705008in}}%
\pgfpathlineto{\pgfqpoint{2.398744in}{0.698279in}}%
\pgfpathlineto{\pgfqpoint{2.399139in}{0.704740in}}%
\pgfpathlineto{\pgfqpoint{2.400718in}{0.711165in}}%
\pgfpathlineto{\pgfqpoint{2.402297in}{0.729341in}}%
\pgfpathlineto{\pgfqpoint{2.403086in}{0.725476in}}%
\pgfpathlineto{\pgfqpoint{2.404665in}{0.709697in}}%
\pgfpathlineto{\pgfqpoint{2.405060in}{0.713607in}}%
\pgfpathlineto{\pgfqpoint{2.410981in}{0.767764in}}%
\pgfpathlineto{\pgfqpoint{2.411771in}{0.764096in}}%
\pgfpathlineto{\pgfqpoint{2.413350in}{0.754988in}}%
\pgfpathlineto{\pgfqpoint{2.413744in}{0.755281in}}%
\pgfpathlineto{\pgfqpoint{2.414139in}{0.758574in}}%
\pgfpathlineto{\pgfqpoint{2.414929in}{0.756179in}}%
\pgfpathlineto{\pgfqpoint{2.418481in}{0.741488in}}%
\pgfpathlineto{\pgfqpoint{2.420060in}{0.721155in}}%
\pgfpathlineto{\pgfqpoint{2.420850in}{0.728862in}}%
\pgfpathlineto{\pgfqpoint{2.422429in}{0.716787in}}%
\pgfpathlineto{\pgfqpoint{2.424008in}{0.686029in}}%
\pgfpathlineto{\pgfqpoint{2.424797in}{0.688373in}}%
\pgfpathlineto{\pgfqpoint{2.427166in}{0.698497in}}%
\pgfpathlineto{\pgfqpoint{2.427955in}{0.696311in}}%
\pgfpathlineto{\pgfqpoint{2.428745in}{0.691410in}}%
\pgfpathlineto{\pgfqpoint{2.429140in}{0.695766in}}%
\pgfpathlineto{\pgfqpoint{2.431903in}{0.708856in}}%
\pgfpathlineto{\pgfqpoint{2.432298in}{0.710900in}}%
\pgfpathlineto{\pgfqpoint{2.433482in}{0.708949in}}%
\pgfpathlineto{\pgfqpoint{2.435061in}{0.708365in}}%
\pgfpathlineto{\pgfqpoint{2.438614in}{0.678144in}}%
\pgfpathlineto{\pgfqpoint{2.439403in}{0.684699in}}%
\pgfpathlineto{\pgfqpoint{2.440982in}{0.695467in}}%
\pgfpathlineto{\pgfqpoint{2.441772in}{0.693021in}}%
\pgfpathlineto{\pgfqpoint{2.444140in}{0.682983in}}%
\pgfpathlineto{\pgfqpoint{2.445324in}{0.684405in}}%
\pgfpathlineto{\pgfqpoint{2.445719in}{0.685331in}}%
\pgfpathlineto{\pgfqpoint{2.446114in}{0.683564in}}%
\pgfpathlineto{\pgfqpoint{2.447298in}{0.675820in}}%
\pgfpathlineto{\pgfqpoint{2.447693in}{0.678910in}}%
\pgfpathlineto{\pgfqpoint{2.451246in}{0.702352in}}%
\pgfpathlineto{\pgfqpoint{2.452430in}{0.707102in}}%
\pgfpathlineto{\pgfqpoint{2.453219in}{0.704501in}}%
\pgfpathlineto{\pgfqpoint{2.454798in}{0.693666in}}%
\pgfpathlineto{\pgfqpoint{2.455193in}{0.695853in}}%
\pgfpathlineto{\pgfqpoint{2.457167in}{0.704036in}}%
\pgfpathlineto{\pgfqpoint{2.458351in}{0.700828in}}%
\pgfpathlineto{\pgfqpoint{2.458746in}{0.704267in}}%
\pgfpathlineto{\pgfqpoint{2.460720in}{0.713510in}}%
\pgfpathlineto{\pgfqpoint{2.461904in}{0.713302in}}%
\pgfpathlineto{\pgfqpoint{2.462298in}{0.712093in}}%
\pgfpathlineto{\pgfqpoint{2.463483in}{0.713306in}}%
\pgfpathlineto{\pgfqpoint{2.464272in}{0.714392in}}%
\pgfpathlineto{\pgfqpoint{2.464667in}{0.713820in}}%
\pgfpathlineto{\pgfqpoint{2.466641in}{0.707082in}}%
\pgfpathlineto{\pgfqpoint{2.467035in}{0.707526in}}%
\pgfpathlineto{\pgfqpoint{2.467825in}{0.707699in}}%
\pgfpathlineto{\pgfqpoint{2.469404in}{0.704954in}}%
\pgfpathlineto{\pgfqpoint{2.470588in}{0.709757in}}%
\pgfpathlineto{\pgfqpoint{2.470983in}{0.707787in}}%
\pgfpathlineto{\pgfqpoint{2.475325in}{0.686774in}}%
\pgfpathlineto{\pgfqpoint{2.476115in}{0.684161in}}%
\pgfpathlineto{\pgfqpoint{2.476904in}{0.686126in}}%
\pgfpathlineto{\pgfqpoint{2.478483in}{0.686653in}}%
\pgfpathlineto{\pgfqpoint{2.479273in}{0.690108in}}%
\pgfpathlineto{\pgfqpoint{2.480062in}{0.688263in}}%
\pgfpathlineto{\pgfqpoint{2.483220in}{0.669065in}}%
\pgfpathlineto{\pgfqpoint{2.483615in}{0.670124in}}%
\pgfpathlineto{\pgfqpoint{2.484404in}{0.672911in}}%
\pgfpathlineto{\pgfqpoint{2.484799in}{0.670957in}}%
\pgfpathlineto{\pgfqpoint{2.486378in}{0.660092in}}%
\pgfpathlineto{\pgfqpoint{2.487957in}{0.646154in}}%
\pgfpathlineto{\pgfqpoint{2.488352in}{0.652329in}}%
\pgfpathlineto{\pgfqpoint{2.491510in}{0.687560in}}%
\pgfpathlineto{\pgfqpoint{2.494273in}{0.673855in}}%
\pgfpathlineto{\pgfqpoint{2.495063in}{0.675375in}}%
\pgfpathlineto{\pgfqpoint{2.495457in}{0.678184in}}%
\pgfpathlineto{\pgfqpoint{2.496247in}{0.673680in}}%
\pgfpathlineto{\pgfqpoint{2.497036in}{0.673440in}}%
\pgfpathlineto{\pgfqpoint{2.500194in}{0.688389in}}%
\pgfpathlineto{\pgfqpoint{2.501773in}{0.695989in}}%
\pgfpathlineto{\pgfqpoint{2.502563in}{0.695615in}}%
\pgfpathlineto{\pgfqpoint{2.503352in}{0.692826in}}%
\pgfpathlineto{\pgfqpoint{2.504142in}{0.685169in}}%
\pgfpathlineto{\pgfqpoint{2.505326in}{0.686127in}}%
\pgfpathlineto{\pgfqpoint{2.508089in}{0.681148in}}%
\pgfpathlineto{\pgfqpoint{2.506116in}{0.687063in}}%
\pgfpathlineto{\pgfqpoint{2.508484in}{0.683279in}}%
\pgfpathlineto{\pgfqpoint{2.509668in}{0.688108in}}%
\pgfpathlineto{\pgfqpoint{2.510458in}{0.687708in}}%
\pgfpathlineto{\pgfqpoint{2.510853in}{0.686946in}}%
\pgfpathlineto{\pgfqpoint{2.511247in}{0.688405in}}%
\pgfpathlineto{\pgfqpoint{2.518353in}{0.718346in}}%
\pgfpathlineto{\pgfqpoint{2.520721in}{0.713533in}}%
\pgfpathlineto{\pgfqpoint{2.522695in}{0.710942in}}%
\pgfpathlineto{\pgfqpoint{2.523090in}{0.711613in}}%
\pgfpathlineto{\pgfqpoint{2.523879in}{0.710922in}}%
\pgfpathlineto{\pgfqpoint{2.525458in}{0.702553in}}%
\pgfpathlineto{\pgfqpoint{2.526248in}{0.702941in}}%
\pgfpathlineto{\pgfqpoint{2.529801in}{0.733224in}}%
\pgfpathlineto{\pgfqpoint{2.531774in}{0.733295in}}%
\pgfpathlineto{\pgfqpoint{2.533353in}{0.726889in}}%
\pgfpathlineto{\pgfqpoint{2.536117in}{0.696588in}}%
\pgfpathlineto{\pgfqpoint{2.537301in}{0.700478in}}%
\pgfpathlineto{\pgfqpoint{2.539275in}{0.715505in}}%
\pgfpathlineto{\pgfqpoint{2.540064in}{0.712894in}}%
\pgfpathlineto{\pgfqpoint{2.541248in}{0.711368in}}%
\pgfpathlineto{\pgfqpoint{2.545196in}{0.678735in}}%
\pgfpathlineto{\pgfqpoint{2.545590in}{0.679028in}}%
\pgfpathlineto{\pgfqpoint{2.545985in}{0.680263in}}%
\pgfpathlineto{\pgfqpoint{2.547959in}{0.690008in}}%
\pgfpathlineto{\pgfqpoint{2.549538in}{0.698017in}}%
\pgfpathlineto{\pgfqpoint{2.550722in}{0.697540in}}%
\pgfpathlineto{\pgfqpoint{2.551906in}{0.696321in}}%
\pgfpathlineto{\pgfqpoint{2.555854in}{0.681442in}}%
\pgfpathlineto{\pgfqpoint{2.558617in}{0.690322in}}%
\pgfpathlineto{\pgfqpoint{2.559407in}{0.692206in}}%
\pgfpathlineto{\pgfqpoint{2.560196in}{0.690832in}}%
\pgfpathlineto{\pgfqpoint{2.561380in}{0.691787in}}%
\pgfpathlineto{\pgfqpoint{2.561775in}{0.690428in}}%
\pgfpathlineto{\pgfqpoint{2.564538in}{0.687106in}}%
\pgfpathlineto{\pgfqpoint{2.567302in}{0.676270in}}%
\pgfpathlineto{\pgfqpoint{2.568091in}{0.679543in}}%
\pgfpathlineto{\pgfqpoint{2.573223in}{0.720834in}}%
\pgfpathlineto{\pgfqpoint{2.573618in}{0.718377in}}%
\pgfpathlineto{\pgfqpoint{2.574802in}{0.705085in}}%
\pgfpathlineto{\pgfqpoint{2.575591in}{0.707691in}}%
\pgfpathlineto{\pgfqpoint{2.575986in}{0.707912in}}%
\pgfpathlineto{\pgfqpoint{2.576776in}{0.697518in}}%
\pgfpathlineto{\pgfqpoint{2.577565in}{0.700953in}}%
\pgfpathlineto{\pgfqpoint{2.578749in}{0.720711in}}%
\pgfpathlineto{\pgfqpoint{2.579934in}{0.713491in}}%
\pgfpathlineto{\pgfqpoint{2.580723in}{0.708944in}}%
\pgfpathlineto{\pgfqpoint{2.581513in}{0.712893in}}%
\pgfpathlineto{\pgfqpoint{2.583486in}{0.715930in}}%
\pgfpathlineto{\pgfqpoint{2.583881in}{0.716248in}}%
\pgfpathlineto{\pgfqpoint{2.589408in}{0.671746in}}%
\pgfpathlineto{\pgfqpoint{2.590197in}{0.675062in}}%
\pgfpathlineto{\pgfqpoint{2.590987in}{0.673912in}}%
\pgfpathlineto{\pgfqpoint{2.593355in}{0.666680in}}%
\pgfpathlineto{\pgfqpoint{2.594145in}{0.665268in}}%
\pgfpathlineto{\pgfqpoint{2.594934in}{0.667138in}}%
\pgfpathlineto{\pgfqpoint{2.610724in}{0.719758in}}%
\pgfpathlineto{\pgfqpoint{2.613093in}{0.701180in}}%
\pgfpathlineto{\pgfqpoint{2.614672in}{0.702376in}}%
\pgfpathlineto{\pgfqpoint{2.615066in}{0.702443in}}%
\pgfpathlineto{\pgfqpoint{2.616645in}{0.693784in}}%
\pgfpathlineto{\pgfqpoint{2.617435in}{0.696136in}}%
\pgfpathlineto{\pgfqpoint{2.618224in}{0.696036in}}%
\pgfpathlineto{\pgfqpoint{2.620988in}{0.679142in}}%
\pgfpathlineto{\pgfqpoint{2.621382in}{0.679394in}}%
\pgfpathlineto{\pgfqpoint{2.625725in}{0.719021in}}%
\pgfpathlineto{\pgfqpoint{2.626514in}{0.723135in}}%
\pgfpathlineto{\pgfqpoint{2.627698in}{0.720994in}}%
\pgfpathlineto{\pgfqpoint{2.628882in}{0.697115in}}%
\pgfpathlineto{\pgfqpoint{2.629672in}{0.697971in}}%
\pgfpathlineto{\pgfqpoint{2.630067in}{0.698117in}}%
\pgfpathlineto{\pgfqpoint{2.630856in}{0.718389in}}%
\pgfpathlineto{\pgfqpoint{2.632040in}{0.710241in}}%
\pgfpathlineto{\pgfqpoint{2.634014in}{0.702574in}}%
\pgfpathlineto{\pgfqpoint{2.634409in}{0.702679in}}%
\pgfpathlineto{\pgfqpoint{2.635198in}{0.697817in}}%
\pgfpathlineto{\pgfqpoint{2.639146in}{0.646584in}}%
\pgfpathlineto{\pgfqpoint{2.639541in}{0.649728in}}%
\pgfpathlineto{\pgfqpoint{2.641120in}{0.661988in}}%
\pgfpathlineto{\pgfqpoint{2.641514in}{0.661889in}}%
\pgfpathlineto{\pgfqpoint{2.642699in}{0.649679in}}%
\pgfpathlineto{\pgfqpoint{2.643093in}{0.658273in}}%
\pgfpathlineto{\pgfqpoint{2.645857in}{0.693988in}}%
\pgfpathlineto{\pgfqpoint{2.648620in}{0.689403in}}%
\pgfpathlineto{\pgfqpoint{2.649804in}{0.689847in}}%
\pgfpathlineto{\pgfqpoint{2.651778in}{0.682472in}}%
\pgfpathlineto{\pgfqpoint{2.652567in}{0.683157in}}%
\pgfpathlineto{\pgfqpoint{2.652962in}{0.682834in}}%
\pgfpathlineto{\pgfqpoint{2.653357in}{0.684074in}}%
\pgfpathlineto{\pgfqpoint{2.655331in}{0.706506in}}%
\pgfpathlineto{\pgfqpoint{2.656910in}{0.703481in}}%
\pgfpathlineto{\pgfqpoint{2.661252in}{0.658481in}}%
\pgfpathlineto{\pgfqpoint{2.662831in}{0.646930in}}%
\pgfpathlineto{\pgfqpoint{2.663226in}{0.648554in}}%
\pgfpathlineto{\pgfqpoint{2.663620in}{0.649951in}}%
\pgfpathlineto{\pgfqpoint{2.664015in}{0.647017in}}%
\pgfpathlineto{\pgfqpoint{2.666384in}{0.637997in}}%
\pgfpathlineto{\pgfqpoint{2.669936in}{0.668577in}}%
\pgfpathlineto{\pgfqpoint{2.672700in}{0.682762in}}%
\pgfpathlineto{\pgfqpoint{2.674279in}{0.684717in}}%
\pgfpathlineto{\pgfqpoint{2.675463in}{0.694490in}}%
\pgfpathlineto{\pgfqpoint{2.675858in}{0.690946in}}%
\pgfpathlineto{\pgfqpoint{2.677437in}{0.676265in}}%
\pgfpathlineto{\pgfqpoint{2.678226in}{0.678420in}}%
\pgfpathlineto{\pgfqpoint{2.678621in}{0.677745in}}%
\pgfpathlineto{\pgfqpoint{2.683358in}{0.645774in}}%
\pgfpathlineto{\pgfqpoint{2.684147in}{0.638595in}}%
\pgfpathlineto{\pgfqpoint{2.685332in}{0.642676in}}%
\pgfpathlineto{\pgfqpoint{2.690463in}{0.695899in}}%
\pgfpathlineto{\pgfqpoint{2.693227in}{0.714475in}}%
\pgfpathlineto{\pgfqpoint{2.695595in}{0.717431in}}%
\pgfpathlineto{\pgfqpoint{2.697569in}{0.725074in}}%
\pgfpathlineto{\pgfqpoint{2.697964in}{0.724605in}}%
\pgfpathlineto{\pgfqpoint{2.700332in}{0.718178in}}%
\pgfpathlineto{\pgfqpoint{2.704280in}{0.686071in}}%
\pgfpathlineto{\pgfqpoint{2.704674in}{0.686538in}}%
\pgfpathlineto{\pgfqpoint{2.706253in}{0.695735in}}%
\pgfpathlineto{\pgfqpoint{2.707043in}{0.693275in}}%
\pgfpathlineto{\pgfqpoint{2.707832in}{0.688718in}}%
\pgfpathlineto{\pgfqpoint{2.708622in}{0.692286in}}%
\pgfpathlineto{\pgfqpoint{2.709806in}{0.701814in}}%
\pgfpathlineto{\pgfqpoint{2.710596in}{0.698800in}}%
\pgfpathlineto{\pgfqpoint{2.712964in}{0.689925in}}%
\pgfpathlineto{\pgfqpoint{2.713359in}{0.690405in}}%
\pgfpathlineto{\pgfqpoint{2.714148in}{0.690292in}}%
\pgfpathlineto{\pgfqpoint{2.714543in}{0.692478in}}%
\pgfpathlineto{\pgfqpoint{2.714938in}{0.688559in}}%
\pgfpathlineto{\pgfqpoint{2.715332in}{0.688043in}}%
\pgfpathlineto{\pgfqpoint{2.717701in}{0.700660in}}%
\pgfpathlineto{\pgfqpoint{2.722043in}{0.666035in}}%
\pgfpathlineto{\pgfqpoint{2.724412in}{0.653360in}}%
\pgfpathlineto{\pgfqpoint{2.726780in}{0.646248in}}%
\pgfpathlineto{\pgfqpoint{2.727175in}{0.647758in}}%
\pgfpathlineto{\pgfqpoint{2.728359in}{0.654008in}}%
\pgfpathlineto{\pgfqpoint{2.728754in}{0.651004in}}%
\pgfpathlineto{\pgfqpoint{2.729938in}{0.641523in}}%
\pgfpathlineto{\pgfqpoint{2.730728in}{0.646175in}}%
\pgfpathlineto{\pgfqpoint{2.732307in}{0.659477in}}%
\pgfpathlineto{\pgfqpoint{2.733096in}{0.656918in}}%
\pgfpathlineto{\pgfqpoint{2.734280in}{0.652231in}}%
\pgfpathlineto{\pgfqpoint{2.737044in}{0.671388in}}%
\pgfpathlineto{\pgfqpoint{2.739412in}{0.655773in}}%
\pgfpathlineto{\pgfqpoint{2.740202in}{0.657416in}}%
\pgfpathlineto{\pgfqpoint{2.741386in}{0.663300in}}%
\pgfpathlineto{\pgfqpoint{2.742965in}{0.682962in}}%
\pgfpathlineto{\pgfqpoint{2.743754in}{0.677454in}}%
\pgfpathlineto{\pgfqpoint{2.744939in}{0.670922in}}%
\pgfpathlineto{\pgfqpoint{2.745728in}{0.671744in}}%
\pgfpathlineto{\pgfqpoint{2.746912in}{0.664658in}}%
\pgfpathlineto{\pgfqpoint{2.747702in}{0.667915in}}%
\pgfpathlineto{\pgfqpoint{2.749676in}{0.676218in}}%
\pgfpathlineto{\pgfqpoint{2.750860in}{0.674314in}}%
\pgfpathlineto{\pgfqpoint{2.752044in}{0.672016in}}%
\pgfpathlineto{\pgfqpoint{2.752439in}{0.673668in}}%
\pgfpathlineto{\pgfqpoint{2.761123in}{0.724458in}}%
\pgfpathlineto{\pgfqpoint{2.761913in}{0.723502in}}%
\pgfpathlineto{\pgfqpoint{2.762702in}{0.726962in}}%
\pgfpathlineto{\pgfqpoint{2.763492in}{0.725198in}}%
\pgfpathlineto{\pgfqpoint{2.765071in}{0.716303in}}%
\pgfpathlineto{\pgfqpoint{2.766650in}{0.706717in}}%
\pgfpathlineto{\pgfqpoint{2.767439in}{0.709218in}}%
\pgfpathlineto{\pgfqpoint{2.771387in}{0.722186in}}%
\pgfpathlineto{\pgfqpoint{2.771782in}{0.726975in}}%
\pgfpathlineto{\pgfqpoint{2.772571in}{0.721453in}}%
\pgfpathlineto{\pgfqpoint{2.776124in}{0.699044in}}%
\pgfpathlineto{\pgfqpoint{2.776519in}{0.699517in}}%
\pgfpathlineto{\pgfqpoint{2.778098in}{0.702989in}}%
\pgfpathlineto{\pgfqpoint{2.781650in}{0.727581in}}%
\pgfpathlineto{\pgfqpoint{2.782045in}{0.726230in}}%
\pgfpathlineto{\pgfqpoint{2.784019in}{0.718248in}}%
\pgfpathlineto{\pgfqpoint{2.786387in}{0.704500in}}%
\pgfpathlineto{\pgfqpoint{2.788361in}{0.701868in}}%
\pgfpathlineto{\pgfqpoint{2.790730in}{0.684054in}}%
\pgfpathlineto{\pgfqpoint{2.791519in}{0.684912in}}%
\pgfpathlineto{\pgfqpoint{2.791914in}{0.685569in}}%
\pgfpathlineto{\pgfqpoint{2.792309in}{0.689441in}}%
\pgfpathlineto{\pgfqpoint{2.793493in}{0.686861in}}%
\pgfpathlineto{\pgfqpoint{2.793888in}{0.686298in}}%
\pgfpathlineto{\pgfqpoint{2.794677in}{0.687589in}}%
\pgfpathlineto{\pgfqpoint{2.799414in}{0.710660in}}%
\pgfpathlineto{\pgfqpoint{2.799809in}{0.709332in}}%
\pgfpathlineto{\pgfqpoint{2.800203in}{0.705815in}}%
\pgfpathlineto{\pgfqpoint{2.800598in}{0.708940in}}%
\pgfpathlineto{\pgfqpoint{2.802572in}{0.718854in}}%
\pgfpathlineto{\pgfqpoint{2.802967in}{0.718064in}}%
\pgfpathlineto{\pgfqpoint{2.803756in}{0.716311in}}%
\pgfpathlineto{\pgfqpoint{2.804151in}{0.717859in}}%
\pgfpathlineto{\pgfqpoint{2.804546in}{0.718518in}}%
\pgfpathlineto{\pgfqpoint{2.804940in}{0.716710in}}%
\pgfpathlineto{\pgfqpoint{2.807309in}{0.703850in}}%
\pgfpathlineto{\pgfqpoint{2.807704in}{0.705720in}}%
\pgfpathlineto{\pgfqpoint{2.810862in}{0.734622in}}%
\pgfpathlineto{\pgfqpoint{2.811651in}{0.729301in}}%
\pgfpathlineto{\pgfqpoint{2.812046in}{0.725999in}}%
\pgfpathlineto{\pgfqpoint{2.812835in}{0.729369in}}%
\pgfpathlineto{\pgfqpoint{2.813230in}{0.729332in}}%
\pgfpathlineto{\pgfqpoint{2.816388in}{0.695031in}}%
\pgfpathlineto{\pgfqpoint{2.819151in}{0.671038in}}%
\pgfpathlineto{\pgfqpoint{2.819941in}{0.673865in}}%
\pgfpathlineto{\pgfqpoint{2.826257in}{0.714512in}}%
\pgfpathlineto{\pgfqpoint{2.826652in}{0.711475in}}%
\pgfpathlineto{\pgfqpoint{2.828625in}{0.698850in}}%
\pgfpathlineto{\pgfqpoint{2.829415in}{0.702908in}}%
\pgfpathlineto{\pgfqpoint{2.829810in}{0.705949in}}%
\pgfpathlineto{\pgfqpoint{2.830204in}{0.701625in}}%
\pgfpathlineto{\pgfqpoint{2.831783in}{0.694021in}}%
\pgfpathlineto{\pgfqpoint{2.832178in}{0.694621in}}%
\pgfpathlineto{\pgfqpoint{2.832573in}{0.693724in}}%
\pgfpathlineto{\pgfqpoint{2.832968in}{0.694239in}}%
\pgfpathlineto{\pgfqpoint{2.833757in}{0.697793in}}%
\pgfpathlineto{\pgfqpoint{2.834152in}{0.695429in}}%
\pgfpathlineto{\pgfqpoint{2.834941in}{0.690199in}}%
\pgfpathlineto{\pgfqpoint{2.835731in}{0.691743in}}%
\pgfpathlineto{\pgfqpoint{2.837310in}{0.696861in}}%
\pgfpathlineto{\pgfqpoint{2.839678in}{0.687213in}}%
\pgfpathlineto{\pgfqpoint{2.840468in}{0.693216in}}%
\pgfpathlineto{\pgfqpoint{2.841257in}{0.690365in}}%
\pgfpathlineto{\pgfqpoint{2.842442in}{0.684213in}}%
\pgfpathlineto{\pgfqpoint{2.843231in}{0.686901in}}%
\pgfpathlineto{\pgfqpoint{2.844415in}{0.686866in}}%
\pgfpathlineto{\pgfqpoint{2.845600in}{0.682384in}}%
\pgfpathlineto{\pgfqpoint{2.846784in}{0.684970in}}%
\pgfpathlineto{\pgfqpoint{2.848758in}{0.691239in}}%
\pgfpathlineto{\pgfqpoint{2.850731in}{0.715984in}}%
\pgfpathlineto{\pgfqpoint{2.851521in}{0.712775in}}%
\pgfpathlineto{\pgfqpoint{2.852705in}{0.708977in}}%
\pgfpathlineto{\pgfqpoint{2.853100in}{0.710671in}}%
\pgfpathlineto{\pgfqpoint{2.855074in}{0.710096in}}%
\pgfpathlineto{\pgfqpoint{2.859416in}{0.689538in}}%
\pgfpathlineto{\pgfqpoint{2.859811in}{0.692808in}}%
\pgfpathlineto{\pgfqpoint{2.860205in}{0.700513in}}%
\pgfpathlineto{\pgfqpoint{2.861390in}{0.697534in}}%
\pgfpathlineto{\pgfqpoint{2.861784in}{0.696769in}}%
\pgfpathlineto{\pgfqpoint{2.862179in}{0.699226in}}%
\pgfpathlineto{\pgfqpoint{2.862574in}{0.698729in}}%
\pgfpathlineto{\pgfqpoint{2.862969in}{0.699367in}}%
\pgfpathlineto{\pgfqpoint{2.864153in}{0.703522in}}%
\pgfpathlineto{\pgfqpoint{2.865732in}{0.715247in}}%
\pgfpathlineto{\pgfqpoint{2.866127in}{0.712789in}}%
\pgfpathlineto{\pgfqpoint{2.866916in}{0.704821in}}%
\pgfpathlineto{\pgfqpoint{2.867706in}{0.709280in}}%
\pgfpathlineto{\pgfqpoint{2.868100in}{0.710856in}}%
\pgfpathlineto{\pgfqpoint{2.868495in}{0.708466in}}%
\pgfpathlineto{\pgfqpoint{2.872443in}{0.698127in}}%
\pgfpathlineto{\pgfqpoint{2.873232in}{0.700174in}}%
\pgfpathlineto{\pgfqpoint{2.874416in}{0.699650in}}%
\pgfpathlineto{\pgfqpoint{2.874811in}{0.697591in}}%
\pgfpathlineto{\pgfqpoint{2.875206in}{0.701610in}}%
\pgfpathlineto{\pgfqpoint{2.876785in}{0.715183in}}%
\pgfpathlineto{\pgfqpoint{2.877574in}{0.713413in}}%
\pgfpathlineto{\pgfqpoint{2.877969in}{0.712183in}}%
\pgfpathlineto{\pgfqpoint{2.878364in}{0.713759in}}%
\pgfpathlineto{\pgfqpoint{2.879943in}{0.722930in}}%
\pgfpathlineto{\pgfqpoint{2.880337in}{0.720969in}}%
\pgfpathlineto{\pgfqpoint{2.886259in}{0.688837in}}%
\pgfpathlineto{\pgfqpoint{2.886653in}{0.693059in}}%
\pgfpathlineto{\pgfqpoint{2.887838in}{0.697432in}}%
\pgfpathlineto{\pgfqpoint{2.888232in}{0.695359in}}%
\pgfpathlineto{\pgfqpoint{2.890206in}{0.680070in}}%
\pgfpathlineto{\pgfqpoint{2.891390in}{0.681677in}}%
\pgfpathlineto{\pgfqpoint{2.893364in}{0.680321in}}%
\pgfpathlineto{\pgfqpoint{2.894943in}{0.679969in}}%
\pgfpathlineto{\pgfqpoint{2.895338in}{0.679402in}}%
\pgfpathlineto{\pgfqpoint{2.895733in}{0.680894in}}%
\pgfpathlineto{\pgfqpoint{2.899285in}{0.695669in}}%
\pgfpathlineto{\pgfqpoint{2.899680in}{0.693444in}}%
\pgfpathlineto{\pgfqpoint{2.900470in}{0.691576in}}%
\pgfpathlineto{\pgfqpoint{2.900864in}{0.692955in}}%
\pgfpathlineto{\pgfqpoint{2.902443in}{0.696870in}}%
\pgfpathlineto{\pgfqpoint{2.902838in}{0.696017in}}%
\pgfpathlineto{\pgfqpoint{2.905601in}{0.689964in}}%
\pgfpathlineto{\pgfqpoint{2.906391in}{0.691495in}}%
\pgfpathlineto{\pgfqpoint{2.907180in}{0.693674in}}%
\pgfpathlineto{\pgfqpoint{2.907970in}{0.692832in}}%
\pgfpathlineto{\pgfqpoint{2.912707in}{0.675126in}}%
\pgfpathlineto{\pgfqpoint{2.913102in}{0.676327in}}%
\pgfpathlineto{\pgfqpoint{2.913891in}{0.673831in}}%
\pgfpathlineto{\pgfqpoint{2.914681in}{0.671292in}}%
\pgfpathlineto{\pgfqpoint{2.918233in}{0.701617in}}%
\pgfpathlineto{\pgfqpoint{2.918628in}{0.702433in}}%
\pgfpathlineto{\pgfqpoint{2.921391in}{0.679527in}}%
\pgfpathlineto{\pgfqpoint{2.921786in}{0.681783in}}%
\pgfpathlineto{\pgfqpoint{2.925339in}{0.708713in}}%
\pgfpathlineto{\pgfqpoint{2.927313in}{0.710111in}}%
\pgfpathlineto{\pgfqpoint{2.928497in}{0.712956in}}%
\pgfpathlineto{\pgfqpoint{2.929286in}{0.711325in}}%
\pgfpathlineto{\pgfqpoint{2.929681in}{0.712269in}}%
\pgfpathlineto{\pgfqpoint{2.930076in}{0.710729in}}%
\pgfpathlineto{\pgfqpoint{2.932839in}{0.702930in}}%
\pgfpathlineto{\pgfqpoint{2.933234in}{0.701785in}}%
\pgfpathlineto{\pgfqpoint{2.934418in}{0.702603in}}%
\pgfpathlineto{\pgfqpoint{2.934813in}{0.703588in}}%
\pgfpathlineto{\pgfqpoint{2.935602in}{0.702171in}}%
\pgfpathlineto{\pgfqpoint{2.936787in}{0.695119in}}%
\pgfpathlineto{\pgfqpoint{2.945866in}{0.654957in}}%
\pgfpathlineto{\pgfqpoint{2.946261in}{0.654504in}}%
\pgfpathlineto{\pgfqpoint{2.946655in}{0.655637in}}%
\pgfpathlineto{\pgfqpoint{2.947445in}{0.658342in}}%
\pgfpathlineto{\pgfqpoint{2.948234in}{0.656765in}}%
\pgfpathlineto{\pgfqpoint{2.949813in}{0.653644in}}%
\pgfpathlineto{\pgfqpoint{2.950998in}{0.655763in}}%
\pgfpathlineto{\pgfqpoint{2.951392in}{0.657232in}}%
\pgfpathlineto{\pgfqpoint{2.952577in}{0.655727in}}%
\pgfpathlineto{\pgfqpoint{2.953366in}{0.655953in}}%
\pgfpathlineto{\pgfqpoint{2.958103in}{0.668003in}}%
\pgfpathlineto{\pgfqpoint{2.960866in}{0.662914in}}%
\pgfpathlineto{\pgfqpoint{2.963629in}{0.669349in}}%
\pgfpathlineto{\pgfqpoint{2.964024in}{0.667817in}}%
\pgfpathlineto{\pgfqpoint{2.964814in}{0.666282in}}%
\pgfpathlineto{\pgfqpoint{2.965603in}{0.666915in}}%
\pgfpathlineto{\pgfqpoint{2.967182in}{0.675308in}}%
\pgfpathlineto{\pgfqpoint{2.967577in}{0.677500in}}%
\pgfpathlineto{\pgfqpoint{2.967972in}{0.673494in}}%
\pgfpathlineto{\pgfqpoint{2.970735in}{0.663577in}}%
\pgfpathlineto{\pgfqpoint{2.971130in}{0.663504in}}%
\pgfpathlineto{\pgfqpoint{2.971524in}{0.661322in}}%
\pgfpathlineto{\pgfqpoint{2.972314in}{0.664869in}}%
\pgfpathlineto{\pgfqpoint{2.973893in}{0.667111in}}%
\pgfpathlineto{\pgfqpoint{2.975472in}{0.672639in}}%
\pgfpathlineto{\pgfqpoint{2.979814in}{0.692894in}}%
\pgfpathlineto{\pgfqpoint{2.980209in}{0.691835in}}%
\pgfpathlineto{\pgfqpoint{2.981393in}{0.689399in}}%
\pgfpathlineto{\pgfqpoint{2.982972in}{0.697792in}}%
\pgfpathlineto{\pgfqpoint{2.983367in}{0.695602in}}%
\pgfpathlineto{\pgfqpoint{2.985341in}{0.682525in}}%
\pgfpathlineto{\pgfqpoint{2.985735in}{0.684031in}}%
\pgfpathlineto{\pgfqpoint{2.988893in}{0.697994in}}%
\pgfpathlineto{\pgfqpoint{2.990078in}{0.697448in}}%
\pgfpathlineto{\pgfqpoint{2.992051in}{0.693789in}}%
\pgfpathlineto{\pgfqpoint{2.996788in}{0.687910in}}%
\pgfpathlineto{\pgfqpoint{2.998367in}{0.682281in}}%
\pgfpathlineto{\pgfqpoint{3.000736in}{0.672775in}}%
\pgfpathlineto{\pgfqpoint{3.001525in}{0.672140in}}%
\pgfpathlineto{\pgfqpoint{3.001920in}{0.673040in}}%
\pgfpathlineto{\pgfqpoint{3.003104in}{0.675381in}}%
\pgfpathlineto{\pgfqpoint{3.003499in}{0.672818in}}%
\pgfpathlineto{\pgfqpoint{3.003894in}{0.672125in}}%
\pgfpathlineto{\pgfqpoint{3.004289in}{0.674033in}}%
\pgfpathlineto{\pgfqpoint{3.004683in}{0.675249in}}%
\pgfpathlineto{\pgfqpoint{3.005473in}{0.673467in}}%
\pgfpathlineto{\pgfqpoint{3.005868in}{0.672345in}}%
\pgfpathlineto{\pgfqpoint{3.006262in}{0.673628in}}%
\pgfpathlineto{\pgfqpoint{3.007447in}{0.683269in}}%
\pgfpathlineto{\pgfqpoint{3.009815in}{0.690570in}}%
\pgfpathlineto{\pgfqpoint{3.010210in}{0.692168in}}%
\pgfpathlineto{\pgfqpoint{3.010999in}{0.680956in}}%
\pgfpathlineto{\pgfqpoint{3.012184in}{0.684756in}}%
\pgfpathlineto{\pgfqpoint{3.014947in}{0.697297in}}%
\pgfpathlineto{\pgfqpoint{3.015736in}{0.695172in}}%
\pgfpathlineto{\pgfqpoint{3.016526in}{0.692523in}}%
\pgfpathlineto{\pgfqpoint{3.020473in}{0.674659in}}%
\pgfpathlineto{\pgfqpoint{3.022447in}{0.670699in}}%
\pgfpathlineto{\pgfqpoint{3.025210in}{0.672803in}}%
\pgfpathlineto{\pgfqpoint{3.027184in}{0.678254in}}%
\pgfpathlineto{\pgfqpoint{3.029947in}{0.698284in}}%
\pgfpathlineto{\pgfqpoint{3.031526in}{0.698953in}}%
\pgfpathlineto{\pgfqpoint{3.031921in}{0.699076in}}%
\pgfpathlineto{\pgfqpoint{3.033500in}{0.694087in}}%
\pgfpathlineto{\pgfqpoint{3.033895in}{0.694487in}}%
\pgfpathlineto{\pgfqpoint{3.035869in}{0.700773in}}%
\pgfpathlineto{\pgfqpoint{3.037448in}{0.704097in}}%
\pgfpathlineto{\pgfqpoint{3.039027in}{0.692929in}}%
\pgfpathlineto{\pgfqpoint{3.039816in}{0.698939in}}%
\pgfpathlineto{\pgfqpoint{3.040211in}{0.701659in}}%
\pgfpathlineto{\pgfqpoint{3.041000in}{0.699944in}}%
\pgfpathlineto{\pgfqpoint{3.042974in}{0.696286in}}%
\pgfpathlineto{\pgfqpoint{3.043764in}{0.695825in}}%
\pgfpathlineto{\pgfqpoint{3.044158in}{0.697588in}}%
\pgfpathlineto{\pgfqpoint{3.044948in}{0.696676in}}%
\pgfpathlineto{\pgfqpoint{3.045737in}{0.693014in}}%
\pgfpathlineto{\pgfqpoint{3.046132in}{0.694739in}}%
\pgfpathlineto{\pgfqpoint{3.047316in}{0.701633in}}%
\pgfpathlineto{\pgfqpoint{3.048106in}{0.698825in}}%
\pgfpathlineto{\pgfqpoint{3.048895in}{0.697303in}}%
\pgfpathlineto{\pgfqpoint{3.054422in}{0.676293in}}%
\pgfpathlineto{\pgfqpoint{3.055606in}{0.679498in}}%
\pgfpathlineto{\pgfqpoint{3.056001in}{0.679877in}}%
\pgfpathlineto{\pgfqpoint{3.056395in}{0.678252in}}%
\pgfpathlineto{\pgfqpoint{3.056790in}{0.678568in}}%
\pgfpathlineto{\pgfqpoint{3.057580in}{0.676757in}}%
\pgfpathlineto{\pgfqpoint{3.058369in}{0.675882in}}%
\pgfpathlineto{\pgfqpoint{3.058764in}{0.676452in}}%
\pgfpathlineto{\pgfqpoint{3.060738in}{0.685061in}}%
\pgfpathlineto{\pgfqpoint{3.061527in}{0.684134in}}%
\pgfpathlineto{\pgfqpoint{3.062317in}{0.684818in}}%
\pgfpathlineto{\pgfqpoint{3.062711in}{0.685854in}}%
\pgfpathlineto{\pgfqpoint{3.063106in}{0.683648in}}%
\pgfpathlineto{\pgfqpoint{3.063501in}{0.683680in}}%
\pgfpathlineto{\pgfqpoint{3.064685in}{0.680638in}}%
\pgfpathlineto{\pgfqpoint{3.065475in}{0.678612in}}%
\pgfpathlineto{\pgfqpoint{3.065869in}{0.679928in}}%
\pgfpathlineto{\pgfqpoint{3.066659in}{0.681697in}}%
\pgfpathlineto{\pgfqpoint{3.067054in}{0.681206in}}%
\pgfpathlineto{\pgfqpoint{3.070212in}{0.668312in}}%
\pgfpathlineto{\pgfqpoint{3.071001in}{0.667571in}}%
\pgfpathlineto{\pgfqpoint{3.071396in}{0.668854in}}%
\pgfpathlineto{\pgfqpoint{3.072975in}{0.671965in}}%
\pgfpathlineto{\pgfqpoint{3.073370in}{0.671787in}}%
\pgfpathlineto{\pgfqpoint{3.074554in}{0.669017in}}%
\pgfpathlineto{\pgfqpoint{3.076133in}{0.662763in}}%
\pgfpathlineto{\pgfqpoint{3.076528in}{0.666482in}}%
\pgfpathlineto{\pgfqpoint{3.077317in}{0.667197in}}%
\pgfpathlineto{\pgfqpoint{3.077712in}{0.668464in}}%
\pgfpathlineto{\pgfqpoint{3.078501in}{0.666592in}}%
\pgfpathlineto{\pgfqpoint{3.080475in}{0.658420in}}%
\pgfpathlineto{\pgfqpoint{3.081265in}{0.663022in}}%
\pgfpathlineto{\pgfqpoint{3.082449in}{0.661395in}}%
\pgfpathlineto{\pgfqpoint{3.083238in}{0.668313in}}%
\pgfpathlineto{\pgfqpoint{3.086002in}{0.681241in}}%
\pgfpathlineto{\pgfqpoint{3.087581in}{0.644885in}}%
\pgfpathlineto{\pgfqpoint{3.088370in}{0.649307in}}%
\pgfpathlineto{\pgfqpoint{3.089949in}{0.660514in}}%
\pgfpathlineto{\pgfqpoint{3.090739in}{0.671076in}}%
\pgfpathlineto{\pgfqpoint{3.091528in}{0.670627in}}%
\pgfpathlineto{\pgfqpoint{3.092712in}{0.667072in}}%
\pgfpathlineto{\pgfqpoint{3.093107in}{0.663809in}}%
\pgfpathlineto{\pgfqpoint{3.094291in}{0.666095in}}%
\pgfpathlineto{\pgfqpoint{3.097055in}{0.677628in}}%
\pgfpathlineto{\pgfqpoint{3.097449in}{0.678773in}}%
\pgfpathlineto{\pgfqpoint{3.097844in}{0.677523in}}%
\pgfpathlineto{\pgfqpoint{3.101002in}{0.655752in}}%
\pgfpathlineto{\pgfqpoint{3.103371in}{0.654646in}}%
\pgfpathlineto{\pgfqpoint{3.105344in}{0.661697in}}%
\pgfpathlineto{\pgfqpoint{3.107318in}{0.675287in}}%
\pgfpathlineto{\pgfqpoint{3.108108in}{0.672234in}}%
\pgfpathlineto{\pgfqpoint{3.108502in}{0.670628in}}%
\pgfpathlineto{\pgfqpoint{3.108897in}{0.671995in}}%
\pgfpathlineto{\pgfqpoint{3.112055in}{0.689873in}}%
\pgfpathlineto{\pgfqpoint{3.112450in}{0.689549in}}%
\pgfpathlineto{\pgfqpoint{3.114424in}{0.680923in}}%
\pgfpathlineto{\pgfqpoint{3.115608in}{0.682290in}}%
\pgfpathlineto{\pgfqpoint{3.116003in}{0.681015in}}%
\pgfpathlineto{\pgfqpoint{3.116792in}{0.682168in}}%
\pgfpathlineto{\pgfqpoint{3.117187in}{0.682434in}}%
\pgfpathlineto{\pgfqpoint{3.118371in}{0.694697in}}%
\pgfpathlineto{\pgfqpoint{3.119161in}{0.689433in}}%
\pgfpathlineto{\pgfqpoint{3.121924in}{0.675862in}}%
\pgfpathlineto{\pgfqpoint{3.122319in}{0.676709in}}%
\pgfpathlineto{\pgfqpoint{3.123108in}{0.675532in}}%
\pgfpathlineto{\pgfqpoint{3.125082in}{0.667692in}}%
\pgfpathlineto{\pgfqpoint{3.125477in}{0.669867in}}%
\pgfpathlineto{\pgfqpoint{3.126266in}{0.674633in}}%
\pgfpathlineto{\pgfqpoint{3.126661in}{0.673246in}}%
\pgfpathlineto{\pgfqpoint{3.132187in}{0.629541in}}%
\pgfpathlineto{\pgfqpoint{3.132977in}{0.628842in}}%
\pgfpathlineto{\pgfqpoint{3.133371in}{0.629438in}}%
\pgfpathlineto{\pgfqpoint{3.136529in}{0.643561in}}%
\pgfpathlineto{\pgfqpoint{3.137714in}{0.638685in}}%
\pgfpathlineto{\pgfqpoint{3.139687in}{0.619967in}}%
\pgfpathlineto{\pgfqpoint{3.140082in}{0.626759in}}%
\pgfpathlineto{\pgfqpoint{3.141266in}{0.638799in}}%
\pgfpathlineto{\pgfqpoint{3.142056in}{0.638034in}}%
\pgfpathlineto{\pgfqpoint{3.142845in}{0.637264in}}%
\pgfpathlineto{\pgfqpoint{3.145609in}{0.647746in}}%
\pgfpathlineto{\pgfqpoint{3.147977in}{0.659516in}}%
\pgfpathlineto{\pgfqpoint{3.148372in}{0.657662in}}%
\pgfpathlineto{\pgfqpoint{3.149556in}{0.655864in}}%
\pgfpathlineto{\pgfqpoint{3.149951in}{0.657773in}}%
\pgfpathlineto{\pgfqpoint{3.150346in}{0.655073in}}%
\pgfpathlineto{\pgfqpoint{3.152319in}{0.648157in}}%
\pgfpathlineto{\pgfqpoint{3.152714in}{0.648458in}}%
\pgfpathlineto{\pgfqpoint{3.153109in}{0.647023in}}%
\pgfpathlineto{\pgfqpoint{3.155477in}{0.643265in}}%
\pgfpathlineto{\pgfqpoint{3.158241in}{0.646542in}}%
\pgfpathlineto{\pgfqpoint{3.159425in}{0.647201in}}%
\pgfpathlineto{\pgfqpoint{3.159820in}{0.646592in}}%
\pgfpathlineto{\pgfqpoint{3.160609in}{0.644708in}}%
\pgfpathlineto{\pgfqpoint{3.161793in}{0.645493in}}%
\pgfpathlineto{\pgfqpoint{3.163767in}{0.647428in}}%
\pgfpathlineto{\pgfqpoint{3.164162in}{0.646149in}}%
\pgfpathlineto{\pgfqpoint{3.165346in}{0.644865in}}%
\pgfpathlineto{\pgfqpoint{3.172057in}{0.680067in}}%
\pgfpathlineto{\pgfqpoint{3.174031in}{0.686821in}}%
\pgfpathlineto{\pgfqpoint{3.175215in}{0.691491in}}%
\pgfpathlineto{\pgfqpoint{3.176399in}{0.689855in}}%
\pgfpathlineto{\pgfqpoint{3.179952in}{0.672066in}}%
\pgfpathlineto{\pgfqpoint{3.182715in}{0.651244in}}%
\pgfpathlineto{\pgfqpoint{3.184294in}{0.644487in}}%
\pgfpathlineto{\pgfqpoint{3.184689in}{0.645424in}}%
\pgfpathlineto{\pgfqpoint{3.185873in}{0.658130in}}%
\pgfpathlineto{\pgfqpoint{3.186663in}{0.655895in}}%
\pgfpathlineto{\pgfqpoint{3.190610in}{0.634589in}}%
\pgfpathlineto{\pgfqpoint{3.191400in}{0.638286in}}%
\pgfpathlineto{\pgfqpoint{3.197321in}{0.656710in}}%
\pgfpathlineto{\pgfqpoint{3.197321in}{0.656710in}}%
\pgfusepath{stroke}%
\end{pgfscope}%
\begin{pgfscope}%
\pgfpathrectangle{\pgfqpoint{0.608025in}{0.484444in}}{\pgfqpoint{2.712595in}{1.541287in}}%
\pgfusepath{clip}%
\pgfsetbuttcap%
\pgfsetmiterjoin%
\definecolor{currentfill}{rgb}{0.839216,0.152941,0.156863}%
\pgfsetfillcolor{currentfill}%
\pgfsetlinewidth{1.003750pt}%
\definecolor{currentstroke}{rgb}{0.839216,0.152941,0.156863}%
\pgfsetstrokecolor{currentstroke}%
\pgfsetdash{}{0pt}%
\pgfsys@defobject{currentmarker}{\pgfqpoint{-0.020833in}{-0.020833in}}{\pgfqpoint{0.020833in}{0.020833in}}{%
\pgfpathmoveto{\pgfqpoint{0.020833in}{-0.000000in}}%
\pgfpathlineto{\pgfqpoint{-0.020833in}{0.020833in}}%
\pgfpathlineto{\pgfqpoint{-0.020833in}{-0.020833in}}%
\pgfpathlineto{\pgfqpoint{0.020833in}{-0.000000in}}%
\pgfpathclose%
\pgfusepath{stroke,fill}%
}%
\begin{pgfscope}%
\pgfsys@transformshift{0.790537in}{1.806614in}%
\pgfsys@useobject{currentmarker}{}%
\end{pgfscope}%
\begin{pgfscope}%
\pgfsys@transformshift{0.987912in}{1.347218in}%
\pgfsys@useobject{currentmarker}{}%
\end{pgfscope}%
\begin{pgfscope}%
\pgfsys@transformshift{1.185286in}{1.173972in}%
\pgfsys@useobject{currentmarker}{}%
\end{pgfscope}%
\begin{pgfscope}%
\pgfsys@transformshift{1.382660in}{0.979159in}%
\pgfsys@useobject{currentmarker}{}%
\end{pgfscope}%
\begin{pgfscope}%
\pgfsys@transformshift{1.580035in}{0.963385in}%
\pgfsys@useobject{currentmarker}{}%
\end{pgfscope}%
\begin{pgfscope}%
\pgfsys@transformshift{1.777409in}{0.863007in}%
\pgfsys@useobject{currentmarker}{}%
\end{pgfscope}%
\begin{pgfscope}%
\pgfsys@transformshift{1.974784in}{0.782842in}%
\pgfsys@useobject{currentmarker}{}%
\end{pgfscope}%
\begin{pgfscope}%
\pgfsys@transformshift{2.172158in}{0.732332in}%
\pgfsys@useobject{currentmarker}{}%
\end{pgfscope}%
\begin{pgfscope}%
\pgfsys@transformshift{2.369533in}{0.721872in}%
\pgfsys@useobject{currentmarker}{}%
\end{pgfscope}%
\begin{pgfscope}%
\pgfsys@transformshift{2.566907in}{0.679033in}%
\pgfsys@useobject{currentmarker}{}%
\end{pgfscope}%
\begin{pgfscope}%
\pgfsys@transformshift{2.764281in}{0.721918in}%
\pgfsys@useobject{currentmarker}{}%
\end{pgfscope}%
\begin{pgfscope}%
\pgfsys@transformshift{2.961656in}{0.665603in}%
\pgfsys@useobject{currentmarker}{}%
\end{pgfscope}%
\begin{pgfscope}%
\pgfsys@transformshift{3.159030in}{0.646564in}%
\pgfsys@useobject{currentmarker}{}%
\end{pgfscope}%
\end{pgfscope}%
\begin{pgfscope}%
\pgfpathrectangle{\pgfqpoint{0.608025in}{0.484444in}}{\pgfqpoint{2.712595in}{1.541287in}}%
\pgfusepath{clip}%
\pgfsetrectcap%
\pgfsetroundjoin%
\pgfsetlinewidth{1.505625pt}%
\definecolor{currentstroke}{rgb}{0.580392,0.403922,0.741176}%
\pgfsetstrokecolor{currentstroke}%
\pgfsetdash{}{0pt}%
\pgfpathmoveto{\pgfqpoint{0.731325in}{1.955466in}}%
\pgfpathlineto{\pgfqpoint{0.744746in}{1.945271in}}%
\pgfpathlineto{\pgfqpoint{0.749089in}{1.935578in}}%
\pgfpathlineto{\pgfqpoint{0.764879in}{1.883745in}}%
\pgfpathlineto{\pgfqpoint{0.778695in}{1.842266in}}%
\pgfpathlineto{\pgfqpoint{0.795669in}{1.808149in}}%
\pgfpathlineto{\pgfqpoint{0.809485in}{1.794937in}}%
\pgfpathlineto{\pgfqpoint{0.813827in}{1.786038in}}%
\pgfpathlineto{\pgfqpoint{0.814617in}{1.786939in}}%
\pgfpathlineto{\pgfqpoint{0.815801in}{1.786665in}}%
\pgfpathlineto{\pgfqpoint{0.818170in}{1.772703in}}%
\pgfpathlineto{\pgfqpoint{0.822117in}{1.734140in}}%
\pgfpathlineto{\pgfqpoint{0.826854in}{1.690475in}}%
\pgfpathlineto{\pgfqpoint{0.832775in}{1.637663in}}%
\pgfpathlineto{\pgfqpoint{0.839486in}{1.615184in}}%
\pgfpathlineto{\pgfqpoint{0.845407in}{1.583698in}}%
\pgfpathlineto{\pgfqpoint{0.853302in}{1.543774in}}%
\pgfpathlineto{\pgfqpoint{0.855671in}{1.539904in}}%
\pgfpathlineto{\pgfqpoint{0.856066in}{1.541674in}}%
\pgfpathlineto{\pgfqpoint{0.859224in}{1.548548in}}%
\pgfpathlineto{\pgfqpoint{0.870277in}{1.485003in}}%
\pgfpathlineto{\pgfqpoint{0.870671in}{1.488410in}}%
\pgfpathlineto{\pgfqpoint{0.873829in}{1.531750in}}%
\pgfpathlineto{\pgfqpoint{0.874619in}{1.531190in}}%
\pgfpathlineto{\pgfqpoint{0.875408in}{1.530426in}}%
\pgfpathlineto{\pgfqpoint{0.878961in}{1.543461in}}%
\pgfpathlineto{\pgfqpoint{0.882514in}{1.541822in}}%
\pgfpathlineto{\pgfqpoint{0.884487in}{1.532838in}}%
\pgfpathlineto{\pgfqpoint{0.887251in}{1.501577in}}%
\pgfpathlineto{\pgfqpoint{0.887645in}{1.495863in}}%
\pgfpathlineto{\pgfqpoint{0.888830in}{1.499626in}}%
\pgfpathlineto{\pgfqpoint{0.893961in}{1.514815in}}%
\pgfpathlineto{\pgfqpoint{0.894751in}{1.518885in}}%
\pgfpathlineto{\pgfqpoint{0.895540in}{1.517825in}}%
\pgfpathlineto{\pgfqpoint{0.896330in}{1.515476in}}%
\pgfpathlineto{\pgfqpoint{0.902251in}{1.413359in}}%
\pgfpathlineto{\pgfqpoint{0.902646in}{1.415108in}}%
\pgfpathlineto{\pgfqpoint{0.906988in}{1.439330in}}%
\pgfpathlineto{\pgfqpoint{0.907383in}{1.436196in}}%
\pgfpathlineto{\pgfqpoint{0.912120in}{1.370644in}}%
\pgfpathlineto{\pgfqpoint{0.914883in}{1.361779in}}%
\pgfpathlineto{\pgfqpoint{0.915673in}{1.358447in}}%
\pgfpathlineto{\pgfqpoint{0.916067in}{1.361105in}}%
\pgfpathlineto{\pgfqpoint{0.920015in}{1.376662in}}%
\pgfpathlineto{\pgfqpoint{0.920410in}{1.377409in}}%
\pgfpathlineto{\pgfqpoint{0.923962in}{1.318359in}}%
\pgfpathlineto{\pgfqpoint{0.925147in}{1.327400in}}%
\pgfpathlineto{\pgfqpoint{0.927910in}{1.368438in}}%
\pgfpathlineto{\pgfqpoint{0.928699in}{1.378005in}}%
\pgfpathlineto{\pgfqpoint{0.929489in}{1.371666in}}%
\pgfpathlineto{\pgfqpoint{0.935015in}{1.311456in}}%
\pgfpathlineto{\pgfqpoint{0.935410in}{1.316918in}}%
\pgfpathlineto{\pgfqpoint{0.938173in}{1.377714in}}%
\pgfpathlineto{\pgfqpoint{0.939752in}{1.407401in}}%
\pgfpathlineto{\pgfqpoint{0.940542in}{1.407027in}}%
\pgfpathlineto{\pgfqpoint{0.941726in}{1.404301in}}%
\pgfpathlineto{\pgfqpoint{0.942910in}{1.391538in}}%
\pgfpathlineto{\pgfqpoint{0.943700in}{1.394384in}}%
\pgfpathlineto{\pgfqpoint{0.946463in}{1.400026in}}%
\pgfpathlineto{\pgfqpoint{0.946858in}{1.399888in}}%
\pgfpathlineto{\pgfqpoint{0.948042in}{1.397993in}}%
\pgfpathlineto{\pgfqpoint{0.950411in}{1.385429in}}%
\pgfpathlineto{\pgfqpoint{0.952384in}{1.374350in}}%
\pgfpathlineto{\pgfqpoint{0.952779in}{1.375963in}}%
\pgfpathlineto{\pgfqpoint{0.954753in}{1.385428in}}%
\pgfpathlineto{\pgfqpoint{0.956332in}{1.407072in}}%
\pgfpathlineto{\pgfqpoint{0.957121in}{1.401818in}}%
\pgfpathlineto{\pgfqpoint{0.960279in}{1.369028in}}%
\pgfpathlineto{\pgfqpoint{0.966200in}{1.301631in}}%
\pgfpathlineto{\pgfqpoint{0.967385in}{1.309210in}}%
\pgfpathlineto{\pgfqpoint{0.967779in}{1.306538in}}%
\pgfpathlineto{\pgfqpoint{0.970937in}{1.277443in}}%
\pgfpathlineto{\pgfqpoint{0.972122in}{1.279973in}}%
\pgfpathlineto{\pgfqpoint{0.974095in}{1.288962in}}%
\pgfpathlineto{\pgfqpoint{0.974490in}{1.287168in}}%
\pgfpathlineto{\pgfqpoint{0.976069in}{1.280027in}}%
\pgfpathlineto{\pgfqpoint{0.976859in}{1.283418in}}%
\pgfpathlineto{\pgfqpoint{0.979622in}{1.338353in}}%
\pgfpathlineto{\pgfqpoint{0.980806in}{1.333309in}}%
\pgfpathlineto{\pgfqpoint{0.983175in}{1.265348in}}%
\pgfpathlineto{\pgfqpoint{0.984754in}{1.276358in}}%
\pgfpathlineto{\pgfqpoint{0.987912in}{1.308204in}}%
\pgfpathlineto{\pgfqpoint{0.988701in}{1.307212in}}%
\pgfpathlineto{\pgfqpoint{0.989885in}{1.302575in}}%
\pgfpathlineto{\pgfqpoint{0.991464in}{1.324385in}}%
\pgfpathlineto{\pgfqpoint{0.992649in}{1.319483in}}%
\pgfpathlineto{\pgfqpoint{0.996596in}{1.299788in}}%
\pgfpathlineto{\pgfqpoint{1.000938in}{1.279158in}}%
\pgfpathlineto{\pgfqpoint{1.002912in}{1.239459in}}%
\pgfpathlineto{\pgfqpoint{1.003702in}{1.242224in}}%
\pgfpathlineto{\pgfqpoint{1.006070in}{1.300463in}}%
\pgfpathlineto{\pgfqpoint{1.006860in}{1.288368in}}%
\pgfpathlineto{\pgfqpoint{1.008833in}{1.281119in}}%
\pgfpathlineto{\pgfqpoint{1.009228in}{1.282748in}}%
\pgfpathlineto{\pgfqpoint{1.009623in}{1.278678in}}%
\pgfpathlineto{\pgfqpoint{1.015149in}{1.225402in}}%
\pgfpathlineto{\pgfqpoint{1.017518in}{1.213328in}}%
\pgfpathlineto{\pgfqpoint{1.019492in}{1.210232in}}%
\pgfpathlineto{\pgfqpoint{1.019886in}{1.209241in}}%
\pgfpathlineto{\pgfqpoint{1.020281in}{1.212393in}}%
\pgfpathlineto{\pgfqpoint{1.025808in}{1.270485in}}%
\pgfpathlineto{\pgfqpoint{1.026992in}{1.267995in}}%
\pgfpathlineto{\pgfqpoint{1.031334in}{1.189848in}}%
\pgfpathlineto{\pgfqpoint{1.035676in}{1.135078in}}%
\pgfpathlineto{\pgfqpoint{1.036466in}{1.137173in}}%
\pgfpathlineto{\pgfqpoint{1.040808in}{1.174589in}}%
\pgfpathlineto{\pgfqpoint{1.045940in}{1.258144in}}%
\pgfpathlineto{\pgfqpoint{1.047519in}{1.270097in}}%
\pgfpathlineto{\pgfqpoint{1.049098in}{1.283253in}}%
\pgfpathlineto{\pgfqpoint{1.049492in}{1.280533in}}%
\pgfpathlineto{\pgfqpoint{1.052650in}{1.238777in}}%
\pgfpathlineto{\pgfqpoint{1.053045in}{1.239835in}}%
\pgfpathlineto{\pgfqpoint{1.054229in}{1.260332in}}%
\pgfpathlineto{\pgfqpoint{1.055019in}{1.254064in}}%
\pgfpathlineto{\pgfqpoint{1.058572in}{1.203883in}}%
\pgfpathlineto{\pgfqpoint{1.059361in}{1.213322in}}%
\pgfpathlineto{\pgfqpoint{1.060545in}{1.221636in}}%
\pgfpathlineto{\pgfqpoint{1.060940in}{1.219609in}}%
\pgfpathlineto{\pgfqpoint{1.061730in}{1.207233in}}%
\pgfpathlineto{\pgfqpoint{1.062519in}{1.211341in}}%
\pgfpathlineto{\pgfqpoint{1.066467in}{1.241190in}}%
\pgfpathlineto{\pgfqpoint{1.067256in}{1.234446in}}%
\pgfpathlineto{\pgfqpoint{1.069625in}{1.205031in}}%
\pgfpathlineto{\pgfqpoint{1.070019in}{1.209744in}}%
\pgfpathlineto{\pgfqpoint{1.077125in}{1.275297in}}%
\pgfpathlineto{\pgfqpoint{1.078309in}{1.282450in}}%
\pgfpathlineto{\pgfqpoint{1.078704in}{1.280105in}}%
\pgfpathlineto{\pgfqpoint{1.082257in}{1.226592in}}%
\pgfpathlineto{\pgfqpoint{1.085415in}{1.132346in}}%
\pgfpathlineto{\pgfqpoint{1.086599in}{1.135680in}}%
\pgfpathlineto{\pgfqpoint{1.087388in}{1.137720in}}%
\pgfpathlineto{\pgfqpoint{1.088967in}{1.158593in}}%
\pgfpathlineto{\pgfqpoint{1.089757in}{1.153171in}}%
\pgfpathlineto{\pgfqpoint{1.092125in}{1.134164in}}%
\pgfpathlineto{\pgfqpoint{1.092520in}{1.137744in}}%
\pgfpathlineto{\pgfqpoint{1.092915in}{1.144295in}}%
\pgfpathlineto{\pgfqpoint{1.093310in}{1.142681in}}%
\pgfpathlineto{\pgfqpoint{1.094494in}{1.117072in}}%
\pgfpathlineto{\pgfqpoint{1.095283in}{1.121784in}}%
\pgfpathlineto{\pgfqpoint{1.095678in}{1.126491in}}%
\pgfpathlineto{\pgfqpoint{1.096468in}{1.118051in}}%
\pgfpathlineto{\pgfqpoint{1.097257in}{1.109453in}}%
\pgfpathlineto{\pgfqpoint{1.097652in}{1.119893in}}%
\pgfpathlineto{\pgfqpoint{1.103968in}{1.197823in}}%
\pgfpathlineto{\pgfqpoint{1.104363in}{1.197924in}}%
\pgfpathlineto{\pgfqpoint{1.105152in}{1.209944in}}%
\pgfpathlineto{\pgfqpoint{1.106336in}{1.204442in}}%
\pgfpathlineto{\pgfqpoint{1.106731in}{1.203576in}}%
\pgfpathlineto{\pgfqpoint{1.107126in}{1.206261in}}%
\pgfpathlineto{\pgfqpoint{1.112258in}{1.228341in}}%
\pgfpathlineto{\pgfqpoint{1.113837in}{1.199574in}}%
\pgfpathlineto{\pgfqpoint{1.114626in}{1.204908in}}%
\pgfpathlineto{\pgfqpoint{1.116600in}{1.230463in}}%
\pgfpathlineto{\pgfqpoint{1.116995in}{1.218031in}}%
\pgfpathlineto{\pgfqpoint{1.119758in}{1.173350in}}%
\pgfpathlineto{\pgfqpoint{1.124890in}{1.138553in}}%
\pgfpathlineto{\pgfqpoint{1.125284in}{1.146112in}}%
\pgfpathlineto{\pgfqpoint{1.126863in}{1.177162in}}%
\pgfpathlineto{\pgfqpoint{1.128047in}{1.175517in}}%
\pgfpathlineto{\pgfqpoint{1.129232in}{1.167075in}}%
\pgfpathlineto{\pgfqpoint{1.133179in}{1.095382in}}%
\pgfpathlineto{\pgfqpoint{1.133969in}{1.105008in}}%
\pgfpathlineto{\pgfqpoint{1.135942in}{1.138742in}}%
\pgfpathlineto{\pgfqpoint{1.137127in}{1.132798in}}%
\pgfpathlineto{\pgfqpoint{1.137521in}{1.130125in}}%
\pgfpathlineto{\pgfqpoint{1.137916in}{1.133996in}}%
\pgfpathlineto{\pgfqpoint{1.138706in}{1.132591in}}%
\pgfpathlineto{\pgfqpoint{1.140285in}{1.146314in}}%
\pgfpathlineto{\pgfqpoint{1.140679in}{1.143616in}}%
\pgfpathlineto{\pgfqpoint{1.144627in}{1.110219in}}%
\pgfpathlineto{\pgfqpoint{1.145416in}{1.118138in}}%
\pgfpathlineto{\pgfqpoint{1.150548in}{1.166042in}}%
\pgfpathlineto{\pgfqpoint{1.150943in}{1.165033in}}%
\pgfpathlineto{\pgfqpoint{1.153311in}{1.153366in}}%
\pgfpathlineto{\pgfqpoint{1.157654in}{1.101778in}}%
\pgfpathlineto{\pgfqpoint{1.158838in}{1.110870in}}%
\pgfpathlineto{\pgfqpoint{1.159233in}{1.118317in}}%
\pgfpathlineto{\pgfqpoint{1.160022in}{1.117898in}}%
\pgfpathlineto{\pgfqpoint{1.163180in}{1.077244in}}%
\pgfpathlineto{\pgfqpoint{1.163970in}{1.078381in}}%
\pgfpathlineto{\pgfqpoint{1.165154in}{1.084077in}}%
\pgfpathlineto{\pgfqpoint{1.168707in}{1.128564in}}%
\pgfpathlineto{\pgfqpoint{1.170286in}{1.131990in}}%
\pgfpathlineto{\pgfqpoint{1.170680in}{1.130012in}}%
\pgfpathlineto{\pgfqpoint{1.171075in}{1.127992in}}%
\pgfpathlineto{\pgfqpoint{1.171470in}{1.130457in}}%
\pgfpathlineto{\pgfqpoint{1.174628in}{1.146762in}}%
\pgfpathlineto{\pgfqpoint{1.175023in}{1.145200in}}%
\pgfpathlineto{\pgfqpoint{1.176602in}{1.134723in}}%
\pgfpathlineto{\pgfqpoint{1.176996in}{1.135329in}}%
\pgfpathlineto{\pgfqpoint{1.179760in}{1.157302in}}%
\pgfpathlineto{\pgfqpoint{1.180154in}{1.159632in}}%
\pgfpathlineto{\pgfqpoint{1.180549in}{1.155989in}}%
\pgfpathlineto{\pgfqpoint{1.183312in}{1.134252in}}%
\pgfpathlineto{\pgfqpoint{1.184891in}{1.134862in}}%
\pgfpathlineto{\pgfqpoint{1.187655in}{1.096920in}}%
\pgfpathlineto{\pgfqpoint{1.188049in}{1.095206in}}%
\pgfpathlineto{\pgfqpoint{1.189234in}{1.068213in}}%
\pgfpathlineto{\pgfqpoint{1.190418in}{1.074707in}}%
\pgfpathlineto{\pgfqpoint{1.196734in}{1.006174in}}%
\pgfpathlineto{\pgfqpoint{1.197523in}{1.014287in}}%
\pgfpathlineto{\pgfqpoint{1.201076in}{1.056602in}}%
\pgfpathlineto{\pgfqpoint{1.205024in}{1.081068in}}%
\pgfpathlineto{\pgfqpoint{1.206208in}{1.097999in}}%
\pgfpathlineto{\pgfqpoint{1.207392in}{1.096744in}}%
\pgfpathlineto{\pgfqpoint{1.211734in}{1.138759in}}%
\pgfpathlineto{\pgfqpoint{1.213313in}{1.128950in}}%
\pgfpathlineto{\pgfqpoint{1.214103in}{1.131782in}}%
\pgfpathlineto{\pgfqpoint{1.217261in}{1.155280in}}%
\pgfpathlineto{\pgfqpoint{1.218445in}{1.153694in}}%
\pgfpathlineto{\pgfqpoint{1.220024in}{1.132956in}}%
\pgfpathlineto{\pgfqpoint{1.221603in}{1.105419in}}%
\pgfpathlineto{\pgfqpoint{1.221998in}{1.108222in}}%
\pgfpathlineto{\pgfqpoint{1.222392in}{1.111687in}}%
\pgfpathlineto{\pgfqpoint{1.222787in}{1.106858in}}%
\pgfpathlineto{\pgfqpoint{1.226340in}{1.070231in}}%
\pgfpathlineto{\pgfqpoint{1.226735in}{1.079109in}}%
\pgfpathlineto{\pgfqpoint{1.227129in}{1.082145in}}%
\pgfpathlineto{\pgfqpoint{1.228314in}{1.080505in}}%
\pgfpathlineto{\pgfqpoint{1.228708in}{1.080766in}}%
\pgfpathlineto{\pgfqpoint{1.229498in}{1.086325in}}%
\pgfpathlineto{\pgfqpoint{1.229893in}{1.084990in}}%
\pgfpathlineto{\pgfqpoint{1.230682in}{1.076318in}}%
\pgfpathlineto{\pgfqpoint{1.231472in}{1.079763in}}%
\pgfpathlineto{\pgfqpoint{1.232261in}{1.086505in}}%
\pgfpathlineto{\pgfqpoint{1.232656in}{1.081895in}}%
\pgfpathlineto{\pgfqpoint{1.233840in}{1.062122in}}%
\pgfpathlineto{\pgfqpoint{1.234235in}{1.067841in}}%
\pgfpathlineto{\pgfqpoint{1.236603in}{1.090745in}}%
\pgfpathlineto{\pgfqpoint{1.238577in}{1.070601in}}%
\pgfpathlineto{\pgfqpoint{1.239367in}{1.075283in}}%
\pgfpathlineto{\pgfqpoint{1.242919in}{1.097156in}}%
\pgfpathlineto{\pgfqpoint{1.243314in}{1.096914in}}%
\pgfpathlineto{\pgfqpoint{1.247656in}{1.061629in}}%
\pgfpathlineto{\pgfqpoint{1.248841in}{1.048835in}}%
\pgfpathlineto{\pgfqpoint{1.249630in}{1.051177in}}%
\pgfpathlineto{\pgfqpoint{1.250025in}{1.051718in}}%
\pgfpathlineto{\pgfqpoint{1.253183in}{0.988409in}}%
\pgfpathlineto{\pgfqpoint{1.254367in}{0.993923in}}%
\pgfpathlineto{\pgfqpoint{1.256736in}{0.998921in}}%
\pgfpathlineto{\pgfqpoint{1.257525in}{0.988371in}}%
\pgfpathlineto{\pgfqpoint{1.257920in}{0.993832in}}%
\pgfpathlineto{\pgfqpoint{1.263446in}{1.048303in}}%
\pgfpathlineto{\pgfqpoint{1.266210in}{1.070795in}}%
\pgfpathlineto{\pgfqpoint{1.268578in}{1.088444in}}%
\pgfpathlineto{\pgfqpoint{1.270157in}{1.109748in}}%
\pgfpathlineto{\pgfqpoint{1.274105in}{1.166948in}}%
\pgfpathlineto{\pgfqpoint{1.274499in}{1.170861in}}%
\pgfpathlineto{\pgfqpoint{1.275289in}{1.165251in}}%
\pgfpathlineto{\pgfqpoint{1.278447in}{1.122246in}}%
\pgfpathlineto{\pgfqpoint{1.281605in}{1.087170in}}%
\pgfpathlineto{\pgfqpoint{1.282394in}{1.085037in}}%
\pgfpathlineto{\pgfqpoint{1.283973in}{1.070737in}}%
\pgfpathlineto{\pgfqpoint{1.284368in}{1.071836in}}%
\pgfpathlineto{\pgfqpoint{1.285158in}{1.065889in}}%
\pgfpathlineto{\pgfqpoint{1.285552in}{1.073897in}}%
\pgfpathlineto{\pgfqpoint{1.288710in}{1.115202in}}%
\pgfpathlineto{\pgfqpoint{1.289105in}{1.112331in}}%
\pgfpathlineto{\pgfqpoint{1.293052in}{1.035431in}}%
\pgfpathlineto{\pgfqpoint{1.293447in}{1.039806in}}%
\pgfpathlineto{\pgfqpoint{1.297000in}{1.078639in}}%
\pgfpathlineto{\pgfqpoint{1.297789in}{1.071735in}}%
\pgfpathlineto{\pgfqpoint{1.300553in}{1.038980in}}%
\pgfpathlineto{\pgfqpoint{1.301342in}{1.039080in}}%
\pgfpathlineto{\pgfqpoint{1.301737in}{1.039758in}}%
\pgfpathlineto{\pgfqpoint{1.304500in}{1.068796in}}%
\pgfpathlineto{\pgfqpoint{1.305684in}{1.069557in}}%
\pgfpathlineto{\pgfqpoint{1.308842in}{1.035672in}}%
\pgfpathlineto{\pgfqpoint{1.309237in}{1.035332in}}%
\pgfpathlineto{\pgfqpoint{1.313579in}{1.004530in}}%
\pgfpathlineto{\pgfqpoint{1.315158in}{0.989538in}}%
\pgfpathlineto{\pgfqpoint{1.315948in}{0.965162in}}%
\pgfpathlineto{\pgfqpoint{1.316737in}{0.976843in}}%
\pgfpathlineto{\pgfqpoint{1.317132in}{0.981845in}}%
\pgfpathlineto{\pgfqpoint{1.317922in}{0.977083in}}%
\pgfpathlineto{\pgfqpoint{1.318316in}{0.977168in}}%
\pgfpathlineto{\pgfqpoint{1.318711in}{0.974082in}}%
\pgfpathlineto{\pgfqpoint{1.319106in}{0.975035in}}%
\pgfpathlineto{\pgfqpoint{1.319895in}{0.988156in}}%
\pgfpathlineto{\pgfqpoint{1.320685in}{0.983181in}}%
\pgfpathlineto{\pgfqpoint{1.322659in}{0.962521in}}%
\pgfpathlineto{\pgfqpoint{1.323053in}{0.969083in}}%
\pgfpathlineto{\pgfqpoint{1.325422in}{0.987720in}}%
\pgfpathlineto{\pgfqpoint{1.326211in}{0.985978in}}%
\pgfpathlineto{\pgfqpoint{1.328975in}{1.057858in}}%
\pgfpathlineto{\pgfqpoint{1.329369in}{1.057094in}}%
\pgfpathlineto{\pgfqpoint{1.330948in}{1.028501in}}%
\pgfpathlineto{\pgfqpoint{1.332133in}{1.000835in}}%
\pgfpathlineto{\pgfqpoint{1.332922in}{1.010920in}}%
\pgfpathlineto{\pgfqpoint{1.336080in}{1.060907in}}%
\pgfpathlineto{\pgfqpoint{1.336870in}{1.062092in}}%
\pgfpathlineto{\pgfqpoint{1.339238in}{1.080167in}}%
\pgfpathlineto{\pgfqpoint{1.341212in}{1.082172in}}%
\pgfpathlineto{\pgfqpoint{1.342001in}{1.081669in}}%
\pgfpathlineto{\pgfqpoint{1.345554in}{1.051899in}}%
\pgfpathlineto{\pgfqpoint{1.346738in}{1.065193in}}%
\pgfpathlineto{\pgfqpoint{1.347133in}{1.062252in}}%
\pgfpathlineto{\pgfqpoint{1.351475in}{0.994606in}}%
\pgfpathlineto{\pgfqpoint{1.351870in}{0.994736in}}%
\pgfpathlineto{\pgfqpoint{1.352265in}{1.000128in}}%
\pgfpathlineto{\pgfqpoint{1.352660in}{0.994770in}}%
\pgfpathlineto{\pgfqpoint{1.356212in}{0.958465in}}%
\pgfpathlineto{\pgfqpoint{1.356607in}{0.958442in}}%
\pgfpathlineto{\pgfqpoint{1.357397in}{0.969978in}}%
\pgfpathlineto{\pgfqpoint{1.358186in}{0.961058in}}%
\pgfpathlineto{\pgfqpoint{1.358581in}{0.956022in}}%
\pgfpathlineto{\pgfqpoint{1.359370in}{0.959910in}}%
\pgfpathlineto{\pgfqpoint{1.362923in}{0.986431in}}%
\pgfpathlineto{\pgfqpoint{1.364897in}{0.997345in}}%
\pgfpathlineto{\pgfqpoint{1.365292in}{0.997069in}}%
\pgfpathlineto{\pgfqpoint{1.366081in}{0.994230in}}%
\pgfpathlineto{\pgfqpoint{1.366476in}{0.998170in}}%
\pgfpathlineto{\pgfqpoint{1.366871in}{0.998620in}}%
\pgfpathlineto{\pgfqpoint{1.367265in}{0.996818in}}%
\pgfpathlineto{\pgfqpoint{1.368055in}{0.997751in}}%
\pgfpathlineto{\pgfqpoint{1.369239in}{0.984590in}}%
\pgfpathlineto{\pgfqpoint{1.373581in}{0.943935in}}%
\pgfpathlineto{\pgfqpoint{1.374371in}{0.951421in}}%
\pgfpathlineto{\pgfqpoint{1.376739in}{0.983480in}}%
\pgfpathlineto{\pgfqpoint{1.377923in}{0.974378in}}%
\pgfpathlineto{\pgfqpoint{1.379108in}{0.966806in}}%
\pgfpathlineto{\pgfqpoint{1.380687in}{0.957910in}}%
\pgfpathlineto{\pgfqpoint{1.381081in}{0.961433in}}%
\pgfpathlineto{\pgfqpoint{1.381476in}{0.963980in}}%
\pgfpathlineto{\pgfqpoint{1.381871in}{0.957619in}}%
\pgfpathlineto{\pgfqpoint{1.385818in}{0.914412in}}%
\pgfpathlineto{\pgfqpoint{1.387003in}{0.912997in}}%
\pgfpathlineto{\pgfqpoint{1.388187in}{0.938726in}}%
\pgfpathlineto{\pgfqpoint{1.389371in}{0.927657in}}%
\pgfpathlineto{\pgfqpoint{1.390950in}{0.893246in}}%
\pgfpathlineto{\pgfqpoint{1.392134in}{0.906351in}}%
\pgfpathlineto{\pgfqpoint{1.392529in}{0.905350in}}%
\pgfpathlineto{\pgfqpoint{1.402398in}{1.004619in}}%
\pgfpathlineto{\pgfqpoint{1.402793in}{1.004145in}}%
\pgfpathlineto{\pgfqpoint{1.406740in}{0.973083in}}%
\pgfpathlineto{\pgfqpoint{1.407530in}{0.976311in}}%
\pgfpathlineto{\pgfqpoint{1.407924in}{0.975978in}}%
\pgfpathlineto{\pgfqpoint{1.411082in}{0.941877in}}%
\pgfpathlineto{\pgfqpoint{1.411477in}{0.943113in}}%
\pgfpathlineto{\pgfqpoint{1.413846in}{0.947055in}}%
\pgfpathlineto{\pgfqpoint{1.414240in}{0.946838in}}%
\pgfpathlineto{\pgfqpoint{1.414635in}{0.945993in}}%
\pgfpathlineto{\pgfqpoint{1.417398in}{0.916182in}}%
\pgfpathlineto{\pgfqpoint{1.417793in}{0.918502in}}%
\pgfpathlineto{\pgfqpoint{1.420556in}{0.950020in}}%
\pgfpathlineto{\pgfqpoint{1.420951in}{0.950827in}}%
\pgfpathlineto{\pgfqpoint{1.421346in}{0.956887in}}%
\pgfpathlineto{\pgfqpoint{1.422135in}{0.947079in}}%
\pgfpathlineto{\pgfqpoint{1.423714in}{0.930851in}}%
\pgfpathlineto{\pgfqpoint{1.424899in}{0.932594in}}%
\pgfpathlineto{\pgfqpoint{1.425293in}{0.932940in}}%
\pgfpathlineto{\pgfqpoint{1.426083in}{0.921510in}}%
\pgfpathlineto{\pgfqpoint{1.426478in}{0.930699in}}%
\pgfpathlineto{\pgfqpoint{1.429241in}{0.976129in}}%
\pgfpathlineto{\pgfqpoint{1.433583in}{0.938886in}}%
\pgfpathlineto{\pgfqpoint{1.439899in}{0.978764in}}%
\pgfpathlineto{\pgfqpoint{1.441083in}{0.975997in}}%
\pgfpathlineto{\pgfqpoint{1.445820in}{0.928397in}}%
\pgfpathlineto{\pgfqpoint{1.446215in}{0.930796in}}%
\pgfpathlineto{\pgfqpoint{1.446610in}{0.931695in}}%
\pgfpathlineto{\pgfqpoint{1.448978in}{0.905220in}}%
\pgfpathlineto{\pgfqpoint{1.450557in}{0.879571in}}%
\pgfpathlineto{\pgfqpoint{1.450952in}{0.887253in}}%
\pgfpathlineto{\pgfqpoint{1.452926in}{0.939664in}}%
\pgfpathlineto{\pgfqpoint{1.453715in}{0.938293in}}%
\pgfpathlineto{\pgfqpoint{1.454110in}{0.934062in}}%
\pgfpathlineto{\pgfqpoint{1.454900in}{0.935489in}}%
\pgfpathlineto{\pgfqpoint{1.455689in}{0.942268in}}%
\pgfpathlineto{\pgfqpoint{1.456873in}{0.939829in}}%
\pgfpathlineto{\pgfqpoint{1.458058in}{0.928642in}}%
\pgfpathlineto{\pgfqpoint{1.458847in}{0.938179in}}%
\pgfpathlineto{\pgfqpoint{1.460031in}{0.941165in}}%
\pgfpathlineto{\pgfqpoint{1.464373in}{0.999222in}}%
\pgfpathlineto{\pgfqpoint{1.464768in}{0.999043in}}%
\pgfpathlineto{\pgfqpoint{1.465163in}{0.998748in}}%
\pgfpathlineto{\pgfqpoint{1.468321in}{0.951854in}}%
\pgfpathlineto{\pgfqpoint{1.468716in}{0.952942in}}%
\pgfpathlineto{\pgfqpoint{1.470689in}{0.956346in}}%
\pgfpathlineto{\pgfqpoint{1.475032in}{0.922222in}}%
\pgfpathlineto{\pgfqpoint{1.477005in}{0.901011in}}%
\pgfpathlineto{\pgfqpoint{1.478584in}{0.926125in}}%
\pgfpathlineto{\pgfqpoint{1.479769in}{0.920819in}}%
\pgfpathlineto{\pgfqpoint{1.480163in}{0.918822in}}%
\pgfpathlineto{\pgfqpoint{1.480558in}{0.921233in}}%
\pgfpathlineto{\pgfqpoint{1.484506in}{0.941459in}}%
\pgfpathlineto{\pgfqpoint{1.484900in}{0.934317in}}%
\pgfpathlineto{\pgfqpoint{1.487664in}{0.904782in}}%
\pgfpathlineto{\pgfqpoint{1.488848in}{0.895293in}}%
\pgfpathlineto{\pgfqpoint{1.489243in}{0.896054in}}%
\pgfpathlineto{\pgfqpoint{1.490032in}{0.910659in}}%
\pgfpathlineto{\pgfqpoint{1.490822in}{0.902341in}}%
\pgfpathlineto{\pgfqpoint{1.495164in}{0.879122in}}%
\pgfpathlineto{\pgfqpoint{1.495559in}{0.881010in}}%
\pgfpathlineto{\pgfqpoint{1.498322in}{0.915379in}}%
\pgfpathlineto{\pgfqpoint{1.498717in}{0.913920in}}%
\pgfpathlineto{\pgfqpoint{1.499111in}{0.912036in}}%
\pgfpathlineto{\pgfqpoint{1.499901in}{0.915446in}}%
\pgfpathlineto{\pgfqpoint{1.502664in}{0.930977in}}%
\pgfpathlineto{\pgfqpoint{1.503454in}{0.924028in}}%
\pgfpathlineto{\pgfqpoint{1.504243in}{0.916115in}}%
\pgfpathlineto{\pgfqpoint{1.505033in}{0.919092in}}%
\pgfpathlineto{\pgfqpoint{1.505822in}{0.927763in}}%
\pgfpathlineto{\pgfqpoint{1.509375in}{0.961611in}}%
\pgfpathlineto{\pgfqpoint{1.509770in}{0.963390in}}%
\pgfpathlineto{\pgfqpoint{1.510559in}{0.960081in}}%
\pgfpathlineto{\pgfqpoint{1.514901in}{0.895397in}}%
\pgfpathlineto{\pgfqpoint{1.515296in}{0.900015in}}%
\pgfpathlineto{\pgfqpoint{1.517665in}{0.953474in}}%
\pgfpathlineto{\pgfqpoint{1.518849in}{0.929663in}}%
\pgfpathlineto{\pgfqpoint{1.520033in}{0.921042in}}%
\pgfpathlineto{\pgfqpoint{1.520428in}{0.923614in}}%
\pgfpathlineto{\pgfqpoint{1.522402in}{0.950685in}}%
\pgfpathlineto{\pgfqpoint{1.523191in}{0.944195in}}%
\pgfpathlineto{\pgfqpoint{1.526349in}{0.918123in}}%
\pgfpathlineto{\pgfqpoint{1.526744in}{0.921610in}}%
\pgfpathlineto{\pgfqpoint{1.530691in}{0.971622in}}%
\pgfpathlineto{\pgfqpoint{1.535034in}{0.920438in}}%
\pgfpathlineto{\pgfqpoint{1.535428in}{0.922887in}}%
\pgfpathlineto{\pgfqpoint{1.535823in}{0.924775in}}%
\pgfpathlineto{\pgfqpoint{1.536218in}{0.923588in}}%
\pgfpathlineto{\pgfqpoint{1.538981in}{0.908443in}}%
\pgfpathlineto{\pgfqpoint{1.539376in}{0.911312in}}%
\pgfpathlineto{\pgfqpoint{1.540165in}{0.908038in}}%
\pgfpathlineto{\pgfqpoint{1.541744in}{0.901562in}}%
\pgfpathlineto{\pgfqpoint{1.544113in}{0.866355in}}%
\pgfpathlineto{\pgfqpoint{1.544902in}{0.867167in}}%
\pgfpathlineto{\pgfqpoint{1.545297in}{0.865818in}}%
\pgfpathlineto{\pgfqpoint{1.546086in}{0.868359in}}%
\pgfpathlineto{\pgfqpoint{1.547665in}{0.880643in}}%
\pgfpathlineto{\pgfqpoint{1.548060in}{0.877861in}}%
\pgfpathlineto{\pgfqpoint{1.548455in}{0.873412in}}%
\pgfpathlineto{\pgfqpoint{1.549244in}{0.875277in}}%
\pgfpathlineto{\pgfqpoint{1.550034in}{0.890870in}}%
\pgfpathlineto{\pgfqpoint{1.550823in}{0.884076in}}%
\pgfpathlineto{\pgfqpoint{1.551218in}{0.879294in}}%
\pgfpathlineto{\pgfqpoint{1.551613in}{0.886338in}}%
\pgfpathlineto{\pgfqpoint{1.554376in}{0.920192in}}%
\pgfpathlineto{\pgfqpoint{1.554771in}{0.913840in}}%
\pgfpathlineto{\pgfqpoint{1.557929in}{0.878540in}}%
\pgfpathlineto{\pgfqpoint{1.558718in}{0.874606in}}%
\pgfpathlineto{\pgfqpoint{1.559113in}{0.879108in}}%
\pgfpathlineto{\pgfqpoint{1.559508in}{0.883477in}}%
\pgfpathlineto{\pgfqpoint{1.560297in}{0.877010in}}%
\pgfpathlineto{\pgfqpoint{1.561482in}{0.867614in}}%
\pgfpathlineto{\pgfqpoint{1.562271in}{0.872207in}}%
\pgfpathlineto{\pgfqpoint{1.564640in}{0.879251in}}%
\pgfpathlineto{\pgfqpoint{1.565429in}{0.877224in}}%
\pgfpathlineto{\pgfqpoint{1.567403in}{0.913093in}}%
\pgfpathlineto{\pgfqpoint{1.568192in}{0.912856in}}%
\pgfpathlineto{\pgfqpoint{1.568587in}{0.914216in}}%
\pgfpathlineto{\pgfqpoint{1.568982in}{0.910069in}}%
\pgfpathlineto{\pgfqpoint{1.572140in}{0.890725in}}%
\pgfpathlineto{\pgfqpoint{1.576087in}{0.917446in}}%
\pgfpathlineto{\pgfqpoint{1.577272in}{0.931165in}}%
\pgfpathlineto{\pgfqpoint{1.578061in}{0.925844in}}%
\pgfpathlineto{\pgfqpoint{1.579245in}{0.931407in}}%
\pgfpathlineto{\pgfqpoint{1.582798in}{0.953950in}}%
\pgfpathlineto{\pgfqpoint{1.583588in}{0.954534in}}%
\pgfpathlineto{\pgfqpoint{1.583982in}{0.948737in}}%
\pgfpathlineto{\pgfqpoint{1.587140in}{0.922214in}}%
\pgfpathlineto{\pgfqpoint{1.588719in}{0.940090in}}%
\pgfpathlineto{\pgfqpoint{1.589114in}{0.933346in}}%
\pgfpathlineto{\pgfqpoint{1.590693in}{0.909114in}}%
\pgfpathlineto{\pgfqpoint{1.591877in}{0.913261in}}%
\pgfpathlineto{\pgfqpoint{1.592667in}{0.919327in}}%
\pgfpathlineto{\pgfqpoint{1.593062in}{0.916780in}}%
\pgfpathlineto{\pgfqpoint{1.595035in}{0.898398in}}%
\pgfpathlineto{\pgfqpoint{1.595825in}{0.903942in}}%
\pgfpathlineto{\pgfqpoint{1.596220in}{0.909630in}}%
\pgfpathlineto{\pgfqpoint{1.597009in}{0.904160in}}%
\pgfpathlineto{\pgfqpoint{1.598193in}{0.900272in}}%
\pgfpathlineto{\pgfqpoint{1.598588in}{0.901340in}}%
\pgfpathlineto{\pgfqpoint{1.600167in}{0.905618in}}%
\pgfpathlineto{\pgfqpoint{1.603720in}{0.877735in}}%
\pgfpathlineto{\pgfqpoint{1.608062in}{0.853199in}}%
\pgfpathlineto{\pgfqpoint{1.608457in}{0.857519in}}%
\pgfpathlineto{\pgfqpoint{1.608852in}{0.853357in}}%
\pgfpathlineto{\pgfqpoint{1.610036in}{0.838151in}}%
\pgfpathlineto{\pgfqpoint{1.610431in}{0.840115in}}%
\pgfpathlineto{\pgfqpoint{1.611615in}{0.871253in}}%
\pgfpathlineto{\pgfqpoint{1.612799in}{0.861605in}}%
\pgfpathlineto{\pgfqpoint{1.613194in}{0.861611in}}%
\pgfpathlineto{\pgfqpoint{1.614378in}{0.877684in}}%
\pgfpathlineto{\pgfqpoint{1.615168in}{0.868354in}}%
\pgfpathlineto{\pgfqpoint{1.617141in}{0.839289in}}%
\pgfpathlineto{\pgfqpoint{1.617536in}{0.839679in}}%
\pgfpathlineto{\pgfqpoint{1.619115in}{0.847518in}}%
\pgfpathlineto{\pgfqpoint{1.623063in}{0.903223in}}%
\pgfpathlineto{\pgfqpoint{1.623852in}{0.895812in}}%
\pgfpathlineto{\pgfqpoint{1.627405in}{0.861154in}}%
\pgfpathlineto{\pgfqpoint{1.627799in}{0.861643in}}%
\pgfpathlineto{\pgfqpoint{1.629378in}{0.890297in}}%
\pgfpathlineto{\pgfqpoint{1.629773in}{0.886388in}}%
\pgfpathlineto{\pgfqpoint{1.630168in}{0.882034in}}%
\pgfpathlineto{\pgfqpoint{1.630957in}{0.887293in}}%
\pgfpathlineto{\pgfqpoint{1.633326in}{0.924041in}}%
\pgfpathlineto{\pgfqpoint{1.633721in}{0.917800in}}%
\pgfpathlineto{\pgfqpoint{1.636879in}{0.882660in}}%
\pgfpathlineto{\pgfqpoint{1.638063in}{0.870156in}}%
\pgfpathlineto{\pgfqpoint{1.640037in}{0.846138in}}%
\pgfpathlineto{\pgfqpoint{1.640431in}{0.848954in}}%
\pgfpathlineto{\pgfqpoint{1.643195in}{0.878923in}}%
\pgfpathlineto{\pgfqpoint{1.645958in}{0.906518in}}%
\pgfpathlineto{\pgfqpoint{1.649116in}{0.864507in}}%
\pgfpathlineto{\pgfqpoint{1.649511in}{0.873099in}}%
\pgfpathlineto{\pgfqpoint{1.651484in}{0.904953in}}%
\pgfpathlineto{\pgfqpoint{1.652274in}{0.903986in}}%
\pgfpathlineto{\pgfqpoint{1.653458in}{0.896821in}}%
\pgfpathlineto{\pgfqpoint{1.657011in}{0.862040in}}%
\pgfpathlineto{\pgfqpoint{1.657800in}{0.869716in}}%
\pgfpathlineto{\pgfqpoint{1.660169in}{0.875720in}}%
\pgfpathlineto{\pgfqpoint{1.658590in}{0.869044in}}%
\pgfpathlineto{\pgfqpoint{1.660958in}{0.874787in}}%
\pgfpathlineto{\pgfqpoint{1.662932in}{0.881124in}}%
\pgfpathlineto{\pgfqpoint{1.664511in}{0.867798in}}%
\pgfpathlineto{\pgfqpoint{1.665695in}{0.871301in}}%
\pgfpathlineto{\pgfqpoint{1.666485in}{0.868651in}}%
\pgfpathlineto{\pgfqpoint{1.667669in}{0.884865in}}%
\pgfpathlineto{\pgfqpoint{1.669248in}{0.901380in}}%
\pgfpathlineto{\pgfqpoint{1.669643in}{0.897824in}}%
\pgfpathlineto{\pgfqpoint{1.673590in}{0.845147in}}%
\pgfpathlineto{\pgfqpoint{1.673985in}{0.847393in}}%
\pgfpathlineto{\pgfqpoint{1.674775in}{0.852385in}}%
\pgfpathlineto{\pgfqpoint{1.675169in}{0.845143in}}%
\pgfpathlineto{\pgfqpoint{1.676354in}{0.835381in}}%
\pgfpathlineto{\pgfqpoint{1.676748in}{0.841054in}}%
\pgfpathlineto{\pgfqpoint{1.677143in}{0.846906in}}%
\pgfpathlineto{\pgfqpoint{1.678327in}{0.841577in}}%
\pgfpathlineto{\pgfqpoint{1.678722in}{0.844347in}}%
\pgfpathlineto{\pgfqpoint{1.679512in}{0.840844in}}%
\pgfpathlineto{\pgfqpoint{1.681485in}{0.838600in}}%
\pgfpathlineto{\pgfqpoint{1.683064in}{0.857467in}}%
\pgfpathlineto{\pgfqpoint{1.684249in}{0.854720in}}%
\pgfpathlineto{\pgfqpoint{1.685433in}{0.864610in}}%
\pgfpathlineto{\pgfqpoint{1.686222in}{0.861949in}}%
\pgfpathlineto{\pgfqpoint{1.687012in}{0.856488in}}%
\pgfpathlineto{\pgfqpoint{1.687407in}{0.860216in}}%
\pgfpathlineto{\pgfqpoint{1.689775in}{0.880986in}}%
\pgfpathlineto{\pgfqpoint{1.690565in}{0.880508in}}%
\pgfpathlineto{\pgfqpoint{1.691749in}{0.876219in}}%
\pgfpathlineto{\pgfqpoint{1.696881in}{0.851722in}}%
\pgfpathlineto{\pgfqpoint{1.697275in}{0.853005in}}%
\pgfpathlineto{\pgfqpoint{1.700828in}{0.863096in}}%
\pgfpathlineto{\pgfqpoint{1.701618in}{0.862059in}}%
\pgfpathlineto{\pgfqpoint{1.702407in}{0.865246in}}%
\pgfpathlineto{\pgfqpoint{1.702802in}{0.861968in}}%
\pgfpathlineto{\pgfqpoint{1.703197in}{0.858636in}}%
\pgfpathlineto{\pgfqpoint{1.703986in}{0.863677in}}%
\pgfpathlineto{\pgfqpoint{1.708723in}{0.839211in}}%
\pgfpathlineto{\pgfqpoint{1.709513in}{0.843082in}}%
\pgfpathlineto{\pgfqpoint{1.711486in}{0.858913in}}%
\pgfpathlineto{\pgfqpoint{1.712276in}{0.864094in}}%
\pgfpathlineto{\pgfqpoint{1.713065in}{0.863137in}}%
\pgfpathlineto{\pgfqpoint{1.713855in}{0.863510in}}%
\pgfpathlineto{\pgfqpoint{1.716223in}{0.889973in}}%
\pgfpathlineto{\pgfqpoint{1.716618in}{0.887393in}}%
\pgfpathlineto{\pgfqpoint{1.717802in}{0.874619in}}%
\pgfpathlineto{\pgfqpoint{1.718592in}{0.877348in}}%
\pgfpathlineto{\pgfqpoint{1.720960in}{0.885036in}}%
\pgfpathlineto{\pgfqpoint{1.721355in}{0.883402in}}%
\pgfpathlineto{\pgfqpoint{1.724513in}{0.874026in}}%
\pgfpathlineto{\pgfqpoint{1.724908in}{0.874264in}}%
\pgfpathlineto{\pgfqpoint{1.726487in}{0.879260in}}%
\pgfpathlineto{\pgfqpoint{1.726881in}{0.875627in}}%
\pgfpathlineto{\pgfqpoint{1.727671in}{0.879219in}}%
\pgfpathlineto{\pgfqpoint{1.728855in}{0.882761in}}%
\pgfpathlineto{\pgfqpoint{1.729250in}{0.879493in}}%
\pgfpathlineto{\pgfqpoint{1.736750in}{0.807909in}}%
\pgfpathlineto{\pgfqpoint{1.737145in}{0.814094in}}%
\pgfpathlineto{\pgfqpoint{1.737934in}{0.806585in}}%
\pgfpathlineto{\pgfqpoint{1.739513in}{0.795606in}}%
\pgfpathlineto{\pgfqpoint{1.739908in}{0.797214in}}%
\pgfpathlineto{\pgfqpoint{1.740698in}{0.808630in}}%
\pgfpathlineto{\pgfqpoint{1.741092in}{0.817219in}}%
\pgfpathlineto{\pgfqpoint{1.742277in}{0.810232in}}%
\pgfpathlineto{\pgfqpoint{1.743066in}{0.808976in}}%
\pgfpathlineto{\pgfqpoint{1.743461in}{0.809702in}}%
\pgfpathlineto{\pgfqpoint{1.745829in}{0.820536in}}%
\pgfpathlineto{\pgfqpoint{1.746224in}{0.817041in}}%
\pgfpathlineto{\pgfqpoint{1.747408in}{0.820041in}}%
\pgfpathlineto{\pgfqpoint{1.751751in}{0.844799in}}%
\pgfpathlineto{\pgfqpoint{1.755303in}{0.891880in}}%
\pgfpathlineto{\pgfqpoint{1.756093in}{0.885316in}}%
\pgfpathlineto{\pgfqpoint{1.761619in}{0.846870in}}%
\pgfpathlineto{\pgfqpoint{1.762014in}{0.844974in}}%
\pgfpathlineto{\pgfqpoint{1.762409in}{0.847779in}}%
\pgfpathlineto{\pgfqpoint{1.763593in}{0.860822in}}%
\pgfpathlineto{\pgfqpoint{1.764383in}{0.855318in}}%
\pgfpathlineto{\pgfqpoint{1.767541in}{0.832899in}}%
\pgfpathlineto{\pgfqpoint{1.767935in}{0.833895in}}%
\pgfpathlineto{\pgfqpoint{1.768330in}{0.831428in}}%
\pgfpathlineto{\pgfqpoint{1.768725in}{0.829101in}}%
\pgfpathlineto{\pgfqpoint{1.769120in}{0.829198in}}%
\pgfpathlineto{\pgfqpoint{1.769514in}{0.833803in}}%
\pgfpathlineto{\pgfqpoint{1.770304in}{0.830131in}}%
\pgfpathlineto{\pgfqpoint{1.771093in}{0.826047in}}%
\pgfpathlineto{\pgfqpoint{1.771488in}{0.827820in}}%
\pgfpathlineto{\pgfqpoint{1.774251in}{0.856304in}}%
\pgfpathlineto{\pgfqpoint{1.774646in}{0.850966in}}%
\pgfpathlineto{\pgfqpoint{1.775041in}{0.848677in}}%
\pgfpathlineto{\pgfqpoint{1.775436in}{0.852569in}}%
\pgfpathlineto{\pgfqpoint{1.776225in}{0.858409in}}%
\pgfpathlineto{\pgfqpoint{1.778199in}{0.843304in}}%
\pgfpathlineto{\pgfqpoint{1.779383in}{0.860593in}}%
\pgfpathlineto{\pgfqpoint{1.780173in}{0.855990in}}%
\pgfpathlineto{\pgfqpoint{1.783725in}{0.828110in}}%
\pgfpathlineto{\pgfqpoint{1.784120in}{0.832044in}}%
\pgfpathlineto{\pgfqpoint{1.786883in}{0.859145in}}%
\pgfpathlineto{\pgfqpoint{1.787673in}{0.854143in}}%
\pgfpathlineto{\pgfqpoint{1.788068in}{0.854298in}}%
\pgfpathlineto{\pgfqpoint{1.788857in}{0.859279in}}%
\pgfpathlineto{\pgfqpoint{1.789252in}{0.857617in}}%
\pgfpathlineto{\pgfqpoint{1.789647in}{0.853871in}}%
\pgfpathlineto{\pgfqpoint{1.790831in}{0.857267in}}%
\pgfpathlineto{\pgfqpoint{1.791226in}{0.858279in}}%
\pgfpathlineto{\pgfqpoint{1.791620in}{0.856096in}}%
\pgfpathlineto{\pgfqpoint{1.795568in}{0.808923in}}%
\pgfpathlineto{\pgfqpoint{1.796357in}{0.815720in}}%
\pgfpathlineto{\pgfqpoint{1.797541in}{0.827663in}}%
\pgfpathlineto{\pgfqpoint{1.798331in}{0.825745in}}%
\pgfpathlineto{\pgfqpoint{1.799515in}{0.798582in}}%
\pgfpathlineto{\pgfqpoint{1.800305in}{0.777894in}}%
\pgfpathlineto{\pgfqpoint{1.801094in}{0.792247in}}%
\pgfpathlineto{\pgfqpoint{1.804252in}{0.804652in}}%
\pgfpathlineto{\pgfqpoint{1.805436in}{0.808828in}}%
\pgfpathlineto{\pgfqpoint{1.807015in}{0.831353in}}%
\pgfpathlineto{\pgfqpoint{1.807410in}{0.824482in}}%
\pgfpathlineto{\pgfqpoint{1.807805in}{0.820969in}}%
\pgfpathlineto{\pgfqpoint{1.808594in}{0.823626in}}%
\pgfpathlineto{\pgfqpoint{1.808989in}{0.824285in}}%
\pgfpathlineto{\pgfqpoint{1.809384in}{0.822696in}}%
\pgfpathlineto{\pgfqpoint{1.810568in}{0.811408in}}%
\pgfpathlineto{\pgfqpoint{1.811358in}{0.813410in}}%
\pgfpathlineto{\pgfqpoint{1.812147in}{0.817271in}}%
\pgfpathlineto{\pgfqpoint{1.815700in}{0.860943in}}%
\pgfpathlineto{\pgfqpoint{1.816884in}{0.862206in}}%
\pgfpathlineto{\pgfqpoint{1.818068in}{0.865809in}}%
\pgfpathlineto{\pgfqpoint{1.818463in}{0.864777in}}%
\pgfpathlineto{\pgfqpoint{1.819647in}{0.860537in}}%
\pgfpathlineto{\pgfqpoint{1.820832in}{0.869610in}}%
\pgfpathlineto{\pgfqpoint{1.821621in}{0.868368in}}%
\pgfpathlineto{\pgfqpoint{1.822805in}{0.863072in}}%
\pgfpathlineto{\pgfqpoint{1.823200in}{0.861333in}}%
\pgfpathlineto{\pgfqpoint{1.823990in}{0.861706in}}%
\pgfpathlineto{\pgfqpoint{1.824384in}{0.863451in}}%
\pgfpathlineto{\pgfqpoint{1.824779in}{0.860531in}}%
\pgfpathlineto{\pgfqpoint{1.827542in}{0.835831in}}%
\pgfpathlineto{\pgfqpoint{1.828332in}{0.842700in}}%
\pgfpathlineto{\pgfqpoint{1.828727in}{0.842891in}}%
\pgfpathlineto{\pgfqpoint{1.833069in}{0.823015in}}%
\pgfpathlineto{\pgfqpoint{1.833858in}{0.831740in}}%
\pgfpathlineto{\pgfqpoint{1.835043in}{0.845542in}}%
\pgfpathlineto{\pgfqpoint{1.835832in}{0.840557in}}%
\pgfpathlineto{\pgfqpoint{1.838201in}{0.802617in}}%
\pgfpathlineto{\pgfqpoint{1.842938in}{0.863291in}}%
\pgfpathlineto{\pgfqpoint{1.843727in}{0.859558in}}%
\pgfpathlineto{\pgfqpoint{1.848464in}{0.810447in}}%
\pgfpathlineto{\pgfqpoint{1.850438in}{0.791501in}}%
\pgfpathlineto{\pgfqpoint{1.851227in}{0.797644in}}%
\pgfpathlineto{\pgfqpoint{1.851622in}{0.797999in}}%
\pgfpathlineto{\pgfqpoint{1.853991in}{0.772109in}}%
\pgfpathlineto{\pgfqpoint{1.854385in}{0.773262in}}%
\pgfpathlineto{\pgfqpoint{1.855570in}{0.761002in}}%
\pgfpathlineto{\pgfqpoint{1.856359in}{0.750877in}}%
\pgfpathlineto{\pgfqpoint{1.857149in}{0.759468in}}%
\pgfpathlineto{\pgfqpoint{1.859122in}{0.782170in}}%
\pgfpathlineto{\pgfqpoint{1.859517in}{0.779901in}}%
\pgfpathlineto{\pgfqpoint{1.859912in}{0.782693in}}%
\pgfpathlineto{\pgfqpoint{1.860307in}{0.778731in}}%
\pgfpathlineto{\pgfqpoint{1.862280in}{0.764485in}}%
\pgfpathlineto{\pgfqpoint{1.867807in}{0.836154in}}%
\pgfpathlineto{\pgfqpoint{1.870570in}{0.828362in}}%
\pgfpathlineto{\pgfqpoint{1.872544in}{0.835382in}}%
\pgfpathlineto{\pgfqpoint{1.872939in}{0.836458in}}%
\pgfpathlineto{\pgfqpoint{1.873333in}{0.833683in}}%
\pgfpathlineto{\pgfqpoint{1.874518in}{0.826846in}}%
\pgfpathlineto{\pgfqpoint{1.875307in}{0.829581in}}%
\pgfpathlineto{\pgfqpoint{1.876097in}{0.828318in}}%
\pgfpathlineto{\pgfqpoint{1.876886in}{0.824599in}}%
\pgfpathlineto{\pgfqpoint{1.877675in}{0.827717in}}%
\pgfpathlineto{\pgfqpoint{1.878465in}{0.832022in}}%
\pgfpathlineto{\pgfqpoint{1.878860in}{0.827832in}}%
\pgfpathlineto{\pgfqpoint{1.880439in}{0.809332in}}%
\pgfpathlineto{\pgfqpoint{1.881228in}{0.815878in}}%
\pgfpathlineto{\pgfqpoint{1.883202in}{0.830928in}}%
\pgfpathlineto{\pgfqpoint{1.883597in}{0.827603in}}%
\pgfpathlineto{\pgfqpoint{1.885176in}{0.798571in}}%
\pgfpathlineto{\pgfqpoint{1.886360in}{0.802422in}}%
\pgfpathlineto{\pgfqpoint{1.888728in}{0.807319in}}%
\pgfpathlineto{\pgfqpoint{1.889123in}{0.812296in}}%
\pgfpathlineto{\pgfqpoint{1.889913in}{0.806578in}}%
\pgfpathlineto{\pgfqpoint{1.893071in}{0.761792in}}%
\pgfpathlineto{\pgfqpoint{1.893465in}{0.760591in}}%
\pgfpathlineto{\pgfqpoint{1.899387in}{0.806115in}}%
\pgfpathlineto{\pgfqpoint{1.900571in}{0.810265in}}%
\pgfpathlineto{\pgfqpoint{1.900966in}{0.808139in}}%
\pgfpathlineto{\pgfqpoint{1.902939in}{0.790326in}}%
\pgfpathlineto{\pgfqpoint{1.903729in}{0.799131in}}%
\pgfpathlineto{\pgfqpoint{1.904518in}{0.807856in}}%
\pgfpathlineto{\pgfqpoint{1.905703in}{0.805867in}}%
\pgfpathlineto{\pgfqpoint{1.906887in}{0.809539in}}%
\pgfpathlineto{\pgfqpoint{1.910045in}{0.838771in}}%
\pgfpathlineto{\pgfqpoint{1.910440in}{0.836493in}}%
\pgfpathlineto{\pgfqpoint{1.914782in}{0.794188in}}%
\pgfpathlineto{\pgfqpoint{1.916756in}{0.784808in}}%
\pgfpathlineto{\pgfqpoint{1.917150in}{0.786030in}}%
\pgfpathlineto{\pgfqpoint{1.917545in}{0.789321in}}%
\pgfpathlineto{\pgfqpoint{1.918335in}{0.783046in}}%
\pgfpathlineto{\pgfqpoint{1.919124in}{0.778685in}}%
\pgfpathlineto{\pgfqpoint{1.919519in}{0.783652in}}%
\pgfpathlineto{\pgfqpoint{1.921887in}{0.804898in}}%
\pgfpathlineto{\pgfqpoint{1.922677in}{0.806754in}}%
\pgfpathlineto{\pgfqpoint{1.923072in}{0.806135in}}%
\pgfpathlineto{\pgfqpoint{1.926624in}{0.784080in}}%
\pgfpathlineto{\pgfqpoint{1.927019in}{0.788237in}}%
\pgfpathlineto{\pgfqpoint{1.932546in}{0.851412in}}%
\pgfpathlineto{\pgfqpoint{1.932940in}{0.850406in}}%
\pgfpathlineto{\pgfqpoint{1.933730in}{0.852459in}}%
\pgfpathlineto{\pgfqpoint{1.934125in}{0.853348in}}%
\pgfpathlineto{\pgfqpoint{1.934519in}{0.851551in}}%
\pgfpathlineto{\pgfqpoint{1.935704in}{0.844444in}}%
\pgfpathlineto{\pgfqpoint{1.936493in}{0.848079in}}%
\pgfpathlineto{\pgfqpoint{1.938862in}{0.850578in}}%
\pgfpathlineto{\pgfqpoint{1.939651in}{0.849969in}}%
\pgfpathlineto{\pgfqpoint{1.940046in}{0.852009in}}%
\pgfpathlineto{\pgfqpoint{1.940441in}{0.848749in}}%
\pgfpathlineto{\pgfqpoint{1.945572in}{0.798195in}}%
\pgfpathlineto{\pgfqpoint{1.946757in}{0.797389in}}%
\pgfpathlineto{\pgfqpoint{1.951099in}{0.778390in}}%
\pgfpathlineto{\pgfqpoint{1.951888in}{0.782110in}}%
\pgfpathlineto{\pgfqpoint{1.954257in}{0.795095in}}%
\pgfpathlineto{\pgfqpoint{1.955046in}{0.803657in}}%
\pgfpathlineto{\pgfqpoint{1.955836in}{0.796733in}}%
\pgfpathlineto{\pgfqpoint{1.958599in}{0.785613in}}%
\pgfpathlineto{\pgfqpoint{1.960178in}{0.775309in}}%
\pgfpathlineto{\pgfqpoint{1.960573in}{0.776759in}}%
\pgfpathlineto{\pgfqpoint{1.962546in}{0.795550in}}%
\pgfpathlineto{\pgfqpoint{1.962941in}{0.792279in}}%
\pgfpathlineto{\pgfqpoint{1.963336in}{0.787144in}}%
\pgfpathlineto{\pgfqpoint{1.964125in}{0.793588in}}%
\pgfpathlineto{\pgfqpoint{1.964520in}{0.792766in}}%
\pgfpathlineto{\pgfqpoint{1.964915in}{0.794540in}}%
\pgfpathlineto{\pgfqpoint{1.969257in}{0.819477in}}%
\pgfpathlineto{\pgfqpoint{1.970441in}{0.817296in}}%
\pgfpathlineto{\pgfqpoint{1.971231in}{0.817154in}}%
\pgfpathlineto{\pgfqpoint{1.971626in}{0.817984in}}%
\pgfpathlineto{\pgfqpoint{1.972810in}{0.820214in}}%
\pgfpathlineto{\pgfqpoint{1.973994in}{0.808554in}}%
\pgfpathlineto{\pgfqpoint{1.974784in}{0.810845in}}%
\pgfpathlineto{\pgfqpoint{1.979126in}{0.824126in}}%
\pgfpathlineto{\pgfqpoint{1.980705in}{0.811160in}}%
\pgfpathlineto{\pgfqpoint{1.981494in}{0.816413in}}%
\pgfpathlineto{\pgfqpoint{1.982284in}{0.821841in}}%
\pgfpathlineto{\pgfqpoint{1.982679in}{0.818778in}}%
\pgfpathlineto{\pgfqpoint{1.984652in}{0.807713in}}%
\pgfpathlineto{\pgfqpoint{1.985047in}{0.809629in}}%
\pgfpathlineto{\pgfqpoint{1.987021in}{0.834116in}}%
\pgfpathlineto{\pgfqpoint{1.988205in}{0.828632in}}%
\pgfpathlineto{\pgfqpoint{1.988995in}{0.823612in}}%
\pgfpathlineto{\pgfqpoint{1.992153in}{0.858547in}}%
\pgfpathlineto{\pgfqpoint{1.992942in}{0.858836in}}%
\pgfpathlineto{\pgfqpoint{1.993337in}{0.858050in}}%
\pgfpathlineto{\pgfqpoint{1.994521in}{0.853077in}}%
\pgfpathlineto{\pgfqpoint{2.000837in}{0.808888in}}%
\pgfpathlineto{\pgfqpoint{2.002811in}{0.793439in}}%
\pgfpathlineto{\pgfqpoint{2.003206in}{0.793473in}}%
\pgfpathlineto{\pgfqpoint{2.004390in}{0.792089in}}%
\pgfpathlineto{\pgfqpoint{2.005179in}{0.790210in}}%
\pgfpathlineto{\pgfqpoint{2.005574in}{0.795320in}}%
\pgfpathlineto{\pgfqpoint{2.006364in}{0.791102in}}%
\pgfpathlineto{\pgfqpoint{2.008337in}{0.785646in}}%
\pgfpathlineto{\pgfqpoint{2.008732in}{0.787500in}}%
\pgfpathlineto{\pgfqpoint{2.011101in}{0.794978in}}%
\pgfpathlineto{\pgfqpoint{2.011890in}{0.797406in}}%
\pgfpathlineto{\pgfqpoint{2.012285in}{0.796301in}}%
\pgfpathlineto{\pgfqpoint{2.017811in}{0.748806in}}%
\pgfpathlineto{\pgfqpoint{2.018996in}{0.747998in}}%
\pgfpathlineto{\pgfqpoint{2.020180in}{0.740003in}}%
\pgfpathlineto{\pgfqpoint{2.021364in}{0.743471in}}%
\pgfpathlineto{\pgfqpoint{2.022943in}{0.746161in}}%
\pgfpathlineto{\pgfqpoint{2.024127in}{0.757008in}}%
\pgfpathlineto{\pgfqpoint{2.024522in}{0.751937in}}%
\pgfpathlineto{\pgfqpoint{2.024917in}{0.747301in}}%
\pgfpathlineto{\pgfqpoint{2.025706in}{0.755101in}}%
\pgfpathlineto{\pgfqpoint{2.030443in}{0.825754in}}%
\pgfpathlineto{\pgfqpoint{2.031233in}{0.823367in}}%
\pgfpathlineto{\pgfqpoint{2.031628in}{0.822751in}}%
\pgfpathlineto{\pgfqpoint{2.035970in}{0.781930in}}%
\pgfpathlineto{\pgfqpoint{2.036759in}{0.785838in}}%
\pgfpathlineto{\pgfqpoint{2.039523in}{0.793060in}}%
\pgfpathlineto{\pgfqpoint{2.043075in}{0.834003in}}%
\pgfpathlineto{\pgfqpoint{2.047023in}{0.807593in}}%
\pgfpathlineto{\pgfqpoint{2.047812in}{0.810393in}}%
\pgfpathlineto{\pgfqpoint{2.049391in}{0.818437in}}%
\pgfpathlineto{\pgfqpoint{2.049786in}{0.814176in}}%
\pgfpathlineto{\pgfqpoint{2.052944in}{0.787725in}}%
\pgfpathlineto{\pgfqpoint{2.053339in}{0.786896in}}%
\pgfpathlineto{\pgfqpoint{2.053733in}{0.788316in}}%
\pgfpathlineto{\pgfqpoint{2.054918in}{0.791834in}}%
\pgfpathlineto{\pgfqpoint{2.055312in}{0.790928in}}%
\pgfpathlineto{\pgfqpoint{2.056102in}{0.779909in}}%
\pgfpathlineto{\pgfqpoint{2.057286in}{0.781131in}}%
\pgfpathlineto{\pgfqpoint{2.058076in}{0.779366in}}%
\pgfpathlineto{\pgfqpoint{2.060839in}{0.768719in}}%
\pgfpathlineto{\pgfqpoint{2.062418in}{0.763642in}}%
\pgfpathlineto{\pgfqpoint{2.062813in}{0.761321in}}%
\pgfpathlineto{\pgfqpoint{2.063997in}{0.763391in}}%
\pgfpathlineto{\pgfqpoint{2.065181in}{0.772437in}}%
\pgfpathlineto{\pgfqpoint{2.065576in}{0.771411in}}%
\pgfpathlineto{\pgfqpoint{2.068339in}{0.749757in}}%
\pgfpathlineto{\pgfqpoint{2.073866in}{0.777498in}}%
\pgfpathlineto{\pgfqpoint{2.076234in}{0.785296in}}%
\pgfpathlineto{\pgfqpoint{2.076629in}{0.784484in}}%
\pgfpathlineto{\pgfqpoint{2.077024in}{0.784670in}}%
\pgfpathlineto{\pgfqpoint{2.080576in}{0.766510in}}%
\pgfpathlineto{\pgfqpoint{2.080971in}{0.769044in}}%
\pgfpathlineto{\pgfqpoint{2.083340in}{0.778388in}}%
\pgfpathlineto{\pgfqpoint{2.083734in}{0.777329in}}%
\pgfpathlineto{\pgfqpoint{2.089656in}{0.739375in}}%
\pgfpathlineto{\pgfqpoint{2.090445in}{0.742956in}}%
\pgfpathlineto{\pgfqpoint{2.093603in}{0.762156in}}%
\pgfpathlineto{\pgfqpoint{2.094393in}{0.769089in}}%
\pgfpathlineto{\pgfqpoint{2.096761in}{0.784913in}}%
\pgfpathlineto{\pgfqpoint{2.101103in}{0.800799in}}%
\pgfpathlineto{\pgfqpoint{2.101893in}{0.802156in}}%
\pgfpathlineto{\pgfqpoint{2.102288in}{0.800804in}}%
\pgfpathlineto{\pgfqpoint{2.103867in}{0.789547in}}%
\pgfpathlineto{\pgfqpoint{2.104261in}{0.792992in}}%
\pgfpathlineto{\pgfqpoint{2.105051in}{0.798960in}}%
\pgfpathlineto{\pgfqpoint{2.105840in}{0.796067in}}%
\pgfpathlineto{\pgfqpoint{2.110577in}{0.763933in}}%
\pgfpathlineto{\pgfqpoint{2.112551in}{0.745483in}}%
\pgfpathlineto{\pgfqpoint{2.112946in}{0.745512in}}%
\pgfpathlineto{\pgfqpoint{2.113735in}{0.750151in}}%
\pgfpathlineto{\pgfqpoint{2.114920in}{0.775186in}}%
\pgfpathlineto{\pgfqpoint{2.116104in}{0.771491in}}%
\pgfpathlineto{\pgfqpoint{2.117288in}{0.770912in}}%
\pgfpathlineto{\pgfqpoint{2.118472in}{0.778308in}}%
\pgfpathlineto{\pgfqpoint{2.122420in}{0.824755in}}%
\pgfpathlineto{\pgfqpoint{2.123209in}{0.821581in}}%
\pgfpathlineto{\pgfqpoint{2.128736in}{0.791594in}}%
\pgfpathlineto{\pgfqpoint{2.129130in}{0.792869in}}%
\pgfpathlineto{\pgfqpoint{2.131894in}{0.799749in}}%
\pgfpathlineto{\pgfqpoint{2.135052in}{0.805382in}}%
\pgfpathlineto{\pgfqpoint{2.137420in}{0.814331in}}%
\pgfpathlineto{\pgfqpoint{2.137815in}{0.813527in}}%
\pgfpathlineto{\pgfqpoint{2.140183in}{0.789413in}}%
\pgfpathlineto{\pgfqpoint{2.141762in}{0.762784in}}%
\pgfpathlineto{\pgfqpoint{2.142947in}{0.766110in}}%
\pgfpathlineto{\pgfqpoint{2.143736in}{0.761480in}}%
\pgfpathlineto{\pgfqpoint{2.147289in}{0.741417in}}%
\pgfpathlineto{\pgfqpoint{2.148078in}{0.743040in}}%
\pgfpathlineto{\pgfqpoint{2.149263in}{0.748299in}}%
\pgfpathlineto{\pgfqpoint{2.150447in}{0.745803in}}%
\pgfpathlineto{\pgfqpoint{2.150842in}{0.744178in}}%
\pgfpathlineto{\pgfqpoint{2.151236in}{0.747668in}}%
\pgfpathlineto{\pgfqpoint{2.151631in}{0.747569in}}%
\pgfpathlineto{\pgfqpoint{2.152421in}{0.740852in}}%
\pgfpathlineto{\pgfqpoint{2.153210in}{0.743080in}}%
\pgfpathlineto{\pgfqpoint{2.153605in}{0.743590in}}%
\pgfpathlineto{\pgfqpoint{2.154000in}{0.742615in}}%
\pgfpathlineto{\pgfqpoint{2.158342in}{0.732584in}}%
\pgfpathlineto{\pgfqpoint{2.158737in}{0.733268in}}%
\pgfpathlineto{\pgfqpoint{2.159921in}{0.732194in}}%
\pgfpathlineto{\pgfqpoint{2.164658in}{0.772075in}}%
\pgfpathlineto{\pgfqpoint{2.166632in}{0.767752in}}%
\pgfpathlineto{\pgfqpoint{2.169000in}{0.758712in}}%
\pgfpathlineto{\pgfqpoint{2.169395in}{0.761137in}}%
\pgfpathlineto{\pgfqpoint{2.170184in}{0.757072in}}%
\pgfpathlineto{\pgfqpoint{2.174527in}{0.777408in}}%
\pgfpathlineto{\pgfqpoint{2.177290in}{0.813043in}}%
\pgfpathlineto{\pgfqpoint{2.177685in}{0.811071in}}%
\pgfpathlineto{\pgfqpoint{2.178869in}{0.808087in}}%
\pgfpathlineto{\pgfqpoint{2.180053in}{0.815361in}}%
\pgfpathlineto{\pgfqpoint{2.180843in}{0.812040in}}%
\pgfpathlineto{\pgfqpoint{2.182027in}{0.803899in}}%
\pgfpathlineto{\pgfqpoint{2.185580in}{0.775406in}}%
\pgfpathlineto{\pgfqpoint{2.187948in}{0.768502in}}%
\pgfpathlineto{\pgfqpoint{2.188738in}{0.770152in}}%
\pgfpathlineto{\pgfqpoint{2.190317in}{0.777789in}}%
\pgfpathlineto{\pgfqpoint{2.191896in}{0.775558in}}%
\pgfpathlineto{\pgfqpoint{2.193475in}{0.768983in}}%
\pgfpathlineto{\pgfqpoint{2.194264in}{0.771026in}}%
\pgfpathlineto{\pgfqpoint{2.195843in}{0.777659in}}%
\pgfpathlineto{\pgfqpoint{2.197027in}{0.785668in}}%
\pgfpathlineto{\pgfqpoint{2.197817in}{0.783275in}}%
\pgfpathlineto{\pgfqpoint{2.203738in}{0.755986in}}%
\pgfpathlineto{\pgfqpoint{2.204133in}{0.756338in}}%
\pgfpathlineto{\pgfqpoint{2.204922in}{0.760091in}}%
\pgfpathlineto{\pgfqpoint{2.205712in}{0.758554in}}%
\pgfpathlineto{\pgfqpoint{2.207686in}{0.748677in}}%
\pgfpathlineto{\pgfqpoint{2.208475in}{0.752910in}}%
\pgfpathlineto{\pgfqpoint{2.210844in}{0.764862in}}%
\pgfpathlineto{\pgfqpoint{2.211238in}{0.760172in}}%
\pgfpathlineto{\pgfqpoint{2.211633in}{0.758236in}}%
\pgfpathlineto{\pgfqpoint{2.212028in}{0.760310in}}%
\pgfpathlineto{\pgfqpoint{2.217949in}{0.793937in}}%
\pgfpathlineto{\pgfqpoint{2.219528in}{0.773638in}}%
\pgfpathlineto{\pgfqpoint{2.219923in}{0.775610in}}%
\pgfpathlineto{\pgfqpoint{2.220712in}{0.785370in}}%
\pgfpathlineto{\pgfqpoint{2.221896in}{0.784097in}}%
\pgfpathlineto{\pgfqpoint{2.222686in}{0.782026in}}%
\pgfpathlineto{\pgfqpoint{2.223081in}{0.785403in}}%
\pgfpathlineto{\pgfqpoint{2.224660in}{0.794926in}}%
\pgfpathlineto{\pgfqpoint{2.225054in}{0.790157in}}%
\pgfpathlineto{\pgfqpoint{2.226239in}{0.783765in}}%
\pgfpathlineto{\pgfqpoint{2.226633in}{0.787183in}}%
\pgfpathlineto{\pgfqpoint{2.227423in}{0.787118in}}%
\pgfpathlineto{\pgfqpoint{2.227818in}{0.788410in}}%
\pgfpathlineto{\pgfqpoint{2.228212in}{0.786746in}}%
\pgfpathlineto{\pgfqpoint{2.229397in}{0.782650in}}%
\pgfpathlineto{\pgfqpoint{2.229791in}{0.783924in}}%
\pgfpathlineto{\pgfqpoint{2.230976in}{0.795059in}}%
\pgfpathlineto{\pgfqpoint{2.232160in}{0.790953in}}%
\pgfpathlineto{\pgfqpoint{2.234923in}{0.808260in}}%
\pgfpathlineto{\pgfqpoint{2.236107in}{0.804791in}}%
\pgfpathlineto{\pgfqpoint{2.247160in}{0.743219in}}%
\pgfpathlineto{\pgfqpoint{2.247950in}{0.744761in}}%
\pgfpathlineto{\pgfqpoint{2.248739in}{0.750238in}}%
\pgfpathlineto{\pgfqpoint{2.249529in}{0.749723in}}%
\pgfpathlineto{\pgfqpoint{2.251503in}{0.744714in}}%
\pgfpathlineto{\pgfqpoint{2.251897in}{0.748171in}}%
\pgfpathlineto{\pgfqpoint{2.254266in}{0.763901in}}%
\pgfpathlineto{\pgfqpoint{2.255055in}{0.761801in}}%
\pgfpathlineto{\pgfqpoint{2.258213in}{0.747392in}}%
\pgfpathlineto{\pgfqpoint{2.258608in}{0.749693in}}%
\pgfpathlineto{\pgfqpoint{2.262556in}{0.772632in}}%
\pgfpathlineto{\pgfqpoint{2.263345in}{0.769957in}}%
\pgfpathlineto{\pgfqpoint{2.264135in}{0.767090in}}%
\pgfpathlineto{\pgfqpoint{2.264924in}{0.768970in}}%
\pgfpathlineto{\pgfqpoint{2.265714in}{0.769607in}}%
\pgfpathlineto{\pgfqpoint{2.267293in}{0.773581in}}%
\pgfpathlineto{\pgfqpoint{2.267687in}{0.772442in}}%
\pgfpathlineto{\pgfqpoint{2.271240in}{0.750549in}}%
\pgfpathlineto{\pgfqpoint{2.272030in}{0.746490in}}%
\pgfpathlineto{\pgfqpoint{2.272819in}{0.748308in}}%
\pgfpathlineto{\pgfqpoint{2.274793in}{0.757411in}}%
\pgfpathlineto{\pgfqpoint{2.275582in}{0.752774in}}%
\pgfpathlineto{\pgfqpoint{2.282293in}{0.719326in}}%
\pgfpathlineto{\pgfqpoint{2.283083in}{0.720861in}}%
\pgfpathlineto{\pgfqpoint{2.284267in}{0.726379in}}%
\pgfpathlineto{\pgfqpoint{2.287425in}{0.759277in}}%
\pgfpathlineto{\pgfqpoint{2.288609in}{0.770273in}}%
\pgfpathlineto{\pgfqpoint{2.289399in}{0.769689in}}%
\pgfpathlineto{\pgfqpoint{2.289793in}{0.769751in}}%
\pgfpathlineto{\pgfqpoint{2.290583in}{0.775634in}}%
\pgfpathlineto{\pgfqpoint{2.291372in}{0.773145in}}%
\pgfpathlineto{\pgfqpoint{2.294136in}{0.767557in}}%
\pgfpathlineto{\pgfqpoint{2.295714in}{0.768819in}}%
\pgfpathlineto{\pgfqpoint{2.297293in}{0.765037in}}%
\pgfpathlineto{\pgfqpoint{2.300057in}{0.743591in}}%
\pgfpathlineto{\pgfqpoint{2.303609in}{0.716519in}}%
\pgfpathlineto{\pgfqpoint{2.304004in}{0.720290in}}%
\pgfpathlineto{\pgfqpoint{2.309925in}{0.776828in}}%
\pgfpathlineto{\pgfqpoint{2.310320in}{0.774597in}}%
\pgfpathlineto{\pgfqpoint{2.311504in}{0.765873in}}%
\pgfpathlineto{\pgfqpoint{2.312689in}{0.768872in}}%
\pgfpathlineto{\pgfqpoint{2.314268in}{0.772032in}}%
\pgfpathlineto{\pgfqpoint{2.317031in}{0.759457in}}%
\pgfpathlineto{\pgfqpoint{2.317426in}{0.758949in}}%
\pgfpathlineto{\pgfqpoint{2.319794in}{0.738180in}}%
\pgfpathlineto{\pgfqpoint{2.322952in}{0.730240in}}%
\pgfpathlineto{\pgfqpoint{2.323742in}{0.732327in}}%
\pgfpathlineto{\pgfqpoint{2.324531in}{0.747010in}}%
\pgfpathlineto{\pgfqpoint{2.325321in}{0.742689in}}%
\pgfpathlineto{\pgfqpoint{2.327689in}{0.729578in}}%
\pgfpathlineto{\pgfqpoint{2.328084in}{0.729876in}}%
\pgfpathlineto{\pgfqpoint{2.332031in}{0.742024in}}%
\pgfpathlineto{\pgfqpoint{2.332426in}{0.738914in}}%
\pgfpathlineto{\pgfqpoint{2.333610in}{0.734231in}}%
\pgfpathlineto{\pgfqpoint{2.334005in}{0.734772in}}%
\pgfpathlineto{\pgfqpoint{2.334795in}{0.736529in}}%
\pgfpathlineto{\pgfqpoint{2.335189in}{0.735383in}}%
\pgfpathlineto{\pgfqpoint{2.336374in}{0.731415in}}%
\pgfpathlineto{\pgfqpoint{2.336768in}{0.733826in}}%
\pgfpathlineto{\pgfqpoint{2.339926in}{0.737266in}}%
\pgfpathlineto{\pgfqpoint{2.341505in}{0.737270in}}%
\pgfpathlineto{\pgfqpoint{2.343479in}{0.741329in}}%
\pgfpathlineto{\pgfqpoint{2.343874in}{0.741545in}}%
\pgfpathlineto{\pgfqpoint{2.345848in}{0.760414in}}%
\pgfpathlineto{\pgfqpoint{2.349006in}{0.775357in}}%
\pgfpathlineto{\pgfqpoint{2.351769in}{0.785836in}}%
\pgfpathlineto{\pgfqpoint{2.352164in}{0.784534in}}%
\pgfpathlineto{\pgfqpoint{2.352953in}{0.782407in}}%
\pgfpathlineto{\pgfqpoint{2.353348in}{0.784276in}}%
\pgfpathlineto{\pgfqpoint{2.355716in}{0.794357in}}%
\pgfpathlineto{\pgfqpoint{2.356111in}{0.791289in}}%
\pgfpathlineto{\pgfqpoint{2.357690in}{0.771327in}}%
\pgfpathlineto{\pgfqpoint{2.359664in}{0.775634in}}%
\pgfpathlineto{\pgfqpoint{2.360453in}{0.776535in}}%
\pgfpathlineto{\pgfqpoint{2.361243in}{0.775622in}}%
\pgfpathlineto{\pgfqpoint{2.363611in}{0.769849in}}%
\pgfpathlineto{\pgfqpoint{2.364006in}{0.770023in}}%
\pgfpathlineto{\pgfqpoint{2.364401in}{0.773641in}}%
\pgfpathlineto{\pgfqpoint{2.365190in}{0.769854in}}%
\pgfpathlineto{\pgfqpoint{2.365585in}{0.770856in}}%
\pgfpathlineto{\pgfqpoint{2.365980in}{0.770771in}}%
\pgfpathlineto{\pgfqpoint{2.366375in}{0.766934in}}%
\pgfpathlineto{\pgfqpoint{2.367164in}{0.768970in}}%
\pgfpathlineto{\pgfqpoint{2.368348in}{0.780219in}}%
\pgfpathlineto{\pgfqpoint{2.369927in}{0.779889in}}%
\pgfpathlineto{\pgfqpoint{2.370717in}{0.782366in}}%
\pgfpathlineto{\pgfqpoint{2.371112in}{0.778310in}}%
\pgfpathlineto{\pgfqpoint{2.371901in}{0.781737in}}%
\pgfpathlineto{\pgfqpoint{2.375454in}{0.774150in}}%
\pgfpathlineto{\pgfqpoint{2.375849in}{0.779565in}}%
\pgfpathlineto{\pgfqpoint{2.377033in}{0.775652in}}%
\pgfpathlineto{\pgfqpoint{2.382164in}{0.756964in}}%
\pgfpathlineto{\pgfqpoint{2.384533in}{0.752839in}}%
\pgfpathlineto{\pgfqpoint{2.384928in}{0.754416in}}%
\pgfpathlineto{\pgfqpoint{2.386901in}{0.764946in}}%
\pgfpathlineto{\pgfqpoint{2.390849in}{0.720926in}}%
\pgfpathlineto{\pgfqpoint{2.391638in}{0.724342in}}%
\pgfpathlineto{\pgfqpoint{2.392033in}{0.723200in}}%
\pgfpathlineto{\pgfqpoint{2.392428in}{0.725868in}}%
\pgfpathlineto{\pgfqpoint{2.400323in}{0.775806in}}%
\pgfpathlineto{\pgfqpoint{2.400718in}{0.774964in}}%
\pgfpathlineto{\pgfqpoint{2.401112in}{0.773201in}}%
\pgfpathlineto{\pgfqpoint{2.401507in}{0.774512in}}%
\pgfpathlineto{\pgfqpoint{2.403481in}{0.781337in}}%
\pgfpathlineto{\pgfqpoint{2.405455in}{0.777471in}}%
\pgfpathlineto{\pgfqpoint{2.406639in}{0.775905in}}%
\pgfpathlineto{\pgfqpoint{2.407034in}{0.776273in}}%
\pgfpathlineto{\pgfqpoint{2.407823in}{0.778142in}}%
\pgfpathlineto{\pgfqpoint{2.410981in}{0.786656in}}%
\pgfpathlineto{\pgfqpoint{2.412560in}{0.780628in}}%
\pgfpathlineto{\pgfqpoint{2.412955in}{0.781666in}}%
\pgfpathlineto{\pgfqpoint{2.415323in}{0.788091in}}%
\pgfpathlineto{\pgfqpoint{2.420060in}{0.734660in}}%
\pgfpathlineto{\pgfqpoint{2.420455in}{0.735671in}}%
\pgfpathlineto{\pgfqpoint{2.422429in}{0.748488in}}%
\pgfpathlineto{\pgfqpoint{2.422824in}{0.747975in}}%
\pgfpathlineto{\pgfqpoint{2.424797in}{0.727058in}}%
\pgfpathlineto{\pgfqpoint{2.425192in}{0.730529in}}%
\pgfpathlineto{\pgfqpoint{2.425982in}{0.724966in}}%
\pgfpathlineto{\pgfqpoint{2.426376in}{0.725000in}}%
\pgfpathlineto{\pgfqpoint{2.427955in}{0.737824in}}%
\pgfpathlineto{\pgfqpoint{2.429140in}{0.732367in}}%
\pgfpathlineto{\pgfqpoint{2.429534in}{0.731341in}}%
\pgfpathlineto{\pgfqpoint{2.430324in}{0.732570in}}%
\pgfpathlineto{\pgfqpoint{2.431903in}{0.732960in}}%
\pgfpathlineto{\pgfqpoint{2.432692in}{0.729287in}}%
\pgfpathlineto{\pgfqpoint{2.433087in}{0.731235in}}%
\pgfpathlineto{\pgfqpoint{2.433877in}{0.736858in}}%
\pgfpathlineto{\pgfqpoint{2.434666in}{0.734374in}}%
\pgfpathlineto{\pgfqpoint{2.437429in}{0.724259in}}%
\pgfpathlineto{\pgfqpoint{2.440193in}{0.703484in}}%
\pgfpathlineto{\pgfqpoint{2.444930in}{0.678114in}}%
\pgfpathlineto{\pgfqpoint{2.445719in}{0.678860in}}%
\pgfpathlineto{\pgfqpoint{2.449667in}{0.692747in}}%
\pgfpathlineto{\pgfqpoint{2.450456in}{0.699372in}}%
\pgfpathlineto{\pgfqpoint{2.451640in}{0.697515in}}%
\pgfpathlineto{\pgfqpoint{2.452035in}{0.697170in}}%
\pgfpathlineto{\pgfqpoint{2.453219in}{0.689642in}}%
\pgfpathlineto{\pgfqpoint{2.453614in}{0.692385in}}%
\pgfpathlineto{\pgfqpoint{2.459535in}{0.744628in}}%
\pgfpathlineto{\pgfqpoint{2.460720in}{0.740605in}}%
\pgfpathlineto{\pgfqpoint{2.461509in}{0.742887in}}%
\pgfpathlineto{\pgfqpoint{2.463483in}{0.747553in}}%
\pgfpathlineto{\pgfqpoint{2.464272in}{0.746087in}}%
\pgfpathlineto{\pgfqpoint{2.465456in}{0.745355in}}%
\pgfpathlineto{\pgfqpoint{2.469404in}{0.727495in}}%
\pgfpathlineto{\pgfqpoint{2.469799in}{0.728819in}}%
\pgfpathlineto{\pgfqpoint{2.470983in}{0.730817in}}%
\pgfpathlineto{\pgfqpoint{2.471378in}{0.730623in}}%
\pgfpathlineto{\pgfqpoint{2.473746in}{0.717131in}}%
\pgfpathlineto{\pgfqpoint{2.474536in}{0.722091in}}%
\pgfpathlineto{\pgfqpoint{2.474930in}{0.726313in}}%
\pgfpathlineto{\pgfqpoint{2.475325in}{0.718898in}}%
\pgfpathlineto{\pgfqpoint{2.477694in}{0.692765in}}%
\pgfpathlineto{\pgfqpoint{2.478483in}{0.694506in}}%
\pgfpathlineto{\pgfqpoint{2.479667in}{0.700620in}}%
\pgfpathlineto{\pgfqpoint{2.482036in}{0.678902in}}%
\pgfpathlineto{\pgfqpoint{2.482825in}{0.680890in}}%
\pgfpathlineto{\pgfqpoint{2.483615in}{0.678600in}}%
\pgfpathlineto{\pgfqpoint{2.485589in}{0.669907in}}%
\pgfpathlineto{\pgfqpoint{2.485983in}{0.670041in}}%
\pgfpathlineto{\pgfqpoint{2.490720in}{0.710249in}}%
\pgfpathlineto{\pgfqpoint{2.491510in}{0.702075in}}%
\pgfpathlineto{\pgfqpoint{2.492694in}{0.691457in}}%
\pgfpathlineto{\pgfqpoint{2.493484in}{0.692507in}}%
\pgfpathlineto{\pgfqpoint{2.495457in}{0.699609in}}%
\pgfpathlineto{\pgfqpoint{2.499800in}{0.733140in}}%
\pgfpathlineto{\pgfqpoint{2.501379in}{0.736769in}}%
\pgfpathlineto{\pgfqpoint{2.501773in}{0.735815in}}%
\pgfpathlineto{\pgfqpoint{2.507695in}{0.708112in}}%
\pgfpathlineto{\pgfqpoint{2.509668in}{0.715769in}}%
\pgfpathlineto{\pgfqpoint{2.510063in}{0.714908in}}%
\pgfpathlineto{\pgfqpoint{2.510853in}{0.712590in}}%
\pgfpathlineto{\pgfqpoint{2.511642in}{0.714073in}}%
\pgfpathlineto{\pgfqpoint{2.513616in}{0.719145in}}%
\pgfpathlineto{\pgfqpoint{2.514011in}{0.718686in}}%
\pgfpathlineto{\pgfqpoint{2.514800in}{0.720433in}}%
\pgfpathlineto{\pgfqpoint{2.515590in}{0.722151in}}%
\pgfpathlineto{\pgfqpoint{2.516774in}{0.726668in}}%
\pgfpathlineto{\pgfqpoint{2.517169in}{0.724216in}}%
\pgfpathlineto{\pgfqpoint{2.518353in}{0.719369in}}%
\pgfpathlineto{\pgfqpoint{2.519142in}{0.719900in}}%
\pgfpathlineto{\pgfqpoint{2.519932in}{0.722084in}}%
\pgfpathlineto{\pgfqpoint{2.520327in}{0.721148in}}%
\pgfpathlineto{\pgfqpoint{2.521116in}{0.718210in}}%
\pgfpathlineto{\pgfqpoint{2.521511in}{0.718683in}}%
\pgfpathlineto{\pgfqpoint{2.525458in}{0.737165in}}%
\pgfpathlineto{\pgfqpoint{2.526248in}{0.735991in}}%
\pgfpathlineto{\pgfqpoint{2.529011in}{0.758760in}}%
\pgfpathlineto{\pgfqpoint{2.529801in}{0.756833in}}%
\pgfpathlineto{\pgfqpoint{2.530590in}{0.759320in}}%
\pgfpathlineto{\pgfqpoint{2.531380in}{0.761086in}}%
\pgfpathlineto{\pgfqpoint{2.532959in}{0.750343in}}%
\pgfpathlineto{\pgfqpoint{2.533748in}{0.751530in}}%
\pgfpathlineto{\pgfqpoint{2.535722in}{0.752013in}}%
\pgfpathlineto{\pgfqpoint{2.536511in}{0.752455in}}%
\pgfpathlineto{\pgfqpoint{2.537301in}{0.759932in}}%
\pgfpathlineto{\pgfqpoint{2.538090in}{0.757241in}}%
\pgfpathlineto{\pgfqpoint{2.542038in}{0.737680in}}%
\pgfpathlineto{\pgfqpoint{2.544406in}{0.732842in}}%
\pgfpathlineto{\pgfqpoint{2.545196in}{0.728844in}}%
\pgfpathlineto{\pgfqpoint{2.547564in}{0.717210in}}%
\pgfpathlineto{\pgfqpoint{2.549933in}{0.714789in}}%
\pgfpathlineto{\pgfqpoint{2.551512in}{0.716632in}}%
\pgfpathlineto{\pgfqpoint{2.551906in}{0.715409in}}%
\pgfpathlineto{\pgfqpoint{2.552696in}{0.716982in}}%
\pgfpathlineto{\pgfqpoint{2.553880in}{0.721526in}}%
\pgfpathlineto{\pgfqpoint{2.554670in}{0.720533in}}%
\pgfpathlineto{\pgfqpoint{2.555064in}{0.719499in}}%
\pgfpathlineto{\pgfqpoint{2.555854in}{0.721369in}}%
\pgfpathlineto{\pgfqpoint{2.558222in}{0.731421in}}%
\pgfpathlineto{\pgfqpoint{2.559407in}{0.737382in}}%
\pgfpathlineto{\pgfqpoint{2.560196in}{0.737241in}}%
\pgfpathlineto{\pgfqpoint{2.561380in}{0.739967in}}%
\pgfpathlineto{\pgfqpoint{2.561775in}{0.740784in}}%
\pgfpathlineto{\pgfqpoint{2.562959in}{0.739755in}}%
\pgfpathlineto{\pgfqpoint{2.564144in}{0.739415in}}%
\pgfpathlineto{\pgfqpoint{2.566117in}{0.729505in}}%
\pgfpathlineto{\pgfqpoint{2.568091in}{0.720150in}}%
\pgfpathlineto{\pgfqpoint{2.568486in}{0.720213in}}%
\pgfpathlineto{\pgfqpoint{2.569275in}{0.725486in}}%
\pgfpathlineto{\pgfqpoint{2.570460in}{0.724749in}}%
\pgfpathlineto{\pgfqpoint{2.571644in}{0.725491in}}%
\pgfpathlineto{\pgfqpoint{2.572039in}{0.725085in}}%
\pgfpathlineto{\pgfqpoint{2.573618in}{0.719568in}}%
\pgfpathlineto{\pgfqpoint{2.574407in}{0.720260in}}%
\pgfpathlineto{\pgfqpoint{2.577960in}{0.725067in}}%
\pgfpathlineto{\pgfqpoint{2.578749in}{0.724537in}}%
\pgfpathlineto{\pgfqpoint{2.579539in}{0.716628in}}%
\pgfpathlineto{\pgfqpoint{2.580328in}{0.718603in}}%
\pgfpathlineto{\pgfqpoint{2.582302in}{0.712764in}}%
\pgfpathlineto{\pgfqpoint{2.584671in}{0.727098in}}%
\pgfpathlineto{\pgfqpoint{2.587039in}{0.733946in}}%
\pgfpathlineto{\pgfqpoint{2.589013in}{0.723543in}}%
\pgfpathlineto{\pgfqpoint{2.590197in}{0.708247in}}%
\pgfpathlineto{\pgfqpoint{2.590987in}{0.715289in}}%
\pgfpathlineto{\pgfqpoint{2.595329in}{0.744576in}}%
\pgfpathlineto{\pgfqpoint{2.597697in}{0.730526in}}%
\pgfpathlineto{\pgfqpoint{2.598882in}{0.729810in}}%
\pgfpathlineto{\pgfqpoint{2.602829in}{0.750957in}}%
\pgfpathlineto{\pgfqpoint{2.603224in}{0.750853in}}%
\pgfpathlineto{\pgfqpoint{2.604408in}{0.748599in}}%
\pgfpathlineto{\pgfqpoint{2.604803in}{0.752581in}}%
\pgfpathlineto{\pgfqpoint{2.605987in}{0.748895in}}%
\pgfpathlineto{\pgfqpoint{2.608750in}{0.726876in}}%
\pgfpathlineto{\pgfqpoint{2.609145in}{0.730146in}}%
\pgfpathlineto{\pgfqpoint{2.609540in}{0.734599in}}%
\pgfpathlineto{\pgfqpoint{2.610329in}{0.727232in}}%
\pgfpathlineto{\pgfqpoint{2.613093in}{0.732757in}}%
\pgfpathlineto{\pgfqpoint{2.613487in}{0.732499in}}%
\pgfpathlineto{\pgfqpoint{2.615461in}{0.726023in}}%
\pgfpathlineto{\pgfqpoint{2.615856in}{0.727527in}}%
\pgfpathlineto{\pgfqpoint{2.617040in}{0.732679in}}%
\pgfpathlineto{\pgfqpoint{2.617830in}{0.731898in}}%
\pgfpathlineto{\pgfqpoint{2.618224in}{0.732099in}}%
\pgfpathlineto{\pgfqpoint{2.618619in}{0.731180in}}%
\pgfpathlineto{\pgfqpoint{2.620593in}{0.725207in}}%
\pgfpathlineto{\pgfqpoint{2.624540in}{0.739413in}}%
\pgfpathlineto{\pgfqpoint{2.625330in}{0.739144in}}%
\pgfpathlineto{\pgfqpoint{2.626119in}{0.739858in}}%
\pgfpathlineto{\pgfqpoint{2.626514in}{0.739312in}}%
\pgfpathlineto{\pgfqpoint{2.628093in}{0.733519in}}%
\pgfpathlineto{\pgfqpoint{2.628882in}{0.736835in}}%
\pgfpathlineto{\pgfqpoint{2.629277in}{0.738839in}}%
\pgfpathlineto{\pgfqpoint{2.629672in}{0.736578in}}%
\pgfpathlineto{\pgfqpoint{2.630067in}{0.726972in}}%
\pgfpathlineto{\pgfqpoint{2.631251in}{0.734009in}}%
\pgfpathlineto{\pgfqpoint{2.635593in}{0.749675in}}%
\pgfpathlineto{\pgfqpoint{2.637567in}{0.747116in}}%
\pgfpathlineto{\pgfqpoint{2.639935in}{0.739875in}}%
\pgfpathlineto{\pgfqpoint{2.642304in}{0.722360in}}%
\pgfpathlineto{\pgfqpoint{2.643883in}{0.723143in}}%
\pgfpathlineto{\pgfqpoint{2.644278in}{0.723414in}}%
\pgfpathlineto{\pgfqpoint{2.645067in}{0.722116in}}%
\pgfpathlineto{\pgfqpoint{2.647041in}{0.721763in}}%
\pgfpathlineto{\pgfqpoint{2.651383in}{0.706978in}}%
\pgfpathlineto{\pgfqpoint{2.653357in}{0.702916in}}%
\pgfpathlineto{\pgfqpoint{2.655331in}{0.688556in}}%
\pgfpathlineto{\pgfqpoint{2.656120in}{0.683268in}}%
\pgfpathlineto{\pgfqpoint{2.656910in}{0.687215in}}%
\pgfpathlineto{\pgfqpoint{2.657304in}{0.687506in}}%
\pgfpathlineto{\pgfqpoint{2.658489in}{0.683037in}}%
\pgfpathlineto{\pgfqpoint{2.658883in}{0.685740in}}%
\pgfpathlineto{\pgfqpoint{2.662041in}{0.696137in}}%
\pgfpathlineto{\pgfqpoint{2.662436in}{0.696003in}}%
\pgfpathlineto{\pgfqpoint{2.664015in}{0.688048in}}%
\pgfpathlineto{\pgfqpoint{2.664805in}{0.689811in}}%
\pgfpathlineto{\pgfqpoint{2.666778in}{0.689255in}}%
\pgfpathlineto{\pgfqpoint{2.667963in}{0.688253in}}%
\pgfpathlineto{\pgfqpoint{2.668752in}{0.691923in}}%
\pgfpathlineto{\pgfqpoint{2.669542in}{0.690859in}}%
\pgfpathlineto{\pgfqpoint{2.669936in}{0.691869in}}%
\pgfpathlineto{\pgfqpoint{2.673094in}{0.699298in}}%
\pgfpathlineto{\pgfqpoint{2.673489in}{0.698953in}}%
\pgfpathlineto{\pgfqpoint{2.673884in}{0.700604in}}%
\pgfpathlineto{\pgfqpoint{2.675858in}{0.704700in}}%
\pgfpathlineto{\pgfqpoint{2.678226in}{0.706448in}}%
\pgfpathlineto{\pgfqpoint{2.679016in}{0.706701in}}%
\pgfpathlineto{\pgfqpoint{2.682568in}{0.680654in}}%
\pgfpathlineto{\pgfqpoint{2.683753in}{0.683028in}}%
\pgfpathlineto{\pgfqpoint{2.685332in}{0.684946in}}%
\pgfpathlineto{\pgfqpoint{2.686911in}{0.697250in}}%
\pgfpathlineto{\pgfqpoint{2.687305in}{0.695311in}}%
\pgfpathlineto{\pgfqpoint{2.687700in}{0.693459in}}%
\pgfpathlineto{\pgfqpoint{2.688884in}{0.693943in}}%
\pgfpathlineto{\pgfqpoint{2.689279in}{0.693494in}}%
\pgfpathlineto{\pgfqpoint{2.691648in}{0.701123in}}%
\pgfpathlineto{\pgfqpoint{2.694806in}{0.711416in}}%
\pgfpathlineto{\pgfqpoint{2.695990in}{0.713232in}}%
\pgfpathlineto{\pgfqpoint{2.696385in}{0.712691in}}%
\pgfpathlineto{\pgfqpoint{2.697569in}{0.711522in}}%
\pgfpathlineto{\pgfqpoint{2.697964in}{0.712427in}}%
\pgfpathlineto{\pgfqpoint{2.701516in}{0.734321in}}%
\pgfpathlineto{\pgfqpoint{2.702701in}{0.730676in}}%
\pgfpathlineto{\pgfqpoint{2.704280in}{0.726325in}}%
\pgfpathlineto{\pgfqpoint{2.705069in}{0.728009in}}%
\pgfpathlineto{\pgfqpoint{2.707043in}{0.730374in}}%
\pgfpathlineto{\pgfqpoint{2.707438in}{0.729644in}}%
\pgfpathlineto{\pgfqpoint{2.708227in}{0.725307in}}%
\pgfpathlineto{\pgfqpoint{2.709411in}{0.713188in}}%
\pgfpathlineto{\pgfqpoint{2.710201in}{0.716175in}}%
\pgfpathlineto{\pgfqpoint{2.710596in}{0.715628in}}%
\pgfpathlineto{\pgfqpoint{2.712569in}{0.688669in}}%
\pgfpathlineto{\pgfqpoint{2.712964in}{0.690220in}}%
\pgfpathlineto{\pgfqpoint{2.715332in}{0.694524in}}%
\pgfpathlineto{\pgfqpoint{2.715727in}{0.694048in}}%
\pgfpathlineto{\pgfqpoint{2.717306in}{0.692541in}}%
\pgfpathlineto{\pgfqpoint{2.717701in}{0.693067in}}%
\pgfpathlineto{\pgfqpoint{2.719280in}{0.695676in}}%
\pgfpathlineto{\pgfqpoint{2.721254in}{0.701794in}}%
\pgfpathlineto{\pgfqpoint{2.722438in}{0.693035in}}%
\pgfpathlineto{\pgfqpoint{2.722833in}{0.695752in}}%
\pgfpathlineto{\pgfqpoint{2.724412in}{0.703966in}}%
\pgfpathlineto{\pgfqpoint{2.724806in}{0.703482in}}%
\pgfpathlineto{\pgfqpoint{2.727175in}{0.699738in}}%
\pgfpathlineto{\pgfqpoint{2.728754in}{0.693241in}}%
\pgfpathlineto{\pgfqpoint{2.729543in}{0.697956in}}%
\pgfpathlineto{\pgfqpoint{2.736649in}{0.755552in}}%
\pgfpathlineto{\pgfqpoint{2.737833in}{0.752615in}}%
\pgfpathlineto{\pgfqpoint{2.748491in}{0.689182in}}%
\pgfpathlineto{\pgfqpoint{2.749281in}{0.691163in}}%
\pgfpathlineto{\pgfqpoint{2.750860in}{0.696600in}}%
\pgfpathlineto{\pgfqpoint{2.753623in}{0.709636in}}%
\pgfpathlineto{\pgfqpoint{2.755202in}{0.706541in}}%
\pgfpathlineto{\pgfqpoint{2.755597in}{0.708677in}}%
\pgfpathlineto{\pgfqpoint{2.759544in}{0.727795in}}%
\pgfpathlineto{\pgfqpoint{2.760729in}{0.733350in}}%
\pgfpathlineto{\pgfqpoint{2.762702in}{0.738396in}}%
\pgfpathlineto{\pgfqpoint{2.763097in}{0.737736in}}%
\pgfpathlineto{\pgfqpoint{2.764281in}{0.736063in}}%
\pgfpathlineto{\pgfqpoint{2.764676in}{0.736586in}}%
\pgfpathlineto{\pgfqpoint{2.766650in}{0.744771in}}%
\pgfpathlineto{\pgfqpoint{2.767045in}{0.741598in}}%
\pgfpathlineto{\pgfqpoint{2.768229in}{0.732084in}}%
\pgfpathlineto{\pgfqpoint{2.769413in}{0.735510in}}%
\pgfpathlineto{\pgfqpoint{2.772176in}{0.743640in}}%
\pgfpathlineto{\pgfqpoint{2.777703in}{0.713549in}}%
\pgfpathlineto{\pgfqpoint{2.778492in}{0.716163in}}%
\pgfpathlineto{\pgfqpoint{2.780466in}{0.726265in}}%
\pgfpathlineto{\pgfqpoint{2.780861in}{0.725954in}}%
\pgfpathlineto{\pgfqpoint{2.782835in}{0.716129in}}%
\pgfpathlineto{\pgfqpoint{2.784019in}{0.710780in}}%
\pgfpathlineto{\pgfqpoint{2.784414in}{0.711879in}}%
\pgfpathlineto{\pgfqpoint{2.789151in}{0.736336in}}%
\pgfpathlineto{\pgfqpoint{2.791124in}{0.746336in}}%
\pgfpathlineto{\pgfqpoint{2.791914in}{0.743803in}}%
\pgfpathlineto{\pgfqpoint{2.792309in}{0.742463in}}%
\pgfpathlineto{\pgfqpoint{2.792703in}{0.745715in}}%
\pgfpathlineto{\pgfqpoint{2.793493in}{0.750158in}}%
\pgfpathlineto{\pgfqpoint{2.793888in}{0.746529in}}%
\pgfpathlineto{\pgfqpoint{2.798624in}{0.718300in}}%
\pgfpathlineto{\pgfqpoint{2.799019in}{0.719718in}}%
\pgfpathlineto{\pgfqpoint{2.802572in}{0.727633in}}%
\pgfpathlineto{\pgfqpoint{2.804940in}{0.706417in}}%
\pgfpathlineto{\pgfqpoint{2.805730in}{0.711599in}}%
\pgfpathlineto{\pgfqpoint{2.806914in}{0.718104in}}%
\pgfpathlineto{\pgfqpoint{2.807309in}{0.716063in}}%
\pgfpathlineto{\pgfqpoint{2.808493in}{0.703358in}}%
\pgfpathlineto{\pgfqpoint{2.808888in}{0.706025in}}%
\pgfpathlineto{\pgfqpoint{2.809283in}{0.712095in}}%
\pgfpathlineto{\pgfqpoint{2.810467in}{0.710020in}}%
\pgfpathlineto{\pgfqpoint{2.810862in}{0.708706in}}%
\pgfpathlineto{\pgfqpoint{2.812046in}{0.709791in}}%
\pgfpathlineto{\pgfqpoint{2.814809in}{0.715140in}}%
\pgfpathlineto{\pgfqpoint{2.818362in}{0.738079in}}%
\pgfpathlineto{\pgfqpoint{2.819151in}{0.739992in}}%
\pgfpathlineto{\pgfqpoint{2.820336in}{0.742017in}}%
\pgfpathlineto{\pgfqpoint{2.821915in}{0.734476in}}%
\pgfpathlineto{\pgfqpoint{2.822704in}{0.735697in}}%
\pgfpathlineto{\pgfqpoint{2.825862in}{0.741848in}}%
\pgfpathlineto{\pgfqpoint{2.828625in}{0.743218in}}%
\pgfpathlineto{\pgfqpoint{2.831783in}{0.724940in}}%
\pgfpathlineto{\pgfqpoint{2.835336in}{0.701361in}}%
\pgfpathlineto{\pgfqpoint{2.835731in}{0.703518in}}%
\pgfpathlineto{\pgfqpoint{2.844415in}{0.744628in}}%
\pgfpathlineto{\pgfqpoint{2.844810in}{0.743408in}}%
\pgfpathlineto{\pgfqpoint{2.845205in}{0.742822in}}%
\pgfpathlineto{\pgfqpoint{2.845600in}{0.744077in}}%
\pgfpathlineto{\pgfqpoint{2.848363in}{0.756546in}}%
\pgfpathlineto{\pgfqpoint{2.850337in}{0.772877in}}%
\pgfpathlineto{\pgfqpoint{2.851126in}{0.769993in}}%
\pgfpathlineto{\pgfqpoint{2.859811in}{0.717999in}}%
\pgfpathlineto{\pgfqpoint{2.860205in}{0.718062in}}%
\pgfpathlineto{\pgfqpoint{2.860600in}{0.718657in}}%
\pgfpathlineto{\pgfqpoint{2.862179in}{0.739874in}}%
\pgfpathlineto{\pgfqpoint{2.864153in}{0.737631in}}%
\pgfpathlineto{\pgfqpoint{2.864942in}{0.739264in}}%
\pgfpathlineto{\pgfqpoint{2.866521in}{0.743481in}}%
\pgfpathlineto{\pgfqpoint{2.866916in}{0.741666in}}%
\pgfpathlineto{\pgfqpoint{2.869285in}{0.736135in}}%
\pgfpathlineto{\pgfqpoint{2.871258in}{0.733353in}}%
\pgfpathlineto{\pgfqpoint{2.871653in}{0.734450in}}%
\pgfpathlineto{\pgfqpoint{2.874811in}{0.758275in}}%
\pgfpathlineto{\pgfqpoint{2.875601in}{0.757056in}}%
\pgfpathlineto{\pgfqpoint{2.879153in}{0.751620in}}%
\pgfpathlineto{\pgfqpoint{2.882311in}{0.732209in}}%
\pgfpathlineto{\pgfqpoint{2.883495in}{0.721350in}}%
\pgfpathlineto{\pgfqpoint{2.884285in}{0.711329in}}%
\pgfpathlineto{\pgfqpoint{2.885074in}{0.712188in}}%
\pgfpathlineto{\pgfqpoint{2.886259in}{0.714927in}}%
\pgfpathlineto{\pgfqpoint{2.886653in}{0.714463in}}%
\pgfpathlineto{\pgfqpoint{2.887048in}{0.712702in}}%
\pgfpathlineto{\pgfqpoint{2.888627in}{0.723861in}}%
\pgfpathlineto{\pgfqpoint{2.889022in}{0.723269in}}%
\pgfpathlineto{\pgfqpoint{2.893759in}{0.704701in}}%
\pgfpathlineto{\pgfqpoint{2.894548in}{0.705515in}}%
\pgfpathlineto{\pgfqpoint{2.896522in}{0.709159in}}%
\pgfpathlineto{\pgfqpoint{2.898101in}{0.715334in}}%
\pgfpathlineto{\pgfqpoint{2.899285in}{0.716747in}}%
\pgfpathlineto{\pgfqpoint{2.902838in}{0.705260in}}%
\pgfpathlineto{\pgfqpoint{2.903233in}{0.705368in}}%
\pgfpathlineto{\pgfqpoint{2.906391in}{0.729036in}}%
\pgfpathlineto{\pgfqpoint{2.906786in}{0.732781in}}%
\pgfpathlineto{\pgfqpoint{2.907575in}{0.730204in}}%
\pgfpathlineto{\pgfqpoint{2.911523in}{0.712389in}}%
\pgfpathlineto{\pgfqpoint{2.911917in}{0.713898in}}%
\pgfpathlineto{\pgfqpoint{2.912707in}{0.720744in}}%
\pgfpathlineto{\pgfqpoint{2.913496in}{0.716983in}}%
\pgfpathlineto{\pgfqpoint{2.914681in}{0.716495in}}%
\pgfpathlineto{\pgfqpoint{2.915075in}{0.717416in}}%
\pgfpathlineto{\pgfqpoint{2.916260in}{0.718255in}}%
\pgfpathlineto{\pgfqpoint{2.918628in}{0.733567in}}%
\pgfpathlineto{\pgfqpoint{2.919418in}{0.733337in}}%
\pgfpathlineto{\pgfqpoint{2.922181in}{0.728829in}}%
\pgfpathlineto{\pgfqpoint{2.922576in}{0.727908in}}%
\pgfpathlineto{\pgfqpoint{2.923760in}{0.740287in}}%
\pgfpathlineto{\pgfqpoint{2.924549in}{0.737670in}}%
\pgfpathlineto{\pgfqpoint{2.927707in}{0.728874in}}%
\pgfpathlineto{\pgfqpoint{2.930076in}{0.728451in}}%
\pgfpathlineto{\pgfqpoint{2.930865in}{0.729596in}}%
\pgfpathlineto{\pgfqpoint{2.931260in}{0.728787in}}%
\pgfpathlineto{\pgfqpoint{2.933234in}{0.721200in}}%
\pgfpathlineto{\pgfqpoint{2.935602in}{0.711357in}}%
\pgfpathlineto{\pgfqpoint{2.936787in}{0.708758in}}%
\pgfpathlineto{\pgfqpoint{2.937181in}{0.709793in}}%
\pgfpathlineto{\pgfqpoint{2.937576in}{0.717695in}}%
\pgfpathlineto{\pgfqpoint{2.938760in}{0.713109in}}%
\pgfpathlineto{\pgfqpoint{2.941129in}{0.708249in}}%
\pgfpathlineto{\pgfqpoint{2.944287in}{0.713508in}}%
\pgfpathlineto{\pgfqpoint{2.945076in}{0.716379in}}%
\pgfpathlineto{\pgfqpoint{2.945866in}{0.713651in}}%
\pgfpathlineto{\pgfqpoint{2.948234in}{0.715801in}}%
\pgfpathlineto{\pgfqpoint{2.949024in}{0.714173in}}%
\pgfpathlineto{\pgfqpoint{2.949419in}{0.714672in}}%
\pgfpathlineto{\pgfqpoint{2.952577in}{0.731131in}}%
\pgfpathlineto{\pgfqpoint{2.954156in}{0.726626in}}%
\pgfpathlineto{\pgfqpoint{2.956129in}{0.729862in}}%
\pgfpathlineto{\pgfqpoint{2.956919in}{0.728790in}}%
\pgfpathlineto{\pgfqpoint{2.958893in}{0.723977in}}%
\pgfpathlineto{\pgfqpoint{2.959682in}{0.727423in}}%
\pgfpathlineto{\pgfqpoint{2.960077in}{0.721251in}}%
\pgfpathlineto{\pgfqpoint{2.961261in}{0.725485in}}%
\pgfpathlineto{\pgfqpoint{2.962051in}{0.729511in}}%
\pgfpathlineto{\pgfqpoint{2.962840in}{0.728067in}}%
\pgfpathlineto{\pgfqpoint{2.963235in}{0.728226in}}%
\pgfpathlineto{\pgfqpoint{2.965998in}{0.736979in}}%
\pgfpathlineto{\pgfqpoint{2.969551in}{0.746614in}}%
\pgfpathlineto{\pgfqpoint{2.969945in}{0.745949in}}%
\pgfpathlineto{\pgfqpoint{2.977840in}{0.711558in}}%
\pgfpathlineto{\pgfqpoint{2.980209in}{0.729847in}}%
\pgfpathlineto{\pgfqpoint{2.981788in}{0.735191in}}%
\pgfpathlineto{\pgfqpoint{2.982183in}{0.733579in}}%
\pgfpathlineto{\pgfqpoint{2.982972in}{0.725336in}}%
\pgfpathlineto{\pgfqpoint{2.983762in}{0.728202in}}%
\pgfpathlineto{\pgfqpoint{2.988893in}{0.754276in}}%
\pgfpathlineto{\pgfqpoint{2.989683in}{0.765382in}}%
\pgfpathlineto{\pgfqpoint{2.990867in}{0.763484in}}%
\pgfpathlineto{\pgfqpoint{2.992841in}{0.759138in}}%
\pgfpathlineto{\pgfqpoint{2.998367in}{0.745651in}}%
\pgfpathlineto{\pgfqpoint{2.999157in}{0.745372in}}%
\pgfpathlineto{\pgfqpoint{2.999946in}{0.744474in}}%
\pgfpathlineto{\pgfqpoint{3.000341in}{0.745796in}}%
\pgfpathlineto{\pgfqpoint{3.002710in}{0.748673in}}%
\pgfpathlineto{\pgfqpoint{3.006262in}{0.742156in}}%
\pgfpathlineto{\pgfqpoint{3.007052in}{0.741632in}}%
\pgfpathlineto{\pgfqpoint{3.007447in}{0.744446in}}%
\pgfpathlineto{\pgfqpoint{3.007841in}{0.739940in}}%
\pgfpathlineto{\pgfqpoint{3.009815in}{0.734147in}}%
\pgfpathlineto{\pgfqpoint{3.010605in}{0.735652in}}%
\pgfpathlineto{\pgfqpoint{3.011394in}{0.736670in}}%
\pgfpathlineto{\pgfqpoint{3.011789in}{0.736413in}}%
\pgfpathlineto{\pgfqpoint{3.012578in}{0.732610in}}%
\pgfpathlineto{\pgfqpoint{3.012973in}{0.738036in}}%
\pgfpathlineto{\pgfqpoint{3.013368in}{0.737979in}}%
\pgfpathlineto{\pgfqpoint{3.016526in}{0.749822in}}%
\pgfpathlineto{\pgfqpoint{3.016921in}{0.748869in}}%
\pgfpathlineto{\pgfqpoint{3.022447in}{0.728309in}}%
\pgfpathlineto{\pgfqpoint{3.025210in}{0.708647in}}%
\pgfpathlineto{\pgfqpoint{3.025605in}{0.708722in}}%
\pgfpathlineto{\pgfqpoint{3.028368in}{0.718643in}}%
\pgfpathlineto{\pgfqpoint{3.029553in}{0.717770in}}%
\pgfpathlineto{\pgfqpoint{3.029947in}{0.717313in}}%
\pgfpathlineto{\pgfqpoint{3.030342in}{0.718390in}}%
\pgfpathlineto{\pgfqpoint{3.032316in}{0.720475in}}%
\pgfpathlineto{\pgfqpoint{3.032711in}{0.719882in}}%
\pgfpathlineto{\pgfqpoint{3.035079in}{0.707226in}}%
\pgfpathlineto{\pgfqpoint{3.035869in}{0.704638in}}%
\pgfpathlineto{\pgfqpoint{3.036658in}{0.706263in}}%
\pgfpathlineto{\pgfqpoint{3.037448in}{0.706330in}}%
\pgfpathlineto{\pgfqpoint{3.041000in}{0.734893in}}%
\pgfpathlineto{\pgfqpoint{3.042185in}{0.734346in}}%
\pgfpathlineto{\pgfqpoint{3.045737in}{0.724566in}}%
\pgfpathlineto{\pgfqpoint{3.051658in}{0.704370in}}%
\pgfpathlineto{\pgfqpoint{3.052843in}{0.704773in}}%
\pgfpathlineto{\pgfqpoint{3.055211in}{0.713492in}}%
\pgfpathlineto{\pgfqpoint{3.055606in}{0.711528in}}%
\pgfpathlineto{\pgfqpoint{3.056395in}{0.708977in}}%
\pgfpathlineto{\pgfqpoint{3.057185in}{0.711273in}}%
\pgfpathlineto{\pgfqpoint{3.059553in}{0.722024in}}%
\pgfpathlineto{\pgfqpoint{3.059948in}{0.721257in}}%
\pgfpathlineto{\pgfqpoint{3.061132in}{0.715509in}}%
\pgfpathlineto{\pgfqpoint{3.061922in}{0.719822in}}%
\pgfpathlineto{\pgfqpoint{3.064685in}{0.727287in}}%
\pgfpathlineto{\pgfqpoint{3.067448in}{0.737350in}}%
\pgfpathlineto{\pgfqpoint{3.069817in}{0.746826in}}%
\pgfpathlineto{\pgfqpoint{3.070212in}{0.746239in}}%
\pgfpathlineto{\pgfqpoint{3.082054in}{0.705736in}}%
\pgfpathlineto{\pgfqpoint{3.084028in}{0.691536in}}%
\pgfpathlineto{\pgfqpoint{3.084423in}{0.691718in}}%
\pgfpathlineto{\pgfqpoint{3.084817in}{0.690838in}}%
\pgfpathlineto{\pgfqpoint{3.086396in}{0.688715in}}%
\pgfpathlineto{\pgfqpoint{3.086791in}{0.689361in}}%
\pgfpathlineto{\pgfqpoint{3.087581in}{0.703762in}}%
\pgfpathlineto{\pgfqpoint{3.089160in}{0.715708in}}%
\pgfpathlineto{\pgfqpoint{3.089554in}{0.712803in}}%
\pgfpathlineto{\pgfqpoint{3.091923in}{0.700726in}}%
\pgfpathlineto{\pgfqpoint{3.092712in}{0.701021in}}%
\pgfpathlineto{\pgfqpoint{3.093107in}{0.700666in}}%
\pgfpathlineto{\pgfqpoint{3.093502in}{0.702292in}}%
\pgfpathlineto{\pgfqpoint{3.095476in}{0.710389in}}%
\pgfpathlineto{\pgfqpoint{3.097449in}{0.720106in}}%
\pgfpathlineto{\pgfqpoint{3.098239in}{0.718977in}}%
\pgfpathlineto{\pgfqpoint{3.101792in}{0.716710in}}%
\pgfpathlineto{\pgfqpoint{3.102186in}{0.717674in}}%
\pgfpathlineto{\pgfqpoint{3.108502in}{0.741264in}}%
\pgfpathlineto{\pgfqpoint{3.110871in}{0.710653in}}%
\pgfpathlineto{\pgfqpoint{3.111266in}{0.711337in}}%
\pgfpathlineto{\pgfqpoint{3.113634in}{0.722970in}}%
\pgfpathlineto{\pgfqpoint{3.114424in}{0.716708in}}%
\pgfpathlineto{\pgfqpoint{3.115608in}{0.705564in}}%
\pgfpathlineto{\pgfqpoint{3.116397in}{0.712693in}}%
\pgfpathlineto{\pgfqpoint{3.118766in}{0.723874in}}%
\pgfpathlineto{\pgfqpoint{3.119950in}{0.722420in}}%
\pgfpathlineto{\pgfqpoint{3.121924in}{0.717861in}}%
\pgfpathlineto{\pgfqpoint{3.122319in}{0.718198in}}%
\pgfpathlineto{\pgfqpoint{3.123108in}{0.720359in}}%
\pgfpathlineto{\pgfqpoint{3.123898in}{0.717954in}}%
\pgfpathlineto{\pgfqpoint{3.132187in}{0.689023in}}%
\pgfpathlineto{\pgfqpoint{3.134556in}{0.669817in}}%
\pgfpathlineto{\pgfqpoint{3.135345in}{0.668317in}}%
\pgfpathlineto{\pgfqpoint{3.135740in}{0.669883in}}%
\pgfpathlineto{\pgfqpoint{3.138108in}{0.678358in}}%
\pgfpathlineto{\pgfqpoint{3.138503in}{0.676009in}}%
\pgfpathlineto{\pgfqpoint{3.139293in}{0.671427in}}%
\pgfpathlineto{\pgfqpoint{3.140082in}{0.673092in}}%
\pgfpathlineto{\pgfqpoint{3.140872in}{0.674137in}}%
\pgfpathlineto{\pgfqpoint{3.146398in}{0.696478in}}%
\pgfpathlineto{\pgfqpoint{3.147582in}{0.689748in}}%
\pgfpathlineto{\pgfqpoint{3.149161in}{0.681251in}}%
\pgfpathlineto{\pgfqpoint{3.149951in}{0.682709in}}%
\pgfpathlineto{\pgfqpoint{3.151925in}{0.705255in}}%
\pgfpathlineto{\pgfqpoint{3.154293in}{0.725612in}}%
\pgfpathlineto{\pgfqpoint{3.156267in}{0.734372in}}%
\pgfpathlineto{\pgfqpoint{3.156662in}{0.733127in}}%
\pgfpathlineto{\pgfqpoint{3.159425in}{0.708646in}}%
\pgfpathlineto{\pgfqpoint{3.162188in}{0.683157in}}%
\pgfpathlineto{\pgfqpoint{3.162583in}{0.681542in}}%
\pgfpathlineto{\pgfqpoint{3.162978in}{0.683299in}}%
\pgfpathlineto{\pgfqpoint{3.165741in}{0.713679in}}%
\pgfpathlineto{\pgfqpoint{3.166925in}{0.713483in}}%
\pgfpathlineto{\pgfqpoint{3.171267in}{0.694891in}}%
\pgfpathlineto{\pgfqpoint{3.171662in}{0.696851in}}%
\pgfpathlineto{\pgfqpoint{3.173636in}{0.698441in}}%
\pgfpathlineto{\pgfqpoint{3.174425in}{0.699962in}}%
\pgfpathlineto{\pgfqpoint{3.174820in}{0.700512in}}%
\pgfpathlineto{\pgfqpoint{3.175215in}{0.698598in}}%
\pgfpathlineto{\pgfqpoint{3.176794in}{0.694376in}}%
\pgfpathlineto{\pgfqpoint{3.177189in}{0.694438in}}%
\pgfpathlineto{\pgfqpoint{3.181531in}{0.701468in}}%
\pgfpathlineto{\pgfqpoint{3.181926in}{0.702862in}}%
\pgfpathlineto{\pgfqpoint{3.182320in}{0.702098in}}%
\pgfpathlineto{\pgfqpoint{3.184294in}{0.688703in}}%
\pgfpathlineto{\pgfqpoint{3.185873in}{0.685226in}}%
\pgfpathlineto{\pgfqpoint{3.186268in}{0.686281in}}%
\pgfpathlineto{\pgfqpoint{3.188242in}{0.689338in}}%
\pgfpathlineto{\pgfqpoint{3.191005in}{0.709607in}}%
\pgfpathlineto{\pgfqpoint{3.191400in}{0.710523in}}%
\pgfpathlineto{\pgfqpoint{3.191794in}{0.708449in}}%
\pgfpathlineto{\pgfqpoint{3.192189in}{0.707820in}}%
\pgfpathlineto{\pgfqpoint{3.192979in}{0.709086in}}%
\pgfpathlineto{\pgfqpoint{3.193373in}{0.710389in}}%
\pgfpathlineto{\pgfqpoint{3.194163in}{0.709044in}}%
\pgfpathlineto{\pgfqpoint{3.196926in}{0.685001in}}%
\pgfpathlineto{\pgfqpoint{3.197321in}{0.682398in}}%
\pgfpathlineto{\pgfqpoint{3.197321in}{0.682398in}}%
\pgfusepath{stroke}%
\end{pgfscope}%
\begin{pgfscope}%
\pgfpathrectangle{\pgfqpoint{0.608025in}{0.484444in}}{\pgfqpoint{2.712595in}{1.541287in}}%
\pgfusepath{clip}%
\pgfsetbuttcap%
\pgfsetmiterjoin%
\definecolor{currentfill}{rgb}{0.580392,0.403922,0.741176}%
\pgfsetfillcolor{currentfill}%
\pgfsetlinewidth{1.003750pt}%
\definecolor{currentstroke}{rgb}{0.580392,0.403922,0.741176}%
\pgfsetstrokecolor{currentstroke}%
\pgfsetdash{}{0pt}%
\pgfsys@defobject{currentmarker}{\pgfqpoint{-0.020833in}{-0.020833in}}{\pgfqpoint{0.020833in}{0.020833in}}{%
\pgfpathmoveto{\pgfqpoint{-0.020833in}{0.000000in}}%
\pgfpathlineto{\pgfqpoint{0.020833in}{-0.020833in}}%
\pgfpathlineto{\pgfqpoint{0.020833in}{0.020833in}}%
\pgfpathlineto{\pgfqpoint{-0.020833in}{0.000000in}}%
\pgfpathclose%
\pgfusepath{stroke,fill}%
}%
\begin{pgfscope}%
\pgfsys@transformshift{0.810275in}{1.793006in}%
\pgfsys@useobject{currentmarker}{}%
\end{pgfscope}%
\begin{pgfscope}%
\pgfsys@transformshift{1.007649in}{1.283119in}%
\pgfsys@useobject{currentmarker}{}%
\end{pgfscope}%
\begin{pgfscope}%
\pgfsys@transformshift{1.205024in}{1.081068in}%
\pgfsys@useobject{currentmarker}{}%
\end{pgfscope}%
\begin{pgfscope}%
\pgfsys@transformshift{1.402398in}{1.004619in}%
\pgfsys@useobject{currentmarker}{}%
\end{pgfscope}%
\begin{pgfscope}%
\pgfsys@transformshift{1.599772in}{0.904710in}%
\pgfsys@useobject{currentmarker}{}%
\end{pgfscope}%
\begin{pgfscope}%
\pgfsys@transformshift{1.797147in}{0.826861in}%
\pgfsys@useobject{currentmarker}{}%
\end{pgfscope}%
\begin{pgfscope}%
\pgfsys@transformshift{1.994521in}{0.853077in}%
\pgfsys@useobject{currentmarker}{}%
\end{pgfscope}%
\begin{pgfscope}%
\pgfsys@transformshift{2.191896in}{0.775558in}%
\pgfsys@useobject{currentmarker}{}%
\end{pgfscope}%
\begin{pgfscope}%
\pgfsys@transformshift{2.389270in}{0.736063in}%
\pgfsys@useobject{currentmarker}{}%
\end{pgfscope}%
\begin{pgfscope}%
\pgfsys@transformshift{2.586644in}{0.733066in}%
\pgfsys@useobject{currentmarker}{}%
\end{pgfscope}%
\begin{pgfscope}%
\pgfsys@transformshift{2.784019in}{0.710780in}%
\pgfsys@useobject{currentmarker}{}%
\end{pgfscope}%
\begin{pgfscope}%
\pgfsys@transformshift{2.981393in}{0.733348in}%
\pgfsys@useobject{currentmarker}{}%
\end{pgfscope}%
\begin{pgfscope}%
\pgfsys@transformshift{3.178768in}{0.697506in}%
\pgfsys@useobject{currentmarker}{}%
\end{pgfscope}%
\end{pgfscope}%
\begin{pgfscope}%
\pgfsetrectcap%
\pgfsetmiterjoin%
\pgfsetlinewidth{0.803000pt}%
\definecolor{currentstroke}{rgb}{0.000000,0.000000,0.000000}%
\pgfsetstrokecolor{currentstroke}%
\pgfsetdash{}{0pt}%
\pgfpathmoveto{\pgfqpoint{0.608025in}{0.484444in}}%
\pgfpathlineto{\pgfqpoint{0.608025in}{2.025731in}}%
\pgfusepath{stroke}%
\end{pgfscope}%
\begin{pgfscope}%
\pgfsetrectcap%
\pgfsetmiterjoin%
\pgfsetlinewidth{0.803000pt}%
\definecolor{currentstroke}{rgb}{0.000000,0.000000,0.000000}%
\pgfsetstrokecolor{currentstroke}%
\pgfsetdash{}{0pt}%
\pgfpathmoveto{\pgfqpoint{3.320621in}{0.484444in}}%
\pgfpathlineto{\pgfqpoint{3.320621in}{2.025731in}}%
\pgfusepath{stroke}%
\end{pgfscope}%
\begin{pgfscope}%
\pgfsetrectcap%
\pgfsetmiterjoin%
\pgfsetlinewidth{0.803000pt}%
\definecolor{currentstroke}{rgb}{0.000000,0.000000,0.000000}%
\pgfsetstrokecolor{currentstroke}%
\pgfsetdash{}{0pt}%
\pgfpathmoveto{\pgfqpoint{0.608025in}{0.484444in}}%
\pgfpathlineto{\pgfqpoint{3.320621in}{0.484444in}}%
\pgfusepath{stroke}%
\end{pgfscope}%
\begin{pgfscope}%
\pgfsetrectcap%
\pgfsetmiterjoin%
\pgfsetlinewidth{0.803000pt}%
\definecolor{currentstroke}{rgb}{0.000000,0.000000,0.000000}%
\pgfsetstrokecolor{currentstroke}%
\pgfsetdash{}{0pt}%
\pgfpathmoveto{\pgfqpoint{0.608025in}{2.025731in}}%
\pgfpathlineto{\pgfqpoint{3.320621in}{2.025731in}}%
\pgfusepath{stroke}%
\end{pgfscope}%
\begin{pgfscope}%
\pgfsetbuttcap%
\pgfsetmiterjoin%
\definecolor{currentfill}{rgb}{1.000000,1.000000,1.000000}%
\pgfsetfillcolor{currentfill}%
\pgfsetfillopacity{0.800000}%
\pgfsetlinewidth{1.003750pt}%
\definecolor{currentstroke}{rgb}{0.800000,0.800000,0.800000}%
\pgfsetstrokecolor{currentstroke}%
\pgfsetstrokeopacity{0.800000}%
\pgfsetdash{}{0pt}%
\pgfpathmoveto{\pgfqpoint{1.630076in}{0.944234in}}%
\pgfpathlineto{\pgfqpoint{3.223398in}{0.944234in}}%
\pgfpathquadraticcurveto{\pgfqpoint{3.251176in}{0.944234in}}{\pgfqpoint{3.251176in}{0.972012in}}%
\pgfpathlineto{\pgfqpoint{3.251176in}{1.928509in}}%
\pgfpathquadraticcurveto{\pgfqpoint{3.251176in}{1.956287in}}{\pgfqpoint{3.223398in}{1.956287in}}%
\pgfpathlineto{\pgfqpoint{1.630076in}{1.956287in}}%
\pgfpathquadraticcurveto{\pgfqpoint{1.602298in}{1.956287in}}{\pgfqpoint{1.602298in}{1.928509in}}%
\pgfpathlineto{\pgfqpoint{1.602298in}{0.972012in}}%
\pgfpathquadraticcurveto{\pgfqpoint{1.602298in}{0.944234in}}{\pgfqpoint{1.630076in}{0.944234in}}%
\pgfpathlineto{\pgfqpoint{1.630076in}{0.944234in}}%
\pgfpathclose%
\pgfusepath{stroke,fill}%
\end{pgfscope}%
\begin{pgfscope}%
\pgfsetrectcap%
\pgfsetroundjoin%
\pgfsetlinewidth{1.505625pt}%
\definecolor{currentstroke}{rgb}{0.121569,0.466667,0.705882}%
\pgfsetstrokecolor{currentstroke}%
\pgfsetdash{}{0pt}%
\pgfpathmoveto{\pgfqpoint{1.657854in}{1.852120in}}%
\pgfpathlineto{\pgfqpoint{1.796743in}{1.852120in}}%
\pgfpathlineto{\pgfqpoint{1.935632in}{1.852120in}}%
\pgfusepath{stroke}%
\end{pgfscope}%
\begin{pgfscope}%
\pgfsetbuttcap%
\pgfsetmiterjoin%
\definecolor{currentfill}{rgb}{0.121569,0.466667,0.705882}%
\pgfsetfillcolor{currentfill}%
\pgfsetlinewidth{1.003750pt}%
\definecolor{currentstroke}{rgb}{0.121569,0.466667,0.705882}%
\pgfsetstrokecolor{currentstroke}%
\pgfsetdash{}{0pt}%
\pgfsys@defobject{currentmarker}{\pgfqpoint{-0.020833in}{-0.020833in}}{\pgfqpoint{0.020833in}{0.020833in}}{%
\pgfpathmoveto{\pgfqpoint{-0.020833in}{-0.020833in}}%
\pgfpathlineto{\pgfqpoint{0.020833in}{-0.020833in}}%
\pgfpathlineto{\pgfqpoint{0.020833in}{0.020833in}}%
\pgfpathlineto{\pgfqpoint{-0.020833in}{0.020833in}}%
\pgfpathlineto{\pgfqpoint{-0.020833in}{-0.020833in}}%
\pgfpathclose%
\pgfusepath{stroke,fill}%
}%
\begin{pgfscope}%
\pgfsys@transformshift{1.796743in}{1.852120in}%
\pgfsys@useobject{currentmarker}{}%
\end{pgfscope}%
\end{pgfscope}%
\begin{pgfscope}%
\definecolor{textcolor}{rgb}{0.000000,0.000000,0.000000}%
\pgfsetstrokecolor{textcolor}%
\pgfsetfillcolor{textcolor}%
\pgftext[x=2.046743in,y=1.803509in,left,base]{\color{textcolor}\rmfamily\fontsize{10.000000}{12.000000}\selectfont optimal}%
\end{pgfscope}%
\begin{pgfscope}%
\pgfsetrectcap%
\pgfsetroundjoin%
\pgfsetlinewidth{1.505625pt}%
\definecolor{currentstroke}{rgb}{1.000000,0.498039,0.054902}%
\pgfsetstrokecolor{currentstroke}%
\pgfsetdash{}{0pt}%
\pgfpathmoveto{\pgfqpoint{1.657854in}{1.658509in}}%
\pgfpathlineto{\pgfqpoint{1.796743in}{1.658509in}}%
\pgfpathlineto{\pgfqpoint{1.935632in}{1.658509in}}%
\pgfusepath{stroke}%
\end{pgfscope}%
\begin{pgfscope}%
\pgfsetbuttcap%
\pgfsetroundjoin%
\definecolor{currentfill}{rgb}{1.000000,0.498039,0.054902}%
\pgfsetfillcolor{currentfill}%
\pgfsetlinewidth{1.003750pt}%
\definecolor{currentstroke}{rgb}{1.000000,0.498039,0.054902}%
\pgfsetstrokecolor{currentstroke}%
\pgfsetdash{}{0pt}%
\pgfsys@defobject{currentmarker}{\pgfqpoint{-0.020833in}{-0.020833in}}{\pgfqpoint{0.020833in}{0.020833in}}{%
\pgfpathmoveto{\pgfqpoint{0.000000in}{-0.020833in}}%
\pgfpathcurveto{\pgfqpoint{0.005525in}{-0.020833in}}{\pgfqpoint{0.010825in}{-0.018638in}}{\pgfqpoint{0.014731in}{-0.014731in}}%
\pgfpathcurveto{\pgfqpoint{0.018638in}{-0.010825in}}{\pgfqpoint{0.020833in}{-0.005525in}}{\pgfqpoint{0.020833in}{0.000000in}}%
\pgfpathcurveto{\pgfqpoint{0.020833in}{0.005525in}}{\pgfqpoint{0.018638in}{0.010825in}}{\pgfqpoint{0.014731in}{0.014731in}}%
\pgfpathcurveto{\pgfqpoint{0.010825in}{0.018638in}}{\pgfqpoint{0.005525in}{0.020833in}}{\pgfqpoint{0.000000in}{0.020833in}}%
\pgfpathcurveto{\pgfqpoint{-0.005525in}{0.020833in}}{\pgfqpoint{-0.010825in}{0.018638in}}{\pgfqpoint{-0.014731in}{0.014731in}}%
\pgfpathcurveto{\pgfqpoint{-0.018638in}{0.010825in}}{\pgfqpoint{-0.020833in}{0.005525in}}{\pgfqpoint{-0.020833in}{0.000000in}}%
\pgfpathcurveto{\pgfqpoint{-0.020833in}{-0.005525in}}{\pgfqpoint{-0.018638in}{-0.010825in}}{\pgfqpoint{-0.014731in}{-0.014731in}}%
\pgfpathcurveto{\pgfqpoint{-0.010825in}{-0.018638in}}{\pgfqpoint{-0.005525in}{-0.020833in}}{\pgfqpoint{0.000000in}{-0.020833in}}%
\pgfpathlineto{\pgfqpoint{0.000000in}{-0.020833in}}%
\pgfpathclose%
\pgfusepath{stroke,fill}%
}%
\begin{pgfscope}%
\pgfsys@transformshift{1.796743in}{1.658509in}%
\pgfsys@useobject{currentmarker}{}%
\end{pgfscope}%
\end{pgfscope}%
\begin{pgfscope}%
\definecolor{textcolor}{rgb}{0.000000,0.000000,0.000000}%
\pgfsetstrokecolor{textcolor}%
\pgfsetfillcolor{textcolor}%
\pgftext[x=2.046743in,y=1.609898in,left,base]{\color{textcolor}\rmfamily\fontsize{10.000000}{12.000000}\selectfont adaptive \(\displaystyle \gamma_i,\,K=1\)}%
\end{pgfscope}%
\begin{pgfscope}%
\pgfsetrectcap%
\pgfsetroundjoin%
\pgfsetlinewidth{1.505625pt}%
\definecolor{currentstroke}{rgb}{0.172549,0.627451,0.172549}%
\pgfsetstrokecolor{currentstroke}%
\pgfsetdash{}{0pt}%
\pgfpathmoveto{\pgfqpoint{1.657854in}{1.464142in}}%
\pgfpathlineto{\pgfqpoint{1.796743in}{1.464142in}}%
\pgfpathlineto{\pgfqpoint{1.935632in}{1.464142in}}%
\pgfusepath{stroke}%
\end{pgfscope}%
\begin{pgfscope}%
\pgfsetbuttcap%
\pgfsetmiterjoin%
\definecolor{currentfill}{rgb}{0.172549,0.627451,0.172549}%
\pgfsetfillcolor{currentfill}%
\pgfsetlinewidth{1.003750pt}%
\definecolor{currentstroke}{rgb}{0.172549,0.627451,0.172549}%
\pgfsetstrokecolor{currentstroke}%
\pgfsetdash{}{0pt}%
\pgfsys@defobject{currentmarker}{\pgfqpoint{-0.020833in}{-0.020833in}}{\pgfqpoint{0.020833in}{0.020833in}}{%
\pgfpathmoveto{\pgfqpoint{-0.000000in}{-0.020833in}}%
\pgfpathlineto{\pgfqpoint{0.020833in}{0.020833in}}%
\pgfpathlineto{\pgfqpoint{-0.020833in}{0.020833in}}%
\pgfpathlineto{\pgfqpoint{-0.000000in}{-0.020833in}}%
\pgfpathclose%
\pgfusepath{stroke,fill}%
}%
\begin{pgfscope}%
\pgfsys@transformshift{1.796743in}{1.464142in}%
\pgfsys@useobject{currentmarker}{}%
\end{pgfscope}%
\end{pgfscope}%
\begin{pgfscope}%
\definecolor{textcolor}{rgb}{0.000000,0.000000,0.000000}%
\pgfsetstrokecolor{textcolor}%
\pgfsetfillcolor{textcolor}%
\pgftext[x=2.046743in,y=1.415531in,left,base]{\color{textcolor}\rmfamily\fontsize{10.000000}{12.000000}\selectfont fixed \(\displaystyle \gamma_i,\,K=1\)}%
\end{pgfscope}%
\begin{pgfscope}%
\pgfsetrectcap%
\pgfsetroundjoin%
\pgfsetlinewidth{1.505625pt}%
\definecolor{currentstroke}{rgb}{0.839216,0.152941,0.156863}%
\pgfsetstrokecolor{currentstroke}%
\pgfsetdash{}{0pt}%
\pgfpathmoveto{\pgfqpoint{1.657854in}{1.269774in}}%
\pgfpathlineto{\pgfqpoint{1.796743in}{1.269774in}}%
\pgfpathlineto{\pgfqpoint{1.935632in}{1.269774in}}%
\pgfusepath{stroke}%
\end{pgfscope}%
\begin{pgfscope}%
\pgfsetbuttcap%
\pgfsetmiterjoin%
\definecolor{currentfill}{rgb}{0.839216,0.152941,0.156863}%
\pgfsetfillcolor{currentfill}%
\pgfsetlinewidth{1.003750pt}%
\definecolor{currentstroke}{rgb}{0.839216,0.152941,0.156863}%
\pgfsetstrokecolor{currentstroke}%
\pgfsetdash{}{0pt}%
\pgfsys@defobject{currentmarker}{\pgfqpoint{-0.020833in}{-0.020833in}}{\pgfqpoint{0.020833in}{0.020833in}}{%
\pgfpathmoveto{\pgfqpoint{0.020833in}{-0.000000in}}%
\pgfpathlineto{\pgfqpoint{-0.020833in}{0.020833in}}%
\pgfpathlineto{\pgfqpoint{-0.020833in}{-0.020833in}}%
\pgfpathlineto{\pgfqpoint{0.020833in}{-0.000000in}}%
\pgfpathclose%
\pgfusepath{stroke,fill}%
}%
\begin{pgfscope}%
\pgfsys@transformshift{1.796743in}{1.269774in}%
\pgfsys@useobject{currentmarker}{}%
\end{pgfscope}%
\end{pgfscope}%
\begin{pgfscope}%
\definecolor{textcolor}{rgb}{0.000000,0.000000,0.000000}%
\pgfsetstrokecolor{textcolor}%
\pgfsetfillcolor{textcolor}%
\pgftext[x=2.046743in,y=1.221163in,left,base]{\color{textcolor}\rmfamily\fontsize{10.000000}{12.000000}\selectfont fixed \(\displaystyle \gamma_i,\,K=2\)}%
\end{pgfscope}%
\begin{pgfscope}%
\pgfsetrectcap%
\pgfsetroundjoin%
\pgfsetlinewidth{1.505625pt}%
\definecolor{currentstroke}{rgb}{0.580392,0.403922,0.741176}%
\pgfsetstrokecolor{currentstroke}%
\pgfsetdash{}{0pt}%
\pgfpathmoveto{\pgfqpoint{1.657854in}{1.075407in}}%
\pgfpathlineto{\pgfqpoint{1.796743in}{1.075407in}}%
\pgfpathlineto{\pgfqpoint{1.935632in}{1.075407in}}%
\pgfusepath{stroke}%
\end{pgfscope}%
\begin{pgfscope}%
\pgfsetbuttcap%
\pgfsetmiterjoin%
\definecolor{currentfill}{rgb}{0.580392,0.403922,0.741176}%
\pgfsetfillcolor{currentfill}%
\pgfsetlinewidth{1.003750pt}%
\definecolor{currentstroke}{rgb}{0.580392,0.403922,0.741176}%
\pgfsetstrokecolor{currentstroke}%
\pgfsetdash{}{0pt}%
\pgfsys@defobject{currentmarker}{\pgfqpoint{-0.020833in}{-0.020833in}}{\pgfqpoint{0.020833in}{0.020833in}}{%
\pgfpathmoveto{\pgfqpoint{-0.020833in}{0.000000in}}%
\pgfpathlineto{\pgfqpoint{0.020833in}{-0.020833in}}%
\pgfpathlineto{\pgfqpoint{0.020833in}{0.020833in}}%
\pgfpathlineto{\pgfqpoint{-0.020833in}{0.000000in}}%
\pgfpathclose%
\pgfusepath{stroke,fill}%
}%
\begin{pgfscope}%
\pgfsys@transformshift{1.796743in}{1.075407in}%
\pgfsys@useobject{currentmarker}{}%
\end{pgfscope}%
\end{pgfscope}%
\begin{pgfscope}%
\definecolor{textcolor}{rgb}{0.000000,0.000000,0.000000}%
\pgfsetstrokecolor{textcolor}%
\pgfsetfillcolor{textcolor}%
\pgftext[x=2.046743in,y=1.026796in,left,base]{\color{textcolor}\rmfamily\fontsize{10.000000}{12.000000}\selectfont fixed \(\displaystyle \gamma_i,\,K=10\)}%
\end{pgfscope}%
\end{pgfpicture}%
\makeatother%
\endgroup%

    \vspace*{-0.6cm}
    \caption[]{Median NPM of 30 Monte-Carlo runs over time for optimal case where all node information is available (fully connected network/broadcasting), fixed values of \(\gamma_i\) with \(K \in \{1,10\}\), adaptive \(\gamma_i\) with \(K=1\). A moving average filter was applied to result curves for better readability (compare transparent and opaque lines). Hatched area indicates frames used in computations for \autoref{fig:simulations:avgNPMgamma}.}
    \label{fig:simulations:NPMtime}
\end{figure}
Given that the BSI algorithm being extended \cite{blochbergerDBSI} is adaptive and the unnormalized consensus estimates \(\bar{\h}_i^{(n)}\) are therefore time-varying, computing a distributed sum of their norms has to be performed adaptively.
The baseline approach is to compute \(K\) iterations \eqref{eq:adaptivenormest:distlincomb} per frame index \(n\).
For the baseline iterative approach to converge, \(K\) has to be sufficiently large (e.g., \(K=50\) in \cite{yuDistributedBlindSystem2014,liuDistributedBlindIdentification2016}).
This is suboptimal in terms of communication cost, as each iteration requires that a node \(i\) shares the averaging variable \(\phi_i(k)\) with its neighborhood \(\Nset_i\).
Therefore, we first propose to split iterations over time frames.
For each \(n\), we could set the initial values \(\phi_i^{(n)}(0)\) to the result of the \(K\)-th iteration \(\phi_i^{(n-1)}(K)\) of the previous frame \(n-1\).
With sufficiently large \(n\), convergence will be reached even when \(K\) is chosen small.
However, this rudimentary modification does not factor in the adaptive approach to channel identification.
That is, when letting \(\phi_i^{(0)}(0) = \|\bar{\h}_i^{(0)}\|\), with \(\bar{\h}_i^{(0)}\) being estimated at the inital frame \(n=0\), it converges to the sum of these initial - poor - estimates.
Therefore, we factor in the instantaneous approximation, the weighted in-neighborhood sum \(\eta_i^{(n)} = \sum_{j \in \Nset_i} W_{ij} \|\bar{\h}_i^{(n)}\|^2\), which will introduce new information gained by more accurate estimates of \(\bar{\h}_i^{(n)}\).
We add this information by linearly combining it with the \(K\)-th iterate of \(\phi_i^{(n)}(k)\) and set it as the initial value for the next frame:
\begin{equation}
    \phi_i^{(n)}(0) = \gamma_i \eta_i^{(n)} + (1-\gamma_i) \phi_i^{(n-1)}(K),
\end{equation}
where \(\gamma_i\) is a mixing factor.
We exploit the instantaneous information \(\eta_i^{(n)}\) to find values for the mixing factor.
If the absolute difference between subsequent frames \(| \eta_i^{(n)} - \eta_i^{(n-1)} | \to 0\), then the estimation of \(\bar{\h}_i\) is converging to a steady state.
The algorithm's emphasis should then lie on norm estimation, i.e., the distributed averaging recursion, which means \(\gamma_i \to 0\).
We heuristically set \(\gamma_i^{(n)}\) proportional to the absolute difference of instantaneous approximations between subsequent frames and set an upper limit at 1,
\begin{equation}
    \gamma_i^{(n)} = \min \left\lbrace \frac{| \eta_i^{(n)} - \eta_i^{(n-1)} |}{\eta_i^{(n-1)}},\,1\right\rbrace.\label{eq:adaptivenormest:adaptivegamma}
\end{equation}
The initial value for \(\eta_i^{(n-1)}\) for \(n=0\) can be set to an arbitrary real number, in our case 1.
Refer to \autoref{alg:davg_norm_est} for the order of computations of the extended algorithm we introduced here.

\section{Evaluation}
\label{sec:simulations}
\begin{figure}[t]
    \centering
    %% Creator: Matplotlib, PGF backend
%%
%% To include the figure in your LaTeX document, write
%%   \input{<filename>.pgf}
%%
%% Make sure the required packages are loaded in your preamble
%%   \usepackage{pgf}
%%
%% Also ensure that all the required font packages are loaded; for instance,
%% the lmodern package is sometimes necessary when using math font.
%%   \usepackage{lmodern}
%%
%% Figures using additional raster images can only be included by \input if
%% they are in the same directory as the main LaTeX file. For loading figures
%% from other directories you can use the `import` package
%%   \usepackage{import}
%%
%% and then include the figures with
%%   \import{<path to file>}{<filename>.pgf}
%%
%% Matplotlib used the following preamble
%%   \usepackage{fontspec}
%%
\begingroup%
\makeatletter%
\begin{pgfpicture}%
\pgfpathrectangle{\pgfpointorigin}{\pgfqpoint{3.390065in}{1.695033in}}%
\pgfusepath{use as bounding box, clip}%
\begin{pgfscope}%
\pgfsetbuttcap%
\pgfsetmiterjoin%
\definecolor{currentfill}{rgb}{1.000000,1.000000,1.000000}%
\pgfsetfillcolor{currentfill}%
\pgfsetlinewidth{0.000000pt}%
\definecolor{currentstroke}{rgb}{1.000000,1.000000,1.000000}%
\pgfsetstrokecolor{currentstroke}%
\pgfsetstrokeopacity{0.000000}%
\pgfsetdash{}{0pt}%
\pgfpathmoveto{\pgfqpoint{0.000000in}{0.000000in}}%
\pgfpathlineto{\pgfqpoint{3.390065in}{0.000000in}}%
\pgfpathlineto{\pgfqpoint{3.390065in}{1.695033in}}%
\pgfpathlineto{\pgfqpoint{0.000000in}{1.695033in}}%
\pgfpathlineto{\pgfqpoint{0.000000in}{0.000000in}}%
\pgfpathclose%
\pgfusepath{fill}%
\end{pgfscope}%
\begin{pgfscope}%
\pgfsetbuttcap%
\pgfsetmiterjoin%
\definecolor{currentfill}{rgb}{1.000000,1.000000,1.000000}%
\pgfsetfillcolor{currentfill}%
\pgfsetlinewidth{0.000000pt}%
\definecolor{currentstroke}{rgb}{0.000000,0.000000,0.000000}%
\pgfsetstrokecolor{currentstroke}%
\pgfsetstrokeopacity{0.000000}%
\pgfsetdash{}{0pt}%
\pgfpathmoveto{\pgfqpoint{0.568671in}{0.451277in}}%
\pgfpathlineto{\pgfqpoint{3.213368in}{0.451277in}}%
\pgfpathlineto{\pgfqpoint{3.213368in}{1.589158in}}%
\pgfpathlineto{\pgfqpoint{0.568671in}{1.589158in}}%
\pgfpathlineto{\pgfqpoint{0.568671in}{0.451277in}}%
\pgfpathclose%
\pgfusepath{fill}%
\end{pgfscope}%
\begin{pgfscope}%
\pgfpathrectangle{\pgfqpoint{0.568671in}{0.451277in}}{\pgfqpoint{2.644697in}{1.137880in}}%
\pgfusepath{clip}%
\pgfsetrectcap%
\pgfsetroundjoin%
\pgfsetlinewidth{0.803000pt}%
\definecolor{currentstroke}{rgb}{0.690196,0.690196,0.690196}%
\pgfsetstrokecolor{currentstroke}%
\pgfsetdash{}{0pt}%
\pgfpathmoveto{\pgfqpoint{0.568671in}{0.451277in}}%
\pgfpathlineto{\pgfqpoint{0.568671in}{1.589158in}}%
\pgfusepath{stroke}%
\end{pgfscope}%
\begin{pgfscope}%
\pgfsetbuttcap%
\pgfsetroundjoin%
\definecolor{currentfill}{rgb}{0.000000,0.000000,0.000000}%
\pgfsetfillcolor{currentfill}%
\pgfsetlinewidth{0.803000pt}%
\definecolor{currentstroke}{rgb}{0.000000,0.000000,0.000000}%
\pgfsetstrokecolor{currentstroke}%
\pgfsetdash{}{0pt}%
\pgfsys@defobject{currentmarker}{\pgfqpoint{0.000000in}{-0.048611in}}{\pgfqpoint{0.000000in}{0.000000in}}{%
\pgfpathmoveto{\pgfqpoint{0.000000in}{0.000000in}}%
\pgfpathlineto{\pgfqpoint{0.000000in}{-0.048611in}}%
\pgfusepath{stroke,fill}%
}%
\begin{pgfscope}%
\pgfsys@transformshift{0.568671in}{0.451277in}%
\pgfsys@useobject{currentmarker}{}%
\end{pgfscope}%
\end{pgfscope}%
\begin{pgfscope}%
\definecolor{textcolor}{rgb}{0.000000,0.000000,0.000000}%
\pgfsetstrokecolor{textcolor}%
\pgfsetfillcolor{textcolor}%
\pgftext[x=0.568671in,y=0.354055in,,top]{\color{textcolor}\rmfamily\fontsize{9.000000}{10.800000}\selectfont \(\displaystyle {0.00}\)}%
\end{pgfscope}%
\begin{pgfscope}%
\pgfpathrectangle{\pgfqpoint{0.568671in}{0.451277in}}{\pgfqpoint{2.644697in}{1.137880in}}%
\pgfusepath{clip}%
\pgfsetrectcap%
\pgfsetroundjoin%
\pgfsetlinewidth{0.803000pt}%
\definecolor{currentstroke}{rgb}{0.690196,0.690196,0.690196}%
\pgfsetstrokecolor{currentstroke}%
\pgfsetdash{}{0pt}%
\pgfpathmoveto{\pgfqpoint{1.229845in}{0.451277in}}%
\pgfpathlineto{\pgfqpoint{1.229845in}{1.589158in}}%
\pgfusepath{stroke}%
\end{pgfscope}%
\begin{pgfscope}%
\pgfsetbuttcap%
\pgfsetroundjoin%
\definecolor{currentfill}{rgb}{0.000000,0.000000,0.000000}%
\pgfsetfillcolor{currentfill}%
\pgfsetlinewidth{0.803000pt}%
\definecolor{currentstroke}{rgb}{0.000000,0.000000,0.000000}%
\pgfsetstrokecolor{currentstroke}%
\pgfsetdash{}{0pt}%
\pgfsys@defobject{currentmarker}{\pgfqpoint{0.000000in}{-0.048611in}}{\pgfqpoint{0.000000in}{0.000000in}}{%
\pgfpathmoveto{\pgfqpoint{0.000000in}{0.000000in}}%
\pgfpathlineto{\pgfqpoint{0.000000in}{-0.048611in}}%
\pgfusepath{stroke,fill}%
}%
\begin{pgfscope}%
\pgfsys@transformshift{1.229845in}{0.451277in}%
\pgfsys@useobject{currentmarker}{}%
\end{pgfscope}%
\end{pgfscope}%
\begin{pgfscope}%
\definecolor{textcolor}{rgb}{0.000000,0.000000,0.000000}%
\pgfsetstrokecolor{textcolor}%
\pgfsetfillcolor{textcolor}%
\pgftext[x=1.229845in,y=0.354055in,,top]{\color{textcolor}\rmfamily\fontsize{9.000000}{10.800000}\selectfont \(\displaystyle {0.01}\)}%
\end{pgfscope}%
\begin{pgfscope}%
\pgfpathrectangle{\pgfqpoint{0.568671in}{0.451277in}}{\pgfqpoint{2.644697in}{1.137880in}}%
\pgfusepath{clip}%
\pgfsetrectcap%
\pgfsetroundjoin%
\pgfsetlinewidth{0.803000pt}%
\definecolor{currentstroke}{rgb}{0.690196,0.690196,0.690196}%
\pgfsetstrokecolor{currentstroke}%
\pgfsetdash{}{0pt}%
\pgfpathmoveto{\pgfqpoint{1.891020in}{0.451277in}}%
\pgfpathlineto{\pgfqpoint{1.891020in}{1.589158in}}%
\pgfusepath{stroke}%
\end{pgfscope}%
\begin{pgfscope}%
\pgfsetbuttcap%
\pgfsetroundjoin%
\definecolor{currentfill}{rgb}{0.000000,0.000000,0.000000}%
\pgfsetfillcolor{currentfill}%
\pgfsetlinewidth{0.803000pt}%
\definecolor{currentstroke}{rgb}{0.000000,0.000000,0.000000}%
\pgfsetstrokecolor{currentstroke}%
\pgfsetdash{}{0pt}%
\pgfsys@defobject{currentmarker}{\pgfqpoint{0.000000in}{-0.048611in}}{\pgfqpoint{0.000000in}{0.000000in}}{%
\pgfpathmoveto{\pgfqpoint{0.000000in}{0.000000in}}%
\pgfpathlineto{\pgfqpoint{0.000000in}{-0.048611in}}%
\pgfusepath{stroke,fill}%
}%
\begin{pgfscope}%
\pgfsys@transformshift{1.891020in}{0.451277in}%
\pgfsys@useobject{currentmarker}{}%
\end{pgfscope}%
\end{pgfscope}%
\begin{pgfscope}%
\definecolor{textcolor}{rgb}{0.000000,0.000000,0.000000}%
\pgfsetstrokecolor{textcolor}%
\pgfsetfillcolor{textcolor}%
\pgftext[x=1.891020in,y=0.354055in,,top]{\color{textcolor}\rmfamily\fontsize{9.000000}{10.800000}\selectfont \(\displaystyle {0.02}\)}%
\end{pgfscope}%
\begin{pgfscope}%
\pgfpathrectangle{\pgfqpoint{0.568671in}{0.451277in}}{\pgfqpoint{2.644697in}{1.137880in}}%
\pgfusepath{clip}%
\pgfsetrectcap%
\pgfsetroundjoin%
\pgfsetlinewidth{0.803000pt}%
\definecolor{currentstroke}{rgb}{0.690196,0.690196,0.690196}%
\pgfsetstrokecolor{currentstroke}%
\pgfsetdash{}{0pt}%
\pgfpathmoveto{\pgfqpoint{2.552194in}{0.451277in}}%
\pgfpathlineto{\pgfqpoint{2.552194in}{1.589158in}}%
\pgfusepath{stroke}%
\end{pgfscope}%
\begin{pgfscope}%
\pgfsetbuttcap%
\pgfsetroundjoin%
\definecolor{currentfill}{rgb}{0.000000,0.000000,0.000000}%
\pgfsetfillcolor{currentfill}%
\pgfsetlinewidth{0.803000pt}%
\definecolor{currentstroke}{rgb}{0.000000,0.000000,0.000000}%
\pgfsetstrokecolor{currentstroke}%
\pgfsetdash{}{0pt}%
\pgfsys@defobject{currentmarker}{\pgfqpoint{0.000000in}{-0.048611in}}{\pgfqpoint{0.000000in}{0.000000in}}{%
\pgfpathmoveto{\pgfqpoint{0.000000in}{0.000000in}}%
\pgfpathlineto{\pgfqpoint{0.000000in}{-0.048611in}}%
\pgfusepath{stroke,fill}%
}%
\begin{pgfscope}%
\pgfsys@transformshift{2.552194in}{0.451277in}%
\pgfsys@useobject{currentmarker}{}%
\end{pgfscope}%
\end{pgfscope}%
\begin{pgfscope}%
\definecolor{textcolor}{rgb}{0.000000,0.000000,0.000000}%
\pgfsetstrokecolor{textcolor}%
\pgfsetfillcolor{textcolor}%
\pgftext[x=2.552194in,y=0.354055in,,top]{\color{textcolor}\rmfamily\fontsize{9.000000}{10.800000}\selectfont \(\displaystyle {0.03}\)}%
\end{pgfscope}%
\begin{pgfscope}%
\pgfpathrectangle{\pgfqpoint{0.568671in}{0.451277in}}{\pgfqpoint{2.644697in}{1.137880in}}%
\pgfusepath{clip}%
\pgfsetrectcap%
\pgfsetroundjoin%
\pgfsetlinewidth{0.803000pt}%
\definecolor{currentstroke}{rgb}{0.690196,0.690196,0.690196}%
\pgfsetstrokecolor{currentstroke}%
\pgfsetdash{}{0pt}%
\pgfpathmoveto{\pgfqpoint{3.213368in}{0.451277in}}%
\pgfpathlineto{\pgfqpoint{3.213368in}{1.589158in}}%
\pgfusepath{stroke}%
\end{pgfscope}%
\begin{pgfscope}%
\pgfsetbuttcap%
\pgfsetroundjoin%
\definecolor{currentfill}{rgb}{0.000000,0.000000,0.000000}%
\pgfsetfillcolor{currentfill}%
\pgfsetlinewidth{0.803000pt}%
\definecolor{currentstroke}{rgb}{0.000000,0.000000,0.000000}%
\pgfsetstrokecolor{currentstroke}%
\pgfsetdash{}{0pt}%
\pgfsys@defobject{currentmarker}{\pgfqpoint{0.000000in}{-0.048611in}}{\pgfqpoint{0.000000in}{0.000000in}}{%
\pgfpathmoveto{\pgfqpoint{0.000000in}{0.000000in}}%
\pgfpathlineto{\pgfqpoint{0.000000in}{-0.048611in}}%
\pgfusepath{stroke,fill}%
}%
\begin{pgfscope}%
\pgfsys@transformshift{3.213368in}{0.451277in}%
\pgfsys@useobject{currentmarker}{}%
\end{pgfscope}%
\end{pgfscope}%
\begin{pgfscope}%
\definecolor{textcolor}{rgb}{0.000000,0.000000,0.000000}%
\pgfsetstrokecolor{textcolor}%
\pgfsetfillcolor{textcolor}%
\pgftext[x=3.213368in,y=0.354055in,,top]{\color{textcolor}\rmfamily\fontsize{9.000000}{10.800000}\selectfont \(\displaystyle {0.04}\)}%
\end{pgfscope}%
\begin{pgfscope}%
\definecolor{textcolor}{rgb}{0.000000,0.000000,0.000000}%
\pgfsetstrokecolor{textcolor}%
\pgfsetfillcolor{textcolor}%
\pgftext[x=1.891020in,y=0.187500in,,top]{\color{textcolor}\rmfamily\fontsize{9.000000}{10.800000}\selectfont Mixing factor \(\displaystyle \gamma\) [1]}%
\end{pgfscope}%
\begin{pgfscope}%
\pgfpathrectangle{\pgfqpoint{0.568671in}{0.451277in}}{\pgfqpoint{2.644697in}{1.137880in}}%
\pgfusepath{clip}%
\pgfsetrectcap%
\pgfsetroundjoin%
\pgfsetlinewidth{0.803000pt}%
\definecolor{currentstroke}{rgb}{0.690196,0.690196,0.690196}%
\pgfsetstrokecolor{currentstroke}%
\pgfsetdash{}{0pt}%
\pgfpathmoveto{\pgfqpoint{0.568671in}{0.451277in}}%
\pgfpathlineto{\pgfqpoint{3.213368in}{0.451277in}}%
\pgfusepath{stroke}%
\end{pgfscope}%
\begin{pgfscope}%
\pgfsetbuttcap%
\pgfsetroundjoin%
\definecolor{currentfill}{rgb}{0.000000,0.000000,0.000000}%
\pgfsetfillcolor{currentfill}%
\pgfsetlinewidth{0.803000pt}%
\definecolor{currentstroke}{rgb}{0.000000,0.000000,0.000000}%
\pgfsetstrokecolor{currentstroke}%
\pgfsetdash{}{0pt}%
\pgfsys@defobject{currentmarker}{\pgfqpoint{-0.048611in}{0.000000in}}{\pgfqpoint{-0.000000in}{0.000000in}}{%
\pgfpathmoveto{\pgfqpoint{-0.000000in}{0.000000in}}%
\pgfpathlineto{\pgfqpoint{-0.048611in}{0.000000in}}%
\pgfusepath{stroke,fill}%
}%
\begin{pgfscope}%
\pgfsys@transformshift{0.568671in}{0.451277in}%
\pgfsys@useobject{currentmarker}{}%
\end{pgfscope}%
\end{pgfscope}%
\begin{pgfscope}%
\definecolor{textcolor}{rgb}{0.000000,0.000000,0.000000}%
\pgfsetstrokecolor{textcolor}%
\pgfsetfillcolor{textcolor}%
\pgftext[x=0.243055in, y=0.407902in, left, base]{\color{textcolor}\rmfamily\fontsize{9.000000}{10.800000}\selectfont \(\displaystyle {\ensuremath{-}40}\)}%
\end{pgfscope}%
\begin{pgfscope}%
\pgfpathrectangle{\pgfqpoint{0.568671in}{0.451277in}}{\pgfqpoint{2.644697in}{1.137880in}}%
\pgfusepath{clip}%
\pgfsetrectcap%
\pgfsetroundjoin%
\pgfsetlinewidth{0.803000pt}%
\definecolor{currentstroke}{rgb}{0.690196,0.690196,0.690196}%
\pgfsetstrokecolor{currentstroke}%
\pgfsetdash{}{0pt}%
\pgfpathmoveto{\pgfqpoint{0.568671in}{0.735747in}}%
\pgfpathlineto{\pgfqpoint{3.213368in}{0.735747in}}%
\pgfusepath{stroke}%
\end{pgfscope}%
\begin{pgfscope}%
\pgfsetbuttcap%
\pgfsetroundjoin%
\definecolor{currentfill}{rgb}{0.000000,0.000000,0.000000}%
\pgfsetfillcolor{currentfill}%
\pgfsetlinewidth{0.803000pt}%
\definecolor{currentstroke}{rgb}{0.000000,0.000000,0.000000}%
\pgfsetstrokecolor{currentstroke}%
\pgfsetdash{}{0pt}%
\pgfsys@defobject{currentmarker}{\pgfqpoint{-0.048611in}{0.000000in}}{\pgfqpoint{-0.000000in}{0.000000in}}{%
\pgfpathmoveto{\pgfqpoint{-0.000000in}{0.000000in}}%
\pgfpathlineto{\pgfqpoint{-0.048611in}{0.000000in}}%
\pgfusepath{stroke,fill}%
}%
\begin{pgfscope}%
\pgfsys@transformshift{0.568671in}{0.735747in}%
\pgfsys@useobject{currentmarker}{}%
\end{pgfscope}%
\end{pgfscope}%
\begin{pgfscope}%
\definecolor{textcolor}{rgb}{0.000000,0.000000,0.000000}%
\pgfsetstrokecolor{textcolor}%
\pgfsetfillcolor{textcolor}%
\pgftext[x=0.243055in, y=0.692372in, left, base]{\color{textcolor}\rmfamily\fontsize{9.000000}{10.800000}\selectfont \(\displaystyle {\ensuremath{-}35}\)}%
\end{pgfscope}%
\begin{pgfscope}%
\pgfpathrectangle{\pgfqpoint{0.568671in}{0.451277in}}{\pgfqpoint{2.644697in}{1.137880in}}%
\pgfusepath{clip}%
\pgfsetrectcap%
\pgfsetroundjoin%
\pgfsetlinewidth{0.803000pt}%
\definecolor{currentstroke}{rgb}{0.690196,0.690196,0.690196}%
\pgfsetstrokecolor{currentstroke}%
\pgfsetdash{}{0pt}%
\pgfpathmoveto{\pgfqpoint{0.568671in}{1.020217in}}%
\pgfpathlineto{\pgfqpoint{3.213368in}{1.020217in}}%
\pgfusepath{stroke}%
\end{pgfscope}%
\begin{pgfscope}%
\pgfsetbuttcap%
\pgfsetroundjoin%
\definecolor{currentfill}{rgb}{0.000000,0.000000,0.000000}%
\pgfsetfillcolor{currentfill}%
\pgfsetlinewidth{0.803000pt}%
\definecolor{currentstroke}{rgb}{0.000000,0.000000,0.000000}%
\pgfsetstrokecolor{currentstroke}%
\pgfsetdash{}{0pt}%
\pgfsys@defobject{currentmarker}{\pgfqpoint{-0.048611in}{0.000000in}}{\pgfqpoint{-0.000000in}{0.000000in}}{%
\pgfpathmoveto{\pgfqpoint{-0.000000in}{0.000000in}}%
\pgfpathlineto{\pgfqpoint{-0.048611in}{0.000000in}}%
\pgfusepath{stroke,fill}%
}%
\begin{pgfscope}%
\pgfsys@transformshift{0.568671in}{1.020217in}%
\pgfsys@useobject{currentmarker}{}%
\end{pgfscope}%
\end{pgfscope}%
\begin{pgfscope}%
\definecolor{textcolor}{rgb}{0.000000,0.000000,0.000000}%
\pgfsetstrokecolor{textcolor}%
\pgfsetfillcolor{textcolor}%
\pgftext[x=0.243055in, y=0.976842in, left, base]{\color{textcolor}\rmfamily\fontsize{9.000000}{10.800000}\selectfont \(\displaystyle {\ensuremath{-}30}\)}%
\end{pgfscope}%
\begin{pgfscope}%
\pgfpathrectangle{\pgfqpoint{0.568671in}{0.451277in}}{\pgfqpoint{2.644697in}{1.137880in}}%
\pgfusepath{clip}%
\pgfsetrectcap%
\pgfsetroundjoin%
\pgfsetlinewidth{0.803000pt}%
\definecolor{currentstroke}{rgb}{0.690196,0.690196,0.690196}%
\pgfsetstrokecolor{currentstroke}%
\pgfsetdash{}{0pt}%
\pgfpathmoveto{\pgfqpoint{0.568671in}{1.304687in}}%
\pgfpathlineto{\pgfqpoint{3.213368in}{1.304687in}}%
\pgfusepath{stroke}%
\end{pgfscope}%
\begin{pgfscope}%
\pgfsetbuttcap%
\pgfsetroundjoin%
\definecolor{currentfill}{rgb}{0.000000,0.000000,0.000000}%
\pgfsetfillcolor{currentfill}%
\pgfsetlinewidth{0.803000pt}%
\definecolor{currentstroke}{rgb}{0.000000,0.000000,0.000000}%
\pgfsetstrokecolor{currentstroke}%
\pgfsetdash{}{0pt}%
\pgfsys@defobject{currentmarker}{\pgfqpoint{-0.048611in}{0.000000in}}{\pgfqpoint{-0.000000in}{0.000000in}}{%
\pgfpathmoveto{\pgfqpoint{-0.000000in}{0.000000in}}%
\pgfpathlineto{\pgfqpoint{-0.048611in}{0.000000in}}%
\pgfusepath{stroke,fill}%
}%
\begin{pgfscope}%
\pgfsys@transformshift{0.568671in}{1.304687in}%
\pgfsys@useobject{currentmarker}{}%
\end{pgfscope}%
\end{pgfscope}%
\begin{pgfscope}%
\definecolor{textcolor}{rgb}{0.000000,0.000000,0.000000}%
\pgfsetstrokecolor{textcolor}%
\pgfsetfillcolor{textcolor}%
\pgftext[x=0.243055in, y=1.261312in, left, base]{\color{textcolor}\rmfamily\fontsize{9.000000}{10.800000}\selectfont \(\displaystyle {\ensuremath{-}25}\)}%
\end{pgfscope}%
\begin{pgfscope}%
\pgfpathrectangle{\pgfqpoint{0.568671in}{0.451277in}}{\pgfqpoint{2.644697in}{1.137880in}}%
\pgfusepath{clip}%
\pgfsetrectcap%
\pgfsetroundjoin%
\pgfsetlinewidth{0.803000pt}%
\definecolor{currentstroke}{rgb}{0.690196,0.690196,0.690196}%
\pgfsetstrokecolor{currentstroke}%
\pgfsetdash{}{0pt}%
\pgfpathmoveto{\pgfqpoint{0.568671in}{1.589158in}}%
\pgfpathlineto{\pgfqpoint{3.213368in}{1.589158in}}%
\pgfusepath{stroke}%
\end{pgfscope}%
\begin{pgfscope}%
\pgfsetbuttcap%
\pgfsetroundjoin%
\definecolor{currentfill}{rgb}{0.000000,0.000000,0.000000}%
\pgfsetfillcolor{currentfill}%
\pgfsetlinewidth{0.803000pt}%
\definecolor{currentstroke}{rgb}{0.000000,0.000000,0.000000}%
\pgfsetstrokecolor{currentstroke}%
\pgfsetdash{}{0pt}%
\pgfsys@defobject{currentmarker}{\pgfqpoint{-0.048611in}{0.000000in}}{\pgfqpoint{-0.000000in}{0.000000in}}{%
\pgfpathmoveto{\pgfqpoint{-0.000000in}{0.000000in}}%
\pgfpathlineto{\pgfqpoint{-0.048611in}{0.000000in}}%
\pgfusepath{stroke,fill}%
}%
\begin{pgfscope}%
\pgfsys@transformshift{0.568671in}{1.589158in}%
\pgfsys@useobject{currentmarker}{}%
\end{pgfscope}%
\end{pgfscope}%
\begin{pgfscope}%
\definecolor{textcolor}{rgb}{0.000000,0.000000,0.000000}%
\pgfsetstrokecolor{textcolor}%
\pgfsetfillcolor{textcolor}%
\pgftext[x=0.243055in, y=1.545783in, left, base]{\color{textcolor}\rmfamily\fontsize{9.000000}{10.800000}\selectfont \(\displaystyle {\ensuremath{-}20}\)}%
\end{pgfscope}%
\begin{pgfscope}%
\definecolor{textcolor}{rgb}{0.000000,0.000000,0.000000}%
\pgfsetstrokecolor{textcolor}%
\pgfsetfillcolor{textcolor}%
\pgftext[x=0.187500in,y=1.020217in,,bottom,rotate=90.000000]{\color{textcolor}\rmfamily\fontsize{9.000000}{10.800000}\selectfont Avg. NPM [dB]}%
\end{pgfscope}%
\begin{pgfscope}%
\pgfpathrectangle{\pgfqpoint{0.568671in}{0.451277in}}{\pgfqpoint{2.644697in}{1.137880in}}%
\pgfusepath{clip}%
\pgfsetrectcap%
\pgfsetroundjoin%
\pgfsetlinewidth{1.505625pt}%
\definecolor{currentstroke}{rgb}{0.000000,0.000000,0.000000}%
\pgfsetstrokecolor{currentstroke}%
\pgfsetdash{}{0pt}%
\pgfpathmoveto{\pgfqpoint{0.568671in}{0.687304in}}%
\pgfpathlineto{\pgfqpoint{0.700906in}{0.687304in}}%
\pgfpathlineto{\pgfqpoint{0.833141in}{0.687304in}}%
\pgfpathlineto{\pgfqpoint{0.965376in}{0.687304in}}%
\pgfpathlineto{\pgfqpoint{1.097611in}{0.687304in}}%
\pgfpathlineto{\pgfqpoint{1.229845in}{0.687304in}}%
\pgfpathlineto{\pgfqpoint{1.362080in}{0.687304in}}%
\pgfpathlineto{\pgfqpoint{1.494315in}{0.687304in}}%
\pgfpathlineto{\pgfqpoint{1.626550in}{0.687304in}}%
\pgfpathlineto{\pgfqpoint{1.758785in}{0.687304in}}%
\pgfpathlineto{\pgfqpoint{1.891020in}{0.687304in}}%
\pgfpathlineto{\pgfqpoint{2.023255in}{0.687304in}}%
\pgfpathlineto{\pgfqpoint{2.155489in}{0.687304in}}%
\pgfpathlineto{\pgfqpoint{2.287724in}{0.687304in}}%
\pgfpathlineto{\pgfqpoint{2.419959in}{0.687304in}}%
\pgfpathlineto{\pgfqpoint{2.552194in}{0.687304in}}%
\pgfpathlineto{\pgfqpoint{2.684429in}{0.687304in}}%
\pgfpathlineto{\pgfqpoint{2.816664in}{0.687304in}}%
\pgfpathlineto{\pgfqpoint{2.948899in}{0.687304in}}%
\pgfpathlineto{\pgfqpoint{3.081133in}{0.687304in}}%
\pgfpathlineto{\pgfqpoint{3.213368in}{0.687304in}}%
\pgfusepath{stroke}%
\end{pgfscope}%
\begin{pgfscope}%
\pgfpathrectangle{\pgfqpoint{0.568671in}{0.451277in}}{\pgfqpoint{2.644697in}{1.137880in}}%
\pgfusepath{clip}%
\pgfsetrectcap%
\pgfsetroundjoin%
\pgfsetlinewidth{1.505625pt}%
\definecolor{currentstroke}{rgb}{0.121569,0.466667,0.705882}%
\pgfsetstrokecolor{currentstroke}%
\pgfsetdash{}{0pt}%
\pgfpathmoveto{\pgfqpoint{0.568671in}{0.706710in}}%
\pgfpathlineto{\pgfqpoint{0.700906in}{0.706710in}}%
\pgfpathlineto{\pgfqpoint{0.833141in}{0.706710in}}%
\pgfpathlineto{\pgfqpoint{0.965376in}{0.706710in}}%
\pgfpathlineto{\pgfqpoint{1.097611in}{0.706710in}}%
\pgfpathlineto{\pgfqpoint{1.229845in}{0.706710in}}%
\pgfpathlineto{\pgfqpoint{1.362080in}{0.706710in}}%
\pgfpathlineto{\pgfqpoint{1.494315in}{0.706710in}}%
\pgfpathlineto{\pgfqpoint{1.626550in}{0.706710in}}%
\pgfpathlineto{\pgfqpoint{1.758785in}{0.706710in}}%
\pgfpathlineto{\pgfqpoint{1.891020in}{0.706710in}}%
\pgfpathlineto{\pgfqpoint{2.023255in}{0.706710in}}%
\pgfpathlineto{\pgfqpoint{2.155489in}{0.706710in}}%
\pgfpathlineto{\pgfqpoint{2.287724in}{0.706710in}}%
\pgfpathlineto{\pgfqpoint{2.419959in}{0.706710in}}%
\pgfpathlineto{\pgfqpoint{2.552194in}{0.706710in}}%
\pgfpathlineto{\pgfqpoint{2.684429in}{0.706710in}}%
\pgfpathlineto{\pgfqpoint{2.816664in}{0.706710in}}%
\pgfpathlineto{\pgfqpoint{2.948899in}{0.706710in}}%
\pgfpathlineto{\pgfqpoint{3.081133in}{0.706710in}}%
\pgfpathlineto{\pgfqpoint{3.213368in}{0.706710in}}%
\pgfusepath{stroke}%
\end{pgfscope}%
\begin{pgfscope}%
\pgfpathrectangle{\pgfqpoint{0.568671in}{0.451277in}}{\pgfqpoint{2.644697in}{1.137880in}}%
\pgfusepath{clip}%
\pgfsetbuttcap%
\pgfsetroundjoin%
\definecolor{currentfill}{rgb}{0.000000,0.000000,0.000000}%
\pgfsetfillcolor{currentfill}%
\pgfsetfillopacity{0.000000}%
\pgfsetlinewidth{1.003750pt}%
\definecolor{currentstroke}{rgb}{0.121569,0.466667,0.705882}%
\pgfsetstrokecolor{currentstroke}%
\pgfsetdash{}{0pt}%
\pgfsys@defobject{currentmarker}{\pgfqpoint{-0.027778in}{-0.027778in}}{\pgfqpoint{0.027778in}{0.027778in}}{%
\pgfpathmoveto{\pgfqpoint{0.000000in}{-0.027778in}}%
\pgfpathcurveto{\pgfqpoint{0.007367in}{-0.027778in}}{\pgfqpoint{0.014433in}{-0.024851in}}{\pgfqpoint{0.019642in}{-0.019642in}}%
\pgfpathcurveto{\pgfqpoint{0.024851in}{-0.014433in}}{\pgfqpoint{0.027778in}{-0.007367in}}{\pgfqpoint{0.027778in}{0.000000in}}%
\pgfpathcurveto{\pgfqpoint{0.027778in}{0.007367in}}{\pgfqpoint{0.024851in}{0.014433in}}{\pgfqpoint{0.019642in}{0.019642in}}%
\pgfpathcurveto{\pgfqpoint{0.014433in}{0.024851in}}{\pgfqpoint{0.007367in}{0.027778in}}{\pgfqpoint{0.000000in}{0.027778in}}%
\pgfpathcurveto{\pgfqpoint{-0.007367in}{0.027778in}}{\pgfqpoint{-0.014433in}{0.024851in}}{\pgfqpoint{-0.019642in}{0.019642in}}%
\pgfpathcurveto{\pgfqpoint{-0.024851in}{0.014433in}}{\pgfqpoint{-0.027778in}{0.007367in}}{\pgfqpoint{-0.027778in}{0.000000in}}%
\pgfpathcurveto{\pgfqpoint{-0.027778in}{-0.007367in}}{\pgfqpoint{-0.024851in}{-0.014433in}}{\pgfqpoint{-0.019642in}{-0.019642in}}%
\pgfpathcurveto{\pgfqpoint{-0.014433in}{-0.024851in}}{\pgfqpoint{-0.007367in}{-0.027778in}}{\pgfqpoint{0.000000in}{-0.027778in}}%
\pgfpathlineto{\pgfqpoint{0.000000in}{-0.027778in}}%
\pgfpathclose%
\pgfusepath{stroke,fill}%
}%
\begin{pgfscope}%
\pgfsys@transformshift{0.833141in}{0.706710in}%
\pgfsys@useobject{currentmarker}{}%
\end{pgfscope}%
\begin{pgfscope}%
\pgfsys@transformshift{1.097611in}{0.706710in}%
\pgfsys@useobject{currentmarker}{}%
\end{pgfscope}%
\begin{pgfscope}%
\pgfsys@transformshift{1.362080in}{0.706710in}%
\pgfsys@useobject{currentmarker}{}%
\end{pgfscope}%
\begin{pgfscope}%
\pgfsys@transformshift{1.626550in}{0.706710in}%
\pgfsys@useobject{currentmarker}{}%
\end{pgfscope}%
\begin{pgfscope}%
\pgfsys@transformshift{1.891020in}{0.706710in}%
\pgfsys@useobject{currentmarker}{}%
\end{pgfscope}%
\begin{pgfscope}%
\pgfsys@transformshift{2.155489in}{0.706710in}%
\pgfsys@useobject{currentmarker}{}%
\end{pgfscope}%
\begin{pgfscope}%
\pgfsys@transformshift{2.419959in}{0.706710in}%
\pgfsys@useobject{currentmarker}{}%
\end{pgfscope}%
\begin{pgfscope}%
\pgfsys@transformshift{2.684429in}{0.706710in}%
\pgfsys@useobject{currentmarker}{}%
\end{pgfscope}%
\begin{pgfscope}%
\pgfsys@transformshift{2.948899in}{0.706710in}%
\pgfsys@useobject{currentmarker}{}%
\end{pgfscope}%
\begin{pgfscope}%
\pgfsys@transformshift{3.213368in}{0.706710in}%
\pgfsys@useobject{currentmarker}{}%
\end{pgfscope}%
\end{pgfscope}%
\begin{pgfscope}%
\pgfpathrectangle{\pgfqpoint{0.568671in}{0.451277in}}{\pgfqpoint{2.644697in}{1.137880in}}%
\pgfusepath{clip}%
\pgfsetrectcap%
\pgfsetroundjoin%
\pgfsetlinewidth{1.505625pt}%
\definecolor{currentstroke}{rgb}{1.000000,0.498039,0.054902}%
\pgfsetstrokecolor{currentstroke}%
\pgfsetdash{}{0pt}%
\pgfpathmoveto{\pgfqpoint{0.662834in}{1.603046in}}%
\pgfpathlineto{\pgfqpoint{0.700906in}{1.247141in}}%
\pgfpathlineto{\pgfqpoint{0.833141in}{0.974550in}}%
\pgfpathlineto{\pgfqpoint{0.965376in}{0.903573in}}%
\pgfpathlineto{\pgfqpoint{1.097611in}{0.872147in}}%
\pgfpathlineto{\pgfqpoint{1.229845in}{0.874937in}}%
\pgfpathlineto{\pgfqpoint{1.362080in}{0.895638in}}%
\pgfpathlineto{\pgfqpoint{1.494315in}{0.909955in}}%
\pgfpathlineto{\pgfqpoint{1.626550in}{0.926984in}}%
\pgfpathlineto{\pgfqpoint{1.758785in}{0.954840in}}%
\pgfpathlineto{\pgfqpoint{1.891020in}{0.977520in}}%
\pgfpathlineto{\pgfqpoint{2.023255in}{1.005576in}}%
\pgfpathlineto{\pgfqpoint{2.155489in}{1.028371in}}%
\pgfpathlineto{\pgfqpoint{2.287724in}{1.056243in}}%
\pgfpathlineto{\pgfqpoint{2.419959in}{1.079508in}}%
\pgfpathlineto{\pgfqpoint{2.552194in}{1.095151in}}%
\pgfpathlineto{\pgfqpoint{2.684429in}{1.118413in}}%
\pgfpathlineto{\pgfqpoint{2.816664in}{1.135753in}}%
\pgfpathlineto{\pgfqpoint{2.948899in}{1.155068in}}%
\pgfpathlineto{\pgfqpoint{3.081133in}{1.176282in}}%
\pgfpathlineto{\pgfqpoint{3.213368in}{1.194313in}}%
\pgfusepath{stroke}%
\end{pgfscope}%
\begin{pgfscope}%
\pgfpathrectangle{\pgfqpoint{0.568671in}{0.451277in}}{\pgfqpoint{2.644697in}{1.137880in}}%
\pgfusepath{clip}%
\pgfsetbuttcap%
\pgfsetmiterjoin%
\definecolor{currentfill}{rgb}{0.000000,0.000000,0.000000}%
\pgfsetfillcolor{currentfill}%
\pgfsetfillopacity{0.000000}%
\pgfsetlinewidth{1.003750pt}%
\definecolor{currentstroke}{rgb}{1.000000,0.498039,0.054902}%
\pgfsetstrokecolor{currentstroke}%
\pgfsetdash{}{0pt}%
\pgfsys@defobject{currentmarker}{\pgfqpoint{-0.027778in}{-0.027778in}}{\pgfqpoint{0.027778in}{0.027778in}}{%
\pgfpathmoveto{\pgfqpoint{-0.027778in}{-0.027778in}}%
\pgfpathlineto{\pgfqpoint{0.027778in}{-0.027778in}}%
\pgfpathlineto{\pgfqpoint{0.027778in}{0.027778in}}%
\pgfpathlineto{\pgfqpoint{-0.027778in}{0.027778in}}%
\pgfpathlineto{\pgfqpoint{-0.027778in}{-0.027778in}}%
\pgfpathclose%
\pgfusepath{stroke,fill}%
}%
\begin{pgfscope}%
\pgfsys@transformshift{0.700906in}{1.247141in}%
\pgfsys@useobject{currentmarker}{}%
\end{pgfscope}%
\begin{pgfscope}%
\pgfsys@transformshift{0.965376in}{0.903573in}%
\pgfsys@useobject{currentmarker}{}%
\end{pgfscope}%
\begin{pgfscope}%
\pgfsys@transformshift{1.229845in}{0.874937in}%
\pgfsys@useobject{currentmarker}{}%
\end{pgfscope}%
\begin{pgfscope}%
\pgfsys@transformshift{1.494315in}{0.909955in}%
\pgfsys@useobject{currentmarker}{}%
\end{pgfscope}%
\begin{pgfscope}%
\pgfsys@transformshift{1.758785in}{0.954840in}%
\pgfsys@useobject{currentmarker}{}%
\end{pgfscope}%
\begin{pgfscope}%
\pgfsys@transformshift{2.023255in}{1.005576in}%
\pgfsys@useobject{currentmarker}{}%
\end{pgfscope}%
\begin{pgfscope}%
\pgfsys@transformshift{2.287724in}{1.056243in}%
\pgfsys@useobject{currentmarker}{}%
\end{pgfscope}%
\begin{pgfscope}%
\pgfsys@transformshift{2.552194in}{1.095151in}%
\pgfsys@useobject{currentmarker}{}%
\end{pgfscope}%
\begin{pgfscope}%
\pgfsys@transformshift{2.816664in}{1.135753in}%
\pgfsys@useobject{currentmarker}{}%
\end{pgfscope}%
\begin{pgfscope}%
\pgfsys@transformshift{3.081133in}{1.176282in}%
\pgfsys@useobject{currentmarker}{}%
\end{pgfscope}%
\end{pgfscope}%
\begin{pgfscope}%
\pgfpathrectangle{\pgfqpoint{0.568671in}{0.451277in}}{\pgfqpoint{2.644697in}{1.137880in}}%
\pgfusepath{clip}%
\pgfsetrectcap%
\pgfsetroundjoin%
\pgfsetlinewidth{1.505625pt}%
\definecolor{currentstroke}{rgb}{0.172549,0.627451,0.172549}%
\pgfsetstrokecolor{currentstroke}%
\pgfsetdash{}{0pt}%
\pgfpathmoveto{\pgfqpoint{0.667104in}{1.603046in}}%
\pgfpathlineto{\pgfqpoint{0.700906in}{1.300764in}}%
\pgfpathlineto{\pgfqpoint{0.833141in}{0.976446in}}%
\pgfpathlineto{\pgfqpoint{0.965376in}{0.837688in}}%
\pgfpathlineto{\pgfqpoint{1.097611in}{0.829392in}}%
\pgfpathlineto{\pgfqpoint{1.229845in}{0.809837in}}%
\pgfpathlineto{\pgfqpoint{1.362080in}{0.813352in}}%
\pgfpathlineto{\pgfqpoint{1.494315in}{0.829162in}}%
\pgfpathlineto{\pgfqpoint{1.626550in}{0.829421in}}%
\pgfpathlineto{\pgfqpoint{1.758785in}{0.853237in}}%
\pgfpathlineto{\pgfqpoint{1.891020in}{0.862738in}}%
\pgfpathlineto{\pgfqpoint{2.023255in}{0.876906in}}%
\pgfpathlineto{\pgfqpoint{2.155489in}{0.877495in}}%
\pgfpathlineto{\pgfqpoint{2.287724in}{0.884390in}}%
\pgfpathlineto{\pgfqpoint{2.419959in}{0.896935in}}%
\pgfpathlineto{\pgfqpoint{2.552194in}{0.901086in}}%
\pgfpathlineto{\pgfqpoint{2.684429in}{0.915076in}}%
\pgfpathlineto{\pgfqpoint{2.816664in}{0.922549in}}%
\pgfpathlineto{\pgfqpoint{2.948899in}{0.930926in}}%
\pgfpathlineto{\pgfqpoint{3.081133in}{0.943561in}}%
\pgfpathlineto{\pgfqpoint{3.213368in}{0.957909in}}%
\pgfusepath{stroke}%
\end{pgfscope}%
\begin{pgfscope}%
\pgfpathrectangle{\pgfqpoint{0.568671in}{0.451277in}}{\pgfqpoint{2.644697in}{1.137880in}}%
\pgfusepath{clip}%
\pgfsetbuttcap%
\pgfsetroundjoin%
\definecolor{currentfill}{rgb}{0.000000,0.000000,0.000000}%
\pgfsetfillcolor{currentfill}%
\pgfsetfillopacity{0.000000}%
\pgfsetlinewidth{1.003750pt}%
\definecolor{currentstroke}{rgb}{0.172549,0.627451,0.172549}%
\pgfsetstrokecolor{currentstroke}%
\pgfsetdash{}{0pt}%
\pgfsys@defobject{currentmarker}{\pgfqpoint{-0.027778in}{-0.027778in}}{\pgfqpoint{0.027778in}{0.027778in}}{%
\pgfpathmoveto{\pgfqpoint{-0.027778in}{-0.027778in}}%
\pgfpathlineto{\pgfqpoint{0.027778in}{0.027778in}}%
\pgfpathmoveto{\pgfqpoint{-0.027778in}{0.027778in}}%
\pgfpathlineto{\pgfqpoint{0.027778in}{-0.027778in}}%
\pgfusepath{stroke,fill}%
}%
\begin{pgfscope}%
\pgfsys@transformshift{0.833141in}{0.976446in}%
\pgfsys@useobject{currentmarker}{}%
\end{pgfscope}%
\begin{pgfscope}%
\pgfsys@transformshift{1.097611in}{0.829392in}%
\pgfsys@useobject{currentmarker}{}%
\end{pgfscope}%
\begin{pgfscope}%
\pgfsys@transformshift{1.362080in}{0.813352in}%
\pgfsys@useobject{currentmarker}{}%
\end{pgfscope}%
\begin{pgfscope}%
\pgfsys@transformshift{1.626550in}{0.829421in}%
\pgfsys@useobject{currentmarker}{}%
\end{pgfscope}%
\begin{pgfscope}%
\pgfsys@transformshift{1.891020in}{0.862738in}%
\pgfsys@useobject{currentmarker}{}%
\end{pgfscope}%
\begin{pgfscope}%
\pgfsys@transformshift{2.155489in}{0.877495in}%
\pgfsys@useobject{currentmarker}{}%
\end{pgfscope}%
\begin{pgfscope}%
\pgfsys@transformshift{2.419959in}{0.896935in}%
\pgfsys@useobject{currentmarker}{}%
\end{pgfscope}%
\begin{pgfscope}%
\pgfsys@transformshift{2.684429in}{0.915076in}%
\pgfsys@useobject{currentmarker}{}%
\end{pgfscope}%
\begin{pgfscope}%
\pgfsys@transformshift{2.948899in}{0.930926in}%
\pgfsys@useobject{currentmarker}{}%
\end{pgfscope}%
\begin{pgfscope}%
\pgfsys@transformshift{3.213368in}{0.957909in}%
\pgfsys@useobject{currentmarker}{}%
\end{pgfscope}%
\end{pgfscope}%
\begin{pgfscope}%
\pgfpathrectangle{\pgfqpoint{0.568671in}{0.451277in}}{\pgfqpoint{2.644697in}{1.137880in}}%
\pgfusepath{clip}%
\pgfsetrectcap%
\pgfsetroundjoin%
\pgfsetlinewidth{1.505625pt}%
\definecolor{currentstroke}{rgb}{0.839216,0.152941,0.156863}%
\pgfsetstrokecolor{currentstroke}%
\pgfsetdash{}{0pt}%
\pgfpathmoveto{\pgfqpoint{0.661733in}{1.603046in}}%
\pgfpathlineto{\pgfqpoint{0.700906in}{1.232520in}}%
\pgfpathlineto{\pgfqpoint{0.833141in}{0.912159in}}%
\pgfpathlineto{\pgfqpoint{0.965376in}{0.808965in}}%
\pgfpathlineto{\pgfqpoint{1.097611in}{0.741955in}}%
\pgfpathlineto{\pgfqpoint{1.229845in}{0.717041in}}%
\pgfpathlineto{\pgfqpoint{1.362080in}{0.719570in}}%
\pgfpathlineto{\pgfqpoint{1.494315in}{0.715205in}}%
\pgfpathlineto{\pgfqpoint{1.626550in}{0.725610in}}%
\pgfpathlineto{\pgfqpoint{1.758785in}{0.732946in}}%
\pgfpathlineto{\pgfqpoint{1.891020in}{0.727665in}}%
\pgfpathlineto{\pgfqpoint{2.023255in}{0.734933in}}%
\pgfpathlineto{\pgfqpoint{2.155489in}{0.742924in}}%
\pgfpathlineto{\pgfqpoint{2.287724in}{0.748680in}}%
\pgfpathlineto{\pgfqpoint{2.419959in}{0.753314in}}%
\pgfpathlineto{\pgfqpoint{2.552194in}{0.756300in}}%
\pgfpathlineto{\pgfqpoint{2.684429in}{0.759125in}}%
\pgfpathlineto{\pgfqpoint{2.816664in}{0.760109in}}%
\pgfpathlineto{\pgfqpoint{2.948899in}{0.765494in}}%
\pgfpathlineto{\pgfqpoint{3.081133in}{0.767609in}}%
\pgfpathlineto{\pgfqpoint{3.213368in}{0.770941in}}%
\pgfusepath{stroke}%
\end{pgfscope}%
\begin{pgfscope}%
\pgfpathrectangle{\pgfqpoint{0.568671in}{0.451277in}}{\pgfqpoint{2.644697in}{1.137880in}}%
\pgfusepath{clip}%
\pgfsetbuttcap%
\pgfsetroundjoin%
\definecolor{currentfill}{rgb}{0.000000,0.000000,0.000000}%
\pgfsetfillcolor{currentfill}%
\pgfsetfillopacity{0.000000}%
\pgfsetlinewidth{1.003750pt}%
\definecolor{currentstroke}{rgb}{0.839216,0.152941,0.156863}%
\pgfsetstrokecolor{currentstroke}%
\pgfsetdash{}{0pt}%
\pgfsys@defobject{currentmarker}{\pgfqpoint{-0.027778in}{-0.027778in}}{\pgfqpoint{0.027778in}{0.027778in}}{%
\pgfpathmoveto{\pgfqpoint{-0.027778in}{0.000000in}}%
\pgfpathlineto{\pgfqpoint{0.027778in}{0.000000in}}%
\pgfpathmoveto{\pgfqpoint{0.000000in}{-0.027778in}}%
\pgfpathlineto{\pgfqpoint{0.000000in}{0.027778in}}%
\pgfusepath{stroke,fill}%
}%
\begin{pgfscope}%
\pgfsys@transformshift{0.700906in}{1.232520in}%
\pgfsys@useobject{currentmarker}{}%
\end{pgfscope}%
\begin{pgfscope}%
\pgfsys@transformshift{0.965376in}{0.808965in}%
\pgfsys@useobject{currentmarker}{}%
\end{pgfscope}%
\begin{pgfscope}%
\pgfsys@transformshift{1.229845in}{0.717041in}%
\pgfsys@useobject{currentmarker}{}%
\end{pgfscope}%
\begin{pgfscope}%
\pgfsys@transformshift{1.494315in}{0.715205in}%
\pgfsys@useobject{currentmarker}{}%
\end{pgfscope}%
\begin{pgfscope}%
\pgfsys@transformshift{1.758785in}{0.732946in}%
\pgfsys@useobject{currentmarker}{}%
\end{pgfscope}%
\begin{pgfscope}%
\pgfsys@transformshift{2.023255in}{0.734933in}%
\pgfsys@useobject{currentmarker}{}%
\end{pgfscope}%
\begin{pgfscope}%
\pgfsys@transformshift{2.287724in}{0.748680in}%
\pgfsys@useobject{currentmarker}{}%
\end{pgfscope}%
\begin{pgfscope}%
\pgfsys@transformshift{2.552194in}{0.756300in}%
\pgfsys@useobject{currentmarker}{}%
\end{pgfscope}%
\begin{pgfscope}%
\pgfsys@transformshift{2.816664in}{0.760109in}%
\pgfsys@useobject{currentmarker}{}%
\end{pgfscope}%
\begin{pgfscope}%
\pgfsys@transformshift{3.081133in}{0.767609in}%
\pgfsys@useobject{currentmarker}{}%
\end{pgfscope}%
\end{pgfscope}%
\begin{pgfscope}%
\pgfsetrectcap%
\pgfsetmiterjoin%
\pgfsetlinewidth{0.803000pt}%
\definecolor{currentstroke}{rgb}{0.000000,0.000000,0.000000}%
\pgfsetstrokecolor{currentstroke}%
\pgfsetdash{}{0pt}%
\pgfpathmoveto{\pgfqpoint{0.568671in}{0.451277in}}%
\pgfpathlineto{\pgfqpoint{0.568671in}{1.589158in}}%
\pgfusepath{stroke}%
\end{pgfscope}%
\begin{pgfscope}%
\pgfsetrectcap%
\pgfsetmiterjoin%
\pgfsetlinewidth{0.803000pt}%
\definecolor{currentstroke}{rgb}{0.000000,0.000000,0.000000}%
\pgfsetstrokecolor{currentstroke}%
\pgfsetdash{}{0pt}%
\pgfpathmoveto{\pgfqpoint{3.213368in}{0.451277in}}%
\pgfpathlineto{\pgfqpoint{3.213368in}{1.589158in}}%
\pgfusepath{stroke}%
\end{pgfscope}%
\begin{pgfscope}%
\pgfsetrectcap%
\pgfsetmiterjoin%
\pgfsetlinewidth{0.803000pt}%
\definecolor{currentstroke}{rgb}{0.000000,0.000000,0.000000}%
\pgfsetstrokecolor{currentstroke}%
\pgfsetdash{}{0pt}%
\pgfpathmoveto{\pgfqpoint{0.568671in}{0.451277in}}%
\pgfpathlineto{\pgfqpoint{3.213368in}{0.451277in}}%
\pgfusepath{stroke}%
\end{pgfscope}%
\begin{pgfscope}%
\pgfsetrectcap%
\pgfsetmiterjoin%
\pgfsetlinewidth{0.803000pt}%
\definecolor{currentstroke}{rgb}{0.000000,0.000000,0.000000}%
\pgfsetstrokecolor{currentstroke}%
\pgfsetdash{}{0pt}%
\pgfpathmoveto{\pgfqpoint{0.568671in}{1.589158in}}%
\pgfpathlineto{\pgfqpoint{3.213368in}{1.589158in}}%
\pgfusepath{stroke}%
\end{pgfscope}%
\begin{pgfscope}%
\pgfsetbuttcap%
\pgfsetmiterjoin%
\definecolor{currentfill}{rgb}{1.000000,1.000000,1.000000}%
\pgfsetfillcolor{currentfill}%
\pgfsetfillopacity{0.800000}%
\pgfsetlinewidth{1.003750pt}%
\definecolor{currentstroke}{rgb}{0.800000,0.800000,0.800000}%
\pgfsetstrokecolor{currentstroke}%
\pgfsetstrokeopacity{0.800000}%
\pgfsetdash{}{0pt}%
\pgfpathmoveto{\pgfqpoint{0.783863in}{1.103933in}}%
\pgfpathlineto{\pgfqpoint{3.145313in}{1.103933in}}%
\pgfpathquadraticcurveto{\pgfqpoint{3.164757in}{1.103933in}}{\pgfqpoint{3.164757in}{1.123377in}}%
\pgfpathlineto{\pgfqpoint{3.164757in}{1.521102in}}%
\pgfpathquadraticcurveto{\pgfqpoint{3.164757in}{1.540546in}}{\pgfqpoint{3.145313in}{1.540546in}}%
\pgfpathlineto{\pgfqpoint{0.783863in}{1.540546in}}%
\pgfpathquadraticcurveto{\pgfqpoint{0.764419in}{1.540546in}}{\pgfqpoint{0.764419in}{1.521102in}}%
\pgfpathlineto{\pgfqpoint{0.764419in}{1.123377in}}%
\pgfpathquadraticcurveto{\pgfqpoint{0.764419in}{1.103933in}}{\pgfqpoint{0.783863in}{1.103933in}}%
\pgfpathlineto{\pgfqpoint{0.783863in}{1.103933in}}%
\pgfpathclose%
\pgfusepath{stroke,fill}%
\end{pgfscope}%
\begin{pgfscope}%
\pgfsetrectcap%
\pgfsetroundjoin%
\pgfsetlinewidth{1.505625pt}%
\definecolor{currentstroke}{rgb}{0.000000,0.000000,0.000000}%
\pgfsetstrokecolor{currentstroke}%
\pgfsetdash{}{0pt}%
\pgfpathmoveto{\pgfqpoint{0.803308in}{1.467241in}}%
\pgfpathlineto{\pgfqpoint{0.900530in}{1.467241in}}%
\pgfpathlineto{\pgfqpoint{0.997752in}{1.467241in}}%
\pgfusepath{stroke}%
\end{pgfscope}%
\begin{pgfscope}%
\definecolor{textcolor}{rgb}{0.000000,0.000000,0.000000}%
\pgfsetstrokecolor{textcolor}%
\pgfsetfillcolor{textcolor}%
\pgftext[x=1.075530in,y=1.433213in,left,base]{\color{textcolor}\rmfamily\fontsize{7.000000}{8.400000}\selectfont optimal}%
\end{pgfscope}%
\begin{pgfscope}%
\pgfsetrectcap%
\pgfsetroundjoin%
\pgfsetlinewidth{1.505625pt}%
\definecolor{currentstroke}{rgb}{0.121569,0.466667,0.705882}%
\pgfsetstrokecolor{currentstroke}%
\pgfsetdash{}{0pt}%
\pgfpathmoveto{\pgfqpoint{0.803308in}{1.331713in}}%
\pgfpathlineto{\pgfqpoint{0.900530in}{1.331713in}}%
\pgfpathlineto{\pgfqpoint{0.997752in}{1.331713in}}%
\pgfusepath{stroke}%
\end{pgfscope}%
\begin{pgfscope}%
\pgfsetbuttcap%
\pgfsetroundjoin%
\definecolor{currentfill}{rgb}{0.000000,0.000000,0.000000}%
\pgfsetfillcolor{currentfill}%
\pgfsetfillopacity{0.000000}%
\pgfsetlinewidth{1.003750pt}%
\definecolor{currentstroke}{rgb}{0.121569,0.466667,0.705882}%
\pgfsetstrokecolor{currentstroke}%
\pgfsetdash{}{0pt}%
\pgfsys@defobject{currentmarker}{\pgfqpoint{-0.027778in}{-0.027778in}}{\pgfqpoint{0.027778in}{0.027778in}}{%
\pgfpathmoveto{\pgfqpoint{0.000000in}{-0.027778in}}%
\pgfpathcurveto{\pgfqpoint{0.007367in}{-0.027778in}}{\pgfqpoint{0.014433in}{-0.024851in}}{\pgfqpoint{0.019642in}{-0.019642in}}%
\pgfpathcurveto{\pgfqpoint{0.024851in}{-0.014433in}}{\pgfqpoint{0.027778in}{-0.007367in}}{\pgfqpoint{0.027778in}{0.000000in}}%
\pgfpathcurveto{\pgfqpoint{0.027778in}{0.007367in}}{\pgfqpoint{0.024851in}{0.014433in}}{\pgfqpoint{0.019642in}{0.019642in}}%
\pgfpathcurveto{\pgfqpoint{0.014433in}{0.024851in}}{\pgfqpoint{0.007367in}{0.027778in}}{\pgfqpoint{0.000000in}{0.027778in}}%
\pgfpathcurveto{\pgfqpoint{-0.007367in}{0.027778in}}{\pgfqpoint{-0.014433in}{0.024851in}}{\pgfqpoint{-0.019642in}{0.019642in}}%
\pgfpathcurveto{\pgfqpoint{-0.024851in}{0.014433in}}{\pgfqpoint{-0.027778in}{0.007367in}}{\pgfqpoint{-0.027778in}{0.000000in}}%
\pgfpathcurveto{\pgfqpoint{-0.027778in}{-0.007367in}}{\pgfqpoint{-0.024851in}{-0.014433in}}{\pgfqpoint{-0.019642in}{-0.019642in}}%
\pgfpathcurveto{\pgfqpoint{-0.014433in}{-0.024851in}}{\pgfqpoint{-0.007367in}{-0.027778in}}{\pgfqpoint{0.000000in}{-0.027778in}}%
\pgfpathlineto{\pgfqpoint{0.000000in}{-0.027778in}}%
\pgfpathclose%
\pgfusepath{stroke,fill}%
}%
\begin{pgfscope}%
\pgfsys@transformshift{0.900530in}{1.331713in}%
\pgfsys@useobject{currentmarker}{}%
\end{pgfscope}%
\end{pgfscope}%
\begin{pgfscope}%
\definecolor{textcolor}{rgb}{0.000000,0.000000,0.000000}%
\pgfsetstrokecolor{textcolor}%
\pgfsetfillcolor{textcolor}%
\pgftext[x=1.075530in,y=1.297685in,left,base]{\color{textcolor}\rmfamily\fontsize{7.000000}{8.400000}\selectfont adaptive \(\displaystyle \gamma_i,\,K=1\)}%
\end{pgfscope}%
\begin{pgfscope}%
\pgfsetrectcap%
\pgfsetroundjoin%
\pgfsetlinewidth{1.505625pt}%
\definecolor{currentstroke}{rgb}{1.000000,0.498039,0.054902}%
\pgfsetstrokecolor{currentstroke}%
\pgfsetdash{}{0pt}%
\pgfpathmoveto{\pgfqpoint{0.803308in}{1.195753in}}%
\pgfpathlineto{\pgfqpoint{0.900530in}{1.195753in}}%
\pgfpathlineto{\pgfqpoint{0.997752in}{1.195753in}}%
\pgfusepath{stroke}%
\end{pgfscope}%
\begin{pgfscope}%
\pgfsetbuttcap%
\pgfsetmiterjoin%
\definecolor{currentfill}{rgb}{0.000000,0.000000,0.000000}%
\pgfsetfillcolor{currentfill}%
\pgfsetfillopacity{0.000000}%
\pgfsetlinewidth{1.003750pt}%
\definecolor{currentstroke}{rgb}{1.000000,0.498039,0.054902}%
\pgfsetstrokecolor{currentstroke}%
\pgfsetdash{}{0pt}%
\pgfsys@defobject{currentmarker}{\pgfqpoint{-0.027778in}{-0.027778in}}{\pgfqpoint{0.027778in}{0.027778in}}{%
\pgfpathmoveto{\pgfqpoint{-0.027778in}{-0.027778in}}%
\pgfpathlineto{\pgfqpoint{0.027778in}{-0.027778in}}%
\pgfpathlineto{\pgfqpoint{0.027778in}{0.027778in}}%
\pgfpathlineto{\pgfqpoint{-0.027778in}{0.027778in}}%
\pgfpathlineto{\pgfqpoint{-0.027778in}{-0.027778in}}%
\pgfpathclose%
\pgfusepath{stroke,fill}%
}%
\begin{pgfscope}%
\pgfsys@transformshift{0.900530in}{1.195753in}%
\pgfsys@useobject{currentmarker}{}%
\end{pgfscope}%
\end{pgfscope}%
\begin{pgfscope}%
\definecolor{textcolor}{rgb}{0.000000,0.000000,0.000000}%
\pgfsetstrokecolor{textcolor}%
\pgfsetfillcolor{textcolor}%
\pgftext[x=1.075530in,y=1.161726in,left,base]{\color{textcolor}\rmfamily\fontsize{7.000000}{8.400000}\selectfont fixed \(\displaystyle \gamma_i,\,K=1\)}%
\end{pgfscope}%
\begin{pgfscope}%
\pgfsetrectcap%
\pgfsetroundjoin%
\pgfsetlinewidth{1.505625pt}%
\definecolor{currentstroke}{rgb}{0.172549,0.627451,0.172549}%
\pgfsetstrokecolor{currentstroke}%
\pgfsetdash{}{0pt}%
\pgfpathmoveto{\pgfqpoint{2.051629in}{1.467241in}}%
\pgfpathlineto{\pgfqpoint{2.148851in}{1.467241in}}%
\pgfpathlineto{\pgfqpoint{2.246073in}{1.467241in}}%
\pgfusepath{stroke}%
\end{pgfscope}%
\begin{pgfscope}%
\pgfsetbuttcap%
\pgfsetroundjoin%
\definecolor{currentfill}{rgb}{0.000000,0.000000,0.000000}%
\pgfsetfillcolor{currentfill}%
\pgfsetfillopacity{0.000000}%
\pgfsetlinewidth{1.003750pt}%
\definecolor{currentstroke}{rgb}{0.172549,0.627451,0.172549}%
\pgfsetstrokecolor{currentstroke}%
\pgfsetdash{}{0pt}%
\pgfsys@defobject{currentmarker}{\pgfqpoint{-0.027778in}{-0.027778in}}{\pgfqpoint{0.027778in}{0.027778in}}{%
\pgfpathmoveto{\pgfqpoint{-0.027778in}{-0.027778in}}%
\pgfpathlineto{\pgfqpoint{0.027778in}{0.027778in}}%
\pgfpathmoveto{\pgfqpoint{-0.027778in}{0.027778in}}%
\pgfpathlineto{\pgfqpoint{0.027778in}{-0.027778in}}%
\pgfusepath{stroke,fill}%
}%
\begin{pgfscope}%
\pgfsys@transformshift{2.148851in}{1.467241in}%
\pgfsys@useobject{currentmarker}{}%
\end{pgfscope}%
\end{pgfscope}%
\begin{pgfscope}%
\definecolor{textcolor}{rgb}{0.000000,0.000000,0.000000}%
\pgfsetstrokecolor{textcolor}%
\pgfsetfillcolor{textcolor}%
\pgftext[x=2.323851in,y=1.433213in,left,base]{\color{textcolor}\rmfamily\fontsize{7.000000}{8.400000}\selectfont fixed \(\displaystyle \gamma_i,\,K=2\)}%
\end{pgfscope}%
\begin{pgfscope}%
\pgfsetrectcap%
\pgfsetroundjoin%
\pgfsetlinewidth{1.505625pt}%
\definecolor{currentstroke}{rgb}{0.839216,0.152941,0.156863}%
\pgfsetstrokecolor{currentstroke}%
\pgfsetdash{}{0pt}%
\pgfpathmoveto{\pgfqpoint{2.051629in}{1.331281in}}%
\pgfpathlineto{\pgfqpoint{2.148851in}{1.331281in}}%
\pgfpathlineto{\pgfqpoint{2.246073in}{1.331281in}}%
\pgfusepath{stroke}%
\end{pgfscope}%
\begin{pgfscope}%
\pgfsetbuttcap%
\pgfsetroundjoin%
\definecolor{currentfill}{rgb}{0.000000,0.000000,0.000000}%
\pgfsetfillcolor{currentfill}%
\pgfsetfillopacity{0.000000}%
\pgfsetlinewidth{1.003750pt}%
\definecolor{currentstroke}{rgb}{0.839216,0.152941,0.156863}%
\pgfsetstrokecolor{currentstroke}%
\pgfsetdash{}{0pt}%
\pgfsys@defobject{currentmarker}{\pgfqpoint{-0.027778in}{-0.027778in}}{\pgfqpoint{0.027778in}{0.027778in}}{%
\pgfpathmoveto{\pgfqpoint{-0.027778in}{0.000000in}}%
\pgfpathlineto{\pgfqpoint{0.027778in}{0.000000in}}%
\pgfpathmoveto{\pgfqpoint{0.000000in}{-0.027778in}}%
\pgfpathlineto{\pgfqpoint{0.000000in}{0.027778in}}%
\pgfusepath{stroke,fill}%
}%
\begin{pgfscope}%
\pgfsys@transformshift{2.148851in}{1.331281in}%
\pgfsys@useobject{currentmarker}{}%
\end{pgfscope}%
\end{pgfscope}%
\begin{pgfscope}%
\definecolor{textcolor}{rgb}{0.000000,0.000000,0.000000}%
\pgfsetstrokecolor{textcolor}%
\pgfsetfillcolor{textcolor}%
\pgftext[x=2.323851in,y=1.297253in,left,base]{\color{textcolor}\rmfamily\fontsize{7.000000}{8.400000}\selectfont fixed \(\displaystyle \gamma_i,\,K=10\)}%
\end{pgfscope}%
\end{pgfpicture}%
\makeatother%
\endgroup%

    \vspace*{-0.6cm}
    \caption[]{Mean of 500 post-convergence frames of median NPM of 30 Monte-Carlo runs for fixed \(\gamma_i\) on the interval \([0.0, 0.04]\) and \(K \in \{1,2,10\}\). Results for optimal algorithm, and adaptive \(\gamma_i\) included for comparison.}
    \label{fig:simulations:avgNPMgamma}
\end{figure}
\subsection[]{Communication cost}
\label{sec:transcost}
To describe the communication cost within the network, we count the number of variables transmitted per time frame \(n\).
The most useful measure in the case of this work is to analyze the cost per node.
We compare the following communication schemes within the context of the algorithm \cite{blochbergerDBSI} this paper proposes the extension to:
\begin{itemize}
    \itemsep-0.2em
    \item[(1)] A fully-connected network, where node \(i\) communicates \(\|\bar{\h}_i\|^2\) to all other nodes \(\{j \in \Nset_i = \Mset\}\) directly, i.e., the neighborhood is the full network.
    \item[(2)] The node \(i\) communicates \(\|\bar{\h}_i\|^2\) only to neighboring nodes \(\{j \in \Nset_i \subset \Mset\}\) and applies \(K\) distributed averaging iterations.
\end{itemize}
\autoref{tab:transcost:table} compares the number of transmit and receive operations node \(i\) has to apply in order to compute \eqref{eq:online_admm:consensus_update}.
Further, it shows the complexity of the problem node \(i\) has to solve according to \cite{blochbergerDBSI} given the two types of network topology, assuming it uses all information it receives.
The advantage of the fully distributed algorithm becomes clear when \(M \gg N_i\), as the additional communication and local complexity depends on neighborhood size \(N_i\) and not on network size \(M\).
\vspace*{-0.6em}
\renewcommand{\arraystretch}{1.2}
\begin{table}[h]
    \centering
    \begin{tabular}{ |l|l|l|l| }
        \hline
        & Trans. Ops. & Rec. Ops. & Local Compl. \\
        \hline\hline
        (1) Fully con. & \(M-1\) & \(M-1\) & \(\mathcal{O}(M^2 L^2)\) \\
        \hline
        (2) Neighborh. & \(N_i K\) & \(N_i K\) & \(\mathcal{O}(N_i^2 L^2)\)\\ 
        \hline
    \end{tabular}
    \caption[]{Communication cost per node and frame \(n\) for computation of \eqref{eq:online_admm:consensus_update}.}
    \label{tab:transcost:table}
\end{table}
\renewcommand{\arraystretch}{1.0}

\subsection[]{Simulations}
To evaluate the effectiveness of the proposed extension, simulations were run, where we define the error measure as the normalized projection misalignment
\begin{equation}
    \begin{aligned}
        \text{NPM}(n) &= 20\,\log_{10} \frac{\left\| \mtxb{e} \right\|}{\left\|\hat{\h}^{(n)}\right\|},\\
        \text{with}\quad \mtxb{e} &= \hat{\h}^{(n)} - \frac{\h^\T \hat{\h}^{(n)}}{\h^\T \h}\hat{\h}^{(n)},
    \end{aligned}
\end{equation}
commonly used in BSI to compare the estimated \(\hat{\h}^{(n)}\) and true \(\h\).

The first simulation setup is a network with \(M=5\) nodes arranged in a ring topology, each node having \(N_i=3\) neighbors (including node \(i\)).
The input signal is zero-mean white Gaussian noise (WGN), the impulse responses of length \(L=16\) are drawn from a normal distribution, and at each channel, independent WGN is added with \(\text{SNR}=10\,\text{dB}\).
The norms of the impulse responses are scaled to random values drawn from the uniform distribution \(\mathcal{U}_{[0.5,2.0]}\).
A comparison is made between the following cases:
\begin{itemize}
    \itemsep-0.2em
    \item[(a)] \(\|\bar{\h}_i\|^2\) of all \(i \in \Mset\) is available for global norm computation (``optimal'' in \autoref{fig:simulations:NPMtime} \& \autoref{fig:simulations:avgNPMgamma}),
    \item[(b)] Inter-neighborhood communication with fixed \(\gamma_i \in [0.0, 0.4]\) and \(K \in \{1,2,10\}\),
    \item[(c)] Inter-neighborhood communication with adaptive \(\gamma_i\) \eqref{eq:adaptivenormest:adaptivegamma} and \(K=1\).
\end{itemize}
\begin{algorithm}[t]
    \caption{ADMM BSI with distributed-averaging-based adaptive estimation of norm values}\label{alg:davg_norm_est}
    \(\W \gets\) \eqref{eq:adaptivenormest:fdlaminprob}\;
    \(\eta_i^{(n-1)} \gets 1, \forall i \in \Mset\)\;
    \For(){\(n=0\dots\)}
    {
        \For(){\(i \in \Mset\)}
        {
            \emph{The steps before are as introduced in }\cite{blochbergerDBSI}\\
            \dotfill\\
            \(\eta_i^{(n)} \gets \sum_{j \in \Nset_i} W_{ij} \|\bar{\h}_i^{(n)}\|^2\)\;
            \(\gamma_i^{(n)} \gets \min \left\lbrace \frac{| \eta_i^{(n)} - \eta_i^{(n-1)} |}{\eta_i^{(n-1)}},\,1\right\rbrace\)\;
            \eIf(){\(n = 0\)}
            {
                \(\phi_i^{(0)}(0) \gets \|\bar{\h}_i^{(0)}\|^2\)\;
            }
            {
                \(\phi_i^{(n)}(0) \gets \gamma_i^{(n)} \eta_i^{(n)} + (1-\gamma_i^{(n)}) \phi_i^{(n-1)}(K)\)\;
            }
            \For(){\(k=0,\dots,K\)}
            {
                Transmit \(\phi_i^{(n)}(k)\) to nodes  \(j \in \Nset_i\)\;
                \(\phi_i^{(n)}({k+1}) \gets \sum_{j \in \Nset_i} W_{ij} \phi_j^{(n)}({k})\)\;
            }
            \(\hat{\h_i}^{(n)} \gets \frac{\bar{\h}_i^{(n)}}{\sqrt{M \phi_i^{(n)}(K)}}\)\;
            \dotfill\\
            \emph{The steps after are as introduced in }\cite{blochbergerDBSI}\\
        }
    }
\end{algorithm}
\autoref{fig:simulations:NPMtime} shows the median of 30 Monte-Carlo runs.
We can observe that when fixing \(\gamma_i\), a tradeoff between convergence speed and steady-state error arises at low iteration counts (cf. also \autoref{fig:simulations:avgNPMgamma}).
With \(K=10\), the performance comes close to the optimal case at the cost of additional communication between nodes.
This in contrast to the case with adaptive \(\gamma_i\) where even at \(K=1\), i.e., no additional communication for the recursive estimation scheme, convergence speed and steady-state error are close to the optimal case.

The second simulation setup is a \(3\)-node ring topology with a neighborhood size of \(N_i=2\) (including node \(i\)).
The input signal, impulse responses, additive noise, and SNR have the same parameters as the first simulation.
To test the algorithms response to time-varying scenarios, the 3 impulse responses are then scaled to norms of \([2.2, 0.5, 1.2],\, [2.2, 1.0, 1.2],\, [2.2, 0.5, 2.0]\) at frames \(0,\, 5000,\, 10000\) respectively.
\autoref{fig:simulations:NPMtimedyn} (bottom) shows the median NPM of 30 Monte-Carlo runs, where we can observe very similar behavior of the optimal and distributed-averaging-based algorithms, where both algorithms reach a converged state after the rescaling events.
It further shows the damped oscillations that appear in the estimates of \(\|\hat{\h}^{(n)}\|\) around the desired value 1 and \(\|\hat{\h}_i^{(n)}\|\) around the rescaled true norms in \autoref{fig:simulations:NPMtimedyn} (top), respectively.

In summary, the simulations show that the right choice of fixed \(\gamma_i\) leads to results close to an optimal case even with \(K=1\) distributed averaging iterations.
Further, introducing an adaptive \(\gamma_i^{(n)}\) allows the algorithm to converge faster and deal with time-variant systems.
This is an improvement over \cite{blochbergerDBSI} as communication is only required within neighborhoods and over \cite{yuDistributedBlindSystem2014, liuDistributedBlindIdentification2016} needing only \(K=1\) instead of \(K=50\) iterations per frame.

\begin{figure}[t]
    \centering
    % %% Creator: Matplotlib, PGF backend
%%
%% To include the figure in your LaTeX document, write
%%   \input{<filename>.pgf}
%%
%% Make sure the required packages are loaded in your preamble
%%   \usepackage{pgf}
%%
%% Also ensure that all the required font packages are loaded; for instance,
%% the lmodern package is sometimes necessary when using math font.
%%   \usepackage{lmodern}
%%
%% Figures using additional raster images can only be included by \input if
%% they are in the same directory as the main LaTeX file. For loading figures
%% from other directories you can use the `import` package
%%   \usepackage{import}
%%
%% and then include the figures with
%%   \import{<path to file>}{<filename>.pgf}
%%
%% Matplotlib used the following preamble
%%   \usepackage{fontspec}
%%
\begingroup%
\makeatletter%
\begin{pgfpicture}%
\pgfpathrectangle{\pgfpointorigin}{\pgfqpoint{3.390065in}{1.017020in}}%
\pgfusepath{use as bounding box, clip}%
\begin{pgfscope}%
\pgfsetbuttcap%
\pgfsetmiterjoin%
\definecolor{currentfill}{rgb}{1.000000,1.000000,1.000000}%
\pgfsetfillcolor{currentfill}%
\pgfsetlinewidth{0.000000pt}%
\definecolor{currentstroke}{rgb}{1.000000,1.000000,1.000000}%
\pgfsetstrokecolor{currentstroke}%
\pgfsetstrokeopacity{0.000000}%
\pgfsetdash{}{0pt}%
\pgfpathmoveto{\pgfqpoint{0.000000in}{0.000000in}}%
\pgfpathlineto{\pgfqpoint{3.390065in}{0.000000in}}%
\pgfpathlineto{\pgfqpoint{3.390065in}{1.017020in}}%
\pgfpathlineto{\pgfqpoint{0.000000in}{1.017020in}}%
\pgfpathlineto{\pgfqpoint{0.000000in}{0.000000in}}%
\pgfpathclose%
\pgfusepath{fill}%
\end{pgfscope}%
\begin{pgfscope}%
\pgfsetbuttcap%
\pgfsetmiterjoin%
\definecolor{currentfill}{rgb}{1.000000,1.000000,1.000000}%
\pgfsetfillcolor{currentfill}%
\pgfsetlinewidth{0.000000pt}%
\definecolor{currentstroke}{rgb}{0.000000,0.000000,0.000000}%
\pgfsetstrokecolor{currentstroke}%
\pgfsetstrokeopacity{0.000000}%
\pgfsetdash{}{0pt}%
\pgfpathmoveto{\pgfqpoint{0.568671in}{0.270766in}}%
\pgfpathlineto{\pgfqpoint{3.166976in}{0.270766in}}%
\pgfpathlineto{\pgfqpoint{3.166976in}{0.680403in}}%
\pgfpathlineto{\pgfqpoint{0.568671in}{0.680403in}}%
\pgfpathlineto{\pgfqpoint{0.568671in}{0.270766in}}%
\pgfpathclose%
\pgfusepath{fill}%
\end{pgfscope}%
\begin{pgfscope}%
\pgfpathrectangle{\pgfqpoint{0.568671in}{0.270766in}}{\pgfqpoint{2.598305in}{0.409637in}}%
\pgfusepath{clip}%
\pgfsetrectcap%
\pgfsetroundjoin%
\pgfsetlinewidth{0.803000pt}%
\definecolor{currentstroke}{rgb}{0.690196,0.690196,0.690196}%
\pgfsetstrokecolor{currentstroke}%
\pgfsetdash{}{0pt}%
\pgfpathmoveto{\pgfqpoint{0.568671in}{0.270766in}}%
\pgfpathlineto{\pgfqpoint{0.568671in}{0.680403in}}%
\pgfusepath{stroke}%
\end{pgfscope}%
\begin{pgfscope}%
\pgfsetbuttcap%
\pgfsetroundjoin%
\definecolor{currentfill}{rgb}{0.000000,0.000000,0.000000}%
\pgfsetfillcolor{currentfill}%
\pgfsetlinewidth{0.803000pt}%
\definecolor{currentstroke}{rgb}{0.000000,0.000000,0.000000}%
\pgfsetstrokecolor{currentstroke}%
\pgfsetdash{}{0pt}%
\pgfsys@defobject{currentmarker}{\pgfqpoint{0.000000in}{-0.048611in}}{\pgfqpoint{0.000000in}{0.000000in}}{%
\pgfpathmoveto{\pgfqpoint{0.000000in}{0.000000in}}%
\pgfpathlineto{\pgfqpoint{0.000000in}{-0.048611in}}%
\pgfusepath{stroke,fill}%
}%
\begin{pgfscope}%
\pgfsys@transformshift{0.568671in}{0.270766in}%
\pgfsys@useobject{currentmarker}{}%
\end{pgfscope}%
\end{pgfscope}%
\begin{pgfscope}%
\pgfpathrectangle{\pgfqpoint{0.568671in}{0.270766in}}{\pgfqpoint{2.598305in}{0.409637in}}%
\pgfusepath{clip}%
\pgfsetrectcap%
\pgfsetroundjoin%
\pgfsetlinewidth{0.803000pt}%
\definecolor{currentstroke}{rgb}{0.690196,0.690196,0.690196}%
\pgfsetstrokecolor{currentstroke}%
\pgfsetdash{}{0pt}%
\pgfpathmoveto{\pgfqpoint{1.001722in}{0.270766in}}%
\pgfpathlineto{\pgfqpoint{1.001722in}{0.680403in}}%
\pgfusepath{stroke}%
\end{pgfscope}%
\begin{pgfscope}%
\pgfsetbuttcap%
\pgfsetroundjoin%
\definecolor{currentfill}{rgb}{0.000000,0.000000,0.000000}%
\pgfsetfillcolor{currentfill}%
\pgfsetlinewidth{0.803000pt}%
\definecolor{currentstroke}{rgb}{0.000000,0.000000,0.000000}%
\pgfsetstrokecolor{currentstroke}%
\pgfsetdash{}{0pt}%
\pgfsys@defobject{currentmarker}{\pgfqpoint{0.000000in}{-0.048611in}}{\pgfqpoint{0.000000in}{0.000000in}}{%
\pgfpathmoveto{\pgfqpoint{0.000000in}{0.000000in}}%
\pgfpathlineto{\pgfqpoint{0.000000in}{-0.048611in}}%
\pgfusepath{stroke,fill}%
}%
\begin{pgfscope}%
\pgfsys@transformshift{1.001722in}{0.270766in}%
\pgfsys@useobject{currentmarker}{}%
\end{pgfscope}%
\end{pgfscope}%
\begin{pgfscope}%
\pgfpathrectangle{\pgfqpoint{0.568671in}{0.270766in}}{\pgfqpoint{2.598305in}{0.409637in}}%
\pgfusepath{clip}%
\pgfsetrectcap%
\pgfsetroundjoin%
\pgfsetlinewidth{0.803000pt}%
\definecolor{currentstroke}{rgb}{0.690196,0.690196,0.690196}%
\pgfsetstrokecolor{currentstroke}%
\pgfsetdash{}{0pt}%
\pgfpathmoveto{\pgfqpoint{1.434773in}{0.270766in}}%
\pgfpathlineto{\pgfqpoint{1.434773in}{0.680403in}}%
\pgfusepath{stroke}%
\end{pgfscope}%
\begin{pgfscope}%
\pgfsetbuttcap%
\pgfsetroundjoin%
\definecolor{currentfill}{rgb}{0.000000,0.000000,0.000000}%
\pgfsetfillcolor{currentfill}%
\pgfsetlinewidth{0.803000pt}%
\definecolor{currentstroke}{rgb}{0.000000,0.000000,0.000000}%
\pgfsetstrokecolor{currentstroke}%
\pgfsetdash{}{0pt}%
\pgfsys@defobject{currentmarker}{\pgfqpoint{0.000000in}{-0.048611in}}{\pgfqpoint{0.000000in}{0.000000in}}{%
\pgfpathmoveto{\pgfqpoint{0.000000in}{0.000000in}}%
\pgfpathlineto{\pgfqpoint{0.000000in}{-0.048611in}}%
\pgfusepath{stroke,fill}%
}%
\begin{pgfscope}%
\pgfsys@transformshift{1.434773in}{0.270766in}%
\pgfsys@useobject{currentmarker}{}%
\end{pgfscope}%
\end{pgfscope}%
\begin{pgfscope}%
\pgfpathrectangle{\pgfqpoint{0.568671in}{0.270766in}}{\pgfqpoint{2.598305in}{0.409637in}}%
\pgfusepath{clip}%
\pgfsetrectcap%
\pgfsetroundjoin%
\pgfsetlinewidth{0.803000pt}%
\definecolor{currentstroke}{rgb}{0.690196,0.690196,0.690196}%
\pgfsetstrokecolor{currentstroke}%
\pgfsetdash{}{0pt}%
\pgfpathmoveto{\pgfqpoint{1.867823in}{0.270766in}}%
\pgfpathlineto{\pgfqpoint{1.867823in}{0.680403in}}%
\pgfusepath{stroke}%
\end{pgfscope}%
\begin{pgfscope}%
\pgfsetbuttcap%
\pgfsetroundjoin%
\definecolor{currentfill}{rgb}{0.000000,0.000000,0.000000}%
\pgfsetfillcolor{currentfill}%
\pgfsetlinewidth{0.803000pt}%
\definecolor{currentstroke}{rgb}{0.000000,0.000000,0.000000}%
\pgfsetstrokecolor{currentstroke}%
\pgfsetdash{}{0pt}%
\pgfsys@defobject{currentmarker}{\pgfqpoint{0.000000in}{-0.048611in}}{\pgfqpoint{0.000000in}{0.000000in}}{%
\pgfpathmoveto{\pgfqpoint{0.000000in}{0.000000in}}%
\pgfpathlineto{\pgfqpoint{0.000000in}{-0.048611in}}%
\pgfusepath{stroke,fill}%
}%
\begin{pgfscope}%
\pgfsys@transformshift{1.867823in}{0.270766in}%
\pgfsys@useobject{currentmarker}{}%
\end{pgfscope}%
\end{pgfscope}%
\begin{pgfscope}%
\pgfpathrectangle{\pgfqpoint{0.568671in}{0.270766in}}{\pgfqpoint{2.598305in}{0.409637in}}%
\pgfusepath{clip}%
\pgfsetrectcap%
\pgfsetroundjoin%
\pgfsetlinewidth{0.803000pt}%
\definecolor{currentstroke}{rgb}{0.690196,0.690196,0.690196}%
\pgfsetstrokecolor{currentstroke}%
\pgfsetdash{}{0pt}%
\pgfpathmoveto{\pgfqpoint{2.300874in}{0.270766in}}%
\pgfpathlineto{\pgfqpoint{2.300874in}{0.680403in}}%
\pgfusepath{stroke}%
\end{pgfscope}%
\begin{pgfscope}%
\pgfsetbuttcap%
\pgfsetroundjoin%
\definecolor{currentfill}{rgb}{0.000000,0.000000,0.000000}%
\pgfsetfillcolor{currentfill}%
\pgfsetlinewidth{0.803000pt}%
\definecolor{currentstroke}{rgb}{0.000000,0.000000,0.000000}%
\pgfsetstrokecolor{currentstroke}%
\pgfsetdash{}{0pt}%
\pgfsys@defobject{currentmarker}{\pgfqpoint{0.000000in}{-0.048611in}}{\pgfqpoint{0.000000in}{0.000000in}}{%
\pgfpathmoveto{\pgfqpoint{0.000000in}{0.000000in}}%
\pgfpathlineto{\pgfqpoint{0.000000in}{-0.048611in}}%
\pgfusepath{stroke,fill}%
}%
\begin{pgfscope}%
\pgfsys@transformshift{2.300874in}{0.270766in}%
\pgfsys@useobject{currentmarker}{}%
\end{pgfscope}%
\end{pgfscope}%
\begin{pgfscope}%
\pgfpathrectangle{\pgfqpoint{0.568671in}{0.270766in}}{\pgfqpoint{2.598305in}{0.409637in}}%
\pgfusepath{clip}%
\pgfsetrectcap%
\pgfsetroundjoin%
\pgfsetlinewidth{0.803000pt}%
\definecolor{currentstroke}{rgb}{0.690196,0.690196,0.690196}%
\pgfsetstrokecolor{currentstroke}%
\pgfsetdash{}{0pt}%
\pgfpathmoveto{\pgfqpoint{2.733925in}{0.270766in}}%
\pgfpathlineto{\pgfqpoint{2.733925in}{0.680403in}}%
\pgfusepath{stroke}%
\end{pgfscope}%
\begin{pgfscope}%
\pgfsetbuttcap%
\pgfsetroundjoin%
\definecolor{currentfill}{rgb}{0.000000,0.000000,0.000000}%
\pgfsetfillcolor{currentfill}%
\pgfsetlinewidth{0.803000pt}%
\definecolor{currentstroke}{rgb}{0.000000,0.000000,0.000000}%
\pgfsetstrokecolor{currentstroke}%
\pgfsetdash{}{0pt}%
\pgfsys@defobject{currentmarker}{\pgfqpoint{0.000000in}{-0.048611in}}{\pgfqpoint{0.000000in}{0.000000in}}{%
\pgfpathmoveto{\pgfqpoint{0.000000in}{0.000000in}}%
\pgfpathlineto{\pgfqpoint{0.000000in}{-0.048611in}}%
\pgfusepath{stroke,fill}%
}%
\begin{pgfscope}%
\pgfsys@transformshift{2.733925in}{0.270766in}%
\pgfsys@useobject{currentmarker}{}%
\end{pgfscope}%
\end{pgfscope}%
\begin{pgfscope}%
\pgfpathrectangle{\pgfqpoint{0.568671in}{0.270766in}}{\pgfqpoint{2.598305in}{0.409637in}}%
\pgfusepath{clip}%
\pgfsetrectcap%
\pgfsetroundjoin%
\pgfsetlinewidth{0.803000pt}%
\definecolor{currentstroke}{rgb}{0.690196,0.690196,0.690196}%
\pgfsetstrokecolor{currentstroke}%
\pgfsetdash{}{0pt}%
\pgfpathmoveto{\pgfqpoint{3.166976in}{0.270766in}}%
\pgfpathlineto{\pgfqpoint{3.166976in}{0.680403in}}%
\pgfusepath{stroke}%
\end{pgfscope}%
\begin{pgfscope}%
\pgfsetbuttcap%
\pgfsetroundjoin%
\definecolor{currentfill}{rgb}{0.000000,0.000000,0.000000}%
\pgfsetfillcolor{currentfill}%
\pgfsetlinewidth{0.803000pt}%
\definecolor{currentstroke}{rgb}{0.000000,0.000000,0.000000}%
\pgfsetstrokecolor{currentstroke}%
\pgfsetdash{}{0pt}%
\pgfsys@defobject{currentmarker}{\pgfqpoint{0.000000in}{-0.048611in}}{\pgfqpoint{0.000000in}{0.000000in}}{%
\pgfpathmoveto{\pgfqpoint{0.000000in}{0.000000in}}%
\pgfpathlineto{\pgfqpoint{0.000000in}{-0.048611in}}%
\pgfusepath{stroke,fill}%
}%
\begin{pgfscope}%
\pgfsys@transformshift{3.166976in}{0.270766in}%
\pgfsys@useobject{currentmarker}{}%
\end{pgfscope}%
\end{pgfscope}%
\begin{pgfscope}%
\pgfpathrectangle{\pgfqpoint{0.568671in}{0.270766in}}{\pgfqpoint{2.598305in}{0.409637in}}%
\pgfusepath{clip}%
\pgfsetrectcap%
\pgfsetroundjoin%
\pgfsetlinewidth{0.803000pt}%
\definecolor{currentstroke}{rgb}{0.690196,0.690196,0.690196}%
\pgfsetstrokecolor{currentstroke}%
\pgfsetdash{}{0pt}%
\pgfpathmoveto{\pgfqpoint{0.568671in}{0.270766in}}%
\pgfpathlineto{\pgfqpoint{3.166976in}{0.270766in}}%
\pgfusepath{stroke}%
\end{pgfscope}%
\begin{pgfscope}%
\pgfsetbuttcap%
\pgfsetroundjoin%
\definecolor{currentfill}{rgb}{0.000000,0.000000,0.000000}%
\pgfsetfillcolor{currentfill}%
\pgfsetlinewidth{0.803000pt}%
\definecolor{currentstroke}{rgb}{0.000000,0.000000,0.000000}%
\pgfsetstrokecolor{currentstroke}%
\pgfsetdash{}{0pt}%
\pgfsys@defobject{currentmarker}{\pgfqpoint{-0.048611in}{0.000000in}}{\pgfqpoint{-0.000000in}{0.000000in}}{%
\pgfpathmoveto{\pgfqpoint{-0.000000in}{0.000000in}}%
\pgfpathlineto{\pgfqpoint{-0.048611in}{0.000000in}}%
\pgfusepath{stroke,fill}%
}%
\begin{pgfscope}%
\pgfsys@transformshift{0.568671in}{0.270766in}%
\pgfsys@useobject{currentmarker}{}%
\end{pgfscope}%
\end{pgfscope}%
\begin{pgfscope}%
\definecolor{textcolor}{rgb}{0.000000,0.000000,0.000000}%
\pgfsetstrokecolor{textcolor}%
\pgfsetfillcolor{textcolor}%
\pgftext[x=0.243055in, y=0.227391in, left, base]{\color{textcolor}\rmfamily\fontsize{9.000000}{10.800000}\selectfont \(\displaystyle {0.75}\)}%
\end{pgfscope}%
\begin{pgfscope}%
\pgfpathrectangle{\pgfqpoint{0.568671in}{0.270766in}}{\pgfqpoint{2.598305in}{0.409637in}}%
\pgfusepath{clip}%
\pgfsetrectcap%
\pgfsetroundjoin%
\pgfsetlinewidth{0.803000pt}%
\definecolor{currentstroke}{rgb}{0.690196,0.690196,0.690196}%
\pgfsetstrokecolor{currentstroke}%
\pgfsetdash{}{0pt}%
\pgfpathmoveto{\pgfqpoint{0.568671in}{0.475585in}}%
\pgfpathlineto{\pgfqpoint{3.166976in}{0.475585in}}%
\pgfusepath{stroke}%
\end{pgfscope}%
\begin{pgfscope}%
\pgfsetbuttcap%
\pgfsetroundjoin%
\definecolor{currentfill}{rgb}{0.000000,0.000000,0.000000}%
\pgfsetfillcolor{currentfill}%
\pgfsetlinewidth{0.803000pt}%
\definecolor{currentstroke}{rgb}{0.000000,0.000000,0.000000}%
\pgfsetstrokecolor{currentstroke}%
\pgfsetdash{}{0pt}%
\pgfsys@defobject{currentmarker}{\pgfqpoint{-0.048611in}{0.000000in}}{\pgfqpoint{-0.000000in}{0.000000in}}{%
\pgfpathmoveto{\pgfqpoint{-0.000000in}{0.000000in}}%
\pgfpathlineto{\pgfqpoint{-0.048611in}{0.000000in}}%
\pgfusepath{stroke,fill}%
}%
\begin{pgfscope}%
\pgfsys@transformshift{0.568671in}{0.475585in}%
\pgfsys@useobject{currentmarker}{}%
\end{pgfscope}%
\end{pgfscope}%
\begin{pgfscope}%
\definecolor{textcolor}{rgb}{0.000000,0.000000,0.000000}%
\pgfsetstrokecolor{textcolor}%
\pgfsetfillcolor{textcolor}%
\pgftext[x=0.243055in, y=0.432210in, left, base]{\color{textcolor}\rmfamily\fontsize{9.000000}{10.800000}\selectfont \(\displaystyle {1.00}\)}%
\end{pgfscope}%
\begin{pgfscope}%
\pgfpathrectangle{\pgfqpoint{0.568671in}{0.270766in}}{\pgfqpoint{2.598305in}{0.409637in}}%
\pgfusepath{clip}%
\pgfsetrectcap%
\pgfsetroundjoin%
\pgfsetlinewidth{0.803000pt}%
\definecolor{currentstroke}{rgb}{0.690196,0.690196,0.690196}%
\pgfsetstrokecolor{currentstroke}%
\pgfsetdash{}{0pt}%
\pgfpathmoveto{\pgfqpoint{0.568671in}{0.680403in}}%
\pgfpathlineto{\pgfqpoint{3.166976in}{0.680403in}}%
\pgfusepath{stroke}%
\end{pgfscope}%
\begin{pgfscope}%
\pgfsetbuttcap%
\pgfsetroundjoin%
\definecolor{currentfill}{rgb}{0.000000,0.000000,0.000000}%
\pgfsetfillcolor{currentfill}%
\pgfsetlinewidth{0.803000pt}%
\definecolor{currentstroke}{rgb}{0.000000,0.000000,0.000000}%
\pgfsetstrokecolor{currentstroke}%
\pgfsetdash{}{0pt}%
\pgfsys@defobject{currentmarker}{\pgfqpoint{-0.048611in}{0.000000in}}{\pgfqpoint{-0.000000in}{0.000000in}}{%
\pgfpathmoveto{\pgfqpoint{-0.000000in}{0.000000in}}%
\pgfpathlineto{\pgfqpoint{-0.048611in}{0.000000in}}%
\pgfusepath{stroke,fill}%
}%
\begin{pgfscope}%
\pgfsys@transformshift{0.568671in}{0.680403in}%
\pgfsys@useobject{currentmarker}{}%
\end{pgfscope}%
\end{pgfscope}%
\begin{pgfscope}%
\definecolor{textcolor}{rgb}{0.000000,0.000000,0.000000}%
\pgfsetstrokecolor{textcolor}%
\pgfsetfillcolor{textcolor}%
\pgftext[x=0.243055in, y=0.637028in, left, base]{\color{textcolor}\rmfamily\fontsize{9.000000}{10.800000}\selectfont \(\displaystyle {1.25}\)}%
\end{pgfscope}%
\begin{pgfscope}%
\definecolor{textcolor}{rgb}{0.000000,0.000000,0.000000}%
\pgfsetstrokecolor{textcolor}%
\pgfsetfillcolor{textcolor}%
\pgftext[x=0.187500in,y=0.475585in,,bottom,rotate=90.000000]{\color{textcolor}\rmfamily\fontsize{9.000000}{10.800000}\selectfont \(\displaystyle \|\mathbf{h}\|\) [1]}%
\end{pgfscope}%
\begin{pgfscope}%
\pgfpathrectangle{\pgfqpoint{0.568671in}{0.270766in}}{\pgfqpoint{2.598305in}{0.409637in}}%
\pgfusepath{clip}%
\pgfsetbuttcap%
\pgfsetroundjoin%
\pgfsetlinewidth{0.752812pt}%
\definecolor{currentstroke}{rgb}{0.000000,0.000000,0.000000}%
\pgfsetstrokecolor{currentstroke}%
\pgfsetdash{{2.775000pt}{1.200000pt}}{0.000000pt}%
\pgfpathmoveto{\pgfqpoint{1.434773in}{0.270766in}}%
\pgfpathlineto{\pgfqpoint{1.434773in}{0.680403in}}%
\pgfusepath{stroke}%
\end{pgfscope}%
\begin{pgfscope}%
\pgfpathrectangle{\pgfqpoint{0.568671in}{0.270766in}}{\pgfqpoint{2.598305in}{0.409637in}}%
\pgfusepath{clip}%
\pgfsetbuttcap%
\pgfsetroundjoin%
\pgfsetlinewidth{0.752812pt}%
\definecolor{currentstroke}{rgb}{0.000000,0.000000,0.000000}%
\pgfsetstrokecolor{currentstroke}%
\pgfsetdash{{2.775000pt}{1.200000pt}}{0.000000pt}%
\pgfpathmoveto{\pgfqpoint{2.300874in}{0.270766in}}%
\pgfpathlineto{\pgfqpoint{2.300874in}{0.680403in}}%
\pgfusepath{stroke}%
\end{pgfscope}%
\begin{pgfscope}%
\pgfpathrectangle{\pgfqpoint{0.568671in}{0.270766in}}{\pgfqpoint{2.598305in}{0.409637in}}%
\pgfusepath{clip}%
\pgfsetrectcap%
\pgfsetroundjoin%
\pgfsetlinewidth{1.505625pt}%
\definecolor{currentstroke}{rgb}{0.121569,0.466667,0.705882}%
\pgfsetstrokecolor{currentstroke}%
\pgfsetdash{}{0pt}%
\pgfpathmoveto{\pgfqpoint{0.568671in}{0.477637in}}%
\pgfpathlineto{\pgfqpoint{0.568844in}{0.301844in}}%
\pgfpathlineto{\pgfqpoint{0.569884in}{0.639221in}}%
\pgfpathlineto{\pgfqpoint{0.570577in}{0.679451in}}%
\pgfpathlineto{\pgfqpoint{0.571789in}{0.654140in}}%
\pgfpathlineto{\pgfqpoint{0.572309in}{0.646467in}}%
\pgfpathlineto{\pgfqpoint{0.575253in}{0.594009in}}%
\pgfpathlineto{\pgfqpoint{0.575427in}{0.595082in}}%
\pgfpathlineto{\pgfqpoint{0.575600in}{0.592258in}}%
\pgfpathlineto{\pgfqpoint{0.576466in}{0.592296in}}%
\pgfpathlineto{\pgfqpoint{0.579238in}{0.585827in}}%
\pgfpathlineto{\pgfqpoint{0.579411in}{0.586032in}}%
\pgfpathlineto{\pgfqpoint{0.583914in}{0.588502in}}%
\pgfpathlineto{\pgfqpoint{0.587725in}{0.587969in}}%
\pgfpathlineto{\pgfqpoint{0.589631in}{0.583731in}}%
\pgfpathlineto{\pgfqpoint{0.592749in}{0.576043in}}%
\pgfpathlineto{\pgfqpoint{0.593442in}{0.576715in}}%
\pgfpathlineto{\pgfqpoint{0.596040in}{0.579600in}}%
\pgfpathlineto{\pgfqpoint{0.596386in}{0.579139in}}%
\pgfpathlineto{\pgfqpoint{0.598985in}{0.571541in}}%
\pgfpathlineto{\pgfqpoint{0.611976in}{0.530095in}}%
\pgfpathlineto{\pgfqpoint{0.613882in}{0.522469in}}%
\pgfpathlineto{\pgfqpoint{0.617866in}{0.509496in}}%
\pgfpathlineto{\pgfqpoint{0.620637in}{0.506968in}}%
\pgfpathlineto{\pgfqpoint{0.620810in}{0.507093in}}%
\pgfpathlineto{\pgfqpoint{0.623062in}{0.510794in}}%
\pgfpathlineto{\pgfqpoint{0.628952in}{0.524023in}}%
\pgfpathlineto{\pgfqpoint{0.629471in}{0.523786in}}%
\pgfpathlineto{\pgfqpoint{0.631030in}{0.520399in}}%
\pgfpathlineto{\pgfqpoint{0.636747in}{0.505554in}}%
\pgfpathlineto{\pgfqpoint{0.639518in}{0.506653in}}%
\pgfpathlineto{\pgfqpoint{0.644888in}{0.508599in}}%
\pgfpathlineto{\pgfqpoint{0.646793in}{0.502912in}}%
\pgfpathlineto{\pgfqpoint{0.652337in}{0.484445in}}%
\pgfpathlineto{\pgfqpoint{0.662557in}{0.469657in}}%
\pgfpathlineto{\pgfqpoint{0.665328in}{0.471468in}}%
\pgfpathlineto{\pgfqpoint{0.669312in}{0.478855in}}%
\pgfpathlineto{\pgfqpoint{0.681784in}{0.498439in}}%
\pgfpathlineto{\pgfqpoint{0.684902in}{0.497694in}}%
\pgfpathlineto{\pgfqpoint{0.688540in}{0.497171in}}%
\pgfpathlineto{\pgfqpoint{0.694949in}{0.499418in}}%
\pgfpathlineto{\pgfqpoint{0.697720in}{0.496027in}}%
\pgfpathlineto{\pgfqpoint{0.701531in}{0.491791in}}%
\pgfpathlineto{\pgfqpoint{0.703610in}{0.492925in}}%
\pgfpathlineto{\pgfqpoint{0.713830in}{0.502104in}}%
\pgfpathlineto{\pgfqpoint{0.718853in}{0.499867in}}%
\pgfpathlineto{\pgfqpoint{0.724223in}{0.499554in}}%
\pgfpathlineto{\pgfqpoint{0.728034in}{0.500149in}}%
\pgfpathlineto{\pgfqpoint{0.738427in}{0.501330in}}%
\pgfpathlineto{\pgfqpoint{0.749167in}{0.495208in}}%
\pgfpathlineto{\pgfqpoint{0.759040in}{0.484612in}}%
\pgfpathlineto{\pgfqpoint{0.770473in}{0.483416in}}%
\pgfpathlineto{\pgfqpoint{0.778268in}{0.485126in}}%
\pgfpathlineto{\pgfqpoint{0.781559in}{0.488673in}}%
\pgfpathlineto{\pgfqpoint{0.794550in}{0.502628in}}%
\pgfpathlineto{\pgfqpoint{0.820533in}{0.491627in}}%
\pgfpathlineto{\pgfqpoint{0.824518in}{0.492677in}}%
\pgfpathlineto{\pgfqpoint{0.833179in}{0.495927in}}%
\pgfpathlineto{\pgfqpoint{0.837682in}{0.494601in}}%
\pgfpathlineto{\pgfqpoint{0.853792in}{0.485099in}}%
\pgfpathlineto{\pgfqpoint{0.869555in}{0.492084in}}%
\pgfpathlineto{\pgfqpoint{0.874751in}{0.492780in}}%
\pgfpathlineto{\pgfqpoint{0.889648in}{0.490230in}}%
\pgfpathlineto{\pgfqpoint{0.898483in}{0.495507in}}%
\pgfpathlineto{\pgfqpoint{0.903679in}{0.497725in}}%
\pgfpathlineto{\pgfqpoint{0.910954in}{0.495524in}}%
\pgfpathlineto{\pgfqpoint{0.928450in}{0.486651in}}%
\pgfpathlineto{\pgfqpoint{0.936418in}{0.479407in}}%
\pgfpathlineto{\pgfqpoint{0.944386in}{0.472354in}}%
\pgfpathlineto{\pgfqpoint{0.949063in}{0.471142in}}%
\pgfpathlineto{\pgfqpoint{0.955992in}{0.473587in}}%
\pgfpathlineto{\pgfqpoint{0.961535in}{0.479657in}}%
\pgfpathlineto{\pgfqpoint{0.966558in}{0.488791in}}%
\pgfpathlineto{\pgfqpoint{0.975566in}{0.503856in}}%
\pgfpathlineto{\pgfqpoint{0.980416in}{0.506161in}}%
\pgfpathlineto{\pgfqpoint{0.983534in}{0.505268in}}%
\pgfpathlineto{\pgfqpoint{0.990116in}{0.500232in}}%
\pgfpathlineto{\pgfqpoint{1.003454in}{0.485407in}}%
\pgfpathlineto{\pgfqpoint{1.007438in}{0.484332in}}%
\pgfpathlineto{\pgfqpoint{1.011942in}{0.485859in}}%
\pgfpathlineto{\pgfqpoint{1.018178in}{0.492320in}}%
\pgfpathlineto{\pgfqpoint{1.033941in}{0.506791in}}%
\pgfpathlineto{\pgfqpoint{1.039137in}{0.505177in}}%
\pgfpathlineto{\pgfqpoint{1.052475in}{0.492573in}}%
\pgfpathlineto{\pgfqpoint{1.065813in}{0.477926in}}%
\pgfpathlineto{\pgfqpoint{1.077419in}{0.475650in}}%
\pgfpathlineto{\pgfqpoint{1.084348in}{0.481241in}}%
\pgfpathlineto{\pgfqpoint{1.098552in}{0.494710in}}%
\pgfpathlineto{\pgfqpoint{1.107386in}{0.497904in}}%
\pgfpathlineto{\pgfqpoint{1.113449in}{0.496280in}}%
\pgfpathlineto{\pgfqpoint{1.117087in}{0.491906in}}%
\pgfpathlineto{\pgfqpoint{1.124708in}{0.483244in}}%
\pgfpathlineto{\pgfqpoint{1.135448in}{0.480976in}}%
\pgfpathlineto{\pgfqpoint{1.141337in}{0.481875in}}%
\pgfpathlineto{\pgfqpoint{1.161431in}{0.494125in}}%
\pgfpathlineto{\pgfqpoint{1.169053in}{0.497895in}}%
\pgfpathlineto{\pgfqpoint{1.173903in}{0.498947in}}%
\pgfpathlineto{\pgfqpoint{1.183430in}{0.499456in}}%
\pgfpathlineto{\pgfqpoint{1.190359in}{0.499338in}}%
\pgfpathlineto{\pgfqpoint{1.194170in}{0.496334in}}%
\pgfpathlineto{\pgfqpoint{1.202657in}{0.492516in}}%
\pgfpathlineto{\pgfqpoint{1.207161in}{0.494663in}}%
\pgfpathlineto{\pgfqpoint{1.216688in}{0.500268in}}%
\pgfpathlineto{\pgfqpoint{1.225523in}{0.502758in}}%
\pgfpathlineto{\pgfqpoint{1.235569in}{0.498754in}}%
\pgfpathlineto{\pgfqpoint{1.247522in}{0.491332in}}%
\pgfpathlineto{\pgfqpoint{1.252372in}{0.488060in}}%
\pgfpathlineto{\pgfqpoint{1.268481in}{0.486284in}}%
\pgfpathlineto{\pgfqpoint{1.295504in}{0.496477in}}%
\pgfpathlineto{\pgfqpoint{1.301566in}{0.494494in}}%
\pgfpathlineto{\pgfqpoint{1.314038in}{0.488305in}}%
\pgfpathlineto{\pgfqpoint{1.319928in}{0.485004in}}%
\pgfpathlineto{\pgfqpoint{1.324085in}{0.486349in}}%
\pgfpathlineto{\pgfqpoint{1.335517in}{0.493120in}}%
\pgfpathlineto{\pgfqpoint{1.381248in}{0.494411in}}%
\pgfpathlineto{\pgfqpoint{1.392680in}{0.497834in}}%
\pgfpathlineto{\pgfqpoint{1.398570in}{0.498098in}}%
\pgfpathlineto{\pgfqpoint{1.404979in}{0.493881in}}%
\pgfpathlineto{\pgfqpoint{1.424206in}{0.482260in}}%
\pgfpathlineto{\pgfqpoint{1.431135in}{0.486026in}}%
\pgfpathlineto{\pgfqpoint{1.434253in}{0.487644in}}%
\pgfpathlineto{\pgfqpoint{1.434426in}{0.486994in}}%
\pgfpathlineto{\pgfqpoint{1.435292in}{0.475333in}}%
\pgfpathlineto{\pgfqpoint{1.443953in}{0.330633in}}%
\pgfpathlineto{\pgfqpoint{1.445339in}{0.329725in}}%
\pgfpathlineto{\pgfqpoint{1.445859in}{0.330347in}}%
\pgfpathlineto{\pgfqpoint{1.447591in}{0.336955in}}%
\pgfpathlineto{\pgfqpoint{1.450016in}{0.362328in}}%
\pgfpathlineto{\pgfqpoint{1.453307in}{0.437258in}}%
\pgfpathlineto{\pgfqpoint{1.460409in}{0.604716in}}%
\pgfpathlineto{\pgfqpoint{1.463700in}{0.621243in}}%
\pgfpathlineto{\pgfqpoint{1.466645in}{0.623523in}}%
\pgfpathlineto{\pgfqpoint{1.469590in}{0.621598in}}%
\pgfpathlineto{\pgfqpoint{1.473228in}{0.615389in}}%
\pgfpathlineto{\pgfqpoint{1.476346in}{0.600437in}}%
\pgfpathlineto{\pgfqpoint{1.480330in}{0.562392in}}%
\pgfpathlineto{\pgfqpoint{1.489164in}{0.443996in}}%
\pgfpathlineto{\pgfqpoint{1.494880in}{0.384451in}}%
\pgfpathlineto{\pgfqpoint{1.498518in}{0.370972in}}%
\pgfpathlineto{\pgfqpoint{1.500423in}{0.370518in}}%
\pgfpathlineto{\pgfqpoint{1.500596in}{0.370672in}}%
\pgfpathlineto{\pgfqpoint{1.502848in}{0.374718in}}%
\pgfpathlineto{\pgfqpoint{1.506659in}{0.388401in}}%
\pgfpathlineto{\pgfqpoint{1.512029in}{0.420877in}}%
\pgfpathlineto{\pgfqpoint{1.528485in}{0.532504in}}%
\pgfpathlineto{\pgfqpoint{1.533681in}{0.546823in}}%
\pgfpathlineto{\pgfqpoint{1.536453in}{0.547943in}}%
\pgfpathlineto{\pgfqpoint{1.538358in}{0.545566in}}%
\pgfpathlineto{\pgfqpoint{1.542342in}{0.534824in}}%
\pgfpathlineto{\pgfqpoint{1.548925in}{0.507081in}}%
\pgfpathlineto{\pgfqpoint{1.561050in}{0.452232in}}%
\pgfpathlineto{\pgfqpoint{1.567633in}{0.436869in}}%
\pgfpathlineto{\pgfqpoint{1.573695in}{0.429676in}}%
\pgfpathlineto{\pgfqpoint{1.576987in}{0.429767in}}%
\pgfpathlineto{\pgfqpoint{1.580278in}{0.433743in}}%
\pgfpathlineto{\pgfqpoint{1.585474in}{0.445793in}}%
\pgfpathlineto{\pgfqpoint{1.607127in}{0.496631in}}%
\pgfpathlineto{\pgfqpoint{1.614575in}{0.504572in}}%
\pgfpathlineto{\pgfqpoint{1.618559in}{0.505484in}}%
\pgfpathlineto{\pgfqpoint{1.622543in}{0.503853in}}%
\pgfpathlineto{\pgfqpoint{1.629472in}{0.495888in}}%
\pgfpathlineto{\pgfqpoint{1.654762in}{0.461060in}}%
\pgfpathlineto{\pgfqpoint{1.663250in}{0.459173in}}%
\pgfpathlineto{\pgfqpoint{1.681612in}{0.463856in}}%
\pgfpathlineto{\pgfqpoint{1.688194in}{0.471581in}}%
\pgfpathlineto{\pgfqpoint{1.696855in}{0.480884in}}%
\pgfpathlineto{\pgfqpoint{1.717988in}{0.489521in}}%
\pgfpathlineto{\pgfqpoint{1.727515in}{0.488306in}}%
\pgfpathlineto{\pgfqpoint{1.757309in}{0.470889in}}%
\pgfpathlineto{\pgfqpoint{1.775670in}{0.471279in}}%
\pgfpathlineto{\pgfqpoint{1.784678in}{0.472130in}}%
\pgfpathlineto{\pgfqpoint{1.790394in}{0.473243in}}%
\pgfpathlineto{\pgfqpoint{1.813432in}{0.483726in}}%
\pgfpathlineto{\pgfqpoint{1.820881in}{0.483161in}}%
\pgfpathlineto{\pgfqpoint{1.831620in}{0.481674in}}%
\pgfpathlineto{\pgfqpoint{1.841321in}{0.482247in}}%
\pgfpathlineto{\pgfqpoint{1.849635in}{0.480314in}}%
\pgfpathlineto{\pgfqpoint{1.858296in}{0.477239in}}%
\pgfpathlineto{\pgfqpoint{1.870249in}{0.473263in}}%
\pgfpathlineto{\pgfqpoint{1.879256in}{0.469011in}}%
\pgfpathlineto{\pgfqpoint{1.896405in}{0.465218in}}%
\pgfpathlineto{\pgfqpoint{1.902294in}{0.468295in}}%
\pgfpathlineto{\pgfqpoint{1.909916in}{0.475382in}}%
\pgfpathlineto{\pgfqpoint{1.919097in}{0.483002in}}%
\pgfpathlineto{\pgfqpoint{1.927065in}{0.487467in}}%
\pgfpathlineto{\pgfqpoint{1.940576in}{0.490756in}}%
\pgfpathlineto{\pgfqpoint{1.946292in}{0.487589in}}%
\pgfpathlineto{\pgfqpoint{1.964480in}{0.476931in}}%
\pgfpathlineto{\pgfqpoint{1.969677in}{0.475256in}}%
\pgfpathlineto{\pgfqpoint{1.983188in}{0.476051in}}%
\pgfpathlineto{\pgfqpoint{2.005187in}{0.487660in}}%
\pgfpathlineto{\pgfqpoint{2.012982in}{0.487567in}}%
\pgfpathlineto{\pgfqpoint{2.030651in}{0.480012in}}%
\pgfpathlineto{\pgfqpoint{2.040351in}{0.471214in}}%
\pgfpathlineto{\pgfqpoint{2.046240in}{0.468137in}}%
\pgfpathlineto{\pgfqpoint{2.055075in}{0.469131in}}%
\pgfpathlineto{\pgfqpoint{2.072223in}{0.476249in}}%
\pgfpathlineto{\pgfqpoint{2.094915in}{0.491403in}}%
\pgfpathlineto{\pgfqpoint{2.102364in}{0.490821in}}%
\pgfpathlineto{\pgfqpoint{2.111025in}{0.486094in}}%
\pgfpathlineto{\pgfqpoint{2.132331in}{0.469864in}}%
\pgfpathlineto{\pgfqpoint{2.141858in}{0.469619in}}%
\pgfpathlineto{\pgfqpoint{2.161952in}{0.473511in}}%
\pgfpathlineto{\pgfqpoint{2.171825in}{0.478675in}}%
\pgfpathlineto{\pgfqpoint{2.184990in}{0.487181in}}%
\pgfpathlineto{\pgfqpoint{2.191572in}{0.487126in}}%
\pgfpathlineto{\pgfqpoint{2.203178in}{0.486255in}}%
\pgfpathlineto{\pgfqpoint{2.212359in}{0.486146in}}%
\pgfpathlineto{\pgfqpoint{2.227256in}{0.481184in}}%
\pgfpathlineto{\pgfqpoint{2.259301in}{0.466490in}}%
\pgfpathlineto{\pgfqpoint{2.272986in}{0.475068in}}%
\pgfpathlineto{\pgfqpoint{2.280088in}{0.480001in}}%
\pgfpathlineto{\pgfqpoint{2.299488in}{0.492163in}}%
\pgfpathlineto{\pgfqpoint{2.300528in}{0.491073in}}%
\pgfpathlineto{\pgfqpoint{2.301221in}{0.479599in}}%
\pgfpathlineto{\pgfqpoint{2.308669in}{0.283363in}}%
\pgfpathlineto{\pgfqpoint{2.308842in}{0.283425in}}%
\pgfpathlineto{\pgfqpoint{2.309882in}{0.286676in}}%
\pgfpathlineto{\pgfqpoint{2.311614in}{0.306483in}}%
\pgfpathlineto{\pgfqpoint{2.314039in}{0.378542in}}%
\pgfpathlineto{\pgfqpoint{2.322180in}{0.666429in}}%
\pgfpathlineto{\pgfqpoint{2.325472in}{0.676339in}}%
\pgfpathlineto{\pgfqpoint{2.328589in}{0.678802in}}%
\pgfpathlineto{\pgfqpoint{2.331881in}{0.676084in}}%
\pgfpathlineto{\pgfqpoint{2.334999in}{0.667283in}}%
\pgfpathlineto{\pgfqpoint{2.341408in}{0.640984in}}%
\pgfpathlineto{\pgfqpoint{2.345045in}{0.606667in}}%
\pgfpathlineto{\pgfqpoint{2.350588in}{0.519693in}}%
\pgfpathlineto{\pgfqpoint{2.359942in}{0.379499in}}%
\pgfpathlineto{\pgfqpoint{2.363407in}{0.365532in}}%
\pgfpathlineto{\pgfqpoint{2.365312in}{0.365572in}}%
\pgfpathlineto{\pgfqpoint{2.368603in}{0.371364in}}%
\pgfpathlineto{\pgfqpoint{2.373280in}{0.386941in}}%
\pgfpathlineto{\pgfqpoint{2.377438in}{0.411831in}}%
\pgfpathlineto{\pgfqpoint{2.382807in}{0.463640in}}%
\pgfpathlineto{\pgfqpoint{2.390776in}{0.533798in}}%
\pgfpathlineto{\pgfqpoint{2.395106in}{0.548888in}}%
\pgfpathlineto{\pgfqpoint{2.398570in}{0.551843in}}%
\pgfpathlineto{\pgfqpoint{2.401169in}{0.550004in}}%
\pgfpathlineto{\pgfqpoint{2.404114in}{0.545036in}}%
\pgfpathlineto{\pgfqpoint{2.408098in}{0.532316in}}%
\pgfpathlineto{\pgfqpoint{2.418318in}{0.481866in}}%
\pgfpathlineto{\pgfqpoint{2.426632in}{0.449992in}}%
\pgfpathlineto{\pgfqpoint{2.432522in}{0.439134in}}%
\pgfpathlineto{\pgfqpoint{2.436679in}{0.436229in}}%
\pgfpathlineto{\pgfqpoint{2.439797in}{0.436799in}}%
\pgfpathlineto{\pgfqpoint{2.443261in}{0.439828in}}%
\pgfpathlineto{\pgfqpoint{2.448978in}{0.451261in}}%
\pgfpathlineto{\pgfqpoint{2.465434in}{0.486922in}}%
\pgfpathlineto{\pgfqpoint{2.472536in}{0.500816in}}%
\pgfpathlineto{\pgfqpoint{2.479118in}{0.509274in}}%
\pgfpathlineto{\pgfqpoint{2.484488in}{0.510061in}}%
\pgfpathlineto{\pgfqpoint{2.488645in}{0.507514in}}%
\pgfpathlineto{\pgfqpoint{2.493668in}{0.501009in}}%
\pgfpathlineto{\pgfqpoint{2.503715in}{0.479106in}}%
\pgfpathlineto{\pgfqpoint{2.513589in}{0.457481in}}%
\pgfpathlineto{\pgfqpoint{2.520691in}{0.448925in}}%
\pgfpathlineto{\pgfqpoint{2.525021in}{0.447877in}}%
\pgfpathlineto{\pgfqpoint{2.530218in}{0.451599in}}%
\pgfpathlineto{\pgfqpoint{2.537840in}{0.464259in}}%
\pgfpathlineto{\pgfqpoint{2.548926in}{0.487561in}}%
\pgfpathlineto{\pgfqpoint{2.555681in}{0.497709in}}%
\pgfpathlineto{\pgfqpoint{2.562437in}{0.503762in}}%
\pgfpathlineto{\pgfqpoint{2.566594in}{0.503961in}}%
\pgfpathlineto{\pgfqpoint{2.570405in}{0.500976in}}%
\pgfpathlineto{\pgfqpoint{2.579932in}{0.493602in}}%
\pgfpathlineto{\pgfqpoint{2.586168in}{0.488978in}}%
\pgfpathlineto{\pgfqpoint{2.593443in}{0.479319in}}%
\pgfpathlineto{\pgfqpoint{2.600719in}{0.471752in}}%
\pgfpathlineto{\pgfqpoint{2.608167in}{0.469584in}}%
\pgfpathlineto{\pgfqpoint{2.618560in}{0.469242in}}%
\pgfpathlineto{\pgfqpoint{2.639174in}{0.478017in}}%
\pgfpathlineto{\pgfqpoint{2.665676in}{0.492359in}}%
\pgfpathlineto{\pgfqpoint{2.676762in}{0.488239in}}%
\pgfpathlineto{\pgfqpoint{2.683345in}{0.485837in}}%
\pgfpathlineto{\pgfqpoint{2.689061in}{0.486020in}}%
\pgfpathlineto{\pgfqpoint{2.699627in}{0.484143in}}%
\pgfpathlineto{\pgfqpoint{2.716257in}{0.475789in}}%
\pgfpathlineto{\pgfqpoint{2.721280in}{0.477063in}}%
\pgfpathlineto{\pgfqpoint{2.729768in}{0.478259in}}%
\pgfpathlineto{\pgfqpoint{2.732712in}{0.478288in}}%
\pgfpathlineto{\pgfqpoint{2.740507in}{0.478844in}}%
\pgfpathlineto{\pgfqpoint{2.744318in}{0.479191in}}%
\pgfpathlineto{\pgfqpoint{2.752113in}{0.479133in}}%
\pgfpathlineto{\pgfqpoint{2.758003in}{0.479294in}}%
\pgfpathlineto{\pgfqpoint{2.774632in}{0.484496in}}%
\pgfpathlineto{\pgfqpoint{2.785718in}{0.489780in}}%
\pgfpathlineto{\pgfqpoint{2.791434in}{0.489751in}}%
\pgfpathlineto{\pgfqpoint{2.811701in}{0.487975in}}%
\pgfpathlineto{\pgfqpoint{2.819496in}{0.487286in}}%
\pgfpathlineto{\pgfqpoint{2.828330in}{0.484976in}}%
\pgfpathlineto{\pgfqpoint{2.836298in}{0.480656in}}%
\pgfpathlineto{\pgfqpoint{2.844266in}{0.477698in}}%
\pgfpathlineto{\pgfqpoint{2.854660in}{0.478791in}}%
\pgfpathlineto{\pgfqpoint{2.864360in}{0.483070in}}%
\pgfpathlineto{\pgfqpoint{2.878564in}{0.491152in}}%
\pgfpathlineto{\pgfqpoint{2.890516in}{0.490699in}}%
\pgfpathlineto{\pgfqpoint{2.901429in}{0.486148in}}%
\pgfpathlineto{\pgfqpoint{2.917885in}{0.476549in}}%
\pgfpathlineto{\pgfqpoint{2.928625in}{0.476455in}}%
\pgfpathlineto{\pgfqpoint{2.951317in}{0.478626in}}%
\pgfpathlineto{\pgfqpoint{2.960324in}{0.484288in}}%
\pgfpathlineto{\pgfqpoint{2.966040in}{0.488461in}}%
\pgfpathlineto{\pgfqpoint{2.975741in}{0.495638in}}%
\pgfpathlineto{\pgfqpoint{2.985094in}{0.492506in}}%
\pgfpathlineto{\pgfqpoint{2.989079in}{0.491139in}}%
\pgfpathlineto{\pgfqpoint{3.003629in}{0.486092in}}%
\pgfpathlineto{\pgfqpoint{3.011424in}{0.483950in}}%
\pgfpathlineto{\pgfqpoint{3.038100in}{0.475691in}}%
\pgfpathlineto{\pgfqpoint{3.044682in}{0.476560in}}%
\pgfpathlineto{\pgfqpoint{3.050225in}{0.480003in}}%
\pgfpathlineto{\pgfqpoint{3.064256in}{0.488698in}}%
\pgfpathlineto{\pgfqpoint{3.077248in}{0.495373in}}%
\pgfpathlineto{\pgfqpoint{3.082964in}{0.494184in}}%
\pgfpathlineto{\pgfqpoint{3.101152in}{0.486792in}}%
\pgfpathlineto{\pgfqpoint{3.116395in}{0.479655in}}%
\pgfpathlineto{\pgfqpoint{3.152772in}{0.475610in}}%
\pgfpathlineto{\pgfqpoint{3.160393in}{0.480037in}}%
\pgfpathlineto{\pgfqpoint{3.166456in}{0.484132in}}%
\pgfpathlineto{\pgfqpoint{3.166456in}{0.484132in}}%
\pgfusepath{stroke}%
\end{pgfscope}%
\begin{pgfscope}%
\pgfsetrectcap%
\pgfsetmiterjoin%
\pgfsetlinewidth{0.803000pt}%
\definecolor{currentstroke}{rgb}{0.000000,0.000000,0.000000}%
\pgfsetstrokecolor{currentstroke}%
\pgfsetdash{}{0pt}%
\pgfpathmoveto{\pgfqpoint{0.568671in}{0.270766in}}%
\pgfpathlineto{\pgfqpoint{0.568671in}{0.680403in}}%
\pgfusepath{stroke}%
\end{pgfscope}%
\begin{pgfscope}%
\pgfsetrectcap%
\pgfsetmiterjoin%
\pgfsetlinewidth{0.803000pt}%
\definecolor{currentstroke}{rgb}{0.000000,0.000000,0.000000}%
\pgfsetstrokecolor{currentstroke}%
\pgfsetdash{}{0pt}%
\pgfpathmoveto{\pgfqpoint{3.166976in}{0.270766in}}%
\pgfpathlineto{\pgfqpoint{3.166976in}{0.680403in}}%
\pgfusepath{stroke}%
\end{pgfscope}%
\begin{pgfscope}%
\pgfsetrectcap%
\pgfsetmiterjoin%
\pgfsetlinewidth{0.803000pt}%
\definecolor{currentstroke}{rgb}{0.000000,0.000000,0.000000}%
\pgfsetstrokecolor{currentstroke}%
\pgfsetdash{}{0pt}%
\pgfpathmoveto{\pgfqpoint{0.568671in}{0.270766in}}%
\pgfpathlineto{\pgfqpoint{3.166976in}{0.270766in}}%
\pgfusepath{stroke}%
\end{pgfscope}%
\begin{pgfscope}%
\pgfsetrectcap%
\pgfsetmiterjoin%
\pgfsetlinewidth{0.803000pt}%
\definecolor{currentstroke}{rgb}{0.000000,0.000000,0.000000}%
\pgfsetstrokecolor{currentstroke}%
\pgfsetdash{}{0pt}%
\pgfpathmoveto{\pgfqpoint{0.568671in}{0.680403in}}%
\pgfpathlineto{\pgfqpoint{3.166976in}{0.680403in}}%
\pgfusepath{stroke}%
\end{pgfscope}%
\end{pgfpicture}%
\makeatother%
\endgroup%
\\\vspace*{-1.0cm}
    %% Creator: Matplotlib, PGF backend
%%
%% To include the figure in your LaTeX document, write
%%   \input{<filename>.pgf}
%%
%% Make sure the required packages are loaded in your preamble
%%   \usepackage{pgf}
%%
%% Also ensure that all the required font packages are loaded; for instance,
%% the lmodern package is sometimes necessary when using math font.
%%   \usepackage{lmodern}
%%
%% Figures using additional raster images can only be included by \input if
%% they are in the same directory as the main LaTeX file. For loading figures
%% from other directories you can use the `import` package
%%   \usepackage{import}
%%
%% and then include the figures with
%%   \import{<path to file>}{<filename>.pgf}
%%
%% Matplotlib used the following preamble
%%   \usepackage{fontspec}
%%
\begingroup%
\makeatletter%
\begin{pgfpicture}%
\pgfpathrectangle{\pgfpointorigin}{\pgfqpoint{3.390065in}{0.847516in}}%
\pgfusepath{use as bounding box, clip}%
\begin{pgfscope}%
\pgfsetbuttcap%
\pgfsetmiterjoin%
\definecolor{currentfill}{rgb}{1.000000,1.000000,1.000000}%
\pgfsetfillcolor{currentfill}%
\pgfsetlinewidth{0.000000pt}%
\definecolor{currentstroke}{rgb}{1.000000,1.000000,1.000000}%
\pgfsetstrokecolor{currentstroke}%
\pgfsetstrokeopacity{0.000000}%
\pgfsetdash{}{0pt}%
\pgfpathmoveto{\pgfqpoint{0.000000in}{0.000000in}}%
\pgfpathlineto{\pgfqpoint{3.390065in}{0.000000in}}%
\pgfpathlineto{\pgfqpoint{3.390065in}{0.847516in}}%
\pgfpathlineto{\pgfqpoint{0.000000in}{0.847516in}}%
\pgfpathlineto{\pgfqpoint{0.000000in}{0.000000in}}%
\pgfpathclose%
\pgfusepath{fill}%
\end{pgfscope}%
\begin{pgfscope}%
\pgfsetbuttcap%
\pgfsetmiterjoin%
\definecolor{currentfill}{rgb}{1.000000,1.000000,1.000000}%
\pgfsetfillcolor{currentfill}%
\pgfsetlinewidth{0.000000pt}%
\definecolor{currentstroke}{rgb}{0.000000,0.000000,0.000000}%
\pgfsetstrokecolor{currentstroke}%
\pgfsetstrokeopacity{0.000000}%
\pgfsetdash{}{0pt}%
\pgfpathmoveto{\pgfqpoint{0.568671in}{0.225639in}}%
\pgfpathlineto{\pgfqpoint{3.166976in}{0.225639in}}%
\pgfpathlineto{\pgfqpoint{3.166976in}{0.680791in}}%
\pgfpathlineto{\pgfqpoint{0.568671in}{0.680791in}}%
\pgfpathlineto{\pgfqpoint{0.568671in}{0.225639in}}%
\pgfpathclose%
\pgfusepath{fill}%
\end{pgfscope}%
\begin{pgfscope}%
\pgfpathrectangle{\pgfqpoint{0.568671in}{0.225639in}}{\pgfqpoint{2.598305in}{0.455152in}}%
\pgfusepath{clip}%
\pgfsetrectcap%
\pgfsetroundjoin%
\pgfsetlinewidth{0.803000pt}%
\definecolor{currentstroke}{rgb}{0.690196,0.690196,0.690196}%
\pgfsetstrokecolor{currentstroke}%
\pgfsetdash{}{0pt}%
\pgfpathmoveto{\pgfqpoint{0.568671in}{0.225639in}}%
\pgfpathlineto{\pgfqpoint{0.568671in}{0.680791in}}%
\pgfusepath{stroke}%
\end{pgfscope}%
\begin{pgfscope}%
\pgfsetbuttcap%
\pgfsetroundjoin%
\definecolor{currentfill}{rgb}{0.000000,0.000000,0.000000}%
\pgfsetfillcolor{currentfill}%
\pgfsetlinewidth{0.803000pt}%
\definecolor{currentstroke}{rgb}{0.000000,0.000000,0.000000}%
\pgfsetstrokecolor{currentstroke}%
\pgfsetdash{}{0pt}%
\pgfsys@defobject{currentmarker}{\pgfqpoint{0.000000in}{-0.048611in}}{\pgfqpoint{0.000000in}{0.000000in}}{%
\pgfpathmoveto{\pgfqpoint{0.000000in}{0.000000in}}%
\pgfpathlineto{\pgfqpoint{0.000000in}{-0.048611in}}%
\pgfusepath{stroke,fill}%
}%
\begin{pgfscope}%
\pgfsys@transformshift{0.568671in}{0.225639in}%
\pgfsys@useobject{currentmarker}{}%
\end{pgfscope}%
\end{pgfscope}%
\begin{pgfscope}%
\pgfpathrectangle{\pgfqpoint{0.568671in}{0.225639in}}{\pgfqpoint{2.598305in}{0.455152in}}%
\pgfusepath{clip}%
\pgfsetrectcap%
\pgfsetroundjoin%
\pgfsetlinewidth{0.803000pt}%
\definecolor{currentstroke}{rgb}{0.690196,0.690196,0.690196}%
\pgfsetstrokecolor{currentstroke}%
\pgfsetdash{}{0pt}%
\pgfpathmoveto{\pgfqpoint{1.001722in}{0.225639in}}%
\pgfpathlineto{\pgfqpoint{1.001722in}{0.680791in}}%
\pgfusepath{stroke}%
\end{pgfscope}%
\begin{pgfscope}%
\pgfsetbuttcap%
\pgfsetroundjoin%
\definecolor{currentfill}{rgb}{0.000000,0.000000,0.000000}%
\pgfsetfillcolor{currentfill}%
\pgfsetlinewidth{0.803000pt}%
\definecolor{currentstroke}{rgb}{0.000000,0.000000,0.000000}%
\pgfsetstrokecolor{currentstroke}%
\pgfsetdash{}{0pt}%
\pgfsys@defobject{currentmarker}{\pgfqpoint{0.000000in}{-0.048611in}}{\pgfqpoint{0.000000in}{0.000000in}}{%
\pgfpathmoveto{\pgfqpoint{0.000000in}{0.000000in}}%
\pgfpathlineto{\pgfqpoint{0.000000in}{-0.048611in}}%
\pgfusepath{stroke,fill}%
}%
\begin{pgfscope}%
\pgfsys@transformshift{1.001722in}{0.225639in}%
\pgfsys@useobject{currentmarker}{}%
\end{pgfscope}%
\end{pgfscope}%
\begin{pgfscope}%
\pgfpathrectangle{\pgfqpoint{0.568671in}{0.225639in}}{\pgfqpoint{2.598305in}{0.455152in}}%
\pgfusepath{clip}%
\pgfsetrectcap%
\pgfsetroundjoin%
\pgfsetlinewidth{0.803000pt}%
\definecolor{currentstroke}{rgb}{0.690196,0.690196,0.690196}%
\pgfsetstrokecolor{currentstroke}%
\pgfsetdash{}{0pt}%
\pgfpathmoveto{\pgfqpoint{1.434773in}{0.225639in}}%
\pgfpathlineto{\pgfqpoint{1.434773in}{0.680791in}}%
\pgfusepath{stroke}%
\end{pgfscope}%
\begin{pgfscope}%
\pgfsetbuttcap%
\pgfsetroundjoin%
\definecolor{currentfill}{rgb}{0.000000,0.000000,0.000000}%
\pgfsetfillcolor{currentfill}%
\pgfsetlinewidth{0.803000pt}%
\definecolor{currentstroke}{rgb}{0.000000,0.000000,0.000000}%
\pgfsetstrokecolor{currentstroke}%
\pgfsetdash{}{0pt}%
\pgfsys@defobject{currentmarker}{\pgfqpoint{0.000000in}{-0.048611in}}{\pgfqpoint{0.000000in}{0.000000in}}{%
\pgfpathmoveto{\pgfqpoint{0.000000in}{0.000000in}}%
\pgfpathlineto{\pgfqpoint{0.000000in}{-0.048611in}}%
\pgfusepath{stroke,fill}%
}%
\begin{pgfscope}%
\pgfsys@transformshift{1.434773in}{0.225639in}%
\pgfsys@useobject{currentmarker}{}%
\end{pgfscope}%
\end{pgfscope}%
\begin{pgfscope}%
\pgfpathrectangle{\pgfqpoint{0.568671in}{0.225639in}}{\pgfqpoint{2.598305in}{0.455152in}}%
\pgfusepath{clip}%
\pgfsetrectcap%
\pgfsetroundjoin%
\pgfsetlinewidth{0.803000pt}%
\definecolor{currentstroke}{rgb}{0.690196,0.690196,0.690196}%
\pgfsetstrokecolor{currentstroke}%
\pgfsetdash{}{0pt}%
\pgfpathmoveto{\pgfqpoint{1.867823in}{0.225639in}}%
\pgfpathlineto{\pgfqpoint{1.867823in}{0.680791in}}%
\pgfusepath{stroke}%
\end{pgfscope}%
\begin{pgfscope}%
\pgfsetbuttcap%
\pgfsetroundjoin%
\definecolor{currentfill}{rgb}{0.000000,0.000000,0.000000}%
\pgfsetfillcolor{currentfill}%
\pgfsetlinewidth{0.803000pt}%
\definecolor{currentstroke}{rgb}{0.000000,0.000000,0.000000}%
\pgfsetstrokecolor{currentstroke}%
\pgfsetdash{}{0pt}%
\pgfsys@defobject{currentmarker}{\pgfqpoint{0.000000in}{-0.048611in}}{\pgfqpoint{0.000000in}{0.000000in}}{%
\pgfpathmoveto{\pgfqpoint{0.000000in}{0.000000in}}%
\pgfpathlineto{\pgfqpoint{0.000000in}{-0.048611in}}%
\pgfusepath{stroke,fill}%
}%
\begin{pgfscope}%
\pgfsys@transformshift{1.867823in}{0.225639in}%
\pgfsys@useobject{currentmarker}{}%
\end{pgfscope}%
\end{pgfscope}%
\begin{pgfscope}%
\pgfpathrectangle{\pgfqpoint{0.568671in}{0.225639in}}{\pgfqpoint{2.598305in}{0.455152in}}%
\pgfusepath{clip}%
\pgfsetrectcap%
\pgfsetroundjoin%
\pgfsetlinewidth{0.803000pt}%
\definecolor{currentstroke}{rgb}{0.690196,0.690196,0.690196}%
\pgfsetstrokecolor{currentstroke}%
\pgfsetdash{}{0pt}%
\pgfpathmoveto{\pgfqpoint{2.300874in}{0.225639in}}%
\pgfpathlineto{\pgfqpoint{2.300874in}{0.680791in}}%
\pgfusepath{stroke}%
\end{pgfscope}%
\begin{pgfscope}%
\pgfsetbuttcap%
\pgfsetroundjoin%
\definecolor{currentfill}{rgb}{0.000000,0.000000,0.000000}%
\pgfsetfillcolor{currentfill}%
\pgfsetlinewidth{0.803000pt}%
\definecolor{currentstroke}{rgb}{0.000000,0.000000,0.000000}%
\pgfsetstrokecolor{currentstroke}%
\pgfsetdash{}{0pt}%
\pgfsys@defobject{currentmarker}{\pgfqpoint{0.000000in}{-0.048611in}}{\pgfqpoint{0.000000in}{0.000000in}}{%
\pgfpathmoveto{\pgfqpoint{0.000000in}{0.000000in}}%
\pgfpathlineto{\pgfqpoint{0.000000in}{-0.048611in}}%
\pgfusepath{stroke,fill}%
}%
\begin{pgfscope}%
\pgfsys@transformshift{2.300874in}{0.225639in}%
\pgfsys@useobject{currentmarker}{}%
\end{pgfscope}%
\end{pgfscope}%
\begin{pgfscope}%
\pgfpathrectangle{\pgfqpoint{0.568671in}{0.225639in}}{\pgfqpoint{2.598305in}{0.455152in}}%
\pgfusepath{clip}%
\pgfsetrectcap%
\pgfsetroundjoin%
\pgfsetlinewidth{0.803000pt}%
\definecolor{currentstroke}{rgb}{0.690196,0.690196,0.690196}%
\pgfsetstrokecolor{currentstroke}%
\pgfsetdash{}{0pt}%
\pgfpathmoveto{\pgfqpoint{2.733925in}{0.225639in}}%
\pgfpathlineto{\pgfqpoint{2.733925in}{0.680791in}}%
\pgfusepath{stroke}%
\end{pgfscope}%
\begin{pgfscope}%
\pgfsetbuttcap%
\pgfsetroundjoin%
\definecolor{currentfill}{rgb}{0.000000,0.000000,0.000000}%
\pgfsetfillcolor{currentfill}%
\pgfsetlinewidth{0.803000pt}%
\definecolor{currentstroke}{rgb}{0.000000,0.000000,0.000000}%
\pgfsetstrokecolor{currentstroke}%
\pgfsetdash{}{0pt}%
\pgfsys@defobject{currentmarker}{\pgfqpoint{0.000000in}{-0.048611in}}{\pgfqpoint{0.000000in}{0.000000in}}{%
\pgfpathmoveto{\pgfqpoint{0.000000in}{0.000000in}}%
\pgfpathlineto{\pgfqpoint{0.000000in}{-0.048611in}}%
\pgfusepath{stroke,fill}%
}%
\begin{pgfscope}%
\pgfsys@transformshift{2.733925in}{0.225639in}%
\pgfsys@useobject{currentmarker}{}%
\end{pgfscope}%
\end{pgfscope}%
\begin{pgfscope}%
\pgfpathrectangle{\pgfqpoint{0.568671in}{0.225639in}}{\pgfqpoint{2.598305in}{0.455152in}}%
\pgfusepath{clip}%
\pgfsetrectcap%
\pgfsetroundjoin%
\pgfsetlinewidth{0.803000pt}%
\definecolor{currentstroke}{rgb}{0.690196,0.690196,0.690196}%
\pgfsetstrokecolor{currentstroke}%
\pgfsetdash{}{0pt}%
\pgfpathmoveto{\pgfqpoint{3.166976in}{0.225639in}}%
\pgfpathlineto{\pgfqpoint{3.166976in}{0.680791in}}%
\pgfusepath{stroke}%
\end{pgfscope}%
\begin{pgfscope}%
\pgfsetbuttcap%
\pgfsetroundjoin%
\definecolor{currentfill}{rgb}{0.000000,0.000000,0.000000}%
\pgfsetfillcolor{currentfill}%
\pgfsetlinewidth{0.803000pt}%
\definecolor{currentstroke}{rgb}{0.000000,0.000000,0.000000}%
\pgfsetstrokecolor{currentstroke}%
\pgfsetdash{}{0pt}%
\pgfsys@defobject{currentmarker}{\pgfqpoint{0.000000in}{-0.048611in}}{\pgfqpoint{0.000000in}{0.000000in}}{%
\pgfpathmoveto{\pgfqpoint{0.000000in}{0.000000in}}%
\pgfpathlineto{\pgfqpoint{0.000000in}{-0.048611in}}%
\pgfusepath{stroke,fill}%
}%
\begin{pgfscope}%
\pgfsys@transformshift{3.166976in}{0.225639in}%
\pgfsys@useobject{currentmarker}{}%
\end{pgfscope}%
\end{pgfscope}%
\begin{pgfscope}%
\pgfpathrectangle{\pgfqpoint{0.568671in}{0.225639in}}{\pgfqpoint{2.598305in}{0.455152in}}%
\pgfusepath{clip}%
\pgfsetrectcap%
\pgfsetroundjoin%
\pgfsetlinewidth{0.803000pt}%
\definecolor{currentstroke}{rgb}{0.690196,0.690196,0.690196}%
\pgfsetstrokecolor{currentstroke}%
\pgfsetdash{}{0pt}%
\pgfpathmoveto{\pgfqpoint{0.568671in}{0.225639in}}%
\pgfpathlineto{\pgfqpoint{3.166976in}{0.225639in}}%
\pgfusepath{stroke}%
\end{pgfscope}%
\begin{pgfscope}%
\pgfsetbuttcap%
\pgfsetroundjoin%
\definecolor{currentfill}{rgb}{0.000000,0.000000,0.000000}%
\pgfsetfillcolor{currentfill}%
\pgfsetlinewidth{0.803000pt}%
\definecolor{currentstroke}{rgb}{0.000000,0.000000,0.000000}%
\pgfsetstrokecolor{currentstroke}%
\pgfsetdash{}{0pt}%
\pgfsys@defobject{currentmarker}{\pgfqpoint{-0.048611in}{0.000000in}}{\pgfqpoint{-0.000000in}{0.000000in}}{%
\pgfpathmoveto{\pgfqpoint{-0.000000in}{0.000000in}}%
\pgfpathlineto{\pgfqpoint{-0.048611in}{0.000000in}}%
\pgfusepath{stroke,fill}%
}%
\begin{pgfscope}%
\pgfsys@transformshift{0.568671in}{0.225639in}%
\pgfsys@useobject{currentmarker}{}%
\end{pgfscope}%
\end{pgfscope}%
\begin{pgfscope}%
\definecolor{textcolor}{rgb}{0.000000,0.000000,0.000000}%
\pgfsetstrokecolor{textcolor}%
\pgfsetfillcolor{textcolor}%
\pgftext[x=0.407213in, y=0.182264in, left, base]{\color{textcolor}\rmfamily\fontsize{9.000000}{10.800000}\selectfont \(\displaystyle {0}\)}%
\end{pgfscope}%
\begin{pgfscope}%
\pgfpathrectangle{\pgfqpoint{0.568671in}{0.225639in}}{\pgfqpoint{2.598305in}{0.455152in}}%
\pgfusepath{clip}%
\pgfsetrectcap%
\pgfsetroundjoin%
\pgfsetlinewidth{0.803000pt}%
\definecolor{currentstroke}{rgb}{0.690196,0.690196,0.690196}%
\pgfsetstrokecolor{currentstroke}%
\pgfsetdash{}{0pt}%
\pgfpathmoveto{\pgfqpoint{0.568671in}{0.680791in}}%
\pgfpathlineto{\pgfqpoint{3.166976in}{0.680791in}}%
\pgfusepath{stroke}%
\end{pgfscope}%
\begin{pgfscope}%
\pgfsetbuttcap%
\pgfsetroundjoin%
\definecolor{currentfill}{rgb}{0.000000,0.000000,0.000000}%
\pgfsetfillcolor{currentfill}%
\pgfsetlinewidth{0.803000pt}%
\definecolor{currentstroke}{rgb}{0.000000,0.000000,0.000000}%
\pgfsetstrokecolor{currentstroke}%
\pgfsetdash{}{0pt}%
\pgfsys@defobject{currentmarker}{\pgfqpoint{-0.048611in}{0.000000in}}{\pgfqpoint{-0.000000in}{0.000000in}}{%
\pgfpathmoveto{\pgfqpoint{-0.000000in}{0.000000in}}%
\pgfpathlineto{\pgfqpoint{-0.048611in}{0.000000in}}%
\pgfusepath{stroke,fill}%
}%
\begin{pgfscope}%
\pgfsys@transformshift{0.568671in}{0.680791in}%
\pgfsys@useobject{currentmarker}{}%
\end{pgfscope}%
\end{pgfscope}%
\begin{pgfscope}%
\definecolor{textcolor}{rgb}{0.000000,0.000000,0.000000}%
\pgfsetstrokecolor{textcolor}%
\pgfsetfillcolor{textcolor}%
\pgftext[x=0.407213in, y=0.637416in, left, base]{\color{textcolor}\rmfamily\fontsize{9.000000}{10.800000}\selectfont \(\displaystyle {1}\)}%
\end{pgfscope}%
\begin{pgfscope}%
\definecolor{textcolor}{rgb}{0.000000,0.000000,0.000000}%
\pgfsetstrokecolor{textcolor}%
\pgfsetfillcolor{textcolor}%
\pgftext[x=0.351658in,y=0.453215in,,bottom,rotate=90.000000]{\color{textcolor}\rmfamily\fontsize{9.000000}{10.800000}\selectfont \(\displaystyle \|\mathbf{h}_i\|\) [1]}%
\end{pgfscope}%
\begin{pgfscope}%
\pgfpathrectangle{\pgfqpoint{0.568671in}{0.225639in}}{\pgfqpoint{2.598305in}{0.455152in}}%
\pgfusepath{clip}%
\pgfsetbuttcap%
\pgfsetroundjoin%
\pgfsetlinewidth{0.752812pt}%
\definecolor{currentstroke}{rgb}{0.000000,0.000000,0.000000}%
\pgfsetstrokecolor{currentstroke}%
\pgfsetdash{{2.775000pt}{1.200000pt}}{0.000000pt}%
\pgfpathmoveto{\pgfqpoint{1.434773in}{0.225639in}}%
\pgfpathlineto{\pgfqpoint{1.434773in}{0.680791in}}%
\pgfusepath{stroke}%
\end{pgfscope}%
\begin{pgfscope}%
\pgfpathrectangle{\pgfqpoint{0.568671in}{0.225639in}}{\pgfqpoint{2.598305in}{0.455152in}}%
\pgfusepath{clip}%
\pgfsetbuttcap%
\pgfsetroundjoin%
\pgfsetlinewidth{0.752812pt}%
\definecolor{currentstroke}{rgb}{0.000000,0.000000,0.000000}%
\pgfsetstrokecolor{currentstroke}%
\pgfsetdash{{2.775000pt}{1.200000pt}}{0.000000pt}%
\pgfpathmoveto{\pgfqpoint{2.300874in}{0.225639in}}%
\pgfpathlineto{\pgfqpoint{2.300874in}{0.680791in}}%
\pgfusepath{stroke}%
\end{pgfscope}%
\begin{pgfscope}%
\pgfpathrectangle{\pgfqpoint{0.568671in}{0.225639in}}{\pgfqpoint{2.598305in}{0.455152in}}%
\pgfusepath{clip}%
\pgfsetrectcap%
\pgfsetroundjoin%
\pgfsetlinewidth{1.505625pt}%
\definecolor{currentstroke}{rgb}{0.121569,0.466667,0.705882}%
\pgfsetstrokecolor{currentstroke}%
\pgfsetdash{}{0pt}%
\pgfpathmoveto{\pgfqpoint{0.568671in}{0.288838in}}%
\pgfpathlineto{\pgfqpoint{0.568844in}{0.256893in}}%
\pgfpathlineto{\pgfqpoint{0.569537in}{0.409088in}}%
\pgfpathlineto{\pgfqpoint{0.571443in}{0.639418in}}%
\pgfpathlineto{\pgfqpoint{0.571789in}{0.637623in}}%
\pgfpathlineto{\pgfqpoint{0.575773in}{0.617439in}}%
\pgfpathlineto{\pgfqpoint{0.576639in}{0.620773in}}%
\pgfpathlineto{\pgfqpoint{0.579584in}{0.626742in}}%
\pgfpathlineto{\pgfqpoint{0.582702in}{0.631541in}}%
\pgfpathlineto{\pgfqpoint{0.584954in}{0.633312in}}%
\pgfpathlineto{\pgfqpoint{0.588938in}{0.636520in}}%
\pgfpathlineto{\pgfqpoint{0.589458in}{0.635687in}}%
\pgfpathlineto{\pgfqpoint{0.593442in}{0.628518in}}%
\pgfpathlineto{\pgfqpoint{0.594308in}{0.629732in}}%
\pgfpathlineto{\pgfqpoint{0.596040in}{0.632081in}}%
\pgfpathlineto{\pgfqpoint{0.596733in}{0.631632in}}%
\pgfpathlineto{\pgfqpoint{0.598465in}{0.627919in}}%
\pgfpathlineto{\pgfqpoint{0.604354in}{0.604157in}}%
\pgfpathlineto{\pgfqpoint{0.607992in}{0.593263in}}%
\pgfpathlineto{\pgfqpoint{0.613016in}{0.586656in}}%
\pgfpathlineto{\pgfqpoint{0.619598in}{0.569540in}}%
\pgfpathlineto{\pgfqpoint{0.621503in}{0.570192in}}%
\pgfpathlineto{\pgfqpoint{0.624448in}{0.575774in}}%
\pgfpathlineto{\pgfqpoint{0.629991in}{0.586202in}}%
\pgfpathlineto{\pgfqpoint{0.631377in}{0.583926in}}%
\pgfpathlineto{\pgfqpoint{0.636920in}{0.573623in}}%
\pgfpathlineto{\pgfqpoint{0.640558in}{0.575172in}}%
\pgfpathlineto{\pgfqpoint{0.644715in}{0.576306in}}%
\pgfpathlineto{\pgfqpoint{0.646274in}{0.573564in}}%
\pgfpathlineto{\pgfqpoint{0.655281in}{0.550975in}}%
\pgfpathlineto{\pgfqpoint{0.663942in}{0.543407in}}%
\pgfpathlineto{\pgfqpoint{0.666714in}{0.546291in}}%
\pgfpathlineto{\pgfqpoint{0.681091in}{0.565805in}}%
\pgfpathlineto{\pgfqpoint{0.685248in}{0.566589in}}%
\pgfpathlineto{\pgfqpoint{0.688713in}{0.567074in}}%
\pgfpathlineto{\pgfqpoint{0.695122in}{0.570161in}}%
\pgfpathlineto{\pgfqpoint{0.698240in}{0.567364in}}%
\pgfpathlineto{\pgfqpoint{0.701878in}{0.564621in}}%
\pgfpathlineto{\pgfqpoint{0.703956in}{0.565823in}}%
\pgfpathlineto{\pgfqpoint{0.712964in}{0.572770in}}%
\pgfpathlineto{\pgfqpoint{0.716774in}{0.571660in}}%
\pgfpathlineto{\pgfqpoint{0.723357in}{0.571077in}}%
\pgfpathlineto{\pgfqpoint{0.728900in}{0.571841in}}%
\pgfpathlineto{\pgfqpoint{0.737214in}{0.572943in}}%
\pgfpathlineto{\pgfqpoint{0.747954in}{0.569352in}}%
\pgfpathlineto{\pgfqpoint{0.759733in}{0.559811in}}%
\pgfpathlineto{\pgfqpoint{0.770473in}{0.558522in}}%
\pgfpathlineto{\pgfqpoint{0.779480in}{0.560062in}}%
\pgfpathlineto{\pgfqpoint{0.786063in}{0.567826in}}%
\pgfpathlineto{\pgfqpoint{0.792645in}{0.573947in}}%
\pgfpathlineto{\pgfqpoint{0.795243in}{0.573706in}}%
\pgfpathlineto{\pgfqpoint{0.801306in}{0.569693in}}%
\pgfpathlineto{\pgfqpoint{0.805290in}{0.568602in}}%
\pgfpathlineto{\pgfqpoint{0.816030in}{0.565921in}}%
\pgfpathlineto{\pgfqpoint{0.822612in}{0.564723in}}%
\pgfpathlineto{\pgfqpoint{0.834564in}{0.568779in}}%
\pgfpathlineto{\pgfqpoint{0.839761in}{0.566887in}}%
\pgfpathlineto{\pgfqpoint{0.851713in}{0.559232in}}%
\pgfpathlineto{\pgfqpoint{0.856390in}{0.560565in}}%
\pgfpathlineto{\pgfqpoint{0.872153in}{0.564973in}}%
\pgfpathlineto{\pgfqpoint{0.879948in}{0.563077in}}%
\pgfpathlineto{\pgfqpoint{0.884452in}{0.561883in}}%
\pgfpathlineto{\pgfqpoint{0.893979in}{0.563554in}}%
\pgfpathlineto{\pgfqpoint{0.905585in}{0.567723in}}%
\pgfpathlineto{\pgfqpoint{0.925678in}{0.561281in}}%
\pgfpathlineto{\pgfqpoint{0.934686in}{0.555174in}}%
\pgfpathlineto{\pgfqpoint{0.943520in}{0.548216in}}%
\pgfpathlineto{\pgfqpoint{0.949236in}{0.546384in}}%
\pgfpathlineto{\pgfqpoint{0.955819in}{0.548123in}}%
\pgfpathlineto{\pgfqpoint{0.960842in}{0.551826in}}%
\pgfpathlineto{\pgfqpoint{0.967078in}{0.560562in}}%
\pgfpathlineto{\pgfqpoint{0.976259in}{0.573490in}}%
\pgfpathlineto{\pgfqpoint{0.982148in}{0.575225in}}%
\pgfpathlineto{\pgfqpoint{0.991502in}{0.569667in}}%
\pgfpathlineto{\pgfqpoint{1.007958in}{0.558112in}}%
\pgfpathlineto{\pgfqpoint{1.012808in}{0.559430in}}%
\pgfpathlineto{\pgfqpoint{1.019390in}{0.564621in}}%
\pgfpathlineto{\pgfqpoint{1.033421in}{0.575340in}}%
\pgfpathlineto{\pgfqpoint{1.038445in}{0.575312in}}%
\pgfpathlineto{\pgfqpoint{1.043122in}{0.572082in}}%
\pgfpathlineto{\pgfqpoint{1.057152in}{0.559594in}}%
\pgfpathlineto{\pgfqpoint{1.065467in}{0.553239in}}%
\pgfpathlineto{\pgfqpoint{1.082269in}{0.552960in}}%
\pgfpathlineto{\pgfqpoint{1.102883in}{0.568199in}}%
\pgfpathlineto{\pgfqpoint{1.108772in}{0.568921in}}%
\pgfpathlineto{\pgfqpoint{1.114488in}{0.566999in}}%
\pgfpathlineto{\pgfqpoint{1.128173in}{0.556173in}}%
\pgfpathlineto{\pgfqpoint{1.138393in}{0.555223in}}%
\pgfpathlineto{\pgfqpoint{1.144109in}{0.557213in}}%
\pgfpathlineto{\pgfqpoint{1.166454in}{0.567588in}}%
\pgfpathlineto{\pgfqpoint{1.182564in}{0.569220in}}%
\pgfpathlineto{\pgfqpoint{1.193650in}{0.567058in}}%
\pgfpathlineto{\pgfqpoint{1.200059in}{0.564004in}}%
\pgfpathlineto{\pgfqpoint{1.204736in}{0.564284in}}%
\pgfpathlineto{\pgfqpoint{1.210106in}{0.567114in}}%
\pgfpathlineto{\pgfqpoint{1.219113in}{0.570832in}}%
\pgfpathlineto{\pgfqpoint{1.225869in}{0.571976in}}%
\pgfpathlineto{\pgfqpoint{1.232451in}{0.570033in}}%
\pgfpathlineto{\pgfqpoint{1.237302in}{0.568991in}}%
\pgfpathlineto{\pgfqpoint{1.244750in}{0.566427in}}%
\pgfpathlineto{\pgfqpoint{1.253584in}{0.560144in}}%
\pgfpathlineto{\pgfqpoint{1.271946in}{0.560071in}}%
\pgfpathlineto{\pgfqpoint{1.288402in}{0.566895in}}%
\pgfpathlineto{\pgfqpoint{1.300180in}{0.565923in}}%
\pgfpathlineto{\pgfqpoint{1.307109in}{0.563416in}}%
\pgfpathlineto{\pgfqpoint{1.316636in}{0.558469in}}%
\pgfpathlineto{\pgfqpoint{1.320967in}{0.557368in}}%
\pgfpathlineto{\pgfqpoint{1.332399in}{0.563869in}}%
\pgfpathlineto{\pgfqpoint{1.336384in}{0.564798in}}%
\pgfpathlineto{\pgfqpoint{1.374319in}{0.564209in}}%
\pgfpathlineto{\pgfqpoint{1.379515in}{0.564864in}}%
\pgfpathlineto{\pgfqpoint{1.386617in}{0.566615in}}%
\pgfpathlineto{\pgfqpoint{1.395798in}{0.568184in}}%
\pgfpathlineto{\pgfqpoint{1.399609in}{0.567680in}}%
\pgfpathlineto{\pgfqpoint{1.410695in}{0.563201in}}%
\pgfpathlineto{\pgfqpoint{1.414852in}{0.561109in}}%
\pgfpathlineto{\pgfqpoint{1.425072in}{0.555432in}}%
\pgfpathlineto{\pgfqpoint{1.430442in}{0.557763in}}%
\pgfpathlineto{\pgfqpoint{1.434253in}{0.559149in}}%
\pgfpathlineto{\pgfqpoint{1.434426in}{0.558745in}}%
\pgfpathlineto{\pgfqpoint{1.435119in}{0.552004in}}%
\pgfpathlineto{\pgfqpoint{1.438584in}{0.476173in}}%
\pgfpathlineto{\pgfqpoint{1.442221in}{0.430308in}}%
\pgfpathlineto{\pgfqpoint{1.445339in}{0.422142in}}%
\pgfpathlineto{\pgfqpoint{1.446898in}{0.422772in}}%
\pgfpathlineto{\pgfqpoint{1.448804in}{0.427971in}}%
\pgfpathlineto{\pgfqpoint{1.451229in}{0.446409in}}%
\pgfpathlineto{\pgfqpoint{1.454520in}{0.500615in}}%
\pgfpathlineto{\pgfqpoint{1.461622in}{0.620913in}}%
\pgfpathlineto{\pgfqpoint{1.464913in}{0.631767in}}%
\pgfpathlineto{\pgfqpoint{1.467338in}{0.632809in}}%
\pgfpathlineto{\pgfqpoint{1.472361in}{0.629274in}}%
\pgfpathlineto{\pgfqpoint{1.475133in}{0.622574in}}%
\pgfpathlineto{\pgfqpoint{1.477905in}{0.608044in}}%
\pgfpathlineto{\pgfqpoint{1.484660in}{0.551368in}}%
\pgfpathlineto{\pgfqpoint{1.495919in}{0.453816in}}%
\pgfpathlineto{\pgfqpoint{1.499557in}{0.445870in}}%
\pgfpathlineto{\pgfqpoint{1.501809in}{0.446487in}}%
\pgfpathlineto{\pgfqpoint{1.504754in}{0.451519in}}%
\pgfpathlineto{\pgfqpoint{1.510124in}{0.468413in}}%
\pgfpathlineto{\pgfqpoint{1.517052in}{0.503660in}}%
\pgfpathlineto{\pgfqpoint{1.527619in}{0.552998in}}%
\pgfpathlineto{\pgfqpoint{1.533681in}{0.566573in}}%
\pgfpathlineto{\pgfqpoint{1.537146in}{0.568091in}}%
\pgfpathlineto{\pgfqpoint{1.540610in}{0.563937in}}%
\pgfpathlineto{\pgfqpoint{1.546500in}{0.549049in}}%
\pgfpathlineto{\pgfqpoint{1.566420in}{0.491230in}}%
\pgfpathlineto{\pgfqpoint{1.573869in}{0.484058in}}%
\pgfpathlineto{\pgfqpoint{1.578026in}{0.484341in}}%
\pgfpathlineto{\pgfqpoint{1.583049in}{0.489640in}}%
\pgfpathlineto{\pgfqpoint{1.588766in}{0.500097in}}%
\pgfpathlineto{\pgfqpoint{1.607300in}{0.529407in}}%
\pgfpathlineto{\pgfqpoint{1.620985in}{0.536222in}}%
\pgfpathlineto{\pgfqpoint{1.625488in}{0.533935in}}%
\pgfpathlineto{\pgfqpoint{1.639346in}{0.521466in}}%
\pgfpathlineto{\pgfqpoint{1.650952in}{0.507509in}}%
\pgfpathlineto{\pgfqpoint{1.655802in}{0.505134in}}%
\pgfpathlineto{\pgfqpoint{1.664116in}{0.503606in}}%
\pgfpathlineto{\pgfqpoint{1.678494in}{0.504428in}}%
\pgfpathlineto{\pgfqpoint{1.687848in}{0.510485in}}%
\pgfpathlineto{\pgfqpoint{1.697721in}{0.518144in}}%
\pgfpathlineto{\pgfqpoint{1.720240in}{0.524730in}}%
\pgfpathlineto{\pgfqpoint{1.731153in}{0.522716in}}%
\pgfpathlineto{\pgfqpoint{1.758695in}{0.510819in}}%
\pgfpathlineto{\pgfqpoint{1.789701in}{0.512680in}}%
\pgfpathlineto{\pgfqpoint{1.796803in}{0.516224in}}%
\pgfpathlineto{\pgfqpoint{1.807543in}{0.519740in}}%
\pgfpathlineto{\pgfqpoint{1.813259in}{0.520265in}}%
\pgfpathlineto{\pgfqpoint{1.818109in}{0.520148in}}%
\pgfpathlineto{\pgfqpoint{1.831967in}{0.517921in}}%
\pgfpathlineto{\pgfqpoint{1.839762in}{0.518565in}}%
\pgfpathlineto{\pgfqpoint{1.849462in}{0.517178in}}%
\pgfpathlineto{\pgfqpoint{1.865052in}{0.512700in}}%
\pgfpathlineto{\pgfqpoint{1.897617in}{0.505135in}}%
\pgfpathlineto{\pgfqpoint{1.900735in}{0.506388in}}%
\pgfpathlineto{\pgfqpoint{1.908011in}{0.510598in}}%
\pgfpathlineto{\pgfqpoint{1.915632in}{0.515940in}}%
\pgfpathlineto{\pgfqpoint{1.920829in}{0.518356in}}%
\pgfpathlineto{\pgfqpoint{1.939710in}{0.524405in}}%
\pgfpathlineto{\pgfqpoint{1.950623in}{0.519339in}}%
\pgfpathlineto{\pgfqpoint{1.956512in}{0.516579in}}%
\pgfpathlineto{\pgfqpoint{1.963441in}{0.514024in}}%
\pgfpathlineto{\pgfqpoint{1.970197in}{0.512718in}}%
\pgfpathlineto{\pgfqpoint{1.973315in}{0.512214in}}%
\pgfpathlineto{\pgfqpoint{1.983881in}{0.513390in}}%
\pgfpathlineto{\pgfqpoint{2.004148in}{0.520826in}}%
\pgfpathlineto{\pgfqpoint{2.014195in}{0.521190in}}%
\pgfpathlineto{\pgfqpoint{2.028399in}{0.517895in}}%
\pgfpathlineto{\pgfqpoint{2.034461in}{0.514843in}}%
\pgfpathlineto{\pgfqpoint{2.050571in}{0.507572in}}%
\pgfpathlineto{\pgfqpoint{2.056807in}{0.508687in}}%
\pgfpathlineto{\pgfqpoint{2.068586in}{0.510683in}}%
\pgfpathlineto{\pgfqpoint{2.077940in}{0.516763in}}%
\pgfpathlineto{\pgfqpoint{2.085735in}{0.521823in}}%
\pgfpathlineto{\pgfqpoint{2.095781in}{0.524800in}}%
\pgfpathlineto{\pgfqpoint{2.105655in}{0.523514in}}%
\pgfpathlineto{\pgfqpoint{2.115529in}{0.518763in}}%
\pgfpathlineto{\pgfqpoint{2.135795in}{0.508931in}}%
\pgfpathlineto{\pgfqpoint{2.152251in}{0.509857in}}%
\pgfpathlineto{\pgfqpoint{2.163337in}{0.512008in}}%
\pgfpathlineto{\pgfqpoint{2.167321in}{0.513007in}}%
\pgfpathlineto{\pgfqpoint{2.193304in}{0.520698in}}%
\pgfpathlineto{\pgfqpoint{2.206989in}{0.521163in}}%
\pgfpathlineto{\pgfqpoint{2.219461in}{0.519576in}}%
\pgfpathlineto{\pgfqpoint{2.228122in}{0.517499in}}%
\pgfpathlineto{\pgfqpoint{2.245963in}{0.510884in}}%
\pgfpathlineto{\pgfqpoint{2.262593in}{0.507221in}}%
\pgfpathlineto{\pgfqpoint{2.278009in}{0.515491in}}%
\pgfpathlineto{\pgfqpoint{2.287883in}{0.521103in}}%
\pgfpathlineto{\pgfqpoint{2.298796in}{0.525289in}}%
\pgfpathlineto{\pgfqpoint{2.300701in}{0.523645in}}%
\pgfpathlineto{\pgfqpoint{2.302087in}{0.489843in}}%
\pgfpathlineto{\pgfqpoint{2.306937in}{0.366092in}}%
\pgfpathlineto{\pgfqpoint{2.309362in}{0.358804in}}%
\pgfpathlineto{\pgfqpoint{2.310575in}{0.360118in}}%
\pgfpathlineto{\pgfqpoint{2.312307in}{0.368347in}}%
\pgfpathlineto{\pgfqpoint{2.314559in}{0.397868in}}%
\pgfpathlineto{\pgfqpoint{2.318543in}{0.508601in}}%
\pgfpathlineto{\pgfqpoint{2.322180in}{0.571588in}}%
\pgfpathlineto{\pgfqpoint{2.325645in}{0.583383in}}%
\pgfpathlineto{\pgfqpoint{2.328763in}{0.586035in}}%
\pgfpathlineto{\pgfqpoint{2.331707in}{0.585265in}}%
\pgfpathlineto{\pgfqpoint{2.334825in}{0.579309in}}%
\pgfpathlineto{\pgfqpoint{2.343140in}{0.556941in}}%
\pgfpathlineto{\pgfqpoint{2.347297in}{0.529524in}}%
\pgfpathlineto{\pgfqpoint{2.363580in}{0.402044in}}%
\pgfpathlineto{\pgfqpoint{2.366178in}{0.401713in}}%
\pgfpathlineto{\pgfqpoint{2.371548in}{0.407851in}}%
\pgfpathlineto{\pgfqpoint{2.376052in}{0.417380in}}%
\pgfpathlineto{\pgfqpoint{2.380209in}{0.432619in}}%
\pgfpathlineto{\pgfqpoint{2.395799in}{0.495533in}}%
\pgfpathlineto{\pgfqpoint{2.399263in}{0.497225in}}%
\pgfpathlineto{\pgfqpoint{2.403074in}{0.495220in}}%
\pgfpathlineto{\pgfqpoint{2.407405in}{0.489651in}}%
\pgfpathlineto{\pgfqpoint{2.415026in}{0.469801in}}%
\pgfpathlineto{\pgfqpoint{2.425593in}{0.443982in}}%
\pgfpathlineto{\pgfqpoint{2.431482in}{0.436679in}}%
\pgfpathlineto{\pgfqpoint{2.436506in}{0.433666in}}%
\pgfpathlineto{\pgfqpoint{2.443781in}{0.435168in}}%
\pgfpathlineto{\pgfqpoint{2.451229in}{0.443961in}}%
\pgfpathlineto{\pgfqpoint{2.460064in}{0.455424in}}%
\pgfpathlineto{\pgfqpoint{2.465607in}{0.459725in}}%
\pgfpathlineto{\pgfqpoint{2.474268in}{0.467918in}}%
\pgfpathlineto{\pgfqpoint{2.479464in}{0.470650in}}%
\pgfpathlineto{\pgfqpoint{2.486740in}{0.470723in}}%
\pgfpathlineto{\pgfqpoint{2.496094in}{0.464436in}}%
\pgfpathlineto{\pgfqpoint{2.504928in}{0.453932in}}%
\pgfpathlineto{\pgfqpoint{2.515667in}{0.441193in}}%
\pgfpathlineto{\pgfqpoint{2.522769in}{0.436979in}}%
\pgfpathlineto{\pgfqpoint{2.526754in}{0.437343in}}%
\pgfpathlineto{\pgfqpoint{2.534895in}{0.442535in}}%
\pgfpathlineto{\pgfqpoint{2.558799in}{0.465553in}}%
\pgfpathlineto{\pgfqpoint{2.563823in}{0.466786in}}%
\pgfpathlineto{\pgfqpoint{2.568326in}{0.466129in}}%
\pgfpathlineto{\pgfqpoint{2.591711in}{0.454894in}}%
\pgfpathlineto{\pgfqpoint{2.599160in}{0.450092in}}%
\pgfpathlineto{\pgfqpoint{2.625143in}{0.447690in}}%
\pgfpathlineto{\pgfqpoint{2.660653in}{0.457476in}}%
\pgfpathlineto{\pgfqpoint{2.664810in}{0.458401in}}%
\pgfpathlineto{\pgfqpoint{2.672085in}{0.457857in}}%
\pgfpathlineto{\pgfqpoint{2.686463in}{0.455688in}}%
\pgfpathlineto{\pgfqpoint{2.691486in}{0.456086in}}%
\pgfpathlineto{\pgfqpoint{2.710540in}{0.451441in}}%
\pgfpathlineto{\pgfqpoint{2.718162in}{0.450415in}}%
\pgfpathlineto{\pgfqpoint{2.729595in}{0.451315in}}%
\pgfpathlineto{\pgfqpoint{2.739815in}{0.450884in}}%
\pgfpathlineto{\pgfqpoint{2.777750in}{0.454624in}}%
\pgfpathlineto{\pgfqpoint{2.788489in}{0.456782in}}%
\pgfpathlineto{\pgfqpoint{2.797843in}{0.455235in}}%
\pgfpathlineto{\pgfqpoint{2.806851in}{0.454915in}}%
\pgfpathlineto{\pgfqpoint{2.814819in}{0.455723in}}%
\pgfpathlineto{\pgfqpoint{2.837511in}{0.451320in}}%
\pgfpathlineto{\pgfqpoint{2.846691in}{0.450439in}}%
\pgfpathlineto{\pgfqpoint{2.869383in}{0.454240in}}%
\pgfpathlineto{\pgfqpoint{2.875792in}{0.456209in}}%
\pgfpathlineto{\pgfqpoint{2.886186in}{0.457393in}}%
\pgfpathlineto{\pgfqpoint{2.892248in}{0.457112in}}%
\pgfpathlineto{\pgfqpoint{2.922562in}{0.449523in}}%
\pgfpathlineto{\pgfqpoint{2.953742in}{0.450647in}}%
\pgfpathlineto{\pgfqpoint{2.959285in}{0.452468in}}%
\pgfpathlineto{\pgfqpoint{2.967426in}{0.455583in}}%
\pgfpathlineto{\pgfqpoint{2.976433in}{0.458883in}}%
\pgfpathlineto{\pgfqpoint{2.986480in}{0.457102in}}%
\pgfpathlineto{\pgfqpoint{3.005708in}{0.453989in}}%
\pgfpathlineto{\pgfqpoint{3.014195in}{0.452631in}}%
\pgfpathlineto{\pgfqpoint{3.019912in}{0.451183in}}%
\pgfpathlineto{\pgfqpoint{3.046588in}{0.448968in}}%
\pgfpathlineto{\pgfqpoint{3.058020in}{0.452746in}}%
\pgfpathlineto{\pgfqpoint{3.065295in}{0.455398in}}%
\pgfpathlineto{\pgfqpoint{3.075515in}{0.459215in}}%
\pgfpathlineto{\pgfqpoint{3.095263in}{0.456479in}}%
\pgfpathlineto{\pgfqpoint{3.119167in}{0.450099in}}%
\pgfpathlineto{\pgfqpoint{3.126096in}{0.450165in}}%
\pgfpathlineto{\pgfqpoint{3.136835in}{0.449921in}}%
\pgfpathlineto{\pgfqpoint{3.142379in}{0.449110in}}%
\pgfpathlineto{\pgfqpoint{3.152079in}{0.448370in}}%
\pgfpathlineto{\pgfqpoint{3.160913in}{0.450636in}}%
\pgfpathlineto{\pgfqpoint{3.166456in}{0.452672in}}%
\pgfpathlineto{\pgfqpoint{3.166456in}{0.452672in}}%
\pgfusepath{stroke}%
\end{pgfscope}%
\begin{pgfscope}%
\pgfpathrectangle{\pgfqpoint{0.568671in}{0.225639in}}{\pgfqpoint{2.598305in}{0.455152in}}%
\pgfusepath{clip}%
\pgfsetrectcap%
\pgfsetroundjoin%
\pgfsetlinewidth{1.505625pt}%
\definecolor{currentstroke}{rgb}{1.000000,0.498039,0.054902}%
\pgfsetstrokecolor{currentstroke}%
\pgfsetdash{}{0pt}%
\pgfpathmoveto{\pgfqpoint{0.568671in}{0.255903in}}%
\pgfpathlineto{\pgfqpoint{0.568844in}{0.232723in}}%
\pgfpathlineto{\pgfqpoint{0.570230in}{0.250796in}}%
\pgfpathlineto{\pgfqpoint{0.570577in}{0.251753in}}%
\pgfpathlineto{\pgfqpoint{0.571269in}{0.249603in}}%
\pgfpathlineto{\pgfqpoint{0.574561in}{0.246043in}}%
\pgfpathlineto{\pgfqpoint{0.577159in}{0.246491in}}%
\pgfpathlineto{\pgfqpoint{0.584434in}{0.248064in}}%
\pgfpathlineto{\pgfqpoint{0.598119in}{0.249088in}}%
\pgfpathlineto{\pgfqpoint{0.617000in}{0.245919in}}%
\pgfpathlineto{\pgfqpoint{0.626873in}{0.245678in}}%
\pgfpathlineto{\pgfqpoint{0.635707in}{0.244896in}}%
\pgfpathlineto{\pgfqpoint{0.671391in}{0.243606in}}%
\pgfpathlineto{\pgfqpoint{0.736002in}{0.243761in}}%
\pgfpathlineto{\pgfqpoint{0.778787in}{0.242657in}}%
\pgfpathlineto{\pgfqpoint{0.807542in}{0.243213in}}%
\pgfpathlineto{\pgfqpoint{0.840800in}{0.242709in}}%
\pgfpathlineto{\pgfqpoint{0.939709in}{0.242271in}}%
\pgfpathlineto{\pgfqpoint{0.958936in}{0.242187in}}%
\pgfpathlineto{\pgfqpoint{0.993407in}{0.243121in}}%
\pgfpathlineto{\pgfqpoint{1.157101in}{0.242689in}}%
\pgfpathlineto{\pgfqpoint{1.244230in}{0.242779in}}%
\pgfpathlineto{\pgfqpoint{1.391468in}{0.242942in}}%
\pgfpathlineto{\pgfqpoint{1.413986in}{0.242795in}}%
\pgfpathlineto{\pgfqpoint{1.435119in}{0.243520in}}%
\pgfpathlineto{\pgfqpoint{1.438930in}{0.248493in}}%
\pgfpathlineto{\pgfqpoint{1.446725in}{0.253169in}}%
\pgfpathlineto{\pgfqpoint{1.450882in}{0.259091in}}%
\pgfpathlineto{\pgfqpoint{1.454693in}{0.271141in}}%
\pgfpathlineto{\pgfqpoint{1.461968in}{0.295373in}}%
\pgfpathlineto{\pgfqpoint{1.466299in}{0.298648in}}%
\pgfpathlineto{\pgfqpoint{1.473747in}{0.299676in}}%
\pgfpathlineto{\pgfqpoint{1.478771in}{0.297204in}}%
\pgfpathlineto{\pgfqpoint{1.485873in}{0.286682in}}%
\pgfpathlineto{\pgfqpoint{1.497305in}{0.268875in}}%
\pgfpathlineto{\pgfqpoint{1.501636in}{0.267950in}}%
\pgfpathlineto{\pgfqpoint{1.507872in}{0.270135in}}%
\pgfpathlineto{\pgfqpoint{1.514281in}{0.275391in}}%
\pgfpathlineto{\pgfqpoint{1.532122in}{0.291287in}}%
\pgfpathlineto{\pgfqpoint{1.537839in}{0.293016in}}%
\pgfpathlineto{\pgfqpoint{1.543901in}{0.291302in}}%
\pgfpathlineto{\pgfqpoint{1.552562in}{0.285715in}}%
\pgfpathlineto{\pgfqpoint{1.565381in}{0.278524in}}%
\pgfpathlineto{\pgfqpoint{1.577333in}{0.276079in}}%
\pgfpathlineto{\pgfqpoint{1.586340in}{0.278236in}}%
\pgfpathlineto{\pgfqpoint{1.613536in}{0.286941in}}%
\pgfpathlineto{\pgfqpoint{1.626354in}{0.286577in}}%
\pgfpathlineto{\pgfqpoint{1.638999in}{0.284021in}}%
\pgfpathlineto{\pgfqpoint{1.658747in}{0.280322in}}%
\pgfpathlineto{\pgfqpoint{1.682824in}{0.281281in}}%
\pgfpathlineto{\pgfqpoint{1.695989in}{0.283558in}}%
\pgfpathlineto{\pgfqpoint{1.713657in}{0.284672in}}%
\pgfpathlineto{\pgfqpoint{1.796110in}{0.282745in}}%
\pgfpathlineto{\pgfqpoint{1.844439in}{0.284452in}}%
\pgfpathlineto{\pgfqpoint{1.901082in}{0.282686in}}%
\pgfpathlineto{\pgfqpoint{1.937458in}{0.285396in}}%
\pgfpathlineto{\pgfqpoint{1.953394in}{0.284496in}}%
\pgfpathlineto{\pgfqpoint{1.980243in}{0.283585in}}%
\pgfpathlineto{\pgfqpoint{2.026320in}{0.284217in}}%
\pgfpathlineto{\pgfqpoint{2.054555in}{0.282785in}}%
\pgfpathlineto{\pgfqpoint{2.105135in}{0.285352in}}%
\pgfpathlineto{\pgfqpoint{2.125922in}{0.283285in}}%
\pgfpathlineto{\pgfqpoint{2.140299in}{0.282942in}}%
\pgfpathlineto{\pgfqpoint{2.173211in}{0.284410in}}%
\pgfpathlineto{\pgfqpoint{2.190879in}{0.285612in}}%
\pgfpathlineto{\pgfqpoint{2.214091in}{0.284987in}}%
\pgfpathlineto{\pgfqpoint{2.275411in}{0.283594in}}%
\pgfpathlineto{\pgfqpoint{2.292040in}{0.285434in}}%
\pgfpathlineto{\pgfqpoint{2.300874in}{0.284306in}}%
\pgfpathlineto{\pgfqpoint{2.309016in}{0.236599in}}%
\pgfpathlineto{\pgfqpoint{2.312826in}{0.234877in}}%
\pgfpathlineto{\pgfqpoint{2.315771in}{0.236754in}}%
\pgfpathlineto{\pgfqpoint{2.323046in}{0.243186in}}%
\pgfpathlineto{\pgfqpoint{2.343486in}{0.241196in}}%
\pgfpathlineto{\pgfqpoint{2.355958in}{0.235604in}}%
\pgfpathlineto{\pgfqpoint{2.363407in}{0.233584in}}%
\pgfpathlineto{\pgfqpoint{2.376572in}{0.234520in}}%
\pgfpathlineto{\pgfqpoint{2.389390in}{0.237671in}}%
\pgfpathlineto{\pgfqpoint{2.402728in}{0.238811in}}%
\pgfpathlineto{\pgfqpoint{2.459717in}{0.236655in}}%
\pgfpathlineto{\pgfqpoint{2.495574in}{0.237373in}}%
\pgfpathlineto{\pgfqpoint{2.541650in}{0.236478in}}%
\pgfpathlineto{\pgfqpoint{2.568846in}{0.237555in}}%
\pgfpathlineto{\pgfqpoint{2.635016in}{0.236729in}}%
\pgfpathlineto{\pgfqpoint{2.714698in}{0.236921in}}%
\pgfpathlineto{\pgfqpoint{2.756270in}{0.237031in}}%
\pgfpathlineto{\pgfqpoint{3.066162in}{0.237336in}}%
\pgfpathlineto{\pgfqpoint{3.113624in}{0.237165in}}%
\pgfpathlineto{\pgfqpoint{3.157102in}{0.237166in}}%
\pgfpathlineto{\pgfqpoint{3.166456in}{0.237361in}}%
\pgfpathlineto{\pgfqpoint{3.166456in}{0.237361in}}%
\pgfusepath{stroke}%
\end{pgfscope}%
\begin{pgfscope}%
\pgfpathrectangle{\pgfqpoint{0.568671in}{0.225639in}}{\pgfqpoint{2.598305in}{0.455152in}}%
\pgfusepath{clip}%
\pgfsetrectcap%
\pgfsetroundjoin%
\pgfsetlinewidth{1.505625pt}%
\definecolor{currentstroke}{rgb}{0.172549,0.627451,0.172549}%
\pgfsetstrokecolor{currentstroke}%
\pgfsetdash{}{0pt}%
\pgfpathmoveto{\pgfqpoint{0.568671in}{0.269827in}}%
\pgfpathlineto{\pgfqpoint{0.568844in}{0.235630in}}%
\pgfpathlineto{\pgfqpoint{0.569884in}{0.331520in}}%
\pgfpathlineto{\pgfqpoint{0.570923in}{0.389927in}}%
\pgfpathlineto{\pgfqpoint{0.571789in}{0.373718in}}%
\pgfpathlineto{\pgfqpoint{0.571962in}{0.378063in}}%
\pgfpathlineto{\pgfqpoint{0.572828in}{0.366065in}}%
\pgfpathlineto{\pgfqpoint{0.573002in}{0.371523in}}%
\pgfpathlineto{\pgfqpoint{0.574907in}{0.353984in}}%
\pgfpathlineto{\pgfqpoint{0.575080in}{0.354753in}}%
\pgfpathlineto{\pgfqpoint{0.575600in}{0.351117in}}%
\pgfpathlineto{\pgfqpoint{0.575773in}{0.351675in}}%
\pgfpathlineto{\pgfqpoint{0.575946in}{0.350762in}}%
\pgfpathlineto{\pgfqpoint{0.576986in}{0.353296in}}%
\pgfpathlineto{\pgfqpoint{0.577505in}{0.353135in}}%
\pgfpathlineto{\pgfqpoint{0.577852in}{0.352465in}}%
\pgfpathlineto{\pgfqpoint{0.580277in}{0.350839in}}%
\pgfpathlineto{\pgfqpoint{0.589977in}{0.353296in}}%
\pgfpathlineto{\pgfqpoint{0.593788in}{0.354041in}}%
\pgfpathlineto{\pgfqpoint{0.598292in}{0.356825in}}%
\pgfpathlineto{\pgfqpoint{0.618385in}{0.340227in}}%
\pgfpathlineto{\pgfqpoint{0.622543in}{0.337329in}}%
\pgfpathlineto{\pgfqpoint{0.633802in}{0.334068in}}%
\pgfpathlineto{\pgfqpoint{0.637786in}{0.332402in}}%
\pgfpathlineto{\pgfqpoint{0.650778in}{0.329233in}}%
\pgfpathlineto{\pgfqpoint{0.660478in}{0.324894in}}%
\pgfpathlineto{\pgfqpoint{0.666714in}{0.323827in}}%
\pgfpathlineto{\pgfqpoint{0.673469in}{0.325383in}}%
\pgfpathlineto{\pgfqpoint{0.687673in}{0.327162in}}%
\pgfpathlineto{\pgfqpoint{0.708980in}{0.325680in}}%
\pgfpathlineto{\pgfqpoint{0.723010in}{0.326157in}}%
\pgfpathlineto{\pgfqpoint{0.746742in}{0.325192in}}%
\pgfpathlineto{\pgfqpoint{0.775150in}{0.322084in}}%
\pgfpathlineto{\pgfqpoint{0.785023in}{0.324040in}}%
\pgfpathlineto{\pgfqpoint{0.794897in}{0.326416in}}%
\pgfpathlineto{\pgfqpoint{0.815164in}{0.324619in}}%
\pgfpathlineto{\pgfqpoint{0.826076in}{0.323873in}}%
\pgfpathlineto{\pgfqpoint{0.842532in}{0.323606in}}%
\pgfpathlineto{\pgfqpoint{0.852406in}{0.322519in}}%
\pgfpathlineto{\pgfqpoint{0.919269in}{0.325775in}}%
\pgfpathlineto{\pgfqpoint{0.957551in}{0.321731in}}%
\pgfpathlineto{\pgfqpoint{0.985612in}{0.328190in}}%
\pgfpathlineto{\pgfqpoint{1.013328in}{0.324164in}}%
\pgfpathlineto{\pgfqpoint{1.040004in}{0.328174in}}%
\pgfpathlineto{\pgfqpoint{1.080017in}{0.321813in}}%
\pgfpathlineto{\pgfqpoint{1.114662in}{0.326196in}}%
\pgfpathlineto{\pgfqpoint{1.139086in}{0.322922in}}%
\pgfpathlineto{\pgfqpoint{1.160911in}{0.324955in}}%
\pgfpathlineto{\pgfqpoint{1.190532in}{0.327287in}}%
\pgfpathlineto{\pgfqpoint{1.211838in}{0.327015in}}%
\pgfpathlineto{\pgfqpoint{1.234530in}{0.327224in}}%
\pgfpathlineto{\pgfqpoint{1.250466in}{0.325088in}}%
\pgfpathlineto{\pgfqpoint{1.266922in}{0.323792in}}%
\pgfpathlineto{\pgfqpoint{1.276449in}{0.324334in}}%
\pgfpathlineto{\pgfqpoint{1.314038in}{0.325123in}}%
\pgfpathlineto{\pgfqpoint{1.324258in}{0.324368in}}%
\pgfpathlineto{\pgfqpoint{1.337423in}{0.325646in}}%
\pgfpathlineto{\pgfqpoint{1.352839in}{0.325522in}}%
\pgfpathlineto{\pgfqpoint{1.366004in}{0.325357in}}%
\pgfpathlineto{\pgfqpoint{1.386791in}{0.326876in}}%
\pgfpathlineto{\pgfqpoint{1.403420in}{0.326885in}}%
\pgfpathlineto{\pgfqpoint{1.435119in}{0.322653in}}%
\pgfpathlineto{\pgfqpoint{1.438930in}{0.298817in}}%
\pgfpathlineto{\pgfqpoint{1.442741in}{0.285393in}}%
\pgfpathlineto{\pgfqpoint{1.446205in}{0.282959in}}%
\pgfpathlineto{\pgfqpoint{1.448977in}{0.284139in}}%
\pgfpathlineto{\pgfqpoint{1.451402in}{0.289044in}}%
\pgfpathlineto{\pgfqpoint{1.454693in}{0.303934in}}%
\pgfpathlineto{\pgfqpoint{1.461968in}{0.337726in}}%
\pgfpathlineto{\pgfqpoint{1.465779in}{0.340881in}}%
\pgfpathlineto{\pgfqpoint{1.472708in}{0.340533in}}%
\pgfpathlineto{\pgfqpoint{1.476692in}{0.337553in}}%
\pgfpathlineto{\pgfqpoint{1.481542in}{0.328530in}}%
\pgfpathlineto{\pgfqpoint{1.498864in}{0.290030in}}%
\pgfpathlineto{\pgfqpoint{1.503195in}{0.289961in}}%
\pgfpathlineto{\pgfqpoint{1.508565in}{0.293414in}}%
\pgfpathlineto{\pgfqpoint{1.513761in}{0.300054in}}%
\pgfpathlineto{\pgfqpoint{1.529697in}{0.322080in}}%
\pgfpathlineto{\pgfqpoint{1.535587in}{0.325329in}}%
\pgfpathlineto{\pgfqpoint{1.540091in}{0.324835in}}%
\pgfpathlineto{\pgfqpoint{1.548752in}{0.318412in}}%
\pgfpathlineto{\pgfqpoint{1.565208in}{0.304387in}}%
\pgfpathlineto{\pgfqpoint{1.576120in}{0.301029in}}%
\pgfpathlineto{\pgfqpoint{1.582183in}{0.302108in}}%
\pgfpathlineto{\pgfqpoint{1.618733in}{0.316741in}}%
\pgfpathlineto{\pgfqpoint{1.625315in}{0.315684in}}%
\pgfpathlineto{\pgfqpoint{1.663250in}{0.307360in}}%
\pgfpathlineto{\pgfqpoint{1.686635in}{0.309162in}}%
\pgfpathlineto{\pgfqpoint{1.707248in}{0.312555in}}%
\pgfpathlineto{\pgfqpoint{1.736003in}{0.312048in}}%
\pgfpathlineto{\pgfqpoint{1.749860in}{0.309934in}}%
\pgfpathlineto{\pgfqpoint{1.776190in}{0.309601in}}%
\pgfpathlineto{\pgfqpoint{1.852753in}{0.311387in}}%
\pgfpathlineto{\pgfqpoint{1.865398in}{0.310613in}}%
\pgfpathlineto{\pgfqpoint{1.874059in}{0.309901in}}%
\pgfpathlineto{\pgfqpoint{1.885492in}{0.309225in}}%
\pgfpathlineto{\pgfqpoint{1.903680in}{0.309698in}}%
\pgfpathlineto{\pgfqpoint{1.943174in}{0.314036in}}%
\pgfpathlineto{\pgfqpoint{1.987172in}{0.311299in}}%
\pgfpathlineto{\pgfqpoint{2.022336in}{0.312416in}}%
\pgfpathlineto{\pgfqpoint{2.054209in}{0.309389in}}%
\pgfpathlineto{\pgfqpoint{2.069279in}{0.310452in}}%
\pgfpathlineto{\pgfqpoint{2.102537in}{0.313773in}}%
\pgfpathlineto{\pgfqpoint{2.149480in}{0.309648in}}%
\pgfpathlineto{\pgfqpoint{2.179966in}{0.312359in}}%
\pgfpathlineto{\pgfqpoint{2.187242in}{0.313064in}}%
\pgfpathlineto{\pgfqpoint{2.275584in}{0.310102in}}%
\pgfpathlineto{\pgfqpoint{2.301394in}{0.314738in}}%
\pgfpathlineto{\pgfqpoint{2.302260in}{0.314786in}}%
\pgfpathlineto{\pgfqpoint{2.302606in}{0.313924in}}%
\pgfpathlineto{\pgfqpoint{2.307457in}{0.293133in}}%
\pgfpathlineto{\pgfqpoint{2.308669in}{0.294103in}}%
\pgfpathlineto{\pgfqpoint{2.311267in}{0.300487in}}%
\pgfpathlineto{\pgfqpoint{2.313693in}{0.315760in}}%
\pgfpathlineto{\pgfqpoint{2.316291in}{0.355616in}}%
\pgfpathlineto{\pgfqpoint{2.322873in}{0.462783in}}%
\pgfpathlineto{\pgfqpoint{2.327204in}{0.475901in}}%
\pgfpathlineto{\pgfqpoint{2.332227in}{0.483209in}}%
\pgfpathlineto{\pgfqpoint{2.335865in}{0.483809in}}%
\pgfpathlineto{\pgfqpoint{2.339329in}{0.482063in}}%
\pgfpathlineto{\pgfqpoint{2.341754in}{0.477609in}}%
\pgfpathlineto{\pgfqpoint{2.345912in}{0.461124in}}%
\pgfpathlineto{\pgfqpoint{2.351801in}{0.419894in}}%
\pgfpathlineto{\pgfqpoint{2.360808in}{0.360382in}}%
\pgfpathlineto{\pgfqpoint{2.364100in}{0.355066in}}%
\pgfpathlineto{\pgfqpoint{2.367044in}{0.355382in}}%
\pgfpathlineto{\pgfqpoint{2.372241in}{0.361223in}}%
\pgfpathlineto{\pgfqpoint{2.376398in}{0.368630in}}%
\pgfpathlineto{\pgfqpoint{2.381248in}{0.384888in}}%
\pgfpathlineto{\pgfqpoint{2.395106in}{0.435879in}}%
\pgfpathlineto{\pgfqpoint{2.399610in}{0.439013in}}%
\pgfpathlineto{\pgfqpoint{2.403594in}{0.438154in}}%
\pgfpathlineto{\pgfqpoint{2.408444in}{0.431960in}}%
\pgfpathlineto{\pgfqpoint{2.417798in}{0.411230in}}%
\pgfpathlineto{\pgfqpoint{2.427152in}{0.393938in}}%
\pgfpathlineto{\pgfqpoint{2.434254in}{0.388084in}}%
\pgfpathlineto{\pgfqpoint{2.441009in}{0.386989in}}%
\pgfpathlineto{\pgfqpoint{2.446553in}{0.390170in}}%
\pgfpathlineto{\pgfqpoint{2.452788in}{0.396989in}}%
\pgfpathlineto{\pgfqpoint{2.466126in}{0.409121in}}%
\pgfpathlineto{\pgfqpoint{2.470977in}{0.413464in}}%
\pgfpathlineto{\pgfqpoint{2.477386in}{0.418450in}}%
\pgfpathlineto{\pgfqpoint{2.484834in}{0.421009in}}%
\pgfpathlineto{\pgfqpoint{2.489858in}{0.419830in}}%
\pgfpathlineto{\pgfqpoint{2.496267in}{0.415341in}}%
\pgfpathlineto{\pgfqpoint{2.514801in}{0.396888in}}%
\pgfpathlineto{\pgfqpoint{2.522423in}{0.393112in}}%
\pgfpathlineto{\pgfqpoint{2.527793in}{0.393165in}}%
\pgfpathlineto{\pgfqpoint{2.534202in}{0.396201in}}%
\pgfpathlineto{\pgfqpoint{2.547367in}{0.407802in}}%
\pgfpathlineto{\pgfqpoint{2.558106in}{0.415635in}}%
\pgfpathlineto{\pgfqpoint{2.569712in}{0.417494in}}%
\pgfpathlineto{\pgfqpoint{2.583743in}{0.413423in}}%
\pgfpathlineto{\pgfqpoint{2.587900in}{0.412039in}}%
\pgfpathlineto{\pgfqpoint{2.595349in}{0.407216in}}%
\pgfpathlineto{\pgfqpoint{2.602970in}{0.403512in}}%
\pgfpathlineto{\pgfqpoint{2.607821in}{0.403101in}}%
\pgfpathlineto{\pgfqpoint{2.627741in}{0.405163in}}%
\pgfpathlineto{\pgfqpoint{2.645583in}{0.409800in}}%
\pgfpathlineto{\pgfqpoint{2.669487in}{0.414547in}}%
\pgfpathlineto{\pgfqpoint{2.675896in}{0.412850in}}%
\pgfpathlineto{\pgfqpoint{2.684730in}{0.410692in}}%
\pgfpathlineto{\pgfqpoint{2.700320in}{0.410103in}}%
\pgfpathlineto{\pgfqpoint{2.714351in}{0.406741in}}%
\pgfpathlineto{\pgfqpoint{2.721800in}{0.407246in}}%
\pgfpathlineto{\pgfqpoint{2.730287in}{0.407880in}}%
\pgfpathlineto{\pgfqpoint{2.737736in}{0.408673in}}%
\pgfpathlineto{\pgfqpoint{2.773939in}{0.410841in}}%
\pgfpathlineto{\pgfqpoint{2.780348in}{0.412230in}}%
\pgfpathlineto{\pgfqpoint{2.825385in}{0.412224in}}%
\pgfpathlineto{\pgfqpoint{2.838723in}{0.408862in}}%
\pgfpathlineto{\pgfqpoint{2.848077in}{0.407859in}}%
\pgfpathlineto{\pgfqpoint{2.870249in}{0.412212in}}%
\pgfpathlineto{\pgfqpoint{2.878044in}{0.414660in}}%
\pgfpathlineto{\pgfqpoint{2.887225in}{0.414668in}}%
\pgfpathlineto{\pgfqpoint{2.895020in}{0.413268in}}%
\pgfpathlineto{\pgfqpoint{2.900217in}{0.412583in}}%
\pgfpathlineto{\pgfqpoint{2.929837in}{0.407871in}}%
\pgfpathlineto{\pgfqpoint{2.936420in}{0.407458in}}%
\pgfpathlineto{\pgfqpoint{2.958072in}{0.410657in}}%
\pgfpathlineto{\pgfqpoint{2.968465in}{0.414125in}}%
\pgfpathlineto{\pgfqpoint{2.977646in}{0.416543in}}%
\pgfpathlineto{\pgfqpoint{2.989079in}{0.414788in}}%
\pgfpathlineto{\pgfqpoint{2.997047in}{0.412723in}}%
\pgfpathlineto{\pgfqpoint{3.003975in}{0.412188in}}%
\pgfpathlineto{\pgfqpoint{3.015928in}{0.410166in}}%
\pgfpathlineto{\pgfqpoint{3.021817in}{0.409966in}}%
\pgfpathlineto{\pgfqpoint{3.026841in}{0.409940in}}%
\pgfpathlineto{\pgfqpoint{3.036541in}{0.408382in}}%
\pgfpathlineto{\pgfqpoint{3.045202in}{0.408430in}}%
\pgfpathlineto{\pgfqpoint{3.055249in}{0.410758in}}%
\pgfpathlineto{\pgfqpoint{3.061485in}{0.412185in}}%
\pgfpathlineto{\pgfqpoint{3.081405in}{0.416496in}}%
\pgfpathlineto{\pgfqpoint{3.087641in}{0.415760in}}%
\pgfpathlineto{\pgfqpoint{3.096648in}{0.413889in}}%
\pgfpathlineto{\pgfqpoint{3.147748in}{0.406622in}}%
\pgfpathlineto{\pgfqpoint{3.161260in}{0.409301in}}%
\pgfpathlineto{\pgfqpoint{3.166456in}{0.410605in}}%
\pgfpathlineto{\pgfqpoint{3.166456in}{0.410605in}}%
\pgfusepath{stroke}%
\end{pgfscope}%
\begin{pgfscope}%
\pgfsetrectcap%
\pgfsetmiterjoin%
\pgfsetlinewidth{0.803000pt}%
\definecolor{currentstroke}{rgb}{0.000000,0.000000,0.000000}%
\pgfsetstrokecolor{currentstroke}%
\pgfsetdash{}{0pt}%
\pgfpathmoveto{\pgfqpoint{0.568671in}{0.225639in}}%
\pgfpathlineto{\pgfqpoint{0.568671in}{0.680791in}}%
\pgfusepath{stroke}%
\end{pgfscope}%
\begin{pgfscope}%
\pgfsetrectcap%
\pgfsetmiterjoin%
\pgfsetlinewidth{0.803000pt}%
\definecolor{currentstroke}{rgb}{0.000000,0.000000,0.000000}%
\pgfsetstrokecolor{currentstroke}%
\pgfsetdash{}{0pt}%
\pgfpathmoveto{\pgfqpoint{3.166976in}{0.225639in}}%
\pgfpathlineto{\pgfqpoint{3.166976in}{0.680791in}}%
\pgfusepath{stroke}%
\end{pgfscope}%
\begin{pgfscope}%
\pgfsetrectcap%
\pgfsetmiterjoin%
\pgfsetlinewidth{0.803000pt}%
\definecolor{currentstroke}{rgb}{0.000000,0.000000,0.000000}%
\pgfsetstrokecolor{currentstroke}%
\pgfsetdash{}{0pt}%
\pgfpathmoveto{\pgfqpoint{0.568671in}{0.225639in}}%
\pgfpathlineto{\pgfqpoint{3.166976in}{0.225639in}}%
\pgfusepath{stroke}%
\end{pgfscope}%
\begin{pgfscope}%
\pgfsetrectcap%
\pgfsetmiterjoin%
\pgfsetlinewidth{0.803000pt}%
\definecolor{currentstroke}{rgb}{0.000000,0.000000,0.000000}%
\pgfsetstrokecolor{currentstroke}%
\pgfsetdash{}{0pt}%
\pgfpathmoveto{\pgfqpoint{0.568671in}{0.680791in}}%
\pgfpathlineto{\pgfqpoint{3.166976in}{0.680791in}}%
\pgfusepath{stroke}%
\end{pgfscope}%
\begin{pgfscope}%
\pgfsetbuttcap%
\pgfsetmiterjoin%
\definecolor{currentfill}{rgb}{1.000000,1.000000,1.000000}%
\pgfsetfillcolor{currentfill}%
\pgfsetfillopacity{0.800000}%
\pgfsetlinewidth{1.003750pt}%
\definecolor{currentstroke}{rgb}{0.800000,0.800000,0.800000}%
\pgfsetstrokecolor{currentstroke}%
\pgfsetstrokeopacity{0.800000}%
\pgfsetdash{}{0pt}%
\pgfpathmoveto{\pgfqpoint{0.636727in}{0.365715in}}%
\pgfpathlineto{\pgfqpoint{2.302975in}{0.365715in}}%
\pgfpathquadraticcurveto{\pgfqpoint{2.322419in}{0.365715in}}{\pgfqpoint{2.322419in}{0.385159in}}%
\pgfpathlineto{\pgfqpoint{2.322419in}{0.521270in}}%
\pgfpathquadraticcurveto{\pgfqpoint{2.322419in}{0.540715in}}{\pgfqpoint{2.302975in}{0.540715in}}%
\pgfpathlineto{\pgfqpoint{0.636727in}{0.540715in}}%
\pgfpathquadraticcurveto{\pgfqpoint{0.617282in}{0.540715in}}{\pgfqpoint{0.617282in}{0.521270in}}%
\pgfpathlineto{\pgfqpoint{0.617282in}{0.385159in}}%
\pgfpathquadraticcurveto{\pgfqpoint{0.617282in}{0.365715in}}{\pgfqpoint{0.636727in}{0.365715in}}%
\pgfpathlineto{\pgfqpoint{0.636727in}{0.365715in}}%
\pgfpathclose%
\pgfusepath{stroke,fill}%
\end{pgfscope}%
\begin{pgfscope}%
\pgfsetrectcap%
\pgfsetroundjoin%
\pgfsetlinewidth{1.505625pt}%
\definecolor{currentstroke}{rgb}{0.121569,0.466667,0.705882}%
\pgfsetstrokecolor{currentstroke}%
\pgfsetdash{}{0pt}%
\pgfpathmoveto{\pgfqpoint{0.656171in}{0.462937in}}%
\pgfpathlineto{\pgfqpoint{0.753393in}{0.462937in}}%
\pgfpathlineto{\pgfqpoint{0.850616in}{0.462937in}}%
\pgfusepath{stroke}%
\end{pgfscope}%
\begin{pgfscope}%
\definecolor{textcolor}{rgb}{0.000000,0.000000,0.000000}%
\pgfsetstrokecolor{textcolor}%
\pgfsetfillcolor{textcolor}%
\pgftext[x=0.928393in,y=0.428909in,left,base]{\color{textcolor}\rmfamily\fontsize{7.000000}{8.400000}\selectfont \(\displaystyle \|\mathbf{h}_0\|\)}%
\end{pgfscope}%
\begin{pgfscope}%
\pgfsetrectcap%
\pgfsetroundjoin%
\pgfsetlinewidth{1.505625pt}%
\definecolor{currentstroke}{rgb}{1.000000,0.498039,0.054902}%
\pgfsetstrokecolor{currentstroke}%
\pgfsetdash{}{0pt}%
\pgfpathmoveto{\pgfqpoint{1.214828in}{0.462937in}}%
\pgfpathlineto{\pgfqpoint{1.312050in}{0.462937in}}%
\pgfpathlineto{\pgfqpoint{1.409272in}{0.462937in}}%
\pgfusepath{stroke}%
\end{pgfscope}%
\begin{pgfscope}%
\definecolor{textcolor}{rgb}{0.000000,0.000000,0.000000}%
\pgfsetstrokecolor{textcolor}%
\pgfsetfillcolor{textcolor}%
\pgftext[x=1.487050in,y=0.428909in,left,base]{\color{textcolor}\rmfamily\fontsize{7.000000}{8.400000}\selectfont \(\displaystyle \|\mathbf{h}_1\|\)}%
\end{pgfscope}%
\begin{pgfscope}%
\pgfsetrectcap%
\pgfsetroundjoin%
\pgfsetlinewidth{1.505625pt}%
\definecolor{currentstroke}{rgb}{0.172549,0.627451,0.172549}%
\pgfsetstrokecolor{currentstroke}%
\pgfsetdash{}{0pt}%
\pgfpathmoveto{\pgfqpoint{1.773485in}{0.462937in}}%
\pgfpathlineto{\pgfqpoint{1.870707in}{0.462937in}}%
\pgfpathlineto{\pgfqpoint{1.967929in}{0.462937in}}%
\pgfusepath{stroke}%
\end{pgfscope}%
\begin{pgfscope}%
\definecolor{textcolor}{rgb}{0.000000,0.000000,0.000000}%
\pgfsetstrokecolor{textcolor}%
\pgfsetfillcolor{textcolor}%
\pgftext[x=2.045707in,y=0.428909in,left,base]{\color{textcolor}\rmfamily\fontsize{7.000000}{8.400000}\selectfont \(\displaystyle \|\mathbf{h}_2\|\)}%
\end{pgfscope}%
\end{pgfpicture}%
\makeatother%
\endgroup%
\\\vspace*{-1.2cm}
    %% Creator: Matplotlib, PGF backend
%%
%% To include the figure in your LaTeX document, write
%%   \input{<filename>.pgf}
%%
%% Make sure the required packages are loaded in your preamble
%%   \usepackage{pgf}
%%
%% Also ensure that all the required font packages are loaded; for instance,
%% the lmodern package is sometimes necessary when using math font.
%%   \usepackage{lmodern}
%%
%% Figures using additional raster images can only be included by \input if
%% they are in the same directory as the main LaTeX file. For loading figures
%% from other directories you can use the `import` package
%%   \usepackage{import}
%%
%% and then include the figures with
%%   \import{<path to file>}{<filename>.pgf}
%%
%% Matplotlib used the following preamble
%%   \usepackage{fontspec}
%%
\begingroup%
\makeatletter%
\begin{pgfpicture}%
\pgfpathrectangle{\pgfpointorigin}{\pgfqpoint{3.390065in}{1.356026in}}%
\pgfusepath{use as bounding box, clip}%
\begin{pgfscope}%
\pgfsetbuttcap%
\pgfsetmiterjoin%
\definecolor{currentfill}{rgb}{1.000000,1.000000,1.000000}%
\pgfsetfillcolor{currentfill}%
\pgfsetlinewidth{0.000000pt}%
\definecolor{currentstroke}{rgb}{1.000000,1.000000,1.000000}%
\pgfsetstrokecolor{currentstroke}%
\pgfsetstrokeopacity{0.000000}%
\pgfsetdash{}{0pt}%
\pgfpathmoveto{\pgfqpoint{0.000000in}{0.000000in}}%
\pgfpathlineto{\pgfqpoint{3.390065in}{0.000000in}}%
\pgfpathlineto{\pgfqpoint{3.390065in}{1.356026in}}%
\pgfpathlineto{\pgfqpoint{0.000000in}{1.356026in}}%
\pgfpathlineto{\pgfqpoint{0.000000in}{0.000000in}}%
\pgfpathclose%
\pgfusepath{fill}%
\end{pgfscope}%
\begin{pgfscope}%
\pgfsetbuttcap%
\pgfsetmiterjoin%
\definecolor{currentfill}{rgb}{1.000000,1.000000,1.000000}%
\pgfsetfillcolor{currentfill}%
\pgfsetlinewidth{0.000000pt}%
\definecolor{currentstroke}{rgb}{0.000000,0.000000,0.000000}%
\pgfsetstrokecolor{currentstroke}%
\pgfsetstrokeopacity{0.000000}%
\pgfsetdash{}{0pt}%
\pgfpathmoveto{\pgfqpoint{0.568671in}{0.451277in}}%
\pgfpathlineto{\pgfqpoint{3.166976in}{0.451277in}}%
\pgfpathlineto{\pgfqpoint{3.166976in}{1.250151in}}%
\pgfpathlineto{\pgfqpoint{0.568671in}{1.250151in}}%
\pgfpathlineto{\pgfqpoint{0.568671in}{0.451277in}}%
\pgfpathclose%
\pgfusepath{fill}%
\end{pgfscope}%
\begin{pgfscope}%
\pgfpathrectangle{\pgfqpoint{0.568671in}{0.451277in}}{\pgfqpoint{2.598305in}{0.798874in}}%
\pgfusepath{clip}%
\pgfsetrectcap%
\pgfsetroundjoin%
\pgfsetlinewidth{0.803000pt}%
\definecolor{currentstroke}{rgb}{0.690196,0.690196,0.690196}%
\pgfsetstrokecolor{currentstroke}%
\pgfsetdash{}{0pt}%
\pgfpathmoveto{\pgfqpoint{0.568671in}{0.451277in}}%
\pgfpathlineto{\pgfqpoint{0.568671in}{1.250151in}}%
\pgfusepath{stroke}%
\end{pgfscope}%
\begin{pgfscope}%
\pgfsetbuttcap%
\pgfsetroundjoin%
\definecolor{currentfill}{rgb}{0.000000,0.000000,0.000000}%
\pgfsetfillcolor{currentfill}%
\pgfsetlinewidth{0.803000pt}%
\definecolor{currentstroke}{rgb}{0.000000,0.000000,0.000000}%
\pgfsetstrokecolor{currentstroke}%
\pgfsetdash{}{0pt}%
\pgfsys@defobject{currentmarker}{\pgfqpoint{0.000000in}{-0.048611in}}{\pgfqpoint{0.000000in}{0.000000in}}{%
\pgfpathmoveto{\pgfqpoint{0.000000in}{0.000000in}}%
\pgfpathlineto{\pgfqpoint{0.000000in}{-0.048611in}}%
\pgfusepath{stroke,fill}%
}%
\begin{pgfscope}%
\pgfsys@transformshift{0.568671in}{0.451277in}%
\pgfsys@useobject{currentmarker}{}%
\end{pgfscope}%
\end{pgfscope}%
\begin{pgfscope}%
\definecolor{textcolor}{rgb}{0.000000,0.000000,0.000000}%
\pgfsetstrokecolor{textcolor}%
\pgfsetfillcolor{textcolor}%
\pgftext[x=0.568671in,y=0.354055in,,top]{\color{textcolor}\rmfamily\fontsize{9.000000}{10.800000}\selectfont \(\displaystyle {0}\)}%
\end{pgfscope}%
\begin{pgfscope}%
\pgfpathrectangle{\pgfqpoint{0.568671in}{0.451277in}}{\pgfqpoint{2.598305in}{0.798874in}}%
\pgfusepath{clip}%
\pgfsetrectcap%
\pgfsetroundjoin%
\pgfsetlinewidth{0.803000pt}%
\definecolor{currentstroke}{rgb}{0.690196,0.690196,0.690196}%
\pgfsetstrokecolor{currentstroke}%
\pgfsetdash{}{0pt}%
\pgfpathmoveto{\pgfqpoint{1.001722in}{0.451277in}}%
\pgfpathlineto{\pgfqpoint{1.001722in}{1.250151in}}%
\pgfusepath{stroke}%
\end{pgfscope}%
\begin{pgfscope}%
\pgfsetbuttcap%
\pgfsetroundjoin%
\definecolor{currentfill}{rgb}{0.000000,0.000000,0.000000}%
\pgfsetfillcolor{currentfill}%
\pgfsetlinewidth{0.803000pt}%
\definecolor{currentstroke}{rgb}{0.000000,0.000000,0.000000}%
\pgfsetstrokecolor{currentstroke}%
\pgfsetdash{}{0pt}%
\pgfsys@defobject{currentmarker}{\pgfqpoint{0.000000in}{-0.048611in}}{\pgfqpoint{0.000000in}{0.000000in}}{%
\pgfpathmoveto{\pgfqpoint{0.000000in}{0.000000in}}%
\pgfpathlineto{\pgfqpoint{0.000000in}{-0.048611in}}%
\pgfusepath{stroke,fill}%
}%
\begin{pgfscope}%
\pgfsys@transformshift{1.001722in}{0.451277in}%
\pgfsys@useobject{currentmarker}{}%
\end{pgfscope}%
\end{pgfscope}%
\begin{pgfscope}%
\definecolor{textcolor}{rgb}{0.000000,0.000000,0.000000}%
\pgfsetstrokecolor{textcolor}%
\pgfsetfillcolor{textcolor}%
\pgftext[x=1.001722in,y=0.354055in,,top]{\color{textcolor}\rmfamily\fontsize{9.000000}{10.800000}\selectfont \(\displaystyle {2500}\)}%
\end{pgfscope}%
\begin{pgfscope}%
\pgfpathrectangle{\pgfqpoint{0.568671in}{0.451277in}}{\pgfqpoint{2.598305in}{0.798874in}}%
\pgfusepath{clip}%
\pgfsetrectcap%
\pgfsetroundjoin%
\pgfsetlinewidth{0.803000pt}%
\definecolor{currentstroke}{rgb}{0.690196,0.690196,0.690196}%
\pgfsetstrokecolor{currentstroke}%
\pgfsetdash{}{0pt}%
\pgfpathmoveto{\pgfqpoint{1.434773in}{0.451277in}}%
\pgfpathlineto{\pgfqpoint{1.434773in}{1.250151in}}%
\pgfusepath{stroke}%
\end{pgfscope}%
\begin{pgfscope}%
\pgfsetbuttcap%
\pgfsetroundjoin%
\definecolor{currentfill}{rgb}{0.000000,0.000000,0.000000}%
\pgfsetfillcolor{currentfill}%
\pgfsetlinewidth{0.803000pt}%
\definecolor{currentstroke}{rgb}{0.000000,0.000000,0.000000}%
\pgfsetstrokecolor{currentstroke}%
\pgfsetdash{}{0pt}%
\pgfsys@defobject{currentmarker}{\pgfqpoint{0.000000in}{-0.048611in}}{\pgfqpoint{0.000000in}{0.000000in}}{%
\pgfpathmoveto{\pgfqpoint{0.000000in}{0.000000in}}%
\pgfpathlineto{\pgfqpoint{0.000000in}{-0.048611in}}%
\pgfusepath{stroke,fill}%
}%
\begin{pgfscope}%
\pgfsys@transformshift{1.434773in}{0.451277in}%
\pgfsys@useobject{currentmarker}{}%
\end{pgfscope}%
\end{pgfscope}%
\begin{pgfscope}%
\definecolor{textcolor}{rgb}{0.000000,0.000000,0.000000}%
\pgfsetstrokecolor{textcolor}%
\pgfsetfillcolor{textcolor}%
\pgftext[x=1.434773in,y=0.354055in,,top]{\color{textcolor}\rmfamily\fontsize{9.000000}{10.800000}\selectfont \(\displaystyle {5000}\)}%
\end{pgfscope}%
\begin{pgfscope}%
\pgfpathrectangle{\pgfqpoint{0.568671in}{0.451277in}}{\pgfqpoint{2.598305in}{0.798874in}}%
\pgfusepath{clip}%
\pgfsetrectcap%
\pgfsetroundjoin%
\pgfsetlinewidth{0.803000pt}%
\definecolor{currentstroke}{rgb}{0.690196,0.690196,0.690196}%
\pgfsetstrokecolor{currentstroke}%
\pgfsetdash{}{0pt}%
\pgfpathmoveto{\pgfqpoint{1.867823in}{0.451277in}}%
\pgfpathlineto{\pgfqpoint{1.867823in}{1.250151in}}%
\pgfusepath{stroke}%
\end{pgfscope}%
\begin{pgfscope}%
\pgfsetbuttcap%
\pgfsetroundjoin%
\definecolor{currentfill}{rgb}{0.000000,0.000000,0.000000}%
\pgfsetfillcolor{currentfill}%
\pgfsetlinewidth{0.803000pt}%
\definecolor{currentstroke}{rgb}{0.000000,0.000000,0.000000}%
\pgfsetstrokecolor{currentstroke}%
\pgfsetdash{}{0pt}%
\pgfsys@defobject{currentmarker}{\pgfqpoint{0.000000in}{-0.048611in}}{\pgfqpoint{0.000000in}{0.000000in}}{%
\pgfpathmoveto{\pgfqpoint{0.000000in}{0.000000in}}%
\pgfpathlineto{\pgfqpoint{0.000000in}{-0.048611in}}%
\pgfusepath{stroke,fill}%
}%
\begin{pgfscope}%
\pgfsys@transformshift{1.867823in}{0.451277in}%
\pgfsys@useobject{currentmarker}{}%
\end{pgfscope}%
\end{pgfscope}%
\begin{pgfscope}%
\definecolor{textcolor}{rgb}{0.000000,0.000000,0.000000}%
\pgfsetstrokecolor{textcolor}%
\pgfsetfillcolor{textcolor}%
\pgftext[x=1.867823in,y=0.354055in,,top]{\color{textcolor}\rmfamily\fontsize{9.000000}{10.800000}\selectfont \(\displaystyle {7500}\)}%
\end{pgfscope}%
\begin{pgfscope}%
\pgfpathrectangle{\pgfqpoint{0.568671in}{0.451277in}}{\pgfqpoint{2.598305in}{0.798874in}}%
\pgfusepath{clip}%
\pgfsetrectcap%
\pgfsetroundjoin%
\pgfsetlinewidth{0.803000pt}%
\definecolor{currentstroke}{rgb}{0.690196,0.690196,0.690196}%
\pgfsetstrokecolor{currentstroke}%
\pgfsetdash{}{0pt}%
\pgfpathmoveto{\pgfqpoint{2.300874in}{0.451277in}}%
\pgfpathlineto{\pgfqpoint{2.300874in}{1.250151in}}%
\pgfusepath{stroke}%
\end{pgfscope}%
\begin{pgfscope}%
\pgfsetbuttcap%
\pgfsetroundjoin%
\definecolor{currentfill}{rgb}{0.000000,0.000000,0.000000}%
\pgfsetfillcolor{currentfill}%
\pgfsetlinewidth{0.803000pt}%
\definecolor{currentstroke}{rgb}{0.000000,0.000000,0.000000}%
\pgfsetstrokecolor{currentstroke}%
\pgfsetdash{}{0pt}%
\pgfsys@defobject{currentmarker}{\pgfqpoint{0.000000in}{-0.048611in}}{\pgfqpoint{0.000000in}{0.000000in}}{%
\pgfpathmoveto{\pgfqpoint{0.000000in}{0.000000in}}%
\pgfpathlineto{\pgfqpoint{0.000000in}{-0.048611in}}%
\pgfusepath{stroke,fill}%
}%
\begin{pgfscope}%
\pgfsys@transformshift{2.300874in}{0.451277in}%
\pgfsys@useobject{currentmarker}{}%
\end{pgfscope}%
\end{pgfscope}%
\begin{pgfscope}%
\definecolor{textcolor}{rgb}{0.000000,0.000000,0.000000}%
\pgfsetstrokecolor{textcolor}%
\pgfsetfillcolor{textcolor}%
\pgftext[x=2.300874in,y=0.354055in,,top]{\color{textcolor}\rmfamily\fontsize{9.000000}{10.800000}\selectfont \(\displaystyle {10000}\)}%
\end{pgfscope}%
\begin{pgfscope}%
\pgfpathrectangle{\pgfqpoint{0.568671in}{0.451277in}}{\pgfqpoint{2.598305in}{0.798874in}}%
\pgfusepath{clip}%
\pgfsetrectcap%
\pgfsetroundjoin%
\pgfsetlinewidth{0.803000pt}%
\definecolor{currentstroke}{rgb}{0.690196,0.690196,0.690196}%
\pgfsetstrokecolor{currentstroke}%
\pgfsetdash{}{0pt}%
\pgfpathmoveto{\pgfqpoint{2.733925in}{0.451277in}}%
\pgfpathlineto{\pgfqpoint{2.733925in}{1.250151in}}%
\pgfusepath{stroke}%
\end{pgfscope}%
\begin{pgfscope}%
\pgfsetbuttcap%
\pgfsetroundjoin%
\definecolor{currentfill}{rgb}{0.000000,0.000000,0.000000}%
\pgfsetfillcolor{currentfill}%
\pgfsetlinewidth{0.803000pt}%
\definecolor{currentstroke}{rgb}{0.000000,0.000000,0.000000}%
\pgfsetstrokecolor{currentstroke}%
\pgfsetdash{}{0pt}%
\pgfsys@defobject{currentmarker}{\pgfqpoint{0.000000in}{-0.048611in}}{\pgfqpoint{0.000000in}{0.000000in}}{%
\pgfpathmoveto{\pgfqpoint{0.000000in}{0.000000in}}%
\pgfpathlineto{\pgfqpoint{0.000000in}{-0.048611in}}%
\pgfusepath{stroke,fill}%
}%
\begin{pgfscope}%
\pgfsys@transformshift{2.733925in}{0.451277in}%
\pgfsys@useobject{currentmarker}{}%
\end{pgfscope}%
\end{pgfscope}%
\begin{pgfscope}%
\definecolor{textcolor}{rgb}{0.000000,0.000000,0.000000}%
\pgfsetstrokecolor{textcolor}%
\pgfsetfillcolor{textcolor}%
\pgftext[x=2.733925in,y=0.354055in,,top]{\color{textcolor}\rmfamily\fontsize{9.000000}{10.800000}\selectfont \(\displaystyle {12500}\)}%
\end{pgfscope}%
\begin{pgfscope}%
\pgfpathrectangle{\pgfqpoint{0.568671in}{0.451277in}}{\pgfqpoint{2.598305in}{0.798874in}}%
\pgfusepath{clip}%
\pgfsetrectcap%
\pgfsetroundjoin%
\pgfsetlinewidth{0.803000pt}%
\definecolor{currentstroke}{rgb}{0.690196,0.690196,0.690196}%
\pgfsetstrokecolor{currentstroke}%
\pgfsetdash{}{0pt}%
\pgfpathmoveto{\pgfqpoint{3.166976in}{0.451277in}}%
\pgfpathlineto{\pgfqpoint{3.166976in}{1.250151in}}%
\pgfusepath{stroke}%
\end{pgfscope}%
\begin{pgfscope}%
\pgfsetbuttcap%
\pgfsetroundjoin%
\definecolor{currentfill}{rgb}{0.000000,0.000000,0.000000}%
\pgfsetfillcolor{currentfill}%
\pgfsetlinewidth{0.803000pt}%
\definecolor{currentstroke}{rgb}{0.000000,0.000000,0.000000}%
\pgfsetstrokecolor{currentstroke}%
\pgfsetdash{}{0pt}%
\pgfsys@defobject{currentmarker}{\pgfqpoint{0.000000in}{-0.048611in}}{\pgfqpoint{0.000000in}{0.000000in}}{%
\pgfpathmoveto{\pgfqpoint{0.000000in}{0.000000in}}%
\pgfpathlineto{\pgfqpoint{0.000000in}{-0.048611in}}%
\pgfusepath{stroke,fill}%
}%
\begin{pgfscope}%
\pgfsys@transformshift{3.166976in}{0.451277in}%
\pgfsys@useobject{currentmarker}{}%
\end{pgfscope}%
\end{pgfscope}%
\begin{pgfscope}%
\definecolor{textcolor}{rgb}{0.000000,0.000000,0.000000}%
\pgfsetstrokecolor{textcolor}%
\pgfsetfillcolor{textcolor}%
\pgftext[x=3.166976in,y=0.354055in,,top]{\color{textcolor}\rmfamily\fontsize{9.000000}{10.800000}\selectfont \(\displaystyle {15000}\)}%
\end{pgfscope}%
\begin{pgfscope}%
\definecolor{textcolor}{rgb}{0.000000,0.000000,0.000000}%
\pgfsetstrokecolor{textcolor}%
\pgfsetfillcolor{textcolor}%
\pgftext[x=1.867823in,y=0.187500in,,top]{\color{textcolor}\rmfamily\fontsize{9.000000}{10.800000}\selectfont Time [frames]}%
\end{pgfscope}%
\begin{pgfscope}%
\pgfpathrectangle{\pgfqpoint{0.568671in}{0.451277in}}{\pgfqpoint{2.598305in}{0.798874in}}%
\pgfusepath{clip}%
\pgfsetrectcap%
\pgfsetroundjoin%
\pgfsetlinewidth{0.803000pt}%
\definecolor{currentstroke}{rgb}{0.690196,0.690196,0.690196}%
\pgfsetstrokecolor{currentstroke}%
\pgfsetdash{}{0pt}%
\pgfpathmoveto{\pgfqpoint{0.568671in}{0.451277in}}%
\pgfpathlineto{\pgfqpoint{3.166976in}{0.451277in}}%
\pgfusepath{stroke}%
\end{pgfscope}%
\begin{pgfscope}%
\pgfsetbuttcap%
\pgfsetroundjoin%
\definecolor{currentfill}{rgb}{0.000000,0.000000,0.000000}%
\pgfsetfillcolor{currentfill}%
\pgfsetlinewidth{0.803000pt}%
\definecolor{currentstroke}{rgb}{0.000000,0.000000,0.000000}%
\pgfsetstrokecolor{currentstroke}%
\pgfsetdash{}{0pt}%
\pgfsys@defobject{currentmarker}{\pgfqpoint{-0.048611in}{0.000000in}}{\pgfqpoint{-0.000000in}{0.000000in}}{%
\pgfpathmoveto{\pgfqpoint{-0.000000in}{0.000000in}}%
\pgfpathlineto{\pgfqpoint{-0.048611in}{0.000000in}}%
\pgfusepath{stroke,fill}%
}%
\begin{pgfscope}%
\pgfsys@transformshift{0.568671in}{0.451277in}%
\pgfsys@useobject{currentmarker}{}%
\end{pgfscope}%
\end{pgfscope}%
\begin{pgfscope}%
\definecolor{textcolor}{rgb}{0.000000,0.000000,0.000000}%
\pgfsetstrokecolor{textcolor}%
\pgfsetfillcolor{textcolor}%
\pgftext[x=0.243055in, y=0.407902in, left, base]{\color{textcolor}\rmfamily\fontsize{9.000000}{10.800000}\selectfont \(\displaystyle {\ensuremath{-}30}\)}%
\end{pgfscope}%
\begin{pgfscope}%
\pgfpathrectangle{\pgfqpoint{0.568671in}{0.451277in}}{\pgfqpoint{2.598305in}{0.798874in}}%
\pgfusepath{clip}%
\pgfsetrectcap%
\pgfsetroundjoin%
\pgfsetlinewidth{0.803000pt}%
\definecolor{currentstroke}{rgb}{0.690196,0.690196,0.690196}%
\pgfsetstrokecolor{currentstroke}%
\pgfsetdash{}{0pt}%
\pgfpathmoveto{\pgfqpoint{0.568671in}{0.717569in}}%
\pgfpathlineto{\pgfqpoint{3.166976in}{0.717569in}}%
\pgfusepath{stroke}%
\end{pgfscope}%
\begin{pgfscope}%
\pgfsetbuttcap%
\pgfsetroundjoin%
\definecolor{currentfill}{rgb}{0.000000,0.000000,0.000000}%
\pgfsetfillcolor{currentfill}%
\pgfsetlinewidth{0.803000pt}%
\definecolor{currentstroke}{rgb}{0.000000,0.000000,0.000000}%
\pgfsetstrokecolor{currentstroke}%
\pgfsetdash{}{0pt}%
\pgfsys@defobject{currentmarker}{\pgfqpoint{-0.048611in}{0.000000in}}{\pgfqpoint{-0.000000in}{0.000000in}}{%
\pgfpathmoveto{\pgfqpoint{-0.000000in}{0.000000in}}%
\pgfpathlineto{\pgfqpoint{-0.048611in}{0.000000in}}%
\pgfusepath{stroke,fill}%
}%
\begin{pgfscope}%
\pgfsys@transformshift{0.568671in}{0.717569in}%
\pgfsys@useobject{currentmarker}{}%
\end{pgfscope}%
\end{pgfscope}%
\begin{pgfscope}%
\definecolor{textcolor}{rgb}{0.000000,0.000000,0.000000}%
\pgfsetstrokecolor{textcolor}%
\pgfsetfillcolor{textcolor}%
\pgftext[x=0.243055in, y=0.674194in, left, base]{\color{textcolor}\rmfamily\fontsize{9.000000}{10.800000}\selectfont \(\displaystyle {\ensuremath{-}20}\)}%
\end{pgfscope}%
\begin{pgfscope}%
\pgfpathrectangle{\pgfqpoint{0.568671in}{0.451277in}}{\pgfqpoint{2.598305in}{0.798874in}}%
\pgfusepath{clip}%
\pgfsetrectcap%
\pgfsetroundjoin%
\pgfsetlinewidth{0.803000pt}%
\definecolor{currentstroke}{rgb}{0.690196,0.690196,0.690196}%
\pgfsetstrokecolor{currentstroke}%
\pgfsetdash{}{0pt}%
\pgfpathmoveto{\pgfqpoint{0.568671in}{0.983860in}}%
\pgfpathlineto{\pgfqpoint{3.166976in}{0.983860in}}%
\pgfusepath{stroke}%
\end{pgfscope}%
\begin{pgfscope}%
\pgfsetbuttcap%
\pgfsetroundjoin%
\definecolor{currentfill}{rgb}{0.000000,0.000000,0.000000}%
\pgfsetfillcolor{currentfill}%
\pgfsetlinewidth{0.803000pt}%
\definecolor{currentstroke}{rgb}{0.000000,0.000000,0.000000}%
\pgfsetstrokecolor{currentstroke}%
\pgfsetdash{}{0pt}%
\pgfsys@defobject{currentmarker}{\pgfqpoint{-0.048611in}{0.000000in}}{\pgfqpoint{-0.000000in}{0.000000in}}{%
\pgfpathmoveto{\pgfqpoint{-0.000000in}{0.000000in}}%
\pgfpathlineto{\pgfqpoint{-0.048611in}{0.000000in}}%
\pgfusepath{stroke,fill}%
}%
\begin{pgfscope}%
\pgfsys@transformshift{0.568671in}{0.983860in}%
\pgfsys@useobject{currentmarker}{}%
\end{pgfscope}%
\end{pgfscope}%
\begin{pgfscope}%
\definecolor{textcolor}{rgb}{0.000000,0.000000,0.000000}%
\pgfsetstrokecolor{textcolor}%
\pgfsetfillcolor{textcolor}%
\pgftext[x=0.243055in, y=0.940485in, left, base]{\color{textcolor}\rmfamily\fontsize{9.000000}{10.800000}\selectfont \(\displaystyle {\ensuremath{-}10}\)}%
\end{pgfscope}%
\begin{pgfscope}%
\pgfpathrectangle{\pgfqpoint{0.568671in}{0.451277in}}{\pgfqpoint{2.598305in}{0.798874in}}%
\pgfusepath{clip}%
\pgfsetrectcap%
\pgfsetroundjoin%
\pgfsetlinewidth{0.803000pt}%
\definecolor{currentstroke}{rgb}{0.690196,0.690196,0.690196}%
\pgfsetstrokecolor{currentstroke}%
\pgfsetdash{}{0pt}%
\pgfpathmoveto{\pgfqpoint{0.568671in}{1.250151in}}%
\pgfpathlineto{\pgfqpoint{3.166976in}{1.250151in}}%
\pgfusepath{stroke}%
\end{pgfscope}%
\begin{pgfscope}%
\pgfsetbuttcap%
\pgfsetroundjoin%
\definecolor{currentfill}{rgb}{0.000000,0.000000,0.000000}%
\pgfsetfillcolor{currentfill}%
\pgfsetlinewidth{0.803000pt}%
\definecolor{currentstroke}{rgb}{0.000000,0.000000,0.000000}%
\pgfsetstrokecolor{currentstroke}%
\pgfsetdash{}{0pt}%
\pgfsys@defobject{currentmarker}{\pgfqpoint{-0.048611in}{0.000000in}}{\pgfqpoint{-0.000000in}{0.000000in}}{%
\pgfpathmoveto{\pgfqpoint{-0.000000in}{0.000000in}}%
\pgfpathlineto{\pgfqpoint{-0.048611in}{0.000000in}}%
\pgfusepath{stroke,fill}%
}%
\begin{pgfscope}%
\pgfsys@transformshift{0.568671in}{1.250151in}%
\pgfsys@useobject{currentmarker}{}%
\end{pgfscope}%
\end{pgfscope}%
\begin{pgfscope}%
\definecolor{textcolor}{rgb}{0.000000,0.000000,0.000000}%
\pgfsetstrokecolor{textcolor}%
\pgfsetfillcolor{textcolor}%
\pgftext[x=0.407213in, y=1.206776in, left, base]{\color{textcolor}\rmfamily\fontsize{9.000000}{10.800000}\selectfont \(\displaystyle {0}\)}%
\end{pgfscope}%
\begin{pgfscope}%
\definecolor{textcolor}{rgb}{0.000000,0.000000,0.000000}%
\pgfsetstrokecolor{textcolor}%
\pgfsetfillcolor{textcolor}%
\pgftext[x=0.187500in,y=0.850714in,,bottom,rotate=90.000000]{\color{textcolor}\rmfamily\fontsize{9.000000}{10.800000}\selectfont NPM [dB]}%
\end{pgfscope}%
\begin{pgfscope}%
\pgfpathrectangle{\pgfqpoint{0.568671in}{0.451277in}}{\pgfqpoint{2.598305in}{0.798874in}}%
\pgfusepath{clip}%
\pgfsetbuttcap%
\pgfsetroundjoin%
\pgfsetlinewidth{0.752812pt}%
\definecolor{currentstroke}{rgb}{0.000000,0.000000,0.000000}%
\pgfsetstrokecolor{currentstroke}%
\pgfsetdash{{2.775000pt}{1.200000pt}}{0.000000pt}%
\pgfpathmoveto{\pgfqpoint{1.434773in}{0.451277in}}%
\pgfpathlineto{\pgfqpoint{1.434773in}{1.250151in}}%
\pgfusepath{stroke}%
\end{pgfscope}%
\begin{pgfscope}%
\pgfpathrectangle{\pgfqpoint{0.568671in}{0.451277in}}{\pgfqpoint{2.598305in}{0.798874in}}%
\pgfusepath{clip}%
\pgfsetbuttcap%
\pgfsetroundjoin%
\pgfsetlinewidth{0.752812pt}%
\definecolor{currentstroke}{rgb}{0.000000,0.000000,0.000000}%
\pgfsetstrokecolor{currentstroke}%
\pgfsetdash{{2.775000pt}{1.200000pt}}{0.000000pt}%
\pgfpathmoveto{\pgfqpoint{2.300874in}{0.451277in}}%
\pgfpathlineto{\pgfqpoint{2.300874in}{1.250151in}}%
\pgfusepath{stroke}%
\end{pgfscope}%
\begin{pgfscope}%
\pgfpathrectangle{\pgfqpoint{0.568671in}{0.451277in}}{\pgfqpoint{2.598305in}{0.798874in}}%
\pgfusepath{clip}%
\pgfsetrectcap%
\pgfsetroundjoin%
\pgfsetlinewidth{1.505625pt}%
\definecolor{currentstroke}{rgb}{0.121569,0.466667,0.705882}%
\pgfsetstrokecolor{currentstroke}%
\pgfsetstrokeopacity{0.250000}%
\pgfsetdash{}{0pt}%
\pgfpathmoveto{\pgfqpoint{0.651825in}{1.264040in}}%
\pgfpathlineto{\pgfqpoint{0.657187in}{1.255808in}}%
\pgfpathlineto{\pgfqpoint{0.663076in}{1.246299in}}%
\pgfpathlineto{\pgfqpoint{0.672257in}{1.229344in}}%
\pgfpathlineto{\pgfqpoint{0.683343in}{1.199714in}}%
\pgfpathlineto{\pgfqpoint{0.687327in}{1.191171in}}%
\pgfpathlineto{\pgfqpoint{0.693217in}{1.178143in}}%
\pgfpathlineto{\pgfqpoint{0.711058in}{1.142390in}}%
\pgfpathlineto{\pgfqpoint{0.730805in}{1.110956in}}%
\pgfpathlineto{\pgfqpoint{0.732538in}{1.107670in}}%
\pgfpathlineto{\pgfqpoint{0.736002in}{1.103305in}}%
\pgfpathlineto{\pgfqpoint{0.737214in}{1.102034in}}%
\pgfpathlineto{\pgfqpoint{0.741025in}{1.096196in}}%
\pgfpathlineto{\pgfqpoint{0.745356in}{1.091019in}}%
\pgfpathlineto{\pgfqpoint{0.752285in}{1.073106in}}%
\pgfpathlineto{\pgfqpoint{0.765796in}{1.057695in}}%
\pgfpathlineto{\pgfqpoint{0.770126in}{1.051593in}}%
\pgfpathlineto{\pgfqpoint{0.774457in}{1.047908in}}%
\pgfpathlineto{\pgfqpoint{0.781905in}{1.040426in}}%
\pgfpathlineto{\pgfqpoint{0.783984in}{1.038872in}}%
\pgfpathlineto{\pgfqpoint{0.790047in}{1.028325in}}%
\pgfpathlineto{\pgfqpoint{0.793338in}{1.027919in}}%
\pgfpathlineto{\pgfqpoint{0.801133in}{1.022962in}}%
\pgfpathlineto{\pgfqpoint{0.807195in}{1.020223in}}%
\pgfpathlineto{\pgfqpoint{0.808928in}{1.020656in}}%
\pgfpathlineto{\pgfqpoint{0.810140in}{1.018544in}}%
\pgfpathlineto{\pgfqpoint{0.811699in}{1.016741in}}%
\pgfpathlineto{\pgfqpoint{0.812046in}{1.017220in}}%
\pgfpathlineto{\pgfqpoint{0.812739in}{1.017023in}}%
\pgfpathlineto{\pgfqpoint{0.813085in}{1.016232in}}%
\pgfpathlineto{\pgfqpoint{0.816896in}{1.010547in}}%
\pgfpathlineto{\pgfqpoint{0.818455in}{1.011246in}}%
\pgfpathlineto{\pgfqpoint{0.820360in}{1.012524in}}%
\pgfpathlineto{\pgfqpoint{0.820880in}{1.011693in}}%
\pgfpathlineto{\pgfqpoint{0.823132in}{1.010410in}}%
\pgfpathlineto{\pgfqpoint{0.826076in}{1.007870in}}%
\pgfpathlineto{\pgfqpoint{0.829714in}{1.007027in}}%
\pgfpathlineto{\pgfqpoint{0.830580in}{1.007078in}}%
\pgfpathlineto{\pgfqpoint{0.831100in}{1.006460in}}%
\pgfpathlineto{\pgfqpoint{0.835777in}{1.001301in}}%
\pgfpathlineto{\pgfqpoint{0.840107in}{0.999524in}}%
\pgfpathlineto{\pgfqpoint{0.840800in}{0.999413in}}%
\pgfpathlineto{\pgfqpoint{0.841320in}{0.998583in}}%
\pgfpathlineto{\pgfqpoint{0.844784in}{0.994192in}}%
\pgfpathlineto{\pgfqpoint{0.847383in}{0.990836in}}%
\pgfpathlineto{\pgfqpoint{0.852579in}{0.990816in}}%
\pgfpathlineto{\pgfqpoint{0.853272in}{0.991076in}}%
\pgfpathlineto{\pgfqpoint{0.853792in}{0.990222in}}%
\pgfpathlineto{\pgfqpoint{0.858642in}{0.982410in}}%
\pgfpathlineto{\pgfqpoint{0.862280in}{0.979924in}}%
\pgfpathlineto{\pgfqpoint{0.862972in}{0.979423in}}%
\pgfpathlineto{\pgfqpoint{0.863146in}{0.978965in}}%
\pgfpathlineto{\pgfqpoint{0.865917in}{0.976127in}}%
\pgfpathlineto{\pgfqpoint{0.869035in}{0.975515in}}%
\pgfpathlineto{\pgfqpoint{0.870767in}{0.978239in}}%
\pgfpathlineto{\pgfqpoint{0.871460in}{0.976696in}}%
\pgfpathlineto{\pgfqpoint{0.873366in}{0.975664in}}%
\pgfpathlineto{\pgfqpoint{0.884105in}{0.973663in}}%
\pgfpathlineto{\pgfqpoint{0.887570in}{0.970220in}}%
\pgfpathlineto{\pgfqpoint{0.888782in}{0.967651in}}%
\pgfpathlineto{\pgfqpoint{0.892247in}{0.961287in}}%
\pgfpathlineto{\pgfqpoint{0.892420in}{0.961442in}}%
\pgfpathlineto{\pgfqpoint{0.893459in}{0.962563in}}%
\pgfpathlineto{\pgfqpoint{0.893979in}{0.961058in}}%
\pgfpathlineto{\pgfqpoint{0.896924in}{0.958047in}}%
\pgfpathlineto{\pgfqpoint{0.898136in}{0.957808in}}%
\pgfpathlineto{\pgfqpoint{0.898309in}{0.957442in}}%
\pgfpathlineto{\pgfqpoint{0.900042in}{0.955874in}}%
\pgfpathlineto{\pgfqpoint{0.900561in}{0.956263in}}%
\pgfpathlineto{\pgfqpoint{0.903506in}{0.954432in}}%
\pgfpathlineto{\pgfqpoint{0.903852in}{0.955408in}}%
\pgfpathlineto{\pgfqpoint{0.905065in}{0.957275in}}%
\pgfpathlineto{\pgfqpoint{0.905758in}{0.956264in}}%
\pgfpathlineto{\pgfqpoint{0.908183in}{0.954087in}}%
\pgfpathlineto{\pgfqpoint{0.908356in}{0.954226in}}%
\pgfpathlineto{\pgfqpoint{0.909742in}{0.955062in}}%
\pgfpathlineto{\pgfqpoint{0.910781in}{0.955313in}}%
\pgfpathlineto{\pgfqpoint{0.911301in}{0.954826in}}%
\pgfpathlineto{\pgfqpoint{0.915285in}{0.950621in}}%
\pgfpathlineto{\pgfqpoint{0.917883in}{0.945354in}}%
\pgfpathlineto{\pgfqpoint{0.918056in}{0.945494in}}%
\pgfpathlineto{\pgfqpoint{0.922041in}{0.944612in}}%
\pgfpathlineto{\pgfqpoint{0.924639in}{0.941291in}}%
\pgfpathlineto{\pgfqpoint{0.924985in}{0.941617in}}%
\pgfpathlineto{\pgfqpoint{0.925505in}{0.942298in}}%
\pgfpathlineto{\pgfqpoint{0.926371in}{0.941144in}}%
\pgfpathlineto{\pgfqpoint{0.930009in}{0.932498in}}%
\pgfpathlineto{\pgfqpoint{0.932087in}{0.928343in}}%
\pgfpathlineto{\pgfqpoint{0.932434in}{0.928585in}}%
\pgfpathlineto{\pgfqpoint{0.933300in}{0.927266in}}%
\pgfpathlineto{\pgfqpoint{0.939189in}{0.915689in}}%
\pgfpathlineto{\pgfqpoint{0.944559in}{0.908838in}}%
\pgfpathlineto{\pgfqpoint{0.947504in}{0.904860in}}%
\pgfpathlineto{\pgfqpoint{0.949409in}{0.905824in}}%
\pgfpathlineto{\pgfqpoint{0.949756in}{0.904850in}}%
\pgfpathlineto{\pgfqpoint{0.954779in}{0.887618in}}%
\pgfpathlineto{\pgfqpoint{0.958936in}{0.872464in}}%
\pgfpathlineto{\pgfqpoint{0.960149in}{0.870192in}}%
\pgfpathlineto{\pgfqpoint{0.961188in}{0.867163in}}%
\pgfpathlineto{\pgfqpoint{0.962228in}{0.867821in}}%
\pgfpathlineto{\pgfqpoint{0.962401in}{0.868057in}}%
\pgfpathlineto{\pgfqpoint{0.963094in}{0.866520in}}%
\pgfpathlineto{\pgfqpoint{0.965172in}{0.861891in}}%
\pgfpathlineto{\pgfqpoint{0.965346in}{0.862064in}}%
\pgfpathlineto{\pgfqpoint{0.966039in}{0.862069in}}%
\pgfpathlineto{\pgfqpoint{0.966731in}{0.861405in}}%
\pgfpathlineto{\pgfqpoint{0.968117in}{0.859899in}}%
\pgfpathlineto{\pgfqpoint{0.969849in}{0.859466in}}%
\pgfpathlineto{\pgfqpoint{0.973833in}{0.853903in}}%
\pgfpathlineto{\pgfqpoint{0.976951in}{0.846730in}}%
\pgfpathlineto{\pgfqpoint{0.979203in}{0.846149in}}%
\pgfpathlineto{\pgfqpoint{0.981628in}{0.841156in}}%
\pgfpathlineto{\pgfqpoint{0.984573in}{0.840125in}}%
\pgfpathlineto{\pgfqpoint{0.988384in}{0.831625in}}%
\pgfpathlineto{\pgfqpoint{0.988904in}{0.831964in}}%
\pgfpathlineto{\pgfqpoint{0.989770in}{0.831041in}}%
\pgfpathlineto{\pgfqpoint{0.990463in}{0.830098in}}%
\pgfpathlineto{\pgfqpoint{0.990982in}{0.831301in}}%
\pgfpathlineto{\pgfqpoint{0.993234in}{0.834770in}}%
\pgfpathlineto{\pgfqpoint{0.994100in}{0.832432in}}%
\pgfpathlineto{\pgfqpoint{0.997218in}{0.823508in}}%
\pgfpathlineto{\pgfqpoint{1.002415in}{0.805180in}}%
\pgfpathlineto{\pgfqpoint{1.002588in}{0.805411in}}%
\pgfpathlineto{\pgfqpoint{1.007092in}{0.813855in}}%
\pgfpathlineto{\pgfqpoint{1.007438in}{0.813420in}}%
\pgfpathlineto{\pgfqpoint{1.009170in}{0.814649in}}%
\pgfpathlineto{\pgfqpoint{1.009517in}{0.813392in}}%
\pgfpathlineto{\pgfqpoint{1.012462in}{0.811028in}}%
\pgfpathlineto{\pgfqpoint{1.014887in}{0.810428in}}%
\pgfpathlineto{\pgfqpoint{1.017831in}{0.807118in}}%
\pgfpathlineto{\pgfqpoint{1.020256in}{0.805745in}}%
\pgfpathlineto{\pgfqpoint{1.021469in}{0.808110in}}%
\pgfpathlineto{\pgfqpoint{1.023374in}{0.812420in}}%
\pgfpathlineto{\pgfqpoint{1.023721in}{0.811876in}}%
\pgfpathlineto{\pgfqpoint{1.027532in}{0.804738in}}%
\pgfpathlineto{\pgfqpoint{1.034461in}{0.788521in}}%
\pgfpathlineto{\pgfqpoint{1.035153in}{0.790787in}}%
\pgfpathlineto{\pgfqpoint{1.035846in}{0.792731in}}%
\pgfpathlineto{\pgfqpoint{1.036366in}{0.790214in}}%
\pgfpathlineto{\pgfqpoint{1.037405in}{0.788345in}}%
\pgfpathlineto{\pgfqpoint{1.037925in}{0.789779in}}%
\pgfpathlineto{\pgfqpoint{1.038964in}{0.791433in}}%
\pgfpathlineto{\pgfqpoint{1.039311in}{0.790140in}}%
\pgfpathlineto{\pgfqpoint{1.041909in}{0.781601in}}%
\pgfpathlineto{\pgfqpoint{1.042082in}{0.781658in}}%
\pgfpathlineto{\pgfqpoint{1.043988in}{0.781931in}}%
\pgfpathlineto{\pgfqpoint{1.044161in}{0.781491in}}%
\pgfpathlineto{\pgfqpoint{1.045200in}{0.780884in}}%
\pgfpathlineto{\pgfqpoint{1.045547in}{0.781560in}}%
\pgfpathlineto{\pgfqpoint{1.046586in}{0.783741in}}%
\pgfpathlineto{\pgfqpoint{1.047279in}{0.781737in}}%
\pgfpathlineto{\pgfqpoint{1.051436in}{0.769736in}}%
\pgfpathlineto{\pgfqpoint{1.052475in}{0.770997in}}%
\pgfpathlineto{\pgfqpoint{1.053515in}{0.772690in}}%
\pgfpathlineto{\pgfqpoint{1.054034in}{0.771226in}}%
\pgfpathlineto{\pgfqpoint{1.057326in}{0.764167in}}%
\pgfpathlineto{\pgfqpoint{1.057499in}{0.764224in}}%
\pgfpathlineto{\pgfqpoint{1.058538in}{0.765822in}}%
\pgfpathlineto{\pgfqpoint{1.059058in}{0.765052in}}%
\pgfpathlineto{\pgfqpoint{1.062869in}{0.750704in}}%
\pgfpathlineto{\pgfqpoint{1.063388in}{0.752166in}}%
\pgfpathlineto{\pgfqpoint{1.067892in}{0.764097in}}%
\pgfpathlineto{\pgfqpoint{1.068238in}{0.763486in}}%
\pgfpathlineto{\pgfqpoint{1.068412in}{0.763747in}}%
\pgfpathlineto{\pgfqpoint{1.068758in}{0.762817in}}%
\pgfpathlineto{\pgfqpoint{1.070837in}{0.753508in}}%
\pgfpathlineto{\pgfqpoint{1.071183in}{0.754840in}}%
\pgfpathlineto{\pgfqpoint{1.072049in}{0.752738in}}%
\pgfpathlineto{\pgfqpoint{1.074301in}{0.743743in}}%
\pgfpathlineto{\pgfqpoint{1.074821in}{0.743968in}}%
\pgfpathlineto{\pgfqpoint{1.077766in}{0.737840in}}%
\pgfpathlineto{\pgfqpoint{1.077939in}{0.737930in}}%
\pgfpathlineto{\pgfqpoint{1.079151in}{0.738544in}}%
\pgfpathlineto{\pgfqpoint{1.079325in}{0.737905in}}%
\pgfpathlineto{\pgfqpoint{1.081230in}{0.733582in}}%
\pgfpathlineto{\pgfqpoint{1.081403in}{0.733840in}}%
\pgfpathlineto{\pgfqpoint{1.083135in}{0.738008in}}%
\pgfpathlineto{\pgfqpoint{1.083828in}{0.741895in}}%
\pgfpathlineto{\pgfqpoint{1.084868in}{0.740327in}}%
\pgfpathlineto{\pgfqpoint{1.085561in}{0.742060in}}%
\pgfpathlineto{\pgfqpoint{1.085734in}{0.742781in}}%
\pgfpathlineto{\pgfqpoint{1.086600in}{0.740236in}}%
\pgfpathlineto{\pgfqpoint{1.087812in}{0.736510in}}%
\pgfpathlineto{\pgfqpoint{1.088159in}{0.738603in}}%
\pgfpathlineto{\pgfqpoint{1.089198in}{0.741028in}}%
\pgfpathlineto{\pgfqpoint{1.089891in}{0.739397in}}%
\pgfpathlineto{\pgfqpoint{1.093529in}{0.733159in}}%
\pgfpathlineto{\pgfqpoint{1.094222in}{0.734306in}}%
\pgfpathlineto{\pgfqpoint{1.095607in}{0.731851in}}%
\pgfpathlineto{\pgfqpoint{1.096300in}{0.734096in}}%
\pgfpathlineto{\pgfqpoint{1.098725in}{0.741773in}}%
\pgfpathlineto{\pgfqpoint{1.099591in}{0.740252in}}%
\pgfpathlineto{\pgfqpoint{1.099765in}{0.740077in}}%
\pgfpathlineto{\pgfqpoint{1.100111in}{0.741767in}}%
\pgfpathlineto{\pgfqpoint{1.101670in}{0.742470in}}%
\pgfpathlineto{\pgfqpoint{1.101843in}{0.742363in}}%
\pgfpathlineto{\pgfqpoint{1.105308in}{0.731748in}}%
\pgfpathlineto{\pgfqpoint{1.106347in}{0.727383in}}%
\pgfpathlineto{\pgfqpoint{1.107040in}{0.728896in}}%
\pgfpathlineto{\pgfqpoint{1.107213in}{0.729520in}}%
\pgfpathlineto{\pgfqpoint{1.107560in}{0.726922in}}%
\pgfpathlineto{\pgfqpoint{1.109638in}{0.718181in}}%
\pgfpathlineto{\pgfqpoint{1.110331in}{0.716658in}}%
\pgfpathlineto{\pgfqpoint{1.111024in}{0.717609in}}%
\pgfpathlineto{\pgfqpoint{1.112756in}{0.727539in}}%
\pgfpathlineto{\pgfqpoint{1.113449in}{0.723530in}}%
\pgfpathlineto{\pgfqpoint{1.114142in}{0.721509in}}%
\pgfpathlineto{\pgfqpoint{1.114662in}{0.723476in}}%
\pgfpathlineto{\pgfqpoint{1.116221in}{0.729903in}}%
\pgfpathlineto{\pgfqpoint{1.116740in}{0.728533in}}%
\pgfpathlineto{\pgfqpoint{1.117087in}{0.727555in}}%
\pgfpathlineto{\pgfqpoint{1.119165in}{0.719499in}}%
\pgfpathlineto{\pgfqpoint{1.119512in}{0.720910in}}%
\pgfpathlineto{\pgfqpoint{1.120378in}{0.726070in}}%
\pgfpathlineto{\pgfqpoint{1.121071in}{0.721662in}}%
\pgfpathlineto{\pgfqpoint{1.121764in}{0.722288in}}%
\pgfpathlineto{\pgfqpoint{1.122110in}{0.720650in}}%
\pgfpathlineto{\pgfqpoint{1.125228in}{0.708467in}}%
\pgfpathlineto{\pgfqpoint{1.125921in}{0.709136in}}%
\pgfpathlineto{\pgfqpoint{1.126787in}{0.702907in}}%
\pgfpathlineto{\pgfqpoint{1.127653in}{0.698443in}}%
\pgfpathlineto{\pgfqpoint{1.128346in}{0.700494in}}%
\pgfpathlineto{\pgfqpoint{1.128866in}{0.702100in}}%
\pgfpathlineto{\pgfqpoint{1.129905in}{0.700395in}}%
\pgfpathlineto{\pgfqpoint{1.130251in}{0.699781in}}%
\pgfpathlineto{\pgfqpoint{1.132157in}{0.694649in}}%
\pgfpathlineto{\pgfqpoint{1.134062in}{0.698988in}}%
\pgfpathlineto{\pgfqpoint{1.134235in}{0.697742in}}%
\pgfpathlineto{\pgfqpoint{1.137007in}{0.690041in}}%
\pgfpathlineto{\pgfqpoint{1.137353in}{0.690162in}}%
\pgfpathlineto{\pgfqpoint{1.138739in}{0.689346in}}%
\pgfpathlineto{\pgfqpoint{1.140645in}{0.686142in}}%
\pgfpathlineto{\pgfqpoint{1.141684in}{0.681920in}}%
\pgfpathlineto{\pgfqpoint{1.142896in}{0.682668in}}%
\pgfpathlineto{\pgfqpoint{1.144629in}{0.685164in}}%
\pgfpathlineto{\pgfqpoint{1.146188in}{0.694252in}}%
\pgfpathlineto{\pgfqpoint{1.148093in}{0.692375in}}%
\pgfpathlineto{\pgfqpoint{1.148266in}{0.692447in}}%
\pgfpathlineto{\pgfqpoint{1.148613in}{0.691165in}}%
\pgfpathlineto{\pgfqpoint{1.151384in}{0.687215in}}%
\pgfpathlineto{\pgfqpoint{1.153463in}{0.690538in}}%
\pgfpathlineto{\pgfqpoint{1.154329in}{0.688987in}}%
\pgfpathlineto{\pgfqpoint{1.154849in}{0.682930in}}%
\pgfpathlineto{\pgfqpoint{1.156927in}{0.672576in}}%
\pgfpathlineto{\pgfqpoint{1.157967in}{0.667682in}}%
\pgfpathlineto{\pgfqpoint{1.159006in}{0.661004in}}%
\pgfpathlineto{\pgfqpoint{1.159872in}{0.664497in}}%
\pgfpathlineto{\pgfqpoint{1.160565in}{0.666596in}}%
\pgfpathlineto{\pgfqpoint{1.160738in}{0.667625in}}%
\pgfpathlineto{\pgfqpoint{1.161431in}{0.663872in}}%
\pgfpathlineto{\pgfqpoint{1.163163in}{0.654286in}}%
\pgfpathlineto{\pgfqpoint{1.164029in}{0.657654in}}%
\pgfpathlineto{\pgfqpoint{1.164722in}{0.653637in}}%
\pgfpathlineto{\pgfqpoint{1.165415in}{0.646993in}}%
\pgfpathlineto{\pgfqpoint{1.166281in}{0.652763in}}%
\pgfpathlineto{\pgfqpoint{1.167667in}{0.646339in}}%
\pgfpathlineto{\pgfqpoint{1.168187in}{0.650290in}}%
\pgfpathlineto{\pgfqpoint{1.170612in}{0.662201in}}%
\pgfpathlineto{\pgfqpoint{1.170958in}{0.661668in}}%
\pgfpathlineto{\pgfqpoint{1.171651in}{0.663341in}}%
\pgfpathlineto{\pgfqpoint{1.174076in}{0.676853in}}%
\pgfpathlineto{\pgfqpoint{1.174249in}{0.677167in}}%
\pgfpathlineto{\pgfqpoint{1.174596in}{0.675989in}}%
\pgfpathlineto{\pgfqpoint{1.174942in}{0.676161in}}%
\pgfpathlineto{\pgfqpoint{1.175982in}{0.671671in}}%
\pgfpathlineto{\pgfqpoint{1.176674in}{0.674942in}}%
\pgfpathlineto{\pgfqpoint{1.178233in}{0.688684in}}%
\pgfpathlineto{\pgfqpoint{1.178753in}{0.687672in}}%
\pgfpathlineto{\pgfqpoint{1.179966in}{0.685187in}}%
\pgfpathlineto{\pgfqpoint{1.180832in}{0.686509in}}%
\pgfpathlineto{\pgfqpoint{1.182217in}{0.694148in}}%
\pgfpathlineto{\pgfqpoint{1.182737in}{0.690521in}}%
\pgfpathlineto{\pgfqpoint{1.184643in}{0.677204in}}%
\pgfpathlineto{\pgfqpoint{1.185162in}{0.678795in}}%
\pgfpathlineto{\pgfqpoint{1.187934in}{0.682954in}}%
\pgfpathlineto{\pgfqpoint{1.188453in}{0.681770in}}%
\pgfpathlineto{\pgfqpoint{1.188800in}{0.684141in}}%
\pgfpathlineto{\pgfqpoint{1.189839in}{0.689348in}}%
\pgfpathlineto{\pgfqpoint{1.191052in}{0.696576in}}%
\pgfpathlineto{\pgfqpoint{1.191745in}{0.693764in}}%
\pgfpathlineto{\pgfqpoint{1.194343in}{0.686172in}}%
\pgfpathlineto{\pgfqpoint{1.195036in}{0.687749in}}%
\pgfpathlineto{\pgfqpoint{1.195729in}{0.689170in}}%
\pgfpathlineto{\pgfqpoint{1.196768in}{0.687677in}}%
\pgfpathlineto{\pgfqpoint{1.197288in}{0.687127in}}%
\pgfpathlineto{\pgfqpoint{1.197461in}{0.688254in}}%
\pgfpathlineto{\pgfqpoint{1.198154in}{0.689861in}}%
\pgfpathlineto{\pgfqpoint{1.198847in}{0.687436in}}%
\pgfpathlineto{\pgfqpoint{1.201618in}{0.675146in}}%
\pgfpathlineto{\pgfqpoint{1.204736in}{0.660021in}}%
\pgfpathlineto{\pgfqpoint{1.205083in}{0.661213in}}%
\pgfpathlineto{\pgfqpoint{1.205602in}{0.664326in}}%
\pgfpathlineto{\pgfqpoint{1.206468in}{0.660782in}}%
\pgfpathlineto{\pgfqpoint{1.208720in}{0.656613in}}%
\pgfpathlineto{\pgfqpoint{1.212011in}{0.662188in}}%
\pgfpathlineto{\pgfqpoint{1.212185in}{0.661389in}}%
\pgfpathlineto{\pgfqpoint{1.213570in}{0.657824in}}%
\pgfpathlineto{\pgfqpoint{1.214090in}{0.659571in}}%
\pgfpathlineto{\pgfqpoint{1.214956in}{0.661125in}}%
\pgfpathlineto{\pgfqpoint{1.215303in}{0.658863in}}%
\pgfpathlineto{\pgfqpoint{1.215822in}{0.657560in}}%
\pgfpathlineto{\pgfqpoint{1.216342in}{0.660060in}}%
\pgfpathlineto{\pgfqpoint{1.217901in}{0.671325in}}%
\pgfpathlineto{\pgfqpoint{1.218594in}{0.667369in}}%
\pgfpathlineto{\pgfqpoint{1.219460in}{0.661581in}}%
\pgfpathlineto{\pgfqpoint{1.220326in}{0.663280in}}%
\pgfpathlineto{\pgfqpoint{1.221192in}{0.664234in}}%
\pgfpathlineto{\pgfqpoint{1.221538in}{0.662960in}}%
\pgfpathlineto{\pgfqpoint{1.224137in}{0.640166in}}%
\pgfpathlineto{\pgfqpoint{1.226042in}{0.631236in}}%
\pgfpathlineto{\pgfqpoint{1.226215in}{0.631442in}}%
\pgfpathlineto{\pgfqpoint{1.231066in}{0.650886in}}%
\pgfpathlineto{\pgfqpoint{1.231412in}{0.650420in}}%
\pgfpathlineto{\pgfqpoint{1.237475in}{0.672322in}}%
\pgfpathlineto{\pgfqpoint{1.240593in}{0.682429in}}%
\pgfpathlineto{\pgfqpoint{1.240766in}{0.681788in}}%
\pgfpathlineto{\pgfqpoint{1.243884in}{0.652355in}}%
\pgfpathlineto{\pgfqpoint{1.244577in}{0.649016in}}%
\pgfpathlineto{\pgfqpoint{1.245270in}{0.653407in}}%
\pgfpathlineto{\pgfqpoint{1.249773in}{0.680727in}}%
\pgfpathlineto{\pgfqpoint{1.251332in}{0.684291in}}%
\pgfpathlineto{\pgfqpoint{1.252025in}{0.681864in}}%
\pgfpathlineto{\pgfqpoint{1.252372in}{0.681014in}}%
\pgfpathlineto{\pgfqpoint{1.252718in}{0.682408in}}%
\pgfpathlineto{\pgfqpoint{1.253065in}{0.683798in}}%
\pgfpathlineto{\pgfqpoint{1.253757in}{0.681186in}}%
\pgfpathlineto{\pgfqpoint{1.253931in}{0.681422in}}%
\pgfpathlineto{\pgfqpoint{1.254624in}{0.677602in}}%
\pgfpathlineto{\pgfqpoint{1.255316in}{0.680375in}}%
\pgfpathlineto{\pgfqpoint{1.256356in}{0.681899in}}%
\pgfpathlineto{\pgfqpoint{1.256875in}{0.680985in}}%
\pgfpathlineto{\pgfqpoint{1.257049in}{0.681276in}}%
\pgfpathlineto{\pgfqpoint{1.257395in}{0.679442in}}%
\pgfpathlineto{\pgfqpoint{1.257742in}{0.679541in}}%
\pgfpathlineto{\pgfqpoint{1.259300in}{0.675201in}}%
\pgfpathlineto{\pgfqpoint{1.259820in}{0.677871in}}%
\pgfpathlineto{\pgfqpoint{1.263111in}{0.700383in}}%
\pgfpathlineto{\pgfqpoint{1.263631in}{0.701164in}}%
\pgfpathlineto{\pgfqpoint{1.264497in}{0.699914in}}%
\pgfpathlineto{\pgfqpoint{1.267962in}{0.682363in}}%
\pgfpathlineto{\pgfqpoint{1.268135in}{0.682455in}}%
\pgfpathlineto{\pgfqpoint{1.271253in}{0.693356in}}%
\pgfpathlineto{\pgfqpoint{1.272119in}{0.690259in}}%
\pgfpathlineto{\pgfqpoint{1.272292in}{0.690341in}}%
\pgfpathlineto{\pgfqpoint{1.272465in}{0.689353in}}%
\pgfpathlineto{\pgfqpoint{1.273331in}{0.682803in}}%
\pgfpathlineto{\pgfqpoint{1.274197in}{0.687061in}}%
\pgfpathlineto{\pgfqpoint{1.276623in}{0.697103in}}%
\pgfpathlineto{\pgfqpoint{1.276796in}{0.696684in}}%
\pgfpathlineto{\pgfqpoint{1.277835in}{0.690167in}}%
\pgfpathlineto{\pgfqpoint{1.278701in}{0.692341in}}%
\pgfpathlineto{\pgfqpoint{1.278874in}{0.692830in}}%
\pgfpathlineto{\pgfqpoint{1.279567in}{0.690368in}}%
\pgfpathlineto{\pgfqpoint{1.280260in}{0.688004in}}%
\pgfpathlineto{\pgfqpoint{1.281646in}{0.679135in}}%
\pgfpathlineto{\pgfqpoint{1.283032in}{0.682706in}}%
\pgfpathlineto{\pgfqpoint{1.283205in}{0.683095in}}%
\pgfpathlineto{\pgfqpoint{1.283898in}{0.681160in}}%
\pgfpathlineto{\pgfqpoint{1.284071in}{0.680665in}}%
\pgfpathlineto{\pgfqpoint{1.284417in}{0.683359in}}%
\pgfpathlineto{\pgfqpoint{1.285110in}{0.691613in}}%
\pgfpathlineto{\pgfqpoint{1.286150in}{0.685819in}}%
\pgfpathlineto{\pgfqpoint{1.286496in}{0.686769in}}%
\pgfpathlineto{\pgfqpoint{1.286843in}{0.684733in}}%
\pgfpathlineto{\pgfqpoint{1.288748in}{0.677744in}}%
\pgfpathlineto{\pgfqpoint{1.290827in}{0.663367in}}%
\pgfpathlineto{\pgfqpoint{1.292039in}{0.660759in}}%
\pgfpathlineto{\pgfqpoint{1.295504in}{0.637597in}}%
\pgfpathlineto{\pgfqpoint{1.300527in}{0.662069in}}%
\pgfpathlineto{\pgfqpoint{1.301047in}{0.660400in}}%
\pgfpathlineto{\pgfqpoint{1.301739in}{0.658395in}}%
\pgfpathlineto{\pgfqpoint{1.302086in}{0.661277in}}%
\pgfpathlineto{\pgfqpoint{1.304684in}{0.667671in}}%
\pgfpathlineto{\pgfqpoint{1.305377in}{0.665385in}}%
\pgfpathlineto{\pgfqpoint{1.306763in}{0.665896in}}%
\pgfpathlineto{\pgfqpoint{1.306936in}{0.666368in}}%
\pgfpathlineto{\pgfqpoint{1.307802in}{0.665286in}}%
\pgfpathlineto{\pgfqpoint{1.308495in}{0.661050in}}%
\pgfpathlineto{\pgfqpoint{1.309188in}{0.665691in}}%
\pgfpathlineto{\pgfqpoint{1.310920in}{0.672776in}}%
\pgfpathlineto{\pgfqpoint{1.311786in}{0.671919in}}%
\pgfpathlineto{\pgfqpoint{1.313345in}{0.668012in}}%
\pgfpathlineto{\pgfqpoint{1.314038in}{0.670894in}}%
\pgfpathlineto{\pgfqpoint{1.319062in}{0.689967in}}%
\pgfpathlineto{\pgfqpoint{1.319235in}{0.689240in}}%
\pgfpathlineto{\pgfqpoint{1.320620in}{0.684167in}}%
\pgfpathlineto{\pgfqpoint{1.321140in}{0.684584in}}%
\pgfpathlineto{\pgfqpoint{1.324085in}{0.673731in}}%
\pgfpathlineto{\pgfqpoint{1.324951in}{0.676157in}}%
\pgfpathlineto{\pgfqpoint{1.325124in}{0.676576in}}%
\pgfpathlineto{\pgfqpoint{1.325817in}{0.675006in}}%
\pgfpathlineto{\pgfqpoint{1.328415in}{0.670335in}}%
\pgfpathlineto{\pgfqpoint{1.328589in}{0.670501in}}%
\pgfpathlineto{\pgfqpoint{1.329282in}{0.670239in}}%
\pgfpathlineto{\pgfqpoint{1.329628in}{0.671412in}}%
\pgfpathlineto{\pgfqpoint{1.329974in}{0.672215in}}%
\pgfpathlineto{\pgfqpoint{1.330494in}{0.669627in}}%
\pgfpathlineto{\pgfqpoint{1.330840in}{0.668765in}}%
\pgfpathlineto{\pgfqpoint{1.331707in}{0.670048in}}%
\pgfpathlineto{\pgfqpoint{1.332226in}{0.672474in}}%
\pgfpathlineto{\pgfqpoint{1.333266in}{0.670606in}}%
\pgfpathlineto{\pgfqpoint{1.335691in}{0.673046in}}%
\pgfpathlineto{\pgfqpoint{1.337076in}{0.667270in}}%
\pgfpathlineto{\pgfqpoint{1.337769in}{0.669980in}}%
\pgfpathlineto{\pgfqpoint{1.338809in}{0.669561in}}%
\pgfpathlineto{\pgfqpoint{1.340714in}{0.661835in}}%
\pgfpathlineto{\pgfqpoint{1.340887in}{0.662243in}}%
\pgfpathlineto{\pgfqpoint{1.342446in}{0.668895in}}%
\pgfpathlineto{\pgfqpoint{1.343486in}{0.667307in}}%
\pgfpathlineto{\pgfqpoint{1.344352in}{0.665284in}}%
\pgfpathlineto{\pgfqpoint{1.344871in}{0.666966in}}%
\pgfpathlineto{\pgfqpoint{1.345045in}{0.667506in}}%
\pgfpathlineto{\pgfqpoint{1.345391in}{0.665031in}}%
\pgfpathlineto{\pgfqpoint{1.347643in}{0.654329in}}%
\pgfpathlineto{\pgfqpoint{1.348163in}{0.656578in}}%
\pgfpathlineto{\pgfqpoint{1.349721in}{0.663964in}}%
\pgfpathlineto{\pgfqpoint{1.350241in}{0.661834in}}%
\pgfpathlineto{\pgfqpoint{1.352666in}{0.649810in}}%
\pgfpathlineto{\pgfqpoint{1.352839in}{0.650391in}}%
\pgfpathlineto{\pgfqpoint{1.353532in}{0.654704in}}%
\pgfpathlineto{\pgfqpoint{1.354398in}{0.651895in}}%
\pgfpathlineto{\pgfqpoint{1.355265in}{0.654218in}}%
\pgfpathlineto{\pgfqpoint{1.356131in}{0.652134in}}%
\pgfpathlineto{\pgfqpoint{1.356824in}{0.651487in}}%
\pgfpathlineto{\pgfqpoint{1.357516in}{0.652369in}}%
\pgfpathlineto{\pgfqpoint{1.358209in}{0.656206in}}%
\pgfpathlineto{\pgfqpoint{1.359249in}{0.660944in}}%
\pgfpathlineto{\pgfqpoint{1.359941in}{0.658879in}}%
\pgfpathlineto{\pgfqpoint{1.360288in}{0.657782in}}%
\pgfpathlineto{\pgfqpoint{1.360808in}{0.660440in}}%
\pgfpathlineto{\pgfqpoint{1.361674in}{0.667094in}}%
\pgfpathlineto{\pgfqpoint{1.362193in}{0.660959in}}%
\pgfpathlineto{\pgfqpoint{1.362367in}{0.659486in}}%
\pgfpathlineto{\pgfqpoint{1.363579in}{0.661826in}}%
\pgfpathlineto{\pgfqpoint{1.364099in}{0.664653in}}%
\pgfpathlineto{\pgfqpoint{1.364792in}{0.660025in}}%
\pgfpathlineto{\pgfqpoint{1.366870in}{0.641576in}}%
\pgfpathlineto{\pgfqpoint{1.367563in}{0.643447in}}%
\pgfpathlineto{\pgfqpoint{1.367736in}{0.642798in}}%
\pgfpathlineto{\pgfqpoint{1.368256in}{0.645111in}}%
\pgfpathlineto{\pgfqpoint{1.368776in}{0.643548in}}%
\pgfpathlineto{\pgfqpoint{1.375358in}{0.677491in}}%
\pgfpathlineto{\pgfqpoint{1.375878in}{0.676409in}}%
\pgfpathlineto{\pgfqpoint{1.380035in}{0.659726in}}%
\pgfpathlineto{\pgfqpoint{1.381421in}{0.655176in}}%
\pgfpathlineto{\pgfqpoint{1.381594in}{0.656413in}}%
\pgfpathlineto{\pgfqpoint{1.383846in}{0.661300in}}%
\pgfpathlineto{\pgfqpoint{1.384539in}{0.662408in}}%
\pgfpathlineto{\pgfqpoint{1.384712in}{0.661799in}}%
\pgfpathlineto{\pgfqpoint{1.387310in}{0.648026in}}%
\pgfpathlineto{\pgfqpoint{1.387484in}{0.648699in}}%
\pgfpathlineto{\pgfqpoint{1.392160in}{0.668888in}}%
\pgfpathlineto{\pgfqpoint{1.393373in}{0.667646in}}%
\pgfpathlineto{\pgfqpoint{1.394412in}{0.660488in}}%
\pgfpathlineto{\pgfqpoint{1.395452in}{0.664294in}}%
\pgfpathlineto{\pgfqpoint{1.395798in}{0.665297in}}%
\pgfpathlineto{\pgfqpoint{1.396145in}{0.663922in}}%
\pgfpathlineto{\pgfqpoint{1.397357in}{0.656290in}}%
\pgfpathlineto{\pgfqpoint{1.397877in}{0.659320in}}%
\pgfpathlineto{\pgfqpoint{1.401514in}{0.690829in}}%
\pgfpathlineto{\pgfqpoint{1.403073in}{0.692488in}}%
\pgfpathlineto{\pgfqpoint{1.403247in}{0.691105in}}%
\pgfpathlineto{\pgfqpoint{1.405152in}{0.684934in}}%
\pgfpathlineto{\pgfqpoint{1.405845in}{0.687024in}}%
\pgfpathlineto{\pgfqpoint{1.407057in}{0.694176in}}%
\pgfpathlineto{\pgfqpoint{1.407750in}{0.692845in}}%
\pgfpathlineto{\pgfqpoint{1.408616in}{0.692676in}}%
\pgfpathlineto{\pgfqpoint{1.408790in}{0.692088in}}%
\pgfpathlineto{\pgfqpoint{1.409829in}{0.686119in}}%
\pgfpathlineto{\pgfqpoint{1.410868in}{0.687569in}}%
\pgfpathlineto{\pgfqpoint{1.412774in}{0.691171in}}%
\pgfpathlineto{\pgfqpoint{1.413467in}{0.689982in}}%
\pgfpathlineto{\pgfqpoint{1.418144in}{0.674678in}}%
\pgfpathlineto{\pgfqpoint{1.418317in}{0.675135in}}%
\pgfpathlineto{\pgfqpoint{1.419010in}{0.683187in}}%
\pgfpathlineto{\pgfqpoint{1.420222in}{0.680761in}}%
\pgfpathlineto{\pgfqpoint{1.422301in}{0.671171in}}%
\pgfpathlineto{\pgfqpoint{1.422647in}{0.672184in}}%
\pgfpathlineto{\pgfqpoint{1.423687in}{0.673824in}}%
\pgfpathlineto{\pgfqpoint{1.424553in}{0.676896in}}%
\pgfpathlineto{\pgfqpoint{1.425246in}{0.675065in}}%
\pgfpathlineto{\pgfqpoint{1.428017in}{0.665976in}}%
\pgfpathlineto{\pgfqpoint{1.430096in}{0.661689in}}%
\pgfpathlineto{\pgfqpoint{1.430269in}{0.662015in}}%
\pgfpathlineto{\pgfqpoint{1.430789in}{0.659864in}}%
\pgfpathlineto{\pgfqpoint{1.431481in}{0.660947in}}%
\pgfpathlineto{\pgfqpoint{1.434080in}{0.671768in}}%
\pgfpathlineto{\pgfqpoint{1.434426in}{0.670456in}}%
\pgfpathlineto{\pgfqpoint{1.434773in}{0.667097in}}%
\pgfpathlineto{\pgfqpoint{1.435292in}{0.678959in}}%
\pgfpathlineto{\pgfqpoint{1.436158in}{0.676314in}}%
\pgfpathlineto{\pgfqpoint{1.436851in}{0.702516in}}%
\pgfpathlineto{\pgfqpoint{1.440489in}{0.802318in}}%
\pgfpathlineto{\pgfqpoint{1.447764in}{0.916832in}}%
\pgfpathlineto{\pgfqpoint{1.450016in}{0.918610in}}%
\pgfpathlineto{\pgfqpoint{1.450363in}{0.918399in}}%
\pgfpathlineto{\pgfqpoint{1.452614in}{0.920087in}}%
\pgfpathlineto{\pgfqpoint{1.455906in}{0.945331in}}%
\pgfpathlineto{\pgfqpoint{1.457465in}{0.941016in}}%
\pgfpathlineto{\pgfqpoint{1.458331in}{0.937138in}}%
\pgfpathlineto{\pgfqpoint{1.459370in}{0.939349in}}%
\pgfpathlineto{\pgfqpoint{1.461102in}{0.940085in}}%
\pgfpathlineto{\pgfqpoint{1.461449in}{0.939360in}}%
\pgfpathlineto{\pgfqpoint{1.466818in}{0.920172in}}%
\pgfpathlineto{\pgfqpoint{1.470110in}{0.897729in}}%
\pgfpathlineto{\pgfqpoint{1.479810in}{0.812946in}}%
\pgfpathlineto{\pgfqpoint{1.486046in}{0.759306in}}%
\pgfpathlineto{\pgfqpoint{1.492628in}{0.717908in}}%
\pgfpathlineto{\pgfqpoint{1.502155in}{0.666892in}}%
\pgfpathlineto{\pgfqpoint{1.502848in}{0.668111in}}%
\pgfpathlineto{\pgfqpoint{1.507525in}{0.659157in}}%
\pgfpathlineto{\pgfqpoint{1.508218in}{0.662250in}}%
\pgfpathlineto{\pgfqpoint{1.509257in}{0.666654in}}%
\pgfpathlineto{\pgfqpoint{1.510470in}{0.665996in}}%
\pgfpathlineto{\pgfqpoint{1.511856in}{0.668205in}}%
\pgfpathlineto{\pgfqpoint{1.514454in}{0.677892in}}%
\pgfpathlineto{\pgfqpoint{1.514800in}{0.679098in}}%
\pgfpathlineto{\pgfqpoint{1.515840in}{0.677397in}}%
\pgfpathlineto{\pgfqpoint{1.516186in}{0.676230in}}%
\pgfpathlineto{\pgfqpoint{1.516879in}{0.677712in}}%
\pgfpathlineto{\pgfqpoint{1.521036in}{0.705096in}}%
\pgfpathlineto{\pgfqpoint{1.521383in}{0.704117in}}%
\pgfpathlineto{\pgfqpoint{1.525194in}{0.683907in}}%
\pgfpathlineto{\pgfqpoint{1.527965in}{0.659296in}}%
\pgfpathlineto{\pgfqpoint{1.528831in}{0.664254in}}%
\pgfpathlineto{\pgfqpoint{1.531083in}{0.670620in}}%
\pgfpathlineto{\pgfqpoint{1.531430in}{0.670093in}}%
\pgfpathlineto{\pgfqpoint{1.534894in}{0.657854in}}%
\pgfpathlineto{\pgfqpoint{1.535067in}{0.658327in}}%
\pgfpathlineto{\pgfqpoint{1.536799in}{0.663686in}}%
\pgfpathlineto{\pgfqpoint{1.537319in}{0.661544in}}%
\pgfpathlineto{\pgfqpoint{1.543209in}{0.643182in}}%
\pgfpathlineto{\pgfqpoint{1.543382in}{0.643698in}}%
\pgfpathlineto{\pgfqpoint{1.545287in}{0.650866in}}%
\pgfpathlineto{\pgfqpoint{1.545807in}{0.653299in}}%
\pgfpathlineto{\pgfqpoint{1.546673in}{0.650221in}}%
\pgfpathlineto{\pgfqpoint{1.548925in}{0.644074in}}%
\pgfpathlineto{\pgfqpoint{1.554988in}{0.613673in}}%
\pgfpathlineto{\pgfqpoint{1.555334in}{0.615506in}}%
\pgfpathlineto{\pgfqpoint{1.556547in}{0.619211in}}%
\pgfpathlineto{\pgfqpoint{1.557413in}{0.618885in}}%
\pgfpathlineto{\pgfqpoint{1.559145in}{0.622424in}}%
\pgfpathlineto{\pgfqpoint{1.559838in}{0.619857in}}%
\pgfpathlineto{\pgfqpoint{1.560184in}{0.618485in}}%
\pgfpathlineto{\pgfqpoint{1.560877in}{0.622090in}}%
\pgfpathlineto{\pgfqpoint{1.563822in}{0.630958in}}%
\pgfpathlineto{\pgfqpoint{1.565381in}{0.633858in}}%
\pgfpathlineto{\pgfqpoint{1.568672in}{0.642471in}}%
\pgfpathlineto{\pgfqpoint{1.569365in}{0.639095in}}%
\pgfpathlineto{\pgfqpoint{1.570231in}{0.642432in}}%
\pgfpathlineto{\pgfqpoint{1.570404in}{0.642153in}}%
\pgfpathlineto{\pgfqpoint{1.571097in}{0.643837in}}%
\pgfpathlineto{\pgfqpoint{1.571963in}{0.647050in}}%
\pgfpathlineto{\pgfqpoint{1.572829in}{0.644932in}}%
\pgfpathlineto{\pgfqpoint{1.573002in}{0.644834in}}%
\pgfpathlineto{\pgfqpoint{1.573349in}{0.645788in}}%
\pgfpathlineto{\pgfqpoint{1.573869in}{0.648011in}}%
\pgfpathlineto{\pgfqpoint{1.574561in}{0.645369in}}%
\pgfpathlineto{\pgfqpoint{1.574735in}{0.645468in}}%
\pgfpathlineto{\pgfqpoint{1.575601in}{0.647709in}}%
\pgfpathlineto{\pgfqpoint{1.578026in}{0.658735in}}%
\pgfpathlineto{\pgfqpoint{1.579065in}{0.654504in}}%
\pgfpathlineto{\pgfqpoint{1.579758in}{0.652283in}}%
\pgfpathlineto{\pgfqpoint{1.580797in}{0.653643in}}%
\pgfpathlineto{\pgfqpoint{1.581144in}{0.654424in}}%
\pgfpathlineto{\pgfqpoint{1.581663in}{0.651676in}}%
\pgfpathlineto{\pgfqpoint{1.584955in}{0.641513in}}%
\pgfpathlineto{\pgfqpoint{1.586687in}{0.638439in}}%
\pgfpathlineto{\pgfqpoint{1.589458in}{0.633558in}}%
\pgfpathlineto{\pgfqpoint{1.590151in}{0.634134in}}%
\pgfpathlineto{\pgfqpoint{1.590844in}{0.635244in}}%
\pgfpathlineto{\pgfqpoint{1.591537in}{0.633575in}}%
\pgfpathlineto{\pgfqpoint{1.592057in}{0.632696in}}%
\pgfpathlineto{\pgfqpoint{1.593616in}{0.624457in}}%
\pgfpathlineto{\pgfqpoint{1.594309in}{0.627645in}}%
\pgfpathlineto{\pgfqpoint{1.596734in}{0.635163in}}%
\pgfpathlineto{\pgfqpoint{1.597427in}{0.636309in}}%
\pgfpathlineto{\pgfqpoint{1.597773in}{0.634858in}}%
\pgfpathlineto{\pgfqpoint{1.600025in}{0.622677in}}%
\pgfpathlineto{\pgfqpoint{1.600718in}{0.623901in}}%
\pgfpathlineto{\pgfqpoint{1.602103in}{0.619533in}}%
\pgfpathlineto{\pgfqpoint{1.604355in}{0.604456in}}%
\pgfpathlineto{\pgfqpoint{1.605221in}{0.608406in}}%
\pgfpathlineto{\pgfqpoint{1.607127in}{0.612440in}}%
\pgfpathlineto{\pgfqpoint{1.607473in}{0.611203in}}%
\pgfpathlineto{\pgfqpoint{1.608339in}{0.613080in}}%
\pgfpathlineto{\pgfqpoint{1.608686in}{0.614883in}}%
\pgfpathlineto{\pgfqpoint{1.609552in}{0.611844in}}%
\pgfpathlineto{\pgfqpoint{1.610072in}{0.608865in}}%
\pgfpathlineto{\pgfqpoint{1.610418in}{0.614007in}}%
\pgfpathlineto{\pgfqpoint{1.610591in}{0.613823in}}%
\pgfpathlineto{\pgfqpoint{1.613363in}{0.621836in}}%
\pgfpathlineto{\pgfqpoint{1.613536in}{0.621790in}}%
\pgfpathlineto{\pgfqpoint{1.614056in}{0.622064in}}%
\pgfpathlineto{\pgfqpoint{1.614402in}{0.623072in}}%
\pgfpathlineto{\pgfqpoint{1.614749in}{0.623525in}}%
\pgfpathlineto{\pgfqpoint{1.614922in}{0.622767in}}%
\pgfpathlineto{\pgfqpoint{1.616308in}{0.612738in}}%
\pgfpathlineto{\pgfqpoint{1.617174in}{0.613769in}}%
\pgfpathlineto{\pgfqpoint{1.618213in}{0.608161in}}%
\pgfpathlineto{\pgfqpoint{1.622370in}{0.587374in}}%
\pgfpathlineto{\pgfqpoint{1.623063in}{0.590675in}}%
\pgfpathlineto{\pgfqpoint{1.623410in}{0.592104in}}%
\pgfpathlineto{\pgfqpoint{1.624276in}{0.590591in}}%
\pgfpathlineto{\pgfqpoint{1.627047in}{0.579477in}}%
\pgfpathlineto{\pgfqpoint{1.627913in}{0.577447in}}%
\pgfpathlineto{\pgfqpoint{1.628433in}{0.579647in}}%
\pgfpathlineto{\pgfqpoint{1.630685in}{0.593549in}}%
\pgfpathlineto{\pgfqpoint{1.633110in}{0.602249in}}%
\pgfpathlineto{\pgfqpoint{1.633456in}{0.604103in}}%
\pgfpathlineto{\pgfqpoint{1.634149in}{0.601039in}}%
\pgfpathlineto{\pgfqpoint{1.634322in}{0.600231in}}%
\pgfpathlineto{\pgfqpoint{1.635189in}{0.603364in}}%
\pgfpathlineto{\pgfqpoint{1.635881in}{0.605250in}}%
\pgfpathlineto{\pgfqpoint{1.636401in}{0.607875in}}%
\pgfpathlineto{\pgfqpoint{1.637440in}{0.605990in}}%
\pgfpathlineto{\pgfqpoint{1.637614in}{0.605581in}}%
\pgfpathlineto{\pgfqpoint{1.637960in}{0.607484in}}%
\pgfpathlineto{\pgfqpoint{1.638653in}{0.611009in}}%
\pgfpathlineto{\pgfqpoint{1.639519in}{0.608394in}}%
\pgfpathlineto{\pgfqpoint{1.642464in}{0.592415in}}%
\pgfpathlineto{\pgfqpoint{1.642637in}{0.592668in}}%
\pgfpathlineto{\pgfqpoint{1.644889in}{0.587563in}}%
\pgfpathlineto{\pgfqpoint{1.645582in}{0.588025in}}%
\pgfpathlineto{\pgfqpoint{1.645755in}{0.588902in}}%
\pgfpathlineto{\pgfqpoint{1.646448in}{0.584540in}}%
\pgfpathlineto{\pgfqpoint{1.646794in}{0.582191in}}%
\pgfpathlineto{\pgfqpoint{1.647834in}{0.585051in}}%
\pgfpathlineto{\pgfqpoint{1.649393in}{0.589905in}}%
\pgfpathlineto{\pgfqpoint{1.649912in}{0.586611in}}%
\pgfpathlineto{\pgfqpoint{1.652337in}{0.581116in}}%
\pgfpathlineto{\pgfqpoint{1.652857in}{0.583187in}}%
\pgfpathlineto{\pgfqpoint{1.654416in}{0.592368in}}%
\pgfpathlineto{\pgfqpoint{1.655455in}{0.591032in}}%
\pgfpathlineto{\pgfqpoint{1.658227in}{0.585462in}}%
\pgfpathlineto{\pgfqpoint{1.658400in}{0.584941in}}%
\pgfpathlineto{\pgfqpoint{1.659093in}{0.587304in}}%
\pgfpathlineto{\pgfqpoint{1.659613in}{0.585892in}}%
\pgfpathlineto{\pgfqpoint{1.661691in}{0.590161in}}%
\pgfpathlineto{\pgfqpoint{1.660479in}{0.585571in}}%
\pgfpathlineto{\pgfqpoint{1.662038in}{0.589340in}}%
\pgfpathlineto{\pgfqpoint{1.662384in}{0.588263in}}%
\pgfpathlineto{\pgfqpoint{1.662904in}{0.590797in}}%
\pgfpathlineto{\pgfqpoint{1.663423in}{0.590182in}}%
\pgfpathlineto{\pgfqpoint{1.665849in}{0.590556in}}%
\pgfpathlineto{\pgfqpoint{1.666541in}{0.596344in}}%
\pgfpathlineto{\pgfqpoint{1.667234in}{0.591775in}}%
\pgfpathlineto{\pgfqpoint{1.668967in}{0.580702in}}%
\pgfpathlineto{\pgfqpoint{1.669659in}{0.583967in}}%
\pgfpathlineto{\pgfqpoint{1.670179in}{0.586766in}}%
\pgfpathlineto{\pgfqpoint{1.670699in}{0.583226in}}%
\pgfpathlineto{\pgfqpoint{1.671565in}{0.576366in}}%
\pgfpathlineto{\pgfqpoint{1.672258in}{0.582673in}}%
\pgfpathlineto{\pgfqpoint{1.674683in}{0.594334in}}%
\pgfpathlineto{\pgfqpoint{1.674856in}{0.594246in}}%
\pgfpathlineto{\pgfqpoint{1.676761in}{0.590206in}}%
\pgfpathlineto{\pgfqpoint{1.676935in}{0.590375in}}%
\pgfpathlineto{\pgfqpoint{1.677801in}{0.594748in}}%
\pgfpathlineto{\pgfqpoint{1.680053in}{0.603413in}}%
\pgfpathlineto{\pgfqpoint{1.680226in}{0.603016in}}%
\pgfpathlineto{\pgfqpoint{1.680746in}{0.602304in}}%
\pgfpathlineto{\pgfqpoint{1.684730in}{0.584605in}}%
\pgfpathlineto{\pgfqpoint{1.685249in}{0.585603in}}%
\pgfpathlineto{\pgfqpoint{1.685769in}{0.587935in}}%
\pgfpathlineto{\pgfqpoint{1.688887in}{0.619231in}}%
\pgfpathlineto{\pgfqpoint{1.689060in}{0.618041in}}%
\pgfpathlineto{\pgfqpoint{1.690099in}{0.610690in}}%
\pgfpathlineto{\pgfqpoint{1.690966in}{0.615229in}}%
\pgfpathlineto{\pgfqpoint{1.692351in}{0.620890in}}%
\pgfpathlineto{\pgfqpoint{1.693044in}{0.618106in}}%
\pgfpathlineto{\pgfqpoint{1.693737in}{0.618686in}}%
\pgfpathlineto{\pgfqpoint{1.694950in}{0.610420in}}%
\pgfpathlineto{\pgfqpoint{1.698068in}{0.590736in}}%
\pgfpathlineto{\pgfqpoint{1.698414in}{0.592348in}}%
\pgfpathlineto{\pgfqpoint{1.700146in}{0.608158in}}%
\pgfpathlineto{\pgfqpoint{1.701186in}{0.603387in}}%
\pgfpathlineto{\pgfqpoint{1.703264in}{0.593674in}}%
\pgfpathlineto{\pgfqpoint{1.703957in}{0.596528in}}%
\pgfpathlineto{\pgfqpoint{1.704130in}{0.596784in}}%
\pgfpathlineto{\pgfqpoint{1.704477in}{0.594877in}}%
\pgfpathlineto{\pgfqpoint{1.707941in}{0.575384in}}%
\pgfpathlineto{\pgfqpoint{1.708114in}{0.575217in}}%
\pgfpathlineto{\pgfqpoint{1.708461in}{0.575932in}}%
\pgfpathlineto{\pgfqpoint{1.709673in}{0.574793in}}%
\pgfpathlineto{\pgfqpoint{1.710193in}{0.577160in}}%
\pgfpathlineto{\pgfqpoint{1.714350in}{0.589957in}}%
\pgfpathlineto{\pgfqpoint{1.714523in}{0.589882in}}%
\pgfpathlineto{\pgfqpoint{1.715216in}{0.586058in}}%
\pgfpathlineto{\pgfqpoint{1.715909in}{0.578606in}}%
\pgfpathlineto{\pgfqpoint{1.716949in}{0.582840in}}%
\pgfpathlineto{\pgfqpoint{1.718161in}{0.584485in}}%
\pgfpathlineto{\pgfqpoint{1.718508in}{0.582853in}}%
\pgfpathlineto{\pgfqpoint{1.720759in}{0.581078in}}%
\pgfpathlineto{\pgfqpoint{1.720933in}{0.579482in}}%
\pgfpathlineto{\pgfqpoint{1.721799in}{0.584439in}}%
\pgfpathlineto{\pgfqpoint{1.721972in}{0.584863in}}%
\pgfpathlineto{\pgfqpoint{1.722665in}{0.582998in}}%
\pgfpathlineto{\pgfqpoint{1.725263in}{0.574488in}}%
\pgfpathlineto{\pgfqpoint{1.725956in}{0.575121in}}%
\pgfpathlineto{\pgfqpoint{1.727169in}{0.581418in}}%
\pgfpathlineto{\pgfqpoint{1.728554in}{0.592083in}}%
\pgfpathlineto{\pgfqpoint{1.729767in}{0.591664in}}%
\pgfpathlineto{\pgfqpoint{1.730460in}{0.595804in}}%
\pgfpathlineto{\pgfqpoint{1.731153in}{0.591657in}}%
\pgfpathlineto{\pgfqpoint{1.732019in}{0.590172in}}%
\pgfpathlineto{\pgfqpoint{1.731499in}{0.592960in}}%
\pgfpathlineto{\pgfqpoint{1.732192in}{0.591059in}}%
\pgfpathlineto{\pgfqpoint{1.733058in}{0.597268in}}%
\pgfpathlineto{\pgfqpoint{1.733578in}{0.590470in}}%
\pgfpathlineto{\pgfqpoint{1.735137in}{0.580222in}}%
\pgfpathlineto{\pgfqpoint{1.735656in}{0.580839in}}%
\pgfpathlineto{\pgfqpoint{1.736003in}{0.580751in}}%
\pgfpathlineto{\pgfqpoint{1.736522in}{0.581368in}}%
\pgfpathlineto{\pgfqpoint{1.741026in}{0.599334in}}%
\pgfpathlineto{\pgfqpoint{1.741892in}{0.598784in}}%
\pgfpathlineto{\pgfqpoint{1.742239in}{0.598263in}}%
\pgfpathlineto{\pgfqpoint{1.742758in}{0.599788in}}%
\pgfpathlineto{\pgfqpoint{1.743278in}{0.599263in}}%
\pgfpathlineto{\pgfqpoint{1.744664in}{0.603526in}}%
\pgfpathlineto{\pgfqpoint{1.745703in}{0.600698in}}%
\pgfpathlineto{\pgfqpoint{1.746569in}{0.599059in}}%
\pgfpathlineto{\pgfqpoint{1.747089in}{0.600173in}}%
\pgfpathlineto{\pgfqpoint{1.749514in}{0.605106in}}%
\pgfpathlineto{\pgfqpoint{1.751939in}{0.607745in}}%
\pgfpathlineto{\pgfqpoint{1.752112in}{0.607355in}}%
\pgfpathlineto{\pgfqpoint{1.755577in}{0.592800in}}%
\pgfpathlineto{\pgfqpoint{1.755923in}{0.594613in}}%
\pgfpathlineto{\pgfqpoint{1.756270in}{0.596949in}}%
\pgfpathlineto{\pgfqpoint{1.757136in}{0.593773in}}%
\pgfpathlineto{\pgfqpoint{1.757655in}{0.591517in}}%
\pgfpathlineto{\pgfqpoint{1.758175in}{0.595159in}}%
\pgfpathlineto{\pgfqpoint{1.760080in}{0.600297in}}%
\pgfpathlineto{\pgfqpoint{1.760427in}{0.599649in}}%
\pgfpathlineto{\pgfqpoint{1.760600in}{0.599282in}}%
\pgfpathlineto{\pgfqpoint{1.760947in}{0.601601in}}%
\pgfpathlineto{\pgfqpoint{1.763025in}{0.610348in}}%
\pgfpathlineto{\pgfqpoint{1.763198in}{0.608615in}}%
\pgfpathlineto{\pgfqpoint{1.765797in}{0.591607in}}%
\pgfpathlineto{\pgfqpoint{1.765970in}{0.591730in}}%
\pgfpathlineto{\pgfqpoint{1.767529in}{0.581323in}}%
\pgfpathlineto{\pgfqpoint{1.769261in}{0.583759in}}%
\pgfpathlineto{\pgfqpoint{1.769434in}{0.583874in}}%
\pgfpathlineto{\pgfqpoint{1.769608in}{0.582798in}}%
\pgfpathlineto{\pgfqpoint{1.772033in}{0.575243in}}%
\pgfpathlineto{\pgfqpoint{1.772726in}{0.577093in}}%
\pgfpathlineto{\pgfqpoint{1.773765in}{0.576141in}}%
\pgfpathlineto{\pgfqpoint{1.775151in}{0.577158in}}%
\pgfpathlineto{\pgfqpoint{1.775670in}{0.573928in}}%
\pgfpathlineto{\pgfqpoint{1.776190in}{0.579961in}}%
\pgfpathlineto{\pgfqpoint{1.778788in}{0.594415in}}%
\pgfpathlineto{\pgfqpoint{1.778961in}{0.594377in}}%
\pgfpathlineto{\pgfqpoint{1.781040in}{0.581095in}}%
\pgfpathlineto{\pgfqpoint{1.781560in}{0.584747in}}%
\pgfpathlineto{\pgfqpoint{1.782946in}{0.590196in}}%
\pgfpathlineto{\pgfqpoint{1.783638in}{0.590115in}}%
\pgfpathlineto{\pgfqpoint{1.783812in}{0.590667in}}%
\pgfpathlineto{\pgfqpoint{1.784331in}{0.588371in}}%
\pgfpathlineto{\pgfqpoint{1.786063in}{0.580581in}}%
\pgfpathlineto{\pgfqpoint{1.786410in}{0.582038in}}%
\pgfpathlineto{\pgfqpoint{1.786583in}{0.582741in}}%
\pgfpathlineto{\pgfqpoint{1.787103in}{0.578967in}}%
\pgfpathlineto{\pgfqpoint{1.787276in}{0.578199in}}%
\pgfpathlineto{\pgfqpoint{1.788315in}{0.580121in}}%
\pgfpathlineto{\pgfqpoint{1.789181in}{0.582049in}}%
\pgfpathlineto{\pgfqpoint{1.789874in}{0.580284in}}%
\pgfpathlineto{\pgfqpoint{1.790567in}{0.578256in}}%
\pgfpathlineto{\pgfqpoint{1.791260in}{0.580655in}}%
\pgfpathlineto{\pgfqpoint{1.792992in}{0.582711in}}%
\pgfpathlineto{\pgfqpoint{1.795417in}{0.580846in}}%
\pgfpathlineto{\pgfqpoint{1.797496in}{0.569478in}}%
\pgfpathlineto{\pgfqpoint{1.797842in}{0.573524in}}%
\pgfpathlineto{\pgfqpoint{1.799748in}{0.584521in}}%
\pgfpathlineto{\pgfqpoint{1.800094in}{0.583722in}}%
\pgfpathlineto{\pgfqpoint{1.801653in}{0.575345in}}%
\pgfpathlineto{\pgfqpoint{1.802000in}{0.579172in}}%
\pgfpathlineto{\pgfqpoint{1.803212in}{0.586695in}}%
\pgfpathlineto{\pgfqpoint{1.803905in}{0.584783in}}%
\pgfpathlineto{\pgfqpoint{1.804771in}{0.578716in}}%
\pgfpathlineto{\pgfqpoint{1.805984in}{0.581208in}}%
\pgfpathlineto{\pgfqpoint{1.808236in}{0.588882in}}%
\pgfpathlineto{\pgfqpoint{1.808409in}{0.588842in}}%
\pgfpathlineto{\pgfqpoint{1.808755in}{0.589664in}}%
\pgfpathlineto{\pgfqpoint{1.808929in}{0.588234in}}%
\pgfpathlineto{\pgfqpoint{1.809621in}{0.585147in}}%
\pgfpathlineto{\pgfqpoint{1.810314in}{0.587627in}}%
\pgfpathlineto{\pgfqpoint{1.811180in}{0.588693in}}%
\pgfpathlineto{\pgfqpoint{1.811527in}{0.586762in}}%
\pgfpathlineto{\pgfqpoint{1.812047in}{0.583430in}}%
\pgfpathlineto{\pgfqpoint{1.812913in}{0.587940in}}%
\pgfpathlineto{\pgfqpoint{1.813432in}{0.589891in}}%
\pgfpathlineto{\pgfqpoint{1.814472in}{0.587916in}}%
\pgfpathlineto{\pgfqpoint{1.814818in}{0.587547in}}%
\pgfpathlineto{\pgfqpoint{1.815338in}{0.588934in}}%
\pgfpathlineto{\pgfqpoint{1.815511in}{0.588854in}}%
\pgfpathlineto{\pgfqpoint{1.817936in}{0.593038in}}%
\pgfpathlineto{\pgfqpoint{1.818109in}{0.592497in}}%
\pgfpathlineto{\pgfqpoint{1.818802in}{0.594481in}}%
\pgfpathlineto{\pgfqpoint{1.821920in}{0.603167in}}%
\pgfpathlineto{\pgfqpoint{1.822093in}{0.603009in}}%
\pgfpathlineto{\pgfqpoint{1.822267in}{0.603832in}}%
\pgfpathlineto{\pgfqpoint{1.824692in}{0.613033in}}%
\pgfpathlineto{\pgfqpoint{1.824865in}{0.612970in}}%
\pgfpathlineto{\pgfqpoint{1.825558in}{0.611641in}}%
\pgfpathlineto{\pgfqpoint{1.826077in}{0.609726in}}%
\pgfpathlineto{\pgfqpoint{1.826943in}{0.611418in}}%
\pgfpathlineto{\pgfqpoint{1.829195in}{0.618806in}}%
\pgfpathlineto{\pgfqpoint{1.830061in}{0.615839in}}%
\pgfpathlineto{\pgfqpoint{1.839242in}{0.586973in}}%
\pgfpathlineto{\pgfqpoint{1.839762in}{0.587828in}}%
\pgfpathlineto{\pgfqpoint{1.840455in}{0.593683in}}%
\pgfpathlineto{\pgfqpoint{1.841494in}{0.589383in}}%
\pgfpathlineto{\pgfqpoint{1.842533in}{0.589407in}}%
\pgfpathlineto{\pgfqpoint{1.842707in}{0.590292in}}%
\pgfpathlineto{\pgfqpoint{1.843053in}{0.592653in}}%
\pgfpathlineto{\pgfqpoint{1.843399in}{0.588403in}}%
\pgfpathlineto{\pgfqpoint{1.845998in}{0.566607in}}%
\pgfpathlineto{\pgfqpoint{1.846171in}{0.566750in}}%
\pgfpathlineto{\pgfqpoint{1.847557in}{0.574485in}}%
\pgfpathlineto{\pgfqpoint{1.848250in}{0.578118in}}%
\pgfpathlineto{\pgfqpoint{1.848942in}{0.575111in}}%
\pgfpathlineto{\pgfqpoint{1.849116in}{0.574442in}}%
\pgfpathlineto{\pgfqpoint{1.849462in}{0.578512in}}%
\pgfpathlineto{\pgfqpoint{1.852060in}{0.590376in}}%
\pgfpathlineto{\pgfqpoint{1.852580in}{0.588469in}}%
\pgfpathlineto{\pgfqpoint{1.857084in}{0.552347in}}%
\pgfpathlineto{\pgfqpoint{1.857777in}{0.556620in}}%
\pgfpathlineto{\pgfqpoint{1.858470in}{0.559633in}}%
\pgfpathlineto{\pgfqpoint{1.859682in}{0.559000in}}%
\pgfpathlineto{\pgfqpoint{1.861761in}{0.560744in}}%
\pgfpathlineto{\pgfqpoint{1.861934in}{0.558992in}}%
\pgfpathlineto{\pgfqpoint{1.862280in}{0.556700in}}%
\pgfpathlineto{\pgfqpoint{1.862973in}{0.562325in}}%
\pgfpathlineto{\pgfqpoint{1.863839in}{0.561129in}}%
\pgfpathlineto{\pgfqpoint{1.864879in}{0.567559in}}%
\pgfpathlineto{\pgfqpoint{1.865918in}{0.573600in}}%
\pgfpathlineto{\pgfqpoint{1.866611in}{0.570061in}}%
\pgfpathlineto{\pgfqpoint{1.866784in}{0.569228in}}%
\pgfpathlineto{\pgfqpoint{1.867304in}{0.572028in}}%
\pgfpathlineto{\pgfqpoint{1.869382in}{0.580543in}}%
\pgfpathlineto{\pgfqpoint{1.871115in}{0.584207in}}%
\pgfpathlineto{\pgfqpoint{1.871634in}{0.586154in}}%
\pgfpathlineto{\pgfqpoint{1.872154in}{0.582717in}}%
\pgfpathlineto{\pgfqpoint{1.872327in}{0.582921in}}%
\pgfpathlineto{\pgfqpoint{1.874752in}{0.574863in}}%
\pgfpathlineto{\pgfqpoint{1.875099in}{0.573829in}}%
\pgfpathlineto{\pgfqpoint{1.876138in}{0.570872in}}%
\pgfpathlineto{\pgfqpoint{1.876831in}{0.572434in}}%
\pgfpathlineto{\pgfqpoint{1.877351in}{0.576618in}}%
\pgfpathlineto{\pgfqpoint{1.878217in}{0.572068in}}%
\pgfpathlineto{\pgfqpoint{1.878390in}{0.572499in}}%
\pgfpathlineto{\pgfqpoint{1.879083in}{0.573436in}}%
\pgfpathlineto{\pgfqpoint{1.881681in}{0.584684in}}%
\pgfpathlineto{\pgfqpoint{1.882201in}{0.583599in}}%
\pgfpathlineto{\pgfqpoint{1.882547in}{0.581971in}}%
\pgfpathlineto{\pgfqpoint{1.885665in}{0.564379in}}%
\pgfpathlineto{\pgfqpoint{1.886185in}{0.566343in}}%
\pgfpathlineto{\pgfqpoint{1.886878in}{0.562765in}}%
\pgfpathlineto{\pgfqpoint{1.888090in}{0.557694in}}%
\pgfpathlineto{\pgfqpoint{1.888610in}{0.560260in}}%
\pgfpathlineto{\pgfqpoint{1.889476in}{0.567667in}}%
\pgfpathlineto{\pgfqpoint{1.890689in}{0.567103in}}%
\pgfpathlineto{\pgfqpoint{1.892940in}{0.560840in}}%
\pgfpathlineto{\pgfqpoint{1.893287in}{0.562733in}}%
\pgfpathlineto{\pgfqpoint{1.895019in}{0.565510in}}%
\pgfpathlineto{\pgfqpoint{1.894499in}{0.561788in}}%
\pgfpathlineto{\pgfqpoint{1.895192in}{0.565400in}}%
\pgfpathlineto{\pgfqpoint{1.895885in}{0.570961in}}%
\pgfpathlineto{\pgfqpoint{1.897964in}{0.578798in}}%
\pgfpathlineto{\pgfqpoint{1.898137in}{0.578589in}}%
\pgfpathlineto{\pgfqpoint{1.901082in}{0.566278in}}%
\pgfpathlineto{\pgfqpoint{1.902121in}{0.568646in}}%
\pgfpathlineto{\pgfqpoint{1.902294in}{0.569706in}}%
\pgfpathlineto{\pgfqpoint{1.902987in}{0.565171in}}%
\pgfpathlineto{\pgfqpoint{1.905759in}{0.558842in}}%
\pgfpathlineto{\pgfqpoint{1.905932in}{0.559577in}}%
\pgfpathlineto{\pgfqpoint{1.906625in}{0.558960in}}%
\pgfpathlineto{\pgfqpoint{1.908357in}{0.571396in}}%
\pgfpathlineto{\pgfqpoint{1.911302in}{0.586531in}}%
\pgfpathlineto{\pgfqpoint{1.912168in}{0.589724in}}%
\pgfpathlineto{\pgfqpoint{1.912688in}{0.593177in}}%
\pgfpathlineto{\pgfqpoint{1.913554in}{0.588636in}}%
\pgfpathlineto{\pgfqpoint{1.915286in}{0.586261in}}%
\pgfpathlineto{\pgfqpoint{1.916152in}{0.592066in}}%
\pgfpathlineto{\pgfqpoint{1.918404in}{0.589892in}}%
\pgfpathlineto{\pgfqpoint{1.918923in}{0.591041in}}%
\pgfpathlineto{\pgfqpoint{1.919790in}{0.595174in}}%
\pgfpathlineto{\pgfqpoint{1.920482in}{0.592631in}}%
\pgfpathlineto{\pgfqpoint{1.922041in}{0.588969in}}%
\pgfpathlineto{\pgfqpoint{1.924813in}{0.612513in}}%
\pgfpathlineto{\pgfqpoint{1.926025in}{0.607076in}}%
\pgfpathlineto{\pgfqpoint{1.935033in}{0.570782in}}%
\pgfpathlineto{\pgfqpoint{1.935553in}{0.572759in}}%
\pgfpathlineto{\pgfqpoint{1.938844in}{0.584660in}}%
\pgfpathlineto{\pgfqpoint{1.943867in}{0.559307in}}%
\pgfpathlineto{\pgfqpoint{1.945253in}{0.559956in}}%
\pgfpathlineto{\pgfqpoint{1.945946in}{0.562538in}}%
\pgfpathlineto{\pgfqpoint{1.946639in}{0.560029in}}%
\pgfpathlineto{\pgfqpoint{1.947851in}{0.550835in}}%
\pgfpathlineto{\pgfqpoint{1.948371in}{0.544397in}}%
\pgfpathlineto{\pgfqpoint{1.949583in}{0.546202in}}%
\pgfpathlineto{\pgfqpoint{1.954087in}{0.571911in}}%
\pgfpathlineto{\pgfqpoint{1.954953in}{0.569301in}}%
\pgfpathlineto{\pgfqpoint{1.955473in}{0.571755in}}%
\pgfpathlineto{\pgfqpoint{1.958244in}{0.588198in}}%
\pgfpathlineto{\pgfqpoint{1.961536in}{0.582383in}}%
\pgfpathlineto{\pgfqpoint{1.961709in}{0.583174in}}%
\pgfpathlineto{\pgfqpoint{1.965693in}{0.605927in}}%
\pgfpathlineto{\pgfqpoint{1.966213in}{0.605117in}}%
\pgfpathlineto{\pgfqpoint{1.966732in}{0.606820in}}%
\pgfpathlineto{\pgfqpoint{1.967425in}{0.609376in}}%
\pgfpathlineto{\pgfqpoint{1.968291in}{0.616344in}}%
\pgfpathlineto{\pgfqpoint{1.968984in}{0.610351in}}%
\pgfpathlineto{\pgfqpoint{1.970890in}{0.604457in}}%
\pgfpathlineto{\pgfqpoint{1.971236in}{0.605068in}}%
\pgfpathlineto{\pgfqpoint{1.971756in}{0.603452in}}%
\pgfpathlineto{\pgfqpoint{1.972968in}{0.600568in}}%
\pgfpathlineto{\pgfqpoint{1.973488in}{0.601600in}}%
\pgfpathlineto{\pgfqpoint{1.974008in}{0.604083in}}%
\pgfpathlineto{\pgfqpoint{1.974874in}{0.601460in}}%
\pgfpathlineto{\pgfqpoint{1.979031in}{0.574899in}}%
\pgfpathlineto{\pgfqpoint{1.979897in}{0.566211in}}%
\pgfpathlineto{\pgfqpoint{1.981283in}{0.568192in}}%
\pgfpathlineto{\pgfqpoint{1.982495in}{0.569855in}}%
\pgfpathlineto{\pgfqpoint{1.985786in}{0.580581in}}%
\pgfpathlineto{\pgfqpoint{1.986306in}{0.578624in}}%
\pgfpathlineto{\pgfqpoint{1.987692in}{0.575301in}}%
\pgfpathlineto{\pgfqpoint{1.988385in}{0.576626in}}%
\pgfpathlineto{\pgfqpoint{1.989771in}{0.580441in}}%
\pgfpathlineto{\pgfqpoint{1.990290in}{0.578172in}}%
\pgfpathlineto{\pgfqpoint{1.991156in}{0.576797in}}%
\pgfpathlineto{\pgfqpoint{1.991503in}{0.578457in}}%
\pgfpathlineto{\pgfqpoint{1.992022in}{0.577344in}}%
\pgfpathlineto{\pgfqpoint{1.992889in}{0.579190in}}%
\pgfpathlineto{\pgfqpoint{1.994274in}{0.586159in}}%
\pgfpathlineto{\pgfqpoint{1.994794in}{0.584250in}}%
\pgfpathlineto{\pgfqpoint{1.996353in}{0.581013in}}%
\pgfpathlineto{\pgfqpoint{1.996526in}{0.581060in}}%
\pgfpathlineto{\pgfqpoint{1.997565in}{0.577016in}}%
\pgfpathlineto{\pgfqpoint{1.998258in}{0.579610in}}%
\pgfpathlineto{\pgfqpoint{2.000164in}{0.586916in}}%
\pgfpathlineto{\pgfqpoint{2.000683in}{0.585036in}}%
\pgfpathlineto{\pgfqpoint{2.005880in}{0.577041in}}%
\pgfpathlineto{\pgfqpoint{2.006226in}{0.577679in}}%
\pgfpathlineto{\pgfqpoint{2.007093in}{0.577232in}}%
\pgfpathlineto{\pgfqpoint{2.007266in}{0.576471in}}%
\pgfpathlineto{\pgfqpoint{2.008305in}{0.574260in}}%
\pgfpathlineto{\pgfqpoint{2.009171in}{0.575744in}}%
\pgfpathlineto{\pgfqpoint{2.010557in}{0.577390in}}%
\pgfpathlineto{\pgfqpoint{2.009864in}{0.574580in}}%
\pgfpathlineto{\pgfqpoint{2.010730in}{0.576990in}}%
\pgfpathlineto{\pgfqpoint{2.011077in}{0.575235in}}%
\pgfpathlineto{\pgfqpoint{2.011770in}{0.579062in}}%
\pgfpathlineto{\pgfqpoint{2.011943in}{0.579898in}}%
\pgfpathlineto{\pgfqpoint{2.012982in}{0.578408in}}%
\pgfpathlineto{\pgfqpoint{2.013329in}{0.577045in}}%
\pgfpathlineto{\pgfqpoint{2.014195in}{0.579378in}}%
\pgfpathlineto{\pgfqpoint{2.014714in}{0.580734in}}%
\pgfpathlineto{\pgfqpoint{2.014887in}{0.579289in}}%
\pgfpathlineto{\pgfqpoint{2.016446in}{0.569298in}}%
\pgfpathlineto{\pgfqpoint{2.016966in}{0.569776in}}%
\pgfpathlineto{\pgfqpoint{2.021297in}{0.538471in}}%
\pgfpathlineto{\pgfqpoint{2.022163in}{0.548284in}}%
\pgfpathlineto{\pgfqpoint{2.022682in}{0.550922in}}%
\pgfpathlineto{\pgfqpoint{2.023375in}{0.546546in}}%
\pgfpathlineto{\pgfqpoint{2.025281in}{0.541632in}}%
\pgfpathlineto{\pgfqpoint{2.026320in}{0.547144in}}%
\pgfpathlineto{\pgfqpoint{2.027013in}{0.543997in}}%
\pgfpathlineto{\pgfqpoint{2.027706in}{0.542114in}}%
\pgfpathlineto{\pgfqpoint{2.028225in}{0.544518in}}%
\pgfpathlineto{\pgfqpoint{2.029438in}{0.557537in}}%
\pgfpathlineto{\pgfqpoint{2.030131in}{0.566113in}}%
\pgfpathlineto{\pgfqpoint{2.030997in}{0.559175in}}%
\pgfpathlineto{\pgfqpoint{2.033942in}{0.541438in}}%
\pgfpathlineto{\pgfqpoint{2.035327in}{0.548210in}}%
\pgfpathlineto{\pgfqpoint{2.035847in}{0.545380in}}%
\pgfpathlineto{\pgfqpoint{2.037579in}{0.542234in}}%
\pgfpathlineto{\pgfqpoint{2.037753in}{0.542277in}}%
\pgfpathlineto{\pgfqpoint{2.038619in}{0.548879in}}%
\pgfpathlineto{\pgfqpoint{2.039658in}{0.545223in}}%
\pgfpathlineto{\pgfqpoint{2.039831in}{0.544863in}}%
\pgfpathlineto{\pgfqpoint{2.040004in}{0.546408in}}%
\pgfpathlineto{\pgfqpoint{2.040871in}{0.551801in}}%
\pgfpathlineto{\pgfqpoint{2.041737in}{0.549616in}}%
\pgfpathlineto{\pgfqpoint{2.043122in}{0.544410in}}%
\pgfpathlineto{\pgfqpoint{2.043815in}{0.547132in}}%
\pgfpathlineto{\pgfqpoint{2.045894in}{0.551876in}}%
\pgfpathlineto{\pgfqpoint{2.047626in}{0.561659in}}%
\pgfpathlineto{\pgfqpoint{2.048839in}{0.559555in}}%
\pgfpathlineto{\pgfqpoint{2.050224in}{0.553396in}}%
\pgfpathlineto{\pgfqpoint{2.050571in}{0.555924in}}%
\pgfpathlineto{\pgfqpoint{2.052130in}{0.563090in}}%
\pgfpathlineto{\pgfqpoint{2.052650in}{0.562664in}}%
\pgfpathlineto{\pgfqpoint{2.052996in}{0.560804in}}%
\pgfpathlineto{\pgfqpoint{2.053689in}{0.564863in}}%
\pgfpathlineto{\pgfqpoint{2.055075in}{0.566157in}}%
\pgfpathlineto{\pgfqpoint{2.058539in}{0.575687in}}%
\pgfpathlineto{\pgfqpoint{2.059059in}{0.572435in}}%
\pgfpathlineto{\pgfqpoint{2.060964in}{0.570889in}}%
\pgfpathlineto{\pgfqpoint{2.062696in}{0.566442in}}%
\pgfpathlineto{\pgfqpoint{2.063043in}{0.568739in}}%
\pgfpathlineto{\pgfqpoint{2.063562in}{0.573336in}}%
\pgfpathlineto{\pgfqpoint{2.064429in}{0.567572in}}%
\pgfpathlineto{\pgfqpoint{2.066680in}{0.548569in}}%
\pgfpathlineto{\pgfqpoint{2.067546in}{0.552488in}}%
\pgfpathlineto{\pgfqpoint{2.067893in}{0.553205in}}%
\pgfpathlineto{\pgfqpoint{2.068239in}{0.552240in}}%
\pgfpathlineto{\pgfqpoint{2.070145in}{0.556387in}}%
\pgfpathlineto{\pgfqpoint{2.070491in}{0.558808in}}%
\pgfpathlineto{\pgfqpoint{2.071531in}{0.556768in}}%
\pgfpathlineto{\pgfqpoint{2.073263in}{0.544695in}}%
\pgfpathlineto{\pgfqpoint{2.074649in}{0.540876in}}%
\pgfpathlineto{\pgfqpoint{2.075168in}{0.541772in}}%
\pgfpathlineto{\pgfqpoint{2.075688in}{0.546240in}}%
\pgfpathlineto{\pgfqpoint{2.078459in}{0.569391in}}%
\pgfpathlineto{\pgfqpoint{2.078806in}{0.566961in}}%
\pgfpathlineto{\pgfqpoint{2.080538in}{0.558248in}}%
\pgfpathlineto{\pgfqpoint{2.081058in}{0.560660in}}%
\pgfpathlineto{\pgfqpoint{2.083310in}{0.566373in}}%
\pgfpathlineto{\pgfqpoint{2.084002in}{0.565430in}}%
\pgfpathlineto{\pgfqpoint{2.084349in}{0.565948in}}%
\pgfpathlineto{\pgfqpoint{2.087986in}{0.555680in}}%
\pgfpathlineto{\pgfqpoint{2.088333in}{0.557523in}}%
\pgfpathlineto{\pgfqpoint{2.088506in}{0.557890in}}%
\pgfpathlineto{\pgfqpoint{2.089199in}{0.556275in}}%
\pgfpathlineto{\pgfqpoint{2.089892in}{0.558177in}}%
\pgfpathlineto{\pgfqpoint{2.090758in}{0.555184in}}%
\pgfpathlineto{\pgfqpoint{2.094742in}{0.570521in}}%
\pgfpathlineto{\pgfqpoint{2.096128in}{0.563916in}}%
\pgfpathlineto{\pgfqpoint{2.097687in}{0.560692in}}%
\pgfpathlineto{\pgfqpoint{2.097860in}{0.562275in}}%
\pgfpathlineto{\pgfqpoint{2.099419in}{0.570139in}}%
\pgfpathlineto{\pgfqpoint{2.099939in}{0.566488in}}%
\pgfpathlineto{\pgfqpoint{2.101671in}{0.573689in}}%
\pgfpathlineto{\pgfqpoint{2.103403in}{0.580989in}}%
\pgfpathlineto{\pgfqpoint{2.103750in}{0.579685in}}%
\pgfpathlineto{\pgfqpoint{2.105828in}{0.573148in}}%
\pgfpathlineto{\pgfqpoint{2.106001in}{0.571929in}}%
\pgfpathlineto{\pgfqpoint{2.107387in}{0.572696in}}%
\pgfpathlineto{\pgfqpoint{2.108946in}{0.582485in}}%
\pgfpathlineto{\pgfqpoint{2.109985in}{0.577072in}}%
\pgfpathlineto{\pgfqpoint{2.112064in}{0.563017in}}%
\pgfpathlineto{\pgfqpoint{2.112411in}{0.565070in}}%
\pgfpathlineto{\pgfqpoint{2.112584in}{0.565829in}}%
\pgfpathlineto{\pgfqpoint{2.113103in}{0.563075in}}%
\pgfpathlineto{\pgfqpoint{2.113450in}{0.559647in}}%
\pgfpathlineto{\pgfqpoint{2.114489in}{0.562567in}}%
\pgfpathlineto{\pgfqpoint{2.115702in}{0.570401in}}%
\pgfpathlineto{\pgfqpoint{2.116221in}{0.568729in}}%
\pgfpathlineto{\pgfqpoint{2.117261in}{0.565179in}}%
\pgfpathlineto{\pgfqpoint{2.117780in}{0.568343in}}%
\pgfpathlineto{\pgfqpoint{2.125402in}{0.619460in}}%
\pgfpathlineto{\pgfqpoint{2.125922in}{0.619290in}}%
\pgfpathlineto{\pgfqpoint{2.127134in}{0.620412in}}%
\pgfpathlineto{\pgfqpoint{2.127481in}{0.619294in}}%
\pgfpathlineto{\pgfqpoint{2.130945in}{0.607331in}}%
\pgfpathlineto{\pgfqpoint{2.131118in}{0.607813in}}%
\pgfpathlineto{\pgfqpoint{2.132851in}{0.612290in}}%
\pgfpathlineto{\pgfqpoint{2.133543in}{0.610475in}}%
\pgfpathlineto{\pgfqpoint{2.134756in}{0.609492in}}%
\pgfpathlineto{\pgfqpoint{2.134929in}{0.608669in}}%
\pgfpathlineto{\pgfqpoint{2.135449in}{0.613099in}}%
\pgfpathlineto{\pgfqpoint{2.136142in}{0.614384in}}%
\pgfpathlineto{\pgfqpoint{2.138047in}{0.624201in}}%
\pgfpathlineto{\pgfqpoint{2.138567in}{0.622861in}}%
\pgfpathlineto{\pgfqpoint{2.141338in}{0.605181in}}%
\pgfpathlineto{\pgfqpoint{2.145149in}{0.610395in}}%
\pgfpathlineto{\pgfqpoint{2.147401in}{0.599275in}}%
\pgfpathlineto{\pgfqpoint{2.147921in}{0.601943in}}%
\pgfpathlineto{\pgfqpoint{2.148440in}{0.606981in}}%
\pgfpathlineto{\pgfqpoint{2.149480in}{0.602819in}}%
\pgfpathlineto{\pgfqpoint{2.151558in}{0.592835in}}%
\pgfpathlineto{\pgfqpoint{2.153464in}{0.578725in}}%
\pgfpathlineto{\pgfqpoint{2.154330in}{0.581313in}}%
\pgfpathlineto{\pgfqpoint{2.156408in}{0.589121in}}%
\pgfpathlineto{\pgfqpoint{2.157101in}{0.588272in}}%
\pgfpathlineto{\pgfqpoint{2.157967in}{0.585845in}}%
\pgfpathlineto{\pgfqpoint{2.158660in}{0.587653in}}%
\pgfpathlineto{\pgfqpoint{2.159353in}{0.589055in}}%
\pgfpathlineto{\pgfqpoint{2.161952in}{0.606097in}}%
\pgfpathlineto{\pgfqpoint{2.162818in}{0.603899in}}%
\pgfpathlineto{\pgfqpoint{2.163684in}{0.602412in}}%
\pgfpathlineto{\pgfqpoint{2.163857in}{0.604307in}}%
\pgfpathlineto{\pgfqpoint{2.165070in}{0.616080in}}%
\pgfpathlineto{\pgfqpoint{2.165762in}{0.611106in}}%
\pgfpathlineto{\pgfqpoint{2.169920in}{0.595878in}}%
\pgfpathlineto{\pgfqpoint{2.171305in}{0.593015in}}%
\pgfpathlineto{\pgfqpoint{2.171998in}{0.595556in}}%
\pgfpathlineto{\pgfqpoint{2.173211in}{0.599898in}}%
\pgfpathlineto{\pgfqpoint{2.173904in}{0.598783in}}%
\pgfpathlineto{\pgfqpoint{2.174770in}{0.601684in}}%
\pgfpathlineto{\pgfqpoint{2.174250in}{0.598610in}}%
\pgfpathlineto{\pgfqpoint{2.175809in}{0.599445in}}%
\pgfpathlineto{\pgfqpoint{2.176156in}{0.597738in}}%
\pgfpathlineto{\pgfqpoint{2.177022in}{0.601200in}}%
\pgfpathlineto{\pgfqpoint{2.177195in}{0.601459in}}%
\pgfpathlineto{\pgfqpoint{2.177888in}{0.600052in}}%
\pgfpathlineto{\pgfqpoint{2.179447in}{0.599003in}}%
\pgfpathlineto{\pgfqpoint{2.179620in}{0.599176in}}%
\pgfpathlineto{\pgfqpoint{2.180313in}{0.598007in}}%
\pgfpathlineto{\pgfqpoint{2.180659in}{0.597323in}}%
\pgfpathlineto{\pgfqpoint{2.181525in}{0.598714in}}%
\pgfpathlineto{\pgfqpoint{2.182218in}{0.599305in}}%
\pgfpathlineto{\pgfqpoint{2.183084in}{0.598714in}}%
\pgfpathlineto{\pgfqpoint{2.184124in}{0.596906in}}%
\pgfpathlineto{\pgfqpoint{2.186549in}{0.591680in}}%
\pgfpathlineto{\pgfqpoint{2.187415in}{0.591008in}}%
\pgfpathlineto{\pgfqpoint{2.187761in}{0.592453in}}%
\pgfpathlineto{\pgfqpoint{2.189320in}{0.595695in}}%
\pgfpathlineto{\pgfqpoint{2.189840in}{0.594822in}}%
\pgfpathlineto{\pgfqpoint{2.191226in}{0.591341in}}%
\pgfpathlineto{\pgfqpoint{2.191745in}{0.592642in}}%
\pgfpathlineto{\pgfqpoint{2.196249in}{0.600278in}}%
\pgfpathlineto{\pgfqpoint{2.196942in}{0.598202in}}%
\pgfpathlineto{\pgfqpoint{2.200406in}{0.594570in}}%
\pgfpathlineto{\pgfqpoint{2.197462in}{0.599084in}}%
\pgfpathlineto{\pgfqpoint{2.200753in}{0.596428in}}%
\pgfpathlineto{\pgfqpoint{2.201965in}{0.601249in}}%
\pgfpathlineto{\pgfqpoint{2.202658in}{0.600766in}}%
\pgfpathlineto{\pgfqpoint{2.205083in}{0.604478in}}%
\pgfpathlineto{\pgfqpoint{2.205257in}{0.604211in}}%
\pgfpathlineto{\pgfqpoint{2.206469in}{0.599741in}}%
\pgfpathlineto{\pgfqpoint{2.207508in}{0.602196in}}%
\pgfpathlineto{\pgfqpoint{2.207682in}{0.602554in}}%
\pgfpathlineto{\pgfqpoint{2.208201in}{0.600939in}}%
\pgfpathlineto{\pgfqpoint{2.209414in}{0.597427in}}%
\pgfpathlineto{\pgfqpoint{2.209934in}{0.597809in}}%
\pgfpathlineto{\pgfqpoint{2.210626in}{0.598035in}}%
\pgfpathlineto{\pgfqpoint{2.210280in}{0.597545in}}%
\pgfpathlineto{\pgfqpoint{2.211146in}{0.597066in}}%
\pgfpathlineto{\pgfqpoint{2.211319in}{0.596528in}}%
\pgfpathlineto{\pgfqpoint{2.211839in}{0.598544in}}%
\pgfpathlineto{\pgfqpoint{2.212532in}{0.603474in}}%
\pgfpathlineto{\pgfqpoint{2.213398in}{0.598655in}}%
\pgfpathlineto{\pgfqpoint{2.214957in}{0.590584in}}%
\pgfpathlineto{\pgfqpoint{2.215650in}{0.592347in}}%
\pgfpathlineto{\pgfqpoint{2.217728in}{0.591111in}}%
\pgfpathlineto{\pgfqpoint{2.219461in}{0.590230in}}%
\pgfpathlineto{\pgfqpoint{2.223964in}{0.608484in}}%
\pgfpathlineto{\pgfqpoint{2.220154in}{0.589994in}}%
\pgfpathlineto{\pgfqpoint{2.225177in}{0.603758in}}%
\pgfpathlineto{\pgfqpoint{2.228468in}{0.586952in}}%
\pgfpathlineto{\pgfqpoint{2.228815in}{0.587376in}}%
\pgfpathlineto{\pgfqpoint{2.229681in}{0.591870in}}%
\pgfpathlineto{\pgfqpoint{2.230720in}{0.590731in}}%
\pgfpathlineto{\pgfqpoint{2.231066in}{0.592039in}}%
\pgfpathlineto{\pgfqpoint{2.233145in}{0.602185in}}%
\pgfpathlineto{\pgfqpoint{2.234011in}{0.598969in}}%
\pgfpathlineto{\pgfqpoint{2.235917in}{0.592502in}}%
\pgfpathlineto{\pgfqpoint{2.241460in}{0.575784in}}%
\pgfpathlineto{\pgfqpoint{2.241633in}{0.576162in}}%
\pgfpathlineto{\pgfqpoint{2.242672in}{0.577572in}}%
\pgfpathlineto{\pgfqpoint{2.243192in}{0.576087in}}%
\pgfpathlineto{\pgfqpoint{2.245617in}{0.576764in}}%
\pgfpathlineto{\pgfqpoint{2.245790in}{0.576337in}}%
\pgfpathlineto{\pgfqpoint{2.247349in}{0.570215in}}%
\pgfpathlineto{\pgfqpoint{2.248215in}{0.570775in}}%
\pgfpathlineto{\pgfqpoint{2.248388in}{0.572073in}}%
\pgfpathlineto{\pgfqpoint{2.249255in}{0.568049in}}%
\pgfpathlineto{\pgfqpoint{2.249947in}{0.566284in}}%
\pgfpathlineto{\pgfqpoint{2.250121in}{0.567190in}}%
\pgfpathlineto{\pgfqpoint{2.251160in}{0.575244in}}%
\pgfpathlineto{\pgfqpoint{2.252026in}{0.573544in}}%
\pgfpathlineto{\pgfqpoint{2.252373in}{0.572061in}}%
\pgfpathlineto{\pgfqpoint{2.253239in}{0.568718in}}%
\pgfpathlineto{\pgfqpoint{2.254624in}{0.569995in}}%
\pgfpathlineto{\pgfqpoint{2.255317in}{0.570528in}}%
\pgfpathlineto{\pgfqpoint{2.255664in}{0.568792in}}%
\pgfpathlineto{\pgfqpoint{2.257223in}{0.562221in}}%
\pgfpathlineto{\pgfqpoint{2.257742in}{0.565080in}}%
\pgfpathlineto{\pgfqpoint{2.260167in}{0.572988in}}%
\pgfpathlineto{\pgfqpoint{2.261380in}{0.574202in}}%
\pgfpathlineto{\pgfqpoint{2.261553in}{0.573212in}}%
\pgfpathlineto{\pgfqpoint{2.263285in}{0.567370in}}%
\pgfpathlineto{\pgfqpoint{2.263805in}{0.567751in}}%
\pgfpathlineto{\pgfqpoint{2.263978in}{0.568061in}}%
\pgfpathlineto{\pgfqpoint{2.264671in}{0.566161in}}%
\pgfpathlineto{\pgfqpoint{2.266577in}{0.556897in}}%
\pgfpathlineto{\pgfqpoint{2.266923in}{0.558730in}}%
\pgfpathlineto{\pgfqpoint{2.268655in}{0.563638in}}%
\pgfpathlineto{\pgfqpoint{2.268828in}{0.563354in}}%
\pgfpathlineto{\pgfqpoint{2.269348in}{0.561042in}}%
\pgfpathlineto{\pgfqpoint{2.270214in}{0.562442in}}%
\pgfpathlineto{\pgfqpoint{2.274025in}{0.580068in}}%
\pgfpathlineto{\pgfqpoint{2.274198in}{0.579556in}}%
\pgfpathlineto{\pgfqpoint{2.275064in}{0.575731in}}%
\pgfpathlineto{\pgfqpoint{2.275584in}{0.579874in}}%
\pgfpathlineto{\pgfqpoint{2.275931in}{0.581678in}}%
\pgfpathlineto{\pgfqpoint{2.276797in}{0.578891in}}%
\pgfpathlineto{\pgfqpoint{2.276970in}{0.579185in}}%
\pgfpathlineto{\pgfqpoint{2.279568in}{0.591207in}}%
\pgfpathlineto{\pgfqpoint{2.280088in}{0.593255in}}%
\pgfpathlineto{\pgfqpoint{2.281127in}{0.592474in}}%
\pgfpathlineto{\pgfqpoint{2.281820in}{0.593942in}}%
\pgfpathlineto{\pgfqpoint{2.283033in}{0.597616in}}%
\pgfpathlineto{\pgfqpoint{2.283379in}{0.595770in}}%
\pgfpathlineto{\pgfqpoint{2.285284in}{0.592520in}}%
\pgfpathlineto{\pgfqpoint{2.286151in}{0.593016in}}%
\pgfpathlineto{\pgfqpoint{2.287017in}{0.590779in}}%
\pgfpathlineto{\pgfqpoint{2.288749in}{0.595203in}}%
\pgfpathlineto{\pgfqpoint{2.289095in}{0.592789in}}%
\pgfpathlineto{\pgfqpoint{2.292386in}{0.587426in}}%
\pgfpathlineto{\pgfqpoint{2.289615in}{0.593400in}}%
\pgfpathlineto{\pgfqpoint{2.292560in}{0.587626in}}%
\pgfpathlineto{\pgfqpoint{2.294638in}{0.584710in}}%
\pgfpathlineto{\pgfqpoint{2.295678in}{0.578771in}}%
\pgfpathlineto{\pgfqpoint{2.296544in}{0.581888in}}%
\pgfpathlineto{\pgfqpoint{2.297063in}{0.583975in}}%
\pgfpathlineto{\pgfqpoint{2.297583in}{0.587420in}}%
\pgfpathlineto{\pgfqpoint{2.298103in}{0.583536in}}%
\pgfpathlineto{\pgfqpoint{2.300701in}{0.572472in}}%
\pgfpathlineto{\pgfqpoint{2.300874in}{0.572994in}}%
\pgfpathlineto{\pgfqpoint{2.302780in}{0.699129in}}%
\pgfpathlineto{\pgfqpoint{2.305551in}{0.828287in}}%
\pgfpathlineto{\pgfqpoint{2.311614in}{1.015795in}}%
\pgfpathlineto{\pgfqpoint{2.316811in}{1.077730in}}%
\pgfpathlineto{\pgfqpoint{2.319755in}{1.094116in}}%
\pgfpathlineto{\pgfqpoint{2.320275in}{1.093802in}}%
\pgfpathlineto{\pgfqpoint{2.324952in}{1.090049in}}%
\pgfpathlineto{\pgfqpoint{2.328936in}{1.079804in}}%
\pgfpathlineto{\pgfqpoint{2.347817in}{1.032004in}}%
\pgfpathlineto{\pgfqpoint{2.354573in}{0.993538in}}%
\pgfpathlineto{\pgfqpoint{2.368777in}{0.945633in}}%
\pgfpathlineto{\pgfqpoint{2.370162in}{0.942018in}}%
\pgfpathlineto{\pgfqpoint{2.371548in}{0.938232in}}%
\pgfpathlineto{\pgfqpoint{2.371895in}{0.938996in}}%
\pgfpathlineto{\pgfqpoint{2.373280in}{0.940965in}}%
\pgfpathlineto{\pgfqpoint{2.373973in}{0.940476in}}%
\pgfpathlineto{\pgfqpoint{2.379689in}{0.927124in}}%
\pgfpathlineto{\pgfqpoint{2.382115in}{0.919826in}}%
\pgfpathlineto{\pgfqpoint{2.382461in}{0.919988in}}%
\pgfpathlineto{\pgfqpoint{2.384020in}{0.917890in}}%
\pgfpathlineto{\pgfqpoint{2.384540in}{0.918492in}}%
\pgfpathlineto{\pgfqpoint{2.389736in}{0.932270in}}%
\pgfpathlineto{\pgfqpoint{2.390602in}{0.934180in}}%
\pgfpathlineto{\pgfqpoint{2.391642in}{0.933079in}}%
\pgfpathlineto{\pgfqpoint{2.392681in}{0.934670in}}%
\pgfpathlineto{\pgfqpoint{2.392854in}{0.935030in}}%
\pgfpathlineto{\pgfqpoint{2.393201in}{0.934362in}}%
\pgfpathlineto{\pgfqpoint{2.393894in}{0.934261in}}%
\pgfpathlineto{\pgfqpoint{2.394760in}{0.931968in}}%
\pgfpathlineto{\pgfqpoint{2.395972in}{0.933135in}}%
\pgfpathlineto{\pgfqpoint{2.397878in}{0.925246in}}%
\pgfpathlineto{\pgfqpoint{2.401169in}{0.913499in}}%
\pgfpathlineto{\pgfqpoint{2.404287in}{0.899296in}}%
\pgfpathlineto{\pgfqpoint{2.407924in}{0.884405in}}%
\pgfpathlineto{\pgfqpoint{2.408444in}{0.884837in}}%
\pgfpathlineto{\pgfqpoint{2.409137in}{0.884139in}}%
\pgfpathlineto{\pgfqpoint{2.415373in}{0.859664in}}%
\pgfpathlineto{\pgfqpoint{2.415546in}{0.859866in}}%
\pgfpathlineto{\pgfqpoint{2.416585in}{0.862125in}}%
\pgfpathlineto{\pgfqpoint{2.417105in}{0.860401in}}%
\pgfpathlineto{\pgfqpoint{2.420743in}{0.837247in}}%
\pgfpathlineto{\pgfqpoint{2.421609in}{0.838886in}}%
\pgfpathlineto{\pgfqpoint{2.424727in}{0.823711in}}%
\pgfpathlineto{\pgfqpoint{2.428711in}{0.807781in}}%
\pgfpathlineto{\pgfqpoint{2.435986in}{0.785710in}}%
\pgfpathlineto{\pgfqpoint{2.436679in}{0.786134in}}%
\pgfpathlineto{\pgfqpoint{2.437025in}{0.784607in}}%
\pgfpathlineto{\pgfqpoint{2.437545in}{0.780671in}}%
\pgfpathlineto{\pgfqpoint{2.438411in}{0.784300in}}%
\pgfpathlineto{\pgfqpoint{2.439104in}{0.788907in}}%
\pgfpathlineto{\pgfqpoint{2.439970in}{0.785152in}}%
\pgfpathlineto{\pgfqpoint{2.442742in}{0.777623in}}%
\pgfpathlineto{\pgfqpoint{2.445167in}{0.788754in}}%
\pgfpathlineto{\pgfqpoint{2.445860in}{0.785422in}}%
\pgfpathlineto{\pgfqpoint{2.447245in}{0.778623in}}%
\pgfpathlineto{\pgfqpoint{2.447938in}{0.780703in}}%
\pgfpathlineto{\pgfqpoint{2.448978in}{0.787294in}}%
\pgfpathlineto{\pgfqpoint{2.449670in}{0.783050in}}%
\pgfpathlineto{\pgfqpoint{2.451922in}{0.777887in}}%
\pgfpathlineto{\pgfqpoint{2.452096in}{0.778584in}}%
\pgfpathlineto{\pgfqpoint{2.456599in}{0.792797in}}%
\pgfpathlineto{\pgfqpoint{2.457465in}{0.789864in}}%
\pgfpathlineto{\pgfqpoint{2.459198in}{0.786099in}}%
\pgfpathlineto{\pgfqpoint{2.459371in}{0.786587in}}%
\pgfpathlineto{\pgfqpoint{2.459544in}{0.786521in}}%
\pgfpathlineto{\pgfqpoint{2.459717in}{0.787326in}}%
\pgfpathlineto{\pgfqpoint{2.460757in}{0.788954in}}%
\pgfpathlineto{\pgfqpoint{2.461276in}{0.787355in}}%
\pgfpathlineto{\pgfqpoint{2.461623in}{0.788418in}}%
\pgfpathlineto{\pgfqpoint{2.462142in}{0.786698in}}%
\pgfpathlineto{\pgfqpoint{2.464741in}{0.781780in}}%
\pgfpathlineto{\pgfqpoint{2.464914in}{0.782858in}}%
\pgfpathlineto{\pgfqpoint{2.465087in}{0.784306in}}%
\pgfpathlineto{\pgfqpoint{2.466126in}{0.780252in}}%
\pgfpathlineto{\pgfqpoint{2.468551in}{0.772891in}}%
\pgfpathlineto{\pgfqpoint{2.470110in}{0.767772in}}%
\pgfpathlineto{\pgfqpoint{2.470630in}{0.769205in}}%
\pgfpathlineto{\pgfqpoint{2.472016in}{0.773690in}}%
\pgfpathlineto{\pgfqpoint{2.472536in}{0.772068in}}%
\pgfpathlineto{\pgfqpoint{2.472882in}{0.772593in}}%
\pgfpathlineto{\pgfqpoint{2.473402in}{0.771351in}}%
\pgfpathlineto{\pgfqpoint{2.474614in}{0.766674in}}%
\pgfpathlineto{\pgfqpoint{2.475307in}{0.768316in}}%
\pgfpathlineto{\pgfqpoint{2.476520in}{0.771104in}}%
\pgfpathlineto{\pgfqpoint{2.477039in}{0.768480in}}%
\pgfpathlineto{\pgfqpoint{2.477732in}{0.770756in}}%
\pgfpathlineto{\pgfqpoint{2.479118in}{0.765707in}}%
\pgfpathlineto{\pgfqpoint{2.481716in}{0.773199in}}%
\pgfpathlineto{\pgfqpoint{2.482063in}{0.777201in}}%
\pgfpathlineto{\pgfqpoint{2.483102in}{0.773865in}}%
\pgfpathlineto{\pgfqpoint{2.484315in}{0.767571in}}%
\pgfpathlineto{\pgfqpoint{2.485354in}{0.768182in}}%
\pgfpathlineto{\pgfqpoint{2.486220in}{0.769068in}}%
\pgfpathlineto{\pgfqpoint{2.488818in}{0.754671in}}%
\pgfpathlineto{\pgfqpoint{2.490377in}{0.756314in}}%
\pgfpathlineto{\pgfqpoint{2.492976in}{0.748678in}}%
\pgfpathlineto{\pgfqpoint{2.490897in}{0.757414in}}%
\pgfpathlineto{\pgfqpoint{2.494188in}{0.750799in}}%
\pgfpathlineto{\pgfqpoint{2.495227in}{0.751581in}}%
\pgfpathlineto{\pgfqpoint{2.495401in}{0.751070in}}%
\pgfpathlineto{\pgfqpoint{2.497306in}{0.745544in}}%
\pgfpathlineto{\pgfqpoint{2.497479in}{0.746263in}}%
\pgfpathlineto{\pgfqpoint{2.497826in}{0.746184in}}%
\pgfpathlineto{\pgfqpoint{2.499211in}{0.740446in}}%
\pgfpathlineto{\pgfqpoint{2.499558in}{0.742341in}}%
\pgfpathlineto{\pgfqpoint{2.501637in}{0.746166in}}%
\pgfpathlineto{\pgfqpoint{2.501983in}{0.745706in}}%
\pgfpathlineto{\pgfqpoint{2.502849in}{0.736855in}}%
\pgfpathlineto{\pgfqpoint{2.504235in}{0.742505in}}%
\pgfpathlineto{\pgfqpoint{2.505794in}{0.751584in}}%
\pgfpathlineto{\pgfqpoint{2.506487in}{0.749297in}}%
\pgfpathlineto{\pgfqpoint{2.506660in}{0.748737in}}%
\pgfpathlineto{\pgfqpoint{2.507180in}{0.751035in}}%
\pgfpathlineto{\pgfqpoint{2.507699in}{0.752692in}}%
\pgfpathlineto{\pgfqpoint{2.508739in}{0.752349in}}%
\pgfpathlineto{\pgfqpoint{2.508912in}{0.752199in}}%
\pgfpathlineto{\pgfqpoint{2.509085in}{0.753375in}}%
\pgfpathlineto{\pgfqpoint{2.509605in}{0.756241in}}%
\pgfpathlineto{\pgfqpoint{2.510471in}{0.753421in}}%
\pgfpathlineto{\pgfqpoint{2.513935in}{0.744362in}}%
\pgfpathlineto{\pgfqpoint{2.510990in}{0.753636in}}%
\pgfpathlineto{\pgfqpoint{2.514628in}{0.746388in}}%
\pgfpathlineto{\pgfqpoint{2.514975in}{0.748373in}}%
\pgfpathlineto{\pgfqpoint{2.515841in}{0.746733in}}%
\pgfpathlineto{\pgfqpoint{2.517746in}{0.741970in}}%
\pgfpathlineto{\pgfqpoint{2.517919in}{0.742608in}}%
\pgfpathlineto{\pgfqpoint{2.518785in}{0.740348in}}%
\pgfpathlineto{\pgfqpoint{2.518959in}{0.740635in}}%
\pgfpathlineto{\pgfqpoint{2.520864in}{0.748855in}}%
\pgfpathlineto{\pgfqpoint{2.521557in}{0.750746in}}%
\pgfpathlineto{\pgfqpoint{2.522250in}{0.748228in}}%
\pgfpathlineto{\pgfqpoint{2.527100in}{0.729679in}}%
\pgfpathlineto{\pgfqpoint{2.528139in}{0.734453in}}%
\pgfpathlineto{\pgfqpoint{2.529179in}{0.742277in}}%
\pgfpathlineto{\pgfqpoint{2.530564in}{0.753110in}}%
\pgfpathlineto{\pgfqpoint{2.531084in}{0.749611in}}%
\pgfpathlineto{\pgfqpoint{2.534029in}{0.734938in}}%
\pgfpathlineto{\pgfqpoint{2.534722in}{0.736445in}}%
\pgfpathlineto{\pgfqpoint{2.535241in}{0.733692in}}%
\pgfpathlineto{\pgfqpoint{2.535761in}{0.733082in}}%
\pgfpathlineto{\pgfqpoint{2.536281in}{0.736000in}}%
\pgfpathlineto{\pgfqpoint{2.537840in}{0.743872in}}%
\pgfpathlineto{\pgfqpoint{2.538359in}{0.741211in}}%
\pgfpathlineto{\pgfqpoint{2.538879in}{0.742514in}}%
\pgfpathlineto{\pgfqpoint{2.539745in}{0.736127in}}%
\pgfpathlineto{\pgfqpoint{2.541997in}{0.721803in}}%
\pgfpathlineto{\pgfqpoint{2.542343in}{0.721436in}}%
\pgfpathlineto{\pgfqpoint{2.542690in}{0.722889in}}%
\pgfpathlineto{\pgfqpoint{2.551004in}{0.763009in}}%
\pgfpathlineto{\pgfqpoint{2.553603in}{0.745126in}}%
\pgfpathlineto{\pgfqpoint{2.553949in}{0.747701in}}%
\pgfpathlineto{\pgfqpoint{2.554469in}{0.746318in}}%
\pgfpathlineto{\pgfqpoint{2.555162in}{0.748923in}}%
\pgfpathlineto{\pgfqpoint{2.558280in}{0.757923in}}%
\pgfpathlineto{\pgfqpoint{2.558453in}{0.757729in}}%
\pgfpathlineto{\pgfqpoint{2.559319in}{0.758637in}}%
\pgfpathlineto{\pgfqpoint{2.559665in}{0.757215in}}%
\pgfpathlineto{\pgfqpoint{2.560012in}{0.755627in}}%
\pgfpathlineto{\pgfqpoint{2.560705in}{0.759417in}}%
\pgfpathlineto{\pgfqpoint{2.561398in}{0.762026in}}%
\pgfpathlineto{\pgfqpoint{2.561917in}{0.759032in}}%
\pgfpathlineto{\pgfqpoint{2.562090in}{0.759279in}}%
\pgfpathlineto{\pgfqpoint{2.563823in}{0.755094in}}%
\pgfpathlineto{\pgfqpoint{2.565035in}{0.746625in}}%
\pgfpathlineto{\pgfqpoint{2.565728in}{0.751919in}}%
\pgfpathlineto{\pgfqpoint{2.567460in}{0.755473in}}%
\pgfpathlineto{\pgfqpoint{2.568326in}{0.760042in}}%
\pgfpathlineto{\pgfqpoint{2.569192in}{0.756343in}}%
\pgfpathlineto{\pgfqpoint{2.573350in}{0.735196in}}%
\pgfpathlineto{\pgfqpoint{2.573523in}{0.735306in}}%
\pgfpathlineto{\pgfqpoint{2.577680in}{0.749277in}}%
\pgfpathlineto{\pgfqpoint{2.577854in}{0.749100in}}%
\pgfpathlineto{\pgfqpoint{2.578200in}{0.750146in}}%
\pgfpathlineto{\pgfqpoint{2.578720in}{0.750216in}}%
\pgfpathlineto{\pgfqpoint{2.579759in}{0.751653in}}%
\pgfpathlineto{\pgfqpoint{2.580279in}{0.749852in}}%
\pgfpathlineto{\pgfqpoint{2.582704in}{0.745167in}}%
\pgfpathlineto{\pgfqpoint{2.584782in}{0.752847in}}%
\pgfpathlineto{\pgfqpoint{2.585475in}{0.750621in}}%
\pgfpathlineto{\pgfqpoint{2.588766in}{0.734218in}}%
\pgfpathlineto{\pgfqpoint{2.589459in}{0.733922in}}%
\pgfpathlineto{\pgfqpoint{2.589632in}{0.733046in}}%
\pgfpathlineto{\pgfqpoint{2.591538in}{0.726511in}}%
\pgfpathlineto{\pgfqpoint{2.593790in}{0.709966in}}%
\pgfpathlineto{\pgfqpoint{2.594136in}{0.713171in}}%
\pgfpathlineto{\pgfqpoint{2.596042in}{0.722881in}}%
\pgfpathlineto{\pgfqpoint{2.596561in}{0.722086in}}%
\pgfpathlineto{\pgfqpoint{2.597774in}{0.723001in}}%
\pgfpathlineto{\pgfqpoint{2.598986in}{0.730331in}}%
\pgfpathlineto{\pgfqpoint{2.600199in}{0.726970in}}%
\pgfpathlineto{\pgfqpoint{2.600545in}{0.724536in}}%
\pgfpathlineto{\pgfqpoint{2.601585in}{0.727671in}}%
\pgfpathlineto{\pgfqpoint{2.602104in}{0.725531in}}%
\pgfpathlineto{\pgfqpoint{2.602451in}{0.727289in}}%
\pgfpathlineto{\pgfqpoint{2.604183in}{0.735222in}}%
\pgfpathlineto{\pgfqpoint{2.604356in}{0.735016in}}%
\pgfpathlineto{\pgfqpoint{2.606088in}{0.723879in}}%
\pgfpathlineto{\pgfqpoint{2.606435in}{0.724198in}}%
\pgfpathlineto{\pgfqpoint{2.607647in}{0.729256in}}%
\pgfpathlineto{\pgfqpoint{2.608167in}{0.726744in}}%
\pgfpathlineto{\pgfqpoint{2.609380in}{0.716922in}}%
\pgfpathlineto{\pgfqpoint{2.610246in}{0.719820in}}%
\pgfpathlineto{\pgfqpoint{2.612671in}{0.724709in}}%
\pgfpathlineto{\pgfqpoint{2.612844in}{0.724593in}}%
\pgfpathlineto{\pgfqpoint{2.615096in}{0.719470in}}%
\pgfpathlineto{\pgfqpoint{2.615442in}{0.720407in}}%
\pgfpathlineto{\pgfqpoint{2.616482in}{0.721388in}}%
\pgfpathlineto{\pgfqpoint{2.616655in}{0.720088in}}%
\pgfpathlineto{\pgfqpoint{2.618214in}{0.716491in}}%
\pgfpathlineto{\pgfqpoint{2.618387in}{0.717603in}}%
\pgfpathlineto{\pgfqpoint{2.618907in}{0.719437in}}%
\pgfpathlineto{\pgfqpoint{2.619426in}{0.716629in}}%
\pgfpathlineto{\pgfqpoint{2.622198in}{0.702221in}}%
\pgfpathlineto{\pgfqpoint{2.622718in}{0.704711in}}%
\pgfpathlineto{\pgfqpoint{2.624623in}{0.708283in}}%
\pgfpathlineto{\pgfqpoint{2.624796in}{0.708250in}}%
\pgfpathlineto{\pgfqpoint{2.627221in}{0.714202in}}%
\pgfpathlineto{\pgfqpoint{2.625489in}{0.706749in}}%
\pgfpathlineto{\pgfqpoint{2.627395in}{0.713001in}}%
\pgfpathlineto{\pgfqpoint{2.629300in}{0.704125in}}%
\pgfpathlineto{\pgfqpoint{2.629646in}{0.700731in}}%
\pgfpathlineto{\pgfqpoint{2.630686in}{0.706224in}}%
\pgfpathlineto{\pgfqpoint{2.631898in}{0.704066in}}%
\pgfpathlineto{\pgfqpoint{2.632071in}{0.703115in}}%
\pgfpathlineto{\pgfqpoint{2.632764in}{0.707657in}}%
\pgfpathlineto{\pgfqpoint{2.633284in}{0.708708in}}%
\pgfpathlineto{\pgfqpoint{2.633457in}{0.706554in}}%
\pgfpathlineto{\pgfqpoint{2.634670in}{0.693351in}}%
\pgfpathlineto{\pgfqpoint{2.635189in}{0.699782in}}%
\pgfpathlineto{\pgfqpoint{2.635882in}{0.703045in}}%
\pgfpathlineto{\pgfqpoint{2.636402in}{0.698796in}}%
\pgfpathlineto{\pgfqpoint{2.636575in}{0.698857in}}%
\pgfpathlineto{\pgfqpoint{2.637615in}{0.702287in}}%
\pgfpathlineto{\pgfqpoint{2.638654in}{0.701000in}}%
\pgfpathlineto{\pgfqpoint{2.643504in}{0.688244in}}%
\pgfpathlineto{\pgfqpoint{2.643850in}{0.688728in}}%
\pgfpathlineto{\pgfqpoint{2.644024in}{0.686932in}}%
\pgfpathlineto{\pgfqpoint{2.644717in}{0.684973in}}%
\pgfpathlineto{\pgfqpoint{2.645063in}{0.687316in}}%
\pgfpathlineto{\pgfqpoint{2.649220in}{0.702681in}}%
\pgfpathlineto{\pgfqpoint{2.650086in}{0.700383in}}%
\pgfpathlineto{\pgfqpoint{2.650606in}{0.703444in}}%
\pgfpathlineto{\pgfqpoint{2.650779in}{0.703501in}}%
\pgfpathlineto{\pgfqpoint{2.651645in}{0.705686in}}%
\pgfpathlineto{\pgfqpoint{2.652511in}{0.700066in}}%
\pgfpathlineto{\pgfqpoint{2.652685in}{0.700017in}}%
\pgfpathlineto{\pgfqpoint{2.654590in}{0.706433in}}%
\pgfpathlineto{\pgfqpoint{2.654763in}{0.707190in}}%
\pgfpathlineto{\pgfqpoint{2.655283in}{0.702851in}}%
\pgfpathlineto{\pgfqpoint{2.655456in}{0.701659in}}%
\pgfpathlineto{\pgfqpoint{2.656496in}{0.704985in}}%
\pgfpathlineto{\pgfqpoint{2.656669in}{0.703948in}}%
\pgfpathlineto{\pgfqpoint{2.658055in}{0.706604in}}%
\pgfpathlineto{\pgfqpoint{2.657535in}{0.702932in}}%
\pgfpathlineto{\pgfqpoint{2.658228in}{0.704536in}}%
\pgfpathlineto{\pgfqpoint{2.658921in}{0.698463in}}%
\pgfpathlineto{\pgfqpoint{2.659960in}{0.701819in}}%
\pgfpathlineto{\pgfqpoint{2.660133in}{0.703060in}}%
\pgfpathlineto{\pgfqpoint{2.660999in}{0.699233in}}%
\pgfpathlineto{\pgfqpoint{2.666716in}{0.681875in}}%
\pgfpathlineto{\pgfqpoint{2.668448in}{0.693515in}}%
\pgfpathlineto{\pgfqpoint{2.668621in}{0.692298in}}%
\pgfpathlineto{\pgfqpoint{2.669660in}{0.679953in}}%
\pgfpathlineto{\pgfqpoint{2.670353in}{0.689015in}}%
\pgfpathlineto{\pgfqpoint{2.671739in}{0.691260in}}%
\pgfpathlineto{\pgfqpoint{2.671392in}{0.687978in}}%
\pgfpathlineto{\pgfqpoint{2.671912in}{0.690850in}}%
\pgfpathlineto{\pgfqpoint{2.672085in}{0.688356in}}%
\pgfpathlineto{\pgfqpoint{2.673298in}{0.691927in}}%
\pgfpathlineto{\pgfqpoint{2.673471in}{0.691842in}}%
\pgfpathlineto{\pgfqpoint{2.674857in}{0.699692in}}%
\pgfpathlineto{\pgfqpoint{2.675377in}{0.696104in}}%
\pgfpathlineto{\pgfqpoint{2.677628in}{0.701497in}}%
\pgfpathlineto{\pgfqpoint{2.678321in}{0.699831in}}%
\pgfpathlineto{\pgfqpoint{2.680573in}{0.689441in}}%
\pgfpathlineto{\pgfqpoint{2.680746in}{0.689530in}}%
\pgfpathlineto{\pgfqpoint{2.680920in}{0.689175in}}%
\pgfpathlineto{\pgfqpoint{2.683518in}{0.678454in}}%
\pgfpathlineto{\pgfqpoint{2.684557in}{0.674674in}}%
\pgfpathlineto{\pgfqpoint{2.684904in}{0.677034in}}%
\pgfpathlineto{\pgfqpoint{2.685250in}{0.680732in}}%
\pgfpathlineto{\pgfqpoint{2.686289in}{0.677816in}}%
\pgfpathlineto{\pgfqpoint{2.687329in}{0.668927in}}%
\pgfpathlineto{\pgfqpoint{2.687848in}{0.674595in}}%
\pgfpathlineto{\pgfqpoint{2.688715in}{0.690312in}}%
\pgfpathlineto{\pgfqpoint{2.689754in}{0.684613in}}%
\pgfpathlineto{\pgfqpoint{2.690793in}{0.690955in}}%
\pgfpathlineto{\pgfqpoint{2.692699in}{0.703158in}}%
\pgfpathlineto{\pgfqpoint{2.692872in}{0.702390in}}%
\pgfpathlineto{\pgfqpoint{2.695470in}{0.693289in}}%
\pgfpathlineto{\pgfqpoint{2.693738in}{0.702567in}}%
\pgfpathlineto{\pgfqpoint{2.695643in}{0.694187in}}%
\pgfpathlineto{\pgfqpoint{2.696163in}{0.693029in}}%
\pgfpathlineto{\pgfqpoint{2.698761in}{0.686077in}}%
\pgfpathlineto{\pgfqpoint{2.700147in}{0.686556in}}%
\pgfpathlineto{\pgfqpoint{2.700320in}{0.687831in}}%
\pgfpathlineto{\pgfqpoint{2.701013in}{0.681771in}}%
\pgfpathlineto{\pgfqpoint{2.702399in}{0.676760in}}%
\pgfpathlineto{\pgfqpoint{2.701533in}{0.682579in}}%
\pgfpathlineto{\pgfqpoint{2.702572in}{0.681127in}}%
\pgfpathlineto{\pgfqpoint{2.703265in}{0.688197in}}%
\pgfpathlineto{\pgfqpoint{2.704131in}{0.682641in}}%
\pgfpathlineto{\pgfqpoint{2.704304in}{0.678218in}}%
\pgfpathlineto{\pgfqpoint{2.705344in}{0.685431in}}%
\pgfpathlineto{\pgfqpoint{2.705517in}{0.684958in}}%
\pgfpathlineto{\pgfqpoint{2.707076in}{0.691931in}}%
\pgfpathlineto{\pgfqpoint{2.708115in}{0.690612in}}%
\pgfpathlineto{\pgfqpoint{2.708808in}{0.688230in}}%
\pgfpathlineto{\pgfqpoint{2.709328in}{0.691143in}}%
\pgfpathlineto{\pgfqpoint{2.709501in}{0.690098in}}%
\pgfpathlineto{\pgfqpoint{2.711060in}{0.706466in}}%
\pgfpathlineto{\pgfqpoint{2.711753in}{0.707810in}}%
\pgfpathlineto{\pgfqpoint{2.712099in}{0.705735in}}%
\pgfpathlineto{\pgfqpoint{2.713485in}{0.697446in}}%
\pgfpathlineto{\pgfqpoint{2.714178in}{0.699907in}}%
\pgfpathlineto{\pgfqpoint{2.716083in}{0.706292in}}%
\pgfpathlineto{\pgfqpoint{2.716257in}{0.704151in}}%
\pgfpathlineto{\pgfqpoint{2.717123in}{0.709861in}}%
\pgfpathlineto{\pgfqpoint{2.717642in}{0.714359in}}%
\pgfpathlineto{\pgfqpoint{2.718855in}{0.711802in}}%
\pgfpathlineto{\pgfqpoint{2.721107in}{0.704317in}}%
\pgfpathlineto{\pgfqpoint{2.721626in}{0.705981in}}%
\pgfpathlineto{\pgfqpoint{2.723185in}{0.718170in}}%
\pgfpathlineto{\pgfqpoint{2.723532in}{0.715244in}}%
\pgfpathlineto{\pgfqpoint{2.724225in}{0.712599in}}%
\pgfpathlineto{\pgfqpoint{2.726303in}{0.703645in}}%
\pgfpathlineto{\pgfqpoint{2.726996in}{0.706675in}}%
\pgfpathlineto{\pgfqpoint{2.728728in}{0.710634in}}%
\pgfpathlineto{\pgfqpoint{2.729768in}{0.714676in}}%
\pgfpathlineto{\pgfqpoint{2.730287in}{0.710462in}}%
\pgfpathlineto{\pgfqpoint{2.731153in}{0.711134in}}%
\pgfpathlineto{\pgfqpoint{2.732366in}{0.704064in}}%
\pgfpathlineto{\pgfqpoint{2.732539in}{0.704387in}}%
\pgfpathlineto{\pgfqpoint{2.732886in}{0.702344in}}%
\pgfpathlineto{\pgfqpoint{2.736350in}{0.687284in}}%
\pgfpathlineto{\pgfqpoint{2.736523in}{0.687701in}}%
\pgfpathlineto{\pgfqpoint{2.737043in}{0.687900in}}%
\pgfpathlineto{\pgfqpoint{2.737216in}{0.688761in}}%
\pgfpathlineto{\pgfqpoint{2.737736in}{0.689074in}}%
\pgfpathlineto{\pgfqpoint{2.738429in}{0.685313in}}%
\pgfpathlineto{\pgfqpoint{2.739468in}{0.685497in}}%
\pgfpathlineto{\pgfqpoint{2.739641in}{0.684914in}}%
\pgfpathlineto{\pgfqpoint{2.740161in}{0.687337in}}%
\pgfpathlineto{\pgfqpoint{2.740854in}{0.685642in}}%
\pgfpathlineto{\pgfqpoint{2.743279in}{0.700624in}}%
\pgfpathlineto{\pgfqpoint{2.745184in}{0.707637in}}%
\pgfpathlineto{\pgfqpoint{2.745358in}{0.707102in}}%
\pgfpathlineto{\pgfqpoint{2.747263in}{0.699682in}}%
\pgfpathlineto{\pgfqpoint{2.746397in}{0.707897in}}%
\pgfpathlineto{\pgfqpoint{2.747956in}{0.703299in}}%
\pgfpathlineto{\pgfqpoint{2.748302in}{0.706369in}}%
\pgfpathlineto{\pgfqpoint{2.748995in}{0.696713in}}%
\pgfpathlineto{\pgfqpoint{2.749168in}{0.694905in}}%
\pgfpathlineto{\pgfqpoint{2.749861in}{0.701409in}}%
\pgfpathlineto{\pgfqpoint{2.751420in}{0.711635in}}%
\pgfpathlineto{\pgfqpoint{2.751593in}{0.710862in}}%
\pgfpathlineto{\pgfqpoint{2.752979in}{0.707542in}}%
\pgfpathlineto{\pgfqpoint{2.753326in}{0.707898in}}%
\pgfpathlineto{\pgfqpoint{2.757483in}{0.677439in}}%
\pgfpathlineto{\pgfqpoint{2.758176in}{0.679633in}}%
\pgfpathlineto{\pgfqpoint{2.760774in}{0.699876in}}%
\pgfpathlineto{\pgfqpoint{2.761121in}{0.699132in}}%
\pgfpathlineto{\pgfqpoint{2.761813in}{0.695775in}}%
\pgfpathlineto{\pgfqpoint{2.762853in}{0.697118in}}%
\pgfpathlineto{\pgfqpoint{2.763892in}{0.702856in}}%
\pgfpathlineto{\pgfqpoint{2.764758in}{0.699458in}}%
\pgfpathlineto{\pgfqpoint{2.766490in}{0.693507in}}%
\pgfpathlineto{\pgfqpoint{2.769089in}{0.706024in}}%
\pgfpathlineto{\pgfqpoint{2.769435in}{0.707162in}}%
\pgfpathlineto{\pgfqpoint{2.769782in}{0.703474in}}%
\pgfpathlineto{\pgfqpoint{2.769955in}{0.703276in}}%
\pgfpathlineto{\pgfqpoint{2.770301in}{0.707419in}}%
\pgfpathlineto{\pgfqpoint{2.771167in}{0.698783in}}%
\pgfpathlineto{\pgfqpoint{2.771687in}{0.697130in}}%
\pgfpathlineto{\pgfqpoint{2.772553in}{0.698856in}}%
\pgfpathlineto{\pgfqpoint{2.772900in}{0.702604in}}%
\pgfpathlineto{\pgfqpoint{2.773592in}{0.697601in}}%
\pgfpathlineto{\pgfqpoint{2.774285in}{0.701161in}}%
\pgfpathlineto{\pgfqpoint{2.774805in}{0.699353in}}%
\pgfpathlineto{\pgfqpoint{2.775498in}{0.703770in}}%
\pgfpathlineto{\pgfqpoint{2.776364in}{0.697145in}}%
\pgfpathlineto{\pgfqpoint{2.778962in}{0.710369in}}%
\pgfpathlineto{\pgfqpoint{2.779655in}{0.707311in}}%
\pgfpathlineto{\pgfqpoint{2.779828in}{0.707096in}}%
\pgfpathlineto{\pgfqpoint{2.780002in}{0.704404in}}%
\pgfpathlineto{\pgfqpoint{2.781041in}{0.709112in}}%
\pgfpathlineto{\pgfqpoint{2.782600in}{0.712503in}}%
\pgfpathlineto{\pgfqpoint{2.782080in}{0.707610in}}%
\pgfpathlineto{\pgfqpoint{2.782946in}{0.711593in}}%
\pgfpathlineto{\pgfqpoint{2.783986in}{0.706893in}}%
\pgfpathlineto{\pgfqpoint{2.783293in}{0.712273in}}%
\pgfpathlineto{\pgfqpoint{2.785025in}{0.709225in}}%
\pgfpathlineto{\pgfqpoint{2.785545in}{0.712413in}}%
\pgfpathlineto{\pgfqpoint{2.786238in}{0.706480in}}%
\pgfpathlineto{\pgfqpoint{2.786584in}{0.709393in}}%
\pgfpathlineto{\pgfqpoint{2.786757in}{0.709403in}}%
\pgfpathlineto{\pgfqpoint{2.788143in}{0.698802in}}%
\pgfpathlineto{\pgfqpoint{2.788663in}{0.700954in}}%
\pgfpathlineto{\pgfqpoint{2.790222in}{0.712799in}}%
\pgfpathlineto{\pgfqpoint{2.790395in}{0.710270in}}%
\pgfpathlineto{\pgfqpoint{2.791434in}{0.703237in}}%
\pgfpathlineto{\pgfqpoint{2.792300in}{0.705540in}}%
\pgfpathlineto{\pgfqpoint{2.793513in}{0.710973in}}%
\pgfpathlineto{\pgfqpoint{2.794206in}{0.708998in}}%
\pgfpathlineto{\pgfqpoint{2.798536in}{0.680414in}}%
\pgfpathlineto{\pgfqpoint{2.798709in}{0.681048in}}%
\pgfpathlineto{\pgfqpoint{2.799576in}{0.677763in}}%
\pgfpathlineto{\pgfqpoint{2.799922in}{0.680312in}}%
\pgfpathlineto{\pgfqpoint{2.801827in}{0.686219in}}%
\pgfpathlineto{\pgfqpoint{2.802001in}{0.685660in}}%
\pgfpathlineto{\pgfqpoint{2.802174in}{0.687843in}}%
\pgfpathlineto{\pgfqpoint{2.803213in}{0.690588in}}%
\pgfpathlineto{\pgfqpoint{2.803560in}{0.688519in}}%
\pgfpathlineto{\pgfqpoint{2.804079in}{0.686941in}}%
\pgfpathlineto{\pgfqpoint{2.804599in}{0.690166in}}%
\pgfpathlineto{\pgfqpoint{2.807544in}{0.701218in}}%
\pgfpathlineto{\pgfqpoint{2.808063in}{0.698102in}}%
\pgfpathlineto{\pgfqpoint{2.811354in}{0.689281in}}%
\pgfpathlineto{\pgfqpoint{2.813260in}{0.692999in}}%
\pgfpathlineto{\pgfqpoint{2.813433in}{0.692868in}}%
\pgfpathlineto{\pgfqpoint{2.817071in}{0.675735in}}%
\pgfpathlineto{\pgfqpoint{2.818110in}{0.679686in}}%
\pgfpathlineto{\pgfqpoint{2.821574in}{0.698272in}}%
\pgfpathlineto{\pgfqpoint{2.818976in}{0.679397in}}%
\pgfpathlineto{\pgfqpoint{2.821921in}{0.693208in}}%
\pgfpathlineto{\pgfqpoint{2.822267in}{0.691741in}}%
\pgfpathlineto{\pgfqpoint{2.823133in}{0.695591in}}%
\pgfpathlineto{\pgfqpoint{2.823480in}{0.698283in}}%
\pgfpathlineto{\pgfqpoint{2.824519in}{0.695123in}}%
\pgfpathlineto{\pgfqpoint{2.824692in}{0.695862in}}%
\pgfpathlineto{\pgfqpoint{2.825559in}{0.697057in}}%
\pgfpathlineto{\pgfqpoint{2.825732in}{0.694853in}}%
\pgfpathlineto{\pgfqpoint{2.825905in}{0.694020in}}%
\pgfpathlineto{\pgfqpoint{2.826251in}{0.697025in}}%
\pgfpathlineto{\pgfqpoint{2.826944in}{0.705276in}}%
\pgfpathlineto{\pgfqpoint{2.828330in}{0.704292in}}%
\pgfpathlineto{\pgfqpoint{2.830582in}{0.709315in}}%
\pgfpathlineto{\pgfqpoint{2.831275in}{0.704912in}}%
\pgfpathlineto{\pgfqpoint{2.831794in}{0.708496in}}%
\pgfpathlineto{\pgfqpoint{2.832487in}{0.716383in}}%
\pgfpathlineto{\pgfqpoint{2.833873in}{0.715999in}}%
\pgfpathlineto{\pgfqpoint{2.834566in}{0.720026in}}%
\pgfpathlineto{\pgfqpoint{2.835086in}{0.717067in}}%
\pgfpathlineto{\pgfqpoint{2.837164in}{0.696397in}}%
\pgfpathlineto{\pgfqpoint{2.837684in}{0.700212in}}%
\pgfpathlineto{\pgfqpoint{2.838204in}{0.700813in}}%
\pgfpathlineto{\pgfqpoint{2.839589in}{0.704881in}}%
\pgfpathlineto{\pgfqpoint{2.839763in}{0.702631in}}%
\pgfpathlineto{\pgfqpoint{2.840109in}{0.700988in}}%
\pgfpathlineto{\pgfqpoint{2.840802in}{0.705651in}}%
\pgfpathlineto{\pgfqpoint{2.841668in}{0.710289in}}%
\pgfpathlineto{\pgfqpoint{2.842014in}{0.703550in}}%
\pgfpathlineto{\pgfqpoint{2.842534in}{0.697831in}}%
\pgfpathlineto{\pgfqpoint{2.843747in}{0.700666in}}%
\pgfpathlineto{\pgfqpoint{2.843920in}{0.701281in}}%
\pgfpathlineto{\pgfqpoint{2.844093in}{0.698586in}}%
\pgfpathlineto{\pgfqpoint{2.846345in}{0.681923in}}%
\pgfpathlineto{\pgfqpoint{2.847731in}{0.675894in}}%
\pgfpathlineto{\pgfqpoint{2.848250in}{0.677493in}}%
\pgfpathlineto{\pgfqpoint{2.848424in}{0.677431in}}%
\pgfpathlineto{\pgfqpoint{2.848597in}{0.678349in}}%
\pgfpathlineto{\pgfqpoint{2.851368in}{0.692273in}}%
\pgfpathlineto{\pgfqpoint{2.851542in}{0.689712in}}%
\pgfpathlineto{\pgfqpoint{2.852234in}{0.688230in}}%
\pgfpathlineto{\pgfqpoint{2.853101in}{0.674673in}}%
\pgfpathlineto{\pgfqpoint{2.855006in}{0.664863in}}%
\pgfpathlineto{\pgfqpoint{2.855872in}{0.660706in}}%
\pgfpathlineto{\pgfqpoint{2.856219in}{0.667439in}}%
\pgfpathlineto{\pgfqpoint{2.858644in}{0.683040in}}%
\pgfpathlineto{\pgfqpoint{2.858817in}{0.682311in}}%
\pgfpathlineto{\pgfqpoint{2.858990in}{0.682326in}}%
\pgfpathlineto{\pgfqpoint{2.859510in}{0.687867in}}%
\pgfpathlineto{\pgfqpoint{2.860029in}{0.679144in}}%
\pgfpathlineto{\pgfqpoint{2.860376in}{0.680046in}}%
\pgfpathlineto{\pgfqpoint{2.860896in}{0.674756in}}%
\pgfpathlineto{\pgfqpoint{2.861588in}{0.679907in}}%
\pgfpathlineto{\pgfqpoint{2.862801in}{0.688223in}}%
\pgfpathlineto{\pgfqpoint{2.863667in}{0.685789in}}%
\pgfpathlineto{\pgfqpoint{2.865746in}{0.676586in}}%
\pgfpathlineto{\pgfqpoint{2.868171in}{0.691081in}}%
\pgfpathlineto{\pgfqpoint{2.868690in}{0.689739in}}%
\pgfpathlineto{\pgfqpoint{2.869903in}{0.689240in}}%
\pgfpathlineto{\pgfqpoint{2.871116in}{0.697269in}}%
\pgfpathlineto{\pgfqpoint{2.871462in}{0.693486in}}%
\pgfpathlineto{\pgfqpoint{2.872155in}{0.697641in}}%
\pgfpathlineto{\pgfqpoint{2.872674in}{0.696668in}}%
\pgfpathlineto{\pgfqpoint{2.873021in}{0.697633in}}%
\pgfpathlineto{\pgfqpoint{2.873714in}{0.695436in}}%
\pgfpathlineto{\pgfqpoint{2.873887in}{0.694198in}}%
\pgfpathlineto{\pgfqpoint{2.874407in}{0.699470in}}%
\pgfpathlineto{\pgfqpoint{2.875273in}{0.705663in}}%
\pgfpathlineto{\pgfqpoint{2.876139in}{0.704688in}}%
\pgfpathlineto{\pgfqpoint{2.882375in}{0.678483in}}%
\pgfpathlineto{\pgfqpoint{2.882721in}{0.679485in}}%
\pgfpathlineto{\pgfqpoint{2.885146in}{0.699241in}}%
\pgfpathlineto{\pgfqpoint{2.886532in}{0.694529in}}%
\pgfpathlineto{\pgfqpoint{2.886879in}{0.690339in}}%
\pgfpathlineto{\pgfqpoint{2.887745in}{0.696436in}}%
\pgfpathlineto{\pgfqpoint{2.887918in}{0.696076in}}%
\pgfpathlineto{\pgfqpoint{2.888091in}{0.697032in}}%
\pgfpathlineto{\pgfqpoint{2.888611in}{0.693383in}}%
\pgfpathlineto{\pgfqpoint{2.890516in}{0.686889in}}%
\pgfpathlineto{\pgfqpoint{2.890863in}{0.685916in}}%
\pgfpathlineto{\pgfqpoint{2.891382in}{0.688620in}}%
\pgfpathlineto{\pgfqpoint{2.892941in}{0.692959in}}%
\pgfpathlineto{\pgfqpoint{2.893288in}{0.692408in}}%
\pgfpathlineto{\pgfqpoint{2.893634in}{0.688238in}}%
\pgfpathlineto{\pgfqpoint{2.893807in}{0.688960in}}%
\pgfpathlineto{\pgfqpoint{2.894327in}{0.681620in}}%
\pgfpathlineto{\pgfqpoint{2.895540in}{0.683974in}}%
\pgfpathlineto{\pgfqpoint{2.895886in}{0.686436in}}%
\pgfpathlineto{\pgfqpoint{2.896232in}{0.682466in}}%
\pgfpathlineto{\pgfqpoint{2.898484in}{0.668819in}}%
\pgfpathlineto{\pgfqpoint{2.900736in}{0.681777in}}%
\pgfpathlineto{\pgfqpoint{2.902295in}{0.677233in}}%
\pgfpathlineto{\pgfqpoint{2.902468in}{0.678169in}}%
\pgfpathlineto{\pgfqpoint{2.902988in}{0.680363in}}%
\pgfpathlineto{\pgfqpoint{2.903508in}{0.675364in}}%
\pgfpathlineto{\pgfqpoint{2.903681in}{0.673390in}}%
\pgfpathlineto{\pgfqpoint{2.904547in}{0.679685in}}%
\pgfpathlineto{\pgfqpoint{2.905933in}{0.687201in}}%
\pgfpathlineto{\pgfqpoint{2.906626in}{0.682994in}}%
\pgfpathlineto{\pgfqpoint{2.908704in}{0.674458in}}%
\pgfpathlineto{\pgfqpoint{2.908878in}{0.674784in}}%
\pgfpathlineto{\pgfqpoint{2.909397in}{0.673071in}}%
\pgfpathlineto{\pgfqpoint{2.909744in}{0.669068in}}%
\pgfpathlineto{\pgfqpoint{2.910263in}{0.678712in}}%
\pgfpathlineto{\pgfqpoint{2.910783in}{0.672410in}}%
\pgfpathlineto{\pgfqpoint{2.913554in}{0.688567in}}%
\pgfpathlineto{\pgfqpoint{2.913901in}{0.684340in}}%
\pgfpathlineto{\pgfqpoint{2.914074in}{0.685216in}}%
\pgfpathlineto{\pgfqpoint{2.914594in}{0.683099in}}%
\pgfpathlineto{\pgfqpoint{2.917019in}{0.667897in}}%
\pgfpathlineto{\pgfqpoint{2.917365in}{0.665209in}}%
\pgfpathlineto{\pgfqpoint{2.918058in}{0.670874in}}%
\pgfpathlineto{\pgfqpoint{2.918405in}{0.674153in}}%
\pgfpathlineto{\pgfqpoint{2.919444in}{0.671487in}}%
\pgfpathlineto{\pgfqpoint{2.919790in}{0.668987in}}%
\pgfpathlineto{\pgfqpoint{2.920483in}{0.672269in}}%
\pgfpathlineto{\pgfqpoint{2.920657in}{0.672102in}}%
\pgfpathlineto{\pgfqpoint{2.921696in}{0.679013in}}%
\pgfpathlineto{\pgfqpoint{2.922042in}{0.674131in}}%
\pgfpathlineto{\pgfqpoint{2.922389in}{0.669922in}}%
\pgfpathlineto{\pgfqpoint{2.923255in}{0.675021in}}%
\pgfpathlineto{\pgfqpoint{2.923601in}{0.673780in}}%
\pgfpathlineto{\pgfqpoint{2.924641in}{0.677565in}}%
\pgfpathlineto{\pgfqpoint{2.923948in}{0.672582in}}%
\pgfpathlineto{\pgfqpoint{2.924987in}{0.675016in}}%
\pgfpathlineto{\pgfqpoint{2.925160in}{0.673069in}}%
\pgfpathlineto{\pgfqpoint{2.925853in}{0.678336in}}%
\pgfpathlineto{\pgfqpoint{2.926373in}{0.675532in}}%
\pgfpathlineto{\pgfqpoint{2.928798in}{0.686978in}}%
\pgfpathlineto{\pgfqpoint{2.929318in}{0.684758in}}%
\pgfpathlineto{\pgfqpoint{2.931223in}{0.674442in}}%
\pgfpathlineto{\pgfqpoint{2.931569in}{0.676453in}}%
\pgfpathlineto{\pgfqpoint{2.932609in}{0.680411in}}%
\pgfpathlineto{\pgfqpoint{2.933302in}{0.679374in}}%
\pgfpathlineto{\pgfqpoint{2.933475in}{0.679023in}}%
\pgfpathlineto{\pgfqpoint{2.933821in}{0.681505in}}%
\pgfpathlineto{\pgfqpoint{2.934168in}{0.682603in}}%
\pgfpathlineto{\pgfqpoint{2.934341in}{0.684571in}}%
\pgfpathlineto{\pgfqpoint{2.935034in}{0.676611in}}%
\pgfpathlineto{\pgfqpoint{2.935553in}{0.674518in}}%
\pgfpathlineto{\pgfqpoint{2.936246in}{0.676828in}}%
\pgfpathlineto{\pgfqpoint{2.936593in}{0.675725in}}%
\pgfpathlineto{\pgfqpoint{2.937459in}{0.673404in}}%
\pgfpathlineto{\pgfqpoint{2.937979in}{0.676241in}}%
\pgfpathlineto{\pgfqpoint{2.938325in}{0.674788in}}%
\pgfpathlineto{\pgfqpoint{2.942655in}{0.690618in}}%
\pgfpathlineto{\pgfqpoint{2.943522in}{0.695651in}}%
\pgfpathlineto{\pgfqpoint{2.944388in}{0.692680in}}%
\pgfpathlineto{\pgfqpoint{2.944907in}{0.695450in}}%
\pgfpathlineto{\pgfqpoint{2.945947in}{0.692644in}}%
\pgfpathlineto{\pgfqpoint{2.946466in}{0.693323in}}%
\pgfpathlineto{\pgfqpoint{2.946640in}{0.692387in}}%
\pgfpathlineto{\pgfqpoint{2.948545in}{0.686454in}}%
\pgfpathlineto{\pgfqpoint{2.948891in}{0.688389in}}%
\pgfpathlineto{\pgfqpoint{2.949758in}{0.683554in}}%
\pgfpathlineto{\pgfqpoint{2.950104in}{0.684704in}}%
\pgfpathlineto{\pgfqpoint{2.950277in}{0.682401in}}%
\pgfpathlineto{\pgfqpoint{2.950624in}{0.682403in}}%
\pgfpathlineto{\pgfqpoint{2.952183in}{0.659902in}}%
\pgfpathlineto{\pgfqpoint{2.952702in}{0.671569in}}%
\pgfpathlineto{\pgfqpoint{2.953568in}{0.683727in}}%
\pgfpathlineto{\pgfqpoint{2.954434in}{0.681031in}}%
\pgfpathlineto{\pgfqpoint{2.954954in}{0.680020in}}%
\pgfpathlineto{\pgfqpoint{2.955647in}{0.684381in}}%
\pgfpathlineto{\pgfqpoint{2.957033in}{0.667026in}}%
\pgfpathlineto{\pgfqpoint{2.957726in}{0.668336in}}%
\pgfpathlineto{\pgfqpoint{2.958245in}{0.668321in}}%
\pgfpathlineto{\pgfqpoint{2.958419in}{0.669095in}}%
\pgfpathlineto{\pgfqpoint{2.958592in}{0.668990in}}%
\pgfpathlineto{\pgfqpoint{2.960151in}{0.678618in}}%
\pgfpathlineto{\pgfqpoint{2.960497in}{0.676986in}}%
\pgfpathlineto{\pgfqpoint{2.963615in}{0.698203in}}%
\pgfpathlineto{\pgfqpoint{2.964654in}{0.693990in}}%
\pgfpathlineto{\pgfqpoint{2.967080in}{0.685037in}}%
\pgfpathlineto{\pgfqpoint{2.967253in}{0.685105in}}%
\pgfpathlineto{\pgfqpoint{2.967426in}{0.684908in}}%
\pgfpathlineto{\pgfqpoint{2.967599in}{0.685812in}}%
\pgfpathlineto{\pgfqpoint{2.970371in}{0.696391in}}%
\pgfpathlineto{\pgfqpoint{2.970544in}{0.695526in}}%
\pgfpathlineto{\pgfqpoint{2.971064in}{0.696081in}}%
\pgfpathlineto{\pgfqpoint{2.972449in}{0.691218in}}%
\pgfpathlineto{\pgfqpoint{2.974528in}{0.700296in}}%
\pgfpathlineto{\pgfqpoint{2.973315in}{0.687911in}}%
\pgfpathlineto{\pgfqpoint{2.974874in}{0.698483in}}%
\pgfpathlineto{\pgfqpoint{2.976087in}{0.692846in}}%
\pgfpathlineto{\pgfqpoint{2.976433in}{0.694337in}}%
\pgfpathlineto{\pgfqpoint{2.977473in}{0.703519in}}%
\pgfpathlineto{\pgfqpoint{2.978339in}{0.699392in}}%
\pgfpathlineto{\pgfqpoint{2.979378in}{0.693417in}}%
\pgfpathlineto{\pgfqpoint{2.980071in}{0.698952in}}%
\pgfpathlineto{\pgfqpoint{2.980418in}{0.700181in}}%
\pgfpathlineto{\pgfqpoint{2.981110in}{0.706929in}}%
\pgfpathlineto{\pgfqpoint{2.981630in}{0.699145in}}%
\pgfpathlineto{\pgfqpoint{2.982323in}{0.696757in}}%
\pgfpathlineto{\pgfqpoint{2.982669in}{0.699688in}}%
\pgfpathlineto{\pgfqpoint{2.985094in}{0.708775in}}%
\pgfpathlineto{\pgfqpoint{2.989079in}{0.694023in}}%
\pgfpathlineto{\pgfqpoint{2.989252in}{0.697139in}}%
\pgfpathlineto{\pgfqpoint{2.989598in}{0.697572in}}%
\pgfpathlineto{\pgfqpoint{2.989945in}{0.695056in}}%
\pgfpathlineto{\pgfqpoint{2.992023in}{0.684772in}}%
\pgfpathlineto{\pgfqpoint{2.993236in}{0.689544in}}%
\pgfpathlineto{\pgfqpoint{2.993409in}{0.691001in}}%
\pgfpathlineto{\pgfqpoint{2.994102in}{0.684122in}}%
\pgfpathlineto{\pgfqpoint{2.995488in}{0.692091in}}%
\pgfpathlineto{\pgfqpoint{2.997220in}{0.696376in}}%
\pgfpathlineto{\pgfqpoint{2.997393in}{0.695149in}}%
\pgfpathlineto{\pgfqpoint{2.997740in}{0.691539in}}%
\pgfpathlineto{\pgfqpoint{2.998432in}{0.698157in}}%
\pgfpathlineto{\pgfqpoint{3.000858in}{0.708711in}}%
\pgfpathlineto{\pgfqpoint{3.001204in}{0.710276in}}%
\pgfpathlineto{\pgfqpoint{3.001897in}{0.706311in}}%
\pgfpathlineto{\pgfqpoint{3.002070in}{0.704335in}}%
\pgfpathlineto{\pgfqpoint{3.002936in}{0.711590in}}%
\pgfpathlineto{\pgfqpoint{3.003802in}{0.719229in}}%
\pgfpathlineto{\pgfqpoint{3.004668in}{0.718638in}}%
\pgfpathlineto{\pgfqpoint{3.005361in}{0.723444in}}%
\pgfpathlineto{\pgfqpoint{3.006227in}{0.719987in}}%
\pgfpathlineto{\pgfqpoint{3.006574in}{0.718558in}}%
\pgfpathlineto{\pgfqpoint{3.006920in}{0.722366in}}%
\pgfpathlineto{\pgfqpoint{3.008479in}{0.736043in}}%
\pgfpathlineto{\pgfqpoint{3.008999in}{0.733240in}}%
\pgfpathlineto{\pgfqpoint{3.011597in}{0.727262in}}%
\pgfpathlineto{\pgfqpoint{3.012117in}{0.724852in}}%
\pgfpathlineto{\pgfqpoint{3.012636in}{0.728295in}}%
\pgfpathlineto{\pgfqpoint{3.013156in}{0.731561in}}%
\pgfpathlineto{\pgfqpoint{3.014195in}{0.729662in}}%
\pgfpathlineto{\pgfqpoint{3.016274in}{0.726856in}}%
\pgfpathlineto{\pgfqpoint{3.016967in}{0.726253in}}%
\pgfpathlineto{\pgfqpoint{3.017313in}{0.727950in}}%
\pgfpathlineto{\pgfqpoint{3.020778in}{0.745901in}}%
\pgfpathlineto{\pgfqpoint{3.021990in}{0.740996in}}%
\pgfpathlineto{\pgfqpoint{3.022164in}{0.739919in}}%
\pgfpathlineto{\pgfqpoint{3.023030in}{0.741232in}}%
\pgfpathlineto{\pgfqpoint{3.023376in}{0.741165in}}%
\pgfpathlineto{\pgfqpoint{3.024069in}{0.743071in}}%
\pgfpathlineto{\pgfqpoint{3.024935in}{0.741843in}}%
\pgfpathlineto{\pgfqpoint{3.029092in}{0.722692in}}%
\pgfpathlineto{\pgfqpoint{3.029439in}{0.719357in}}%
\pgfpathlineto{\pgfqpoint{3.030305in}{0.724601in}}%
\pgfpathlineto{\pgfqpoint{3.033423in}{0.739172in}}%
\pgfpathlineto{\pgfqpoint{3.033769in}{0.736909in}}%
\pgfpathlineto{\pgfqpoint{3.037927in}{0.721688in}}%
\pgfpathlineto{\pgfqpoint{3.038793in}{0.728108in}}%
\pgfpathlineto{\pgfqpoint{3.039486in}{0.722903in}}%
\pgfpathlineto{\pgfqpoint{3.039832in}{0.719190in}}%
\pgfpathlineto{\pgfqpoint{3.040698in}{0.725111in}}%
\pgfpathlineto{\pgfqpoint{3.040871in}{0.724988in}}%
\pgfpathlineto{\pgfqpoint{3.041218in}{0.726561in}}%
\pgfpathlineto{\pgfqpoint{3.042604in}{0.734114in}}%
\pgfpathlineto{\pgfqpoint{3.042777in}{0.736399in}}%
\pgfpathlineto{\pgfqpoint{3.043643in}{0.729686in}}%
\pgfpathlineto{\pgfqpoint{3.045375in}{0.726945in}}%
\pgfpathlineto{\pgfqpoint{3.047800in}{0.717452in}}%
\pgfpathlineto{\pgfqpoint{3.048320in}{0.715002in}}%
\pgfpathlineto{\pgfqpoint{3.048840in}{0.718918in}}%
\pgfpathlineto{\pgfqpoint{3.049706in}{0.716945in}}%
\pgfpathlineto{\pgfqpoint{3.050572in}{0.720319in}}%
\pgfpathlineto{\pgfqpoint{3.052997in}{0.707472in}}%
\pgfpathlineto{\pgfqpoint{3.053170in}{0.707694in}}%
\pgfpathlineto{\pgfqpoint{3.053343in}{0.705879in}}%
\pgfpathlineto{\pgfqpoint{3.054729in}{0.701048in}}%
\pgfpathlineto{\pgfqpoint{3.055249in}{0.701350in}}%
\pgfpathlineto{\pgfqpoint{3.055595in}{0.700338in}}%
\pgfpathlineto{\pgfqpoint{3.055768in}{0.699012in}}%
\pgfpathlineto{\pgfqpoint{3.056808in}{0.701785in}}%
\pgfpathlineto{\pgfqpoint{3.056981in}{0.702718in}}%
\pgfpathlineto{\pgfqpoint{3.057847in}{0.699038in}}%
\pgfpathlineto{\pgfqpoint{3.059752in}{0.696770in}}%
\pgfpathlineto{\pgfqpoint{3.060272in}{0.699595in}}%
\pgfpathlineto{\pgfqpoint{3.060965in}{0.696819in}}%
\pgfpathlineto{\pgfqpoint{3.062178in}{0.692221in}}%
\pgfpathlineto{\pgfqpoint{3.062697in}{0.694038in}}%
\pgfpathlineto{\pgfqpoint{3.063044in}{0.694796in}}%
\pgfpathlineto{\pgfqpoint{3.063390in}{0.691853in}}%
\pgfpathlineto{\pgfqpoint{3.063563in}{0.690597in}}%
\pgfpathlineto{\pgfqpoint{3.064083in}{0.697275in}}%
\pgfpathlineto{\pgfqpoint{3.065988in}{0.713147in}}%
\pgfpathlineto{\pgfqpoint{3.066162in}{0.711968in}}%
\pgfpathlineto{\pgfqpoint{3.066681in}{0.713634in}}%
\pgfpathlineto{\pgfqpoint{3.067028in}{0.710568in}}%
\pgfpathlineto{\pgfqpoint{3.068587in}{0.684518in}}%
\pgfpathlineto{\pgfqpoint{3.069106in}{0.691154in}}%
\pgfpathlineto{\pgfqpoint{3.069453in}{0.696355in}}%
\pgfpathlineto{\pgfqpoint{3.070146in}{0.686837in}}%
\pgfpathlineto{\pgfqpoint{3.071531in}{0.674679in}}%
\pgfpathlineto{\pgfqpoint{3.072051in}{0.677702in}}%
\pgfpathlineto{\pgfqpoint{3.074130in}{0.690495in}}%
\pgfpathlineto{\pgfqpoint{3.075515in}{0.702462in}}%
\pgfpathlineto{\pgfqpoint{3.076728in}{0.699723in}}%
\pgfpathlineto{\pgfqpoint{3.077594in}{0.693319in}}%
\pgfpathlineto{\pgfqpoint{3.078287in}{0.698291in}}%
\pgfpathlineto{\pgfqpoint{3.078633in}{0.701380in}}%
\pgfpathlineto{\pgfqpoint{3.079500in}{0.697438in}}%
\pgfpathlineto{\pgfqpoint{3.079673in}{0.698187in}}%
\pgfpathlineto{\pgfqpoint{3.079846in}{0.697104in}}%
\pgfpathlineto{\pgfqpoint{3.080712in}{0.700970in}}%
\pgfpathlineto{\pgfqpoint{3.082271in}{0.705233in}}%
\pgfpathlineto{\pgfqpoint{3.081232in}{0.699642in}}%
\pgfpathlineto{\pgfqpoint{3.082791in}{0.704290in}}%
\pgfpathlineto{\pgfqpoint{3.083657in}{0.693659in}}%
\pgfpathlineto{\pgfqpoint{3.084869in}{0.698916in}}%
\pgfpathlineto{\pgfqpoint{3.085909in}{0.699912in}}%
\pgfpathlineto{\pgfqpoint{3.085389in}{0.697896in}}%
\pgfpathlineto{\pgfqpoint{3.086082in}{0.699777in}}%
\pgfpathlineto{\pgfqpoint{3.088334in}{0.690021in}}%
\pgfpathlineto{\pgfqpoint{3.090932in}{0.702502in}}%
\pgfpathlineto{\pgfqpoint{3.089373in}{0.689084in}}%
\pgfpathlineto{\pgfqpoint{3.091452in}{0.701406in}}%
\pgfpathlineto{\pgfqpoint{3.092318in}{0.692482in}}%
\pgfpathlineto{\pgfqpoint{3.093184in}{0.695565in}}%
\pgfpathlineto{\pgfqpoint{3.094223in}{0.702191in}}%
\pgfpathlineto{\pgfqpoint{3.094743in}{0.695511in}}%
\pgfpathlineto{\pgfqpoint{3.096129in}{0.678618in}}%
\pgfpathlineto{\pgfqpoint{3.097168in}{0.680730in}}%
\pgfpathlineto{\pgfqpoint{3.099593in}{0.696945in}}%
\pgfpathlineto{\pgfqpoint{3.099766in}{0.695574in}}%
\pgfpathlineto{\pgfqpoint{3.100806in}{0.694867in}}%
\pgfpathlineto{\pgfqpoint{3.100286in}{0.695946in}}%
\pgfpathlineto{\pgfqpoint{3.100979in}{0.695529in}}%
\pgfpathlineto{\pgfqpoint{3.101325in}{0.697855in}}%
\pgfpathlineto{\pgfqpoint{3.101845in}{0.691364in}}%
\pgfpathlineto{\pgfqpoint{3.102538in}{0.696769in}}%
\pgfpathlineto{\pgfqpoint{3.105656in}{0.705272in}}%
\pgfpathlineto{\pgfqpoint{3.105829in}{0.703707in}}%
\pgfpathlineto{\pgfqpoint{3.109467in}{0.685384in}}%
\pgfpathlineto{\pgfqpoint{3.109986in}{0.686786in}}%
\pgfpathlineto{\pgfqpoint{3.112758in}{0.699429in}}%
\pgfpathlineto{\pgfqpoint{3.116915in}{0.680740in}}%
\pgfpathlineto{\pgfqpoint{3.119340in}{0.694566in}}%
\pgfpathlineto{\pgfqpoint{3.120033in}{0.693812in}}%
\pgfpathlineto{\pgfqpoint{3.121246in}{0.690840in}}%
\pgfpathlineto{\pgfqpoint{3.121592in}{0.693447in}}%
\pgfpathlineto{\pgfqpoint{3.121939in}{0.694338in}}%
\pgfpathlineto{\pgfqpoint{3.122285in}{0.693119in}}%
\pgfpathlineto{\pgfqpoint{3.123844in}{0.672929in}}%
\pgfpathlineto{\pgfqpoint{3.125056in}{0.675339in}}%
\pgfpathlineto{\pgfqpoint{3.127828in}{0.693205in}}%
\pgfpathlineto{\pgfqpoint{3.128001in}{0.693974in}}%
\pgfpathlineto{\pgfqpoint{3.128694in}{0.690317in}}%
\pgfpathlineto{\pgfqpoint{3.129907in}{0.692119in}}%
\pgfpathlineto{\pgfqpoint{3.129214in}{0.688940in}}%
\pgfpathlineto{\pgfqpoint{3.130080in}{0.691182in}}%
\pgfpathlineto{\pgfqpoint{3.131985in}{0.683882in}}%
\pgfpathlineto{\pgfqpoint{3.132332in}{0.684901in}}%
\pgfpathlineto{\pgfqpoint{3.133371in}{0.689687in}}%
\pgfpathlineto{\pgfqpoint{3.133891in}{0.687264in}}%
\pgfpathlineto{\pgfqpoint{3.134757in}{0.677140in}}%
\pgfpathlineto{\pgfqpoint{3.135450in}{0.682496in}}%
\pgfpathlineto{\pgfqpoint{3.137182in}{0.693694in}}%
\pgfpathlineto{\pgfqpoint{3.137702in}{0.692345in}}%
\pgfpathlineto{\pgfqpoint{3.139261in}{0.677575in}}%
\pgfpathlineto{\pgfqpoint{3.140127in}{0.668582in}}%
\pgfpathlineto{\pgfqpoint{3.140993in}{0.674666in}}%
\pgfpathlineto{\pgfqpoint{3.145843in}{0.697743in}}%
\pgfpathlineto{\pgfqpoint{3.146016in}{0.697611in}}%
\pgfpathlineto{\pgfqpoint{3.148095in}{0.684382in}}%
\pgfpathlineto{\pgfqpoint{3.148268in}{0.684644in}}%
\pgfpathlineto{\pgfqpoint{3.149827in}{0.698269in}}%
\pgfpathlineto{\pgfqpoint{3.150347in}{0.695342in}}%
\pgfpathlineto{\pgfqpoint{3.151213in}{0.679517in}}%
\pgfpathlineto{\pgfqpoint{3.152425in}{0.682210in}}%
\pgfpathlineto{\pgfqpoint{3.155197in}{0.665728in}}%
\pgfpathlineto{\pgfqpoint{3.155543in}{0.670802in}}%
\pgfpathlineto{\pgfqpoint{3.156236in}{0.675329in}}%
\pgfpathlineto{\pgfqpoint{3.156756in}{0.670564in}}%
\pgfpathlineto{\pgfqpoint{3.156929in}{0.668059in}}%
\pgfpathlineto{\pgfqpoint{3.157968in}{0.672724in}}%
\pgfpathlineto{\pgfqpoint{3.158661in}{0.676661in}}%
\pgfpathlineto{\pgfqpoint{3.159181in}{0.671207in}}%
\pgfpathlineto{\pgfqpoint{3.160047in}{0.666998in}}%
\pgfpathlineto{\pgfqpoint{3.160567in}{0.671473in}}%
\pgfpathlineto{\pgfqpoint{3.161952in}{0.670152in}}%
\pgfpathlineto{\pgfqpoint{3.161433in}{0.672608in}}%
\pgfpathlineto{\pgfqpoint{3.162126in}{0.671167in}}%
\pgfpathlineto{\pgfqpoint{3.164551in}{0.681727in}}%
\pgfpathlineto{\pgfqpoint{3.164897in}{0.679739in}}%
\pgfpathlineto{\pgfqpoint{3.165417in}{0.682895in}}%
\pgfpathlineto{\pgfqpoint{3.165936in}{0.689067in}}%
\pgfpathlineto{\pgfqpoint{3.166283in}{0.694858in}}%
\pgfpathlineto{\pgfqpoint{3.166456in}{0.692791in}}%
\pgfusepath{stroke}%
\end{pgfscope}%
\begin{pgfscope}%
\pgfpathrectangle{\pgfqpoint{0.568671in}{0.451277in}}{\pgfqpoint{2.598305in}{0.798874in}}%
\pgfusepath{clip}%
\pgfsetrectcap%
\pgfsetroundjoin%
\pgfsetlinewidth{1.505625pt}%
\definecolor{currentstroke}{rgb}{0.121569,0.466667,0.705882}%
\pgfsetstrokecolor{currentstroke}%
\pgfsetdash{}{0pt}%
\pgfpathmoveto{\pgfqpoint{0.651109in}{1.264040in}}%
\pgfpathlineto{\pgfqpoint{0.661864in}{1.247212in}}%
\pgfpathlineto{\pgfqpoint{0.680745in}{1.206622in}}%
\pgfpathlineto{\pgfqpoint{0.686461in}{1.191396in}}%
\pgfpathlineto{\pgfqpoint{0.699626in}{1.163489in}}%
\pgfpathlineto{\pgfqpoint{0.714523in}{1.135311in}}%
\pgfpathlineto{\pgfqpoint{0.733577in}{1.104817in}}%
\pgfpathlineto{\pgfqpoint{0.744143in}{1.091528in}}%
\pgfpathlineto{\pgfqpoint{0.746568in}{1.083566in}}%
\pgfpathlineto{\pgfqpoint{0.750726in}{1.073720in}}%
\pgfpathlineto{\pgfqpoint{0.754537in}{1.068741in}}%
\pgfpathlineto{\pgfqpoint{0.781212in}{1.040315in}}%
\pgfpathlineto{\pgfqpoint{0.783638in}{1.037764in}}%
\pgfpathlineto{\pgfqpoint{0.788834in}{1.028258in}}%
\pgfpathlineto{\pgfqpoint{0.792991in}{1.027905in}}%
\pgfpathlineto{\pgfqpoint{0.795243in}{1.025590in}}%
\pgfpathlineto{\pgfqpoint{0.798015in}{1.023594in}}%
\pgfpathlineto{\pgfqpoint{0.804770in}{1.022014in}}%
\pgfpathlineto{\pgfqpoint{0.807888in}{1.020659in}}%
\pgfpathlineto{\pgfqpoint{0.809274in}{1.018802in}}%
\pgfpathlineto{\pgfqpoint{0.812219in}{1.016288in}}%
\pgfpathlineto{\pgfqpoint{0.817242in}{1.010934in}}%
\pgfpathlineto{\pgfqpoint{0.817415in}{1.011060in}}%
\pgfpathlineto{\pgfqpoint{0.820880in}{1.011730in}}%
\pgfpathlineto{\pgfqpoint{0.822612in}{1.010211in}}%
\pgfpathlineto{\pgfqpoint{0.826423in}{1.007379in}}%
\pgfpathlineto{\pgfqpoint{0.829541in}{1.006980in}}%
\pgfpathlineto{\pgfqpoint{0.831793in}{1.004657in}}%
\pgfpathlineto{\pgfqpoint{0.835950in}{1.000291in}}%
\pgfpathlineto{\pgfqpoint{0.838548in}{0.999266in}}%
\pgfpathlineto{\pgfqpoint{0.840800in}{0.998697in}}%
\pgfpathlineto{\pgfqpoint{0.843225in}{0.995934in}}%
\pgfpathlineto{\pgfqpoint{0.847556in}{0.990529in}}%
\pgfpathlineto{\pgfqpoint{0.850501in}{0.990094in}}%
\pgfpathlineto{\pgfqpoint{0.852752in}{0.990513in}}%
\pgfpathlineto{\pgfqpoint{0.856910in}{0.984210in}}%
\pgfpathlineto{\pgfqpoint{0.859854in}{0.981772in}}%
\pgfpathlineto{\pgfqpoint{0.862280in}{0.978931in}}%
\pgfpathlineto{\pgfqpoint{0.866090in}{0.975540in}}%
\pgfpathlineto{\pgfqpoint{0.868342in}{0.975860in}}%
\pgfpathlineto{\pgfqpoint{0.870074in}{0.977614in}}%
\pgfpathlineto{\pgfqpoint{0.870594in}{0.977001in}}%
\pgfpathlineto{\pgfqpoint{0.873366in}{0.975663in}}%
\pgfpathlineto{\pgfqpoint{0.877003in}{0.975035in}}%
\pgfpathlineto{\pgfqpoint{0.881334in}{0.974336in}}%
\pgfpathlineto{\pgfqpoint{0.883239in}{0.973785in}}%
\pgfpathlineto{\pgfqpoint{0.896058in}{0.958452in}}%
\pgfpathlineto{\pgfqpoint{0.903333in}{0.955898in}}%
\pgfpathlineto{\pgfqpoint{0.904372in}{0.956752in}}%
\pgfpathlineto{\pgfqpoint{0.905065in}{0.955893in}}%
\pgfpathlineto{\pgfqpoint{0.907663in}{0.954186in}}%
\pgfpathlineto{\pgfqpoint{0.913206in}{0.953787in}}%
\pgfpathlineto{\pgfqpoint{0.914939in}{0.948924in}}%
\pgfpathlineto{\pgfqpoint{0.917537in}{0.945592in}}%
\pgfpathlineto{\pgfqpoint{0.921174in}{0.944520in}}%
\pgfpathlineto{\pgfqpoint{0.927757in}{0.938381in}}%
\pgfpathlineto{\pgfqpoint{0.934859in}{0.922409in}}%
\pgfpathlineto{\pgfqpoint{0.940922in}{0.912235in}}%
\pgfpathlineto{\pgfqpoint{0.942827in}{0.910634in}}%
\pgfpathlineto{\pgfqpoint{0.947157in}{0.905260in}}%
\pgfpathlineto{\pgfqpoint{0.948890in}{0.904667in}}%
\pgfpathlineto{\pgfqpoint{0.953393in}{0.890509in}}%
\pgfpathlineto{\pgfqpoint{0.958763in}{0.871717in}}%
\pgfpathlineto{\pgfqpoint{0.964133in}{0.862471in}}%
\pgfpathlineto{\pgfqpoint{0.972794in}{0.854228in}}%
\pgfpathlineto{\pgfqpoint{0.977817in}{0.846977in}}%
\pgfpathlineto{\pgfqpoint{0.989596in}{0.830863in}}%
\pgfpathlineto{\pgfqpoint{0.990463in}{0.832129in}}%
\pgfpathlineto{\pgfqpoint{0.992541in}{0.833862in}}%
\pgfpathlineto{\pgfqpoint{0.992714in}{0.833681in}}%
\pgfpathlineto{\pgfqpoint{0.994793in}{0.827637in}}%
\pgfpathlineto{\pgfqpoint{0.997911in}{0.817832in}}%
\pgfpathlineto{\pgfqpoint{1.001549in}{0.805929in}}%
\pgfpathlineto{\pgfqpoint{1.001895in}{0.806193in}}%
\pgfpathlineto{\pgfqpoint{1.008131in}{0.814218in}}%
\pgfpathlineto{\pgfqpoint{1.008304in}{0.814098in}}%
\pgfpathlineto{\pgfqpoint{1.012808in}{0.810950in}}%
\pgfpathlineto{\pgfqpoint{1.016446in}{0.808078in}}%
\pgfpathlineto{\pgfqpoint{1.019217in}{0.805814in}}%
\pgfpathlineto{\pgfqpoint{1.019390in}{0.805923in}}%
\pgfpathlineto{\pgfqpoint{1.020776in}{0.808459in}}%
\pgfpathlineto{\pgfqpoint{1.022855in}{0.811796in}}%
\pgfpathlineto{\pgfqpoint{1.023374in}{0.811317in}}%
\pgfpathlineto{\pgfqpoint{1.036193in}{0.789355in}}%
\pgfpathlineto{\pgfqpoint{1.037059in}{0.789723in}}%
\pgfpathlineto{\pgfqpoint{1.037405in}{0.790264in}}%
\pgfpathlineto{\pgfqpoint{1.037925in}{0.790664in}}%
\pgfpathlineto{\pgfqpoint{1.038618in}{0.789417in}}%
\pgfpathlineto{\pgfqpoint{1.043295in}{0.781286in}}%
\pgfpathlineto{\pgfqpoint{1.047279in}{0.779775in}}%
\pgfpathlineto{\pgfqpoint{1.050743in}{0.770196in}}%
\pgfpathlineto{\pgfqpoint{1.051436in}{0.770654in}}%
\pgfpathlineto{\pgfqpoint{1.052649in}{0.771876in}}%
\pgfpathlineto{\pgfqpoint{1.053342in}{0.770792in}}%
\pgfpathlineto{\pgfqpoint{1.056979in}{0.764940in}}%
\pgfpathlineto{\pgfqpoint{1.058018in}{0.764756in}}%
\pgfpathlineto{\pgfqpoint{1.058192in}{0.764331in}}%
\pgfpathlineto{\pgfqpoint{1.061656in}{0.751150in}}%
\pgfpathlineto{\pgfqpoint{1.063388in}{0.754163in}}%
\pgfpathlineto{\pgfqpoint{1.067372in}{0.763203in}}%
\pgfpathlineto{\pgfqpoint{1.067892in}{0.762380in}}%
\pgfpathlineto{\pgfqpoint{1.077246in}{0.738228in}}%
\pgfpathlineto{\pgfqpoint{1.078632in}{0.736983in}}%
\pgfpathlineto{\pgfqpoint{1.080884in}{0.734235in}}%
\pgfpathlineto{\pgfqpoint{1.081230in}{0.734655in}}%
\pgfpathlineto{\pgfqpoint{1.085041in}{0.741677in}}%
\pgfpathlineto{\pgfqpoint{1.085561in}{0.740897in}}%
\pgfpathlineto{\pgfqpoint{1.086773in}{0.737823in}}%
\pgfpathlineto{\pgfqpoint{1.087639in}{0.739176in}}%
\pgfpathlineto{\pgfqpoint{1.088505in}{0.740274in}}%
\pgfpathlineto{\pgfqpoint{1.089198in}{0.739337in}}%
\pgfpathlineto{\pgfqpoint{1.094048in}{0.733323in}}%
\pgfpathlineto{\pgfqpoint{1.095088in}{0.733087in}}%
\pgfpathlineto{\pgfqpoint{1.095434in}{0.733785in}}%
\pgfpathlineto{\pgfqpoint{1.101150in}{0.741951in}}%
\pgfpathlineto{\pgfqpoint{1.101324in}{0.741821in}}%
\pgfpathlineto{\pgfqpoint{1.103056in}{0.736948in}}%
\pgfpathlineto{\pgfqpoint{1.109638in}{0.717399in}}%
\pgfpathlineto{\pgfqpoint{1.109985in}{0.717839in}}%
\pgfpathlineto{\pgfqpoint{1.111890in}{0.725474in}}%
\pgfpathlineto{\pgfqpoint{1.113103in}{0.723071in}}%
\pgfpathlineto{\pgfqpoint{1.113969in}{0.724221in}}%
\pgfpathlineto{\pgfqpoint{1.115701in}{0.728365in}}%
\pgfpathlineto{\pgfqpoint{1.116221in}{0.726999in}}%
\pgfpathlineto{\pgfqpoint{1.117953in}{0.721160in}}%
\pgfpathlineto{\pgfqpoint{1.118646in}{0.721950in}}%
\pgfpathlineto{\pgfqpoint{1.119685in}{0.724024in}}%
\pgfpathlineto{\pgfqpoint{1.120378in}{0.723113in}}%
\pgfpathlineto{\pgfqpoint{1.130251in}{0.697216in}}%
\pgfpathlineto{\pgfqpoint{1.131291in}{0.696044in}}%
\pgfpathlineto{\pgfqpoint{1.131984in}{0.696591in}}%
\pgfpathlineto{\pgfqpoint{1.133543in}{0.697487in}}%
\pgfpathlineto{\pgfqpoint{1.133889in}{0.696985in}}%
\pgfpathlineto{\pgfqpoint{1.141511in}{0.682678in}}%
\pgfpathlineto{\pgfqpoint{1.143936in}{0.686167in}}%
\pgfpathlineto{\pgfqpoint{1.146014in}{0.693058in}}%
\pgfpathlineto{\pgfqpoint{1.146534in}{0.692554in}}%
\pgfpathlineto{\pgfqpoint{1.151038in}{0.687887in}}%
\pgfpathlineto{\pgfqpoint{1.152077in}{0.689389in}}%
\pgfpathlineto{\pgfqpoint{1.152770in}{0.690140in}}%
\pgfpathlineto{\pgfqpoint{1.153290in}{0.688904in}}%
\pgfpathlineto{\pgfqpoint{1.158486in}{0.663003in}}%
\pgfpathlineto{\pgfqpoint{1.159179in}{0.664667in}}%
\pgfpathlineto{\pgfqpoint{1.159872in}{0.665793in}}%
\pgfpathlineto{\pgfqpoint{1.160392in}{0.664572in}}%
\pgfpathlineto{\pgfqpoint{1.164895in}{0.650267in}}%
\pgfpathlineto{\pgfqpoint{1.165069in}{0.650306in}}%
\pgfpathlineto{\pgfqpoint{1.165935in}{0.650867in}}%
\pgfpathlineto{\pgfqpoint{1.166454in}{0.649617in}}%
\pgfpathlineto{\pgfqpoint{1.166801in}{0.649407in}}%
\pgfpathlineto{\pgfqpoint{1.167321in}{0.650667in}}%
\pgfpathlineto{\pgfqpoint{1.173210in}{0.675327in}}%
\pgfpathlineto{\pgfqpoint{1.174076in}{0.675424in}}%
\pgfpathlineto{\pgfqpoint{1.174423in}{0.674480in}}%
\pgfpathlineto{\pgfqpoint{1.175115in}{0.673265in}}%
\pgfpathlineto{\pgfqpoint{1.175635in}{0.675469in}}%
\pgfpathlineto{\pgfqpoint{1.177714in}{0.687507in}}%
\pgfpathlineto{\pgfqpoint{1.178753in}{0.686272in}}%
\pgfpathlineto{\pgfqpoint{1.179446in}{0.685918in}}%
\pgfpathlineto{\pgfqpoint{1.179966in}{0.687141in}}%
\pgfpathlineto{\pgfqpoint{1.181351in}{0.691844in}}%
\pgfpathlineto{\pgfqpoint{1.181871in}{0.689888in}}%
\pgfpathlineto{\pgfqpoint{1.183950in}{0.678531in}}%
\pgfpathlineto{\pgfqpoint{1.184816in}{0.679754in}}%
\pgfpathlineto{\pgfqpoint{1.189319in}{0.691612in}}%
\pgfpathlineto{\pgfqpoint{1.190359in}{0.694914in}}%
\pgfpathlineto{\pgfqpoint{1.190878in}{0.693362in}}%
\pgfpathlineto{\pgfqpoint{1.193130in}{0.688614in}}%
\pgfpathlineto{\pgfqpoint{1.196595in}{0.688393in}}%
\pgfpathlineto{\pgfqpoint{1.197288in}{0.688950in}}%
\pgfpathlineto{\pgfqpoint{1.197807in}{0.688080in}}%
\pgfpathlineto{\pgfqpoint{1.200406in}{0.676172in}}%
\pgfpathlineto{\pgfqpoint{1.203697in}{0.660835in}}%
\pgfpathlineto{\pgfqpoint{1.205256in}{0.662318in}}%
\pgfpathlineto{\pgfqpoint{1.208720in}{0.657762in}}%
\pgfpathlineto{\pgfqpoint{1.209240in}{0.658184in}}%
\pgfpathlineto{\pgfqpoint{1.210452in}{0.660063in}}%
\pgfpathlineto{\pgfqpoint{1.211145in}{0.660768in}}%
\pgfpathlineto{\pgfqpoint{1.211838in}{0.659817in}}%
\pgfpathlineto{\pgfqpoint{1.212877in}{0.658588in}}%
\pgfpathlineto{\pgfqpoint{1.213570in}{0.659600in}}%
\pgfpathlineto{\pgfqpoint{1.215822in}{0.662124in}}%
\pgfpathlineto{\pgfqpoint{1.217208in}{0.668646in}}%
\pgfpathlineto{\pgfqpoint{1.217901in}{0.666080in}}%
\pgfpathlineto{\pgfqpoint{1.218940in}{0.662395in}}%
\pgfpathlineto{\pgfqpoint{1.219806in}{0.663227in}}%
\pgfpathlineto{\pgfqpoint{1.220672in}{0.662780in}}%
\pgfpathlineto{\pgfqpoint{1.220846in}{0.662466in}}%
\pgfpathlineto{\pgfqpoint{1.222231in}{0.653771in}}%
\pgfpathlineto{\pgfqpoint{1.225523in}{0.632909in}}%
\pgfpathlineto{\pgfqpoint{1.226215in}{0.634264in}}%
\pgfpathlineto{\pgfqpoint{1.231585in}{0.654166in}}%
\pgfpathlineto{\pgfqpoint{1.234184in}{0.661269in}}%
\pgfpathlineto{\pgfqpoint{1.239207in}{0.682135in}}%
\pgfpathlineto{\pgfqpoint{1.239900in}{0.678858in}}%
\pgfpathlineto{\pgfqpoint{1.243711in}{0.650926in}}%
\pgfpathlineto{\pgfqpoint{1.244404in}{0.652848in}}%
\pgfpathlineto{\pgfqpoint{1.250813in}{0.682595in}}%
\pgfpathlineto{\pgfqpoint{1.251332in}{0.682465in}}%
\pgfpathlineto{\pgfqpoint{1.253065in}{0.680600in}}%
\pgfpathlineto{\pgfqpoint{1.254104in}{0.679613in}}%
\pgfpathlineto{\pgfqpoint{1.254624in}{0.680301in}}%
\pgfpathlineto{\pgfqpoint{1.255490in}{0.681419in}}%
\pgfpathlineto{\pgfqpoint{1.256356in}{0.680534in}}%
\pgfpathlineto{\pgfqpoint{1.258261in}{0.676408in}}%
\pgfpathlineto{\pgfqpoint{1.258954in}{0.678236in}}%
\pgfpathlineto{\pgfqpoint{1.263111in}{0.700413in}}%
\pgfpathlineto{\pgfqpoint{1.264151in}{0.697255in}}%
\pgfpathlineto{\pgfqpoint{1.267269in}{0.683502in}}%
\pgfpathlineto{\pgfqpoint{1.268308in}{0.685954in}}%
\pgfpathlineto{\pgfqpoint{1.270387in}{0.692076in}}%
\pgfpathlineto{\pgfqpoint{1.271079in}{0.691052in}}%
\pgfpathlineto{\pgfqpoint{1.272812in}{0.685065in}}%
\pgfpathlineto{\pgfqpoint{1.273505in}{0.687342in}}%
\pgfpathlineto{\pgfqpoint{1.275583in}{0.696022in}}%
\pgfpathlineto{\pgfqpoint{1.276276in}{0.694338in}}%
\pgfpathlineto{\pgfqpoint{1.281126in}{0.680281in}}%
\pgfpathlineto{\pgfqpoint{1.281819in}{0.681365in}}%
\pgfpathlineto{\pgfqpoint{1.284764in}{0.688348in}}%
\pgfpathlineto{\pgfqpoint{1.285630in}{0.685857in}}%
\pgfpathlineto{\pgfqpoint{1.288575in}{0.670605in}}%
\pgfpathlineto{\pgfqpoint{1.291346in}{0.659065in}}%
\pgfpathlineto{\pgfqpoint{1.294811in}{0.639789in}}%
\pgfpathlineto{\pgfqpoint{1.295504in}{0.641840in}}%
\pgfpathlineto{\pgfqpoint{1.299488in}{0.661489in}}%
\pgfpathlineto{\pgfqpoint{1.299834in}{0.660949in}}%
\pgfpathlineto{\pgfqpoint{1.300700in}{0.660101in}}%
\pgfpathlineto{\pgfqpoint{1.301393in}{0.661267in}}%
\pgfpathlineto{\pgfqpoint{1.304511in}{0.666864in}}%
\pgfpathlineto{\pgfqpoint{1.307975in}{0.663814in}}%
\pgfpathlineto{\pgfqpoint{1.308668in}{0.665712in}}%
\pgfpathlineto{\pgfqpoint{1.310574in}{0.672050in}}%
\pgfpathlineto{\pgfqpoint{1.311440in}{0.670628in}}%
\pgfpathlineto{\pgfqpoint{1.312306in}{0.669031in}}%
\pgfpathlineto{\pgfqpoint{1.312999in}{0.670397in}}%
\pgfpathlineto{\pgfqpoint{1.317676in}{0.688763in}}%
\pgfpathlineto{\pgfqpoint{1.318715in}{0.687295in}}%
\pgfpathlineto{\pgfqpoint{1.327376in}{0.670752in}}%
\pgfpathlineto{\pgfqpoint{1.330494in}{0.669811in}}%
\pgfpathlineto{\pgfqpoint{1.333785in}{0.672435in}}%
\pgfpathlineto{\pgfqpoint{1.334998in}{0.671360in}}%
\pgfpathlineto{\pgfqpoint{1.340194in}{0.663681in}}%
\pgfpathlineto{\pgfqpoint{1.340368in}{0.663924in}}%
\pgfpathlineto{\pgfqpoint{1.342100in}{0.667773in}}%
\pgfpathlineto{\pgfqpoint{1.343139in}{0.666376in}}%
\pgfpathlineto{\pgfqpoint{1.345564in}{0.663077in}}%
\pgfpathlineto{\pgfqpoint{1.347296in}{0.657228in}}%
\pgfpathlineto{\pgfqpoint{1.347989in}{0.659678in}}%
\pgfpathlineto{\pgfqpoint{1.349029in}{0.662149in}}%
\pgfpathlineto{\pgfqpoint{1.349548in}{0.660928in}}%
\pgfpathlineto{\pgfqpoint{1.351973in}{0.652052in}}%
\pgfpathlineto{\pgfqpoint{1.353359in}{0.653220in}}%
\pgfpathlineto{\pgfqpoint{1.356997in}{0.654351in}}%
\pgfpathlineto{\pgfqpoint{1.360634in}{0.663712in}}%
\pgfpathlineto{\pgfqpoint{1.361327in}{0.663116in}}%
\pgfpathlineto{\pgfqpoint{1.361500in}{0.662609in}}%
\pgfpathlineto{\pgfqpoint{1.362193in}{0.660951in}}%
\pgfpathlineto{\pgfqpoint{1.363059in}{0.662703in}}%
\pgfpathlineto{\pgfqpoint{1.363752in}{0.661061in}}%
\pgfpathlineto{\pgfqpoint{1.366524in}{0.642774in}}%
\pgfpathlineto{\pgfqpoint{1.367736in}{0.644318in}}%
\pgfpathlineto{\pgfqpoint{1.369295in}{0.650070in}}%
\pgfpathlineto{\pgfqpoint{1.374665in}{0.675802in}}%
\pgfpathlineto{\pgfqpoint{1.375012in}{0.675754in}}%
\pgfpathlineto{\pgfqpoint{1.375358in}{0.674961in}}%
\pgfpathlineto{\pgfqpoint{1.380208in}{0.656015in}}%
\pgfpathlineto{\pgfqpoint{1.381074in}{0.657349in}}%
\pgfpathlineto{\pgfqpoint{1.383153in}{0.661630in}}%
\pgfpathlineto{\pgfqpoint{1.383846in}{0.660427in}}%
\pgfpathlineto{\pgfqpoint{1.385058in}{0.655000in}}%
\pgfpathlineto{\pgfqpoint{1.386271in}{0.649522in}}%
\pgfpathlineto{\pgfqpoint{1.386964in}{0.651813in}}%
\pgfpathlineto{\pgfqpoint{1.391121in}{0.667557in}}%
\pgfpathlineto{\pgfqpoint{1.391814in}{0.668232in}}%
\pgfpathlineto{\pgfqpoint{1.392334in}{0.667302in}}%
\pgfpathlineto{\pgfqpoint{1.393893in}{0.662673in}}%
\pgfpathlineto{\pgfqpoint{1.394932in}{0.663765in}}%
\pgfpathlineto{\pgfqpoint{1.395971in}{0.660194in}}%
\pgfpathlineto{\pgfqpoint{1.396491in}{0.658877in}}%
\pgfpathlineto{\pgfqpoint{1.397184in}{0.661585in}}%
\pgfpathlineto{\pgfqpoint{1.401688in}{0.691757in}}%
\pgfpathlineto{\pgfqpoint{1.402207in}{0.691064in}}%
\pgfpathlineto{\pgfqpoint{1.404286in}{0.685876in}}%
\pgfpathlineto{\pgfqpoint{1.404979in}{0.687567in}}%
\pgfpathlineto{\pgfqpoint{1.406884in}{0.693056in}}%
\pgfpathlineto{\pgfqpoint{1.407577in}{0.692080in}}%
\pgfpathlineto{\pgfqpoint{1.409483in}{0.686839in}}%
\pgfpathlineto{\pgfqpoint{1.410522in}{0.688133in}}%
\pgfpathlineto{\pgfqpoint{1.412254in}{0.690252in}}%
\pgfpathlineto{\pgfqpoint{1.412600in}{0.689449in}}%
\pgfpathlineto{\pgfqpoint{1.417104in}{0.677145in}}%
\pgfpathlineto{\pgfqpoint{1.417451in}{0.677712in}}%
\pgfpathlineto{\pgfqpoint{1.418836in}{0.681494in}}%
\pgfpathlineto{\pgfqpoint{1.419703in}{0.679567in}}%
\pgfpathlineto{\pgfqpoint{1.421954in}{0.672470in}}%
\pgfpathlineto{\pgfqpoint{1.423167in}{0.674874in}}%
\pgfpathlineto{\pgfqpoint{1.424033in}{0.675622in}}%
\pgfpathlineto{\pgfqpoint{1.424553in}{0.674594in}}%
\pgfpathlineto{\pgfqpoint{1.429923in}{0.660869in}}%
\pgfpathlineto{\pgfqpoint{1.431308in}{0.662097in}}%
\pgfpathlineto{\pgfqpoint{1.434080in}{0.673684in}}%
\pgfpathlineto{\pgfqpoint{1.435292in}{0.683513in}}%
\pgfpathlineto{\pgfqpoint{1.437198in}{0.746668in}}%
\pgfpathlineto{\pgfqpoint{1.440489in}{0.818550in}}%
\pgfpathlineto{\pgfqpoint{1.446552in}{0.913603in}}%
\pgfpathlineto{\pgfqpoint{1.448804in}{0.918498in}}%
\pgfpathlineto{\pgfqpoint{1.451402in}{0.919508in}}%
\pgfpathlineto{\pgfqpoint{1.452961in}{0.933477in}}%
\pgfpathlineto{\pgfqpoint{1.454866in}{0.944983in}}%
\pgfpathlineto{\pgfqpoint{1.455386in}{0.944660in}}%
\pgfpathlineto{\pgfqpoint{1.463700in}{0.929626in}}%
\pgfpathlineto{\pgfqpoint{1.469763in}{0.893568in}}%
\pgfpathlineto{\pgfqpoint{1.480849in}{0.800357in}}%
\pgfpathlineto{\pgfqpoint{1.494360in}{0.702799in}}%
\pgfpathlineto{\pgfqpoint{1.501982in}{0.667348in}}%
\pgfpathlineto{\pgfqpoint{1.503368in}{0.665218in}}%
\pgfpathlineto{\pgfqpoint{1.506659in}{0.660822in}}%
\pgfpathlineto{\pgfqpoint{1.507006in}{0.661127in}}%
\pgfpathlineto{\pgfqpoint{1.512722in}{0.675066in}}%
\pgfpathlineto{\pgfqpoint{1.514454in}{0.678216in}}%
\pgfpathlineto{\pgfqpoint{1.514974in}{0.677620in}}%
\pgfpathlineto{\pgfqpoint{1.515493in}{0.677367in}}%
\pgfpathlineto{\pgfqpoint{1.516013in}{0.678761in}}%
\pgfpathlineto{\pgfqpoint{1.520170in}{0.703849in}}%
\pgfpathlineto{\pgfqpoint{1.521556in}{0.701082in}}%
\pgfpathlineto{\pgfqpoint{1.523808in}{0.687900in}}%
\pgfpathlineto{\pgfqpoint{1.527272in}{0.662048in}}%
\pgfpathlineto{\pgfqpoint{1.528312in}{0.665485in}}%
\pgfpathlineto{\pgfqpoint{1.529871in}{0.670384in}}%
\pgfpathlineto{\pgfqpoint{1.530564in}{0.669441in}}%
\pgfpathlineto{\pgfqpoint{1.534374in}{0.659973in}}%
\pgfpathlineto{\pgfqpoint{1.535240in}{0.661973in}}%
\pgfpathlineto{\pgfqpoint{1.535760in}{0.662777in}}%
\pgfpathlineto{\pgfqpoint{1.536453in}{0.661446in}}%
\pgfpathlineto{\pgfqpoint{1.542169in}{0.643899in}}%
\pgfpathlineto{\pgfqpoint{1.542516in}{0.644117in}}%
\pgfpathlineto{\pgfqpoint{1.543901in}{0.647789in}}%
\pgfpathlineto{\pgfqpoint{1.545287in}{0.651528in}}%
\pgfpathlineto{\pgfqpoint{1.545980in}{0.650376in}}%
\pgfpathlineto{\pgfqpoint{1.548925in}{0.639208in}}%
\pgfpathlineto{\pgfqpoint{1.554121in}{0.616484in}}%
\pgfpathlineto{\pgfqpoint{1.554468in}{0.615947in}}%
\pgfpathlineto{\pgfqpoint{1.555161in}{0.617172in}}%
\pgfpathlineto{\pgfqpoint{1.558279in}{0.621265in}}%
\pgfpathlineto{\pgfqpoint{1.559838in}{0.620904in}}%
\pgfpathlineto{\pgfqpoint{1.560184in}{0.621750in}}%
\pgfpathlineto{\pgfqpoint{1.567113in}{0.640336in}}%
\pgfpathlineto{\pgfqpoint{1.571790in}{0.645833in}}%
\pgfpathlineto{\pgfqpoint{1.574561in}{0.647147in}}%
\pgfpathlineto{\pgfqpoint{1.577333in}{0.657018in}}%
\pgfpathlineto{\pgfqpoint{1.578372in}{0.654567in}}%
\pgfpathlineto{\pgfqpoint{1.589112in}{0.634156in}}%
\pgfpathlineto{\pgfqpoint{1.589632in}{0.634455in}}%
\pgfpathlineto{\pgfqpoint{1.591017in}{0.632926in}}%
\pgfpathlineto{\pgfqpoint{1.592923in}{0.626588in}}%
\pgfpathlineto{\pgfqpoint{1.593789in}{0.628755in}}%
\pgfpathlineto{\pgfqpoint{1.596387in}{0.635068in}}%
\pgfpathlineto{\pgfqpoint{1.596907in}{0.634280in}}%
\pgfpathlineto{\pgfqpoint{1.603662in}{0.606507in}}%
\pgfpathlineto{\pgfqpoint{1.604875in}{0.609123in}}%
\pgfpathlineto{\pgfqpoint{1.607993in}{0.613310in}}%
\pgfpathlineto{\pgfqpoint{1.608339in}{0.612811in}}%
\pgfpathlineto{\pgfqpoint{1.609206in}{0.612018in}}%
\pgfpathlineto{\pgfqpoint{1.609725in}{0.612946in}}%
\pgfpathlineto{\pgfqpoint{1.613363in}{0.622514in}}%
\pgfpathlineto{\pgfqpoint{1.613709in}{0.622203in}}%
\pgfpathlineto{\pgfqpoint{1.614922in}{0.616345in}}%
\pgfpathlineto{\pgfqpoint{1.621158in}{0.589367in}}%
\pgfpathlineto{\pgfqpoint{1.622370in}{0.590429in}}%
\pgfpathlineto{\pgfqpoint{1.622890in}{0.591022in}}%
\pgfpathlineto{\pgfqpoint{1.623410in}{0.589397in}}%
\pgfpathlineto{\pgfqpoint{1.626874in}{0.578816in}}%
\pgfpathlineto{\pgfqpoint{1.627047in}{0.578855in}}%
\pgfpathlineto{\pgfqpoint{1.627913in}{0.580984in}}%
\pgfpathlineto{\pgfqpoint{1.632590in}{0.602256in}}%
\pgfpathlineto{\pgfqpoint{1.633630in}{0.601939in}}%
\pgfpathlineto{\pgfqpoint{1.634496in}{0.603319in}}%
\pgfpathlineto{\pgfqpoint{1.637787in}{0.609510in}}%
\pgfpathlineto{\pgfqpoint{1.638133in}{0.609641in}}%
\pgfpathlineto{\pgfqpoint{1.638480in}{0.608671in}}%
\pgfpathlineto{\pgfqpoint{1.644023in}{0.587880in}}%
\pgfpathlineto{\pgfqpoint{1.645062in}{0.586788in}}%
\pgfpathlineto{\pgfqpoint{1.646275in}{0.583977in}}%
\pgfpathlineto{\pgfqpoint{1.647141in}{0.585099in}}%
\pgfpathlineto{\pgfqpoint{1.648700in}{0.587888in}}%
\pgfpathlineto{\pgfqpoint{1.649219in}{0.586743in}}%
\pgfpathlineto{\pgfqpoint{1.651298in}{0.581925in}}%
\pgfpathlineto{\pgfqpoint{1.651818in}{0.583072in}}%
\pgfpathlineto{\pgfqpoint{1.654243in}{0.591236in}}%
\pgfpathlineto{\pgfqpoint{1.655282in}{0.590298in}}%
\pgfpathlineto{\pgfqpoint{1.659786in}{0.586450in}}%
\pgfpathlineto{\pgfqpoint{1.660479in}{0.587882in}}%
\pgfpathlineto{\pgfqpoint{1.663250in}{0.590311in}}%
\pgfpathlineto{\pgfqpoint{1.663423in}{0.590206in}}%
\pgfpathlineto{\pgfqpoint{1.664463in}{0.589875in}}%
\pgfpathlineto{\pgfqpoint{1.664809in}{0.590744in}}%
\pgfpathlineto{\pgfqpoint{1.665849in}{0.593358in}}%
\pgfpathlineto{\pgfqpoint{1.666541in}{0.591462in}}%
\pgfpathlineto{\pgfqpoint{1.668447in}{0.583515in}}%
\pgfpathlineto{\pgfqpoint{1.669313in}{0.584133in}}%
\pgfpathlineto{\pgfqpoint{1.670699in}{0.579105in}}%
\pgfpathlineto{\pgfqpoint{1.671565in}{0.582840in}}%
\pgfpathlineto{\pgfqpoint{1.674336in}{0.593514in}}%
\pgfpathlineto{\pgfqpoint{1.674510in}{0.593458in}}%
\pgfpathlineto{\pgfqpoint{1.676069in}{0.591050in}}%
\pgfpathlineto{\pgfqpoint{1.676588in}{0.592472in}}%
\pgfpathlineto{\pgfqpoint{1.679187in}{0.602767in}}%
\pgfpathlineto{\pgfqpoint{1.680053in}{0.600780in}}%
\pgfpathlineto{\pgfqpoint{1.683863in}{0.585626in}}%
\pgfpathlineto{\pgfqpoint{1.684556in}{0.587006in}}%
\pgfpathlineto{\pgfqpoint{1.686635in}{0.606151in}}%
\pgfpathlineto{\pgfqpoint{1.688194in}{0.615736in}}%
\pgfpathlineto{\pgfqpoint{1.688887in}{0.614112in}}%
\pgfpathlineto{\pgfqpoint{1.689407in}{0.612958in}}%
\pgfpathlineto{\pgfqpoint{1.690099in}{0.614496in}}%
\pgfpathlineto{\pgfqpoint{1.691658in}{0.619201in}}%
\pgfpathlineto{\pgfqpoint{1.692351in}{0.618665in}}%
\pgfpathlineto{\pgfqpoint{1.693737in}{0.613959in}}%
\pgfpathlineto{\pgfqpoint{1.697028in}{0.592288in}}%
\pgfpathlineto{\pgfqpoint{1.697721in}{0.594700in}}%
\pgfpathlineto{\pgfqpoint{1.699627in}{0.606003in}}%
\pgfpathlineto{\pgfqpoint{1.700319in}{0.603895in}}%
\pgfpathlineto{\pgfqpoint{1.707248in}{0.576106in}}%
\pgfpathlineto{\pgfqpoint{1.709327in}{0.576983in}}%
\pgfpathlineto{\pgfqpoint{1.713657in}{0.588741in}}%
\pgfpathlineto{\pgfqpoint{1.714177in}{0.586799in}}%
\pgfpathlineto{\pgfqpoint{1.715563in}{0.581173in}}%
\pgfpathlineto{\pgfqpoint{1.716256in}{0.582663in}}%
\pgfpathlineto{\pgfqpoint{1.717122in}{0.583709in}}%
\pgfpathlineto{\pgfqpoint{1.717815in}{0.582927in}}%
\pgfpathlineto{\pgfqpoint{1.720240in}{0.581676in}}%
\pgfpathlineto{\pgfqpoint{1.720586in}{0.582103in}}%
\pgfpathlineto{\pgfqpoint{1.721452in}{0.583384in}}%
\pgfpathlineto{\pgfqpoint{1.722145in}{0.581947in}}%
\pgfpathlineto{\pgfqpoint{1.724917in}{0.575414in}}%
\pgfpathlineto{\pgfqpoint{1.725436in}{0.576236in}}%
\pgfpathlineto{\pgfqpoint{1.729940in}{0.593556in}}%
\pgfpathlineto{\pgfqpoint{1.731153in}{0.592182in}}%
\pgfpathlineto{\pgfqpoint{1.732019in}{0.593749in}}%
\pgfpathlineto{\pgfqpoint{1.732538in}{0.592396in}}%
\pgfpathlineto{\pgfqpoint{1.734963in}{0.580983in}}%
\pgfpathlineto{\pgfqpoint{1.735830in}{0.582586in}}%
\pgfpathlineto{\pgfqpoint{1.741892in}{0.599071in}}%
\pgfpathlineto{\pgfqpoint{1.743278in}{0.601217in}}%
\pgfpathlineto{\pgfqpoint{1.744144in}{0.601889in}}%
\pgfpathlineto{\pgfqpoint{1.744664in}{0.601065in}}%
\pgfpathlineto{\pgfqpoint{1.745703in}{0.599705in}}%
\pgfpathlineto{\pgfqpoint{1.746223in}{0.600539in}}%
\pgfpathlineto{\pgfqpoint{1.750553in}{0.607256in}}%
\pgfpathlineto{\pgfqpoint{1.751593in}{0.605045in}}%
\pgfpathlineto{\pgfqpoint{1.754191in}{0.593586in}}%
\pgfpathlineto{\pgfqpoint{1.755230in}{0.594968in}}%
\pgfpathlineto{\pgfqpoint{1.755750in}{0.595219in}}%
\pgfpathlineto{\pgfqpoint{1.756443in}{0.594050in}}%
\pgfpathlineto{\pgfqpoint{1.756962in}{0.593865in}}%
\pgfpathlineto{\pgfqpoint{1.757309in}{0.594647in}}%
\pgfpathlineto{\pgfqpoint{1.761813in}{0.607644in}}%
\pgfpathlineto{\pgfqpoint{1.762159in}{0.607325in}}%
\pgfpathlineto{\pgfqpoint{1.763372in}{0.600474in}}%
\pgfpathlineto{\pgfqpoint{1.767182in}{0.582804in}}%
\pgfpathlineto{\pgfqpoint{1.767529in}{0.583103in}}%
\pgfpathlineto{\pgfqpoint{1.768049in}{0.583559in}}%
\pgfpathlineto{\pgfqpoint{1.768741in}{0.582658in}}%
\pgfpathlineto{\pgfqpoint{1.774284in}{0.576003in}}%
\pgfpathlineto{\pgfqpoint{1.774804in}{0.576892in}}%
\pgfpathlineto{\pgfqpoint{1.778095in}{0.592887in}}%
\pgfpathlineto{\pgfqpoint{1.779135in}{0.589937in}}%
\pgfpathlineto{\pgfqpoint{1.780520in}{0.584682in}}%
\pgfpathlineto{\pgfqpoint{1.781040in}{0.585971in}}%
\pgfpathlineto{\pgfqpoint{1.782599in}{0.590096in}}%
\pgfpathlineto{\pgfqpoint{1.783119in}{0.589162in}}%
\pgfpathlineto{\pgfqpoint{1.786930in}{0.579247in}}%
\pgfpathlineto{\pgfqpoint{1.787449in}{0.579886in}}%
\pgfpathlineto{\pgfqpoint{1.788489in}{0.580896in}}%
\pgfpathlineto{\pgfqpoint{1.789181in}{0.579941in}}%
\pgfpathlineto{\pgfqpoint{1.789701in}{0.579551in}}%
\pgfpathlineto{\pgfqpoint{1.790567in}{0.580628in}}%
\pgfpathlineto{\pgfqpoint{1.793339in}{0.583273in}}%
\pgfpathlineto{\pgfqpoint{1.794378in}{0.581060in}}%
\pgfpathlineto{\pgfqpoint{1.796283in}{0.571834in}}%
\pgfpathlineto{\pgfqpoint{1.797150in}{0.575135in}}%
\pgfpathlineto{\pgfqpoint{1.798882in}{0.583061in}}%
\pgfpathlineto{\pgfqpoint{1.799575in}{0.581975in}}%
\pgfpathlineto{\pgfqpoint{1.800614in}{0.578468in}}%
\pgfpathlineto{\pgfqpoint{1.801307in}{0.580219in}}%
\pgfpathlineto{\pgfqpoint{1.802519in}{0.585358in}}%
\pgfpathlineto{\pgfqpoint{1.803212in}{0.583365in}}%
\pgfpathlineto{\pgfqpoint{1.804425in}{0.580624in}}%
\pgfpathlineto{\pgfqpoint{1.804944in}{0.581674in}}%
\pgfpathlineto{\pgfqpoint{1.807716in}{0.588004in}}%
\pgfpathlineto{\pgfqpoint{1.808755in}{0.586806in}}%
\pgfpathlineto{\pgfqpoint{1.809275in}{0.586853in}}%
\pgfpathlineto{\pgfqpoint{1.809795in}{0.587786in}}%
\pgfpathlineto{\pgfqpoint{1.810314in}{0.587453in}}%
\pgfpathlineto{\pgfqpoint{1.810661in}{0.586590in}}%
\pgfpathlineto{\pgfqpoint{1.811527in}{0.585714in}}%
\pgfpathlineto{\pgfqpoint{1.812047in}{0.587036in}}%
\pgfpathlineto{\pgfqpoint{1.813086in}{0.588813in}}%
\pgfpathlineto{\pgfqpoint{1.813952in}{0.588361in}}%
\pgfpathlineto{\pgfqpoint{1.815511in}{0.590403in}}%
\pgfpathlineto{\pgfqpoint{1.818629in}{0.598474in}}%
\pgfpathlineto{\pgfqpoint{1.821227in}{0.603745in}}%
\pgfpathlineto{\pgfqpoint{1.824345in}{0.612186in}}%
\pgfpathlineto{\pgfqpoint{1.824865in}{0.611432in}}%
\pgfpathlineto{\pgfqpoint{1.825558in}{0.610689in}}%
\pgfpathlineto{\pgfqpoint{1.826251in}{0.611743in}}%
\pgfpathlineto{\pgfqpoint{1.828502in}{0.617508in}}%
\pgfpathlineto{\pgfqpoint{1.829369in}{0.615598in}}%
\pgfpathlineto{\pgfqpoint{1.838376in}{0.587569in}}%
\pgfpathlineto{\pgfqpoint{1.838722in}{0.588344in}}%
\pgfpathlineto{\pgfqpoint{1.840108in}{0.591418in}}%
\pgfpathlineto{\pgfqpoint{1.840801in}{0.590177in}}%
\pgfpathlineto{\pgfqpoint{1.843053in}{0.583516in}}%
\pgfpathlineto{\pgfqpoint{1.844958in}{0.568480in}}%
\pgfpathlineto{\pgfqpoint{1.845651in}{0.569080in}}%
\pgfpathlineto{\pgfqpoint{1.849289in}{0.581702in}}%
\pgfpathlineto{\pgfqpoint{1.851194in}{0.589135in}}%
\pgfpathlineto{\pgfqpoint{1.851714in}{0.587196in}}%
\pgfpathlineto{\pgfqpoint{1.856044in}{0.554314in}}%
\pgfpathlineto{\pgfqpoint{1.857603in}{0.558021in}}%
\pgfpathlineto{\pgfqpoint{1.859855in}{0.559349in}}%
\pgfpathlineto{\pgfqpoint{1.863493in}{0.563850in}}%
\pgfpathlineto{\pgfqpoint{1.870595in}{0.583976in}}%
\pgfpathlineto{\pgfqpoint{1.871461in}{0.582929in}}%
\pgfpathlineto{\pgfqpoint{1.875445in}{0.571945in}}%
\pgfpathlineto{\pgfqpoint{1.875965in}{0.573228in}}%
\pgfpathlineto{\pgfqpoint{1.877004in}{0.574268in}}%
\pgfpathlineto{\pgfqpoint{1.877697in}{0.573325in}}%
\pgfpathlineto{\pgfqpoint{1.878043in}{0.573290in}}%
\pgfpathlineto{\pgfqpoint{1.878390in}{0.574226in}}%
\pgfpathlineto{\pgfqpoint{1.880988in}{0.583575in}}%
\pgfpathlineto{\pgfqpoint{1.881681in}{0.581318in}}%
\pgfpathlineto{\pgfqpoint{1.887051in}{0.559544in}}%
\pgfpathlineto{\pgfqpoint{1.887397in}{0.559833in}}%
\pgfpathlineto{\pgfqpoint{1.889476in}{0.566220in}}%
\pgfpathlineto{\pgfqpoint{1.890862in}{0.564502in}}%
\pgfpathlineto{\pgfqpoint{1.892940in}{0.563038in}}%
\pgfpathlineto{\pgfqpoint{1.893114in}{0.563288in}}%
\pgfpathlineto{\pgfqpoint{1.894846in}{0.568798in}}%
\pgfpathlineto{\pgfqpoint{1.897098in}{0.578096in}}%
\pgfpathlineto{\pgfqpoint{1.897791in}{0.577868in}}%
\pgfpathlineto{\pgfqpoint{1.898830in}{0.575346in}}%
\pgfpathlineto{\pgfqpoint{1.903334in}{0.560957in}}%
\pgfpathlineto{\pgfqpoint{1.905759in}{0.559596in}}%
\pgfpathlineto{\pgfqpoint{1.906971in}{0.565583in}}%
\pgfpathlineto{\pgfqpoint{1.910955in}{0.588029in}}%
\pgfpathlineto{\pgfqpoint{1.911995in}{0.590965in}}%
\pgfpathlineto{\pgfqpoint{1.912688in}{0.589850in}}%
\pgfpathlineto{\pgfqpoint{1.914073in}{0.587288in}}%
\pgfpathlineto{\pgfqpoint{1.914593in}{0.588178in}}%
\pgfpathlineto{\pgfqpoint{1.917191in}{0.590011in}}%
\pgfpathlineto{\pgfqpoint{1.918057in}{0.591624in}}%
\pgfpathlineto{\pgfqpoint{1.919097in}{0.593660in}}%
\pgfpathlineto{\pgfqpoint{1.919790in}{0.592242in}}%
\pgfpathlineto{\pgfqpoint{1.920656in}{0.590375in}}%
\pgfpathlineto{\pgfqpoint{1.921175in}{0.592607in}}%
\pgfpathlineto{\pgfqpoint{1.924293in}{0.610985in}}%
\pgfpathlineto{\pgfqpoint{1.924986in}{0.608807in}}%
\pgfpathlineto{\pgfqpoint{1.933474in}{0.572683in}}%
\pgfpathlineto{\pgfqpoint{1.933994in}{0.571974in}}%
\pgfpathlineto{\pgfqpoint{1.934686in}{0.573076in}}%
\pgfpathlineto{\pgfqpoint{1.937804in}{0.583360in}}%
\pgfpathlineto{\pgfqpoint{1.938497in}{0.580950in}}%
\pgfpathlineto{\pgfqpoint{1.943174in}{0.560253in}}%
\pgfpathlineto{\pgfqpoint{1.945426in}{0.560809in}}%
\pgfpathlineto{\pgfqpoint{1.945599in}{0.560511in}}%
\pgfpathlineto{\pgfqpoint{1.946812in}{0.553119in}}%
\pgfpathlineto{\pgfqpoint{1.948198in}{0.546071in}}%
\pgfpathlineto{\pgfqpoint{1.948891in}{0.548729in}}%
\pgfpathlineto{\pgfqpoint{1.954087in}{0.570979in}}%
\pgfpathlineto{\pgfqpoint{1.955126in}{0.575474in}}%
\pgfpathlineto{\pgfqpoint{1.957725in}{0.587225in}}%
\pgfpathlineto{\pgfqpoint{1.958244in}{0.587129in}}%
\pgfpathlineto{\pgfqpoint{1.959803in}{0.585284in}}%
\pgfpathlineto{\pgfqpoint{1.960496in}{0.584112in}}%
\pgfpathlineto{\pgfqpoint{1.961016in}{0.585450in}}%
\pgfpathlineto{\pgfqpoint{1.967425in}{0.613073in}}%
\pgfpathlineto{\pgfqpoint{1.967945in}{0.611976in}}%
\pgfpathlineto{\pgfqpoint{1.971929in}{0.601228in}}%
\pgfpathlineto{\pgfqpoint{1.974181in}{0.600087in}}%
\pgfpathlineto{\pgfqpoint{1.977299in}{0.582293in}}%
\pgfpathlineto{\pgfqpoint{1.979724in}{0.567715in}}%
\pgfpathlineto{\pgfqpoint{1.980936in}{0.568331in}}%
\pgfpathlineto{\pgfqpoint{1.983708in}{0.575570in}}%
\pgfpathlineto{\pgfqpoint{1.984920in}{0.579174in}}%
\pgfpathlineto{\pgfqpoint{1.985786in}{0.578109in}}%
\pgfpathlineto{\pgfqpoint{1.986999in}{0.576382in}}%
\pgfpathlineto{\pgfqpoint{1.987519in}{0.576975in}}%
\pgfpathlineto{\pgfqpoint{1.988904in}{0.579355in}}%
\pgfpathlineto{\pgfqpoint{1.989597in}{0.578421in}}%
\pgfpathlineto{\pgfqpoint{1.991156in}{0.577653in}}%
\pgfpathlineto{\pgfqpoint{1.991503in}{0.578174in}}%
\pgfpathlineto{\pgfqpoint{1.993408in}{0.584802in}}%
\pgfpathlineto{\pgfqpoint{1.994794in}{0.581791in}}%
\pgfpathlineto{\pgfqpoint{1.996873in}{0.578438in}}%
\pgfpathlineto{\pgfqpoint{1.997392in}{0.579295in}}%
\pgfpathlineto{\pgfqpoint{1.999298in}{0.585776in}}%
\pgfpathlineto{\pgfqpoint{2.000164in}{0.584103in}}%
\pgfpathlineto{\pgfqpoint{2.004494in}{0.577548in}}%
\pgfpathlineto{\pgfqpoint{2.006400in}{0.576497in}}%
\pgfpathlineto{\pgfqpoint{2.008305in}{0.575227in}}%
\pgfpathlineto{\pgfqpoint{2.008478in}{0.575326in}}%
\pgfpathlineto{\pgfqpoint{2.011596in}{0.579000in}}%
\pgfpathlineto{\pgfqpoint{2.012636in}{0.578129in}}%
\pgfpathlineto{\pgfqpoint{2.014195in}{0.577985in}}%
\pgfpathlineto{\pgfqpoint{2.014368in}{0.577370in}}%
\pgfpathlineto{\pgfqpoint{2.019045in}{0.550751in}}%
\pgfpathlineto{\pgfqpoint{2.020431in}{0.542005in}}%
\pgfpathlineto{\pgfqpoint{2.021123in}{0.544890in}}%
\pgfpathlineto{\pgfqpoint{2.022163in}{0.548176in}}%
\pgfpathlineto{\pgfqpoint{2.022856in}{0.546294in}}%
\pgfpathlineto{\pgfqpoint{2.024415in}{0.543469in}}%
\pgfpathlineto{\pgfqpoint{2.024761in}{0.544158in}}%
\pgfpathlineto{\pgfqpoint{2.025627in}{0.545334in}}%
\pgfpathlineto{\pgfqpoint{2.026320in}{0.544244in}}%
\pgfpathlineto{\pgfqpoint{2.026840in}{0.543681in}}%
\pgfpathlineto{\pgfqpoint{2.027359in}{0.545075in}}%
\pgfpathlineto{\pgfqpoint{2.029438in}{0.562210in}}%
\pgfpathlineto{\pgfqpoint{2.030651in}{0.555964in}}%
\pgfpathlineto{\pgfqpoint{2.033422in}{0.543614in}}%
\pgfpathlineto{\pgfqpoint{2.033769in}{0.544420in}}%
\pgfpathlineto{\pgfqpoint{2.034461in}{0.545916in}}%
\pgfpathlineto{\pgfqpoint{2.035327in}{0.545131in}}%
\pgfpathlineto{\pgfqpoint{2.036367in}{0.543417in}}%
\pgfpathlineto{\pgfqpoint{2.037060in}{0.544493in}}%
\pgfpathlineto{\pgfqpoint{2.040351in}{0.550622in}}%
\pgfpathlineto{\pgfqpoint{2.040697in}{0.550089in}}%
\pgfpathlineto{\pgfqpoint{2.042256in}{0.545837in}}%
\pgfpathlineto{\pgfqpoint{2.043122in}{0.548024in}}%
\pgfpathlineto{\pgfqpoint{2.047280in}{0.560039in}}%
\pgfpathlineto{\pgfqpoint{2.047453in}{0.559882in}}%
\pgfpathlineto{\pgfqpoint{2.049185in}{0.554809in}}%
\pgfpathlineto{\pgfqpoint{2.049878in}{0.557266in}}%
\pgfpathlineto{\pgfqpoint{2.052476in}{0.563082in}}%
\pgfpathlineto{\pgfqpoint{2.055941in}{0.569977in}}%
\pgfpathlineto{\pgfqpoint{2.057500in}{0.574276in}}%
\pgfpathlineto{\pgfqpoint{2.058366in}{0.573061in}}%
\pgfpathlineto{\pgfqpoint{2.061657in}{0.567654in}}%
\pgfpathlineto{\pgfqpoint{2.062003in}{0.568624in}}%
\pgfpathlineto{\pgfqpoint{2.062870in}{0.570656in}}%
\pgfpathlineto{\pgfqpoint{2.063389in}{0.569268in}}%
\pgfpathlineto{\pgfqpoint{2.066161in}{0.550791in}}%
\pgfpathlineto{\pgfqpoint{2.067373in}{0.553254in}}%
\pgfpathlineto{\pgfqpoint{2.069279in}{0.556045in}}%
\pgfpathlineto{\pgfqpoint{2.070145in}{0.557600in}}%
\pgfpathlineto{\pgfqpoint{2.070664in}{0.556516in}}%
\pgfpathlineto{\pgfqpoint{2.073782in}{0.541330in}}%
\pgfpathlineto{\pgfqpoint{2.074822in}{0.545314in}}%
\pgfpathlineto{\pgfqpoint{2.077593in}{0.567382in}}%
\pgfpathlineto{\pgfqpoint{2.079152in}{0.561457in}}%
\pgfpathlineto{\pgfqpoint{2.079845in}{0.560407in}}%
\pgfpathlineto{\pgfqpoint{2.080711in}{0.561156in}}%
\pgfpathlineto{\pgfqpoint{2.083136in}{0.565786in}}%
\pgfpathlineto{\pgfqpoint{2.084002in}{0.564387in}}%
\pgfpathlineto{\pgfqpoint{2.089372in}{0.556569in}}%
\pgfpathlineto{\pgfqpoint{2.090758in}{0.560780in}}%
\pgfpathlineto{\pgfqpoint{2.093356in}{0.570065in}}%
\pgfpathlineto{\pgfqpoint{2.093530in}{0.570025in}}%
\pgfpathlineto{\pgfqpoint{2.094742in}{0.567742in}}%
\pgfpathlineto{\pgfqpoint{2.096301in}{0.561336in}}%
\pgfpathlineto{\pgfqpoint{2.096994in}{0.563071in}}%
\pgfpathlineto{\pgfqpoint{2.102364in}{0.579357in}}%
\pgfpathlineto{\pgfqpoint{2.103057in}{0.579109in}}%
\pgfpathlineto{\pgfqpoint{2.103403in}{0.578383in}}%
\pgfpathlineto{\pgfqpoint{2.106001in}{0.572662in}}%
\pgfpathlineto{\pgfqpoint{2.106694in}{0.573976in}}%
\pgfpathlineto{\pgfqpoint{2.108253in}{0.580961in}}%
\pgfpathlineto{\pgfqpoint{2.109119in}{0.577617in}}%
\pgfpathlineto{\pgfqpoint{2.112930in}{0.561309in}}%
\pgfpathlineto{\pgfqpoint{2.113623in}{0.563350in}}%
\pgfpathlineto{\pgfqpoint{2.114836in}{0.568876in}}%
\pgfpathlineto{\pgfqpoint{2.115875in}{0.567158in}}%
\pgfpathlineto{\pgfqpoint{2.116395in}{0.566855in}}%
\pgfpathlineto{\pgfqpoint{2.116741in}{0.567945in}}%
\pgfpathlineto{\pgfqpoint{2.121245in}{0.596673in}}%
\pgfpathlineto{\pgfqpoint{2.125229in}{0.619343in}}%
\pgfpathlineto{\pgfqpoint{2.126615in}{0.618666in}}%
\pgfpathlineto{\pgfqpoint{2.130252in}{0.609043in}}%
\pgfpathlineto{\pgfqpoint{2.131465in}{0.611370in}}%
\pgfpathlineto{\pgfqpoint{2.132504in}{0.611150in}}%
\pgfpathlineto{\pgfqpoint{2.132851in}{0.610609in}}%
\pgfpathlineto{\pgfqpoint{2.133717in}{0.609779in}}%
\pgfpathlineto{\pgfqpoint{2.134236in}{0.610657in}}%
\pgfpathlineto{\pgfqpoint{2.136315in}{0.619485in}}%
\pgfpathlineto{\pgfqpoint{2.137181in}{0.622543in}}%
\pgfpathlineto{\pgfqpoint{2.137874in}{0.620792in}}%
\pgfpathlineto{\pgfqpoint{2.140645in}{0.606617in}}%
\pgfpathlineto{\pgfqpoint{2.141685in}{0.608654in}}%
\pgfpathlineto{\pgfqpoint{2.143763in}{0.609660in}}%
\pgfpathlineto{\pgfqpoint{2.144630in}{0.607883in}}%
\pgfpathlineto{\pgfqpoint{2.146362in}{0.600950in}}%
\pgfpathlineto{\pgfqpoint{2.147228in}{0.603008in}}%
\pgfpathlineto{\pgfqpoint{2.148094in}{0.604841in}}%
\pgfpathlineto{\pgfqpoint{2.148787in}{0.602873in}}%
\pgfpathlineto{\pgfqpoint{2.151905in}{0.584365in}}%
\pgfpathlineto{\pgfqpoint{2.153117in}{0.580122in}}%
\pgfpathlineto{\pgfqpoint{2.153637in}{0.581690in}}%
\pgfpathlineto{\pgfqpoint{2.155542in}{0.588438in}}%
\pgfpathlineto{\pgfqpoint{2.156408in}{0.587518in}}%
\pgfpathlineto{\pgfqpoint{2.157448in}{0.586842in}}%
\pgfpathlineto{\pgfqpoint{2.157967in}{0.587718in}}%
\pgfpathlineto{\pgfqpoint{2.160566in}{0.599428in}}%
\pgfpathlineto{\pgfqpoint{2.162818in}{0.604688in}}%
\pgfpathlineto{\pgfqpoint{2.164550in}{0.612353in}}%
\pgfpathlineto{\pgfqpoint{2.165589in}{0.608922in}}%
\pgfpathlineto{\pgfqpoint{2.170613in}{0.594536in}}%
\pgfpathlineto{\pgfqpoint{2.171652in}{0.597687in}}%
\pgfpathlineto{\pgfqpoint{2.174250in}{0.600147in}}%
\pgfpathlineto{\pgfqpoint{2.181179in}{0.598799in}}%
\pgfpathlineto{\pgfqpoint{2.182565in}{0.597875in}}%
\pgfpathlineto{\pgfqpoint{2.186376in}{0.591650in}}%
\pgfpathlineto{\pgfqpoint{2.187242in}{0.593329in}}%
\pgfpathlineto{\pgfqpoint{2.188454in}{0.595168in}}%
\pgfpathlineto{\pgfqpoint{2.189147in}{0.593909in}}%
\pgfpathlineto{\pgfqpoint{2.190360in}{0.592015in}}%
\pgfpathlineto{\pgfqpoint{2.191053in}{0.593254in}}%
\pgfpathlineto{\pgfqpoint{2.196076in}{0.598982in}}%
\pgfpathlineto{\pgfqpoint{2.196422in}{0.598744in}}%
\pgfpathlineto{\pgfqpoint{2.197808in}{0.597045in}}%
\pgfpathlineto{\pgfqpoint{2.199021in}{0.594761in}}%
\pgfpathlineto{\pgfqpoint{2.199714in}{0.596268in}}%
\pgfpathlineto{\pgfqpoint{2.203005in}{0.602413in}}%
\pgfpathlineto{\pgfqpoint{2.204564in}{0.602490in}}%
\pgfpathlineto{\pgfqpoint{2.204737in}{0.602246in}}%
\pgfpathlineto{\pgfqpoint{2.205776in}{0.601057in}}%
\pgfpathlineto{\pgfqpoint{2.206642in}{0.601884in}}%
\pgfpathlineto{\pgfqpoint{2.207682in}{0.599838in}}%
\pgfpathlineto{\pgfqpoint{2.210107in}{0.597383in}}%
\pgfpathlineto{\pgfqpoint{2.210800in}{0.598584in}}%
\pgfpathlineto{\pgfqpoint{2.211839in}{0.601220in}}%
\pgfpathlineto{\pgfqpoint{2.212532in}{0.599105in}}%
\pgfpathlineto{\pgfqpoint{2.214611in}{0.591840in}}%
\pgfpathlineto{\pgfqpoint{2.215130in}{0.592628in}}%
\pgfpathlineto{\pgfqpoint{2.216170in}{0.593217in}}%
\pgfpathlineto{\pgfqpoint{2.216689in}{0.592426in}}%
\pgfpathlineto{\pgfqpoint{2.218768in}{0.590503in}}%
\pgfpathlineto{\pgfqpoint{2.219114in}{0.590742in}}%
\pgfpathlineto{\pgfqpoint{2.220154in}{0.594750in}}%
\pgfpathlineto{\pgfqpoint{2.223445in}{0.607129in}}%
\pgfpathlineto{\pgfqpoint{2.223618in}{0.606910in}}%
\pgfpathlineto{\pgfqpoint{2.224831in}{0.600734in}}%
\pgfpathlineto{\pgfqpoint{2.227602in}{0.588446in}}%
\pgfpathlineto{\pgfqpoint{2.227775in}{0.588538in}}%
\pgfpathlineto{\pgfqpoint{2.231066in}{0.595477in}}%
\pgfpathlineto{\pgfqpoint{2.232452in}{0.600766in}}%
\pgfpathlineto{\pgfqpoint{2.233145in}{0.598673in}}%
\pgfpathlineto{\pgfqpoint{2.237302in}{0.585395in}}%
\pgfpathlineto{\pgfqpoint{2.238515in}{0.583275in}}%
\pgfpathlineto{\pgfqpoint{2.241806in}{0.576798in}}%
\pgfpathlineto{\pgfqpoint{2.244751in}{0.575780in}}%
\pgfpathlineto{\pgfqpoint{2.248735in}{0.568050in}}%
\pgfpathlineto{\pgfqpoint{2.248908in}{0.568161in}}%
\pgfpathlineto{\pgfqpoint{2.250640in}{0.574243in}}%
\pgfpathlineto{\pgfqpoint{2.251680in}{0.571621in}}%
\pgfpathlineto{\pgfqpoint{2.253065in}{0.569346in}}%
\pgfpathlineto{\pgfqpoint{2.253758in}{0.569904in}}%
\pgfpathlineto{\pgfqpoint{2.254798in}{0.568887in}}%
\pgfpathlineto{\pgfqpoint{2.256530in}{0.564345in}}%
\pgfpathlineto{\pgfqpoint{2.257223in}{0.566745in}}%
\pgfpathlineto{\pgfqpoint{2.259994in}{0.573526in}}%
\pgfpathlineto{\pgfqpoint{2.261380in}{0.571598in}}%
\pgfpathlineto{\pgfqpoint{2.265711in}{0.558188in}}%
\pgfpathlineto{\pgfqpoint{2.266230in}{0.558787in}}%
\pgfpathlineto{\pgfqpoint{2.270387in}{0.566691in}}%
\pgfpathlineto{\pgfqpoint{2.274545in}{0.578245in}}%
\pgfpathlineto{\pgfqpoint{2.277836in}{0.586005in}}%
\pgfpathlineto{\pgfqpoint{2.281647in}{0.596013in}}%
\pgfpathlineto{\pgfqpoint{2.281993in}{0.596345in}}%
\pgfpathlineto{\pgfqpoint{2.282686in}{0.595183in}}%
\pgfpathlineto{\pgfqpoint{2.285631in}{0.591860in}}%
\pgfpathlineto{\pgfqpoint{2.286670in}{0.592963in}}%
\pgfpathlineto{\pgfqpoint{2.287536in}{0.594601in}}%
\pgfpathlineto{\pgfqpoint{2.288576in}{0.593488in}}%
\pgfpathlineto{\pgfqpoint{2.295158in}{0.580325in}}%
\pgfpathlineto{\pgfqpoint{2.296024in}{0.583006in}}%
\pgfpathlineto{\pgfqpoint{2.296717in}{0.584595in}}%
\pgfpathlineto{\pgfqpoint{2.297237in}{0.582935in}}%
\pgfpathlineto{\pgfqpoint{2.299315in}{0.573995in}}%
\pgfpathlineto{\pgfqpoint{2.299662in}{0.575912in}}%
\pgfpathlineto{\pgfqpoint{2.300701in}{0.621887in}}%
\pgfpathlineto{\pgfqpoint{2.306764in}{0.918314in}}%
\pgfpathlineto{\pgfqpoint{2.311267in}{1.022151in}}%
\pgfpathlineto{\pgfqpoint{2.317677in}{1.090769in}}%
\pgfpathlineto{\pgfqpoint{2.319236in}{1.093840in}}%
\pgfpathlineto{\pgfqpoint{2.319755in}{1.093646in}}%
\pgfpathlineto{\pgfqpoint{2.324259in}{1.089905in}}%
\pgfpathlineto{\pgfqpoint{2.339502in}{1.051848in}}%
\pgfpathlineto{\pgfqpoint{2.344353in}{1.042227in}}%
\pgfpathlineto{\pgfqpoint{2.347990in}{1.025599in}}%
\pgfpathlineto{\pgfqpoint{2.356998in}{0.977093in}}%
\pgfpathlineto{\pgfqpoint{2.366005in}{0.949573in}}%
\pgfpathlineto{\pgfqpoint{2.370682in}{0.938842in}}%
\pgfpathlineto{\pgfqpoint{2.371202in}{0.939179in}}%
\pgfpathlineto{\pgfqpoint{2.373627in}{0.940216in}}%
\pgfpathlineto{\pgfqpoint{2.373800in}{0.940031in}}%
\pgfpathlineto{\pgfqpoint{2.376918in}{0.933446in}}%
\pgfpathlineto{\pgfqpoint{2.383327in}{0.918561in}}%
\pgfpathlineto{\pgfqpoint{2.383847in}{0.919096in}}%
\pgfpathlineto{\pgfqpoint{2.392508in}{0.934428in}}%
\pgfpathlineto{\pgfqpoint{2.392681in}{0.934304in}}%
\pgfpathlineto{\pgfqpoint{2.396838in}{0.926738in}}%
\pgfpathlineto{\pgfqpoint{2.401688in}{0.909132in}}%
\pgfpathlineto{\pgfqpoint{2.411562in}{0.871495in}}%
\pgfpathlineto{\pgfqpoint{2.418491in}{0.844073in}}%
\pgfpathlineto{\pgfqpoint{2.420743in}{0.837961in}}%
\pgfpathlineto{\pgfqpoint{2.421609in}{0.836332in}}%
\pgfpathlineto{\pgfqpoint{2.436333in}{0.783554in}}%
\pgfpathlineto{\pgfqpoint{2.437025in}{0.782692in}}%
\pgfpathlineto{\pgfqpoint{2.437545in}{0.784222in}}%
\pgfpathlineto{\pgfqpoint{2.438584in}{0.786978in}}%
\pgfpathlineto{\pgfqpoint{2.439104in}{0.785111in}}%
\pgfpathlineto{\pgfqpoint{2.441529in}{0.778736in}}%
\pgfpathlineto{\pgfqpoint{2.442395in}{0.780807in}}%
\pgfpathlineto{\pgfqpoint{2.444127in}{0.787733in}}%
\pgfpathlineto{\pgfqpoint{2.444820in}{0.786113in}}%
\pgfpathlineto{\pgfqpoint{2.446553in}{0.779847in}}%
\pgfpathlineto{\pgfqpoint{2.447245in}{0.782109in}}%
\pgfpathlineto{\pgfqpoint{2.448285in}{0.785094in}}%
\pgfpathlineto{\pgfqpoint{2.448978in}{0.783640in}}%
\pgfpathlineto{\pgfqpoint{2.451056in}{0.778852in}}%
\pgfpathlineto{\pgfqpoint{2.451403in}{0.779372in}}%
\pgfpathlineto{\pgfqpoint{2.455733in}{0.792250in}}%
\pgfpathlineto{\pgfqpoint{2.456946in}{0.789931in}}%
\pgfpathlineto{\pgfqpoint{2.458851in}{0.787442in}}%
\pgfpathlineto{\pgfqpoint{2.459371in}{0.788002in}}%
\pgfpathlineto{\pgfqpoint{2.461103in}{0.787156in}}%
\pgfpathlineto{\pgfqpoint{2.462835in}{0.784824in}}%
\pgfpathlineto{\pgfqpoint{2.469418in}{0.769338in}}%
\pgfpathlineto{\pgfqpoint{2.469937in}{0.770048in}}%
\pgfpathlineto{\pgfqpoint{2.471496in}{0.772603in}}%
\pgfpathlineto{\pgfqpoint{2.472189in}{0.771715in}}%
\pgfpathlineto{\pgfqpoint{2.474095in}{0.767567in}}%
\pgfpathlineto{\pgfqpoint{2.474787in}{0.768875in}}%
\pgfpathlineto{\pgfqpoint{2.476346in}{0.769761in}}%
\pgfpathlineto{\pgfqpoint{2.476520in}{0.769561in}}%
\pgfpathlineto{\pgfqpoint{2.478425in}{0.767102in}}%
\pgfpathlineto{\pgfqpoint{2.479118in}{0.767779in}}%
\pgfpathlineto{\pgfqpoint{2.481716in}{0.774828in}}%
\pgfpathlineto{\pgfqpoint{2.482582in}{0.772335in}}%
\pgfpathlineto{\pgfqpoint{2.485527in}{0.766485in}}%
\pgfpathlineto{\pgfqpoint{2.492629in}{0.750259in}}%
\pgfpathlineto{\pgfqpoint{2.492976in}{0.750624in}}%
\pgfpathlineto{\pgfqpoint{2.494188in}{0.750883in}}%
\pgfpathlineto{\pgfqpoint{2.494535in}{0.750290in}}%
\pgfpathlineto{\pgfqpoint{2.498345in}{0.742025in}}%
\pgfpathlineto{\pgfqpoint{2.498692in}{0.742280in}}%
\pgfpathlineto{\pgfqpoint{2.500944in}{0.744861in}}%
\pgfpathlineto{\pgfqpoint{2.501290in}{0.743607in}}%
\pgfpathlineto{\pgfqpoint{2.502676in}{0.739154in}}%
\pgfpathlineto{\pgfqpoint{2.503196in}{0.741423in}}%
\pgfpathlineto{\pgfqpoint{2.506487in}{0.751090in}}%
\pgfpathlineto{\pgfqpoint{2.509258in}{0.754677in}}%
\pgfpathlineto{\pgfqpoint{2.509778in}{0.753980in}}%
\pgfpathlineto{\pgfqpoint{2.518266in}{0.741219in}}%
\pgfpathlineto{\pgfqpoint{2.518959in}{0.742622in}}%
\pgfpathlineto{\pgfqpoint{2.520691in}{0.749656in}}%
\pgfpathlineto{\pgfqpoint{2.521557in}{0.748366in}}%
\pgfpathlineto{\pgfqpoint{2.526754in}{0.732195in}}%
\pgfpathlineto{\pgfqpoint{2.527446in}{0.735314in}}%
\pgfpathlineto{\pgfqpoint{2.529698in}{0.751354in}}%
\pgfpathlineto{\pgfqpoint{2.530564in}{0.746288in}}%
\pgfpathlineto{\pgfqpoint{2.532470in}{0.735973in}}%
\pgfpathlineto{\pgfqpoint{2.532989in}{0.736221in}}%
\pgfpathlineto{\pgfqpoint{2.533856in}{0.735117in}}%
\pgfpathlineto{\pgfqpoint{2.534895in}{0.734679in}}%
\pgfpathlineto{\pgfqpoint{2.535241in}{0.735648in}}%
\pgfpathlineto{\pgfqpoint{2.537493in}{0.742270in}}%
\pgfpathlineto{\pgfqpoint{2.538013in}{0.741254in}}%
\pgfpathlineto{\pgfqpoint{2.539572in}{0.729328in}}%
\pgfpathlineto{\pgfqpoint{2.541304in}{0.722508in}}%
\pgfpathlineto{\pgfqpoint{2.541650in}{0.723015in}}%
\pgfpathlineto{\pgfqpoint{2.548579in}{0.755471in}}%
\pgfpathlineto{\pgfqpoint{2.549965in}{0.760778in}}%
\pgfpathlineto{\pgfqpoint{2.550658in}{0.760218in}}%
\pgfpathlineto{\pgfqpoint{2.551870in}{0.754603in}}%
\pgfpathlineto{\pgfqpoint{2.553429in}{0.746607in}}%
\pgfpathlineto{\pgfqpoint{2.554122in}{0.748182in}}%
\pgfpathlineto{\pgfqpoint{2.560358in}{0.759974in}}%
\pgfpathlineto{\pgfqpoint{2.560878in}{0.760333in}}%
\pgfpathlineto{\pgfqpoint{2.561571in}{0.759284in}}%
\pgfpathlineto{\pgfqpoint{2.562783in}{0.755955in}}%
\pgfpathlineto{\pgfqpoint{2.564169in}{0.749300in}}%
\pgfpathlineto{\pgfqpoint{2.565035in}{0.751645in}}%
\pgfpathlineto{\pgfqpoint{2.567634in}{0.758575in}}%
\pgfpathlineto{\pgfqpoint{2.568153in}{0.757553in}}%
\pgfpathlineto{\pgfqpoint{2.572657in}{0.736432in}}%
\pgfpathlineto{\pgfqpoint{2.574043in}{0.740394in}}%
\pgfpathlineto{\pgfqpoint{2.578720in}{0.750506in}}%
\pgfpathlineto{\pgfqpoint{2.578893in}{0.750399in}}%
\pgfpathlineto{\pgfqpoint{2.582184in}{0.746756in}}%
\pgfpathlineto{\pgfqpoint{2.582877in}{0.749216in}}%
\pgfpathlineto{\pgfqpoint{2.583916in}{0.751885in}}%
\pgfpathlineto{\pgfqpoint{2.584436in}{0.750436in}}%
\pgfpathlineto{\pgfqpoint{2.589113in}{0.732926in}}%
\pgfpathlineto{\pgfqpoint{2.590672in}{0.725820in}}%
\pgfpathlineto{\pgfqpoint{2.592577in}{0.712810in}}%
\pgfpathlineto{\pgfqpoint{2.593270in}{0.714861in}}%
\pgfpathlineto{\pgfqpoint{2.598120in}{0.727839in}}%
\pgfpathlineto{\pgfqpoint{2.598986in}{0.726870in}}%
\pgfpathlineto{\pgfqpoint{2.599506in}{0.725976in}}%
\pgfpathlineto{\pgfqpoint{2.600719in}{0.726598in}}%
\pgfpathlineto{\pgfqpoint{2.601585in}{0.729037in}}%
\pgfpathlineto{\pgfqpoint{2.602970in}{0.733879in}}%
\pgfpathlineto{\pgfqpoint{2.603837in}{0.731953in}}%
\pgfpathlineto{\pgfqpoint{2.605569in}{0.725647in}}%
\pgfpathlineto{\pgfqpoint{2.606781in}{0.727590in}}%
\pgfpathlineto{\pgfqpoint{2.607821in}{0.722467in}}%
\pgfpathlineto{\pgfqpoint{2.608860in}{0.718832in}}%
\pgfpathlineto{\pgfqpoint{2.609553in}{0.720865in}}%
\pgfpathlineto{\pgfqpoint{2.611458in}{0.723884in}}%
\pgfpathlineto{\pgfqpoint{2.611978in}{0.723433in}}%
\pgfpathlineto{\pgfqpoint{2.615096in}{0.720650in}}%
\pgfpathlineto{\pgfqpoint{2.616482in}{0.718784in}}%
\pgfpathlineto{\pgfqpoint{2.621678in}{0.704293in}}%
\pgfpathlineto{\pgfqpoint{2.622371in}{0.705702in}}%
\pgfpathlineto{\pgfqpoint{2.626182in}{0.712741in}}%
\pgfpathlineto{\pgfqpoint{2.627395in}{0.709530in}}%
\pgfpathlineto{\pgfqpoint{2.628954in}{0.703145in}}%
\pgfpathlineto{\pgfqpoint{2.629820in}{0.704637in}}%
\pgfpathlineto{\pgfqpoint{2.631898in}{0.706204in}}%
\pgfpathlineto{\pgfqpoint{2.632591in}{0.705277in}}%
\pgfpathlineto{\pgfqpoint{2.633977in}{0.697828in}}%
\pgfpathlineto{\pgfqpoint{2.634843in}{0.700286in}}%
\pgfpathlineto{\pgfqpoint{2.636575in}{0.700847in}}%
\pgfpathlineto{\pgfqpoint{2.637441in}{0.701112in}}%
\pgfpathlineto{\pgfqpoint{2.637788in}{0.700274in}}%
\pgfpathlineto{\pgfqpoint{2.643504in}{0.686913in}}%
\pgfpathlineto{\pgfqpoint{2.644543in}{0.688225in}}%
\pgfpathlineto{\pgfqpoint{2.650606in}{0.703931in}}%
\pgfpathlineto{\pgfqpoint{2.650952in}{0.703458in}}%
\pgfpathlineto{\pgfqpoint{2.653204in}{0.704382in}}%
\pgfpathlineto{\pgfqpoint{2.653551in}{0.705233in}}%
\pgfpathlineto{\pgfqpoint{2.654590in}{0.704215in}}%
\pgfpathlineto{\pgfqpoint{2.662558in}{0.694688in}}%
\pgfpathlineto{\pgfqpoint{2.665676in}{0.684706in}}%
\pgfpathlineto{\pgfqpoint{2.666023in}{0.685591in}}%
\pgfpathlineto{\pgfqpoint{2.667235in}{0.691330in}}%
\pgfpathlineto{\pgfqpoint{2.667928in}{0.688013in}}%
\pgfpathlineto{\pgfqpoint{2.668794in}{0.684322in}}%
\pgfpathlineto{\pgfqpoint{2.669487in}{0.686822in}}%
\pgfpathlineto{\pgfqpoint{2.671739in}{0.690560in}}%
\pgfpathlineto{\pgfqpoint{2.672432in}{0.692151in}}%
\pgfpathlineto{\pgfqpoint{2.676589in}{0.699547in}}%
\pgfpathlineto{\pgfqpoint{2.677282in}{0.699507in}}%
\pgfpathlineto{\pgfqpoint{2.677628in}{0.698678in}}%
\pgfpathlineto{\pgfqpoint{2.683345in}{0.677092in}}%
\pgfpathlineto{\pgfqpoint{2.684557in}{0.678318in}}%
\pgfpathlineto{\pgfqpoint{2.685077in}{0.678753in}}%
\pgfpathlineto{\pgfqpoint{2.685423in}{0.677201in}}%
\pgfpathlineto{\pgfqpoint{2.686289in}{0.673108in}}%
\pgfpathlineto{\pgfqpoint{2.686982in}{0.675889in}}%
\pgfpathlineto{\pgfqpoint{2.689754in}{0.689605in}}%
\pgfpathlineto{\pgfqpoint{2.692525in}{0.701739in}}%
\pgfpathlineto{\pgfqpoint{2.692699in}{0.701608in}}%
\pgfpathlineto{\pgfqpoint{2.701013in}{0.680789in}}%
\pgfpathlineto{\pgfqpoint{2.701879in}{0.683623in}}%
\pgfpathlineto{\pgfqpoint{2.702572in}{0.685159in}}%
\pgfpathlineto{\pgfqpoint{2.703265in}{0.683023in}}%
\pgfpathlineto{\pgfqpoint{2.703785in}{0.682527in}}%
\pgfpathlineto{\pgfqpoint{2.704304in}{0.683587in}}%
\pgfpathlineto{\pgfqpoint{2.707076in}{0.690936in}}%
\pgfpathlineto{\pgfqpoint{2.708115in}{0.690150in}}%
\pgfpathlineto{\pgfqpoint{2.708635in}{0.690980in}}%
\pgfpathlineto{\pgfqpoint{2.709847in}{0.700400in}}%
\pgfpathlineto{\pgfqpoint{2.710887in}{0.705933in}}%
\pgfpathlineto{\pgfqpoint{2.711580in}{0.703847in}}%
\pgfpathlineto{\pgfqpoint{2.713139in}{0.699558in}}%
\pgfpathlineto{\pgfqpoint{2.713831in}{0.700882in}}%
\pgfpathlineto{\pgfqpoint{2.717469in}{0.712272in}}%
\pgfpathlineto{\pgfqpoint{2.718682in}{0.709002in}}%
\pgfpathlineto{\pgfqpoint{2.720067in}{0.705551in}}%
\pgfpathlineto{\pgfqpoint{2.720587in}{0.707024in}}%
\pgfpathlineto{\pgfqpoint{2.722319in}{0.715926in}}%
\pgfpathlineto{\pgfqpoint{2.723359in}{0.712638in}}%
\pgfpathlineto{\pgfqpoint{2.726130in}{0.706549in}}%
\pgfpathlineto{\pgfqpoint{2.726650in}{0.708160in}}%
\pgfpathlineto{\pgfqpoint{2.728728in}{0.712247in}}%
\pgfpathlineto{\pgfqpoint{2.730114in}{0.709772in}}%
\pgfpathlineto{\pgfqpoint{2.732020in}{0.702819in}}%
\pgfpathlineto{\pgfqpoint{2.736870in}{0.687620in}}%
\pgfpathlineto{\pgfqpoint{2.739295in}{0.686016in}}%
\pgfpathlineto{\pgfqpoint{2.739641in}{0.686598in}}%
\pgfpathlineto{\pgfqpoint{2.745011in}{0.706878in}}%
\pgfpathlineto{\pgfqpoint{2.746397in}{0.703040in}}%
\pgfpathlineto{\pgfqpoint{2.748476in}{0.699138in}}%
\pgfpathlineto{\pgfqpoint{2.748995in}{0.701286in}}%
\pgfpathlineto{\pgfqpoint{2.750901in}{0.710499in}}%
\pgfpathlineto{\pgfqpoint{2.751767in}{0.709260in}}%
\pgfpathlineto{\pgfqpoint{2.753845in}{0.700460in}}%
\pgfpathlineto{\pgfqpoint{2.756444in}{0.678741in}}%
\pgfpathlineto{\pgfqpoint{2.757137in}{0.679850in}}%
\pgfpathlineto{\pgfqpoint{2.763199in}{0.700547in}}%
\pgfpathlineto{\pgfqpoint{2.763892in}{0.698488in}}%
\pgfpathlineto{\pgfqpoint{2.766317in}{0.695017in}}%
\pgfpathlineto{\pgfqpoint{2.767357in}{0.699436in}}%
\pgfpathlineto{\pgfqpoint{2.768916in}{0.705708in}}%
\pgfpathlineto{\pgfqpoint{2.769608in}{0.704004in}}%
\pgfpathlineto{\pgfqpoint{2.771167in}{0.698220in}}%
\pgfpathlineto{\pgfqpoint{2.772207in}{0.699811in}}%
\pgfpathlineto{\pgfqpoint{2.774459in}{0.700821in}}%
\pgfpathlineto{\pgfqpoint{2.774632in}{0.700658in}}%
\pgfpathlineto{\pgfqpoint{2.774978in}{0.700197in}}%
\pgfpathlineto{\pgfqpoint{2.775498in}{0.701142in}}%
\pgfpathlineto{\pgfqpoint{2.775844in}{0.701256in}}%
\pgfpathlineto{\pgfqpoint{2.777577in}{0.705605in}}%
\pgfpathlineto{\pgfqpoint{2.778616in}{0.708116in}}%
\pgfpathlineto{\pgfqpoint{2.779482in}{0.706639in}}%
\pgfpathlineto{\pgfqpoint{2.782253in}{0.710284in}}%
\pgfpathlineto{\pgfqpoint{2.782773in}{0.709524in}}%
\pgfpathlineto{\pgfqpoint{2.783466in}{0.708166in}}%
\pgfpathlineto{\pgfqpoint{2.784159in}{0.709780in}}%
\pgfpathlineto{\pgfqpoint{2.785198in}{0.709717in}}%
\pgfpathlineto{\pgfqpoint{2.785371in}{0.709363in}}%
\pgfpathlineto{\pgfqpoint{2.787277in}{0.701215in}}%
\pgfpathlineto{\pgfqpoint{2.787970in}{0.704684in}}%
\pgfpathlineto{\pgfqpoint{2.789009in}{0.710864in}}%
\pgfpathlineto{\pgfqpoint{2.789875in}{0.707665in}}%
\pgfpathlineto{\pgfqpoint{2.790914in}{0.705366in}}%
\pgfpathlineto{\pgfqpoint{2.791607in}{0.706623in}}%
\pgfpathlineto{\pgfqpoint{2.792993in}{0.708288in}}%
\pgfpathlineto{\pgfqpoint{2.793513in}{0.707214in}}%
\pgfpathlineto{\pgfqpoint{2.798363in}{0.679688in}}%
\pgfpathlineto{\pgfqpoint{2.800442in}{0.683282in}}%
\pgfpathlineto{\pgfqpoint{2.806331in}{0.699308in}}%
\pgfpathlineto{\pgfqpoint{2.807197in}{0.698137in}}%
\pgfpathlineto{\pgfqpoint{2.810488in}{0.691412in}}%
\pgfpathlineto{\pgfqpoint{2.813780in}{0.687986in}}%
\pgfpathlineto{\pgfqpoint{2.815858in}{0.677268in}}%
\pgfpathlineto{\pgfqpoint{2.817071in}{0.679223in}}%
\pgfpathlineto{\pgfqpoint{2.819496in}{0.690051in}}%
\pgfpathlineto{\pgfqpoint{2.820535in}{0.694756in}}%
\pgfpathlineto{\pgfqpoint{2.821574in}{0.694225in}}%
\pgfpathlineto{\pgfqpoint{2.821748in}{0.694012in}}%
\pgfpathlineto{\pgfqpoint{2.822441in}{0.695478in}}%
\pgfpathlineto{\pgfqpoint{2.824000in}{0.696271in}}%
\pgfpathlineto{\pgfqpoint{2.824173in}{0.696075in}}%
\pgfpathlineto{\pgfqpoint{2.824519in}{0.695806in}}%
\pgfpathlineto{\pgfqpoint{2.825039in}{0.696918in}}%
\pgfpathlineto{\pgfqpoint{2.828330in}{0.705713in}}%
\pgfpathlineto{\pgfqpoint{2.832834in}{0.715121in}}%
\pgfpathlineto{\pgfqpoint{2.833700in}{0.717408in}}%
\pgfpathlineto{\pgfqpoint{2.834393in}{0.715059in}}%
\pgfpathlineto{\pgfqpoint{2.836818in}{0.699502in}}%
\pgfpathlineto{\pgfqpoint{2.837511in}{0.701908in}}%
\pgfpathlineto{\pgfqpoint{2.840456in}{0.706721in}}%
\pgfpathlineto{\pgfqpoint{2.840629in}{0.706570in}}%
\pgfpathlineto{\pgfqpoint{2.843227in}{0.696247in}}%
\pgfpathlineto{\pgfqpoint{2.846691in}{0.678045in}}%
\pgfpathlineto{\pgfqpoint{2.847384in}{0.677563in}}%
\pgfpathlineto{\pgfqpoint{2.847731in}{0.678793in}}%
\pgfpathlineto{\pgfqpoint{2.850329in}{0.690302in}}%
\pgfpathlineto{\pgfqpoint{2.850849in}{0.689552in}}%
\pgfpathlineto{\pgfqpoint{2.851715in}{0.683362in}}%
\pgfpathlineto{\pgfqpoint{2.854660in}{0.664067in}}%
\pgfpathlineto{\pgfqpoint{2.855179in}{0.666092in}}%
\pgfpathlineto{\pgfqpoint{2.858297in}{0.684344in}}%
\pgfpathlineto{\pgfqpoint{2.858990in}{0.683193in}}%
\pgfpathlineto{\pgfqpoint{2.860029in}{0.677827in}}%
\pgfpathlineto{\pgfqpoint{2.861069in}{0.680924in}}%
\pgfpathlineto{\pgfqpoint{2.862281in}{0.686322in}}%
\pgfpathlineto{\pgfqpoint{2.862974in}{0.684518in}}%
\pgfpathlineto{\pgfqpoint{2.864706in}{0.679087in}}%
\pgfpathlineto{\pgfqpoint{2.865226in}{0.680459in}}%
\pgfpathlineto{\pgfqpoint{2.868171in}{0.691002in}}%
\pgfpathlineto{\pgfqpoint{2.868517in}{0.690876in}}%
\pgfpathlineto{\pgfqpoint{2.875100in}{0.704819in}}%
\pgfpathlineto{\pgfqpoint{2.876485in}{0.702828in}}%
\pgfpathlineto{\pgfqpoint{2.881509in}{0.680955in}}%
\pgfpathlineto{\pgfqpoint{2.882548in}{0.685616in}}%
\pgfpathlineto{\pgfqpoint{2.884627in}{0.696748in}}%
\pgfpathlineto{\pgfqpoint{2.885146in}{0.696189in}}%
\pgfpathlineto{\pgfqpoint{2.886012in}{0.693758in}}%
\pgfpathlineto{\pgfqpoint{2.886879in}{0.694467in}}%
\pgfpathlineto{\pgfqpoint{2.887225in}{0.694678in}}%
\pgfpathlineto{\pgfqpoint{2.887745in}{0.693391in}}%
\pgfpathlineto{\pgfqpoint{2.889997in}{0.687421in}}%
\pgfpathlineto{\pgfqpoint{2.890689in}{0.688832in}}%
\pgfpathlineto{\pgfqpoint{2.891902in}{0.692042in}}%
\pgfpathlineto{\pgfqpoint{2.892595in}{0.690175in}}%
\pgfpathlineto{\pgfqpoint{2.897099in}{0.672381in}}%
\pgfpathlineto{\pgfqpoint{2.897445in}{0.671866in}}%
\pgfpathlineto{\pgfqpoint{2.898138in}{0.673650in}}%
\pgfpathlineto{\pgfqpoint{2.900217in}{0.679138in}}%
\pgfpathlineto{\pgfqpoint{2.901083in}{0.677973in}}%
\pgfpathlineto{\pgfqpoint{2.901602in}{0.678447in}}%
\pgfpathlineto{\pgfqpoint{2.902295in}{0.677291in}}%
\pgfpathlineto{\pgfqpoint{2.903161in}{0.677022in}}%
\pgfpathlineto{\pgfqpoint{2.903508in}{0.678048in}}%
\pgfpathlineto{\pgfqpoint{2.905067in}{0.685095in}}%
\pgfpathlineto{\pgfqpoint{2.905760in}{0.683130in}}%
\pgfpathlineto{\pgfqpoint{2.909397in}{0.673558in}}%
\pgfpathlineto{\pgfqpoint{2.910956in}{0.677354in}}%
\pgfpathlineto{\pgfqpoint{2.913035in}{0.685455in}}%
\pgfpathlineto{\pgfqpoint{2.913728in}{0.681640in}}%
\pgfpathlineto{\pgfqpoint{2.916326in}{0.667417in}}%
\pgfpathlineto{\pgfqpoint{2.916672in}{0.667947in}}%
\pgfpathlineto{\pgfqpoint{2.918058in}{0.672239in}}%
\pgfpathlineto{\pgfqpoint{2.919098in}{0.671239in}}%
\pgfpathlineto{\pgfqpoint{2.919964in}{0.673012in}}%
\pgfpathlineto{\pgfqpoint{2.920483in}{0.674855in}}%
\pgfpathlineto{\pgfqpoint{2.921523in}{0.673960in}}%
\pgfpathlineto{\pgfqpoint{2.922389in}{0.673078in}}%
\pgfpathlineto{\pgfqpoint{2.922908in}{0.674340in}}%
\pgfpathlineto{\pgfqpoint{2.928625in}{0.685014in}}%
\pgfpathlineto{\pgfqpoint{2.929318in}{0.682583in}}%
\pgfpathlineto{\pgfqpoint{2.930703in}{0.677662in}}%
\pgfpathlineto{\pgfqpoint{2.931396in}{0.678906in}}%
\pgfpathlineto{\pgfqpoint{2.933302in}{0.681575in}}%
\pgfpathlineto{\pgfqpoint{2.933821in}{0.680473in}}%
\pgfpathlineto{\pgfqpoint{2.936766in}{0.674714in}}%
\pgfpathlineto{\pgfqpoint{2.937632in}{0.676195in}}%
\pgfpathlineto{\pgfqpoint{2.941443in}{0.688762in}}%
\pgfpathlineto{\pgfqpoint{2.943522in}{0.694120in}}%
\pgfpathlineto{\pgfqpoint{2.944041in}{0.693880in}}%
\pgfpathlineto{\pgfqpoint{2.945081in}{0.693526in}}%
\pgfpathlineto{\pgfqpoint{2.945254in}{0.693137in}}%
\pgfpathlineto{\pgfqpoint{2.949931in}{0.677987in}}%
\pgfpathlineto{\pgfqpoint{2.951143in}{0.665244in}}%
\pgfpathlineto{\pgfqpoint{2.951836in}{0.670367in}}%
\pgfpathlineto{\pgfqpoint{2.953395in}{0.681338in}}%
\pgfpathlineto{\pgfqpoint{2.954261in}{0.681156in}}%
\pgfpathlineto{\pgfqpoint{2.955127in}{0.678411in}}%
\pgfpathlineto{\pgfqpoint{2.957033in}{0.668348in}}%
\pgfpathlineto{\pgfqpoint{2.957726in}{0.670771in}}%
\pgfpathlineto{\pgfqpoint{2.962576in}{0.696717in}}%
\pgfpathlineto{\pgfqpoint{2.963788in}{0.693486in}}%
\pgfpathlineto{\pgfqpoint{2.966040in}{0.686265in}}%
\pgfpathlineto{\pgfqpoint{2.966387in}{0.686750in}}%
\pgfpathlineto{\pgfqpoint{2.969678in}{0.695451in}}%
\pgfpathlineto{\pgfqpoint{2.970198in}{0.695239in}}%
\pgfpathlineto{\pgfqpoint{2.971237in}{0.693585in}}%
\pgfpathlineto{\pgfqpoint{2.972276in}{0.690843in}}%
\pgfpathlineto{\pgfqpoint{2.972969in}{0.693318in}}%
\pgfpathlineto{\pgfqpoint{2.974182in}{0.696881in}}%
\pgfpathlineto{\pgfqpoint{2.974701in}{0.694993in}}%
\pgfpathlineto{\pgfqpoint{2.975048in}{0.694337in}}%
\pgfpathlineto{\pgfqpoint{2.975741in}{0.696273in}}%
\pgfpathlineto{\pgfqpoint{2.976953in}{0.700679in}}%
\pgfpathlineto{\pgfqpoint{2.977646in}{0.698295in}}%
\pgfpathlineto{\pgfqpoint{2.978685in}{0.696211in}}%
\pgfpathlineto{\pgfqpoint{2.979205in}{0.698210in}}%
\pgfpathlineto{\pgfqpoint{2.980071in}{0.702222in}}%
\pgfpathlineto{\pgfqpoint{2.980937in}{0.700526in}}%
\pgfpathlineto{\pgfqpoint{2.981457in}{0.699130in}}%
\pgfpathlineto{\pgfqpoint{2.982150in}{0.701294in}}%
\pgfpathlineto{\pgfqpoint{2.984748in}{0.707384in}}%
\pgfpathlineto{\pgfqpoint{2.985094in}{0.707124in}}%
\pgfpathlineto{\pgfqpoint{2.986827in}{0.702190in}}%
\pgfpathlineto{\pgfqpoint{2.990984in}{0.688099in}}%
\pgfpathlineto{\pgfqpoint{2.991504in}{0.686644in}}%
\pgfpathlineto{\pgfqpoint{2.992543in}{0.687814in}}%
\pgfpathlineto{\pgfqpoint{2.993755in}{0.687034in}}%
\pgfpathlineto{\pgfqpoint{2.994102in}{0.688192in}}%
\pgfpathlineto{\pgfqpoint{2.995834in}{0.694404in}}%
\pgfpathlineto{\pgfqpoint{2.996700in}{0.693968in}}%
\pgfpathlineto{\pgfqpoint{2.998259in}{0.701094in}}%
\pgfpathlineto{\pgfqpoint{3.000338in}{0.708388in}}%
\pgfpathlineto{\pgfqpoint{3.000858in}{0.707884in}}%
\pgfpathlineto{\pgfqpoint{3.001377in}{0.707858in}}%
\pgfpathlineto{\pgfqpoint{3.001550in}{0.708460in}}%
\pgfpathlineto{\pgfqpoint{3.005708in}{0.721474in}}%
\pgfpathlineto{\pgfqpoint{3.008306in}{0.733418in}}%
\pgfpathlineto{\pgfqpoint{3.009865in}{0.731155in}}%
\pgfpathlineto{\pgfqpoint{3.011251in}{0.727186in}}%
\pgfpathlineto{\pgfqpoint{3.012117in}{0.728754in}}%
\pgfpathlineto{\pgfqpoint{3.012810in}{0.730071in}}%
\pgfpathlineto{\pgfqpoint{3.013676in}{0.729022in}}%
\pgfpathlineto{\pgfqpoint{3.016447in}{0.728029in}}%
\pgfpathlineto{\pgfqpoint{3.016794in}{0.729289in}}%
\pgfpathlineto{\pgfqpoint{3.020258in}{0.744450in}}%
\pgfpathlineto{\pgfqpoint{3.021124in}{0.742342in}}%
\pgfpathlineto{\pgfqpoint{3.021990in}{0.740961in}}%
\pgfpathlineto{\pgfqpoint{3.022856in}{0.741999in}}%
\pgfpathlineto{\pgfqpoint{3.023723in}{0.742190in}}%
\pgfpathlineto{\pgfqpoint{3.024069in}{0.741273in}}%
\pgfpathlineto{\pgfqpoint{3.028919in}{0.722436in}}%
\pgfpathlineto{\pgfqpoint{3.029439in}{0.724924in}}%
\pgfpathlineto{\pgfqpoint{3.032210in}{0.737199in}}%
\pgfpathlineto{\pgfqpoint{3.032384in}{0.737331in}}%
\pgfpathlineto{\pgfqpoint{3.032903in}{0.735845in}}%
\pgfpathlineto{\pgfqpoint{3.036368in}{0.724836in}}%
\pgfpathlineto{\pgfqpoint{3.037061in}{0.724080in}}%
\pgfpathlineto{\pgfqpoint{3.037753in}{0.725214in}}%
\pgfpathlineto{\pgfqpoint{3.038100in}{0.725249in}}%
\pgfpathlineto{\pgfqpoint{3.038446in}{0.724354in}}%
\pgfpathlineto{\pgfqpoint{3.039312in}{0.722415in}}%
\pgfpathlineto{\pgfqpoint{3.040005in}{0.724258in}}%
\pgfpathlineto{\pgfqpoint{3.042257in}{0.732952in}}%
\pgfpathlineto{\pgfqpoint{3.042950in}{0.731131in}}%
\pgfpathlineto{\pgfqpoint{3.047454in}{0.717312in}}%
\pgfpathlineto{\pgfqpoint{3.047800in}{0.717499in}}%
\pgfpathlineto{\pgfqpoint{3.050225in}{0.716936in}}%
\pgfpathlineto{\pgfqpoint{3.054729in}{0.700501in}}%
\pgfpathlineto{\pgfqpoint{3.059752in}{0.697355in}}%
\pgfpathlineto{\pgfqpoint{3.062178in}{0.693111in}}%
\pgfpathlineto{\pgfqpoint{3.062524in}{0.693610in}}%
\pgfpathlineto{\pgfqpoint{3.065469in}{0.711164in}}%
\pgfpathlineto{\pgfqpoint{3.066335in}{0.705810in}}%
\pgfpathlineto{\pgfqpoint{3.067894in}{0.689164in}}%
\pgfpathlineto{\pgfqpoint{3.068933in}{0.690150in}}%
\pgfpathlineto{\pgfqpoint{3.069972in}{0.682186in}}%
\pgfpathlineto{\pgfqpoint{3.070839in}{0.678865in}}%
\pgfpathlineto{\pgfqpoint{3.071531in}{0.680985in}}%
\pgfpathlineto{\pgfqpoint{3.075169in}{0.699998in}}%
\pgfpathlineto{\pgfqpoint{3.075862in}{0.697954in}}%
\pgfpathlineto{\pgfqpoint{3.076901in}{0.696215in}}%
\pgfpathlineto{\pgfqpoint{3.077421in}{0.697465in}}%
\pgfpathlineto{\pgfqpoint{3.081578in}{0.703513in}}%
\pgfpathlineto{\pgfqpoint{3.082791in}{0.698026in}}%
\pgfpathlineto{\pgfqpoint{3.083310in}{0.696667in}}%
\pgfpathlineto{\pgfqpoint{3.084176in}{0.698271in}}%
\pgfpathlineto{\pgfqpoint{3.084696in}{0.698813in}}%
\pgfpathlineto{\pgfqpoint{3.085216in}{0.697286in}}%
\pgfpathlineto{\pgfqpoint{3.088161in}{0.690529in}}%
\pgfpathlineto{\pgfqpoint{3.088680in}{0.692017in}}%
\pgfpathlineto{\pgfqpoint{3.090239in}{0.700571in}}%
\pgfpathlineto{\pgfqpoint{3.091105in}{0.696805in}}%
\pgfpathlineto{\pgfqpoint{3.091798in}{0.694737in}}%
\pgfpathlineto{\pgfqpoint{3.092491in}{0.697260in}}%
\pgfpathlineto{\pgfqpoint{3.093184in}{0.699301in}}%
\pgfpathlineto{\pgfqpoint{3.093704in}{0.697215in}}%
\pgfpathlineto{\pgfqpoint{3.095609in}{0.679675in}}%
\pgfpathlineto{\pgfqpoint{3.096648in}{0.682589in}}%
\pgfpathlineto{\pgfqpoint{3.099766in}{0.695853in}}%
\pgfpathlineto{\pgfqpoint{3.099940in}{0.695814in}}%
\pgfpathlineto{\pgfqpoint{3.101499in}{0.694913in}}%
\pgfpathlineto{\pgfqpoint{3.101672in}{0.695344in}}%
\pgfpathlineto{\pgfqpoint{3.104270in}{0.703240in}}%
\pgfpathlineto{\pgfqpoint{3.104616in}{0.702832in}}%
\pgfpathlineto{\pgfqpoint{3.107561in}{0.692386in}}%
\pgfpathlineto{\pgfqpoint{3.108774in}{0.687586in}}%
\pgfpathlineto{\pgfqpoint{3.109467in}{0.689826in}}%
\pgfpathlineto{\pgfqpoint{3.111719in}{0.697711in}}%
\pgfpathlineto{\pgfqpoint{3.112065in}{0.697144in}}%
\pgfpathlineto{\pgfqpoint{3.116569in}{0.681829in}}%
\pgfpathlineto{\pgfqpoint{3.117435in}{0.684875in}}%
\pgfpathlineto{\pgfqpoint{3.119340in}{0.692778in}}%
\pgfpathlineto{\pgfqpoint{3.120206in}{0.692385in}}%
\pgfpathlineto{\pgfqpoint{3.120899in}{0.692659in}}%
\pgfpathlineto{\pgfqpoint{3.121072in}{0.692073in}}%
\pgfpathlineto{\pgfqpoint{3.123497in}{0.674727in}}%
\pgfpathlineto{\pgfqpoint{3.124710in}{0.679740in}}%
\pgfpathlineto{\pgfqpoint{3.126789in}{0.692445in}}%
\pgfpathlineto{\pgfqpoint{3.127482in}{0.691717in}}%
\pgfpathlineto{\pgfqpoint{3.131466in}{0.685082in}}%
\pgfpathlineto{\pgfqpoint{3.132159in}{0.686750in}}%
\pgfpathlineto{\pgfqpoint{3.132332in}{0.687026in}}%
\pgfpathlineto{\pgfqpoint{3.132851in}{0.685562in}}%
\pgfpathlineto{\pgfqpoint{3.134064in}{0.680600in}}%
\pgfpathlineto{\pgfqpoint{3.134757in}{0.684091in}}%
\pgfpathlineto{\pgfqpoint{3.136662in}{0.691539in}}%
\pgfpathlineto{\pgfqpoint{3.137182in}{0.690480in}}%
\pgfpathlineto{\pgfqpoint{3.139434in}{0.671517in}}%
\pgfpathlineto{\pgfqpoint{3.140993in}{0.677175in}}%
\pgfpathlineto{\pgfqpoint{3.144630in}{0.694597in}}%
\pgfpathlineto{\pgfqpoint{3.145843in}{0.693058in}}%
\pgfpathlineto{\pgfqpoint{3.147055in}{0.688600in}}%
\pgfpathlineto{\pgfqpoint{3.147922in}{0.690086in}}%
\pgfpathlineto{\pgfqpoint{3.148961in}{0.695578in}}%
\pgfpathlineto{\pgfqpoint{3.149654in}{0.691443in}}%
\pgfpathlineto{\pgfqpoint{3.152599in}{0.677823in}}%
\pgfpathlineto{\pgfqpoint{3.154157in}{0.669752in}}%
\pgfpathlineto{\pgfqpoint{3.155197in}{0.671952in}}%
\pgfpathlineto{\pgfqpoint{3.155716in}{0.672432in}}%
\pgfpathlineto{\pgfqpoint{3.156409in}{0.671128in}}%
\pgfpathlineto{\pgfqpoint{3.156929in}{0.671800in}}%
\pgfpathlineto{\pgfqpoint{3.157622in}{0.673762in}}%
\pgfpathlineto{\pgfqpoint{3.158315in}{0.672116in}}%
\pgfpathlineto{\pgfqpoint{3.159354in}{0.669406in}}%
\pgfpathlineto{\pgfqpoint{3.160047in}{0.670993in}}%
\pgfpathlineto{\pgfqpoint{3.164204in}{0.681925in}}%
\pgfpathlineto{\pgfqpoint{3.164897in}{0.686467in}}%
\pgfpathlineto{\pgfqpoint{3.164897in}{0.686467in}}%
\pgfusepath{stroke}%
\end{pgfscope}%
\begin{pgfscope}%
\pgfpathrectangle{\pgfqpoint{0.568671in}{0.451277in}}{\pgfqpoint{2.598305in}{0.798874in}}%
\pgfusepath{clip}%
\pgfsetbuttcap%
\pgfsetroundjoin%
\definecolor{currentfill}{rgb}{0.000000,0.000000,0.000000}%
\pgfsetfillcolor{currentfill}%
\pgfsetfillopacity{0.000000}%
\pgfsetlinewidth{1.003750pt}%
\definecolor{currentstroke}{rgb}{0.121569,0.466667,0.705882}%
\pgfsetstrokecolor{currentstroke}%
\pgfsetdash{}{0pt}%
\pgfsys@defobject{currentmarker}{\pgfqpoint{-0.027778in}{-0.027778in}}{\pgfqpoint{0.027778in}{0.027778in}}{%
\pgfpathmoveto{\pgfqpoint{0.000000in}{-0.027778in}}%
\pgfpathcurveto{\pgfqpoint{0.007367in}{-0.027778in}}{\pgfqpoint{0.014433in}{-0.024851in}}{\pgfqpoint{0.019642in}{-0.019642in}}%
\pgfpathcurveto{\pgfqpoint{0.024851in}{-0.014433in}}{\pgfqpoint{0.027778in}{-0.007367in}}{\pgfqpoint{0.027778in}{0.000000in}}%
\pgfpathcurveto{\pgfqpoint{0.027778in}{0.007367in}}{\pgfqpoint{0.024851in}{0.014433in}}{\pgfqpoint{0.019642in}{0.019642in}}%
\pgfpathcurveto{\pgfqpoint{0.014433in}{0.024851in}}{\pgfqpoint{0.007367in}{0.027778in}}{\pgfqpoint{0.000000in}{0.027778in}}%
\pgfpathcurveto{\pgfqpoint{-0.007367in}{0.027778in}}{\pgfqpoint{-0.014433in}{0.024851in}}{\pgfqpoint{-0.019642in}{0.019642in}}%
\pgfpathcurveto{\pgfqpoint{-0.024851in}{0.014433in}}{\pgfqpoint{-0.027778in}{0.007367in}}{\pgfqpoint{-0.027778in}{0.000000in}}%
\pgfpathcurveto{\pgfqpoint{-0.027778in}{-0.007367in}}{\pgfqpoint{-0.024851in}{-0.014433in}}{\pgfqpoint{-0.019642in}{-0.019642in}}%
\pgfpathcurveto{\pgfqpoint{-0.014433in}{-0.024851in}}{\pgfqpoint{-0.007367in}{-0.027778in}}{\pgfqpoint{0.000000in}{-0.027778in}}%
\pgfpathlineto{\pgfqpoint{0.000000in}{-0.027778in}}%
\pgfpathclose%
\pgfusepath{stroke,fill}%
}%
\begin{pgfscope}%
\pgfsys@transformshift{0.572136in}{1.365470in}%
\pgfsys@useobject{currentmarker}{}%
\end{pgfscope}%
\begin{pgfscope}%
\pgfsys@transformshift{0.658746in}{1.251514in}%
\pgfsys@useobject{currentmarker}{}%
\end{pgfscope}%
\begin{pgfscope}%
\pgfsys@transformshift{0.745356in}{1.088110in}%
\pgfsys@useobject{currentmarker}{}%
\end{pgfscope}%
\begin{pgfscope}%
\pgfsys@transformshift{0.831966in}{1.004344in}%
\pgfsys@useobject{currentmarker}{}%
\end{pgfscope}%
\begin{pgfscope}%
\pgfsys@transformshift{0.918576in}{0.945722in}%
\pgfsys@useobject{currentmarker}{}%
\end{pgfscope}%
\begin{pgfscope}%
\pgfsys@transformshift{1.005186in}{0.812831in}%
\pgfsys@useobject{currentmarker}{}%
\end{pgfscope}%
\begin{pgfscope}%
\pgfsys@transformshift{1.091796in}{0.734395in}%
\pgfsys@useobject{currentmarker}{}%
\end{pgfscope}%
\begin{pgfscope}%
\pgfsys@transformshift{1.178407in}{0.686625in}%
\pgfsys@useobject{currentmarker}{}%
\end{pgfscope}%
\begin{pgfscope}%
\pgfsys@transformshift{1.265017in}{0.693515in}%
\pgfsys@useobject{currentmarker}{}%
\end{pgfscope}%
\begin{pgfscope}%
\pgfsys@transformshift{1.351627in}{0.652303in}%
\pgfsys@useobject{currentmarker}{}%
\end{pgfscope}%
\begin{pgfscope}%
\pgfsys@transformshift{1.438237in}{0.781013in}%
\pgfsys@useobject{currentmarker}{}%
\end{pgfscope}%
\begin{pgfscope}%
\pgfsys@transformshift{1.524847in}{0.679768in}%
\pgfsys@useobject{currentmarker}{}%
\end{pgfscope}%
\begin{pgfscope}%
\pgfsys@transformshift{1.611457in}{0.620404in}%
\pgfsys@useobject{currentmarker}{}%
\end{pgfscope}%
\begin{pgfscope}%
\pgfsys@transformshift{1.698068in}{0.597305in}%
\pgfsys@useobject{currentmarker}{}%
\end{pgfscope}%
\begin{pgfscope}%
\pgfsys@transformshift{1.784678in}{0.582352in}%
\pgfsys@useobject{currentmarker}{}%
\end{pgfscope}%
\begin{pgfscope}%
\pgfsys@transformshift{1.871288in}{0.583394in}%
\pgfsys@useobject{currentmarker}{}%
\end{pgfscope}%
\begin{pgfscope}%
\pgfsys@transformshift{1.957898in}{0.587189in}%
\pgfsys@useobject{currentmarker}{}%
\end{pgfscope}%
\begin{pgfscope}%
\pgfsys@transformshift{2.044508in}{0.551141in}%
\pgfsys@useobject{currentmarker}{}%
\end{pgfscope}%
\begin{pgfscope}%
\pgfsys@transformshift{2.131118in}{0.610560in}%
\pgfsys@useobject{currentmarker}{}%
\end{pgfscope}%
\begin{pgfscope}%
\pgfsys@transformshift{2.217728in}{0.590690in}%
\pgfsys@useobject{currentmarker}{}%
\end{pgfscope}%
\begin{pgfscope}%
\pgfsys@transformshift{2.304339in}{0.811826in}%
\pgfsys@useobject{currentmarker}{}%
\end{pgfscope}%
\begin{pgfscope}%
\pgfsys@transformshift{2.390949in}{0.933621in}%
\pgfsys@useobject{currentmarker}{}%
\end{pgfscope}%
\begin{pgfscope}%
\pgfsys@transformshift{2.477559in}{0.768193in}%
\pgfsys@useobject{currentmarker}{}%
\end{pgfscope}%
\begin{pgfscope}%
\pgfsys@transformshift{2.564169in}{0.749300in}%
\pgfsys@useobject{currentmarker}{}%
\end{pgfscope}%
\begin{pgfscope}%
\pgfsys@transformshift{2.650779in}{0.703799in}%
\pgfsys@useobject{currentmarker}{}%
\end{pgfscope}%
\begin{pgfscope}%
\pgfsys@transformshift{2.737389in}{0.687053in}%
\pgfsys@useobject{currentmarker}{}%
\end{pgfscope}%
\begin{pgfscope}%
\pgfsys@transformshift{2.824000in}{0.696271in}%
\pgfsys@useobject{currentmarker}{}%
\end{pgfscope}%
\begin{pgfscope}%
\pgfsys@transformshift{2.910610in}{0.676154in}%
\pgfsys@useobject{currentmarker}{}%
\end{pgfscope}%
\begin{pgfscope}%
\pgfsys@transformshift{2.997220in}{0.695448in}%
\pgfsys@useobject{currentmarker}{}%
\end{pgfscope}%
\begin{pgfscope}%
\pgfsys@transformshift{3.083830in}{0.697609in}%
\pgfsys@useobject{currentmarker}{}%
\end{pgfscope}%
\end{pgfscope}%
\begin{pgfscope}%
\pgfsetrectcap%
\pgfsetmiterjoin%
\pgfsetlinewidth{0.803000pt}%
\definecolor{currentstroke}{rgb}{0.000000,0.000000,0.000000}%
\pgfsetstrokecolor{currentstroke}%
\pgfsetdash{}{0pt}%
\pgfpathmoveto{\pgfqpoint{0.568671in}{0.451277in}}%
\pgfpathlineto{\pgfqpoint{0.568671in}{1.250151in}}%
\pgfusepath{stroke}%
\end{pgfscope}%
\begin{pgfscope}%
\pgfsetrectcap%
\pgfsetmiterjoin%
\pgfsetlinewidth{0.803000pt}%
\definecolor{currentstroke}{rgb}{0.000000,0.000000,0.000000}%
\pgfsetstrokecolor{currentstroke}%
\pgfsetdash{}{0pt}%
\pgfpathmoveto{\pgfqpoint{3.166976in}{0.451277in}}%
\pgfpathlineto{\pgfqpoint{3.166976in}{1.250151in}}%
\pgfusepath{stroke}%
\end{pgfscope}%
\begin{pgfscope}%
\pgfsetrectcap%
\pgfsetmiterjoin%
\pgfsetlinewidth{0.803000pt}%
\definecolor{currentstroke}{rgb}{0.000000,0.000000,0.000000}%
\pgfsetstrokecolor{currentstroke}%
\pgfsetdash{}{0pt}%
\pgfpathmoveto{\pgfqpoint{0.568671in}{0.451277in}}%
\pgfpathlineto{\pgfqpoint{3.166976in}{0.451277in}}%
\pgfusepath{stroke}%
\end{pgfscope}%
\begin{pgfscope}%
\pgfsetrectcap%
\pgfsetmiterjoin%
\pgfsetlinewidth{0.803000pt}%
\definecolor{currentstroke}{rgb}{0.000000,0.000000,0.000000}%
\pgfsetstrokecolor{currentstroke}%
\pgfsetdash{}{0pt}%
\pgfpathmoveto{\pgfqpoint{0.568671in}{1.250151in}}%
\pgfpathlineto{\pgfqpoint{3.166976in}{1.250151in}}%
\pgfusepath{stroke}%
\end{pgfscope}%
\begin{pgfscope}%
\pgfsetbuttcap%
\pgfsetmiterjoin%
\definecolor{currentfill}{rgb}{1.000000,1.000000,1.000000}%
\pgfsetfillcolor{currentfill}%
\pgfsetfillopacity{0.800000}%
\pgfsetlinewidth{1.003750pt}%
\definecolor{currentstroke}{rgb}{0.800000,0.800000,0.800000}%
\pgfsetstrokecolor{currentstroke}%
\pgfsetstrokeopacity{0.800000}%
\pgfsetdash{}{0pt}%
\pgfpathmoveto{\pgfqpoint{1.860321in}{1.036802in}}%
\pgfpathlineto{\pgfqpoint{3.098920in}{1.036802in}}%
\pgfpathquadraticcurveto{\pgfqpoint{3.118365in}{1.036802in}}{\pgfqpoint{3.118365in}{1.056247in}}%
\pgfpathlineto{\pgfqpoint{3.118365in}{1.182095in}}%
\pgfpathquadraticcurveto{\pgfqpoint{3.118365in}{1.201540in}}{\pgfqpoint{3.098920in}{1.201540in}}%
\pgfpathlineto{\pgfqpoint{1.860321in}{1.201540in}}%
\pgfpathquadraticcurveto{\pgfqpoint{1.840877in}{1.201540in}}{\pgfqpoint{1.840877in}{1.182095in}}%
\pgfpathlineto{\pgfqpoint{1.840877in}{1.056247in}}%
\pgfpathquadraticcurveto{\pgfqpoint{1.840877in}{1.036802in}}{\pgfqpoint{1.860321in}{1.036802in}}%
\pgfpathlineto{\pgfqpoint{1.860321in}{1.036802in}}%
\pgfpathclose%
\pgfusepath{stroke,fill}%
\end{pgfscope}%
\begin{pgfscope}%
\pgfsetrectcap%
\pgfsetroundjoin%
\pgfsetlinewidth{1.505625pt}%
\definecolor{currentstroke}{rgb}{0.121569,0.466667,0.705882}%
\pgfsetstrokecolor{currentstroke}%
\pgfsetdash{}{0pt}%
\pgfpathmoveto{\pgfqpoint{1.879766in}{1.128623in}}%
\pgfpathlineto{\pgfqpoint{1.976988in}{1.128623in}}%
\pgfpathlineto{\pgfqpoint{2.074210in}{1.128623in}}%
\pgfusepath{stroke}%
\end{pgfscope}%
\begin{pgfscope}%
\pgfsetbuttcap%
\pgfsetroundjoin%
\definecolor{currentfill}{rgb}{0.000000,0.000000,0.000000}%
\pgfsetfillcolor{currentfill}%
\pgfsetfillopacity{0.000000}%
\pgfsetlinewidth{1.003750pt}%
\definecolor{currentstroke}{rgb}{0.121569,0.466667,0.705882}%
\pgfsetstrokecolor{currentstroke}%
\pgfsetdash{}{0pt}%
\pgfsys@defobject{currentmarker}{\pgfqpoint{-0.027778in}{-0.027778in}}{\pgfqpoint{0.027778in}{0.027778in}}{%
\pgfpathmoveto{\pgfqpoint{0.000000in}{-0.027778in}}%
\pgfpathcurveto{\pgfqpoint{0.007367in}{-0.027778in}}{\pgfqpoint{0.014433in}{-0.024851in}}{\pgfqpoint{0.019642in}{-0.019642in}}%
\pgfpathcurveto{\pgfqpoint{0.024851in}{-0.014433in}}{\pgfqpoint{0.027778in}{-0.007367in}}{\pgfqpoint{0.027778in}{0.000000in}}%
\pgfpathcurveto{\pgfqpoint{0.027778in}{0.007367in}}{\pgfqpoint{0.024851in}{0.014433in}}{\pgfqpoint{0.019642in}{0.019642in}}%
\pgfpathcurveto{\pgfqpoint{0.014433in}{0.024851in}}{\pgfqpoint{0.007367in}{0.027778in}}{\pgfqpoint{0.000000in}{0.027778in}}%
\pgfpathcurveto{\pgfqpoint{-0.007367in}{0.027778in}}{\pgfqpoint{-0.014433in}{0.024851in}}{\pgfqpoint{-0.019642in}{0.019642in}}%
\pgfpathcurveto{\pgfqpoint{-0.024851in}{0.014433in}}{\pgfqpoint{-0.027778in}{0.007367in}}{\pgfqpoint{-0.027778in}{0.000000in}}%
\pgfpathcurveto{\pgfqpoint{-0.027778in}{-0.007367in}}{\pgfqpoint{-0.024851in}{-0.014433in}}{\pgfqpoint{-0.019642in}{-0.019642in}}%
\pgfpathcurveto{\pgfqpoint{-0.014433in}{-0.024851in}}{\pgfqpoint{-0.007367in}{-0.027778in}}{\pgfqpoint{0.000000in}{-0.027778in}}%
\pgfpathlineto{\pgfqpoint{0.000000in}{-0.027778in}}%
\pgfpathclose%
\pgfusepath{stroke,fill}%
}%
\begin{pgfscope}%
\pgfsys@transformshift{1.976988in}{1.128623in}%
\pgfsys@useobject{currentmarker}{}%
\end{pgfscope}%
\end{pgfscope}%
\begin{pgfscope}%
\definecolor{textcolor}{rgb}{0.000000,0.000000,0.000000}%
\pgfsetstrokecolor{textcolor}%
\pgfsetfillcolor{textcolor}%
\pgftext[x=2.151988in,y=1.094595in,left,base]{\color{textcolor}\rmfamily\fontsize{7.000000}{8.400000}\selectfont adaptive \(\displaystyle \gamma_i,\,K=1\)}%
\end{pgfscope}%
\end{pgfpicture}%
\makeatother%
\endgroup%

    \vspace*{-0.6cm}
    \caption[]{Results of 30 Monte-Carlo runs: Top: Estimated \(\|\h^{(n)}\|\), \(\|\hat{\h}_i^{(n)}\|\) over time, shown is the mean. Dot-dashed lines are true values. Bottom: Median NPM, comparing optimal algorithm and proposed extended algorithm. Dashed vertical lines indicate rescaling events.}
    \label{fig:simulations:NPMtimedyn}
\end{figure}
\section[]{Conclusions}
\label{sec:conclusions}
In this contribution, we propose an extension to a distributed adaptive BSI algorithm applied in sensor networks.
Based on distributed averaging, only using the information provided by neighboring nodes in the sensor network, we compute estimates of norm values of channel estimates in order to enforce a norm constraint.
By balancing the averaging with introducing new data into the recursion, we allow the algorithm to follow an adaptive updating scheme.
The mixing factor, which balances new data with recursion data, is set adaptively dependent on instantaneous channel estimate norms.
These introductions reduce inter-node transmissions for each time frame while still delivering steady-state estimation performance close to the optimal case where network-wide information is available at all nodes.
The reduction that can be achieved goes as far as only needing a single iteration and, therefore, one additional inter-neighborhood information exchange per time frame.
We illustrate the performance in simulations with a white Gaussian input signal and random impulse responses drawn from the normal distribution.
% \begin{todo}
%     \begin{itemize}
%         \item Scale up simulation (for first setup) to \(M >> N_i\), like M = 50 and neighborhoods like in Toon's paper.
%     \end{itemize}
% \end{todo}

\vfill\pagebreak

% References should be produced using the bibtex program from suitable
% BiBTeX files (here: strings, refs, manuals). The IEEEbib.bst bibliography
% style file from IEEE produces unsorted bibliography list.
% -------------------------------------------------------------------------
\bibliographystyle{IEEEbib}
\bibliography{bib_abbrev,refs}

\end{document}
